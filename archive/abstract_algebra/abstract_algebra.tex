\documentclass[11pt,oneside,titlepage]{book}
\title{My abstract algebra exercises}
\usepackage{amsmath, amssymb}
\usepackage{geometry}
\usepackage{hyperref}
\author{Evgeny Markin}
\date{2023}

\DeclareMathOperator \map {\mathcal {L}}
\DeclareMathOperator \pow {\mathcal {P}}
\DeclareMathOperator \topol {\mathcal {T}}
\DeclareMathOperator \basis {\mathcal {B}}
\DeclareMathOperator \ns {null}
\DeclareMathOperator \range {range}
\DeclareMathOperator \fld {fld}
\DeclareMathOperator \inv {^{-1}}
\DeclareMathOperator \Span {span}
\DeclareMathOperator \lra {\Leftrightarrow}
\DeclareMathOperator \eqv {\Leftrightarrow}
\DeclareMathOperator \la {\Leftarrow}
\DeclareMathOperator \ra {\Rightarrow}
\DeclareMathOperator \imp {\Rightarrow}
\DeclareMathOperator \true {true}
\DeclareMathOperator \false {false}
\DeclareMathOperator \dom {dom}
\DeclareMathOperator \ran {ran}
\newcommand{\eangle}[1]{\langle #1 \rangle}
\newcommand{\set}[1]{\{ #1 \}}

\begin{document}
\maketitle
\tableofcontents

\chapter{Groups}

\section{Symmetries of  a Regular Polygon}

\textit{Content of this section was pretty much taken care of in a previous try at an
  abstract algebra coutse}

\section{Introduction to Groups}

\textit{Same applies to this section}

\section{Properties of Group Elements}

\subsection{}

\textit{Find the orders of $\overline{5}$ and $\overline{6}$ in $(Z/21Z, +)$}

We follow that order of $\overline{5}$ is $21$ and $7$ for $\overline{6}$.

\subsection{}

\textit{Find the orders of $\overline{21}$ in $Z/52$}

It's' 13

\subsection{}

\textit{Calculate the order of $\overline{285}$ in the group $Z/360Z$}

$$(285 * 24) / 360 = 19$$
thus the order is 19

\textit{The rest of the exercises (or exercises similar to those given in a book) were
taken care of previously in previous books}

\section{}

\end{document}

%%% Local Variables:
%%% mode: latex
%%% TeX-master: t
%%% End:
