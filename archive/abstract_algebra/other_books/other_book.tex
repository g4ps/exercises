\documentclass[11pt,oneside,titlepage]{book}
\title{My (other) exercises in abstract algebra}
\usepackage{amsmath, amssymb}
\usepackage{geometry}
\usepackage{hyperref}
\author{Evgeny Markin}
\date{2023}

\DeclareMathOperator \map {\mathcal {L}}
\DeclareMathOperator \ns {null}
\DeclareMathOperator \range {range}
\DeclareMathOperator \inv {^{-1}}
\DeclareMathOperator \Span {span}
\DeclareMathOperator \imp {\Rightarrow}
\DeclareMathOperator \lra {\Leftrightarrow}
\newcommand{\eangle}[1]{\langle #1 \rangle}


\begin{document}
\maketitle
\tableofcontents


\chapter*{Prefase}

Started with another book on the subject, switched to Dummit and Foote,
since I don't like this book very much. Left here those exercises, because
why not

\chapter{Integers}

\section{Divisors}

\subsection{}

\textit{Let $m, n, r, s \in Z$. If $m^2 + n^2 = r^2 + s^2 = mr + ns$ , prove that $m = r$ and
  $n = s$.}

We can state that there are 3 possible cases: $m = r$ and $n = s$, one of the equation holds and
none of the equations hold. 

If one of the equations does not hold, suppose that $m \neq r$, then we follow that
$$ns = s^2 = n^2$$, therefore
$$m^2 + n^2 = mr + ns$$
$$m^2 + n^2 = mr + n^2$$
$$m^2 = mr$$
$$m = r$$
which is a contradiction. Thus we follow that the case when only one of the equations
does not hold is impossible.

Suppose now that $m \neq r$ and $n \neq s$. We follow that $(m - r) \neq 0$ and
$(n - s) \neq 0$. Thus $(m - r)^2 \neq 0$ and $(n - s)^2 \neq 0$. moreover, since
we've got squares we follow that
$$(m - r)^2 > 0$$
and
$$(n - s)^2 > 0$$
thus
$$(m - r)^2 + (n - s)^2 > 0$$
thus
$$(m - r)^2 + (n - s)^2 \neq 0$$
therefore 
$$(m - r)^2 + (n - s)^2 = m^2 - 2mr + r^2 + n^2 - 2ns + s^2 =
(m^2 + r^2) + (r^2 + s^2) - 2(mr + ns) \neq 0$$
Now if we use our identity  $m^2 + n^2 = r^2 + s^2 = mr + ns$, we gonna get that
$$(m^2 + r^2) + (r^2 + s^2) - 2(mr + ns) = (m^2 + r^2) + (m^2 + r^2) - 2(m^2 + r^2) = 0 \neq 0$$
which gives us a contradiction. Thus we follow that this case is impossible as well.
Thus we conclude that $m = r$ and $n = s$, as desired.

We can prove that simular conclusion holds for reals as well, since we haven't used properties
that are exclusive for $Z$.

\subsection{}

\textit{For each number between 6 and the next perfect number, make a list containing the number,
its proper divisors, and their sum}

\begin{verbatim}
7: 1, sum:  1
8: 1, 2, 4, sum:  7
9: 1, 3, sum:  4
10: 1, 2, 5, sum:  8
11: 1, sum:  1
12: 1, 2, 3, 4, 6, sum:  16
13: 1, sum:  1
14: 1, 2, 7, sum:  10
15: 1, 3, 5, sum:  9
16: 1, 2, 4, 8, sum:  15
17: 1, sum:  1
18: 1, 2, 3, 6, 9, sum:  21
19: 1, sum:  1
20: 1, 2, 4, 5, 10, sum:  22
21: 1, 3, 7, sum:  11
22: 1, 2, 11, sum:  14
23: 1, sum:  1
24: 1, 2, 3, 4, 6, 8, 12, sum:  36
25: 1, 5, sum:  6
26: 1, 2, 13, sum:  16
27: 1, 3, 9, sum:  13
28: 1, 2, 4, 7, 14, sum:  28 PERFECT
\end{verbatim}

\subsection{}

\textit{Find the quotent and remainder when $a$ is divided by $b$}

$$99 = 17 * 5 + 14$$
$$-99 = 17 * (-6) + 3$$
$$17 = 99 * 0 + 17$$
$$-1017 = -11 * 99 + 72$$

\subsection{}

\textit{Use the Eucledian algorithm to find the following greatest common divisors.}

$$(35, 14) = (14, 7) = 7$$
$$(15, 11) = (11, 4) = (4, 3) = (3, 1) = 1$$
$$(252, 180) = (180, 72) = (72, 36) = 36$$
$$(513, 187) = (187, 139) = (139, 48) = (48, 43) = (43, 5) = (5, 3) = (3, 2) = (2, 1) = 1$$
$$(7655, 1001) = (1001, 648) = (648, 353) = (353, 295) = $$
$$ = (295, 58) = (58, 5) = (5, 3) = (3, 2) = (2, 1) = 1$$

\subsection{}

\textit{Use the Eucledian algorith to find the following greatest common divisors}

$$(6643, 2873) = (2873, 897) = (897, 182) = (182, 169) = (169, 13) = 13$$
$$(7684, 4148) = (4148, 3536) = (3536, 612) = (612, 476) = (476, 136) = (136, 68) = 68$$
$$(26460, 12600) = (12600, 1260) = 1260$$
$$(6540, 1206) = (1206, 510) = (510, 186) = (186, 138) = (138, 48) = (48, 42) = (42, 6) = 6$$
$$(12081, 8439) = (8439, 3642) = (3642, 1155) = $$
$$ = (1155, 177) = (177, 93) = (93, 84) = (84, 9) = (9, 3) = 3$$

\subsection{}
\subsection{}
\textit{Skipped}

\subsection{}

\textit{Let $a, b, c \in Z$. Give a proof for these facts about divisors:}

\textit{(a) If $b|a$, then $b|ac$}

Suppose that $b|a$. We follow that $a = qb$ for some $q \in Z$. Thus we follow that
$ca = cqb$. Thus $b|ac$, as desired.

\textit{(b) If $b|a$ and $c|b$, then $c|a$}

We follow that $a = qb$ and $b = wc$ for some $w, q \in Z$. Thus $a = wqc$, thus $c|a$.

\textit{(c) If $c|a$ and $c|b$, then $c|(ma + nb)$}

We follow that since $c|a$ and $c|b$ that $c|(a, b)$. We follow that $(a, b)|(ma + nb)$, since
$ma + nb$ is a linear combination of $a, b$. Thus by previous point we follow
$c | (ma + nb)$.

\subsection{}

\textit{Let $a, b, c \in Z$ are such that $a + b + c = 0$. Show that if $n \in Z$
  and $n$ is a divbvisor of two of the three integers, then it is also a divisot of the third.}

Suppose that $n|a$ and $n|b$. Then we follow that $n|(a, b)$. Since $-a - b = c$ we follow that
$(a, b) | c$, thus $n|c$, as desired.

\subsection{}

\textit{Let $a, b, c \in Z$.}

\textit{(a) Show that if $b|a$ and $b|(a + c)$, then $b|c$.}

We follow that $\exists q, w \in Z$ such that 
$$a = qb $$
$$a + c = wb$$
thus 
$$qb + c = wb$$
$$c = wb - qb$$
$$c = b(w - q)$$
$$b | c$$

\textit{(b) Show that if $b|a$ and $b\not|c$ the $b \not | (a + c)$.}

If $b \not | c$, then we follow that there exists $q, w, r \in Z$ such that $0 < r < b$ and 
$$a = qb \land c = wb + r$$
thus
$$a + c = (q + w)b + r$$
thus $b \not | (a + c)$ as desired.

\subsection{}

\textit{Let $a, b, c \in Z$ and $c \neq 0$. Show that $bc | ac$ iff $b | a$.}

Suppose that $bc | ac$. This means 
$$ac = qbc$$
and since $c \neq 0$ we follow that it is equivalent to 
$$a = qb$$
i.e. $b|a$. Since every implication here is an equivalence, we follow that we've got a
converse as well.

\subsection{}

\textit{Show that if $a > 0$, then $(ab, ac) = a(b, c)$}

We follow that there exist $m, n \in Z$ such that 
$$(b, c) = mb + nc$$
thus
$$a(b, c) = a(mb + nc)$$
$$a(b, c) = m(ab) + n(ac)$$
therefore we follow that $a(b, c)$ is a multiple of $(ac, bc)$, thus $(ac, bc)|a(b, c)$

$(b, c)| b$ and $(b, c)| c$, thus $a(b, c) | ab$ and $a(b, c) | ac$, thus $a(b, c) | (ab, ac)$.
Thus we follow that $(ac, bc) = a(b, c)$, as desired.

\subsection{}

\textit{Show that if $n$ is any integer, then $(10n + 3, 5n + 2)  = 1$}

We know that gcd is a smallest positive linear combination of $10 n + 3$ and $5n + 2$. Thus

$$-(10 n + 3) + 2(5n + 2) = -10n - 3 + 10n + 4 = 1$$
Since gcd is a smallest positive linear combination of $10n + 3$ and $5n + 2$, and there
is no smaller positive number then 1, we follow that $(10n + 3, 5n + 2) = 1$, as desired.

\subsection{}

\textit{Show that if $n$ is any integer then $(a + nb, b) = (a, b)$}

We follow that $(a, b)$ is the least positive linear combination of $a, b$. Also,
$(a + nb, b)$ is the least linear combination of $a + nb$ and $b$. Since
$$q(a + nb) + wb = qa + qnb + wb = qa + (qn - w) b$$
we follow that $(a + nb, b)$ is also the linear combination of $a$ and $b$ (because $qn + w$
with fixed $qn$ can be still any number). Since
there is only one positive linear combination of $a$ and $b$, we follow that
$(a + bn, b) = (a, b)$, as desired.

\subsection{}

\textit{For what positive integers $n$ is it true that $(n, n + 2) = 2$? Prove your claim.}

It appears that it is true for all even numbers. It is certainly true, that if $n$
is even, then $n + 2$ is also even, therefore both of them are divisible by 2.

We know that $(a, b) = (b, a)$, thus we follow that $(n, n + 2) = (n + 2, n)$.

Suppose that $n$ is even
By euclidean algoritm we've got that
$$(n + 2) = 1(n) + 2$$
thus
$$(n + 2, n) = (n, 2) = 2$$
Thus if $n$ is even, then $(n, n + 2) = 2$.

If $n = 1$, then $(n + 2, n) = (3, 1) = 1$.

If $n > 1$ and $n$ is odd, then there exists $k \in N$ such that $k \geq  1$ and  $n = 2k + 1$.
THus we follow that $n + 2 = 2k + 1 + 2 = 2k + 3$. Thus 
$$(2k + 3) = 1*(2k + 1) + 2$$
since $k \geq 1$, we follow that $0 \leq 2 \leq 2k + 1$.
Thus we can conclude that
$$(n + 2, n) = (2k + 3, 2k + 1) = (2k + 1, 2) = 1$$

Therefore we follow that the only positive numbers such that $(n, n + 2) = 2$ are
the even numbers.

\subsection{}

\textit{Show that the positive integer $n$ is the difference of two squares if and
  only if $n$ is odd or divisible by 4.}

Let $n \in Z^+$ and  $a, b \in Z$ be such that $a^2 - b^2 = n$.

We follow that since $n \geq 0$, than $a^2 - b^2 \geq 0$, therefore $a^2 \geq b^2$.
Since $a^2 = (-a)^2$, let us assume that $a, b \geq 0$, because other cases will be trivial.

Since $a^2 \geq b^2$, we follow that $a \geq b$, therefore there exists $r$ such that
$b + r = a$. Thus
$$n = a^2 - b^2 = (b + r)^2 - b^2 = b^2 + 2br + r^2 - b^2 = 2br + r^2$$
We've got that $r$ is either odd or even. If $r$ is odd, then $r^2$ is odd as well. Thus
the sum of the even number $2br$ and odd $r^2$ is odd. Therefore $n$ is odd as well.
If $r$ is even, then there exists $k \in N$ such that $r = 2k$. Thus
$$n = 2br + r^2 = 2b2k + (2k)^2 = 4bk + 4k^2 = 4(bk + k^2)$$
thus we follow that $n$ is divisible by 4. Therefore we follow that $n$ is either
odd or divisible by 4, as desired.

Conversely, suppose that $n \in Z^+$ is either odd or divisible by 4.

If $4|n$, then we follow that there exists $k \in N$ such that $n = 4k$. Thus
we follow that
$$(k + 1)^2 - (k - 1)^2 = k^2 + 2k + 1 - k^2 + 2k - 1 = 4k = n$$
thus we follow that $n$ is the difference of two squares.

If $n$ is odd, then we follow that there exists $k \in N$  such that $n = 2k - 1$. Thus
we follow that
$$k^2 - (k - 1)^2 = k^2 - k^2 + 2k - 1 = 2k - 1 = n$$

Thus we follow that if $n$ is divisible by 4 or is odd, then it is the difference of two squares,
as desired.

\subsection{}

\textit{Show that the positive integer $k$ is the difference of two odd squares if and
  only if $k$ is divisible by 8}

Suppose that $n$ is the difference between two odd squares. Thus we follow that
$$n = (2k + 1)^2 - (2n + 1)^2$$
if we expand and contract this expression, then we'll get
$$n = 4(k - n)(n + k + 1)$$
We follow that if both $n$ and $k$ are odd or both of them are even, then $(n - k)$ is even.
If one of them is odd while the other one is even, then $(n + k + 1)$ is even. Thus we
follow that $(k - n)(n + k + 1)$ is even, therefore there exists $q$ such that
$$2q = (k - n)(n + k + 1)$$
thus
$$n = 4 * 2q = 8q$$
thus we follow that $n$ is divisible by 8.

Suppose that $n$ is divisible by 8. Then we follow that there exists $k$ such that
$n = 8k$. Thus
$$(2k + 1)^2 - (2 k - 1)^2 = 8k = n$$
thus $n$ is the difference between two odd squares.

\subsection{}

\textit{Give a detailed proof of the statement in the text that if $a$ and $b$ are integers,
  then $b|a$ if and only if $aZ \subseteq bZ$.}

Suppose that $b|a$. Then we follow that $a = qb$ for some $q \in Z$.
Suppose that $n \in aZ$. Then we follow that $n = wa$ for some $w \in Z$. Thus
$n = wqb$, therefore $n \in bZ$. Thus we follow that $aZ \subseteq bZ$.

Conversely, suppsoe that $aZ \subseteq bZ$. We follow that because $a = 1a$ we can state that
$a \in aZ$. Thus $a \in bZ$, threfore by definition of $bZ$ we follow that there exists
$q \in Z$ such that $a = qb$. Thus $b | a$, as desired.

\subsection{}

\textit{Let $a, b, c \in Z \land b > 0 \land c > 0 \land a = qb + r$.}

$$a = qb + r \lra ca = c(qb + r) = cqb + cr = (cq)b + cr$$
Since $r < b$, we follow that $cr < cb$, thus everything holds.
(Skipping (b) because I'm lazy )


\subsection{}

\textit{Let $a, b, n \in Z \land n > 1$. Suppose that $a = nq_1 + r_1$ with $0 \leq r1 < n$
  and $b = nq_2 + r_2$ with $0 \leq r_2 < n$. Prove that $n | (a - b)$ if and only if $r_1 = r_2$.
}

Suppose that $n | (a - b)$.
$$(a - b) = n q_1 + r_1 - (nq_2 + r_2) = n(q_1 - q_2) + (r_1 - r_2)$$
Since $0 \leq r_1, r_2 < n$, we follow that $-n < (r_1 - r_2) < n$, thus we follow that
if $r_1 \neq r_2$, then we've got a contradiction. Converse case is trivial.

\subsection{}

\textit{Show that any nonempty set of integers that is closed under substraction must also
  be closed under addition.}

I personally like closure under additive inverse and closure under addition, but whatever.

Suppose that $S$ is closed under substraction and let $a_1, a_2 \in S$. We follow that
$$a_2 \in S$$
$$a_2 - a_2 = 0 \in S$$
$$0 - a_2 = -a_2 \in S$$
$$a_1 -(- a_2) = a_1 + a_2 \in S$$
thus the set is closed under addition, as desired.

\subsection{}

\subsection{}

skip

\subsection{}

\textit{Show that 3 divides the sum of the cubes of any three consecutive positive integers}

Suppose that $n \in Z^+$. Then we follow that the sum of cubes of 3 consecutive
numbers is equal to
$$n^3 + (n + 1)^3 + (n + 2)^3 = n^3 + n^3 + 3n^2 + 3n + 1 + n^3 + 6n^2 + 12n + 8 =$$
$$ = 3n^3 + 9n^2 + 15n + 9 = 3(n^3 + 3n^2 + 5n + 3)$$
thus it is divisible by 3, as desired.

It's also divisible by $n + 1$, since
$$3(n^3 + 3n^2 + 5n + 3) = 3(n + 1)(n^2 + 2n + 3)$$

\subsection{}

\textit{Find all integers $x$ such that $3x + 7$ is divisible by 11}

Suppose that
$$Y = \{y \in Z: (\exists x \in Z)(y = 11x + 5)\}$$.
Then we follow that
$$3(11 x + 5) + 7 = 33x + 22 = 11(3x + 2)$$
thus $x \in Y \to 11|3x + 7$.

Can't find the other inclusion.

\textit{Rest of the exercises is left for better days.}

\section{Primes}

\subsection*{1.2.4}

\textit{Find all positive integers less than 60 and relatively prime to 60}

\begin{verbatim}
( 1 , 60) =  1
( 7 , 60) =  1
( 11 , 60) =  1
( 13 , 60) =  1
( 17 , 60) =  1
( 19 , 60) =  1
( 23 , 60) =  1
( 29 , 60) =  1
( 31 , 60) =  1
( 37 , 60) =  1
( 41 , 60) =  1
( 43 , 60) =  1
( 47 , 60) =  1
( 49 , 60) =  1
( 53 , 60) =  1
( 59 , 60) =  1
\end{verbatim}

\subsection*{1.2.5}

\textit{Let $p_1, ..., $ be the sequence of primes and set $a_1 = p_1 + 1$,
  $a_2 = p_1 p_2 + 1$ and so on. What is the least $n$ such that $a_n$ is composite}


$$2 * 3 * 5 * 7 * 11 * 13 + 1 = 30031 = 59 * 509$$

\subsection*{1.2.9}

$$2, 1, 2, 2$$

\subsection*{1.2.10}

\textit{Prove that $n^4 + 4$ is composite if $n > 1$}

If $n$ is even, then the sum is even, therefore it's composite. If $n$ is odd, then there exists
$k \in N$ such that $n = 2k + 1$. Thus
$$n^4 + 4 = (2k + 1)^4 + 4 = 16k^4 + 32 k^3 + 24k^2 + 8k + 5 = (4k^2 + 1)(4k^2 + 8k + 5)$$
thus we follow that it's composite.

\section{Congruences}

\subsection*{1.3.[1, 3, 4, 5, 7, 15, 16]}
\begin{verbatim}
4 x <eq>  1 (mod  7 ) =  [2]
2 x <eq>  1 (mod  9 ) =  [5]
5 x <eq>  1 (mod  32 ) =  [13]
19 x <eq>  1 (mod  36 ) =  [19]
10 x <eq>  5 (mod  21 ) =  [11]
10 x <eq>  5 (mod  15 ) =  [2, 5, 8, 11, 14]
10 x <eq>  4 (mod  15 ) =  []
10 x <eq>  4 (mod  14 ) =  [6, 13]
20 x <eq>  12 (mod  72 ) =  [15, 33, 51, 69]
25 x <eq>  45 (mod  60 ) =  [9, 21, 33, 45, 57]
8 x <eq>  0 (mod  12 ) =  [0, 3, 6, 9]
7 x <eq>  0 (mod  12 ) =  [0]
21 x <eq>  0 (mod  28 ) =  [0, 4, 8, 12, 16, 20, 24]
12 x <eq>  0 (mod  18 ) =  [0, 3, 6, 9, 12, 15]
lambda x: x ** 2, 1, 16] = [1, 7, 9, 15]
lambda x: x ** 3, 1, 16] = [1]
lambda x: x ** 4, 1, 16] = [1, 3, 5, 7, 9, 11, 13, 15]
lambda x: x ** 8, 1, 16] = [1, 3, 5, 7, 9, 11, 13, 15]
lambda x: x ** 3 + 2 * x + 2, 0, 5] = [1, 3]
lambda x: x ** 4 + x ** 3 + x**2 + x + 1, 0, 2] = []
lambda x: x ** 4 + x ** 3 + 2 * x**2 + 2 * x + 1, 0, 3] = []
\end{verbatim}


\subsection*{1.3.6}

\textit{Find all integers $x$ such that $3x + 7$ is divisible by $11$}

We follow that this is equivalent to congruence
$$3x + 7 \equiv 0 \mod{11}$$
$$3x  \equiv 4 \mod{11}$$
for which the solution is $5$. Thus we follow that integers in form
$$3 * (11q + 5) + 7: q \in Z$$
are the desired solution

\subsection*{1.3.8}

\textit{Prove that if $p$ is a prime number and $a$ is any integer, such taht $p \not | a$,
  then the additive order of $a$ modulo $p$ is equal to $p$.}

Suppose that it isn't then we follow that there exists $0 < p' < p$ such that
$$p' a \equiv 0 \mod{p}$$
$$pq = p'a$$
Thus we've got that $p | p'a$, which is a contradiction of unique prime representation.

\section{Integers Modulo $n$}

\subsection*{1.4.[1, 2]}

modulo.py in progs folder produces answers, not gonna repeat them here

\subsection*{1.4.3}

\textit{Find the multiplicative inverses of given elements (if possible)}

$$[14]_{15} * [14]_{15} = [1]_{15}$$
$$[38]_{83} * [59]_{83} = [1]_{83}$$
351 is a zero divisor in $Z_{6669}$, to be precise we've got that
$$[351]_{6669} * [19]_{6669} = [0]_{6669}$$
$$[91]_{2565} * [451]_{2565} =  [1]_{2565}$$
everyhting was followed from congr.py in progs folder (in essence it comes from
usage of Euclidean algorithm)

\subsection*{1.4.4.}

\textit{Let $a$ and $b$ be integers. }

\textit{(a) Prove that $[a]_n = [b]_n$ iff $a \equiv b \mod n$.}

$$[a]_n = [b]_n$$
$$[a]_n - [b]_n = [0]_n$$
$$[a - b]_n = [0]_n$$
$$n | (a - b)$$
$$a \equiv b \mod n$$
everything here is a equivalence, thus we've got converse case for free.

\textit{(b) Prove that either $[a]_n \cap [b]_n = \emptyset$ or $[a]_n = [b]_n$}

GOTO set theory book, section on equivalence relations and partitions that they make.


\subsection*{1.4.5}

\textit{Prove that each congruence class $[a]_n$ in $Z_n$ has a unique representative $r$
  that satisfies $0 \leq r \leq n$}

Given that $n > 0$, we follow that there exist unique $q$ and $0 \leq r < n$ such that
$$a = nq + r$$
from this we follow that $r \in [a]_n$, as desired. 

\end{document}
%%% Local Variables:
%%% mode: latex
%%% TeX-master: t
%%% End:
