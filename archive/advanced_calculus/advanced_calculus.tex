\documentclass[11pt,oneside,titlepage]{book}
\title{My advanced calculus exercises}
\usepackage{amsmath, amssymb}
\usepackage{geometry}
\usepackage{hyperref}
\author{Evgeny Markin}
\date{2023}

\DeclareMathOperator \inv {^{-1}}
\DeclareMathOperator \prob {\text{Prob}}

\begin{document}
\maketitle
\tableofcontents

\chapter*{Preface}


Exercises are from Advanced Calculus: A Geometric View by Callahan, 1st ed.


\chapter{Starting points}

\section{}

\textit{Evaluate }
$$\int_0^\infty{\frac{dx}{1 + x^2}}$$
\textit{and}
$$\int_{-\infty}^1{\frac{dx}{1 + x^2}}$$

$$\int_0^\infty{\frac{dx}{1 + x^2}} = [\tan \inv]_0^\infty = \pi/2 - 0 = \pi/2 $$

$$\int_{-\infty}^1{\frac{dx}{1 + x^2}} = [\tan \inv]_{-\infty}^1 = [\pi/4 + \pi/2] = \frac{3}{4} \pi$$


\section{}

\textit{Determine}
$$\int{\frac {x dx}{1 + x^2}}$$
\textit{Which type of substitution did you use?}

$$\int{\frac {x dx}{1 + x^2}} = \frac 1 2 \int{ \frac {2 x  dx}{1 + x^2}}$$
let $u(x) = 1 + x^2$. Then $u'(x) = 2x$ and therefore
$$\frac 1 2 \int{ \frac {2 x  dx}{1 + x^2}} =
\frac 1 2 \int{ \frac {du}{u}} = \frac{1}{2} \ln(u) = \frac{1}{2} \ln(1 + x^2)$$

I've used push-forward substitution here.

\section{}

\textit{Carry out a change of variables to evaluate the integral and determite the type
of substitution used.}
$$\int_{-R}^{R}{\sqrt{R^2 - x^2}dx}$$



Let's try $x = R \sin(s)$ (the idea is to use the identity $\sin^2(x) + \cos^2(x) = 1$ here
somewhere). It follows that $s = \arcsin(\frac{x}{R})$



$$\int_{-R}^{R}{\sqrt{R^2 - x^2}dx} =
\int_{-R}^{R}{\sqrt{R^2 - R^2 \sin^2(s)} R\cos(x) ds} =
\int_{-R}^{R}{R^2 \sqrt{1 -  \sin^2(s)} \cos(s) ds} =$$
$$ = 
R^2  \int_{-R}^{R}{\cos^2(s) ds} =
\frac{R^2}{2}  \int_{-R}^{R}{1 + \cos(2s)  ds} =
\frac{R^2}{2}\left[s + \frac{1}{2} \sin(2s)\right]_{-R}^R
=
$$

$$
= \frac{R^2}{2}\left[\arcsin(\frac{x}{R}) + \frac{1}{2} \sin(2\arcsin(\frac{x}{R}))\right]_{-R}^R =
$$
$$
=
\frac{R^2}{2} \left[\arcsin(1) + \frac{1}{2} \sin (2 \arcsin(1)) - \arcsin(-1) -
  \frac{1}{2} \sin (2 \arcsin(-1))\right]= 
$$
$$
=
\frac{R^2}{2} \left[\pi/2 + \frac{1}{2} \sin (\pi) + \pi/2 -
  \frac{1}{2} \sin (-\pi)\right]=
\frac{R^2}{2} \pi = \frac{\pi R^2}{2}
$$
Which is goot enough for me. We used the pullback approach here

\section{}

\textit{Determine }
$$\int {\frac{\arctan(x)}{1 + x^2} dx}$$
\textit{and show }
$$\int_0^\infty {\frac{\arctan(x)}{1 + x^2} dx} = \pi^2/8$$

$$\int {\frac{\arctan(x)}{1 + x^2} dx} $$
let $u = \arctan(x)$. Then
$$\int {\frac{\arctan(x)}{1 + x^2} dx} = \int u = u^2/2 = \arctan(x)^2/2$$
thus
$$\int_0^\infty {\frac{\arctan(x)}{1 + x^2} dx} = \pi^2/8$$
as desired.

\section{}

\textit{(a) Detrmine }
$$\int{\frac{1}{w (\ln(w))^p} dw}$$.
\textit{Which type of substitution did you use?}

Let $u = \ln(w)$. It follows that $du = \frac{1}{w} dw$. Thus
$$\int{\frac{1}{w (\ln(w))^p} dw} =
\int{\frac{1}{w } (\ln(w))^{-p} dw} =
\int{(u)^{-p} du} =  \frac{u^{-p - 1}}{-p - 1} = \frac{(\ln(w))^{-p - 1}}{-p - 1}
$$.
Given that $p \neq -1$.

Othersise it is $\ln(\ln(w))$.

\textit{(b) Evaluate }
$$I = \int_2^\infty{\frac{1}{w (\ln(w))^p} dw}$$.
\textit{for which values of $p$ is $I$ finite?}

For $p = 1$ we've got that
$$\lim{\ln(\ln(w))} = \infty$$
thus it diverges

For $p \neq -1$ we've got
$$I = \int_2^\infty{\frac{1}{w (\ln(w))^p} dw} =
\lim_{w \to \infty}{\left[\frac{(\ln(w))^{-p - 1}}{-p - 1}\right]} - \frac{(\ln(2))^{-p - 1}}{-p - 1}
$$
The only thing that bothers us is that whether
$$(\ln(w))^{-p - 1} = (\frac{1}{(\ln(w))})^{p - 1}$$
converges. This happens whenever $p > 1$ (in that case we got that $\ln(w) \to \infty$).
Otherwise it diverges.

\section{}

\textit{State a condition that guarantees a function $x = \phi(s)$ has an inverse. Then use
  your condition to decide whether each of the following functions is invertible. When possible,
  find a formula for the inverse of each function that is invertible.}

Such a condition is bijectivity on a given domain and codomain.

\textit{(a)}
$$x = 1/s$$

Is bijective on a domain $R \setminus \{0\}$. Inverse is $s = 1/x$.

\textit{(b)}
$$x = s + s^3$$

This one is bijective (because it is strictly increasing and unbounded below and above).
$$x = s + s^3$$
$$s^3 + s - x = 0$$
Maxima gives some god-awful result for an inverse function, but it can be obtained by solving the
cubic polinomial.

\textit{(c)}
$$x = \frac{s}{1 + s^2}$$

It is not surjective on $R$, and in addition, it is not injective. Thus it cannot be used
without some heavy restrictions on the domain and codomain.

\textit{(d)}
$$x = \sinh s = \frac{e^s - e^{-s}}{2}$$

Looks solid to me.

$$s = \text{asinh}(s)$$
is a desired inverse function;

\textit{(e)}
$$x = \frac{s}{\sqrt{1 - s^2}}$$

Is bijective on restricted domain. I'm not sure if we can have an analytical inverse of this
function.

\textit{(f)}
$$x = ms + b$$

Is a standart linear function. If $m \neq 0$, then we've got inverse on whole $R$.

\textit{(g)}
$$x = cosh(s)$$

Is bijective on restricted domain. Reverse is $s = \text{acosh}(x)$.

\textit{(h)}
$$x = s - s^3$$

Is bijective on restricted domain. Inverse is terrible.


\textit{(i)}
$$x = tanh(s)$$

Also bijective on restricted domain. Inverse is
$$s = \text{atanh}(x)$$

\textit{(j)}
$$x = \frac{1 - s}{1 + s}$$

BIjective on restricted domain.

\section{}

\textit{(a) Obtain formulas for $f(s) = \cos(\arcsin(s))$ and $g(s) = \tan(\arcsin(s))$
  directly as funtions of $s$ that involve neither trigonometric nor inverse trigonometric
  functions. Your answer will involve the square root function and polynomical expressions
  in $s$.}

$$f(s) = \cos(\arcsin(s)) $$
$$\cos(\arcsin(s)) = \sin(\pi/2 - \arcsin(s)) =
\sin(\pi/2)\cos(\arcsin(s)) - \sin(\arcsin(s)) \cos(\pi/2) = $$
$$ = \sin(\pi/2)\cos(\arcsin(s)) = \cos(\arcsin(s)) $$

$$\cos(\arcsin(s)) = \sqrt{1 - \sin^2(\arcsin(s))} = \sqrt{1 - s^2}$$

$$g(s) = \tan(\arcsin(s))$$
$$\tan(\arcsin(s)) = \frac{\sin(\arcsin(s))}{\cos(\arcsin(s))} = \frac{s}{\sqrt{1 - s^2}}$$


\textit{(b) Compute teh derivative of $\cos(\arcsin(s))$ using the chain rule and the
  derivatives of $\cos u$ and $\arcsin s$. Then compute the derivative of $f(s)$ using
  your expressions in part (a). Compare the two derivatives. Do the same for $\tan(\arcsin(s))$
  and $g(s)$}

$$f'(s) = -\sin(\arcsin(s)) \frac{1}{\sqrt{1 - s^2}} = - \frac{s}{\sqrt{1 - s^2}}$$
$$f'(s) = -2s \frac{1}{2\sqrt{1 - s^2}} = - \frac{s}{\sqrt{1 - s^2}}$$
They are the same.

$$g'(s) = \sec^2(\arcsin(s)) \frac{1}{\sqrt{1 - s^2}} =
\frac{1}{\cos^2(\arcsin(s))} \frac{1}{\sqrt{1 - s^2}} =
\frac{1}{(1 - s^2)\sqrt{1 - s^2}} 
$$

$$g'(s) = \frac{\sqrt{1 - s^2} - s(- \frac{s}{\sqrt{1 - s^2}})}{1 - s^2} =
\frac{\sqrt{1 - s^2} + \frac{s^2}{\sqrt{1 - s^2}}}{1 - s^2} =
\frac{\frac{1 - s^2 + s^2}{\sqrt{1 - s^2}}}{1 - s^2} =
\frac{\frac{1}{\sqrt{1 - s^2}}}{1 - s^2} =
\frac{1}{(1 - s^2) \sqrt{1 - s^2}} 
$$

They are also the same.

\section{}

\textit{Use $x = \arcsin s$ to show $\int {\cos^3 x dx} = \sin(x) - \frac{\sin^3 x}{3}$.}

$$s = \sin(x)$$

$$dx = \frac{1}{\sqrt{1 - x^2}}ds$$

$$\int{\cos^3 (x) dx} = \int{\cos^3 (\arcsin(s)) \frac{1}{\sqrt{1 - s^2}} ds} =
\int{(\sqrt{1 - s^2})^3 \frac{1}{\sqrt{1 - s^2}} ds} =
$$
$$
\int{1 - s^2 ds} = s - \frac{s^3}{3} = \sin(x) - \frac{\sin^3(x)}{3}
$$

as desired.

\section{}

\textit{(a) Write the microscope equation (i.e. the linear approximation) for
  $\phi(s) = \sqrt{s}$ at $s = 100$.}

$$\phi'(s) = \frac{1}{2 \sqrt{s}}$$
$$\phi'(100) = \frac{1}{2 \sqrt{100}} = 1/20 = 0.05$$
Thus
$$\Delta \phi = 0.05 \Delta s$$

\textit{(b) Use the microscope equation from part (a) to estimate $\sqrt{102}$ and $\sqrt{99.4}$}

For the first one we've got that
$$\Delta s = |102 - 100| = 2$$
thus
$$\Delta \phi = 0.05 * 2 = 0.1$$
Therefore
$$\phi(102) \approx \phi(100) + 0.1 = 10 + 0.1 = 10.1$$

For the second one we've got
$$\Delta s = |100 - 99.4| = 0.06$$
thus
$$\Delta \phi = 0.06 * 0.05 = 0.03$$
and since $99.4 < 100$ we follow that
$$\phi(99.4) \approx \phi(100) - \Delta \phi = 10 - 0.03 = 9.97$$

\textit{(c) How far are your estimate from those given by a calculator?}

$$\sqrt{102} \approx 10.0995049$$
$$\sqrt{99.4} \approx 9.969955$$
thus we've got  error of approximately $10^{-3}$ in the first case and $10^{-4}$ in the second.

\textit{(d) Your estimates should be greater than the calculator values; use the graph of
  $x = \phi(s)$ to explain why this is so.}

This is because derivative of this function is a  decreasing function around 100.

\section{}

\textit{(a) Write a microscope equation for $\phi(s) = 1/s$ as $s = 2$ and use it to estimate
  $1/2.03$ and $1/98$.}

$$\phi'(s) = -\frac{1}{s^2}$$
thus
$$\phi'(2) = -\frac{1}{4} = -0.25$$

And
$$\Delta \phi = -0.25 \Delta s$$

We can get
$$\Delta s = |2 - 2.03| = 0.03$$
therefore

$$\phi(2.03) \approx 1/2 - 0.25 * 0.03 = 0.4925$$

For the second one we've got
$$\Delta s = |2 - 1.98| = 0.02$$
$$\phi(1.98) \approx 1/2 + 0.25 * 0.02 = 0.505$$

\textit{(b) How far are your estimates from the values given by a calculator?}

$$\phi(2.03) = 0.49261083743842...$$
thus our error is
$$0.00011083743842371652$$

and for the latter we've got
$$\phi(1.98) = 0.5050505050505051...$$
and our error is
$$\approx -5 * 10^{-5}$$

\textit{(c) Your estimates should be lower than the calculator values; use the graph of
  $x = \phi(s)$ to explain why this is so.}

This is because the derivative is increasing, and thus our estimates will be lower.


\section{}

\textit{Show that $\sqrt{1 + 2h} \approx 1 + h$ when $h \approx 0$}

Let
$$f(h) = \sqrt{1 + 2h}$$
$$g(h) = 1 + h$$
then
$$f'(x) = \frac{1}{\sqrt{1 + 2h}}$$
and thus
$$f(0) = 1$$
$$f(0) = 1$$
$$f'(0) = 1$$
$$g'(0) = 1$$
Given that the function is contionus and differentiable around zero, we can conclude by the
same reasoning as in our  "microscope equation" section, that functions are approximately
equal in this range.

In case if we've got some shenannigans happening right after the 0, I've looked at the graphs
of those functions side by side in the range $[-0.1, 0.1]$ and concluded that they look pretty
simular at this range.

\section{}

\textit{(a) Determine the microscope equation for $x = \tan(s)$ at $s = \pi/4$}

$$x' = \sec^2(s)$$
$$x'(0) = 1$$
thus
$$\Delta x \approx \Delta s$$

\textit{(b) Show that $\tan(h + \pi/4) \approx 1 + 2h$ when $h \approx 0$. Is this estimate larger
  or smaller than the true value? Explain why.}

Their derivatives are equal at this point, and they are equal at zero too. Thus we can conclude
that they are simular around zero.

Also, I don't know why it didn't cross my mind earlier, but it also happens
because $1 + 2h$ is a linear approximation of this function at this point.

This estimate is lower then the true value, because the derivative of the function is increasing
around zero.

\section{}

\textit{Determine the local lentgth multiplier for $x = \sin(s)$ at each of the
  points $\{0, \pi/4, \pi/2, 2\pi/3, \pi\}$}

$$x' = \cos(x)$$
$$x'(0) = 1$$
$$x'(\pi/4) = 1/\sqrt{2}$$
$$x'(\pi/2) = 0$$
$$x'(2 \pi/3) = -1/2$$
$$x'(\pi) = -1$$

\section{}

\textit{What is true about the map $\phi: s \to x$ at a point $s_0$ where the local length
  multiplier is negative?}

It means that the funciton is decreasing

\section{}

\textit{Consider the hyperbolic sine and hyperbolic cosine functions, $\sinh s$ and
  $\cosh s$. Show each is the derivative of the other, and show }
$$\cosh^2 s - \sinh^2 s = 1$$
\textit{for all $s$}

$$f(x) = \cosh(x) = \frac{e^x + e^{-x}}{2}$$
by algebraic properties of the derivative and the chain rule we've got that
$$f'(x) = \frac{1}{2}(e^x - e^{-x}) = \sinh(x)$$

$$g(x) = \sinh(s) = \frac{e^x - e^{-x}}{2}$$
same applies to this one
$$g'(x) = \frac{1}{2}(e^x +  e^{-x}) = \cosh(x)$$

$$\cosh^2(x) - \sinh^2(x) = \left(\frac{e^x + e^{-x}}{2} \right)^2 -
\left(\frac{e^x - e^{-x}}{2} \right)^2 = 
\frac{(e^x + e^{-x})^2 - (e^x - e^{-x})^2}{4} = 
$$
$$ =
\frac{e^{2x} + 2e^xe^{-x} + e^{-2x} - (e^{2x} - 2e^xe^{-x} + e^{-2x})}{4} =
$$
$$ =
\frac{4e^xe^{-x}}{4} = \frac{4e^0}{4} = \frac{4}{4} = 1
$$
as desired.

\section{}

\textit{Use the substitution $x = \sinh s$ to determine }
$$\int{\frac{dx}{\sqrt{1 + x^2}}}$$

Let $x = \sinh s$. It follows that
$$\frac{dx}{ds} = \cosh(s)$$
$$dx = \cosh(s) ds$$
(just a reminder: this is not a  rigorous equation, just a useful mneumonic)

Thus we follow that
$$\int{\frac{dx}{\sqrt{1 + x^2}}}
= \int{\frac{\cosh(s) ds}{\sqrt{1 + (\sinh( s))^2}}}
$$
we know that 
$$\cosh^2 (x) - \sinh^2(x) = 1 \to \cosh^2 (x) = 1 + \sinh^2(x)$$
thus
$$\int{\frac{\cosh(s) ds}{\sqrt{1 + (\sinh( s))^2}}} =
\int{\frac{\cosh(s) ds}{\cosh(s)}} = \int{ds} = s = \text{asinh}(x)$$

\section{}

\textit{Determine the work done by the constant force $F = (2, -3)$ in displacing an
  object along}

\textit{(a) $\Delta x = (1, 2)$}

$$\Delta x \cdot F = 2 - 6 = -4$$

\textit{(b) $\Delta x = (1, -2)$}

$$\Delta x \cdot F = 2 + 6 = 8$$

\textit{(c) $\Delta x = (-1, 0)$}

$$\Delta x \cdot F = 2 $$

\section{}

\textit{Determine the work done by the constant force $F = (7, -1, 2)$ in displacing an
  object along}

\textit{(a) $\Delta x = (0, 1, 1)$}

$$\Delta x \cdot F = 0 - 1 + 2 = 1 $$

\textit{(b) $\Delta x = (1, -2, 0)$}

$$\Delta x \cdot F = 7 + 2 + 0 = 9$$

\textit{(c) $\Delta x = (0, 0, 1)$}

$$\Delta x \cdot F = 0 + 0 + 2  = 2$$

\section{}

\textit{Suppsoe a constant force $F$ in the place does $7$ units of work in displacing an
  object along $\Delta x = (2, -1)$ and $-3$ units of work along $\Delta x = (4, 1)$. How
  much work does $F$ do in displacing an object along $\Delta x = (1, 0)$? Along
  $\Delta x = (0, 1)$? Find a nonzero displacement $\Delta x$ along which $F$ does no work.}

We can follow that $F = (v_1, v_2)$ for which it will be true that
$$
\begin{cases}
  4 v_1 + v_2 = 7 \\
  2 v_1 - v_2 = -3
\end{cases}
$$

$$
\begin{cases}
  3v_2  = 13 \\
  2 v_1 - v_2 = -3
\end{cases}
$$

$$
\begin{cases}
  v_2  = 4 + 1/3 \\
  2 v_1 - 4 - 1/3 = -3
\end{cases}
$$

$$
\begin{cases}
  v_1 = 2/3 \\ 
  v_2  = 13/3 \\
\end{cases}
$$

$$F = (2/3, 13/3)$$

Thus for $\Delta x = (1, 0)$ we've got
$$W  = 2/3$$
and for $\Delta x = (0, 1)$ we've got
$$W  = 13/3$$

In order to have a desired vector we've gotta have $v = (x_1, x_2)$ such that $F \cdot v = 0$.
Applying some formulas we get 
$$2/3 x_1 + 13/3 x_2 = 0$$
$$2 x_1 + 13 x_2 = 0$$
$$x_1 = -6.5 x_2$$
Thus we can get $v = (1, -6.5)$ with the desired properties.

\section{}

\textit{Let $W(F, \Delta x)$ be the work done by the constant force $F$ along the linear
  displacement $\Delta x$. Show that $W$ is a linear function of the vectors $F$ and
  $\Delta x $.}

Given that $W = F \cdot \Delta x$ and we are working with the real numbers here only,
we can follow, that by properties of the inner product (specifically linearity in the first slot
and conjugate symmetry) we've got the desired linearity in both slots, as desired.

\section{}

\textit{Suppose $F = (P, Q)$. Determine the unit displacement $\Delta u$
  that yield the maximum and minimum values of $W$.}

$$u = (P, Q) \frac{1}{\sqrt{P^2 + Q^2}}$$
gives the maximum. $-u$ will give us minimum.

\section{}

\textit{Suppose the constant force $F = (P, Q)$ does the work $A$ along the displacement
  $(a, c)$ and the work $B$ along the displacement $(b, d)$. Determine $P$ and $Q$. What
  condition (on $a, b, c$ and $d$) must be satisfied for $P$ and $Q$ to be found?}

$$W_1 = (P, Q) \cdot (a, c) = Pa + Qc$$
$$W_2 = (P, Q) \cdot (b, d) = Pc + Qd$$

$$P = \frac{W_1 - Qa}{c}$$
$$P = \frac{W_2 - Qb}{d}$$

$$\frac{W_1 - Qa}{c} = \frac{W_2 - Qb}{d}$$
$$\frac{W_1}{c} - Q \frac{a}{ c} = \frac{W_2}{d} - Q\frac{b}{d}$$
$$\frac{W_1}{c} - \frac{W_2}{d}  =  - Q\frac{b}{d} + Q \frac{a}{ c}$$
$$\frac{W_1}{c} - \frac{W_2}{d}  =   Q \frac{a}{ c} - Q\frac{b}{d} $$
$$\frac{W_1}{c} - \frac{W_2}{d}  =   Q \left( \frac{a}{ c} - \frac{b}{d} \right) $$
$$\frac{W_1 d - W_2 c}{cd}  =   Q  \frac{ad - bc }{ cd} $$
$$\frac{W_1 d - W_2 c cd}{cd (ad - bc)}  =   Q   $$
$$
\begin{cases}
  Q = \frac{W_1 d - W_2 c }{ad - bc} \\
  P = \frac{W_1 - \frac{W_1 d - W_2 c }{ad - bc} a}{c}
\end{cases}
$$

We need $(a, c)$ and $(b, d)$ to be linearly independent. In this particular case we need
the $(a, c) \neq \lambda (b, d)$ for some $\lambda \neq 0 \in R$.


\section{}

\textit{(a) Sketch the curve in the $(x, y)$-plane given parametrically as }
$$x = \frac{2t}{1 + t^2}$$
$$y = \frac{1 - t^2}{1 + t^2}$$

Done it in graphing software, it's a circle around zero; TODO: add pictures

\textit{(b) Each of the following limits exists; determine the loration of each as a
  point in the (x, y)-plane: }

$$\lim_{t \to \infty}{(x(t), y(t))} = (0, -1)$$
$$\lim_{t \to -\infty}{(x(t), y(t))} = (0, -1)$$

\textit{(c) Compute $\alpha = x(t)^2 + y(t)^2$; your result should be constant; What is the curve
  and how does $\alpha$ relate to it?}

once again, it's a circle; $\alpha$ is a square of the radius;

$$\alpha = \frac{4t^2}{1 + 2t^2 + t^4} + \frac{1 - 2t^2 + t^4}{1 + 2t^2 + t^4} =
\frac{1 + 2t^2 + t^4}{1 + 2t^2 + t^4} = 1$$

\section{}

\textit{Determine the work done by the force field $F$ in moving a particle along the
  oriented curve $C$, where }

\textit{(a) $F = (x, 3y)$, $C = (t^2, t^3)$, $1 \leq t \leq 2$}

$$W = \int_1^2{F(x(t)) \cdot x'(t)dt} = \int_1^2{(t^2, 3t^3) \cdot (2t, 3t^2)dt} =
\int_1^2{2t^3 + 9t^5dt} = \left[\frac{t^4}{2} + \frac{3t^6}{2}\right]_1^2 =$$
$$=
(8 + 96) - (1/2 + 3/2) = 104 - 2 = 102$$

\textit{(b) $F = (-y, x)$, $C$: semicircle of radius 2 at origin, counterclockwise from
  $(2, 0)$ to $(-2, 0)$}

firstly, we gotta move the description of $C$ into sensible domain.
Given that it is a semicircle and whatnot, we get that
$$x = -\sin(t)$$
$$y = \cos(t)$$
for $0 < t < \pi$. Thus $x(t) = (-\sin(t), \cos(t))$ and $x'(t) = (-\cos(t), -\sin(t))$.

Therefore
$$W = \int_0^\pi{F(x) \cdot x'(t) dt} =
\int_0^\pi{(-\cos(t), -\sin(t) \cdot (-\cos(t), -\sin(t)) dt} =$$
$$ = 
\int_0^\pi{\cos^2(t) + \sin^2(t)) dt} =  \int_0^\pi{ 1 dt} =  [x]_0^\pi = \pi$$

\textit{(c) $F=(y, x)$, $C$: any path from (5, 2) to (7, 11)}

We'll do the simpliest: $x(t) = (5 + 2t, 2 + 9t)$ for $0 < t < 1$
Thus $x'(t) = (2, 9)$. Therefore
$$W = \int_0^1{F(x(t)) \cdot x'(t) dt} =  \int_0^1{(2 + 9t, 5 + 2t) \cdot (2, 9) dt} = $$
$$=
\int_0^1{4 + 18t + 25 + 18t dt} = \int_0^1{29 + 36t dt} = [29t + 18t^2]_0^1 = 29 + 18 = 47$$

\textit{(d) $F = (0, 0, -mg)$, $C = (2t, t, 4 - t^2)$, $0 \leq t \leq 1$.}

$$W = \int_0^1{F(x(t)) \cdot x'(t) dt} =
\int_0^1{(0, 0, -mg) \cdot (2, 1, -2t) dt} =$$
$$ =
\int_0^1{2mgt dt} = 2mg \int_0^1{t dt} = 2mg [t^2/2]_0^1 = 2mg(1/2) = mg$$

\textit{(e) $F = (-y, x, 1)$, $C = (\cos t, \sin t, 3 t)$, $0 \leq t \leq A$.}

$$W = \int_0^A{F(x(t)) \cdot x'(t) dt} =
\int_0^A{(- \sin t, \cos t, 1) \cdot (- \sin t, \cos t, 3) dt} = $$
$$ =
\int_0^A{\sin^2(t) + \cos^2(t) + 3 dt} =
\int_0^A{4 dt} = [4x]_0^A = 4A
$$

\section{}

\textit{Determine $\int_C{F \cdot dx}$ when}

\textit{(a) $F = (x + 2y, x - y)$, $C: $ straight line from $(-2, 3)$ to $(1, 7)$.}

$$C = (-2 + 3t, 3 + 4t) $$
for $0 \leq t \leq 1$. Thus
$$\int_0^1{F \cdot dx} = \int_0^1{F(x(t)) \cdot x'(t) dt} =
\int_0^1{F(x(t)) \cdot x'(t) dt} = $$
$$ = \int_0^1{(-2 + 3t + 6 + 8t, -2 + 3t - 3 - 4t) \cdot (3, 4) dt} =
\int_0^1{(4 + 11t, -5 - t) \cdot (3, 4) dt} = 
$$
$$
=  \int_0^1{12 + 33t - 20  - 4t dt} =  \int_0^1{- 8 + 29tdt} = [-8t + 29t^2/2]_0^1 = -8 + 29/2 =
13/2
$$

\textit{(b) $F = (xy, z, x)$, $C = (t^2, t, 1 - t)$, $0 \leq t \leq 1$.}

$$\int_0^1{F \cdot dx } = \int_0^1{F(x(t)) \cdot x'(t) dt } =
\int_0^1{(t^3, 1 - t, t^2) \cdot (2t, 1, -1) dt } = $$
$$ =
\int_0^1{2t^4 + 1 - t - t^2dt } = [2t^5/5 + t - t^2/2 - t^3/3]_0^1 = 2/5 + 1 - 1/2 - 1/3 = 17/30
$$

\textit{(c) $F = (\frac{-y}{x^2 + y^2}, \frac{x}{x^2 + y^2})$, $C = (R \cos t, R \sin t)$
  $0 \leq t \leq 8\pi$}

$$\int_0^{8 \pi}{F \cdot dx } = \int_0^{8 \pi}{F(x(t)) \cdot x'(t) dt } =
\int_0^{8 \pi}{((-R \sin t)/R^2, (R \cos t)/R^2) \cdot (-R \sin t, R \cos t) dt } = 
$$
$$ =
\int_0^{8 \pi}{((- \sin t)/R, ( \cos t)/R) \cdot (-R \sin t, R \cos t) dt } =
\int_0^{8 \pi}{\sin^2(t) + \cos^2(t) dt} = \int_0^1{1 dt} = [t]_0^{8\pi} = 8 \pi
$$

\section{}

\textit{Let $C$ be the semicircle of radius 2 centered at the origin, oriented
  counter-clockwise from $(0, -\sqrt{3})$ to $(0, \sqrt{3})$}

\textit{(a) Show that
  $$r(u) = \left(\frac{4u}{u^2 + 1}, \frac{2u^2 - 2}{u^2 + 1}\right),
  2 - \sqrt{3} \leq u \leq 2 + \sqrt{3}$$
parametrizes $C$. }

It doesn't. Error in the exericise; skipping;


\section{}

\textit{(a) In a coordinate system $(x, y, z)$ where the z-axis is vertical, the gravitational
  force field at the surface of the eqarth can be written as $F = (0, 0, -gm)$, where $g$ is the
  acceleration due to gravity and $m$ is the mass of a falling object. Show that $\Phi(x, y, z) =
  -gmz$ is a potential function for $F$, demonstrating that $F$ is a conservative field.}

We can follow that
$$\frac{\partial}{\partial x}\Phi = 0$$
$$\frac{\partial}{\partial y}\Phi = 0$$
$$\frac{\partial}{\partial z}\Phi = -gm$$
thus we follow that
$$F = \Delta \Phi$$
therefore $\Phi$ is a potential function for $F$.

\textit{(b) What is the work done by gravity in moving an object of mass $m$ from
  the point $(a, b, c)$ to $(\alpha, \beta, \gamma)$? Is this negative if $c < \gamma$?
  What is the meaning of "negative" work?}

We follow that
$$W = -mg(\gamma - c)$$
It is negative if $c < \gamma$. By negative work we mean that it takes work to move the particle
through this particular path.

\textit{(c) What is the net work done by gravity in moving an object of mass $m$ from the point
  $(a, b, c)$ to another point $(\alpha, \beta, c)$ at the same vertical height as the first? }

It's zero.

\section{}

\textit{If $x = (x, y, z)$ is the position of a planet in terms of a coordinate system
  centered at the sum, then the force of the sun's gravity on the planet is given by
  $F(x) = \mu x/r^3$, where $\mu$ is a constant, and $r = ||x||$.}

\textit{(a) Write $F$ explicitly in terms of the space variables $x, y$ and $z$}

$$F(x, y, z) =(- \mu x/(x^2 + y^2 + z^2)^{3/2},
- \mu y/(x^2 + y^2 + z^2)^{3/2}, - \mu z/(x^2 + y^2 + z^2)^{3/2})$$

\textit{(b) Show that the gravitational force obeys the "inverse square" law:
  $$||F|| = \mu/r^2$$}

I'm assuming that $\mu \geq 0$.

$$||F|| = \sqrt{\mu^2 x^2/(x^2 + y^2 + z^2)^{3} + \mu^2 y^2/(x^2 + y^2 + z^2)^{3}  + 
  \mu^2 z^2/(x^2 + y^2 + z^2)^{3}} =  $$
$$
= \sqrt{\sum_{i \in \{x, y, z\}}{\mu^2 i^2/(x^2 + y^2 + z^2)^{3}}} =
\mu \frac{1}{(x^2 + y^2 + z^2)^{3/2}} \sqrt{\sum_{i \in \{x, y, z\}}{ i^2}} = 
$$
$$ =
\frac{\mu}{r^3}  r = \frac{\mu}{r^2}
$$
as desired.

\textit{(c) Write $\Phi(x) = \mu /r$ explicitly in terms of the space variable, and show that $\Phi$
  is a potential for $F: \text{ grad } \Phi = F$. This demonstrates that the gravitational
  field is conservative.}

$$\Phi(x) = \mu / \sqrt{x^2 + y^2 + z^2}$$

for $i \in \{x, y, z\}$
$$\frac{\partial}{\partial i} = \frac{- \mu i}{(x^2 + y^2 + z^2)^{3/2}} = \frac{\mu i}{r}$$
thus we follow that $\text{grad} \Phi = F$, therefore making $F$ conservative.

\textit{(d) Suppose we choose a unit for distance in such a way that $r = 10$ when our
  planet is farthest from the sun and $r = 3$ when it is closest to the sun. How much work
does the sun's gravitational field do in moving the planet from aphelion to perihelion?}

It does
$$\left[\Phi(r)\right]_{10}^3 = (\mu/10 - \mu/3) = 7/30 \mu$$

\textit{(e) What is the net work done on the planet by the sun's gravity when the planet
  traverses one complete orbit? }

Zero, it's conservative.

\section{}

\textit{Determine the arc length of each of the following curves.}

\textit{(a) $x(t) = (3t^2, 4t^2), 0 \leq t \leq 1$.}

$$\int_c{ds} = \int_0^1{||x'(t)|| dt} = \int_0^1{||(6t, 4t)|| dt} =
\int_0^1{\sqrt{36t^2 + 64t^2} dt} = \int_0^1{\sqrt{100t^2} dt} =
\int_0^1{10t dt} = [5t^2]_0^1 = 5$$

\textit{(b) $x(t) = (t^2, t^3), 1 \leq t \leq 3$}

$$x'(t) = (2t, 3t^2)$$
$$||x'(t)|| = \sqrt{4t^2 + 9t^4} = t\sqrt{4 + 9t^2}$$
thus
$$\int_1^3{t\sqrt{4 + 9t} dt} = \int_1^3{t\sqrt{4 + 9t^2} dt} =
$$
$$= 
\int_{13}^{85}{1/ 18 u'\sqrt{u} dt} = \int_{13}^{85}{1/ 18 \sqrt{u} du} =
\left[2 * u^{3/2}/54\right]_{13}^{85} = 1/27(85^{3/2} - 13^{3/2}) \approx 27.28848564032828$$

\textit{(c) $x(t) = (e^t \cos t, e^t \sin t), a \leq t \leq b$.}

$$x'(t) = (e^t \cos t - e^t \sin t, e^t \cos t + e^t \sin t)$$
$$||x'(t)|| = \sqrt{e^t(\cos^2 - 2 \sin \cos + \sin^2 + \cos^2 + 2 \cos \sin + \sin^2)} =
\sqrt{e^t(1 +  1 )} =  \sqrt{2} e^{t/2} $$
thus
$$\int_a^b{\sqrt{2} e^{t/2} dt} = \sqrt{2} [2  * e^{t/2}]_a^b = 2 \sqrt{2} (e^{b/2} - e^{a/2})$$

\textit{The rest of this exercise is skipped.}

\section{}

\textit{(a) Determine the arc-length function $s(t)$ for the circle $C$ of radius $R$
  parametrized as $x(t) = (R \cos t, R \sin t)$.}


$$x'(t) = (- R \sin t, R \cos t)$$
$$|| x'(t) || = |R| \sqrt{ \sin^2 + \cos^2 } = |R|$$
thus
$$l(t) = |R|t$$

\textit{(b) Determine the inverse $t = \sigma(s)$ of the arc-length function and then the
  corresponding arc-length parametrization $y(s) = x( \sigma(s))$}

$$\sigma(s) = l(t)/|R| = s/|R|$$
then we follow that
$$y(s) = (R \cos (s/|R|), R \sin (s/|R|))$$

\section{}

\textit{Determine the arc-length function $s(t)$ (with $s(0) = 0$) and the arc-length
  parametrization $y(s)$ of the curve parametrized as}
$$x(t) = (\frac{1 - t^2}{1 + t^2}, \frac{2t}{1 + t^2})$$

we can follow that
$$x'(t) = (\frac{-2t(1 + t^2) - (1 - t^2)2t}{(1 + t^2)^2},
\frac{2(1 + t^2) - 2t(2t)}{(1 + t^2)^2})$$
$$x'(t) = (\frac{-2t -2t^3 - 2t + 2t^3}{(1 + t^2)^2},
\frac{2 + 2t^2 - 4t^2}{(1 + t^2)^2})$$
$$x'(t) = (\frac{-4t}{(1 + t^2)^2},
\frac{2 - 2t^2)}{(1 + t^2)^2})$$
thus
$$||x'(t)|| = \frac{1}{(1 + t^2)^2}\sqrt{16t^2 + 4 - 8t^2 + 4t^4} =
\frac{1}{(1 + t^2)^2}\sqrt{8t^2 + 4 + 4t^4} = \frac{1}{(1 + t^2)^2}\sqrt{4(t^2 + 1)^2} = $$
$$
= \frac{1}{(1 + t^2)^2}2(t^2 + 1) = 2/(1 + t^2)
$$
thus
$$s(t) = \int_0^t{2/(1 + x^2) dx} = 2 \arctan{t}$$
thus we follow that
$$\sigma(s) = \tan(s/2)$$
thus
$$y(s) = (\frac{1 - \tan(s/2)^2}{1 + \tan(s/2)^2}, \frac{2\tan(s/2)}{1 + \tan(s/2)^2})$$
as desired.

\section{}

\textit{Let $C$ be a thin wire formed into the circle of radius $R$ cm centered at the
  origin. Suppose the mass density of the wire at the point $(x, y)$ is $1 + x^2$ gm/cm.
  Determine the total mass of the wire.}

We can follow that parametrization of path is
$$x(t) = (R \cos t, R \sin t)$$
thus we've got that
$$M = \int_0^{2 \pi}{(1 + (R \cos(t))^2, 0) ||x'(t)|| dt}=
\int_0^{2 \pi}{(1 + (R \cos(t))^2) R dt}=  \int_0^{2 \pi}{R + R^3 \cos(t))^2 dt}=$$
$$ = 
2\pi R + R^3[\sin(2x)/4 + x/2]_0^{2\pi} = 2\pi R + R^3\pi$$

\section{}

\textit{Let $C$  be the helix $(x, y, z) = (\cos t, \sin t, t), 0 < t < 4\pi$, and
  let $s$ be arc length on $C$. Determine }
$$\int_C{z ds} \text{ and } \int_C{z^2 ds}$$

$$x'(t) = (-\sin t, \cos t, 1)$$
$$||x'(t)|| = \sqrt{\sin^2 + \cos^2 + 1} = \sqrt{2}$$

$$\int_c{z ds} = \int_0^4{t ||x'(t)|| dt} = \int_c{t \sqrt{2} dt} = [\sqrt{2}t^2/2]_0^{4\pi} =
8\pi^2\sqrt{2}  $$

$$\int_C{z^2 ds} = \int_0^{4\pi}{t^2 \sqrt{2} dt} = \sqrt{2}[t^3/3]_0^{4\pi} = \sqrt{2}64 \pi^3/3$$

\section{}

\textit{Let $C$ be the circle of radius 5 centered at the point $(4, -3)$, and let $s$ be the
  arc-length parameter along $C$, as measured counterclockwise from the origin. Propose a
  definition for the path integrals}
$$\oint{\cos {s} ds } \text{ and } \oint{\cos^2{s} ds}$$
\textit{and then determine their values.}

We can set $\oint = \int_C$, where the start and the end of $C$ are the same (i.e. it completes
the whole path).

Thus we can follow that length of the whole path is $10 \pi$ and thus

$$\oint_C{\cos{s} ds} = \oint_0^{10\pi}{\cos{s} ds} = [\sin s]_0^{10 \pi} = 0 $$
$$\oint_C{\cos^2{s} ds} = \oint_0^{10\pi}{\cos^2{s} ds} = [\sin (2s)/4 + s/2 ]_0^{10 \pi} = 5 \pi $$

\section{}

\textit{Is the change from Cartesian to polar coordinates either a pullback or a push-forward
  substitution, or is it some new type?}

The change from Cartesian to polar coordinates are when we substitute
$$x = r \sin\theta $$
and
$$y = r \cos \theta$$
thus we can conclude that it is pull-back substitution.

\section{}

\textit{(a) Sketch the region $D$ that lies in the first quadrant in the $(x,y)$-plane
  between the circles $x^2 + y^2 = 1$ and $x^2 + y^2 = 10$.}

TODO: add sketch.

\textit{(b) Describe $D$ in polar coordinates.}

We can follow that
$$x = r \sin \theta$$
$$x = y \cos \theta$$
for $0 \leq \theta \leq \pi/2$ and $1 \leq r \leq \sqrt{10}$.

\textit{(c) Change to polar coordinated to evaluate the double integral}
$$\iint_D{\sin{(x^2 + y^2)} dx dy}=
\int_0^{\pi/2}{\int_1^{\sqrt{10}}{\sin{(r^2)r dr}  d\theta}}= [-\cos{r^2}/2]_1^{\sqrt{10}} = $$
$$=  [1/2(\cos(1) - \cos(10))]_0^{\pi/2} = \pi/4(\cos(1) - \cos(10)) $$

\section{}

\textit{Let $g_{\mu, \sigma}(x) = e^{-(x - \mu)^2/2\sigma^2}$, as in the text}

\textit{(a) Show $g_{\mu, \sigma}$ takes its maximum  at $x = \mu$ and  the graph of
  $g_{\mu, \sigma}$ has inflection points at $x = \mu \pm \sigma$. Sketch the graph
  of $z = g_{\mu, \sigma}(x)$ for $\mu - 3 \sigma \leq x \leq \mu + 3\sigma$. Do this first
  with $\mu = 5$ and $\sigma = 2$ and then symbolically with general values for $\mu$ and \
  $\sigma$.}

$$g_{\mu, \sigma}'(x) = \frac{(x - \mu)}{\sigma^2} e^{-(x - \mu)^2/2\sigma^2} =
\frac{(x - \mu)}{\sigma^2} g_{\mu, \sigma}(x)$$
$g_{\mu, \sigma}'(x)$ has critical points at $x = \mu$ since $e^{-(x - \mu)^2/2\sigma^2} > 0$. From
the graph we can follow that it's a maximum.

$$g_{\mu, \sigma}''(x) = \frac{(x - m)^2}{\sigma^4}g_{\mu, \sigma}(x) - \frac{1}{\sigma^2}g_{\mu, \sigma}(x)$$

for which critical points are $x = m \pm s$.

\textit{Gonna skip (b); TODO }

\section{}

\textit{This exercise is a graphing exercise, TODO: add pictures}

\section{}

\textit{Suppose $Z_{0, 1}$ is a normal random variable with mean 0 and stndard deviation 1.
  Continue to assume $X_{\mu, \sigma}$ is a normal random variable with mean $\mu$ and
  standard deviation $\sigma$. Show that }
$$\prob(0 \leq X_{\mu, \sigma} \leq b) = \prob(0 \leq Z_{0, 1} \leq (b - \mu) / \sigma)$$

What we're esentially need to show is that 

$$ \prob(0 \leq X_{\mu, \sigma} \leq b) = \frac{1}{\sigma 2 \sqrt{\pi}}\int_0^b{e^{-(x - \mu)^2/2\sigma^2} dx} = $$
$$ = 
\prob(0 \leq Z_{0, 1} \leq (b - \mu) / \sigma) = \frac{1}{2 \sqrt{\pi}} \int_0^{(b - \mu)/\sigma}{e^{-(x)^2/2} dx}$$

Let $u = (x - \mu)/\sigma$. Then we follow that $du = 1/\sigma dx \to dx = \sigma du$. Thus
$$ \prob(0 \leq X_{\mu, \sigma} \leq b) =
\frac{1}{\sigma 2 \sqrt{\pi}} \int_0^b{e^{-(x - \mu)^2/2\sigma^2} dx} =
\frac{1}{\sigma 2 \sqrt{\pi}} \int_0^{(b - \mu)/\sigma}{e^{-(x - \mu)^2/2\sigma^2} \sigma du} =
$$
$$ = 
\frac{1}{\sigma 2 \sqrt{\pi}} \int_0^{(b - \mu)/\sigma}{\exp{(\frac{-(x - \mu)^2}{2\sigma^2})}
  \sigma du} = \frac{1}{\sigma 2 \sqrt{\pi}}
\int_0^{(b - \mu)/\sigma}{\exp{(- (\frac{x - \mu}{\sqrt{2}\sigma})^2)} \sigma du} =
$$
$$ = 
\frac{1}{\sigma 2 \sqrt{\pi}}
\int_0^{(b - \mu)/\sigma}{\exp{(- \frac{1}{2} (\frac{x - \mu}{\sigma})^2)} \sigma  du} =
$$
$$ = 
\frac{1}{\sigma 2 \sqrt{\pi}}
\int_0^{(b - \mu)/\sigma}{\exp{(- \frac{1}{2} (u)^2)} \sigma  du} =
\frac{1}{\sigma 2 \sqrt{\pi}}
\sigma \int_0^{(b - \mu)/\sigma}{\exp{(- \frac{1}{2} (u)^2)}  du} =
$$
$$ = 
\frac{1}{ 2 \sqrt{\pi}}
\int_0^{(b - \mu)/\sigma}{\exp{(- \frac{1}{2} (u)^2)}  du} =
\frac{1}{ 2 \sqrt{\pi}} \int_0^{(b - \mu)/\sigma}{e^{-x^2/2}  dx} =
\prob{(0 \leq Z_{0, 1} \leq (b - \mu)/\sigma)}
$$

as desired.

\section{}

\textit{For simplicity, we assumed that $a > 0$ when we reduced probabilities for
  $X_{\mu, \sigma}$ to certain z-scores. This assumption is not neccessary; describe how
  to remove it}

We need to use the fact that for $a < 0 < b$ we've got that
$$\int_a^b{f(x) dx} = \int_0^b{f(x) dx} + \int_a^0{f(x) dx} =
\int_0^b{f(x) dx} - \int_0^{-a}{f(x) dx} $$

If we add the necessary numbers, we'll gret the desired result.

\end{document}

%%% Local Variables:
%%% mode: latex
%%% TeX-master: t
%%% End:
