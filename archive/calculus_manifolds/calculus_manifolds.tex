\documentclass[11pt,oneside,titlepage]{book}
\title{My "Calculus on Manifolds" exercises}
\usepackage{amsmath, amssymb}
\usepackage{geometry}
\usepackage{hyperref}
\author{Evgeny Markin}
\date{2023}

\DeclareMathOperator \map {\mathcal {L}}
\DeclareMathOperator \pow {\mathcal {P}}
\DeclareMathOperator \topol {\mathcal {T}}
\DeclareMathOperator \basis {\mathcal {B}}
\DeclareMathOperator \ns {null}
\DeclareMathOperator \range {range}
\DeclareMathOperator \fld {fld}
\DeclareMathOperator \inv {^{-1}}
\DeclareMathOperator \Span {span}
\DeclareMathOperator \lra {\Leftrightarrow}
\DeclareMathOperator \eqv {\Leftrightarrow}
\DeclareMathOperator \la {\Leftarrow}
\DeclareMathOperator \ra {\Rightarrow}
\DeclareMathOperator \imp {\Rightarrow}
\DeclareMathOperator \true {true}
\DeclareMathOperator \false {false}
\DeclareMathOperator \dom {dom}
\DeclareMathOperator \ran {ran}
\newcommand{\eangle}[1]{\langle #1 \rangle}
\newcommand{\set}[1]{\{ #1 \}}
\newcommand{\qed}{\hfill $\blacksquare$}


\begin{document}
\maketitle
\tableofcontents

\chapter{Functions on Euclidean Space}

\section{Norm and Inner Product}

\subsection*{1-1}

\textit{Prove that
  $$|x| \leq \sum_{i = 1}^n{|x^i|}$$}

We follow that both rhs and lhs are nonnegative, thus we follow that
$$|x| \leq \sum_{i = 1}^n{|x^i|} \lra |x|^2 \leq \left(\sum_{i = 1}^n{|x^i|}\right)^2
\lra \sum{|x^i|^2} \leq \left(\sum_{i = 1}^n{|x^i|}\right)^2$$
We can follow that terms in rhs contain lhs, and since all the terms are nonnegative, we
deduce the desired result. More rigorous result can be obtained through some basic induction.

\subsection*{1-2}

\textit{Whn does equality hold in Theorem 1-1(3)?}

When one vector is a scalar product of the other.

\textit{Other exercises were already handled in earlier courses. 1-3, 1-4, 1-5 at readl analysis
  and linear algebra, 1-6 is undetermined, 1-7 is just isometry thing, and everythng else in
  the mix of those two.}

\section{Subsets of Euclidean Space}

\textit{1-14 and 1-15 were handled in a topology course.}

\subsection*{1-16}

First is closed, second is also closed, third one is $R^n$.

\subsection*{1-17}

Diagonal except for $\eangle{0, 0}$ and $\eangle{1, 1}$ will do

\subsection*{1-18}

We follow that $A$ is open, and thus equal to its interior. We can also follow that
$\overline{A} = [0, 1]$, and thus we conclude the desired result.

\subsection*{1-19}

There are sequences of rationals that converge to any irrational number, thus irrationals
are limit points, which produces this result.

\subsection*{1-20}

TBD in topology course pretty soom

\subsection*{1-21}

(a) If there's no such nnomber, then there's a sequence in $A$ that converges to $x$, thus
$x$ is a limit point of $A$ and thus it's contained in $A$.

(b) and (c) skip

\subsection*{1-22}

Each $c \in C$ has got a basis neighborhood inside $U$. Each one of those basis neighborhoods
have smaller basis neighborhoods inside of them. Thus we can create a function $f: C \to \pow(U)$
to those small neighborhoods, then take a closure of the union of the range of $f$,
and this will produce the desired set.

\section{Functions and Continuity}

\subsection*{1-23}

Follows from the definition of product topology. Also was probably handled with a case
of 2 abstract spaces in procut topology and can be extended to this case
by induction

\subsection*{1-24}

handled in topology course.

\textit{ the rest was taken care of in linear algebra course or somewhere else}

\chapter{Differentiation}

\section{Basic Definitions}

\subsection*{2-1}

\textit{Prove that if $f$ is differentiable at $a \in R^n$, then it is continous at $a$.}

We can screw around with original definition of continuity to get
$$\lim \frac{|f(a + h) - f(a) - (\lambda(a + h) - \lambda(a))|}{|h|} =
\lim \frac{|(f - \lambda)(a + h) - (f - \lambda)(a)|}{|h|} = $$
$$ = \lim_{x \to a}
\frac{|(f - \lambda)(x) - (f - \lambda)(a)|}{|x - a|} = 0$$

Let $B$ be a basis element around $f(a)$. Let $x \in f\inv[B]$. We follow that there's
a ball $B'$ around $x$ such that $y \in B' \ra \lambda(y) \in \lambda(B')$.
Using metrics we get
$$|y - a| < \delta \ra |\lambda(y) - \lambda(a)| < \epsilon$$
Since function 

\subsection*{2-2}

We follow that we ca define
$$g(x) = f(x, 0)$$
and we follow that if $q = \eangle{a, b} \in R^2$, then
$$f(q) = f(a, b) = f(a, 0) = g(a)$$

If there's $g$ such that $f(x, y) = g(x)$ and $f$ is not independent of second variable,
then we follow that there exist $y_1, y_2, x \in R$ such that
$$f(x, y_1) \neq f(x, y_2)$$
and thus $g(x) \neq f(x, y_1)$ or $g(x) \neq f(x, y_2)$, which is a contradiction.

We follow that $f'(a, b) = \eangle{g'(x), 0}$.

\subsection*{2-3}

Close to the previous one.

\subsection*{2-4}

(a) We follow that if $x \in R^2$ and $x = 0$, then $h(t) = f(tx) = f(0) = 0$, thus $h$
is constant and therefore continous. If $x \neq 0$, then we follow that
$$h(t) = f(tx) = |tx| \cdot g(\frac{tx}{|tx|}) = |tx| \cdot g(\frac{tx}{|t| |x|})$$
if $t >= 0$, then we follow that
$$h(t) = |tx| \cdot g(\frac{tx}{t |x|}) = t |x| \cdot g(x/|x|)$$
and if $t < 0$, then
$$h(t) = - t |x| \cdot g(\frac{tx}{- t|x|}) = t|x| \cdot g(x/|x|)$$
since $x$ is fixed, we follow that $|x| \cdot g(x/|x|)$ is a constant, and thus
we conclude that $h$ is a linear function, which is differentiable, as desired.

(b) If $g = 0$, then we follow that $f(x) = 0$, and thus it's differentiable at every
point. If $g \neq 0$, then we follow that there



\end{document}