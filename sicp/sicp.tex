\documentclass[11pt,oneside,titlepage]{book}
\title{My SICP exercises}
\usepackage{amsmath, amssymb}
\usepackage{geometry}
\usepackage{hyperref}
\author{Evgeny Markin}
\date{2023}

\begin{document}
\maketitle
\tableofcontents

\chapter*{Preface}

Exercises are from "Structure and Interpretation of Computer Programs"
by Abelson and Sussmans. Most of those exercises are just the programs
that need to be written, but some require you to write something down
as a text.


\chapter{Building Abstractions with Procedures}

\section*{1.1}

\textit{Below is a sequence of expressions. What is
the result printed by the interpreter in response to each ex-
pression? Assume that the sequence is to be evaluated in
the order in which it is presented.}

\begin{verbatim}
10 -> 10
(+ 5 3 4) -> 12
(- 9 1) -> 8
(/ 6 2) -> 3
(+ (* 2 4) (- 4 6)) -> 6
(define a 3) -> 3
(define b (+ a 1)) -> 4
(+ a b (* a b)) -> 19
(= a b) -> #f
(if (and (> b a) (< b (* a b)))
b
a) -> 4
(cond ((= a 4) 6)
      ((= b 4) (+ 6 7 a))
      (else 25)
) -> 16
(+ 2 (if (> b a) b a)) -> 6
(* (cond ((> a b) a)
         ((< a b) b)
         (else -1))
(+ a 1)) -> 16
\end{verbatim}

Verified most of them in guile.

\section*{1.2}

\textit{Translate the following expression into prefix form}

$$\frac{5 + 4 + (2 - (3 - (6 + \frac{4}{5})))}{3(6 - 2)(2 - 7)}$$

\begin{verbatim}
  (/ (+ 5 4 (- 2 (- 3 (+ 6 (/ 4 5))))) (* 3 (- 6 2) (- 2 7))
\end{verbatim}

\section*{1.4}

\textit{Observe that our model for combinations whose operators are compound expres-
sions. Use this observation to describe the behavior of the
following procedure:}

\begin{verbatim}
(define (a-plus-abs-b a b)
  ((if (> b 0) + -) a b)
)
\end{verbatim}

Inside if returns plus or minus depending on the value of $b$, thus this funtion returns
$$a + b$$
in case if $b > 0$
and
$$a - b$$
otherwise, making it effectively equivalent to
$$a + |b|$$

\section*{1.5}
\textit{ Ben Bitdiddle has invented a test to determine
whether the interpreter he is faced with is using applicative-
order evaluation or normal-order evaluation. He defines the
following two procedures:}
\begin{verbatim}
(define (p) (p))
(define (test x y)
(if (= x 0) 0 y))
\end{verbatim}
\textit{Then he evaluates the expression}
\begin{verbatim}
(test 0 (p))
\end{verbatim}
\textit{What behavior will Ben observe with an interpreter that
uses applicative-order evaluation? What behavior will he
observe with an interpreter that uses normal-order evaluation?
Explain your answer. (Assume that the evaluation
rule for the special form $if$ is the same whether the interpreter
is using normal or applicative order: Thee predicate
expression is evaluated first, and the result determines
whether to evaluate the consequent or the alternative expression.)
}

\end{document}


%%% Local Variables:
%%% mode: latex
%%% TeX-master: t
%%% End:
