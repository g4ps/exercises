\documentclass[11pt,oneside,titlepage]{article}
\title{My real analysis exercises}
\usepackage{amssymb}
\usepackage{geometry}
\author{Evgeny Markin}
\date{2022}

\begin{document}
\maketitle

\section*{4.4.1}
Show that $f(x) = x^{3}$ is continuous on all of \textbf{R}.

\subsection*{{a}}
In order to show, that $f$ is continous we need to show, that $\forall \epsilon \in \textbf{R}$ $\exists \delta$ s.t.
$$|x - c| < \delta \to |f(x) - f(c)| < \epsilon$$

Let's rewrite the first formula

$$ |f(x) - f(c)| = |x^{3} - c^{3}| = |(x - c)(x^{2} + cx + c^{2})| = |x - c||x^{2} + cx + c^{2}|$$

We can put $|x - c|$ can be as small as we want it to be. Therefore we need an upper bound for $|x^{2} + cx + c^{2}|$.

$$|x^{2} + cx + c^{2}| \leqslant |x^{2}| + |cx| + |c^{2}| \leqslant (|c| + 1)^{2} + |c|(|c| + 1) + |c|^{2}$$



Therefore if we take
$\delta = min\{1, \epsilon/((|c| + 1)^{2} + |c|(|c| + 1) + |c|^{2})\}$
then
$$|x^3 - c^3| = |x-c||x^2 + cx + c^2| \leqslant \epsilon \frac{((|c| + 1)^{2} + |c|(|c| + 1) + |c|^{2}) }{ ((|c| + 1)^{2} + |c|(|c| + 1) + |c|^{2}}) = \epsilon$$

Therefore $f(x) = x ^3$ is continous on \textbf{R}.

\subsection*{(b)}
Argue, using Theorem 4.4.6, that f is not uniformly continuous on R


\textbf{Theorem 4.4.6 (Sequential Criterion for Nonuniform Continuity).} A function $f:A \to \textbf{R}$ fails to be uniformly continuous on $A$ if $\exists \epsilon > 0 $ and  two sequences $(x_n)$ and $(y_n)$ in $A$ satisfying

$|x_n - y_n| \to 0$ $but$ $|f(x_n) - f(y_n) \leqslant \epsilon_0$

In order to show that $f(x) = x^3$ is not uniformly continous on \textbf{R} let us use sequences

$$x_n = n$$
$$y_n = (n + 1/n)$$

Firstly
$$ |x_n - y_n| = |n - n - 1/n| = |-1/n| = 1/n \to 0$$

on the other hand

$$|f(x_n) - f(y_n)| = |n ^ 3 - (n + 1/n) ^ 3| = |n^3 - (n^3 + 3 \frac{n^2}{n} + 3 \frac{n}{n^2} + \frac{1}{n^3})| = $$
$$ = |-3n - \frac{3}{n} - \frac{1}{n^3} | \leqslant |3n| \to \infty$$

rmaxima seems to eraborate this statement, therefore  $|x_n - y_n| \to 0$ but $|f(x_n) - f(y_n) \to \infty$

Therefore $f(x) = x^3$ is not uniformly continous on \textbf{R}.

\subsection*{(c)}
Show that $f$ is uniformly continuous on any bounded subset of \textbf{R}.

Suppose that $A \subset \textbf{R}$ and $\exists M \in \textbf{R}$ s.t. $\forall x \in A$ $x \leqslant M$
(i.e. $A$ is bounded $M$)

Then, $\forall c \in A$ and $\forall \epsilon \in \textbf{R}$
$$\frac{\epsilon}{((|M| + 1)^{2} + |M|(|M| + 1) + |M|^2}  \leqslant \frac{\epsilon}{((|c| + 1)^{2} + |c|(|c| + 1) + |c|^2} $$

Therefore if we take
$$\delta = min\{1, \frac{\epsilon}{((|M| + 1)^{2} + |M|(|M| + 1) + |M|^2}\} $$
then $|x - c| < \delta$ implies, that $ |f(x) - f(c)| < \epsilon$, therefore making $f(x)$ uniformly
continous by definition


\section*{4.4.2}
Show that $f(x) = 1/x^3$  is uniformly continous on the set $[1, \infty)$, but is not on the set $(0, 1]$

\end{document}  
