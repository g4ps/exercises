\documentclass[11pt,oneside,titlepage]{book}
\title{My topology exercises}
\usepackage{amsmath, amssymb}
\usepackage{geometry}
\usepackage{hyperref}
\author{Evgeny Markin}
\date{2023}

\DeclareMathOperator \map {\mathcal {L}}
\DeclareMathOperator \pow {\mathcal {P}}
\DeclareMathOperator \topol {\mathcal {T}}
\DeclareMathOperator \basis {\mathcal {B}}
\DeclareMathOperator \ns {null}
\DeclareMathOperator \range {range}
\DeclareMathOperator \fld {fld}
\DeclareMathOperator \inv {^{-1}}
\DeclareMathOperator \Span {span}
\DeclareMathOperator \lra {\Leftrightarrow}
\DeclareMathOperator \eqv {\Leftrightarrow}
\DeclareMathOperator \la {\Leftarrow}
\DeclareMathOperator \ra {\Rightarrow}
\DeclareMathOperator \imp {\Rightarrow}
\DeclareMathOperator \true {true}
\DeclareMathOperator \false {false}
\DeclareMathOperator \dom {dom}
\DeclareMathOperator \ran {ran}
\newcommand{\eangle}[1]{\langle #1 \rangle}
\newcommand{\set}[1]{\{ #1 \}}
\newcommand{\qed}{\hfill $\blacksquare$}



\begin{document}
\maketitle
\tableofcontents

\chapter*{Preface}

Those are my solutions for the James Munkres' "Topology", 2nd edition.

Majority of the notation that is used here migrated from my course on the set theory. In
my very personal opinion, notation that is used there is far superior that whatever is
happening in Munkres' book. Sometimes I use some abusive notation when it is painfully
clear what's going on.

If you decide to persue the study of topology yourself, then I highly recommend firstly
to go through a course on axiomatic set theory and logic, because first chapter of this
book is highly insufficient in this regard. My personal recommendations are the
combo by Cunningham, which includes "Set theory: A first course" and
"A Logical Introduction to Proof", or 
"A first course in Mathematical Logic and Set Theory" by Michael L. O’Leary for both subjects.

\chapter*{Notation}

Sometimes I use specific notation, that migrated from my previous endeavours in pure maths.
This notation includes:

$$V_\epsilon(x) = (x - \epsilon, x + \epsilon)$$

Set of natural numbers is defined with the 0. It's denoted by either $N$, or most oftenly,
$\omega$.

Countable means that there's an injection into $\omega$ (i.e. both finite and infinitely countable
are presumed to be countable). Countably infinite means that there's a bijection with $\omega$.

\part{General Topology}

\chapter{Set Theory and Logic}


\section{Fundamental Concepts}

\subsection{}

\textit{Check distributive and DML laws}

\textit{GOTO set theory book}

\subsection{}

\textit{Determine which of the following are true.}

(a) - impl

(b) - impl

(c) - true

(d) - rimpl

(e) - $\subseteq$, true if $B \subseteq A$.

(f) - $\supseteq$;   $A - (B - A) = A$.

(g) - true

(h) - $\supseteq$

(i) - true

(j) - true

(k) - false

(l) - true

(m) - $\subseteq$

(n) - true

(o) - true

(p) - true

(q) - $\supseteq$

\subsection{}

\textit{(a) Write a contrapositive and converse of the following statement:
  "If $x < 0$, then $x^2 - x > 0$" and determine which ones are true}

Contrapositive:
$$x^2 - x \leq 0 \ra x \geq 0$$
Converse
$$x^2 - x > 0 \ra x < 0$$

Contrapositive is correct, converse is incorrect ($2^2 - 2 > 0$)

\textit{(b) Do the same for the statement $x > 0 \ra x^2 - x > 0$}

Contrapositive:
$$x^2 - x \leq 0 \ra x \leq 0$$
Converse
$$ x^2 - x > 0 \ra x > 0 $$

Contrapositive is false ($1^2 - 1 = 0$); Converse is also false ($(-2)^2 - (-2) = 6$).

\subsection{}

\textit{Let $A$ and $B$ be the sets of real numbers. Write the negation of each of the
  following statements: }

\textit{(a)}
$$ (\exists a \in A)(a^2 \notin B)$$
\textit{(b)}
$$ (\forall a \in A)(a^2 \notin B)$$
\textit{(c)}
$$ (\exists a \in A)(a^2 \in B)$$
\textit{(d) }
$$ (\forall a)(a \notin A \ra a^2 \notin B)$$

\subsection{}

\textit{Let $A$ be a nonempty collection of sets. Determine the truths of each of the
  following and their converses}

\textit{(a)
$$x \in \bigcup{A} \lra (\exists B \in A)(x \in B)$$}
\textit{(b)
$$x \in \bigcup{A} \la (\forall B \in A)(x \in B)$$}
\textit{(c)
$$x \in \bigcap{A} \ra (\exists B \in A)(x \in B)$$}
\textit{(d)
$$x \in \bigcap{A} \lra (\forall B \in A)(x \in B)$$}

\subsection{}

Skip

\subsection{}

skip

\subsection{}

GOTO set theory book

\subsection{}

\textit{Formulate DML for arbitrary unions and intersections}

$$A \setminus \bigcap{(B)} = \bigcup{(A \setminus B)} $$
$$A \setminus \bigcup{(B)} = \bigcap{(A \setminus B)} $$

For the proof goto set theory or real analisys book

\subsection{}

(a, b, d) are true

\section{Functions}

\subsection{}

\textit{Let $f: A \to B$. Let $A_0 \subseteq A$ and $B_0 \subseteq B$.}

\textit{(a) Show that $A_0 \subseteq f\inv[f[A_0]]$ and that equality holds
  if $f$ is injective.}

Suppose that $x \in A_0$. We follow that there exists $\eangle{x, y} \in f$ for some
$y \in f[A_0]$. Therefore there exists $\eangle{y, x} \in f\inv$. Because $y \in f[A_0]$,
we follow that $x \in f\inv[f[A_0]]$. Therefore $A_0 \subseteq f\inv[f[A_0]]$.

Suppose that $f$ is injective. Suppose that there exists $x_0 \in f\inv[f[A_0]]$ such that
$x_0 \notin A_0$. We follow that $\eangle{y, x_0}, \eangle{y, x}, \in f\inv$,
therefore $\eangle{x_0, y}, \eangle{x, y} \in f$, and because $x_0 \neq x$ we follow
that we've got a contradiction.

\textit{((b) }

pretty simular to $(a)$

\textit{This chapter practicly mirrors the content of my set theory course
  . Gonna skip it for now, and will come back if the need arises.}

\chapter{Topological Spaces and Continous Functions}

\section{Topological Spaces}

I want to state here that if $\topol \subseteq \pow(X)$ satisfies
properties
$$\set{X, \emptyset} \subseteq \topol$$
$$(\forall Y \in \pow(\topol))( \bigcup{U} \in \topol)$$
$$(\forall Y \in \pow(\topol))(Y \neq \emptyset \land |Y| <_c
|\omega|  \to \bigcap{U} \in \topol)$$
then $\topol$ is a topology on $X$.

\section{Basis for a Topology}

Let $Y \subseteq \pow(X)$. If
$$(\forall x \in X)(\exists y \in Y)(x \in y)$$
and
$$(\forall x \in X)(\exists y_1, y_2, y_3 \in Y)(x \in y_1 \cap y_3 \to
x \in y_3 \land y_3 \subseteq y_1 \cap y_2)$$
then $Y$ is a basis for a topology on $X$.

\subsection{}

\textit{Let $X$ be a topological space; Let $A$ be a subset of $X$. Suppose that for each
  $x \in A$ there is an open set $U$ containing $x$ such that $U \subseteq A$. Show that $A$ is
  open in $X$.}

Let $U: A \to \pow(A)$ be an indexed function such that 
$$x \in U(x) \land U(x) \subseteq A \land U(x) \in \topol(X)$$
We want to show that $A = \bigcup{\ran(U)}$. Suppose that $x \in A$. We follow that
$x \in U(x)$. Thus $x \in \bigcup{\ran(U)}$. Therefore $A \subseteq \bigcup{\ran(U)}$.

Suppose that $z \in \bigcup{\ran(U)}$. We follow that
$$(\exists Y \in \ran(U))(z \in Y) \ra
(\exists x \in A)(z \in U(x))$$
Since $(\forall x \in A)(U(x) \subseteq A)$, we follow that $z \in A$. Thus
$\bigcup{\ran(U)} = A$.

Because $(\forall x \in A)(U(x) \in \topol(X))$, we follow that
$$\ran(U) \subseteq \topol(A)$$, therefore by definition of topology we follow that
$$\bigcup{\ran(U)} \in \topol(X)$$
as desired.

\subsection{}

Too tedious, skip

\subsection{}

\textit{Show that the collection $\topol_c$ given in Example 4 of p. 12 is a topology on the
  set $X$. Is the collection
  $$\topol_\infty = \{U \in \pow(X):
  |X \setminus U| \geq_c |\omega| \lor X \setminus U = \emptyset \lor
  X \setminus U = X\}$$
  a topology on $X$?
}

We firstly state that
$$\topol_c = \{U \in \pow(X): |X \setminus U| \leq_c |\omega| \lor X \setminus U = X\}$$

We can follow that $X \setminus X = \emptyset$, which is countable, thus $X \in \topol_c$.
$X \setminus \emptyset = X$, therefore $\emptyset \in \topol_c$.

Suppose that $U' \subseteq \topol_c$. If $U' = \{\emptyset\}$, then $
X \setminus \bigcap{U'} = X$ and $X \setminus \bigcup{U'} = X$.
Thus assume that $U' \neq \{\emptyset\}$.

We follow that
$$(\forall u \in U')(|X \setminus u| \leq_c |\omega| \lor X \setminus u = X)$$
We follow that if $\emptyset \in U'$, then $\bigcup{U'} = \bigcup{(U' \setminus \{\emptyset\})}$.
Then we follow by DML that
$$X \setminus \bigcup\{U'\} = X \setminus \bigcup\{U' \setminus \{\emptyset\}\} =
\bigcap_{U' \setminus \{\emptyset\}}{X \setminus u}$$
we know that $(\forall u \in U')(|X \setminus u| \leq_c |\omega|)$. For any $u \in U'$ we
follow that
$$\bigcap_{u \in U' \setminus \{\emptyset\}}{X \setminus u} \subseteq X \setminus u'$$
and given that $X \setminus u'$ is countable, we follow that $\bigcap_{u \in U'}{X \setminus u}$
is countable as well, thus $\bigcup{U'} \in \topol_c$.

Now let $U' \subseteq \topol_c$ and $|U'| < |\omega|$ and $U' \neq \{\emptyset\}$.
We follow that if $\emptyset \in U'$, then $\bigcap{U'} = \emptyset$, and therefore
$X \setminus \bigcap{U'} = X$. Therefore assume that $\emptyset \notin U'$.

Then we can follow that
$$X \setminus \bigcap{U'} = \bigcup_{u \in U'}{X \setminus u}$$
Given that $U'$ is countable and $X \setminus u$ is countable we follow that
$\bigcup_{u \in U'}{X \setminus u}$ is countable, thus $X \setminus \bigcap{U'}$ is countable.

Therefore we conclude that $\topol_c$ is a topology on $X$.

Now let us consider $T_\infty$. We can state that $X \in T_\infty$ because
$X \setminus X = \emptyset$. Because $X \setminus \emptyset = X$, we follow that
$\emptyset \in T_\infty$.

Suppose that $X$ is not infinite and $T_\infty \neq \{\emptyset, X\}$. Then there exists
$u \in T_\infty$ such that $u \neq \emptyset$ and $u \neq X$. Therefore $X - u$ is
nonempty finite set, therefore $u \notin T_\infty$, which is a contradiction.
Therefore we conclude that if $X$ is finite, then $T_\infty$ is a trivial topology.

If $X$ is infinite, then we follow that we can have an injection $f: \omega \to X$.
Let $O$ be the set of odd naturals and $E$ be the set of evens. Then we follow that
$$|X \setminus f[O]| = |f[E]| \geq_c |\omega|$$
and
$$|X \setminus f[E]| =_c |f[O]| \geq_c |\omega|$$
which tells us that $f[O]$ and $f[E]$ are both in $X$. We can also follow that
$$|X \setminus f[O \cup \{0\}]| \geq |\omega|$$
thus $ f[O \cup \{0\}] \in \topol_\infty$. This gives us that
$$f[E] \cap f[O \cup \{0\}] = \{f(0)\} \in \topol_\infty$$
but $\{f(0)\}$ is a finite nonempty set for which none of the conditions of $\topol_\infty$
hold. Therefore we conclude that if $X$ is infinite, then $\topol_\infty$ is not a topology.

Therefore we conclude that if $X$ is a finite set, then $T_\infty$ is equal to a
trivial topology; if $X$ is infinite, then $T_\infty$ is not a topology at all, since
it is not closed under finite intersections.

\subsection{}

\textit{(a) if $\{\topol_\alpha\}$ is a family of topologies on $X$, show that
  $\bigcap{\topol_\alpha}$ is a topology on $X$. Is $\bigcup{\topol_\alpha}$ a topology on $X$?}

Since every topology on $X$ has $X$ and $\emptyset$ as elements, we follow that
$$\{X, \emptyset\} \subseteq \bigcap{\topol_\alpha}$$
If $Y \subseteq \bigcap{\topol_\alpha}$, then we follow that
$$(\forall Z \in \{\topol_\alpha\})(\bigcap{\topol_\alpha} \subseteq Z)$$
$$(\forall Z \in \{\topol_\alpha\})(Y \subseteq Z)$$
since every $Z$ is a topology, we follow that
$$(\forall Z \in \{\topol_\alpha\})(\bigcup{Y} \in Z)$$
$$\bigcup{Y} \in \bigcap{\topol_\alpha}$$
If $Y$ is finite and nonempty, we can also follow that
$$(\forall Z \in \{\topol_\alpha\})(Y \in Z) \ra
(\forall Z \in \{\topol_\alpha\})(\bigcap{Y} \in Z) \ra \bigcap{Y} \in \bigcap{\topol_\alpha}$$
thus we conclude that $\bigcap{\topol_\alpha}$ is a topology.

$\bigcup{\topol_\alpha}$ is not necessarily a topology. Although
$\set{X, \emptyset} \in \bigcup{\topol_\alpha}$, we cannot follow that the topology is
closed under unions. Case in point: Let $X = \set{a, b, c}$ and
$$\topol_1 = \set{\emptyset, X, \set{a}}, \topol_1 = \set{\emptyset, X, \set{b}}$$
then $Y = \topol_1 \cup \topol_2$ does not contain $\set{a, b}$, which would be necessary
for this case. Thus we conclude that in general we can't have implications for
$\bigcup{\topol_\alpha}$.

\textit{(b) Let $\set{\topol_\alpha}$ be a family of topologies on $X$. Show that there is a
  unique smallest topology on $X$ containing all the collections $\topol_\alpha$ and
  a unique largest topology contained in all $\topol_\alpha$.}

Let us take $\bigcup{\set{\topol_\alpha}}$. We cannot follow that presented
set is a topology on $X$, nor can we state that it is a basis of a topology. Former
is followed from the discussion in the previous section of this exercise, and the latter
cannot be followed because we don't necessarily satisfy the
second point of the definition of the basis. Namely, we don't have that
$$(\forall x \in X)(\exists y_1, y_2, y_3 \in \bigcup{\set{\topol_\alpha}})(x \in y_1 \cap y_3 \to
x \in y_3 \land y_3 \subseteq y_1 \cap y_2)$$
Let $Q$ be a set of all of the intersections of finite nonempty subsets of
$\bigcup{\set{\topol_\alpha}}$. We follow that $(\forall x \in \bigcup{\set{\topol_\alpha}})
(x = \bigcap{\{x\}})$, therefore $\bigcup{\set{\topol_\alpha}} \subseteq Q$. 
Thus we follow that $Q$ satisfies
the first requirement for the basis of $X$. Now let $x \in X$ be such that there
exist $y_1, y_2 \in Q$ such that $x \in y_1 \cap y_2$. We follow that there exist
finite subsets $Y_1, Y_2 \subseteq \bigcup{\set{\topol_\alpha}}$ such that 
$$y_1 = \bigcap{Y_1} \land y_2 = \bigcap{Y_2}$$
therefore
$$y_1 \cap y_2  = \bigcap{Y_1} \cap \bigcap{Y_2}$$
which is an intersection of a finite subset of $\bigcup{\topol_\alpha}$. Thus we follow that there
exists $y_3 \in Q$ such that $x \in y_3 \land y_3 \subseteq y_1 \cap y_2$. 
Therefore we can follow that the set $Q$ is indeed a basis for a topology on $X$.
Let us name the topology generated by this set as $\topol_q$.

Suppose that there is a topology, which contains
all of the topologies $\set{\topol_\alpha}$. Then we follow that it contains
$\bigcup{\set{\topol_\alpha}}$, therefore we follow that it contains all of the unions
of $\bigcup{\set{\topol_\alpha}}$, and finite intersections of subsets of
$\bigcup{\set{\topol_\alpha}}$, and thus it contains $\topol_q$. Therefore
we follow that $\topol_q$ is the smallest topology, which contains all
the topologies of $\set{\topol_\alpha}$.

Suppose that $\topol_p$ is a topology, which is contained in all of the $\set{\topol_\alpha}$.
Then we follow that $\topol_p \subseteq \bigcap{\topol_\alpha}$. Because $\bigcap{\topol_\alpha}$
is a topology itself, we follow that it is the largest topology, which is contained
in all of the $\set{\topol_\alpha}$.

\textit{(c) If $X = \set{a, b, c}$, let
  $$\topol_1 = \set{\emptyset, X, \set{a}, \set{a, b}}$$
  $$\topol_2 = \set{\emptyset, X, \set{a}, \set{b, c}}$$
  Find the smallest topology containing $\topol_1$ and $\topol_2$, and the largest topology
  contained in $\topol_1, \topol_2$.
}

We can follow from previous discussions that largest contained topology is
$$\set{\emptyset, X, \set{a}}$$
and the smallest containing topology is
$$\set{\emptyset, X, \set{a}, \set{b}, \set{a, b}, \set{b, c}}$$

\subsection{}

\textit{Show that if $A$ is a basis for a topology on $X$, then the topology generated by $A$
  equals the intersection of all topologies on $X$ that contains $A$. Prove the same
  if $A$ is a subbasis.}

Let $A$ be a subbasis.
Let $\set{\topol_\alpha}$ be a set of topologies, that contain $A$ and  $\topol_A$ is
a topology generated by $A$. We can follow that $\topol_A \in \set{\topol_\alpha}$,
therefore $\bigcap{\set{\topol_\alpha}} \subseteq {\topol_A}$. If $x \in \topol_A$, then we
follow that there exists a subset $B \subseteq A$ such that $x$ is equal to some
union of some finite intersections of $B$. Since
$B \subseteq A$, we follow that $(\forall y \in \topol_\alpha)(B \subseteq y)$. Therefore
all of the finite intersections of $B$ are in any topology of $\topol_\alpha$.  Therefore
all of the unions of those intersections are in any $\topol_\alpha$. Therefore
we conclude that $(\forall y \in \topol_\alpha)(x \in y)$.
and thus $x \in \bigcap{\topol_\alpha}$.
Therefore we conclude that $\topol_A \subseteq  \bigcap{\topol_\alpha}$, and by double
inclusion we get that $\topol_A =  \bigcap{\topol_\alpha}$, as desired.

Since every basis of a topology is a subbasis by first clause of the definition, we follow
that the desired result holds for bases as well.

\subsection{}

\textit{Show that the topologies of $R_l$ and $R_k$ are not comparable.}

Let $[0, 1)$ be an element of a basis of topology $R_l$. Then we follow that
there are no elements of basis of standart topology on $R$ that contains $0$ and lies inside
$[0, 1)$. We can follow this by contradiction

Suppose that $0 \in (x, y)$ and $(x, y) \subseteq [0, 1)$. Since $0 \in (x, y) $,
we follow that $x < 0$. Thus we conclude that there exists $n \in Z_+$ such that
$1/n < |x|$. Therefore $-1/n \in (x, y)$ and $-1/n \notin [0, 1)$ which gives us
that $(x, y) \not \subseteq [0, 1)$, which is a contradiction.
The same logic applies to any element of basis of $R_k$.

Now let us look at the basis element $(-1, 1) \setminus K$ and the point $0$. We can
follow that $0 \in (-1, 1) \setminus K$ and suppose that there exists basis element of
$R_l$  $[a, b)$ that has point $0$ and is contained within $(-1, 1) \setminus K$.
Since $0 \in [a, b)$, we follow that $a \leq 0 < b$. Thus we conclude that there exists
$n \in Z_+$ such that $0 < 1/n < b$. Thus we conclude that $1/n \in [a, b)$ and
$1/n \notin (-1, 1) \setminus K$, since $1/n \in K$ for all $n \in Z_+$. Thus we
conclude that $R_k$ and $R_l$ are not comparable, as desired.

\subsection{}

\textit{Consider the following topologies on $R$:
  $$\topol_1 = \text{the standart topology on $R$}$$
  $$\topol_2 = \text{the topology of $R_k$}$$
  $$\topol_3 = \text{the finite complement topology}$$
  $$\topol_4 = \text{the upper limit topology, having all sets $(a, b]$ as basis}$$
  $$\topol_5 = \text{the topology having all sets
    $(-\infty, a) = \set{x: x < a}$ as a basis}$$
    Determine, for each of these topologies, which of the others it contains
}

We can follow that $T_2$ contains $T_1$, since it's finer, as proven in the chapter. The
reverse is not true, as proven in the chapter.

We can follow that $T_3$ does not contain $T_1$, because if it is, then we follow that
$(-\infty, a] \cup [b, \infty)$ has finite number of points. The revese is true, since
we can divide each element of a finite complement into a union of
open intervals. For example, if $x \in T_3$ is such that $x = R \setminus \set{x_1, x_2, x_3}$
and $x_1 < x_2 < x_3$,
then we can state that $x = (-\infty, x_1) \cup (x_1, x_2) \cup (x_2, x_3) \cup (x_3, \infty)$.
We can follow that middle 2 intervals are in the basis of standart topology, and two infinite
intervals are unions of infinite set of intervals of basis.
Thus $\topol_1$ is strictly finer than $\topol_3$.

We can follow that the same logic, that worked with lower limit, works with upper limit as well.
thus we conclude that $T_4$ is strictly finer than $T_1$.

We can follow that for $(-\infty, a) \in T_5$ we can get a sequence $(x_n) = a - n$, then
get a set of intervals $\set{(a, a - 1), (x_{n + 1}, x_n)}$, all of which are in the basis of
standart topology, get another set$\set{V_{0.1}(x_n)}$ to path the holes in this set,
and take union of unions of both sets to get that $(-\infty, a) \in T_1$.

For $(a, b)$ - a set in the basis of standard topology we follow that every set in the
basis of $T_5$ contains $a - 1$, thus we conclude that $(a, b) \notin T_5$. Thus we
conclude that $T_1$ is strictly finer than $T_5$.

Topology $\topol_2$ is strictly finer than $\topol_1$, therefore we follow that
topologies that are finer than $\topol_1$ are a subset of $\topol_2$. This includes
$\topol_3$ and $\topol_5$. (Almost) the same reasoning that worked with $R_k$ and $R_l$
can be applied to show that $\topol_2$ is not finer than $\topol_4$. On the other hand,
suppose that $x \in X$ and $y \in \topol_2$ is such that $x \in y$. We follow that
if $y \in \topol_1$, then there exists an element of $\topol_4$ that is finer than
$y$. Thus assume taht $y \notin \topol_1$ and therefore is in the form
$y = (a, b) \setminus K$ for some $a, b \in R$. If $x \leq 0$, then we can have
set $(a, x] \subseteq y$ that will satisfy. Thus assume that $x > 0$. We follow that there
exists $n \in Z_+$ such that $1/n < x$. By well-ordering properties of $Z_+$ we
follow that there exists lowest $n \in Z_+$ such that $1/n  x$. Therefore we follow that
there are no elements $z \in K$ such that $1/n < z < x$. Since $x \in (a, b) \setminus K$,
we follow that $x \notin K$, therefore
$(\forall y \in (1/n, x])(y \in x \in (a, b) \setminus K)$. Therefore we conclude that
$\topol_4$ is strictly finer than $\topol_2$, which is neat.

$\topol_3$ is strictly coarser than $\topol_1$, $\topol_2$. Since $\topol_4$
is strictly finer than $\topol_2$, we follow that $\topol_3$ is coarser than $\topol_4$.
Suppose that $a < x < b$ and let $y = R \setminus \set{a, b}$ be an element of $\topol_3$.
Then we follow that no element of basis of $\topol_5$ has $x$ and does not have $a$.
If $(-\infty, a)$ is an element of $\topol_5$, then we follow that every element of topology
$\topol_3$ has numbers greater than $a$ in it (since there are infinitly  many of them).
Thus we conclude that no element of $\topol_3$ is a subset of $(-\infty, a)$. Thus we
conclude that $\topol_3$ and $\topol_5$ are not comparable.

And after all of the discussion, we can conclude that
$$[\topol_3 | \topol_5] \subset \topol_1 \subset \topol_2 \subset \topol_4$$
is the desired conclusion.

\subsection{}

\textit{(a) Apply Lemma 13.2 to show tha the countable collection
  $$B = \set{(a, b): a < b \land a, b \in Q}$$
  is a basis that generates the standard topology on $R$.}

Denote $\topol$ as a standard topology on $R$. Let $x \in \topol$. We follow that
there exists an interval $(a, b)$ in basis of standard topology such that
$x \in (a, b)$. We can follow that there exist $a', b' \in Q$ such that
$a < a' < x < b' < b$ (otherwise we run into some problem with density of rationals
in reals). Therefore we follow that $x \in (a', b')$. Lemma 13.2 tells us that
the presented result implies that $B$ is a basis for standard topology, as desired.

\textit{(b) Show that the collection
  $$C = \set{[a, b): a < b \land a, b \in Q}$$
  is a basis that genenrates a topology different from the lower limit topology on $R$.}

Proof that $C$ is a basis is trivial.
Let us look at $[\sqrt{2}, 2)$ - an element of $R_l$. Suppose that $c = [a, b) \in C$ is
such that $\sqrt{2} \in c$. Because $\sqrt{2} \notin Q$, we follow that
$a \neq \sqrt{2}$, therefore $a < \sqrt{2} < b$. Therefore we can conclude that $C$ is
not finer than $R_l$. Proving that $C$ is a subset of $R_l$ is trivial, thus
we conclude that $R_l$ is strictly finer than $C$, and thus $C$ generates a topology
different than $R_l$, as desired.

\section{The Order Topology}

\section{The Product Topology on $X \times Y$}

\section{The Subspace Topology}

\subsection{}

\textit{Show that if $Y$ is a subspace of $X$, and $A$ is a subspace of $Y$, then the topology
  $A$ inherits as a subspace of $Y$ is the same as the topology it
  inherits as a subspace of $X$.}



Suppose that $Q$ is an open set in $A$ with respect to topology, inherited from $X$.
We follow that there exists an open set in $X$  $Q_x \subseteq X$ such that $Q = Q_x \cap A$
by definition of a subspace topology. We follow that there exists open in $Y$ set $Q_y \subseteq Y$
such that $Q_y = Q_x \cap Y$. With respect to $Q_y$ there exists an open in $A$ set
$Q' = Q_y \cap A$. Thus
$$Q' = Q_y \cap A$$
$$Q' = Q_x \cap Y \cap A$$
Since $A \subseteq Y$, we follow that $Y \cap A = A$. THus
$$Q' = Q_x \cap (Y \cap A)$$
$$Q' = Q_x \cap A$$
$$Q' = Q$$
Therefore we conclude that if $Q$ is in topology of $A$ inherited from $X$, then $Q$ is also
in a topology of $A$ inherited from $Y$. Proof of the converse is pretty much the same
proof

Here's another, more logical and rigorous proof.
Denote topology of $A$ inherited from $Y$ by $\topol_A$ and topology of $A$ inherited from $X$
by $\topol_A'$. Also denote topology of $X$ by $\topol_X$ and topology of $Y$ inherited from $X$
by $\topol_Y$. Then we can state that 
$$Q \in \topol_A \lra (\exists Q_y \in \topol_Y)(Q = Q_y \cap A) \lra
(\exists Q_X \in \topol_X)(Q_y = Q_x \cap Y \land Q = Q_y \cap A) \lra$$
$$ \lra 
(\exists Q_X \in \topol_X)(Q = Q_x \cap Y  \cap A) \lra
(\exists Q_X \in \topol_X)(Q = Q_x \cap (Y  \cap A)) \lra
$$
$$ \lra 
(\exists Q_X \in \topol_X)(Q = Q_x \cap A) \lra
Q \in \topol_A'$$
thus $\topol_A' = \topol_A$ by extensionality axiom.

\subsection{}

\textit{if $\topol$ and $\topol'$ are topologies on $X$ and $\topol'$ is strictly finer
  than $\topol$, what can you cay about the corresponding topologies on the subset $Y$ of $X$.}

Denote corresponding topologies by $\topol_Y'$ and $\topol_Y$.
There're three plausible cases:

1 - we can't say nothing

2 - $\topol_Y' \supset \topol_Y$

3 - $\topol_Y' \supseteq \topol_Y$

I'm betting on the second case, so let us try to prove that. In order to do that, let us firstly
prove the third case, which is a "subcase" of the second.

Suppose that $Q \in \topol_Y$. We follow that there exists $Q_X \in \topol$ such that
$Q = Q_X \cap Y$. Since $Q_X \in \topol$, we follow by $\topol \subset \topol'$ that
$Q_x \in \topol'$. Thus $Q = Q_X \cap Y$ implies that $Q \in \topol_Y'$. Therefore we follow that
$\topol_Y' \supseteq \topol_Y$.

Although I'm betting on the second case, it seems that I'm not getting my money back.
We can follow that second case is not always true, if we substitute $\emptyset$ for $Y$.
Then $\topol_Y = \topol_Y' = \emptyset$. If we look into topologies of some
almost-trivial set, such as $X = \set{a, b, c}$, then I think that we can come up with a more
persuasive case as well. Therefore we conclude that
presented conditions imply that $\topol_Y \subseteq \topol_Y'$.

\subsection{}

\textit{Considet the set $Y = [1, 1]$ as a subspace of $R$. Which of the following sets are
  open in $Y$? Which are open in $R$?
  $$A = \set{x : \frac{1}{2} < |x| < 1 }$$
  $$B = \set{x : \frac{1}{2} < |x| \leq 1 }$$
  $$C = \set{x : \frac{1}{2} \leq |x| < 1 }$$
  $$D = \set{x : \frac{1}{2} \leq |x| \leq 1 }$$
  $$E = \set{x : 0 < |x| < 1 \land 1/x \notin Z_+ }$$
}

We can follow that
$A = (-1, -1/2) \cup (1/2, 1)$
is open in both $Y$ and $R$.

$B = [-1, -1/2) \cup (1/2, 1]$
is a union of two rays in $Y$, therefore we follow that it is open in $Y$. For $R$ we've got that
there is no open interval, that contains a point $1$ and does not contain anything larger
then $1$. Therefore we conclude that given set is not a union of open intervals, and therefore
it is not open in $R$.

We can follow pretty easily that $C$ and $D$ are not open in both $Y$ and $R$ since there is
no open interval/ray that contains $1/2$ and does not contain anything in the interval
$(-1/2, 1/2)$.

We can represent $E$ as
$$E = (-1, 0) \cup ((0, 1) \setminus K)$$
We follow that $(-1, 0)$ is an element of a basis of both $Y$ and $R$. Suppose
that $x \in (0, 1) \setminus K$. Then we follow that there exist lowest $n_1 \in Z_+$ such that
$1/n_1 < x < 1/(n_1 + 1)$. Therefore we can conclude that if $x \in E$, then there exist
a basis element $Q$ of both $Y$ and $R$ such that $x \in Q \subseteq Y, R$. Therefore
we follow that $E$ is an open set in both $Y$ and $R$.

\subsection{}

\textit{A map $f: X \to Y$ is said to be an open map if for every open set $U$ of $X$, the
  set $f(U)$ is open in $Y$. Show that $\pi_1: X \times Y \to X$ and
  $\pi_2 : X \times Y \to Y$ are open maps.}

Suppose that $Q \in X \times Y$ is an open set. Therefore we follow that it is a union of
some element of a basis of $X \times Y$, therefore there exist a subset $R$ of a basis of
$X \times Y$ such that $Q = \bigcap{R}$. From a set theory course we know that
$$U[\bigcup{G}] = \bigcup{\set{R[C]: C \in G}}$$
for any relation $U$. Therefore we can follow that the same result holds for functions
$\pi_1, \pi_2$. We can follow that for any $r \in R$ we've got that both $\pi_1(r)$ and
$\pi_2(r)$ are open by the definition of a basis for the product topology. Therefore we
conclude that $\pi_1[Q] = \pi_1[\bigcup{R}] = \bigcup{\set{\pi_1[\bigcup{r}]: r \in R}}$.
Therefore we conclude that $\pi_1[Q]$ is a union of open sets of $X$, therefore we conclude that
it is in topology of $X$. We can follow the simular result for $\pi_2$ using simular logic.

\subsection{}

\textit{Let $X$ and $X'$ denote a single set in the topologies $\topol$ and $\topol'$
  respectively; let $Y$ and $Y'$ denote a single set in the topologies $U$ and $U'$,
  respectively. Asuume that these sets are nonempty. }

There're a couple of ways to deconstruct the text of this exercise: 
we can assume that  $X = X'$,
$Y = Y'$, $X \in \topol$, $X' \in \topol'$, $Y \in U$ and $Y' \in U'$,
or we can assume that $X \in \topol$, $X' \in \topol'$, $Y \in U$ and $Y' \in U'$
without $X = X'$ and $Y = Y'$. The latter case will obviously present some problems
in the proofs, therefore we will assume that the author intended to use the former case.

\textit{(a) Show that if $\topol' \supseteq \topol$ and $U' \supseteq U$, then the product
  topology on $X' \times Y'$ is finer than the product topology on $X \times Y$}

Let $\basis$ denote the basis for $\topol_{X \times Y}$. Let $q \in X \times Y$.
Because $\basis$ is a basis for $\topol_{X \times Y}$ we follow that there exists
$b \in \basis$ such that $q \in b$. Since $b \in \basis$, we follow that there
exist $x \in \topol$ and $y \in U$ such thaht $b = x \times y$.
Since $\topol \subseteq \topol'$ and $U \subseteq U'$,
we follow that $x \in \topol'$ and $y \in U'$. Therefore $x \times y \in \basis'$
where $\basis'$ denotes the basis for $\topol_{X' \times Y'}$. Therefore we conclude that
for every $x \in X \times Y$ and every basis element $q \in \basis$ there exists
$q' \in \basis'$ such that $q' \subseteq q$ and $x \in q'$. Therefore we conclude that
$\topol_{X \times Y} \subseteq \topol_{X' \times Y'}$, as desired.

\textit{(b) Does the converse of (a) hold? Justify your answer.}

Let $\topol$ and $\topol'$ be defined on a set $Q = \set{a, b}$ and
$U$ and $U'$ be defined on $W = \set{c, d}$.
Let $X = X' = \set{a}$, $Y = Y' = \set{c}$, $\topol = \set{\emptyset, \set{a}, \set{b}, Q}$,
$\topol' = \set{\emptyset, \set{a}, Q}$, and $U = U'$. Then we follow that
topology defined on $X \times Y$ is finer than the topology defined $X' \times Y'$ (and vice versa),
but $\topol'$ is not finer than $\topol$.



\subsection{}

\textit{Show that the countable collection
  $$\set{(a, b) \times (c, d): a < b \land c < d \land a, b, c, d \in Q}$$
  is a basis for $R^2$.}

Let us denote this set by $L$.
Suppose that $x \in R^2$. We follow that there exist $x_1, x_2, y_1, y_2 \in Q$ such that
$x_1 < x < x_2$ and $y_1 < y < y_2$, therefore $(\exists l \in L)(x \in l)$. Thus we follow that
the first condition of a definition of a basis is sastisfied. The last condition can be satisfied
by through the argument about the density of rationals in reals.


We can follow that topology, that is presented by given basis is a subset of the standard topology
on $R^2$, and we can follow though pretty much the same argument that given topology is
finer than the standard topology. Therefore I'm pretty sure that we can state that given basis
generates the standard topology (I'll not provide any proof of that, just stating what I think).


\subsection{}

\textit{Let $X$ be an ordered set. If $Y$ is a proper subset of $X$ that is
  convex in $X$, does it follow that $Y$ is an interval or a ray in $X$?}

Don't think so. I think that the author tries to give us a hint to what's going to come afterwards
(probably something about the completeness and whatnot).

Pretty sure, that we don't need to prove that $Q$ is a totally ordered set, so we're going to
take it as a given. Let
$$M = \set{x \in Q: x^2 < 2 \land x \geq 0}$$
(I've added the latter condition in order not to be bogged down  by
several cases, depending on the sign).
Let $x < y \in M$. Suppose that $z \in Q$ is such that  $x < z < y$. Then
we follow that $z > x  \geq  0$, thus $z > 0$. Since all of the numbers are positive,
we're justified to square them and get that
$x^2 < z^2 < y^2$. Given that $y^2 < 2$, we conclude that $z^2 < 2$ as well. Therefore
we follow that $z \in M$. Thus we can follow that $z \in (x, y) \ra z \in M$. Therefore
we can state that presented set is convex.

Given that $M$ is bounded above and below, we follow that it is not a ray. Suppose that
it is an interval. Then we follow that there exists $k \in Q$ such that $M = [0, k)$. Therefore
we follow that $k$ is a least lower bound of $M$, which is not the case, as proven in numerous
real analysis books. Thus we conclude that $M$ is not an interval.

\subsection{}

\textit{If $L$ is a straight line in the plane, describe the topology $L$ inherits as
  a subspace of $R_l \times R$ and as a subspace of $R_l \times R_l$. In each case it is a
  familiar topology.}

Let $\basis$ be the basis for $R_l \times R$ and $\basis'$ be the basis for $R_l \times R$.
Let $q \in \basis$  and suppoe that $b = L \cap q \neq \emptyset$. From plotting elements
of the basis and the line itself on the graph, we can conclude that $b$ is some sort of an interval
on the plane (either closed or open), and it might as well be a ray (once again, open or closed).
In case with $R_l \times R_l$ we conclude that the topology here is once again open or closed
intervals on the plane.

\subsection{}

\textit{Show that the dictionary order topology on the set $R \times R$ is the same
  as the product topology $R_d \times R$, where $R_d$ denotes $r$ in the discrete topology.
  Compare this topology with the standard topology on $R^2$.}

Let $\eangle{x, y} \in R^2$. We follow that there exists $q$ - element of basis of $R_d \times R$
such that $\eangle{x, y} \in q$. Because $q$ is an element of a basis, we follow that
$q = w \times r$, where $w \in \topol_{R_d}$ and $r \in \topol_R$. Because $r \in \topol_R$,
we follow that there exists $(a, b)$ such that $(a, b) \subseteq r$. Therefore we follow that
element $\set{x} \times (a, b)$ is an element of a basis of dictionary order such that
$\set{x} \times (a, b) \subseteq q$. Therefore we follow that $R_d \times R$ is
coarser than dictionary order topology.

Suppose that $\eangle{x, y}$ is in $R^2$ and $q$ is in basis of dictionary topology of $R^2$
such that $\eangle{x, y} \in q$. By definition (and immediate implications of thereof )
we follow that the set $q \cap \set{x} \times R$ is nonempty. Since $\eangle{x, y}$
is in $q$, and $q$ is a basis element, we follow
that there exist $a, b \in R$ such that $\set{x} \times (a, b) \subseteq q$ (follows from
definitions and maybe some trivial manipulations of definition of dictionary order). Since
$\set{x} \times (a, b)$ is an element of a basis of $R_d \times R$, we conclude that
dictionary order topology is coarser than $R_d \times R$, and thus by double inclusion
we conclude that topology over $R_d \times R$ and dictionary order topologies
are equal, as desired.

We can follow that topology in  $R_d$ is strictly finer than standard topology of $R$
since $R_d = \pow(R)$, and thus it is the largest possible topology. Strictness follows
from the fact that $\set{0} \in \topol_{R_d}$ and $\set{0} \notin \topol_R$.

Thus we can be pretty sure that there is no element of basis of standard topology on $R \times R$
that contains $\set{0} \times R$. Since every element of basis of $R \times R$ is also contained
in $R_d \times R$, we conclude that standard topology on $R^2$ is coarser than $R_d \times R$.

\subsection{}

\textit{Let $I = [0, 1]$. Compare the product topology on $I \times I$, the dictionary order
  topology on $I \times I$, and the topology $I \times I$ inherits as a subspace of $R \times R$
  in the dictionary order topology.
}

Wanted to skip this one, since I've solved it incorrecttly the first time, but
instead of skipping I'll just present the half-assed proof here for completeness' sake.

Important thing to remember: sets such as $\set{x} \times [0, 0.1)$ are not elements of basis
of dictionary topology on $I \times I$ 

Let us look at the point $\eangle{0.5, 1}$ and basis element of standard topology
$[0, 1] \times (0.5, 1]$. We can follow that since the elements of dictionary bases
cannot just stop at the corners and must wrap around, we follow that there is no element of
basis of dictionary order topology, that is contained in presented element of basis
and contains the desired point.

We can follow also that $\set{0.5} \times (0, 1)$ cannot be presented in
standard topology as well. Thus the first two are not comparable

Suppose that $\eangle{x, y} \in I \times I$ and $q$ is the basis element with respect to
the dictionary order topology. We follow that we can take a "strand" from
dictionary (i.e. take a set $\set{x} \times R \cap q$, where $q$ is an element of the basis,
that contains point $\eangle{x, y}$) and get that dictionary order over $I \times I$ is coarser than
the topology $I \times I$ inherits as a subspace of $R \times R$ in the dictionary order topology,
since the strand is the element of the basis of the latter. We can pull the same trick
that we've used in the previous paragraph to show that the inherited topology is strictly coarser
than the dictionary topology. Using the "strand" method (i.e. taking basis elements
in form $\set{x} \times (a, b)$ or its closed analogs) we can prove that the last topology
is strictly finer than the standard topology on $I \times I$.


\section{Closed Sets and Limit Points}

\subsection{}

\textit{let $C$ be a collection of subsets of the set $X$. Suppose that $\emptyset$ and
  $X$ are in $C$, and that finite unions and arbitrary intersections of elements of $C$
  are in $C$. Show that ther collectionn
  $$T = \set{X \setminus C: C \in C}$$
  is a topology on $X$.}

We're gonna use the definition of topology on this one. We follow that
since $X$ and $\emptyset $ are in $C$ that
$$X \setminus X = \emptyset \in T$$
$$X \setminus \emptyset = X \in T$$

Assume that $J$ is an arbitrary subset of $T$. We follow that
for every $j \in J$ there exists a unique $k \in C$ such that $j = X \setminus k$. Thus we
follow that there exists $C' \subseteq C$ such that 
$$\set{j: j \in J} = \set{X \setminus k: k \in C'}$$
thus
$$\bigcup{J} = \bigcup{\set{j: j \in J}} = \bigcup{\set{X \setminus k: k \in C'}} =
X \setminus \bigcap{\set{k: k \in C'}} = X \setminus \bigcap{C'}$$
where we've used DML to justify one of the equations.
Since $C'$ is an arbitrary subset of $C$ we follow that $\bigcap{C'} \in C$. Thus we follow that
$X \setminus \bigcap C' \in T$. Thus we follow that
$J \subseteq T \ra \bigcup{J} \in T$. Therefore we've got second property of topology.

If $J$ is a finite subset of $T$, then we follow that
we can define $C'$ by the same definition and that $C'$ is finite as well. Thus 
$$\bigcap{J} = \bigcap{\set{j: j \in J}} = \bigcap{\set{X \setminus k: k \in C'}} =
X \setminus \bigcup{\set{k: k \in C'}} = X \setminus \bigcup{C'}$$
thus we follow that if $J$ is a finite subset of $T$ , then $\bigcap{J} \in T$, therefore
we've got the third and final condition of topology. Thus we follow that $T$ is a
topology,  as desired.


\subsection{}

\textit{Show that if $A$ is closed in $Y$ and $Y$ is closed in $X$, then $A$ is closed in $X$}

Since $Y$ is closed in $X$  we follow that $Y \subseteq X$. Assuming that the
topology on $Y$ is a subset topology, we follow that if $A$ is a closed set in $Y$,
then there exists $A' \subseteq X$ such that $A'$ is closed and $A = A' \cap Y$.
Since both $A'$ and $Y$ are closed in $X$ we follow that $A = A' \cap Y$ is closed in $X$
as well by definition of topology, as desired.

\subsection{}

\textit{Show that if $A$ is closed in $X$ and $B$ is closed in $Y$, then $A \times B$
  is closed in $X \times Y$.}

If $A$ is closed in $X$, then we follow that there exists $A'$ such that
$A = X \setminus A'$, where $A'$ is an open set in $X$. Same goes for $B = Y \setminus B'$.
Thus we follow that $A' \times B'$ is an open set in $X \times Y$. 
one of the exercises in chapter 1 gives us that
$$(A \times B) = (X \setminus A') \times (Y \setminus B') =
(X \times Y \setminus A' \times Y) \setminus X \times B'$$

We follow that $A' \times Y$ is an open set, then $(X \times Y \setminus A' \times Y)$
is a closed set. We follow also that $X \times B'$ is an open set,
thus $X \times Y \setminus X \times B'$ is a closed set. Thus
$$(X \times Y \setminus A' \times Y) \setminus X \times B' =
(X \times Y \setminus A' \times Y) \cap (X \times Y \setminus  X \times B')$$
is an intersection of closed sets and therefore is closed itself. Thus
we conclude that $A \times B$ is a closed set, as desired. (we can also follow the same thing
by the following exercise)

\subsection{}

\textit{Show that if $U$ is open in $X$ and $A$ is closed in $X$, then
  $U \setminus A$ is open in $X$ and $A \setminus U$ is closed in $X$.}


Firstly I want to prove that if $A, B \subseteq X$, then
$$A \setminus B = A \cap (X \setminus B)$$
We follow that by
$$x \in A \setminus B \lra x \in A \land x \notin B \lra x \in A \land x \in X \land x \notin B
\lra x \in A \land (x \in X \setminus B) \lra x \in A \cap (X \setminus B)$$

We can follow by the fact that $U, A \subseteq X$ that 
$$U \setminus A = U \cap (X \setminus A)$$
and
$$A \setminus U = A \cap (X \setminus U)$$
In the former case we've got finite intersection of two open sets,
and in the latter we've got finite intersection of two closed sets,
thus proving that $U \setminus A$ is open and $A \setminus U$ is closed, as desired.

\subsection{}

\textit{Let $X$ be an ordered set in the order topology. Show that
  $\overline{(a, b)} \subseteq [a, b]$. Under what conditions
  does equality hold?}

Let us firstly state that $a, b \in X$.
We follow that $(a, b) \subseteq [a, b]$, thus $[a, b]$ is a closed set that contains
$(a, b)$, therefore by definition of closure we follow that $\overline{(a, b)} \subseteq [a, b]$.

We follow that $\overline{(a, b)} = [a, b]$ if and only if $a, b$ are limit points of
$(a, b)$.

\subsection{}

\textit{Let $A, B$ and $A_\alpha$ denote subsets of a space $X$. Prove the following}

\textit{(a) If $A \subset B$, then $\overline{A} \subseteq \overline{B}$}

Assume that $A \subseteq B$. Let $x \in \overline{A}$. We follow that
every neighborhood of $x$ intersects $A$. Thus every heighborhood of $x$ intersects $B$
by the fact that $A \subseteq B$. 
Therefore $x \in \overline{B}$. Therefore $\overline{A} \subseteq \overline{B}$.

\textit{(b) $\overline{A \cup B} = \overline{A} \cup \overline{B}$}

Let $x \in \overline{A \cup B}$. We follow that every heighborhood of $x$
intersects $A \cup B$. Thus every neighborhood of $x$ intersects $A$ or $B$.
Assume that $x \notin \overline{A}$ and $x \notin \overline{B}$. Then we follow that
there exists a neighborhood $U$ of $x$ such that $U \cap A = \emptyset$.
There also exists neighborhood $U'$ of $x$ such that $U' \cap B$.
Thus we follow that $U \cap U'$ is a neighborhood of $x$ such that it does
not intersect $A$ nor $B$. Thus $x \notin \overline{A \cup B}$, which is a contradiction.
Thus we conclude that $x \in \overline{A \cup B} \ra x \in \overline{A} \cup \overline{B}$.

If $x \in \overline{A} \cup \overline{B}$, then $x \in \overline{A}$ or $x \in \overline{B}$.
Assume that the former is true. Then we follow that $x \in \overline{A}$.
Thus we follow that every neighborhood of $x$ intersects $A$. Since $A \subseteq A \cup B$,
we follow that every neighborhood of $x$ intersects $A \cup B$. Thus $x \in \overline{A \cup B}$,
as desired.

\textit{(c) $\overline{\bigcup A_\alpha} \supseteq \bigcup \overline{A_\alpha}$. }

I think that we need to assume here that $A_\alpha \subseteq \pow(X)$
and what we actually need to prove is that
$$\overline{\bigcup A_\alpha} \supseteq \bigcup \set{\overline{a}: a \in A_\alpha}$$
if that's the case, then we follow that if $x \in \bigcup \set{\overline{a}: a \in A_\alpha}$,
then there exists $x \in a \in A_\alpha$ such that $x \in \overline{a}$.
This means that every neighborhood of $x$ intersects $a$ at some point.
Since $a \subseteq \bigcup A_\alpha$, we follow that every neighborhood of
$x$ intersects $\bigcup A_\alpha$ at some point and thus
$x \in \overline{\bigcup A_\alpha}$.

We can follow that if $A_\alpha = \set{\set{1/n}: n \in Z_+}$ and we've got
standard topology on reals, then $0 \in \overline{\bigcup{A_\alpha}}$,
but $0 \notin \bigcup \set{\overline{a}: a \in A_\alpha}$, since there is
no $a \in A_\alpha$ such that $0 \in \overline{a}$.

\subsection{}

\textit{Critsize the following "proof" that
  $\overline{\bigcup A_\alpha} \subseteq \bigcup \overline{A_\alpha}$:
  if $\set{A_\alpha}$ is a collection of sets in $X$ and if
  $x \in \overline{\bigcup A_\alpha}$,
  then every neighborhood of $U$ intersects $\bigcup \overline{A_\alpha}$.
  Thus $U$ must intersets some $A_\alpha$, so that $x$ must belong to the
  closure of some $A_\alpha$. Therefore $x \in \bigcup{\overline{A_\alpha}}$.}

We don't have implication "Thus $U$ must intersects some $A_\alpha$, so that $x$ must belong to the
closure of some $A_\alpha$", as it was just made up. Althought every neighborhood of $x$
indeed intersects some $A_\alpha$, there's no implication that there exists $A_\alpha$
such that every neighborhood of $x$ intersects $A_\alpha$.

\subsection{}

\textit{Let $A, B$ and $A_\alpha$ denote subsets of a space $X$. Determine whether the
  following equations hold; if an equality fails, determine whether one of the
  inclusions holds.}

\textit{(a) $\overline{A \cap B} = \overline{A} \cap \overline{B}$.}

If $x \in \overline{A \cap B}$, then we follow that every neighborhood of $x$
intersects $A \cap B$ at some point. Thus every neighborhood of $x$ intersects
$A$ and $B$. Thus $x \in \overline{A}$ and $x \in \overline{B}$. Thus we've got
forward inclusion.

If $x \in \overline{A} \cap \overline{B}$, then we follow that every neighborhood of $x$
intersects $A$ and every neighborhood of $x$ intersects $B$. Thus every
neighborhood of $x$ intersects both $A$ and $B$. This does not mean that
every neighborhood of $x$ intersects $A \cup B$ since points of intersection can be different.
We can come up with some counterexample for this claim:
for example we can set $A = \set{1/2n: n \in Z_+}$ and
$B = \set{1/(2n + 1): n \in Z_+}$. We follow that $A \cap B = \emptyset$ and thus
$\overline{A \cap B} = \emptyset$, but $\overline{A} \cap \overline{B} = \set{0}$.

Therefore we follow only the forward inclusion.

\textit{(b) $\overline{\bigcap{A_\alpha}} = \bigcap{\overline{A_\alpha}}$}

We're once again struck with this awful notation, so let's change that
$$\overline{\bigcap{A_\alpha}} = \bigcap_{a \in A_\alpha}{\overline{a}}$$
we follow that reverse inclusion is not true, since that would imply
the correctness of counterexample in previous point.

We follow the forward inclusion by pretty much the same logic as in previous one

If $x \in \overline{\bigcap{A_\alpha}}$, then we follow that every neighborhood of $x$
intersects $\bigcap{A_\alpha}$ at some point. Thus every neighborhood of $x$ intersects
every $a \in A_\alpha$. Thus $(\forall a \in A_\alpha)(x \in \overline{a})$.
Therefore we follow that $a \in \bigcap_{a \in A_\alpha}{\overline{a}}$.
Thus we've got forward inclusion.

\textit{(c) $\overline{A \setminus B} = \overline{A} \setminus \overline{B}$}

We've got a case against
$\overline{A \setminus B} \subseteq  \overline{A} \setminus \overline{B}$
by setting once again $A = \set{1/2n: n \in Z_+}$, $B = \set{1/(2n + 1): n \in Z_+}$.
We follow that $A \setminus B = A$ since they are disjoint. Therefore we follow that
$0 \in \overline{A \setminus B}$ but $0 \notin \overline{A} \setminus \overline{B}$,
since $0 \in \overline{B}$.

If $x \in \overline{A} \setminus \overline{B}$, then we follow that
every neighborhood of $x$ intersects $A$, but $x$ is not in $\overline{B}$.
Assume that some neighborhood of $x$ intersects $A$ only at a points $A \cap B$.
Then we follow that $x \in \overline{A \cap B}$ and thus $x \in \overline{A} \cap
\overline{B}$. Therefore $x \in \overline{B}$, which is a contradiction.
Therefore we follow that if $x \in \overline{A} \setminus \overline{B}$
then every neighborhood of $x$ intersects $A \setminus B$. Therefore
$x \in \overline{A \setminus B}$, which gives us reverse inclusion.

\subsection{}

\textit{Let $A \subseteq X$ and $B \subseteq Y$. Show that in the space $X \times Y$
  $$\overline{A \times B} = \overline{A} \times \overline{B}$$.}

Let $x \in X$ and $y \in Y$ be such that  $\eangle{x, y} \in \overline{A \times B}$.
Let $U$ be an arbitrary neighborhood of $X$ and $V$ be an arbitrary neighborhood for $y$.
We follow that $U \times V$ is an evement of basis of $X \times Y$ that contains
$\eangle{x, y}$, and thus it intersects $A \times B$.
Therefore there exists $\eangle{q, w} \in A \times B \cap U \times V$.
Thus we follow that $\eangle{q, w} \in (A \cap U) \times (B \cap V)$.
Therefore we follow that $U$ intersects $A$ and $V$ intersects $B$.
Since $U$ and $V$ are arbitrary, we follow that every neighborhood of $x$ intersects
$A$ and every neighborhood of $y$ intersects $B$. Thus $x \in \overline{A}$ and
$y \in \overline{B}$ and thus $\eangle{x, y} \in \overline{A} \times \overline{B}$.

Suppose that $x \in X, y \in Y$ are such that $\eangle{x, y} \in \overline{A} \times \overline{B}$.
We follow that $x \in \overline{A}$ and $y \in \overline{B}$. Assume that $U \times V$ is a
basis element of $X \times Y$ that contains $\eangle{x, y}$.
We follow that $U$ intersects $A$ and $V$ intersects $B$. Thus there exist
$u \in U \cap A$ and $v \in V \cap B$. Therefore $\eangle{u, v} \in A \times B \cap U \times V$.
Therefore $U \times V$ intersects $A \times B$. Thus we follow that every
basis element that contains $\eangle{x, y}$ intersects $A \times B$, and thus
$\eangle{x, y} \in \overline{A \times B}$.

Double inclusion produces the desired equality, as desired.

\subsection{}

\textit{Show that every order topology is Hausdorff.}

Let $X$ be a toset, and let $\topol$ be respective order topology.
I think that definition, that is presented in the book implies that if there is only one element
in $X$, then the topology is vacuously Hausdorff (same goes for empty set).
Thus assume that $X$ contains at least two elements, and let $x_1, x_2 \in X$ be such
that $x_1 \neq x_2$. Since $x_1, x_2 \in X$ and $x_1 \neq x_2$, we follow
that $x_1 \prec x_2$ or $x_2 \prec x_1$. Assume the former.

Essentially we want to
produce two open sets, that will prove that the space is Hausdorff, and
those two sets will be just plain old intervals. But we've got two cases:
there could exist $x_3$ such that $x_1 \prec x_3 \prec x_2$, or there might not.
If such an element exists, then set $b_1 = a_2 = x_3$. If there is no such element, then
set $b_1 = x_2$ and $a_2 = x_1$. 
Let $a_1$ be either the lowest element of $X$ if such exists, or some element that
is less than $x_1$ in case that it does not exist.
Let $b_2$ be either the largest element of $X$ if such exists, or some element that
is larger than $x_2$ in case that it does not exist.

Then we can follow that $x_1 \in (a_1, b_1)$ and $x_2 \in (a_2, b_2)$ and
$(a_1, b_1) \cap (a_2, b_2) = \emptyset$ by their respective definitions.
Thus we follow that the space is Hausdorff, as desired.

\subsection{}

\textit{Show that the product of two Hausdorff spaces is Hausdorff.}

Let $X$ be the first spacce and let $Y$ be the second.
Let $\eangle{x_1, y_1}, \eangle{x_2, y_2} \in X \times Y$ be distinct.
We follow that $x_1 \neq x_2$ or $y_2 \neq y_2$. 
If we've got $A_1, A_2 \subseteq X$ and $B_1, B_2 \subseteq Y$ such that
$A_1 \cap A_2 = \emptyset$ or $B_1 \cap B_2 =  \emptyset$, then we follow that
$$A_1 \times B_1 \cap A_2 \times B_2 = (A_1 \cap A_2) \times (B_1 \cap B_2) = \emptyset$$
the desired result follows easily from that.

\subsection{}

\textit{Show that a subspace of a Hausdorff space is Hausdorff.}

Assume that $X$ is a Hausdorff space and let $Y \subseteq X$. Let $y_1, y_2 \in Y$.
We follow that there exist open sets $U, V \subseteq X$ such that
$y_1 \in U, y_2 \in V, U \cap V = \emptyset$ by the fact that $X$ is Hausdorff.
We follow that $U \cap Y$ and $V \cap Y$ are open sets in $Y$, and since
$U \cap V = \emptyset$, we follow that $(U \cap Y) \cap (V \cap Y) = \emptyset$
by commutatitvity and assocoativity of $\cap$. Therefore we conclude that
supspace generated by $Y$ is a Hausdorff. Since $Y$ is arbitrary, we follow that
any subspace of $X$ is Hausdorff, as desired.

\subsection{}

\textit{Show that $X$ is Hausdorff iff diagonal $\Delta = \set{x \times x: x \in X}$
  is closed in $X \times X$.}

Since $X$ is Hausdorff, we follow that for all $x \in X$ we've got that $\set{\eangle{x, x}}$
is closed.

Assume that $\Delta \neq \overline{\Delta}$ and thus there exists
$\eangle{y_1, y_2} \in \overline{\Delta} \setminus \Delta$
We follow that since $\eangle{y_1, y_2} \notin \Delta$ that $y_1 \neq y_2$. We also
follow that $y_1, y_2 \in X$. Since $X$ is Hausdorff, we follow that
$y_1, y_2 \in X$ implies that there exist neighborhoods $U, V \subseteq X$ of $y_1$
and $y_2$ respestively, such taht $U \cap V = \emptyset$. Thus
we follow that $U \times V$ is an open set in $X \times X$.
Then we follow that since $\eangle{y_1, y_2} \in \overline{\Delta}$ that
there exists $d \in \Delta$ such that $d \in U \cap V$. Thus we follow that
there exists $d_1 \in X$ such that $\eangle{d_1, d_1} = d$. Therefore
$\eangle{d_1, d_1} \in U \times V$ and therefore $d_1 \in U \land d_1 \in V$,
which implies that $d_1 \in U \cap V$, which is a contradiction, since $U \cap V = \emptyset$
by the fact that $X$ is Hausdorff, as desired.

Therefore we conclude that $\overline \Delta \setminus \Delta = \emptyset$, therefore
$\Delta = \overline{\Delta}$ and $\Delta$ is a closed set, thus giving us forward
implication.

Now assume that $\Delta$ is a closed set and $X$ is not Hausdorff. Since $X$
is not Hausdorff, we follow taht there exists $x_1, x_2 \in X$ such that
$x_1 \neq x_2$ and
for all $U \in \topol$ we've got that $x_1 \in U \ra x_2 \in U$.

Let $B$ be an arbitrary basis element that contains $\eangle{x_1, x_2}$. We follow that
since it is a basis element, that $B = U \times V$, where $U, V$ are open subsets
of $X$. We follow that $x_1 \in U$ and $x_2 \in V$, that implies that
$x_1, x_2 \in U$ and $x_1, x_2 \in V$. Thus we follow that $\eangle{x_1, x_2} \in B$.
Therefore we follow that if $B$ is a basis element, that contains $\eangle{x_1, x_2}$,
then $B$ intersects $\Delta$ by the fact that $\eangle{x_1, x_1}$ and $\eangle{x_2, x_2}$
are both in $B$. Thus we follow that $\eangle{x_1, x_2} \in \overline{\Delta}$.
Since $x_1 \neq x_2$ we follwo that $\eangle{x_1, x_2} \notin \Delta$, which implies
that $\Delta \neq \overline{\Delta}$, which implies that $\Delta$ is not closed,
which is a contradiction. Thus we follow that if $\Delta$ is a closed set,
then $X$ is Hausdorff, which gives us reverse implication, as desired.

\subsection{}

\textit{In the finite complement topology on $R$, to what point or points does
  the sequence $x_n = 1/n$ converge?}

We can follow pretty easily that finite topology  on $R$ is not Hausdorff
by the fact that if $U, V$ are nonempty open sets in finite complement topology,
then $U = R \setminus S_1$, $V = R \setminus S_1$ for some finite subsets $S_1, S_2$ of $R$.
Thus we follow that $U \cap V = R \setminus S_1 \cap R \setminus S_2 =
R \setminus (S_1 \cup S_2)$, where we follow that $S_1 \cup S_2$ is finite and therefore
$R \setminus (S_1 \cup S_2)$ is infinite and therefore nonempty.

Assume that $U$ is a neighborhood of $0$. We follow that
$U = R \setminus S_1$ such that $0 \notin S_1$. and $S_1$ is finite.
Since  $x_n$ is an infinite sequence, we follow that there must exist
$n$ such that $n_0 > n \ra x_n \notin S_1 \ra x_n \in U$. Thus we conclude that $x_n$
converges to $0$.

By the same logic we follow that if $U$ is a neighborhood of any other number, then
the same logic applies. Therefore we follow that $x_n$ converges to any number in finite
complement topology.

\subsection{}

\textit{Show that $T_1$ axiom is equivalent to the condition that for each pair
  of points of $X$, each has a neighborhood not containing the other.}

Let $x_1, x_2 \in X$ are such that $x_1 \neq x_2$. $T_1$ axiom implies that
$\set{x_1}$ and $\set{x_2}$ are closed and thus $X \setminus \set{x_1}$ and
$X \setminus {x_2}$ are open. Since $x_1 \neq x_2$, we follow that
$x_2 \in X \setminus \set{x_1}$ and $x_1 \in X \setminus \set{x_2}$,
thus implying that each point has a neighborhood, that does not contain the other point.

\subsection{}

\textit{Consider the five topologies on $R$ given in Exercise 7 of paragraph 13.}

\textit{(a) Determine the closure of the set $K = \set{1/n: n \in Z_+}$ under each of
  these topologies.}

Under standard topology we follow that $\overline{K} = K \cap \set{0}$ from real
analysis course.

Under $K$ topology we follow that $R$ is open by default, and thus $R \setminus K$ is
a basis element, therefore it is open and thus $K = R \setminus (R \setminus K)$ is closed.

Under finite complement topology we follow that all closed sets that are not $R$ are
finite, and thus only $R$ contains $K$. Therefore $\overline{K} = R$. 

Since the upper limit topology is finer than the $K$-topology, we follow that
$R \setminus K$ is an open set in upper limit topology, and thus $\overline{K} = K$.

For the topology that has sets $(-\infty, a)$ as a basis, we follow that
closed sets there are either $R$ or $[a, \infty)$ for some $a \in R$, and thus
$\overline{K} = [0, \infty)$.

\textit{(b) Which of these topologies satisfy the Hausdorff axiom? The $T_1$ axiom?}

We follow that standard topology satisfies the Hausdorff axiom, as proven in the
chapter or in the real analysis course. Thus we follow that since standard topology
is coarser than $K$-topology and upper limit topology, that both of them
satisfy the axiom as well. In exercise 14 of this section we've shown that
finite complement topology is definetly not Hausdorff, and for the last topology
we follow that if $0 \in U$, then $-1 \in U$, therefore it is not Hausdorff as well.

In the chapter we've discussed that $T_1$ axiom is weaker than Hausdorff, thus we follow that
standard topology, $K$-topology and upper limit topology all satisfy the $T_1$ axiom.
We can follow that finite complement topology satisfies $T_1$ by some trivial
implications and the last topology is definetly does not satisfy $T_1$.

\subsection{}

\textit{Consider teh lower limit topology on $R$ and the topology given by the
  basis $C$ of exercise 8 of paragraph 13. Determine the closures of the intervals
  $A = (0, \sqrt{2})$ and $B = (\sqrt{2}, 3)$ in these two topologies.
}

Let's talk about lower limit topology first.

Suppose that $0 \in [a, b)$. We follow that $a \leq 0 < b$, and by density of reals we follow that
there exists $d \in R$ such that $0 < d < b$ and thus $d \in [a, b)$.
Thus we follow that every basis element that contains $0$ also intersects $A$, and thus
$0 \in \overline{A}$. We follow that $[\sqrt{2}, 2)$ is a neighborhood of $\sqrt{2}$ such that
it does not intersect $A$ at any point. Thus we follow that $\sqrt{2} \notin \overline{A}$.
There're plenty of methods to state that if $x < 0$ or $x > \sqrt{2}$, then it is not
in closure of $A$, therefore we conclude that $[0, \sqrt{2})$ is clousre of $A$ under
lower limit topology.

For $B$ we follow that pretty much the same logic holds with respect to different numbers,
and thus $\overline{B} = [\sqrt{2}, 3)$.

Now let's talk about the weird topology.

We can follow that pretty much the same argument holds for $0 \in \overline{A}$.
Now assume that $a \leq \sqrt{2} < b$. We follow that since $a \notin I$, that
$a \neq \sqrt{2}$, and thus we follow that $a < \sqrt{2} < b$. Thus 
we follow that $\sqrt{2} \in \overline{A}$. If $\sqrt{2} < c$,
then we follow that there exist rational $a, b$ such that $\sqrt{2} < a < c < b$,
and thus $c \notin \overline{A}$. If $c < 0$, then we have pretty much the same result.
Therefore we follow that $\overline{A} = [0, \sqrt{2}]$.

For the case of $B = (\sqrt{2}, 3)$, we follow that $[3, 5) \cap B = 0$, thus we follow that
$3 \notin \overline{B}$. We can easily follow that $\sqrt{2} \in \overline{B}$,
and the rest is followed by pretty much the same logic as in the previous paragraph.
Therefore we conclude that $\overline{B} = [\sqrt{2}, 3)$.

\subsection{}

\textit{Determine the closures of the following subsets of the ordered square:}

We're talking about the lexicographical order on the set $I \times I$ for $I = [0, 1]$.
We're also not gonna use the dumb notation.

\textit{$A = \set{\eangle{1/n, 0}: n \in Z_+}$}

We follow that $A \subseteq \overline{A}$. Let $\eangle{x_1, x_2} \in I \times I$.

Assume that $x_1 \neq 0$. If $x_1 = 1/n$ for some $n \in Z_+$ and $x_2 = 0$, then we follow that
$\eangle{x_1, x_2} \in A$, thus assume that $x_2 \neq 0$. We follow that
if $x_2 > 0$ and $x_2 < 1$, then there exist $a, b \in R$ such that
$0 < a < x_2 < b < 1$ and thus $\eangle{x_1, x_2} \in (\eangle{x_1, a}, \eangle{x_1, b})$,
where we follow that $(\eangle{x_1, a}, \eangle{x_1, b})$ does not intersect $A$
at any point.

If $x_1 = 1/n$ and $x_2 = 1$  then we follow that there exists some space between
$x_1$ and the previous point in the sequence, therefore we can have interval
$[x_1, y)$ such that $A \cap [x_1, y) = \set{x_1}$. 
Thus if we 
take open inerval $(\eangle{x_1, 0.5}, \eangle{y_1, 0.5})$, then
$(\eangle{x_1, 0.5}, \eangle{y_1, 0.5}) \cap A = \emptyset$. Therefore we follow that
$\set{1/n} \times I \cap A \subseteq A$. We can follow pretty much the
same thing for $x_1 \neq 0$ in a more general case.

Now assume that $x_1 = 0$. For the point $\eangle{0, 0}$ we've got that
$(\eangle{0, 0}, \eangle{0, 0.5}) \cap A = 0$, therefore we conclude
that $\eangle{0, 0} \notin \overline{A}$. Simple logic implies that for $x_2 \neq 1$ we've got
pretty much the same result.

The case with $\eangle{0, 1}$ is an interesting one. Suppose that
we've got an element of the basis $(\eangle{j_1, j_2}, \eangle{k_1, k_2})$
such that 
$$\eangle{j_1, j_2} < \eangle{0, 1} < \eangle{k_1, k_2}$$
We can follow that $j_1 = 0$ by the definition of order. We also follow that
$0 < k_1$. Thus
$$\eangle{0, j_2} < \eangle{0, 1} < \eangle{k_1, k_2}$$
we can follow that for any $k_1 > 0$ there exists $n \in Z_+$ such that $0 < 1/n < k_1$,
and therefore
$$\eangle{0, j_2} < \eangle{1/n, 0} < \eangle{k_1, k_2}$$
Therefore we follow that if $B$ is an element of the basis such that $\eangle{0, 1} \in B$,
then it intersects $A$, and thus $\eangle{0, 1}$ is the only point outside of $A$ that
is in the closure of $A$. Thus we follow that
$$\overline{A} = A \cap \set{\eangle{0, 1}}$$

\textit{$B = \set{1 - 1/n) \times 1/2: n \in Z_+}$}

$$\overline{B} = B \cup \set{\eangle{1, 1}}$$

\textit{$C = \set{x \times 0: 0 < x < 1}$}

$$\overline{C} = C \cup \set{\eangle{x, 1}: 0 \leq x \leq < 1}$$

\textit{$D = \set{x \times 1/2: 0 < x < 1}$}

$$\overline{D} = D \cup \set{\eangle{x, 1}: 0 \leq x \leq 1} \cup
\set{\eangle{x, 0}: 0 < x < 1} $$

\textit{$E = \set{1/2 \times x: 0 < x < 1}$}

$$\overline{E} = E \cup \set{\eangle{1/2, 0}, \eangle{1/2, 1}}$$

last answers are probably wrong, but I just want to move on.

\subsection{}

\textit{If $A \subseteq X$, we define the boundary of $A$ by the equation
  $$Bd A = \overline{A} \cap \overline{(X - A)}$$}

\textit{(a) Show that $Int A$ and $Bd A$ are disjoint and $\overline{A} = Int A \cup Bd A$.}

Suppose that $x \in Int A$. We follow that there exists open set $U$ such that
$x \in U$ and $U \subseteq A$. We follow that $U$ and $X \setminus A$ are disjoint.
Suppose that $x \in \overline{X \setminus A}$. Then we follow that
every neighborhood of $x$ intersects $X \setminus A$. Since $U$ is a neighborhood of $x$
we follow that $U \cap X \setminus A \neq \emptyset$. Therefore there exists
$j \in U$ such that $j \in X \setminus A$. Therefore $j \in U$ and $j \notin A$.
This contradicts the fact that $U \subseteq A$. Thus we follow that if $x \in Int A$
then $x \notin \overline{X \setminus A}$. Thus we follow that
$Int A \cap \overline{X \setminus A} = \emptyset$.
Now we can follow that
$$Bd A \cap Int A  = \overline{A} \cap \overline{X \setminus A} \cap Int A =
\overline{A} \cap \emptyset = \emptyset$$

We can also follow that $Bd A \subseteq \overline{A}$ by definition and
$Int A \subseteq \overline{A}$ by the fact that
$Int A \subseteq A \subseteq \overline{A}$. Therefore we follow that
$Int A \cup Bd A \subseteq \overline{A}$.

Now assume that $x \in \overline{A}$ and $x \notin Int A$. Since $x \notin Int A$
we follow that there is no neighborhood $U$ of $x$ such that $U \subseteq A$.
We follow that every neighborhood $U$ of $x$ has an element $y \in U$ such that
$y \notin A$. And since $U \subseteq X$, we follow that $U \cap (X \setminus A) \neq \emptyset$
for every neighborhood of $x$. Thus we follow that $x \in \overline{(X \setminus A)}$.
And since $x \in \overline{A}$, we follow that $x \in Bd A$ and thus
$$(\forall x \in \overline{A}) (x \notin Int A \ra x \in Bd A)$$
Thus we follow that $\overline{A} \setminus Bd A = Int A$ and since
both of $Bd A$ and $Int A$ are subsets of $A$ we conclude that
$\overline{A} = Bd A \cup Int A$, as desired.

\textit{(b) Show that $Bd A = \emptyset \lra A $ is both open and closed.}

If $Bd A = \emptyset$ we follow that $A = Int A$ and thus it is open.
We also follow that $\overline{A} = Bd A \cup Int A = Int A = A$
since $Int A \subseteq A \subseteq \overline{A}$, and thus $A$ is closed as well,
as desired.

If $A$ is both opened and closed we follow that $Int A = A$ and $\overline{A} = A$,
thus $Bd A = \emptyset$. 

\textit{(c) Show that $U$ is open $\lra$ $Bd U = \overline{U} \setminus U$.}

If $U$ is open, then we follow that $X \setminus U$ is closed and thus
$\overline{X \setminus U } = X \setminus U$. This implies that
$Bd U = \overline{U} \cap (X \setminus U)$ and since $\overline{U} \subseteq X$
and $U \subseteq \overline{U}$ we follow by some identity with a $\setminus$ that 
$Bd U = \overline{U} \setminus U$, as desired.

Suppose that $Bd U = \overline{U} \setminus U$. 
Since $\overline{U} = Bd U \cup Int U$ and $Int U \cap Bd U = \emptyset$, we follow that
$Int U = \overline{U} \setminus Bd U$. Thus
$Int U = \overline{U} \setminus (\overline{U} \setminus U)$ and since $U \subseteq \overline{U}$
we follow that
$Int U = U$, thus proving that $U$ is open, as desired.

\textit{(d) If $U$ is open, is it true that $U = Int \overline{U}$? Justify
  your answer.}

Suppose that $x \in U$. We follow that there exists open $V \subseteq U$
such that $x \in V \subseteq U$. Therefore $V \subseteq \overline{U}$.
Thus we follow that $V \subseteq Int \overline{U}$ and thus $x \in Int \overline{U}$.
Therefore we follow that $U \subseteq Int \overline{U}$. 

If $x \in Int(\overline{U})$. We follow that there exists $V$ such that
$x \in V \subseteq \overline{U}$. If $x \notin U$, then we follow nothing.

Let $U = R \setminus \set{0}$ and assume stanadrd topology. We follow that
$\overline{U} = R = Int \overline{U} \neq U$, which gives us a solid contradiction
of the reverse inclusion. 

\subsection{}

\textit{Skip}

\subsection{}

\textit{(Kuratowski) Consider the collection of all subsets $A$ of the topological space $X$.
  The operations of closure $A \to \overline{A}$ and complementation $A \to X \setminus A$
  are functions from this collection to itself. }

\textit{(a) Show that starting with a given set $A$, ine can form no more than 14
  distinct sets by applying therse two operations successively.}

We follow that if $X \neq \emptyset$, then $A \neq X \setminus A$.
We also know that $A = X \setminus (X \setminus A)$

Let $A = [0, 2] \setminus \set{1}$ so that it is neither open nor closed.
$$\overline{A} = [0, 2]$$
$$X \setminus \overline{A} = (-\infty, 0) \cup (2, \infty)$$
$$\overline{X \setminus \overline{A}} = (-\infty, 0] \cup [2, \infty)$$
$$X \setminus \overline{X \setminus \overline{A}} = (0, 2)$$
$$\overline{X \setminus \overline{X \setminus \overline{A}}} = [0, 2]$$

$$A = [0, 1) \cup \set{2} \cup (3, 4]$$
$$\overline{A} = [0, 1] \cup \set{2} \cup [3, 4]$$
$$X \setminus \overline{A} = (-\infty, 0) \cup (1, 2) \cup (2, 3) \cup (4, \infty)$$
$$\overline{X \setminus \overline{A}} = (-\infty, 0] \cup [1, 3] \cup [4, \infty)$$
$$X \setminus \overline{X \setminus \overline{A}} = (0, 1) \cup (3, 4)$$
$$\overline{X \setminus \overline{X \setminus \overline{A}}} = [0, 1] \cup [3, 4]$$
$$X \setminus \overline{X \setminus \overline{X \setminus \overline{A}}} =
(-\infty, 0) \cup (1, 3)\cup (4, \infty)$$
$$\overline{X \setminus \overline{X \setminus \overline{X \setminus \overline{A}}}} =
(-\infty, 0] \cup [1, 3]\cup [4, \infty)$$

SKIP for now, probably will come back later

\section{Continous Functions}

\subsection{}

\textit{Prove that for functions $f: R \to R$, the $\epsilon-\delta$ definition of
  continuity implies the open set definition.}

Let $f$ be continous with respect to the $\epsilon-\delta$ definition of
continuity.
Let $(a, b)$ be an element of basis of $R$. If $(a, b) \cap \ran(f) = \emptyset$,
then we follow that  $f\inv[(a, b)] = \emptyset$, which is an open set.

Suppose that $(a, b) \cap \ran(f) \neq \emptyset$. Let $y \in (a, b) \cap \ran(f)$.
Since $y \in (a, b)$, we follow that there exists $\epsilon$ such that
$V_\epsilon(y) \subseteq (a, b)$. 

Let $x \in f\inv[\set{y}]$.  For that $V_\epsilon(y)$ there exists $\delta > 0 \in R$
with corresponding  $V_\delta(x)$ such that $z \in V_\delta(x) \ra f(z) \in V_\epsilon(y)$.
By AC (not sure that we actually need AC
at this point, but we're doing topology, so why not) we follow that for each $y$
we can pick exclusive $\delta$ such that everything holds. 
Define $K: (a, b) \to \pow(R)$ by
$K(y) = \bigcup_{x \in f\inv[\set{y}]} {V_\delta(x)}$ in case that $y \in \ran(f)$ and
empty set otherwise.
Since $V_\delta(x)$ are all open intervals and empty sets are also open,
we follow that for all $y \in (a, b)$ $K(y)$ is an open set. Moreover, we follow that
$\bigcup{\ran(K)}$ is an open set as well. By definition of $K$ and $\epsilon-\delta$
continuity of $f$ we follow that $x \in \bigcup{\ran(K)} \ra f(x) \in (a, b)$ 

Now let $x \in R$ be such that $x \in f\inv[(a, b)]$. We follow that $x \in \bigcup{\ran(K)}$
by definition. Therefore we conclude that $f\inv[(a, b)] = \bigcup{\ran(K)}$.
As proven earler, $\bigcup{\ran(K)}$ is an open set, and therefore we conclude that
if $f$ is $\epsilon-\delta$-continous, then  for arbitrary interval $(a, b)$ we've got
that $f\inv[(a, b)]$ is open, and thus $f$ is continous according to our definition.
Taking into account stuff that we were given in the chapter, we follow that
$f$ is $\epsilon-\delta$-continous if and only if $f$ is continous, as desired.

\subsection{}

\textit{Suppose that $f: X \to Y$ is continous. If $x$ is a limit point of the
  subset $A$ of $X$, is it necessary true that $f(x)$ is a limmit point of $f(A)$?}

Short answer: no.

Let $U \subseteq Y$ be a neighborhood of $f(x)$. We follow that $f\inv[U]$ is open and
$x \in f\inv[U]$. Thus we follow that there exists a point $u \in f\inv[U]$
such that $u \neq x$ and $u \in A$. Thus $f(u) \in f[A]$. This means that we can't
follow crap on the account that $f$ might not be injective.

Let $f: R \to R$ be defined by $f(x) = 5$ and let us assume standard topology.
For $A = (0, 1)$ we follow that $x = 0$ is a limit point of $A$, but $\ran(f) = \set{5}$
and it doesn't have no limit points. Thus we've got a contradiction of our conjecture.

\subsection{}

\textit{Let $X$ and $X'$ denote a single set in the two topologies $\topol$ and
  $\topol'$ respectively. Let $X' \to X$ be the identity function. }

Just to be clear: we've assumed that we've got two topological spaces
$\eangle{X, \topol}$ and $\eangle{X', \topol'}$ such that $X = X'$, but $\topol$
might be different to $\topol'$. 

\textit{(a) Show that $i$ is continous $\lra$ $\topol \subseteq \topol'$}

Suppose that $i$ is continous. Let $U \in \topol$.
Thus $U$ is an open set in $X$.
We follow by continuity of $i$ that $i\inv[U] = U$ is an open set in $\topol'$.
Thus we follow that $U \in \topol'$. Thus we follow that $\topol \subseteq \topol'$,
as desired.

Same logic in reverse gets us the reverse implication.

\textit{(b) Show that $i$ is a homeomorphism $\lra \topol' = \topol$.}

If $i$ is a homeomorphism, then we follow that $i\inv: X \to X'$ is continous.
Previous point implies that $\topol' = \topol$.

If $\topol' = \topol$, then previous point implies that both $i$ and $i\inv$ are
continous and thus $i$ is an homeomorphism.

\subsection{}

\textit{Given $x_0 \in X$ and $y_0 \in Y$, show that the maps $f: X \to X \times Y$
  and $g: Y \to X \times Y$ defined by
  $$f(x) = x \times y_0$$
  $$g(x) = x_0 \times y $$
  are imbeddings
}


We follow that $\ran(f) = X \times \set{y_0}$ and $\ran(g) = \set{x_0} \times Y$.
Let $U_x \subseteq X$ and $U_y \subseteq Y$ be open sets.
We follow that
$$f[U_x] = U_x \times \set{y_0} = U_x \times Y \cap X \times \set{y_0} =
U_x \times Y \cap \ran(f) = $$
$$g[U_y] = \set{x_0} \times U_y = X \times U_y  \cap \set{x_0} \times Y =
U_x \times Y \cap \ran(g)$$
since $X \times U_y$ and $U_x \times Y$ are open sets in $X \times Y$ by the fact that
$U_x$ and $U_y$ are open, we follow that $f[U_x]$ and $g[U_y]$ are open as well.

Let $W \subseteq X \times Y \cap \ran(f)$ be an element of basis. We follow that
$W = U \times V \cap \ran(f)$ for some $U, V$ - open sets in $X$ and $Y$ respectively.
Since $\ran(f) = X \times \set{y_0}$, we follow that $W = U \times \set{y_0}$.
We follow that $W = f[U]$. Since $U$ is an open set and taking into
account previous paragraph, we conclude that $U$ is open if and only if $f[U]$ is
open and thus $f$ is an embedding. Simular argument for reverse implication
also holds for $g$, thus we follow that $g$ is an imbedding as well, as desired.

\subsection{}

\textit{Show that the subspace $(a, b)$ of $R$ is homeomorphic with $(0, 1)$ and the subspace
  $[a, b]$ of $R$ is homeomorphic with $[0, 1]$.}

In order to be clear I'll state here that we assume standard topology, and also
assume that $a, b \in R$ are constants such that $a < b$. 

Let $f_p: R \to R$ be defined by
$$f_p(x) = \frac{x - a}{b - a}$$
By the fact that this function is a constant shift of linear and nonconstant function,
we can follow that this function is bijective (we can also follow that through standard methods
in order to be more precise, but I'll skip that proof because it's pretty obvious).
We can also follow that this function is continous by the fact that
identity function is continouys and by algebraic properties of continuity,
that were proven in the course of real analysis (previous exercises and text in the chapter
implies that $\epsilon-\delta$-continuity is equivalent to topological continuity
with respect to standard topology, so that we're clear with using that stuff here).
Since $f_p$ is bijective we follow that it's got an inverse, and basic algebra implies that
$$f_p\inv(x) = (b - a)x + a$$
which is continous as well by algebraic properties of continuity. 

We can follow that $f_p[(a, b)] = (0, 1)$ and $f_p[[a, b]] = [0, 1]$ by some basic algebra.
Thus we follow that we can define bijections $f: (a, b) \to (0, 1)$ and $g: [a, b] \to [0, 1]$
by restriction of domain, i.e.
$$f = f_p|_{(a, b)}$$
$$g = f_p|_{[a, b]}$$
previous discussions imply that both $f, g$ are continous and bijective. We also follow that
$$f\inv = f_p\inv|_{(0, 1)}$$
$$g\inv = f_p\inv|_{[0, 1]}$$
thus implying that inverses of $f$ and $g$ are also continuous. Thus we follow the desired result.

\subsection{}

\textit{Find a function $f: R \to R$ that is continous at precisely one point.}

We've handled that in real analysis course, but I'll state this function here anyways.

We can modify Dirichlet's function to get
$$f(x) =
\begin{cases}
  x \notin Q \ra x \\
  x \in Q \ra 0
\end{cases}
$$
proof of the fact that tis function is continous exclusively at zero was handled
using $\epsilon-\delta$ definition in real analysis course.

\subsection{}

\textit{(a) Suppose that $f: R \to R$ is "continous from the right", that is 
$$\lim_{x \to a^+}f(x) = f(a)$$
for each $a \in R$. Show that $f$ is continous when considered as a function from $R_l \to R$}

Let $(a, b)$ be a basis element of $R$. Assume that $f\inv[(a, b)]$ is nonempty and
let $y \in f\inv[(a, b)]$. Assume that $x \in (a, b)$ is such that $f(x) = y$.
Let $\epsilon$ be such that $V_\epsilon(y) \subseteq (a, b)$. We follow that
there exists $\delta > 0$ such that $j \in (x, x + \delta)$ implies that $f(j) \in V_\epsilon(y)$.
We follow that $j \in [x, x + \delta)$ implies that $f(j) \in V_\epsilon(y)$ since
$f(j) = y \in V_\epsilon(y)$. Thus we follow that for every $y \in (a, b)$ there exists
interval $[x, x + \delta) \subseteq f\inv[(a, b)]$. Now we can define a function
$K: (a, b) \to \pow(R)$ such that
$$K(f(x)) = [x, x + \delta) \subseteq f\inv[(a, b)]$$
We follow that every value of $K$ is a basis element of $R_l$ and thus union of its range
is the open set in topology $R_l$. From the construction of $K$ we follow that
$\ran(K) = f\inv[(a, b)]$ and thus we conclude that $f\inv[(a, b)]$ is the open set
for all intervals $(a, b)$. Thus we follow that $f$ is continous, as desired.

\textit{(b) Can you conjecture what functions $f: R \to R$ are continous when
  considered as maps from $R$ to $R_l$? As maps from $R_l \to R_l$?
  We shall return to this question in chapter 3.}

We're required only to conjecture, so there're no wrong answers here.

Since $R_l$ is finer than $R$, we follow that any function, that is continous in $R \to R$ is
continous in $R_l \to R_l$. Same goes for $R \to R_l$.

For $R_l \to R_l$ i think that we're gonna have something that has only jump discontinuities.
For $R \to R_l$ I don't think that we're going to have more continous functions, since
the continuity is mainly governed by the domain set. 

I wonder if the uniform continuity is just a continuity with respect to some crazy topology,
or maybe it is some sort of a new form of continuity. Maybe it has something to do with
metrics, or maybe even measures.

\subsection{}

\textit{Let $Y$ be an ordered set in the order topology. Let $f, g: X \to Y$ be
  continous.}

\textit{(a) Show that the set $\set{x: f(x) \leq g(x)}$ is closed in $X$.}

Let $z \in X$ be such that $f(z) > g(z)$. Since $Y$ is ordered, we follow that it is Hausdorff,
and thus there exist disjoint neighborhoods $V_p$ of $f(z)$ and $U_p$ of $g(z)$.
By the fact that open intervals constitute a basis for an ordered topology, we follow that 
those respective neighborhoods have intervals $V \subseteq V_p$ and $U \subseteq U_p$
such that $f(z) \in V$ and $g(z) \in U$. Since the original neighborhoods are
disjoint, we follow that $V$ and $U$ are also disjoint, and thus we follow that
$$(\forall v \in V)(\forall u \in U)(v > u)$$
(in order to follow it faithfully we need to delve deeper into the boundaries and whatnot,
but the point is pretty obvious and trivial).
Since $f$ and $g$
are continous, we follow that $f\inv[V]$ and $g\inv[U]$ are open. Thus $f\inv[V] \cap g\inv[U]$
is open. We also have that $z \in f\inv[V] \cap g\inv[U]$. Let $j \in f\inv[V] \cap g\inv[U]$.
We follow that $f(j) \in V$ and $g(j) \in U$ and thus we follow that $f(j) > g(j)$.
Therefore we follow that $f\inv[V] \cap g\inv[U] \subseteq \set{x \in X: f(x) > g(x)}$.
Therefore we conclude that for all $j \in X$ if $f(j) > g(j)$ then
there is a neighborhood $f\inv[V] \cap g\inv[U] \subseteq \set{x \in X: f(x) > g(x)}$
such that $j \in f\inv[V] \cap g\inv[U]$. Thus we follow that we can once again create a
function $K: X \to f\inv[V] \cap g\inv[U]$ by setiing 
$$K(j) = f\inv[V] \cap g\inv[U]$$
if $j \in f\inv[V] \cap g\inv[U]$ or
$$K(j) = \emptyset$$
otherwise. Since $j \in \set{x \in X: f(x) > g(x)} \ra  j \in K(j)$ we follow that
$\set{x \in X: f(x) > g(x)} \subseteq \bigcup{\ran(K)}$ and we follow that
$ \bigcup{\ran(K)} \subseteq  \set{x \in X: f(x) > g(x)}$ by construction of the function.
Therefore we conclude that $\bigcup{\ran(K)} = \set{x \in X: f(x) > g(x)}$, and since
all values of $J$ are open sets we conclude that $\set{x \in X: f(x) > g(x)}$ is an
open set.

We can follow that
$$\set{x: f(x) \leq g(x)} = X \setminus \set{x \in X: f(x) > g(x)}$$
and since $\set{x \in X: f(x) > g(x)}$ is an open set we conclude that $\set{x: f(x) \leq g(x)}$
is closed, as desired.

\textit{(b) Let $h: X \to Y$ be the function
  $$h(x) = \min{f(x), g(x)}$$
  Show that $h$ is continous.
}

We know from a previous point that
$$A = \set{x \in X: f(x) \geq g(x)}$$
is closed. Also by this pont we follow that
$$B = \set{x \in X: f(x) \leq g(x)}$$
is closed. We can follow that functions
$g_p: A \to Y$, $f_p: B \to Y$ are continous with respect to the subset topology (pretty trivial
point).
Thus we follow that by the pasting lemma that the function $h: X \to Y$,
which is produced as in the text of lemma is continous. And by construction of $f_p$
and $g_p$ we follow that
$$h(x) = \min{f(x), g(x)}$$
thus $h$ is continous, as desired.

\subsection{}

\textit{Let $\set{A_\alpha}$ be a collection of subsets of $X$; let $X = \bigcup_\alpha{A_\alpha}$.
  Let $f: X \to Y$; Suppose that $f|_{A_\alpha}$ is continous for each $\alpha$.}

\textit{(a) Show that if the collection $\set{A_\alpha}$ is finite and each set $A_\alpha$ is
  closed, then $f$ is continous.}

If $A, B \in \set{A_\alpha}$, then we follow by pasting lemma that $f|_{A \cup B}: A \cup B \to X$
is continous. Thus we follow by induction that since $\set{A_\alpha}$ is finite that
$f|_{\bigcup{\set{A_\alpha}}} = f$ is continous, as desired.

\textit{(b) Find an example where the collection $\set{A_\alpha}$ is countable and each $A_\alpha$
  is closed, but $f$ is not continous.}

Let $f: [0, 1] \to R$ and let $A_0 = \set{0}$ and  $A_n = [1/n + 1, 1/n]$. Define
$$f(x) =
\begin{cases}
  1 \text{ if } x = 0 \\ 
  0 \text{ otherwise }
\end{cases}
$$

\textit{(c) An indexed family of sets $\set{A_\alpha}$ is said to be locally finite if each
  point $x$ of $X$ has a neighborhood that intersects $A_\alpha$ for only finitely many values
  of $\alpha$. Show that if the family $\set{A_\alpha}$ is locally finite and each $A_\alpha$
  is closed, then $f$ is continous.}

Let $x \in X$. We follow that there
exists $U \subseteq X$ - neighborhood of $x$ such that $U$ intersets
finitely many $A_\alpha$'s. Let $C$ be a union of $A_\alpha$'s that intersect $U$.
Let $u \in U$. We follow that since $\bigcup{A_\alpha} = X$ that
$u$ is in some $A_\alpha$, and thus $u$ is the point of intersection between $U$
and $C$. Thus we follow that $u \in C$, and therefore $U \subseteq C$.

We follow that $f|_C$ is continous by the first point. Thus $f|_U: U \to Y$ is
also continous. Let $K = f[U]$ with subspace topology. We follow that $g: U \to K$ is
continous.

Let $V$ be a neighborhood of $f(x)$. We follow that $V \cap K$ is open in the subspace
topology with respect to $K$. We also have that $V \cap K \subseteq K$. Thus we follow that
$g\inv[V \cap K]$ is open by continuity of $g$. We also follow that $x \in g\inv[V \cap K]$.
Since $U$ is open in $X$, we follow that $g\inv[V \cap K]$ is open in $X$ as well.
Thus we conclude that for all $x \in X$ and each neighborhood $V$ of $f(x)$ there
exists neighborhood $g\inv[V \cap K]$ of $x$ such that $f[g\inv[V \cap K]] \subseteq V$.
Therefore $f$ is continous, as desired.

\subsection{}

\textit{Let $f: A \to B$ and $g: C \to D$ be continous functions. Let us define a map
  $f \times g: A \times C \to B \times D$ by
  $$(f \times g)(a \times c) = f(a) \times g(c)$$
  Show that $f \times g$ is continous.}

Let $f_1: A \times C \to B$ be defined by
$$f_1(\eangle{a, c}) = f(a)$$
let $f_2: A \times C \to D$ be defined by
$$f_2(\eangle{a, c}) = g(c)$$
Let $U$ be an open set in $B$. Then we follow that
$$f_1\inv[U] = f\inv[U] \times C$$
Since $f$ is continous, we follow that $f\inv[U]$ is open. Thus $f\inv[U] \times C$ is
also open. Thus we follow that for all open $U \subseteq B$ we've got that $f\inv[U]$
is also open. Thus $f_1$ is continous. Same logic can be followed to show that $f_2$
is also continous.

Thus we conclude that we can define $h: A \times C \to B \times D$ by
$$h(\eangle{a, c}) = \eangle{f_1(a), f_2(c)}$$
by one of the theorems we follow that $h$ is continous.
We can follow that $h = f \times g$, and thus we conclude that $f \times g$ is
continous, as desired.

\subsection{}

\textit{Let $F: X \times Y \to Z$. We say that $F$ is continous in each variable separately if
  for each $y_0$ in $Y$, the map $h: X \to Z$ defined by $h(x) = F(x \times y_0)$ is
  continous, and for each $x_0 \in X$, the map $k: Y \to Z$ defined by $k(y) = F(x_0 \times y)$
  is continous. Show that if $F$ is continous, then $F$ is continous in each variable
  separately.}

Let $j: X \to X \times Y$ be defined by
$$j(x) = \eangle{x, y_0}$$

Let $U \times V$ be a basis element of $X \times Y$. We follow that if $y_0 \notin V$, then
$j\inv[U \times V] = \emptyset$, and if $y_0 \in V$, then $j\inv[U \times V] = U$.
Thus we follow that $j$ is a continous function. 

We can follow that $h = F \circ j$, and thus it is a composition of continous functions,
therefore making it itself a continous function. Simular logic can show that $k$ is
a continous function  as well.

\subsection{}

\textit{Let $F: R \times R \to R$ be defined by the equation
  $$F(x \times y) =
  \begin{cases}
    x\neq 0 \land y \neq 0 \ra \frac{xy}{x^2 + y^2} \\
    0 \text{ otherwise}
  \end{cases}
  $$
}

\textit{Show that $F$ is continous in each variable separately}

Let $y_0 = 0$. Then we follow that if $x \neq 0$, then
$$F(\eangle{x, y_0}) = \frac{0}{x^2 + 0} = 0$$
and if $x = 0$, then $F(\eangle{x, y_0}) = 0$. Thus we follow that if $y_0 = 0$, then $F$ is
a constant function. Same goes for $x_0 = 0$.

If $y_0 \neq 0$, then we follow that 
$$F(\eangle{x, y_0}) = \frac{xy_0}{x^2 + y^2}$$
which is continous by algebraic properties of continuity. Same thing holds for $x_0 \neq 0$.
Thus we follow that $F$ is continous in each variable separately

\textit{(b) Compute the function $g: R \to R$ defined by $g(x) = F(x \times x)$.}

We follow that if $x \neq 0$, then
$$g(x) = F(x, x) = \frac{x^2}{2x^2} = 1/2$$
and
$$g(0) = 0$$
thus we follow that
$$g(x) =
\begin{cases}
  x = 0 \ra 0  \\
  1/2 \text{ otherwise}
\end{cases}
$$
therefore it is obviously not continous.

\textit{(c) Show that $F$ is not continous}

Let $c = \eangle{0, 0}$. We follow that $F(c) = 0$. We follow that $(-0.1, 0.1)$ is a
neighborhood of $0$. Let $C$ be a neighborhood of $c$. We follow that there is a basis
element $(a_1, b_1) \times (a_2, b_2) \subseteq C$ such that that basis element is a
neighborhood of $c$. Since $c \in (a_1, b_1) \times (a_2, b_2)$,
we follow that $a_1, a_2 < 0 < b_1, b_2$. We can follow that
there is an element $d \in (a_1, b_1) \times (a_2, b_2)$ such that $d = \eangle{x_1, x_1}$
for some $x_1 \neq 0$ (albeit tedious, this little fact can be followed from the
properties of intervals). Thus we follow that $f(d) = 1/2$, and thus $f(d) \notin (-0.1, 0.1)$,
which contradicts one of the equivalent statements of continuity.

\subsection{}

\textit{Let $A \subseteq X$; let $f: A \to Y$ be continous; let $Y$ be Hausdorff. Show that
  if $f$ may be extended to a continous function $f: \overline{A} \to Y$, then $g$ is
  uniquely determined by $f$.}

Let $f: A \to Y$ be defined as in the exercise. Suppose that there exist two
functions $g_1, g_2: \overline{A} \to Y$ such that both of them are
continous and $f \subseteq g_1 \land f \subseteq g_2$.

If $g_1 \neq g_2$, then we follow that there exists $x \in \overline{A}$ such that
$g_1(x) \neq g_2(x)$. Since $f \subseteq g_1, g_2$, we follow that $x \notin A$. 
We follow that $g_1(x), g_2(x) \in Y$, and because
$Y$ is Hausdorff, we follow that there exist two disjoint neighborhoods $U, V \subseteq Y$
of $g_1(x)$ and $g_2(x)$ respectively. Since both $g_1, g_2$ are continous, we follow that
$g_1\inv[U]$ and $g_2\inv[V]$ are neighborhoods of $x$. We also follow that
since $U$ and $V$ are disjoint, we've got that
$$g_1\inv[U] \cap g_1\inv[V] = \emptyset$$
$$g_2\inv[U] \cap g_2\inv[V] = \emptyset$$
but $x \in g_1\inv[U] \cap g_2 \inv[V]$ and thus $g_1\inv[U] \cap g_2 \inv[V]$
is nonempty. By the fact that $g_1\inv[U]$ and $g_2\inv[V]$ are open, we follow that
$g_1\inv[U] \cap g_2\inv[V]$ is a finite intersection of open sets and thus it is
open as well. Thus we follow that $g_1\inv[U] \cap g_2\inv[V]$ is a neighborhood of $x$.
Thus we follow that it intersects $A$ at some point $a \in A$.
Since $x \notin A$, we follow that $a \neq x$.
Therefore we conclude that there's a point $a \in A$ such that $a \in g_1\inv[U] \cap g_2\inv[V]$.
Since $a \in A$, we follow taht $g_1(a) = g_2(a) = f(a)$. Thus we can conclude that
$$a \in g_1\inv[U] \ra g_1(a) \in U$$
$$a \in g_2\inv[V] \ra g_2(a) \in V$$
and since $g_1(a) = f(a) = g_2(a)$, we follow that $f(a) \in U \cap V$, which contradicts
our assumption that $U \cap V = \emptyset$.

Therefore we conclude that if $Y$ is Hausdorff, $f: A \to Y$ and there exist
continous functions
$g_1, g_2: \overline{A} \to Y$ such that $f \subseteq g_1, g_2$, then $g_1 = g_2$,
which is not the exact wording of the conclusion of the exercise, but its somewhat more
exact equivalence.

\section{The Product Topology}

\subsection{}

\textit{Prove Theorem 19.2}

Let $X_\alpha$ be an original indexed set of topological spaces and $B_\alpha$ be an arbitrary
element of the basis. We follow that since each $B_\alpha$ is open that
$\prod B_\alpha$ is an elemebt of a basis of the box topology and therefore it's open.

Let $x$ be an arbitrary point in the original space with corresponding
neighborhood $U$. We follow that there is an element of original basis $\prod U_\alpha$
that is a subset of $U$ and contains $x$. We follow by properties of bases
that for each $U_\alpha$ there is an element of basis $B_\alpha$ such that
$x_\alpha \in B_\alpha \subseteq U_\alpha$. Therefore we can conclude that there is
a set $\prod B_\alpha$ such that $x \in \prod B_\alpha \subseteq \prod U_\alpha$. Therefore
we conclude by Lemma 13.2 that collection of products of elements of bases indeed
constitutes a basis for a box topology, as desired. Pretty much the same logic
with minor tweaks will will imply simular result for product topologies.

\subsection{}

\textit{Prove Theorem 19.3}

Firstly, we state that
$$\prod U_\alpha \cap \prod A_\alpha = \prod{(U_\alpha \cap A_\alpha)}$$
the proof for this statement is somewhat redundant, but I like FOL, so I'll state it
anyways:
$$x \in \prod U_\alpha \cap \prod A_\alpha \lra
x \in \prod U_\alpha \land x \in \prod A_\alpha \lra 
(\forall \alpha \in J)(x_\alpha \in U_\alpha) \land (\forall \alpha \in J)(x_\alpha \in A_\alpha) \lra $$
$$ \lra 
(\forall \alpha \in J)(x_\alpha \in U_\alpha \land x_\alpha \in A_\alpha) \lra
(\forall \alpha \in J)(x_\alpha \in U_\alpha \cap A_\alpha) \lra 
x \in \prod{(U_\alpha \cap A_\alpha)}$$
Lemma 16.1 with our definition and logic in Lemma 16.3 now implies the desired result.

\subsection{}

\textit{Prove Theorem 19.4}

Statement in the previous exercise and logic from exercise 17.11 (2.6.11) imply the desired
result.

\subsection{}

\textit{Show that $(X_1 \times ... \times X_{n - 1}) \times X_n$ is homeomorphic with
  $X_1 \times ... \times X_n$.}

In set theory course we've defined cartesian product for finite amount of elements
recursively, so that exercise boils down to basically nothing. I think that here
we need to show that with our new definition of cartesian product everything works.

Let $f: (X_1 \times ... \times X_{n - 1}) \times X_n \to X_1 \times ... \times X_n$
be defined by
$f(\eangle{u, v}) = \eangle{u_1, u_2, ..., u_{n - 1}, v}$
By common sense we follow that $f$ is bijective. Let $V$ be an open set in
$X_1 \times ... \times X_n$. Let $B$ be an element of a basis of $X_1 \times ... \times X_n$.
We follow that $B = \prod U_\alpha$ for $\alpha \in \set{1, ..., n}$ and some open
sets $U_\alpha$. We follow that
$$f\inv[B] = (\prod_{\alpha \in \set{1, ..., n - 1}} U_\alpha) \times U_n$$ 
and thus it is open. Thus $f$ is continous. We also follow that
$$f\left[\left(\prod_{\alpha \in \set{1, ..., n - 1}}{U_\alpha}\right) \times U_n\right] =
\prod U_\alpha$$
thus we follow that $f\inv$ is also continous, thus $f$ is homeomorphism and both
spaces are homeomorphic.

\subsection{}

\textit{One of the implications stated in Theorem 19.6 holds for the box topology.
  Which one? }

Example 2 gives a contradiction for reverse implication, therefore the forward implication
must be true.

\subsection{}

\textit{Let $x_1, x_2, ... $ be a sequence of points of the product space $\prod X_\alpha$.
  Show that this sequence converges to the point $x$ if and only if the sequence
  $\pi_\alpha(x_1), \pi_\alpha(x_2), ...$  converges to $\pi_\alpha(x)$ for each $\alpha$.
  Is this fact true for if one uses the box topology instead of  the product topology?
}

Suppose that the sequence
$$x_1, x_2, ...$$
converges to a point $x$. Then we follow that for each neighborhood $U$ of $x$ there is a
$n \in \omega$ such that $m \geq n \ra x_m \in U$.

Let $V$ be a neighborhood of $\pi_\alpha(x)$. Just to be clear here, we're gonna state here
that $\pi_\theta: \prod X_\alpha \to X_\theta$, and thus $\pi_\theta(x) \in X_\theta$.
It can be followed easily (if it wasn't done already) that $\pi_\theta$ is a continous function
and thus $\pi_\theta\inv(V)$ is an open set in $\prod X_\alpha$. Thus we follow that there
is $n \in \omega$ such that $m \geq n \ra x_m \in \pi_\theta\inv(V)$. Since
$x_m \in \pi_\theta\inv(V)$, we follow that $\pi_\theta(x_m) \in V$. Since everything here
was chosen arbitrarily, we follow that for each neighborhood $V$ of $\pi_\theta(x)$ there
is $n \in \omega$ such that $m \geq n \ra x_m \in V$.

Suppose that $\pi_\alpha(x_n)$ converges for every $\alpha$. Let $U$ be a neighborhood of
$x$. Since $U$ is a neighborhood in box topology, we follow that it is an
arbitrary union of finite intersection of finite elements in form $\pi_\theta\inv[U_\theta]$.
Thus we follow that there exists a finite set $C$, whose elements
are in form $\pi_\theta\inv[U_\theta]$, and $x \in \bigcap{C} \subseteq U$ (trivial proof).
Thus we follow that there exists a finite set $D \subseteq \omega$ such that
$$n \in D \ra (\exists \mu)(\pi_\mu[U_\mu] \in C)$$
Since $D$ is finite, we follow that there is a maximal element of it, and thus there
is $n \in \omega$ such that
$$m \in \omega \ra x_m \in \bigcap{C} \subseteq U$$
thus we can conclude that our original sequence converges, which proves reverse direction.

We can follow that box topology will work just fine in forward direction, but we can
make a whacky open set (like one in the example 2) in $\prod X_n$ so that reverse
direrction does not work with infinite "dimentions".

\subsection{}

\textit{Let $R^\infty$ be the subset of $R^\omega$ consisting of all sequences that are
  "eventually zero", that is, all sequences $(x_1, x_2, ...)$ such that $x_i \neq 0$ for only
  finitely many values of $i$. What is the closure of $R^\infty$ in $R^\omega$ in the box
  and product topologies?. Justify your answer.
}

Let $x \in R^\omega$ and let $U$ be an arbitrary neighborhood of $x$ with respect to
product topology.
Since $U$ is an open set, we follow that there's a finite set $C$ that consists
of elements in form $\pi_\alpha\inv[U_\alpha]$ and such that $x \in \bigcap{C} \subseteq U$.
Since $C$ is finite, we follow that we can make a function $f: \omega \to R$ such that
$$
\begin{cases}
  \pi_\alpha\inv[U_\alpha] \in C \ra f(\alpha) \in \pi_\alpha\inv[U_\alpha]  \\
  0 \text{ otherwise}
\end{cases}
$$
we then follow that $f \in R^\infty$, and thus we conclude that every neighborhood of $x$
intersects $R^\infty$ at some point. Thus we conclude that
with respect to product topology, $\overline{R^\infty} = R^\omega$.

Now assume box topology and let $x \in R^\omega \setminus R^\infty$. We follow that
there is an infinite set $D \subseteq \omega$ such that $d \in D \ra x_d \neq 0$.
Assuming that topology of $R$ is standart, we follow that for every $d \in D$ there
is an open set $(a, b)$ such that $0 \notin (a, b)$ and $x \in (a, b)$.
We now can define an indexed set (which is a fancy name for a function) $A: \omega \to \pow(R)$
by
$$
\begin{cases}
  a \in D \ra A(a) = (a, b) \\
  A(a) = R \text{ otherwise}
\end{cases}
$$
which won't have no elements of $R^\infty$ on the account that $D$ is infinite and $0 \notin (a, b)$.
Thus we conclude that for every $x \in R^\omega \setminus R^\infty$ there's a neighborhood
of $x$ that doesn't contain no elements of $R^\infty$. Thus we conclude that
$\overline{R^\infty} = R^\infty$ in box topology.

\subsection{}

\textit{Given sequences $(a_1, a_2, ...)$ and $(b_1, b_2, ...)$ of real numbers with $a_i > 0$
  for all $i$ define $h: R^\omega \to R^\omega$ by the equation
  $$h((x_1, x_2, ...)) = (a_1 x_1 + b_1, a_2 x_2 + b_2, ...)$$
  Show that if $R^\omega$ is given the product topology, $h$ is homeomorphism of $R^\omega$ with
  itself. What happens if $R^\omega$ is given the box topology?
}

Since $a_i \neq 0$ for all $i$, we follow that $h$ is a bijection (trivial proof).

Let $A$ be a basis element of $R^\omega$ with respect to product topology. We follow that it
is an indexed set $A_\alpha$ such that $A_\alpha = (c, d)$ or $A_\alpha = R$, and there
are finitely many $\alpha$'s such that $A_\alpha = (c, d)$. Let $n \in \omega$.
If $n = \alpha$ for some $\alpha \in J$, then there's an interval $(e, f)$ such that
$h_\alpha\inv[(e, f)] = (c, d)$ (trivial proof, for clarifications use common sence).
We also follow that $h_\alpha[R] = R$, and thus
there is an open set $C$, that consists of product of those intervals and $R$'s such that
$h\inv[C] = U$. Simular logic holds in reverse direction. Thus we conclude that $h$ is
homeomorphism.

Since we haven't used the fact that a basis element consists of finitely many iintervals,
we conclude that $h$ is a homeomorphism with respect to a box topology as well.

\subsection{}

\textit{Show that the choice axiom is equivalent to the statement that for any indexed family
  $\set{A_\alpha}_{\alpha \in J}$ of nonempty sets, with $J \neq 0$, the cartesial product is
  not empty.}

Forward direction is directly implied by AC (or its lemma with choice function).
Reverse implication is almost rewording of AC (or its lemma with choice function).
For more GOTO set theory course (2nd and 5th chapter if I remember it correctly).

\subsection{}

\textit{Skip for now, but by looking at it, we can probably expect some Zorn Lemma action coming}

\section{The Metric Topology}

\subsection{}

\textit{(a) In $R^n$, define
  $$d'(x, y) = \sum{|x_i - y_i|}$$
  Show that $d'$ is a metric that induces the usual topology of $R^n$. Sketch the basis element
  under $d'$ when $n = 2$. 
}

Since it's a sum of nonnegative numbers, we follow that $d'(x, y) \geq 0$. We also follow that
if $x \neq y$, then there exists  $i < n$ such that $x_i \neq y_i$, thus $|x_i - y_i| > 0$
and thus $d'(x, y) > 0$. We also follow that $x = y$, then $d'(x, x) = 0$.
Since $|x_i - y_i| = |y_i - x_i|$, we follow that $d'(x, y) = d'(y, x)$.
By triangle inequality of normal absolute value function we follow the triangle inequality for
this metric.

We can follow that
$$\max\set{|x_1 - y_1|, ..., |x_n - y_n|} \leq \sum{|x_i - y_i|}$$
and thus if we set $\delta = \epsilon$ then 
$$B_{d'}(x, \delta) \subseteq B_{d}(x, \epsilon)$$
thus we follow that this topology is finer standart this topology.
We also follow that
$$ \frac{\sum{|x_i - y_i|}}{n} \leq \max\set{|x_1 - y_1|, ..., |x_n - y_n|}$$
and thus we can set $\delta = \frac{\epsilon}{n}$ to get
$$B_{d}(x, \delta) \subseteq B_{d'}(x, \epsilon)$$
thus standart topology is finer than given topology. Therefore we conclude that the topologies
are equal, as desired.

I think that this metric is also called a Manhattan distance. For this one an open set
is a diamond in $R^2$.

\textit{(b) More generally, given $p \geq 1$, define
  $$d'(x, y) = \left[\sum_{i = 1}^{n}{|x_i - y_i|}\right]^{1/p}$$
  for $x, y \in R^n$. Assume that $d'$ is a metric. Show that it induces the usual
  topology on $R^n$.
}

We follow that
$$\max\set{|x_1 - y_1|, ..., |x_n - y_n|} \leq
\left(\left[\sum_{i = 1}^{n}{|x_i - y_i|}\right]^{1/p}\right)^p$$
and
$$\frac{\left(\left[\sum_{i = 1}^{n}{|x_i - y_i|}\right]^{1/p}\right)^p}{n} \leq
\max\set{|x_1 - y_1|, ..., |x_n - y_n|}$$
thus we can set $\delta = \epsilon^p$ in the former case and $\delta = \frac{\epsilon^{1/p}}{n}$
in the latter case to get the desired result (or some variation of those, but the point is clear).

\subsection{}

\textit{Show that $R \times R$ in the dictionary order is metrizable.}

We can define a metric by
$$d(\eangle{x_1, x_2}, \eangle{y_1, y_2}) =
\begin{cases}
  x_1 = y_1 \ra \min(|x_2 - y_2|, 1) \\
  1 \text{ otherwise}
\end{cases}
$$
this is a somewhat modified bounded metric, from which we follow that it is a metric.
Proof that this metric indeed generates a dictionary topology is somewhat trivial.

\subsection{}

\textit{Let $X$ be a metric space with metric $d$.}

\textit{(a) Show that $d: X \times X \to R$ is continous.}

Let us firstly assume that $R$ has a standart topology, $X$ has a metric topology
with respect to $d$ and $X \times X$ has a product topology.
Let $U$ be a basis element of $R$.
We follow that $U = V_\epsilon(x)$ for some $x, \epsilon \in R$.
Let $W = d\inv[U]$. Suppose that $\eangle{z, y} \in W$. We follow that
$d(z, y) \in V_\epsilon(x)$, and thus
$$x - \epsilon < d(z, y) < x + \epsilon$$
which a syntactic sugar for
$$x - \epsilon < d(z, y) \land d(z, y)  < x + \epsilon$$

Let $\eangle{z', y'} \in X \times X$ be such that
$$d(z', y') < x + \epsilon$$
Let
$$\delta = \frac{x + \epsilon - d(z', y')}{2}$$
Let $B_\delta(z')$ and $B_\delta(y')$ be basis elements. Let $k \in B_\delta(z')$
and $l \in B_\delta(y')$. Then we follow that
$$d(k, l) \leq d(k, z) + d(z', y') + d(y', l) < d(z', y') + 2 \delta = $$
$$ = 
d(z', y') + 2 \frac{x + \epsilon - d(z', y')}{2} = d(z', y') + x + \epsilon - d(z', y') =
x + \epsilon $$
thus we follow that if $\eangle{k, l} \in B_\delta(z') \times B_\delta(y')$ then 
$$d(k, l) < x + \epsilon $$
Since $B_\delta(z')$ and $B_\delta(y')$ are basis elements, we follow that they are open,
thus $B_\delta(z') \times B_\delta(y')$ is an open element in $X \times X$ and thus we follow that
for every
$$w \in \set{q \in X \times X: d(q) < x + \epsilon}$$
there is an open subset $U$  of $X \times X$ such that $w \in U$
$$U \subseteq \set{q \in X \times X: d(q) < x + \epsilon}$$
Thus we follow that $\set{q \in X \times X: d(q) < x + \epsilon}$ is an open set.

Let $\eangle{z', y'} \in X \times X$ be such that
$$x - \epsilon < d(z', y')$$
from triangular inequality we follow that
we can set 
$$\delta = \frac{d(z', y') - (x - \epsilon)}{2}$$
define $k, l$ and such as in previous paragraph. From triangular inqeuality we follow that 
$$d(z', y') \leq d(k, z') + d(k, l) + d(l, y')$$
$$d(z', y') - d(k, z') - d(l, y') \leq d(k, l)$$
and since $d(k, z'), d(l, y') < \delta$, we follow that
$$d(z', y') - 2\delta \leq d(k, l)$$
thus
$$d(k, l) \geq d(z', y') - 2\delta =  d(z', y') - 2 \frac{d(z', y') - (x - \epsilon)}{2} =
d(z', y') - d(z', y') + (x - \epsilon) =  $$
$$ = x - \epsilon $$
thus by the same logic as in the previous paragraph we follow that 
$$\set{q \in X \times X: d(q) > x - \epsilon}$$
is open.

Now we can follow that
$$\set{q \in X \times X: d(q) < x + \epsilon} \cap \set{q \in X \times X: d(q) > x - \epsilon}$$
is a finite intersection of open sets, and therefore it is open. By doing some set algebra
we get
$$\set{q \in X \times X: d(q) < x + \epsilon} \cap \set{q \in X \times X: d(q) > x - \epsilon} =$$
$$ = \set{q \in X \times X:  d(q) > x - \epsilon \land d(q) < x + \epsilon } = $$
$$ = \set{q \in X \times X:  x - \epsilon < d(q) \land d(q) < x + \epsilon } = $$
$$ = \set{q \in X \times X:  x - \epsilon < d(q) < x + \epsilon }$$
is an open set, which is precisely equal to $W$. Therefore we conclude that $d\inv$
maps open sets to open sets, and therefore $d$ is continous, as desired.

\textit{(b) Let $X'$ denote a space having the same underlying set as $X$. Show that if
  $d: X' \times X' \to R$ is continous, then the topology of $X'$ is finer than the topology
  of $X$.}

Let $x \in X$ and let $U$ be a basis element of $X$ such that $x \in U$.
We follow that since $U$ is a basis element, there's $k \in R$ and $y \in X$ such
that $U = B_k(y)$. We now can follow that $(-1, k)$ is open in $R$ and thus
$d\inv[(-1, y)]$ is open in both $X \times X$ and $X' \times X'$. Now we want to follow that
$$d\inv[(-1, y)] \cap \set{y} \times R = \set{y} \times B_k(y)$$

Let us firstly state and prove a theorem:

\textbf{Strand theorem: }
\textit{
  Let $X, Y$ be topological spaces and let $x \in X$ and $y \in Y$.
  We follow that if $U$ is an open set in $X \times Y$, then
  $$U \cap X \times \set{y} = V \times \set{y}$$
  and
  $$U \cap \set{x} \times Y = \set{x} \times W $$
  where $V$ is an open set in $X$ and $W$ is an open set in $Y$.
}

Suppose that $U$ is an open set in $X \times Y$. We follow that there is an indexed collection
(i.e. a function) $A_j$
of elements of basis of $X \times Y$ such that
$$U = \bigcup_{j \in J}{A_j}$$
We follow that since $A_j$ is a basis element for all $j \in J$ we've got that
$$A_j = X_j \times Y_j$$
such that $X_j$ and $Y_j$ are open sets in $X$ and $Y$ respectively. Let $K \subseteq J$
be such that if $k \in K$ then $y \in Y_k$. Thus we follow that
$$\bigcup_{j \in J}{A_j} \cap X \times \set{y} =
\left(\bigcup_{k \in K}{A_k} \cup \bigcup_{j \in J \setminus K}{A_j}\right)  \cap X \times \set{y} =
$$
$$ = 
\left(\bigcup_{k \in K}{A_k} \cap X \times \set{y}\right)  \cup
\left( \bigcup_{j \in J \setminus K}{A_j} \cap X \times \set{y} \right)=
 \left( \bigcup_{k \in K}{A_k} \cap X \times \set{y} \right) \cup \emptyset=$$
$$ = \bigcup{A_k} \cap X \times \set{y}
$$
Now we want to follow that
$$\bigcup{A_k} \cap X \times \set{y} = \bigcup{X_k} \times \set{y}$$
$$q \in \bigcup{A_k} \cap X \times \set{y} \lra q = \eangle{x, y} \land x \in \bigcup{X_k} \land
x \in X \lra
q = \eangle{x, y} \land x \in \bigcup{X_k} \lra q \in \bigcup{X_k} \times \set{y}$$
since $X_k$ is a collection of open sets in $X$, we follow that we can set $V = \bigcup{X_k}$
to get the desired result. The same logic holds for the latrer case as well.

\qed

From the strand theorem we follow that $B_k(y)$ is open in $X'$. Thus we follow that
topology on $X$ is coarser than $X'$, as desired.

\subsection{}

\textit{Consider the product, uniform and box topologies on $R^\omega$.}

Let $U_p$, $U_u$ and $U_b$ be basis elements of product, uniform and box topologies respectively.

\textit{(a) In which topologies are the following functions from $R$ to $R^\omega$ continous.
  $$f(t) = (t, 2t, 3t, ...)$$
  $$g(t) = (t, t, t, ...)$$
  $$h(t) = (t, \frac{1}{2} t, \frac{1}{3} t, ...)$$
}

Consider the product topology. Let $U$ be a basis element such that $f\inv[U], g\inv[U],
h\inv[U] \neq \emptyset$ (not necessarily at the same time, but the point is clear).
We follow that
$$U = \prod_{j \in J}{A_J}$$
We follow that there're finitely many intervals $A_j$ such that $A_j \neq R$ and such that
$A_j$ is an open interval in $R$. Let $K = \bigcap_{j \in J} A_J$. We follow by the fact that
there are finitely many $j$'s such that $A_J \neq R$ that $f\inv(U)$ is an open set,
restriction on which comes from the last $A_j$ such that $A_j \neq R$. We can follow that 
$$g\inv(U) = K$$
and we can follow  that $h$ is restricted by the first $A_j$ such that $A_j \neq R$.
Therefore we conclude that all of the functions are continous in product topology.

When it comes to the uniform topology, then we follow that
we can create basis  elements 
$$U = B((1/2 - 1/3, 1/3 - 1/3, 1/4 - 1/3, ...), 1/3)$$
such that $f\inv[U] = g\inv[U] = \set{0}$. Thus not $g$ nor $f$ are continous in uniform
topology. We can do a set, simular to the one in box topology set, so that $h$ is
$h\inv[U]$ is not open as well. Therefore we conclude that no given functions are
continous under uniform topology.


Consider the box topology. We follow that there's a basis element of box topology
$$U = (-1/2, 1/2) \times (-1/3, 1/3) ... $$
we can follow that the only $V \subseteq R$ such that $f(t) \in U$ is $V = \set{0}$.
We can follow that $V$ is not an open set in $R$, and thus $f$ and $g$ are not continous
with respect to box topology.

We can also create an open set
$$K = \prod{(1/n - 1/n^{2}, 1/n + 1/n^2)}$$
We follow that $h(1) \in K$. If $j \neq 1$ thought, then let $k = j - 1$.
We follow that if $k > 0$, then
$$h(j) = h(k + 1) = (k + 1, \frac{1}{2}k + \frac{1}{2}, ....)$$
Since $k \in R$, we follow that there is $l \in Q$ such that $0 < l < k$.
We follow that there's $o, p \in N$ such that $l = o/p$ and thus
Thus we follow that
$$\frac{1}{n}l + \frac{1}{n} = \frac{o}{np} + \frac{1}{n}$$
and thus there's some  $n \in N$ (namely some $n > p$) such that
$$ \frac{o}{np} >  \frac{1}{n^2}$$
and since $h(l) < h(k)$ we conclude that $h(k) \notin K$. Thus we follow that
$h\inv[K] = \set{1}$, which proves that $h$ is not continous as well.

\textit{(b) In which of topologies do the following sequences converge?}

Firtly, we must state that all of the sequences either converge to $(0, 0, ...)$
or don't converge at all. For the simplicity of notation I'm gonna use $0$ as a
shorthand for this vector.

Consider the box topology. We follow that there's
$$U = \prod_{j \in \omega} (1, -1)$$
we follow that $0 \in U$ but every element of $w_n$ such that $n > 1$
has the property that $0 \notin w_n$. Thus we conclude that $w_n$ does
not converge in box topology.

If we set
$$U = \prod_{j \in \omega} (1/j^2, -1)$$
then the fact that $1/j^2 < 1/j$ for all $j \in \omega$ will give us that $x_n$
does not converge in box topology. Same $U$ will suffice to show that $y_n$
does not converge in box topology because for every $n \in \omega$ there's $j \in \omega$
such that $1/n > 1/j$, thus giving us that at $n$'th position both $y_n$ and $x_n$ won't
be confined inside $U$.

Case with $z_n$ is different though. We follow that if $U$ is a basis element in box topology,
then there's two open intervals  $A_1, A_2$ such that $A_1, A_2 \in \ran U_j$.
We follow that $A_1 \cap A_2$ is an interval as well, and thus we follow that there's
$n \in \omega$ such that $1/n \in A_1 \cap A_2$. Thus we conclude that $z_n$ will converge
in box topology. Given that both uniform and product topologies are coarser than box
topology, we can follow that $z_n$ converges in those topologies as well.

Consider uniform topology. We follow that
$$0 \in B(0, 1)$$
but $j > 2 \ra w_j \notin B(0, 1)$. Thus we follow that $w_n$ does not converse in uniform
topology. We can use a method of sliding interval (just as in part (a)) to conclude that
$x_n$ does not converge in uniform topology.

We can follow that if $x \neq (k, 0, 0, ...)$ for some $k \in R$, then $0 \notin B(x, \epsilon)$.
Thus we conclude that if $0 \in B(x, \epsilon)$, then $x = (k, 0, 0, ...)$. We can also
follow pretty easily that $\epsilon > k$. Thus we can follow that $y_n$ converges in uniform
topology. Since product topology is coarser than uniform, we follow that $y_n$ converges
in product topology as well.

Consider product topology. Let $U$ be a basis element such that $0 \in U$. We
follow that
$$U = \prod_{j \in \omega} U_j$$
and by the nature of basis elements in product topologies we follow that there's $n \in \omega$
such that $U_j \neq R \ra j < n$. Thus we follow that there's $n \in \omega$ such that
$w_n \in U$. Therefore we conclude that $w_n$ converges in product topology.
Slightly modified logic can tell us that $x_n$ converges in product topology as well.

\subsection{}

\textit{Let $R^\infty$ be the subset of $R^\omega$ consisting of all sequences that
  are eventually zero. What is the closure of $R^\infty$ in $R\omega$ in the
  uniform topology?. Justify your answer.}

I want to say that it's all the sequences that converge to 0.

Suppose that $z_n \in R^\omega$  is such that it does converge to 0. Let $B(x_n, \epsilon)$
be a basis element of uniform topology such that $z_n \in B(x, \epsilon)$.
We follow that there's $n \in \omega$ such that $m \geq n \ra |x_n| < \epsilon$.
This gives us that there's an element $y_n$ of $R^\infty$, specifically
$$
\begin{cases}
  m < n \ra y_m = x_n \\
  m \geq  n \ra y_m = 0 \\
\end{cases}
$$
such that $y_n \in B(x_n, \epsilon)$. Thus we conclude that if $z_n$ converges to $0$, then
$z_n \in \overline{R^\infty}$ in uniform topology.

Let $x_n \in R^\omega$  be such that it does not converge to 0.
We follow that there exists $\epsilon > 0$ such that for all $n \in \omega$
$$|x_n| > \epsilon$$
moreover, we follow that there's $\epsilon \in (0, 1)$ that will do the job.
Thus we conclude that there's a basis element $B(x_n, \epsilon)$ such that there's no
element $z$ of $R^\infty$ such that $z \in B(x_n, \epsilon)$. Therefore we
conclude that if $x_n$ does not converge to 0, then $x_n \notin \overline{R^\infty}$.

Thus we can conclude that $x_n \in \overline{R^\infty}$ if and only if $x_n$
converges to 0 (or so I think).

\subsection{}

\textit{Let $\rho$ be the uniform metric on $R^\omega$. Given $x = (x_1, x_2, ...) \in R^\omega$
  and given $0 < \epsilon < 1$, let
  $$U(x, \epsilon) = (x_1 - \epsilon, x_1 + \epsilon) \times
  (x_2 - \epsilon, x_2 + \epsilon) \times ... $$
}

\textit{(a) Show that $U(x, \epsilon)$ is not equal to $\epsilon$-ball $B_\rho(x, \epsilon)$.}

Let
$$y = (x_1 + \frac{1}{2} \epsilon, x_2 + \frac{2}{3} \epsilon, ...)$$
we follow that $y \in U(x, \epsilon)$. At the same time we follow that
$$\rho(x, y) = \sup\set{\overline{d}(x_n, y_n): n \in \omega} =
\sup\set{\frac{n - 1}{n} \epsilon: n \in \omega} = \epsilon$$
since
$$y \in B_\rho(x, \epsilon) \lra \rho(x, y) < \epsilon$$
we follow that $y \notin B_\rho(x, \epsilon)$.

\textit{(b) Show that $U(x, \epsilon)$ is nor even open in the uniform topology}

We can follow that there's no basis element of uniform topology such that it contains
$y$ and is a subset of $U(x, \epsilon)$.

\textit{(c) Show that
  $$B_\rho(x, \epsilon) = \bigcup_{\delta < \epsilon}{U(x, \delta)}$$
}

Suppose that $y \in B_\rho(x, \epsilon)$. We follow that
$$\rho(y, x) < \epsilon$$
thus we follow that there's $\delta \in R$ such that $\rho(y, x) < \delta < \epsilon$.
We follow that $y \in U(x, \delta)$ and thus $y \in \bigcup_{\delta < \epsilon}{U(x, \delta)}$.
Therefore
$$B_\rho(x, \epsilon) \subseteq \bigcup_{\delta < \epsilon}{U(x, \delta)}$$

If $y \in \bigcup_{\delta < \epsilon}{U(x, \delta)}$, then there's $\delta < \epsilon$
such thaht $y \in U(x, \delta)$. We follow that $\rho(y, x) \leq \delta < \epsilon$ and thus
$y \in B_\rho(x, \epsilon)$. Thus
$$B_\rho(x, \epsilon) \supseteq \bigcup_{\delta < \epsilon}{U(x, \delta)}$$
Double inclusion gives us the desired result.

\subsection{}

\textit{Consider the map $h: R^\omega \to R^\omega$ defined in Exercise 8 of paragraph 19, give
  $R^\omega$ the uniform topology. Under what conditions on the numbers $a_i$ and $b_i$ is $h$
  continous? a homeomorphism?}

$h: R^\omega \to R^\omega$ is defined as
$$h((x_1, x_2, ... )) = (a_1 x_1 + b_1, a_2 x_2 + b_2, ...)$$
We can also use some notation from linear algebra (where we denote piecewise
vector multiplication by $\cdot$) to get
$$h((x_1, x_2, ... )) = ((x_1, x_2, ...) \cdot (a_1, a_2, ...)) + (b_1, b_2, ...) $$
and if none of $a_i$'s are zeroes, then
$$h\inv((x_1, x_2, ...)) = ((x_1, x_2, ... ) - (b_1, b_2, ...))
\cdot (\frac{1}{a_1}, \frac{1}{a_2}, ...) = $$
$$ = ((x_1, x_2, ... ) \cdot (\frac{1}{a_1}, \frac{1}{a_2}, ...) - (b_1, b_2, ...)
\cdot (\frac{1}{a_1}, \frac{1}{a_2}, ...))=
( \frac{x_1}{a_1}, \frac{x_2}{a_2}, ...) - (\frac{b_1}{a_1}, \frac{b_2}{a_2}, ...)$$
although the notation is somewhat abusive, we only want to show the idea; the same
conclusion can be drawn with ol' reliable FOL and stuff like that.

The conjecture is that set of $a_i$'s need to be bounded.
Let's expand our idea a bit and let $a_i$ and $b_i$'s be arbitrary for now.
We follow that if $B(x, \epsilon)$ is a basis element and
$B(x, \epsilon) \cap h\inv(R^\omega) \neq 0$, then there exists $y \in R^\omega$ such that
$$h(y) \in B(x, \epsilon)$$
Suppose that there's a basis element around $y$ such that its image is contained in
$B(x, \epsilon)$. We follow that there's  $\gamma \in R$  such that $\gamma > 0$ and 
$$h(B(y, \gamma)) \subseteq B(x, \epsilon)$$
We follow that there's $\beta \in R$ such that $0 < \beta < \gamma$ and thus we follow that
$$y + (\beta, \beta, ...) \in B(y, \gamma)$$
as proven in previous exercise. We follow then that
$$h(y + (\beta, \beta, ...)) \subseteq B(x, \epsilon)$$
we thus follow that for all $i$'s we've got that
$$(y_i + \beta) a_i + b_i \in (x_i - \gamma, x_i + \gamma)$$
thus
$$x_i - \gamma < (y_i + \beta) a_i + b_i < x_i + \gamma$$
$$x_i - \gamma - b_i < (y_i + \beta) a_i  < x_i + \gamma - b_i$$
$$x_i - \gamma - b_i < y_i a_i + \beta a_i  < x_i + \gamma - b_i$$
$$x_i - \gamma - b_i - y_i a_i <  \beta a_i  < x_i + \gamma - b_i - y_i a_i$$
$$x_i - y_i a_i - b_i - \gamma   <  \beta a_i  < x_i - b_i - y_i a_i + \gamma $$
for all $i$'s.
Let
$$z = x - ay - b$$
i.e.
$$(z_1, z_2, ...) = (x_1, x_2) - (a_1, a_2, ...) \cdot (y_1, y_2, ...) - (b_1, b_2, ...)$$
we follow then that
thus we follow that
$$\beta a \in U(z, \gamma)$$
and by using pretty much the same logic we can follow that 
$$(-\beta) a \in U(z, \gamma)$$
thus for all $i$'s we've got that
$$
\begin{cases}
  z_i - \gamma < \beta a_i < z_i + \gamma \\
  z_i - \gamma < -\beta a_i < z_i + \gamma
\end{cases}
$$
let's do some algebra on the second inequality
$$- z_i + \gamma > \beta a_i > - z_i - \gamma$$
$$- z_i - \gamma < \beta a_i < - z_i + \gamma$$
thus we get
$$
\begin{cases}
  z_i - \gamma < \beta a_i < z_i + \gamma \\
  - z_i - \gamma < \beta a_i < - z_i + \gamma
\end{cases}
$$
thus we follow that we can sum up two inequalities to get
$$z_i - \gamma - z_i - \gamma < 2 \beta a_i < z_i + \gamma - z_i  + \gamma$$
$$ - 2\gamma < 2 \beta a_i <  2 \gamma $$
$$ - \gamma < \beta a_i < \gamma $$
$$ - \gamma / \beta <  a_i < \gamma / \beta $$
thus we follow that
$$|a_i| < \gamma / \beta$$
which means that $a_i$'s ought to be bounded, otherwise we don't have a basis element
around $y$ whose image is contained in the original basis element.

Now suppose that $a_i$'s are bounded by $M$. If $M$ is zero, then we follow that
$a_i$'s are ceirtainly are bounded by some nonzero $M$, thus let us follow that in those
circumstances $M$ is a positive real number. Let $B(x, \epsilon)$ be a basis element.
Now we follow that there's two cases, namely
$$B(x, \epsilon) \cap \ran(h) = \emptyset $$
or
$$B(x, \epsilon) \cap \ran(h) \neq \emptyset $$
if the former case is true, then $h\inv[B(x, \epsilon)] = \emptyset$, which is an open set
and we're done. Thus assume the latter case. Let $y \in B(x, \epsilon) \cap \ran(h)$.
We want to show now that there's a basis element around $y$ that is contained in
$y \in B(x, \epsilon) \cap \ran(h)$, which would imply that $h\inv[B(x, \epsilon)]$ is an open
set. Previous exercise implies that 
$$B_\rho(x, \epsilon) = \bigcup_{\delta < \epsilon}{U(x, \delta)}$$
which means that there's $\delta \in R$ such that $0 < \delta < x$ and
$$h(y) \in U(x, \delta)$$
we also follow that since $\delta, x \in R$ and $\delta \neq x$ there's $\gamma \in R$ such that
$$\delta < \gamma < x$$
and thus
$$U(x, \delta) \subseteq U(x, \gamma)$$
therefore
$$h(y) \in U(x, \gamma)$$
and
$$h(y) \in U(x, \delta)$$
Therefore
$$h(y) \in \prod[(x_i - \gamma, x_i + \gamma)]$$
$$h(y) \in \prod[(x_i - \delta, x_i + \delta)]$$
we follow that for all $i$'s we've got that
$$x_i - \delta < a_i y_i + b_i < x_i + \delta$$
$$- \delta < a_i y_i + b_i - x_i <  \delta$$
$$|a_i y_i + b_i - x_i| <  \delta$$

Let $z \in U(y, \frac{\gamma - \delta}{M})$. We follow that for all $i$'s we've got
$$|y_i - z_i| < \frac{\gamma - \delta}{M}$$
$$M|y_i - z_i| < \gamma - \delta$$
given that $|a_i| < M$ for all $a_i$'s, we follow that 
$$|a_i||y_i - z_i| < \gamma - \delta$$
$$|a_iy_i - a_iz_i| < \gamma - \delta$$
$$|a_iy_i - a_iz_i| + \delta < \gamma$$
using earlier proven inequality $|a_i y_i + b_i - x_i| <  \delta$ we get
$$|a_iy_i - a_iz_i| + |a_i y_i + b_i - x_i| < \gamma$$
now we need to use some absolute value properties to get 
$$|a_iy_i - a_iz_i| + |- a_i y_i - b_i + x_i| < \gamma$$
let us use the magestic triangle ($|a + b| \leq |a| + |b|$) now
$$|a_iy_i - a_iz_i - a_i y_i - b_i + x_i| < \gamma$$
$$| - a_iz_i  - b_i + x_i| < \gamma$$
$$|a_iz_i  + b_i - x_i| < \gamma$$
$$-\gamma < a_iz_i  + b_i - x_i < \gamma$$
$$x_i - \gamma < a_iz_i  + b_i  < x_i + \gamma$$
thus
$$h(z) \in U(x, \gamma)$$
and therefore
$$h(z) \in B(x, \epsilon)$$
therefore we conclude that if $z \in U(y, \frac{\gamma - \delta}{M})$ then
$h(z) \in B(x, \epsilon)$. Given that
$$B(y, \frac{\gamma - \delta}{M}) \subseteq U(y, \frac{\gamma - \delta}{M})$$
(proof of that is pretty straightforward, so I'll omit it)
we follow that
$$z \in B(y, \frac{\gamma - \delta}{M}) \ra h(z) \in B(x, \epsilon)$$
thus we conclude that around every point of $h\inv[B(x, \epsilon)]$ there's a basis element
that is contained in $h\inv[B(x, \epsilon)]$. Thus we conclude that
for all $B(x, \epsilon)$ we've got that $h\inv[B(x, \epsilon)]$ is open and thus
$h$ is continous, as desired.

Now to the question of homeomorphism. We can follow pretty easily that if some  $a_i$'s are
equal to zero, then $h$ is not bijective, and if $a_i$'s are all nonzero, then the function
is bijective (if the proof of that is not included in my linear algebra course, then it's
pretty straightforward anyways). Thus we conclude that all $a_i$'s ought to be nonzero
in order to function to be homeomorphism. If none of the $a_i$'s are zero, then we can follow that
$$h\inv((x_1, x_2, ...)) = 
( x_1 \frac{1}{a_1}, x_2 \frac{1}{a_2}, ...) + (- \frac{b_1}{a_1}, - \frac{b_2}{a_2}, ...)$$
as shown in the beginning of this solution. Thus we follow that $h\inv$ is continous if
and only if $\frac{1}{a_i}$'s are bounded by some  number. This statement is equivalent
to saying that there's $\gamma \in R$ such that $\gamma < |a_i|$ for all $a_i$'s.
This also takes care of the nonzero clause.

Therefore we conclude that $h$ is a homeomorphism if and only if there
exist $M, \gamma \in R$ such that
$$\gamma < |a_i| < M$$
for all $a_i$'s.

\subsection{}

\textit{Let $X$ be the subset of $R^\omega$ consisting of all sequances $x$ such that $\sum{x^2}$
  converges. Then the formula
  $$d(x, y) = \left[\sum_{i = 1}^{\infty}{(x_i - y_i)^2}\right]^{1/2}$$
  defines a metric on $X$. On $X$ we have the three topologies it inherits from the box,
  uniform and product topologies on $R^\omega$. We have also the topology given by the metric $d$,
  which we call the $l^2$ topology}

\textit{(a)) Show that on $X$, we have the inclusions
$$ \text{uniform topology} \subseteq l^2\text{-topology} \subseteq \text{box topology}$$}

Suppose that $B_\rho(x, \epsilon)$ is a basis element of uniform topology such that
$B_\rho(x, \epsilon) \cap X$ is not empty. We follow that $B_\rho(x, \epsilon) \cap X$ is a
basis element of inherited topology on $X$ from uniform topology. Let
$y \in B_\rho(x, \epsilon) \cap X$. We follow that
$$\rho(x, y) < \epsilon$$
thus
$$\sup\set{\overline{d}(x_i, y_i)} < \epsilon$$
Now let $B_d(x, \epsilon/2)$ be a basis element in $l^2$-topology. We follow that for all
$z \in B_d(x, \epsilon/2)$ we've got that
$$\left[\sum_{i = 1}^{\infty}{(x_i - z_i)^2}\right]^{1/2} < \epsilon/2$$
$$\sum_{i = 1}^{\infty}{(x_i - z_i)^2} < (\epsilon/2)^2$$
we thus follow that for all $x_i$ and $z_i$ we've got
$$(x_i - z_i)^2 < (\epsilon/2)^2$$
thus
$$|x_i - z_i| < \epsilon/2$$
therefore $z \in U(x, \epsilon/2)$. Therefore $z \in B_\rho(x, \epsilon)$. Thus we conclude that
topology inherited from uniform topology on $R^\omega$ is coarser than $l^2$-topology on $X$,
which gives us the first inclusion.

Now let $B_l(x, \epsilon)$ be a basis element in $l^2$ topology.
We follow that we can make a basis element in box topology
$$U = \prod{(x_i - \epsilon * 2^{-i}, x_i + \epsilon * 2^{-i})}$$
we then follow that if $y \in U \cap X$, then
$$d(x, y) < \epsilon$$
thus we conclude that $l^2$-topology is coarser than the box topology, as desired.

\textit{(b) The set $R^\infty$ of all sequences that are eventually zero is contained in $X$.
  Show that the four topologies that $R^\infty$ inherits as a subspace of $X$ are all distinct.}

Let
$$U = \prod{(-1/n, 1/n)}$$
be a basis element in the box topology. We follow that $U \cap R^\infty \neq \emptyset$,
by the fact that $0$ vector is in both of those elements. Thus we conclude that
$U \cap R^\infty$ is a nonempty open set in inherited topology with respect to the box
topology.

Let $B_d(x, \epsilon)$ be a basis element around $0$ with respect to the given metric
(the one, that defines $l^2$ topology). Since $\epsilon > 0$ we can follow that
$\epsilon / 2 > 0$ and thus there's $n \in N$ such that $1/n < \epsilon/2$. We follow that
element
$$f_x =
\begin{cases}
  x = n \ra \epsilon/2  \\
  0 \text{ otherwise }
\end{cases}
$$
is an element in $R^\infty$, and we can also conlcude that $f \in B_d(x, \epsilon)$
by the fact that
$$\sqrt{\sum{(x_i - f_i)^2}} = \sqrt{(\epsilon/2)^2} = \epsilon/2$$
thus $d(x, f) < \epsilon$. We can also follow that $f \notin U$, because
we've specifically engineered it not to be in it. Given that $\epsilon$ is
arbitrary, we can conclude that
there's an $x$ such that there's no basis neighborhood in $l^2$-topology that is
a subset of $U \cap R^\infty$ and thus we conclude that $U \cap R^\infty$ is not open
in $l^2$-topology.

We can define
$$V = \prod{(0, 1/n)}$$
to be an open set in $l^2$-topology, whichh is

\textit{TODO: later}

\subsection{}

This one is handled in the linear algebra course.

\section{The Metric Topology (continued)}

\subsection{}

\textit{Let $A \subseteq X$. If $d$ is a metric for the topology of $X$, show that $d|A \times A$
  is a metric for the subspace topology on $A$.}

We follow that if some metric property does not hold for a pair of points in $A$, then
it doesn't hold in $X$ as well, which produces contradiction, and thus $d|A \times A$ is
a metric on set $A$.

Let $U$ be an open set in $A$. We follow that there's an open set $V \subset X$
such that $U = V \cap A$. Since $V$ is an open set, it is a union of balls, and thus
$U$ is union of open balls in $A$. Thus we conclude that $d|A \times A$ imposes the same topology,
as desired.

\subsection{}

\textit{Let $X$ and $Y$ be metric spaces with metrics $d_X$ and $d_Y$ respectively. Let
  $f: X \to Y$ have the property that for every pair of points $x_1, x_2$ of $X$
  $$d_Y(f(x_1), f(x_2)) = d_X(x_1, x_2)$$
  Show that $f$ is an imbedding. It is called isometric imbedding of $X$ in $Y$.}

Suppose thta $x_1, x_2 \in X$ and $x_1 \neq x_2$. We follow that
$$d_X(x_1, x_2) \neq 0$$
and thus
$$d_Y(f(x_1), f(x_2)) \neq 0$$
thus $f(x_1) \neq f(x_2)$. Thus we concldue that $f$ is injective.

Let $x \in X$ and $\epsilon \in R$ be arbitrary. Then we follow that if we set $\delta = \epsilon$
and get a point $x' \in X$ such that
$$d_X(x, x') < \delta$$
then
$$d_X(x, x') < \epsilon$$
and since $d_X(x, x') = d_Y(f(x), f(x'))$ we follow that
$$d_Y(f(x), f(x')) < \epsilon$$
thus we conclude that $f$ is continous. By the same logic we can prove that $f\inv: \ran(f) \to X$
is continous as well.

Thus we follow that $f$ is an imbedding by definition, as desired.

\subsection{}

\textit{Let $X_n$ be a metric space with metric $d_n$ for $n \in Z_+$.}

\textit{(a) Show that
  $$\rho(x, y) = \max\set{d_1(x_1, y_1), ..., d_n(x_n, y_n)}$$
  is a metric for the product space $X_1 \times ... \times X_n$
}

We follow that $\rho(x, y) \geq 0$ for all $x \in R$ by the virtue of the fact that all of the
numbers under the max are non-negative. If $x = y$, then $d_j(x, y) = 0$ for all $j$'s, and thus
$\rho(x, y) = \max\set{0} = 0$. If $x \neq y$, then there is a component $j$ such that
$d_j(x, y) \neq 0$, and thus we've got the first property of the metric.

We can follow commutativity of this thing by common sense.

Suppose that $x, y, z \in \prod{X_i}$. We follow that for all $j$'s we've got 
$$d_j(x_j, z_j) \leq d_j(x_j, y_j) + d_j(y_j, z_j)$$
Then we can follow the triangle by either induction or contradiction.

Now suppose that $U$ is a basis element of $\prod{X_i}$ and $x \in U$.
We follow that there are open sets $V_j$
such that $U = \prod{V_j}$. Given that any of the $V_j$'s are open, we follow that for each one
of them there's a basis element $B_j$ such that $B_j$ is centered on $x_j$ and $B_j \subseteq V_j$
and we follow that $\prod(B_j) \subseteq \prod V_j$. Each $B_j$ has  $\epsilon_j$
such that $B_j = B_j(x_j, \epsilon_j)$ and thus we can follow that if
$\rho(x, y) < \max\set{\epsilon_j}$, then $y \in \prod(B_j)$. Thus we follow that
$$B_\rho(x, \max\set{\epsilon_j}) \subseteq U$$, therefore product topology is a subset of
topology, that is induced by this metric. We can also follow that every basis element of
topology, that is induced by this metric is a basis element in product topology, and thus
we conclude that given metric induces the product topology, as desired.

\textit{(b) Let $\overline{d_i} = \min\set{d_i, 1}$. Show that
  $$D(x, y) = \sup{\overline{d_i}(x_i, y_i) / i}$$
  is a metric for the product space $\prod{X_i}$.}

This proof is practically the same as in theorem 20.5. If it ain't, then I skip it anyways

\subsection{}

\textit{Show that $R_l$ and the ordered square satisfy the first contability axiom}

Let $x \in R$. We follow that $[x, x + 1/n)$ is the countable collection, that satisfies the
countability axiom.

For the ordered square we've got either $(\eangle{x_1, x_2 - 1/n}, \eangle{x_1, x_2 +  1/n})$
or some simular stuff for the edges.

\subsection*{the rest}

\textit{the rest of the exercises were taken care of in the real analysis course. Maybe later
  I'll repeat the proofs for s'n'g}

\section{The Quotient Topology}

\chapter{Connectedness and Compactness}

\section{Connected Spaces}

\subsection{}

\textit{Let $\topol$ and $\topol'$ be two topologies on $X$. If $\topol \subseteq \topol'$, what
  does connectedness of $X$ in one topology imply about connectedness in the other?}

We follow that if $\topol \subseteq \topol'$ and $U$ is a connected subspace of $\topol'$,
then there's no separations in $\topol'$, and thus there're no separations in $\topol$ as well.

If $U$ is open in $\topol$, then and space $X$ has some sane amout of elements (i.e. not zero or
one), then we can't follow nothing from it. For example, every topology is a subset
of a discrete topology on a set, and every subspace of discrete topology is disconnected.

\subsection{}

\textit{Let $\set{A_n}$ be a sequence of connected subspaces of $X$, such that
  $A_n \cap A_{n + 1} \neq \emptyset$ for all $n$. Show that $\bigcup{A_n}$ is connected.}

We firstly assume that $|\set{A_n}| \leq \omega$ because of the word "sequence"

Assume that $\bigcup{A_n}$ is disconnected and has a separation $U$ and $V$.
We thus follow that for all $\set{A_n}$ we've got that $A_n$ is in either $U$ or $V$.
Assume that $A_1$ is in $U$. Thus we follow that there's a minimal $j$ such
that $A_j \subseteq V$ (if there isn't one, then $V$ is empty, which is a contradiction).
Therefore $A_{j - 1} \subseteq U$ and $A_j \subseteq V$, which means that
$A_{j - 1} \cap A_{j} = \emptyset$, which is a contradiction.

\subsection{}

\textit{Let $A_\alpha$ be a collection of connected subspaces of $X$; let $A$
  be a connected subspace of $X$. Show that if $A \cap A_\alpha \neq \emptyset$ for all $\alpha$,
  then $A \cup (\bigcup{A_\alpha})$ is connected.}

Unlike the previous exercise, we can't reasonably assume nothing about the collection $A_\alpha$.
We firstly state that
$$A \cup (\bigcup{A_\alpha}) = \bigcup{(A \cup A_\alpha)}$$
which is either handled in the set theory course, or is justified by some trivial FOL.

Since $A \cap A_\alpha \neq \emptyset$ for all $\alpha$, we follow that $A \cap A_\alpha$
is connected for all $\alpha$, thus 23.3 implies that $\bigcup{(A \cup A_\alpha)}$ is connected as
well.

\subsection{}

\textit{Show that if $X$ is an infnite set, it is connected in the finite complement topology}

Suppose that it ain't. We follow that sets $U$ and $V$ form separation. Given that
$U$ and $V$ are open, we follow that 
$X \setminus U = V$ is finite and thus $V$ is not open, which is a contradiction.

\subsection{}

\textit{A space is totally disconnected if its only connected subspaces are one point sets.
  Show that if $X$ has the discrete topology, then $X$ is totally disconnected. Does the
  converse hold?}

I'm pretty sure that we don't need to prove that one point sets are connected. If $U$ is a
subspace with more than one point in it, then we follow that there's $u \in U$ and thus
$U \setminus \set{u}$ and $\set{u}$ make a separation of $U$.

We've seen already that $Q$ in $R$ is totally disconnected (if there're 2 or more points
in $U \subseteq Q$, then we can take $i \in (\inf(U), \sup(U)) \cap I$, where $I$ is
irrational, and then $(-\infty, i) \cap U$ and $(i, \infty) \cap U$ form the desired separation)
and it just inherits the standart topology, so the converse doesn't hold.

\subsection{}

\textit{Let $A \subseteq X$. Show that if $C$ is a connected subspace of $X$ that intersects
  both $A$ and $X \setminus A$, then $C$ intersects $Bd(A)$}

We firsly state here that
$$Bd(A) = \overline{A} \setminus Int(A)$$
and
$$Bd(A) = \overline{A} \cap \overline{X \setminus A}$$

Suppose that $C$ does not intersect $Bd(A)$. We firstly want to follow that
$X \setminus \overline A$ and $\overline{A}$ make a partition of set $X$, and since
$Bd(A)$ and $Int(A)$ are partitions of $\overline{A}$, we follow that
$X \setminus \overline A$, $Bd(A)$ and $Int(A)$ form a partition of $X$.
We also note that $X \setminus \overline A$ and $Int(A)$ are open sets. 
Since $C$ does not intersect $Bd(A)$, we follow that there're sets
$X \setminus \overline{A} \cap C$ and $Int(A) \cap C$ that form a separation of $C$, which
means that $C$ is not connected, which is a contradiction.

\subsection{}

\textit{Is the space $R_l$ connected? Justify your answer.}

We follow that
$$(-\infty, 0), [0, \infty)$$
are both open in $R_l$ since
$$(-\infty, 0) = \bigcup_{n \in Z_+}{[-n, -n + 1) }$$
$$[0, \infty) = \bigcup_{n \in Z_+}{[n - 1, n) }$$
and
$$(-\infty, 0) \cup [0, \infty) = R$$
thus we follow that those two sets constitute a separation.

\subsection{}

\textit{Determine whether or not $R^\omega$ is connected in the uniform topology}

We can follow that $R^\omega$ can partitioned in the same manner as the box topology.
We can follow that $B(x, \epsilon) \subseteq U(x, \epsilon)$, and then proceed with the same
logic as in the example 6 to show that bounded and unbounded sequences form a separation on
$R^\omega$.

\subsection{}

\textit{Let $A$ be a proper subset of $X$, and let $B$ be a  proper subset of  $Y$. If $X$
  and $Y$ are connected, show that
  $$(X \times Y) \setminus (A \times B)$$
  is connected.}

The idea here is to plant a cross somewhere in the desired set, and then add
strands to this set, untill we prove that the desired set is connected.

Let  $z \notin A$ and $q \notin B$ (those points exist since both $A$ and $B$
are proper subsets).
Thus $\eangle{z, q} \in (X \times Y) \setminus (A \times B)$.
Now we can follow that set
$$\eangle{z, q} \in \set{z} \times Y \cup X \times \set{q}$$
is a subset of
$$(X \times Y) \setminus (A \times B)$$
and the set $\set{z} \times Y \cup X \times \set{q}$ is connected.
Now let us use the identity
$$(X \times Y) \setminus (A \times B) =  [X \times (Y \setminus B)] \cup
[(X \setminus A) \times Y]$$
we now follow that for all $j \in (Y \setminus B)$ it is true that
every set $X \times \set{j}$ is connected and also that 
$$X \times \set{j} \cap \set{z} \times Y \cup X \times \set{q}$$
is nonempty. Therefore we conclude that
$$[X \times (Y \setminus B)] \cup \set{z} \times Y \cup X \times \set{q}$$
is connected. By the same logic we follow that 
$$[(X \setminus A) \times Y] \cup \set{z} \times Y \cup X \times \set{q}$$
is connected. Since both of those sets have a point $\eangle{z, q}$ in common, we conclude that 
the desired set is connected as well, as desired.

\subsection{}

\textit{Let $\set{X_\alpha}_{\alpha \in J}$ be an indexed family of connected spaces; let $X$ be a
  prodcut space
  $$X = \prod_{\alpha \in J}{X_\alpha}$$
  Let $a = (a_\alpha)$ be a fiexd point of $X$.}

Important point: in this exercise we assume the product topology on $X$.

\textit{(a) Given any finite subset $K$ of $J$, let $X_K$ denote the subspace of $X$ consisting
  of all points $x = (x_\alpha)$ such that $x_\alpha = a_\alpha$ for $\alpha \notin K$. Show that
  $X_K$ is connected.}

We can prove the desired result by stating that $\prod_{j \in K}{X_j}$ is homeomorphic
to $X_K$, and since the former is a cartesian product of finite amount of connected spaces
we follow that both of those spaces are connected.

\textit{(b) Show that the union $Y$ of the spaces $X_K$ is connected}

All of $X_k$'s have point $a$ in common, and thus their union is connected.

\textit{(c) Show that $X$ equals the closure of $Y$; conclude that $X$ is connected.}

Suppose that $b \in X$. Let $U$ be a basis neighborhood of $b$ (in the product topology of course).
Since $U$ is a basis element, we follow that there's a finite set $K$ and function
$f: J \to \pow(\bigcup{X_j})$ such that
$$f(j) =
\begin{cases}
  j \notin K \ra X_k \\
  j \in K \ra V_j
\end{cases}
$$
and 
$$U = \prod_{j \in J}{f(j)}$$
since $K$ is fintite, we follow that $U$ intesects $X_K$ for that particular $K$, and
since $Y$ is a union of $X_K$'s, we follow that $U$ intersects $Y$. Since $U$
is arbitrary, we follow that $b$ is in a closure of $Y$. Therefore
$$X \subseteq \overline{Y}$$
We can also state that $\overline{Y} \subseteq X$ since we use the product topology on $X$,
and thus we conclude that $\overline{Y} = X$ by double inclusion, as desired.

\textit{Last two exercises are left for the time, when I'm finished with quotient products.}

\section{Connected Subspaces of the Real Line}

\subsection{}

\textit{(a) Show that no two spaces $(0, 1)$, $(0, 1]$, and $[0, 1]$ are homeomorphic.}

Suppose that $h: (0, 1] \to (0, 1)$ is a homeomorphism. Since both of those sets have the same
cardinality, we follow that there's a bijection betweeen the two, and therefore we won't dispute
that.

Let $u = h(1)$. We follow that $u \in (0, 1)$ and thus $(0, 1) \setminus \set{h(1)}$
is a disconnected set. Thus we follow that
$$h[(0, 1)] = h[(0, 1]] \setminus h(1) = (0, 1) \setminus \set{h(1)}$$
is a disconnected set, which contradicts the fact that $h$ is a homeomorphism.

By pretty much the same logic, but applied to two points we can show that the
resulting two pairs aren't homeomorphic.

\textit{(b) Suppose taht there exists imbeddings $f: X \to Y$ and $g: Y \to X$. Show
  by means of an example that $X$ and $Y$ need not be homeomorphic.}

We've shown earlier that any two  closed intervals  are homeomorphic, and also that any
two open intervals are homeomorphic as well. Thus we can follow that we can create imbeddings
out of homeomorphisms
$$f: [0, 1] \to [1/3, 2/3]$$
and
$$g: (0, 1) \to (1/3, 2/3)$$
by expanding their respective codomains. Thus we've got the desired functions, and previous
point shows that given sets aren't homeomorphic, as desired.

\textit{(c) Show that $R^n$ and $R$ aren't homeomorphic if $n > 1$.}

Since $R^n$ is Hausdorff, we follow that $\set{0}$ is open and thus punctured eucledian
space $R^n \setminus \set{0}$ is connected, which means that if there's a bijection $h: R^n \to R$,
then the set $h[R^n \setminus \set{0}]$ is always disconnected, which means that no bijection
between those two sets can be continous.

\subsection{}

\textit{Let $f: S^1 \to R$ be a continous map. Show there exists a poit $x$ of $S^1$ such that
  $f(x) = f(-x)$.}

Firstly, we want to assume that $f(-x)$ means that we apply $f$ to the $x$'s additive
inverse in vector notation, because no other meaning makes sense.
Since scalar multiplication of vectors is continous, we follow that
$$f(-x) = (f \circ T)(x)$$
where $T(x) = -x$ is continous as well. Thus we follow that we can define $g: S \to R$ by
$$g(x) = f(x) - f(-x)$$
is a composition of continous operator $-$ and two continous functions, and thus $g$ itself
is continous.
We also follow that
$$g(-x) = f(-x) - f(-(-x)) = f(-x) - f(x) = -g(x)$$

Now let $s \in S$. If $g(s) = 0$, then
$$f(x) - f(-x) = 0 \lra f(x) = f(-x)$$
and we're done. If $g(s) > 0$, then we follow that
$$g(-s) = -g(s) < 0$$
therefore IVT implies together with the connectedness of $S$ imply that there's a point $s' \in S$
such that $g(s') = 0$. If $s < 0$, then we can set $s' = -s$ and default to the previous case.
Therefore we derive the desired result.

\subsection{}

\textit{Let $f: X \to X$ be continous. Show that if $X = [0, 1]$, there is a point $x$
  such that $f(x) = x$. The point $x$ is called a fixed point of $f$. What happens if $X$
  equals $[0, 1)$ or $(0, 1)$?}

Set $g: X \to R$ by
$$g(x) = f(x) - x$$
since the indentity function, $f(x)$ and $-: R \times R \to R$ are all continous functions,
we follow that
$g$ is a continous function as well. We can also argue
that condition that $f(x) = x$ is equivalent to condition $g(x) = 0$; thus we
need to show that there's a $x \in X$ such that $g(x) = 0$.

We can follow that
$$g(0) = f(0) - 0 = f(0)$$
and thus $g(0) \in X$. If $g(0) = 0$, then we're done, thus assume that $g(0) \neq 0$.
We then conclude that $g(0) \in (0, 1]$.

Let us consider also $g(1)$. We follow that
$$g(1) = f(1) - 1 \in [-1, 0]$$
We follow that if $g(x) = 0$, then we're done, and thus $g(x) \in [-1, 0)$.

Now we can combine two arguments to follow that $g(0) > 0$ and $g(1) < 0$, which means that
there's $x \in [0, 1]$ such that $g(x) = 0$, as desired.

We can also follow that we can define function $h: R \to R$
$$h(x) = \frac{x + 1}{2}$$
we follow that we can restrict domain of $h$ to either $[0, 1)$ or $(0, 1)$ and in those
cases we'll have that
$$\ran(h|[0, 1)) \subseteq [0, 1)$$
$$\ran(h|(0, 1)) \subseteq (0, 1)$$
Some trivial analsis can show us that $h(x) \neq x$ for all $[0, 1)$. Thus we conclude thhat
if we restrict domain and codomain to presented values, then we can't say nothing about the
existence of the fixed point.


\subsection{}

\textit{Let $X$ be an ordered set in the order topology. Show that if $X$ is connected, then
  $X$ is a linear continuum.}

Suppose that $X$ is not a linear continuum. If the first condition does not hold,
then we follow that there are $x, y \in X$ such that $x < y$ and there's no $z \in X$
such that $x < z < y$. Thus we follow that $(-\infty, y)$ is an open set,
as well as $(x, \infty)$, and they constitute separation of $X$, which is a contradiction.

Suppose that $X$ does not have a least upper boundary property. Thus we follow that
there's a subset $U$ of $X$ such that $U$ is bounded above and does not have a least upper bound.
Define
$$V = \bigcup_{u \in U}{(-\infty, u)}$$
Since $U$ is bounded, we follow that $V \neq X$. Let $i \in X \setminus V$. We follow that $i$
is an upper bound, but because there's no least upper bound, we follow that there's $i' \in X$
such that $i' < i$ and $i'$ is also an upper bound for $V$. AC implies that we've got a
function on our hands, and thus we follow that
$$\bigcup_{i \in X \setminus V}{(f(i), \infty)}$$
is an open set, that is disjoint from $V$, and is equal to $X \setminus V$.
Since $U$ is nonempty, we follow that $V$ is also nonempty, and thus $V$ is both open
and closed, which implies that $X$ is disconnected, which is a contradiction.

Therefore we follow that if $X$ is connected, then it satisfies both conditions of linear
continuum, as desired.

\subsection{}

\textit{Consider the following sets in the dictionary order. Which are linear
  continua?}

\textit{(a) $Z_+ \times [0, 1)$}

$Z_+$ is a woset, and thus the next exercise will imply that the desired set is linear
continuum. The intuitive conclusion is that this thing is homeomorphic to $R$.

\textit{(b) $[0, 1) \times Z_+$}

We follow that there's no elements between $\eangle{0, 1}$ and $\eangle{0, 2}$
on the account that $Z_+$ does not hold neither of the preoperties of
linear continuum.

\textit{(c) $[0, 1) \times [0, 1]$}

We can follow that $[0, 1) \times [0, 1] = \set{i \in I^2: i < \eangle{1, 0}}$, thus we
follow that given set is convex, and thus by sub-theorem of 24.1 we conclude that it
is connected. Since it is a connected set in order topology, we follow that it is
a linear continuum.

\textit{(d) $[0, 1] \times [0, 1)$}

We follow that the subset
$$U = \set{\eangle{0, 1/n}: n \in Z_+}$$
does not have a supremum, and thus we conclude that given set cannot be a linear continuum.

\subsection{}

\textit{Show that if $X$ is a well-ordered set, then $X \times [0, 1)$ in the dictionary
  order is a linear continuum.}

Let $x, y \in X \times [0, 1)$ be such that $x < y$. We follow that
$$x = \eangle{a, b}$$
$$y = \eangle{c, d}$$
If $a = c$, then we follow that there's $q \in [0, 1)$ such that $b < q < d$, and thus
$$x < \eangle{a, q} < y$$
If $a < c$, then we follow that there are two cases: $c \neq a^+$ or $c = a^+$.
If former is the case, then we follow that
$$x < \eangle{a^+, 0} < y$$
if the latter is the case, then we follow that there's $q \in (b, 1)$ such that
$$x < \eangle{a, q} < y$$
thus we conclude that $x < y$ implies that there's $z \in X \times [0, 1)$ such that
$$x < z < y$$
which proves the second requirement of the linear continuum.

Now let $U$ be a bounded set of $X \times [0, 1)$. We can set
$$s = \sup\set{x \in X: \exists y \in [0, 1): \eangle{x, y} \in U}$$
i.e. the supremum of the first parts of elements of a given set.
Now define
$$V = \set{u \in U: \exists y \in [0, 1): \eangle{s, y} \in U}$$
i.e. set of elements of with the supremum in the first part.

Now we can define an element $j$ by cases on $V$. If $V = \emptyset$,
then we can set $j = \eangle{s, 0}$. Construction of $j$ implies that $j$ is
indeed an upper bound, and we follow that if $k < j$, then
there's $l \in X, q \in [0, 1)$ such that $k = \eangle{l, q}$ and thus $l < s$.
Now construction of $s$ implies that $k$ is not an upper bound. Thus we can conclude that
$j$ is indeed a least upper bound.

Let $y$ be equal to the supremum of the second parts of $V$. IF $y \neq 1$, then we set
$j = \eangle{s, y}$, and if $y = 1$, then we can set $j = \set{s^+, 0}$.
Some trivial logic shows that $j$ is indeed the least upper bound for $U$.

Thus we conclude that $X \times [0, 1)$ is indeed a linear continuum, as desired.

\subsection{}

\textit{(a) Let $X$ be ordered sets in the order topology. Show that if $f: X \to Y$
  is order preserving and surjective, then $f$ is a homeomorphism.}

We firstly note that $X$ is a toset (implied by the fact that $X$ is in the order topology,
which was defined so far exclusively on tosets), and thus it's got a trichotomy.
Thus we follow that if $x \neq y$, then we can assume that $x < y$, thus
$f(x) < f(y)$, and thus $f$ is injectieve. Now surjectivity of $f$ with respect to the
chosen codomain makes it a bijection.

Now let $a, b \in Y$. We follow that
$$\forall x \in X: x \in f\inv[(a, b)] \lra f\inv(a) < x < f\inv(b)$$
(all of this stuff can be derived pretty easily). Thus we follow that
if $U$ is a basis oin $Y$, then $f\inv[U]$ is open, thus making $f$ continous. Check that
$f \inv$ is continous is well is the same as this one. Thus we conclude that
$f$ is a homeomorphism, as desired.

\textit{(b)  Let $X = Y = \overline{R_+}$. Given a positive inveger $n$, show that the function
  $f(x) = x^n$ is order preserving and surjective. Conclude that its inverse, the $n$th root
  function is continous.}

We follow that if $n = 1$, then $f$ is an identity, thus it's order preserving and surjective.
Suppose that $f(x) = x^{n - 1}$ is order preserving and surjective. Let $z, y \in \overline{R_+}$
are such that $z < y$. We follow that $z, y, z^j, y^j$ are all nonnegative for all $j \in \omega$.
Thus we follow that 
$$z < y \iff f(z) < f(y) \iff z^{n - 1} < y^{n - 1} $$
now $z < y$ and $z^{n - 1} < y^{n - 1}$ and nonnegativity of those things imply that
$$z * z^{n - 1} < y * y^{n - 1}$$
$$z^{n} < y^{n}$$
given that $z$ and $y$ are arbitrary, we conclude that $f(x) = x^n$ is order preserving.
Thus we conclude that
$$f(x) = x^n$$
is order preserving for all $n \in \omega$ by induction.

Now by induction we can also prove that $f$ is continous and nonnegative. Thus we
can follow that IVT implies that $f$ is surjective. Therefore we conclude that
$f$ is order-preserving and surjective, which means that it's a homeomorphism.
Thus we conclude that $f\inv$ is indeed continous.

\textit{(c) blah blah blah}

Yes, it is, but not with respect to the subset topology, but with respect to the order topology
defined on $(-\infty, 1) \cup [0, \infty)$. goto page 90 of original book for
examples and explanation

\subsection{}

\textit{(a) Is a product of path-connected spaces necessarily path connected?}

It seems to be so. Let $X$ and $Y$ be path-connected spaces. Now let $q, w \in X \times Y$.
We follow that
$$q = \eangle{a, b}$$
$$w = \eangle{c, d}$$
We now follow that there exist continous functions $f: [x_1, x_2] \to X$ and $g: [y_1, y_2] \to Y$
such that 
$$f(x_1) = a, f(x_2) = c, g(y_1) = b, g(y_2) = d$$
We can follow that spaces $X$ and $X \times \set{b}$ with respect to the subset topology of a
product topology in $X \times Y$ are homeomorphic. Thus we follow that we can define
continous 
$f': [x_1, x_2] \to X \times Y$ by
$$f'(x) = \eangle{f(x), b}$$
and $g': [x_2, (y_2 - y_1) +  x_2] \to X \times Y$ by
$$g'(y) = \eangle{c, g(y - x_2 + y_1)}$$
We can follow that $f'(x_2) = \eangle{f(x_2), b} = \eangle{c, b} = g'(x_2)$. Thus we follow that
we can concatenate those two functions to get the desired path.

By induction we can follow that finite product of path-connected spaces is path connected.

\textit{(b) If $A \subseteq X$ and $A$ is path connected, is $\overline{A}$ necessarily
  path connected?}

Let $S$ be the topologist's sine curve. If $x, y\in S$, then we follow that obviously there
is a continous function such that everything connects, but $\overline{S}$ is not
path connected, and thus we've got the contradiction.

\textit{(c) If $f: X \to Y$ is continous and $X$ is path connected, is $f[X]$ necessarily
  path connected?}

Suppose that $y_1, y_2 \in f[X]$. We follow that there are $x_1, x_2 \in X$ (maybe not unique,
but still) such that $f(x_1) = y_1, f(x_2) = y_2$. Since $X$ is path connected, we follow that
there's a continous $g: [a, b] \to X$ such that $g(a) = x_1$ and $g(b) = x_2$.
Since $f$ and $g$ are both continous, we follow that
$f \circ g$ is also continous, and thus we've got the desired path.

\textit{(d) If $\bigcup{A_\alpha}$ is a collection of path-connected subspaces of $X$
  and if $\bigcap{A_\alpha} \neq \emptyset$, is $\bigcup{A_\alpha}$ necessarily
  path-connected?}

Suppose that $x, y \in \bigcup{A_\alpha}$. We follow that there's
$q \in \bigcap{A_\alpha}$. Thus we follow that there's a path between $x$ and $q$ and
a path between $q$ and $y$, which means that there's a concatenated path between $x$ and $y$.
(Although it feels like a somewhat liberal proof, we can follow all the rigorous stuff
pretty easily)

\subsection{}

\textit{Assume that $R$ is uncountable. Show that if $A$ is a countable subset of $R^2$,
  then $R^2 \setminus A$ is path connected.}

Let $x, y \in R^2$. 
We follow that there are uncountable amount of lines that pass through each one of those points.
We can also follow that there's no point $a \in A$ such that there are two lines from
a single point, that intersect $a$. Thus we follow that
the set of lines from $x$ that does not intersect $A$ is uncountable. Same goes for $y$.
We can follow now that there are uncountably many unparrallel pairs of lines that go through
the $x$ and $y$, which gives us the desired path.

\subsection{}

\textit{Show that if $U$ is an open connected subspace of $R^2$, then $U$ is path connected.}

Let $x_0 \in U$ and let $A$ denote the set of points in $U$, from which there's a path to $x_0$.
We follow that $A$ is a subset of $U$, and we also follow that it's nonempty, since
there's always a path from $x_0$ to $x_0$.

Let $y \in A$. Since $y \in U$ and $U$ is open, we follow that there's a
basis element $B$ inside $U$
such that $y \in B$. Since $y$ is quite literally the center of the basis element $B$,
we follow that there's a path to any given element of $B$ from $y$. Thus we follow that
$B \subseteq A$, which allows us to conclude that $A$ is open.

Suppose that $l \in U$ is a limit point of $A$. We follow that there's a basis neighborhood
around $l$ that intersets $A$ and lies fully within $U$. Thus we conclude that
once again we can connect $l$ to some point inside $A$, and thus we conclude that $l \in A$.
Therefore $A$ contains all of its limit points, that are in $U$, and thus we conclude that
$A$ is both open and closed inside subset topology of $U$. THus we conclude that
$A = \emptyset$ or $A = U$, where former is impossible, as we've shown before. Thus we
conclude that $U$ is path connected, as desired.

\subsection{}

\textit{If $A$ is a connected subspace of $X$, does it follow that $Int(A)$ and $Bd(A)$ are
  connected?}

We can define a set
$$\set{\eangle{x, y} \in R^2: 0 \leq y \leq x || 0 \geq y \geq x}$$
(i.e. two triangular regions, that meet at the origin).
THen we follow that the origin is not in the interior of $A$, since no basis neighborhood
of $A$ is contained within $A$. But $A$ in general is connected, which is trivially
proven. Thus we conclude that $A$ being connected does not imply anything about connectedness
of its interior.

We can also punch two holes through $R^2$ to get a boundary that will consist of two points in
$R^2$, which will obviously be disconnected.  Thus we conclude that
connectedness of $A$ does not imply nothing about connectedness of $Int(A)$ or $Bd(A)$.

Suppose that both $Int(A)$ and $Bd(A)$ are connected. We follow that $Int(A)$ or $Bd(A)$ can
be both open and closed, which would imply that $A$ is disconnected. It can be connected as well.
So we can't conclude much from it.

\textit{The last exercise is left for later}

\section{Compact Spaces}

\subsection{}

\textit{(a) Let $\topol$ and $\topol'$ be two topologies on the set $X$; suppose that
  $\topol \subseteq \topol'$. What does compactness of $X$ under one of these topologies
  imply about the other?}

We follow that if $X$ is compact under $\topol'$, and $U$ is a collection of
open sets from $\topol$, that form open cover for $X$, we can conclude that $U$
is an open cover in $\topol'$ as well, thus there's a finite subcover, and thus we
follow that $X$ is compact in $\topol'$ as well. If $X$ is compact only in $\topol$, then we
can't conclude nothing. Example: standart and discrete topologies on $R$.

\textit{(b) Show that if $X$ is compact Hausdorff under both $\topol$ and $\topol'$, then either
  $\topol$ or $\topol'$ are equal or they are not compatible.}

We essentially want to prove that $\topol$ cannot be a proper subset of $\topol'$ and
vice versa. We follow that they can be equal (duh), so assume that $\topol \subset \topol'$.

We know that $X$ is a compact space under $\topol'$. Let $U$ be an open set of $\topol'$
such that $U \notin \topol$. We can state now that $U \neq \emptyset$, since $U \notin \topol$
and $\emptyset \in \topol$ by definition.

We can follow that $X \setminus U$ is a closed set. Since $U \subseteq X$, we follow that
$X \setminus U \subseteq X$. Thus $X \setminus U$ is a closed subspace of $X$, and since $X$
is compact we follow that $X \setminus U$ is compact. Since $U \notin \topol$, we follow that
$X \setminus U$ is not closed in $X$ with respect to $\topol$, and thus we follow that
there's an open cover $C$ for $X \setminus U$ in $\topol$, that does not have a finite
subcover for $X \setminus U$. Since $C \subseteq \topol$, we follow that $C \subseteq \topol'$,
which means that $C$ is an open cover for $X \setminus U$ for which there's no finite
subcover, which implies that $X \setminus U$ is not compact, which is a contradiction.

\subsection{}

\textit{(a) Show that in the finite complement topology on $R$, every subspace is compact.}

Let $U$ be a subset of $R$ and let $C$ be an open cover for $U$.

If $U = \emptyset$, then we folllow that any set in $C$ will constitute an finite subcover.

Suppose that  $U \neq \emptyset$. Let $C_1 \in C$. We follow that
$U \setminus C_1 \subseteq X \setminus C_1$, and thus it's a finite subset. Thus we follow that
we can create a finite collection $V_u$ such that for each $V \in V_u$ there's $u \in U$
such that $u \in V$. Thus we follow that $V_u \cup \set{C_1}$ will be a finite subcover of $U$,
thus making $U$ compact.

Since subcovers and sets were chosen
arbitrarely, we conclude that we've got the desired result. We also can conclude that
since we haven't used the properties of $R$, given statement will follow for any set.

\textit{(b) If $R$ has the topology consisting of all sets $A$ such that $R \setminus A$
  is either countable of all of $R$, is $[0, 1]$ a compact subspace?}

We can define a collection
$$C_n = R \setminus \set{\frac{1}{m}: m \in Z_+ \land m < n}$$
i.e.
$$C_1 = R \setminus \set{1/2, 1/3, 1/4, ...}$$
$$C_2 = R \setminus \set{1/3, 1/4, 1/5, ...}$$
and so on. Then we follow that although $\bigcup{C} = R$, we follow that there's no finite subcover
for $[0, 1]$.

\subsection{}

\textit{Show that a finite union of compact subspaces of $X$ is compact.}

Let $C_n$ be a finite collection of compact subspaces of $X$ and let $V_n$ be a cover of
$\bigcup{C_n}$. We follow that since $V_n$ covers $\bigcup{C_n}$, it also covers
every individual $C_n$. Thus we follow that there's a finite subcover $U_n \subseteq V_n$
for all $C_n$. Therefoore $\bigcup{C_n}$ is covered by $\bigcup{U_n}$, and given that
each individual $U_n$ is finite we conclude that set of subspaces that cover the finite
union is also finite.

\subsection{}

\textit{Show that every compact subspace of a metric space is bounded in that metric and is closed.
  Find a metric space in which not every closed bounded subspace is compact.}

Since every metric space is Hausdorff, we follow that every compact subspace of a metric space
is closed. Now suppose that $X$ is not bounded. Let $\epsilon > 0$. Since $X$ is not
bounded, we follow that it's nonempty and thus there's $x_1 \in X$.
Let $U_1 = B(x_1, \epsilon)$. Since $X$ is unbounded, we follow that $B(x, \epsilon)$
does not cover $X$, and thus there's $x_2 \in X$. We then can define $B(x_2, \epsilon)$
and in the same fashion inductively we can define $B_n$. We then follow that
$B_n$ is an infinite open cover of $X$, and given the cover's construction we conclude that
there's no finite subcover that covers the whole space. Thus we conclude that if
a subspace is compact, then it's closed and bounded.

We follow that in the standart bounded metric on $R$ we've got that $R$ is closed, and it's
bounded by $M = 1.1$. Since standart bounded metric induces standart topology, we conclude that
$R$ is not compact.

\subsection{}

\textit{Let $A$ and $B$ be disjoint compact subspaces of the Hausdorff space $X$. Show that
  there exists disjoint open sets $U$ and $V$ containing $A$ and $B$ respectively.}

Since $A$ is compact, we follow that for all $b \in B$ we've got that $b \notin A$, and thus
for all $b \in B$ there exists a pair $U_b$ and $V_b$ such that $A \subseteq U_b$
and $b \in V_b$. We then conclude that collection of $V_b$'s is an open cover for $B$
and thus it's got a finite subcover $\set{V_\alpha}_{\alpha \in A}$. We follow then that
we can take $\set{U_\alpha}_{\alpha \in A}$, and since $A$ is finite, we follow that
$$\bigcap{U_\alpha}$$
is an open set, that contains $A$ and is disjoint from every $\set{V_\alpha}_{\alpha \in A}$.
Thus we conclude that we can create two disjoint sets
$\bigcap{U_\alpha}$ and $\bigcup{V_\alpha}$, as desired.

\subsection{}

\textit{Show that if $f: X \to Y$ is continous, where $X$ is compact and $Y$ is
  Hausdorff, then $f$ is a closed map (that is, $f$ carries closed sets to closed sets)}

Let $U \subseteq X$ be a closed set. Since $X$ is compact, we follow that $U$ is compact as well.
Since $f$ is continous, we follow that $f[U]$ is compact as well, and since $Y$ is Hausdorff
and $f[U] \subseteq Y$ is compact, we follow that $f[U]$ is closed as well. Thus we've got the
desired result.

\subsection{}

\textit{Show that if $Y$ is compact, then the projection $\pi_1: X \times Y \to X$ is a
  closed map}

Let $U$ be a closed set in $X \times Y$.
If for all $x \in X$ there exists $y \in Y$ such that  $\eangle{x, y} \in U$, then we follow that
$\pi_1(U) = X$, and since both $U$ and $X$ are closed, we follow that we're done.

Thus assume that there exists $x \in X$ such that for all $y \in Y$ there's no $\eangle{x, y} \in U$.
We follow that $\set{x} \times Y \notin U$, and since $(X \times Y) \setminus U$ is open, we follow
that there exists a neighborhood $W$ of $x$ such that
$W \times Y \subseteq (X \times Y) \setminus U$. This essentially implies that
for all $q \in X \setminus \pi_1(U)$ there exists a neighborhood around $q$ such that
this neighborhood is contained in $X \setminus \pi_1(U)$. This in turn implies that
$X \setminus \pi_1(U)$ is open and thus $\pi_1(U)$ is closed, as desired.

\subsection{}

\textit{\textbf{Theorem}. Let $f: X \to Y$; let $Y$ be compact Hausdorff. Then $f$ is
  continous if and only if the graph of $f$,
  $$G_f = \set{\eangle{x, f(x)}: x \in X}$$
  is closed in $X \times Y$.}

Suppose that $f$ is continous. Let $q = \eangle{x, y} \in G_f$. Let $B$ be a
basis neighborhood around $q$.

Suppose that there exists $q = \eangle{x, y'} \in X \times Y$ such that 
$q \notin G_f$. We follow that there also exists $w = \eangle{x, y} \in G_f$ such that
$y \neq y'$. The fact that $Y$ is Hausdorff gives us that there exist disjoint neighborhoods
$U$ and $V$ of $y$ and $y'$ respectively. Since $f$ is continous, we follow that $f\inv[U]$
is open as well. We now can follow that $f\inv[U] \times V$ is a neighborhood of $q$
that does not intersect $G_f$, and thus we conclude that
if $x \notin G_f$, then $x \notin \overline{G_f}$, which implies that $G_f$ is closed.

Now assume that $G_f$ is closed. Let $x_0 \in X$. We follow that $\eangle{x_0, f(x_0)} \in G_f$.
Let $V$ be a neighborhood of $f(x_0)$. Since $V$ is open, we follow that $X \times V$
is open and thus $(X \times Y) \setminus (X \times V) = X \times (Y \setminus V)$
is closed. Thus $G_f \cap X \times (Y \setminus V)$ is closed. Previous
exercise implies that
$$\pi_1[G_f \cap X \times (Y \setminus V)]$$
is closed. Thus
$$\set{x \in X: f(x) \notin V}$$
is closed. Therefore $f\inv[V] = X \setminus \set{x \in X: f(x) \notin V}$ is open,
which implies that $f$ is continous, as desired.

\subsection{}

\textit{Generalize the tube lemma as follows: }

\textit{\textbf{Theorem: } Let $A$ and $B$ be subspaces of $X$ and $Y$, respectively; let $N$
  be an open set in $X \times Y$ containing $A \times B$. If $A$ and $B$ are compact, then there
  exists open sets $U$ and $V$  in $X$ and $Y$ respectively, such that
  $$A \times B \subseteq U \times V \subseteq N$$}

\textit{TODO later}

\section{Compact Subspaces of the Real Line}

\subsection{}

\textit{Prove that if $X$ is an ordered set in which every closed interval is compact, then $X$
  has the least upper bound property.}

Since $X$ is closed, we follow that it's compact as well. Assume that $X$ does not have a
least upper bound property. We then can follow that there exists $U \subseteq X$ such that
$U$ is bounded above (in the sence that there exists $M \in X$ such that $u < M$ for all $u \in U$).
and it does not have a least upper bound. Since $U$ is bounded, we follow that there exists
set $P$, which consists of upper bounds of $U$. We follow that we can define a collection of sets
$V$ by
$$V_p = (p, \infty)$$
for all $p \in P$. Since $X$ has an order topology, we follow that each $V_p$ is open. Now
let us define a collection of open sets $W$ by
$$W_u = (u, \infty)$$
for all $u \in U$. Let $x \in X$. We follow that $x$ is either an upper bound of $U$, in which
case it's located in $\bigcup{V}$, or it's not an upper bound of $U$, in which case there
exists $u \in U$ such that $x < u$, and thus $x \in \bigcup{W}$. Therefore we conclude that
$V \cap W$ constitutes an open cover of $U$. Since $U$ does not have a lowest upper bound,
we can also conclude that $U$ is an infinite set, and thus we follow that $W$ is an infinite
collection as well.

We can now follow pretty easily that for all $x \in U$ there exists a unique set $Q$ in $V \cup W$
such that $x \in Q$, thus implying that $V \cup W$ does not have an open subcover, which
in turn implies that $X$ is not compact, which is a contradiction.

\subsection{}

\textit{Let $X$ be a metric space with metric $d$; let $A \subseteq XX$ be nonempty.}

\textit{(a) Show that $d(x, A) = 0$ if and only if $x \in \overline{A}$.}

Suppose that $d(x, A) = 0$. Let $B(x, \epsilon)$ be an arbitrary basis neighborhood around $x$.
Since $d(x, A) = 0$ and thus
$$\inf\set{d(x, a): a \in A} = 0$$
we follow that there exists $a' \in A$ such that $d(x, a') < \epsilon$. Thus we follow that
$a' \in B(x, \epsilon)$, and thus we conclude that every basis neighborhood and therefore
every neighborhood of $x$ intersects $A$ at some point. Thus we conclude that $x \in \overline{A}$.

Let $x \in \overline{A}$. We follow that every basis neighborhood of $x$ intersects $A$ at some
point. Thus we follow that for all $\epsilon > 0$ there's $a \in A$ such that
$d(x, a) < \epsilon$. Thus we follow that $0$ is the lower bound of
$$\set{d(x, a): a \in A}$$
and since metric is nonnegative we conclude that infinum of this set is $0$, as desired.

\textit{(b) Show that if $A$ is compact, $d(x, A) = d(x, a)$ for some $a \in A$.}

We follow that if $x \in A$, then $d(x, A) = d(x, x) = 0$, and thus we're done. Thus
assume that $x \notin A$.

Let us define a collection of closed sets $C$ by
$$C_\epsilon = \set{q \in X: d(q, x) \geq \epsilon}$$
for $\epsilon \geq d(x, A)$.
We follow that collection $V$ defined by
$$V_\epsilon = R \setminus C_\epsilon$$
is a set of open sets.

Assume that there's no point $q \in X$ such that $d(x, q) = d(x, A)$. 
We follow then that $C_\epsilon$ does not contain any points in $A$, and thus we follow that
$V_\epsilon$ is an open cover for $A$. Since $A$ is compact, we follow that
there's a finite cover of $A$ in $V$. Since $V$ is a set of nested opened sets (it's
not hard to check that $\epsilon > \delta \ra V_\epsilon \subseteq V_\delta$ ), we follow that
there exists $\epsilon$ such that $A \subseteq V_\epsilon$, which implies that
$d(x, A) > d(x, A)$, which is a contradiction.

\textit{(c) Define the $\epsilon$-neighborhood of $A$ in $X$ to be the set
  $$U(A, \epsilon) = \set{x: d(x, A) < \epsilon}$$
  Show that $U(A, \epsilon)$ equals the union of the open balls $B_d(a, \epsilon)$ for
  $a \in A$.}

Let $x \in U(A, \epsilon)$. We follow that $d(x, A) < \epsilon$. Thus
$$\inf\set{d(x, a): a \in A} < \epsilon$$
which in turn implies that there exists $a \in A$ such that $d(x, a) < \epsilon$. This in turn
implies that $x \in B_d(a, \epsilon)$, and thus
$$x \in \bigcup_{a \in A}{B_d(a, \epsilon)}$$
thus we conclude that
$$U(A, \epsilon) \subseteq \bigcup_{a \in A}{B_d(a, \epsilon)}$$

Let $x \in B_d(a, \epsilon)$ for some $a \in A$. We follow that $d(x, a) < \epsilon$,
and thus $d(x, A) < \epsilon$. Therefore  $x \in U(A, \epsilon)$ by definition. Thus
we've got the desired equality by double inclusion.

\textit{(d) Assume that $A$ is compact; let $U$ be an open set containing $A$. Show that
  some $\epsilon$-neighborhood of $A$ is contained in $U$.}

Let $a \in A$. We follow that since $A$ is compact we've got that $A$ has a
Lebesgue number $\delta$. Since $U$ contains $A$, we follow
that $\set{U}$ it's a finite open cover of $A$. We also follow that $a \in B_d(a, \delta)$,
and thus proof of Lemma 27.5 implies that $\delta$-neighborhood of $a$ is contained
in $U$.

Since every $\delta$-neighborhood of each $a \in A$ is contained in $U$, we follow by
previous point that $U(A, \delta)$ is contained in $U$, as desired.

\textit{(e) Show the result in (d) need not to hold if $A$ is closed but not
  compact}

We can actually use the set, that has already beed defined in the chapter. Let $U, V \subseteq R^2$
such that 
$$V = \set{\eangle{x, 1/x}: x \in R_+}$$
and
$$U = \set{\eangle{x, y}: x, y \in R_+}$$
we have already proven that the former is a closed set, and it's not that hard to understand why
the latter is an open set. Showing that those two sets satisfy the desired constraints is trivial

\subsection{}

\textit{Recall that $R_K$ denoted $R$ in the K-topology}

\textit{(a) Show that $[0, 1]$ is not compact as a subspace of $R_K$.}

Just to remind myself, $K = \set{1/n: n \in Z_+}$ and $R_K$ is the topology
whose basis is all of the open intervals in $R$, as well as all intervals minus $K$. 

We can follow that $(-1, 2) \setminus K$ is open in $K$. We also follow that for
every $k \in K$ such that $k = 1/n$ there's an open interval $(1/n - 1/2n, 1/n + 1/2n)$
so that we've got that
$$(1/n - 1/2n, 1/n + 1/2n) \cap K = \set{k}$$
Thus we can define collection $V$ by adding set $(-1, 2) \setminus K$ to it, as well as every
$(1/n - 1/2n, 1/n + 1/2n)$ for every $n \in Z_+$. Some trivial checks will imply that
given collection constitutes an open cover for $[0, 1]$, for which there's no finite
subcover of $[0, 1]$, which implies that $[0, 1]$ is not compact.

\textit{(b) Show that $R_K$ is connected.}

We follow that both $(-\infty, 0)$ and $(0, \infty)$ inherit their usual topologies
as subspaces of $R_K$, as shown shomewhere in the book before. If not, then it's pretty trivial
to check.

Now let $B$ be a basis neighborhood of $0$. We follow that if $B$ is just a standart interval,
that it intersects both of those sets. If $B$ is in form $(a, b) \setminus K$, then we follow that
$b > 0$, and then we can follow that there's an irrational number $i$ such that
$0 < i < b$. Since $i \notin K$ ($K \subseteq Q$), we follow that any basis neighborhood
of $K$ intersects both of sets $(-\infty, 0)$ and $(0, \infty)$, thus it's a
limit point for both of those sets, and thus $(-\infty, 0]$ and $[0, \infty)$ are
both connected, which implies that $R$ is connected under $K$-topology, as desired.

\textit{(c) Show that $R_K$ is not path connected.}

Let $f: [a, b] \to [0, 1]$ be a map, where we treat $[a, b]$ as a subspace in standart topology
and $[0, 1]$ as the subspace in $K$-topology such that $f(a) = 0$ and $f(b) = 1$. If this
map is not continous, if we treat $[0, 1]$ in standart topology, then we can follow that
it isn't continous with respect to $K$-topology as well, since $K$-topology is finer than
the standart topology. Thus assume that it is continous with respect to standart topology.
IVT gives us that since $[a, b]$ is connected and $[0, 1]$ is continous with respect to
the standart topology (which is order topology), that we've got IVP on this function
and thus image of this function is $[0, 1]$. This implies that with respect to the
$K$-topology $f$ maps compact $[a, b]$ onto non-compact $[0, 1$, as proven in part (a).

Thus we can conclude that $f$ is not continous. Since $a, b$ are arbitrary, we conclude that
points $0$ and $1$ are not path connected, and thus $R_K$ itself is not path-connected, as desired.

\subsection{}

\textit{Show that a connected metric space having more than one point is uncountable}

Let $X$ be such a space. Since $X$ is metric, we follow that it's Hausdorff. Let $x \in X$.
We follow that $\set{x}$ is closed, since it's Hausdorff. Thus if $\set{x}$ is open,
then $X$ is not connected, which is a contradiction, which implies that $\set{x}$
is not open, which implies that $X$ doesn't have any isolated points.

Let $x_1, x_2$ be two distinct  points of a given space. We follow that there exists
$c > 0$ such that $d(x_1, x_2) = c$. Now suppose that there exists $b \in R$ such that
$0 < b < c$  and there's no $x_3 \in X$ such that $d(x_1, x3) = b$. Let
$y \in X \setminus B_d(x, b)$. We follow that $d(x_1, y) > b$ and thus we follow that
if $q \in B_d(y, d(x_1, y) - b)$, then
$$d(y, q) < d(x_1, y) - b \leq d(x_1, q) + d(q, y) - b$$
$$d(y, q) < d(x_1, q) + d(q, y) - b$$
$$d(y, q) < d(x_1, q) + d(y, q) - b$$
$$0 < d(x_1, q) - b$$
$$b < d(x_1, q)$$
thus we conclude that $q \in X \setminus B_d(x, b)$. This implies that
$$B_d(y, d(x_1, y) - b) \subseteq X \setminus B(x, b)$$
which implies that $X \setminus B_d(x, b)$ is an open set, which implies that
$B_d(x, b)$ and $X \setminus B_d(x, b)$ form a separation of $X$, which means that $X$
is not connected, which is a contradiction.

This gives us that for all $q \in R$ such that  $0 < q \leq d(x_1, x_2)$ there
exists $x' \in X$ such that $d(x_1, x') = q$. This implies that we can define a function
$f: (0, d(x_1, x_2)) \to \pow(X)$ by
$$f(y) = \set{x \in X: d(x_1, x) = y}$$
and that for all $q$'s we follow that $f(q)$ is nonempty. Now we can employ AC to give us a
function $g(q): (0, d(x_1, x_2)) \to X$ such that
$$d(x_1, g(q)) = q$$
We can follow that if $a, b \in (0, d(x_1, x_2))$ and $a \neq b$, then
$$d(x_1, g(a)) \neq d(x_1, g(b))$$
and thus $g(a) \neq g(b)$, which means that $g$ is injective, which implies that we've got
injective function from $(0, d(x_1, x_2))$ to $X$, which means that
$$|(0, d(x_1, x_2))| \leq_C |X|$$
Since every given interval has a bijection with $R$ we follow that 
$$|\omega| <_c|(0, d(x_1, x_2))|$$
and then we can conclude that
$$|\omega| <_C |X|$$
thus $X$ is uncountable, as desired.

\subsection{}

\textit{Let $X$ be a compact Hausdorff space; let $\set{A_n}$ be a countable collection of
  closed sets of $X$. Show that if each set $A_n$ has empty interior in $X$, then
  the union $\bigcup{A_n}$ has empty interior in $X$.}

Since $\set{A_n}$ is a countable collection, let us denote distinct $A_n$'s by $A_n$ where
$n \in \omega$ for clarity's sake.

If $\bigcup{A_n}$ is empty, then its interior is empty since the interior is the
subset of its original set, thus let us assume that $\bigcup{A_n}$ is nonemty.

Since $X$ is compact, we follow that each $A_n$ is a closed subspace of a compact space, and
thus it is compact as well. Now lemma 26.4 implies that for each $y \notin A_n$ there
exist disjoint open sets $U$ and $V$ such that $y \in U$ and $A_n \subseteq V$.

Let $x \in \bigcup{A_n}$ and let $U$ be a neighborhood of $x$ with respect to topology on $X$.
Since each $A_n$ has an empty interior and $x \in \bigcup{A_n}$ and thus $x \in A_n$ for some
$n \in \omega$, we follow that $U \not \subseteq A_n$, and
thus $U \setminus A_n$ is nonempty. Thus there exists $y \in U \setminus A_n$ and open $V', W$
such taht $y \in V'$ and $A_n \subseteq W$. Since $V'$ and $U$ are open and $y$ is
contained in both, we follow that there exists $V = V' \cap U$.

We can follow that if $w \in W$, then there exists a neighborhood $Q$ of $w$ that is a
subset of $W$, and since $V$ and $W$ are disjoint, we follow that $Q$ and $V$ are
disjoint, which implies that $w$ is not a limit point of $V$, thus implying that
$w \notin \overline{V}$. Therefore we follow that $W$ and $\overline{V}$ are disjoint.
Same logic applies to the set $\overline{V \cap U}$

Thus we follow that we can take some point $x \in \bigcup{A_n}$, take
its neighborhood $U$, apply given argument to a sequence 
$$A_1, A_2, A_3...$$
which will produce a sequence of nested closed sets
$$\overline{V_1} \supseteq \overline{V_2} \supseteq \overline{V_3} ...$$
such that for all $n \in \omega$ we'll have that 
$$\overline{V_n} \cap A_n = \emptyset$$
and
$$\overline{V_n} \subseteq U$$
since this sequence is nested, we follow that it's got finite intersection property,
and since it's closed and $X$ is compact, we can follow $\bigcap{V_n}$ is nonempty.
Then we can follow that if $z \in \bigcap{V_n}$, then $z \in U$ since every $V_n$ is
in $U$, and we can also follow that $z \notin A_n$ for all $n \in \omega$.

Thus we can conclude that if $x \in \bigcup{A_n}$, then every neighborhood of $x$
has a point, that is not contained in $\bigcup{A_n}$. This implies that no open
sets are contained in $\bigcup{A_n}$, which implies that it's got an empty interior,
as desired.

\subsection{}

\textit{This exercise was handled fully in my real analysis course.}

\section{Limit Point Compactness}

\subsection{}

\textit{Give $[0, 1]^\omega$ the uniform topology. Find an infinite subset of this
  space that has no limit point.}

Let sequence of $x_n \in [0, 1]^\omega$ be defined by
$$x_n(i) =
\begin{cases}
  i \leq n \ra 0 \\
  1 \text{ otherwise}
\end{cases}\
$$

\subsection{}

\textit{Show that $[0, 1]$ is not limit point compact as a subspace of $R_l$.}

We follow that we can set
$$x_n = 1 - 1/n$$
Since $\set{1}$ is opened in this topology (i.e. $[1, 2) \cap [0, 1] = \set{1}$) and
none of the $x_n$'s are in $\set{1}$, we follow that $1$ is not a limit point of a given
set. Since $R_l$ is finer than standart topology and $1$ is the only limit point of
a given set in standart topology, we conclude that given set does not have no limit points
with respect to $R_l$, as desired.

\subsection{}

\textit{Let $X$ be limit point compact.}

\textit{(a) If $f: X \to Y$ is continous, does it follow that $f(X)$ is limit point compact?}

Let $Q$ consist of two points and let it have trivial topology. We follow that
$X = Z_+ \times Q$ is limit point compact (follows from example in the chapter). Thus we can define
$f: X \to Z_+$ to be the projection function. We follow that this function is continous
(as proven in the book previously), however $Z_+$ is not limit point compact (since it
has discrete topology), thus proving the counterpoint for this particular case

\textit{(b) If $A$ is a closed subset of $X$, does it follow that $A$ is limit point
  compact?}


Suppose that $U$ is an infinite subset of $A$. Since $A \subseteq X$, we follow that
$U \subseteq X$, and thus there is a limit point $j$ of $U$. Since $U \subseteq A$
and $A$ is closed, we follow that $\overline{U} \subseteq \overline{A}$ (exercise 17.6),
and since $A$ is closed we follow that $\overline{A} = A$ and thus $\overline{U} \subseteq A$.
Since $j$ is a limit point of $U$ we follow that $j \in \overline{U}$ and thus $j \in A$.
Thus we conclude that any infinite subset of $A$ has a limit point, which means that
$A$ is limit point compact, as desired.

\textit{(c) If $X$ is a subspace of Hausdorff space $Z$, does it follow that $X$ is closed
  in $Z$?}

We can follow that with order topology $\overline{S_\Omega}$ is Hausdorff. As it was shown
previously, $S_\Omega$, which is a subspace of $\overline{S_\Omega}$ is limit point compact.
We follow that if $B$ is a basis neighborhood of $\Omega$, then there exists $a \in S_\Omega$
such that $B \subseteq (a, \infty)$. Since $S_\Omega$ does not have a highest element,
we follow that $a$ is not a highest element of $S_\Omega$, and thus there exists
$j \in S_\Omega$ such that $j \in B$. Thus $\Omega$ is a limit point of $S_\Omega$ (there's
a more approachable set-theoretical explanation there if we assume GCH and instead of $S_\Omega$ we
consider $N_1$) It follows that $\Omega$ is a limit point of $S_\Omega$,
which implies that $S_\omega$ does not contain its limit points and thus it is not closed.

\subsection{}

\textit{A space $X$ is said to be countably compact if every countable open covering
  of $X$ contains a finite subcollection that covers $X$. Show that for a $T_1$ spase $X$,
  countable compactness is equivalent to limit point compactness.}

Suppose that $X$ is $T_1$ (meaning that any singleton is closed). This condinition
implies that any given finite subset is closed (i.e. contains its limit points) by
basic properties of closed subsets.

Suppose that $X$ is limit point compact. We firstly state that if $X$ is finite, then its
topology is finite (topology is a subset of a power set, and the size of a power set is $2^{|X|}$)
and thus vacuously there's always a finite subcovering. Thus assume that $X$ is infinite.
Let $U_n : \omega \to \pow(X)$ be a countable
open covering for $X$ and assume that there's no finite subcollection of $U_n$
that covers $X$. Since $U_n$ does not have a finite subcovering, we follow that there exists
a point $x \notin U_1$. In general we can follow that
$$X \setminus \bigcup_{j \leq n}{U_j}$$
is nonemty. Moreover, for the same reason we may conclude that $X \setminus \bigcup_{j < n}{U_n}$
is infinite for any given $n \in \omega$. Thus let us define a sequence $x_n: \omega \to X$ by
$$x_n \in (X \setminus (\bigcup_{j \leq n}{U_n} \cup \set{x_1, x_2, ... x_{n - 1}})$$
(the idea here is to have a sequence such that $x_n \notin U_1 \cup U_2 ... \cup U_n$ and
such that each $x_n$ is unique so that the range of this sequence is infinite). 
We follow that sequence $x_n$ will constitute an infinite subset of $X$, and thus it'll
have a limit point $l \in X$.

Since $X$ has $T_1$ property, we follow that $\set{x_1, x_2, ..., x_j}$ is closed. Thus
$X \setminus \set{x_1, x_2, ..., x_j}$ is open. Thus
$$U_j \cap (X \setminus \set{x_1, x_2, ..., x_j})$$
is open.

Assume now that $l = x_m$ for some $m \in \omega$,
then we follow that $l \in U_j$ for some $j \in \omega$,
which implies that for all $k \geq j$ we've got that $x_k \notin U_j$. Thus $U_j$
intersects $\set{x_1, ..., x_{j - 1}}$ at some point other than $x_m$. Since
$\set{x_1, ..., x_{j - 1}} \setminus \set{x_m}$ is a closed set, we follow that
$$U_j \setminus (\set{x_1, ..., x_{j - 1}} \setminus \set{x_m})$$ is an open
neighborhood of $l = x_m$, which
intersects $(x_n)$ at some point other that $l$, which is a contradiction, since all
of the points of $(x_n)$ (perhaps with exception of $x_m$) are outside of this set.

Thus we conclude that $l \notin (x_n)$. 
Let $j \in \omega$ be such that $l \in U_j$. We follow $U_j$ intersects a sequence $(x_n)$ at
some point other than $l$. We follow that for all $i \geq j$ we've got that $x_i \notin U_j$
by definition, and none of the points $\set{x_1, ..., x_{j - 1}}$ are in the neighborhood
$$U_j \setminus \set{x_1, ..., x_{j - 1}}$$
of $l$.
Thus we conclude that $l$ has a neighborhood that does not intersect $(x_n)$, which implies that
it's not a limit point, which is a contradiction.

Therefore we conclude that our original assumntion that $U_n$ does not have a finite
subcovering is false, thus implying that as long as $X$ is $T_1$, then limit point
compactness of $X$ implies its countable compactness.

Now suppose that $X$ is countably comact. Let $K$ be an countably infinite subset of $X$
that does not have a limit point. That means that it contain all of its limit points
and thus $K$ is closed. Thus we follow that $X \setminus K$ is open, and moreover,
$$X \setminus (K \setminus \set{k_n})$$
is open for all $k_n$. Since $X \setminus (K \setminus \set{k_n})$ constitutes a
countable open cover of $X$, we follow that it's got a finite subcover, and thus we
conclude that $K$ is finite, which is a contradiction.

Thus we conclude that any countably infinite subset of $X$ has a limmit point. Now
if $Q$ is uncountably infinite subset of $X$, then none of its subsets have a limit point,
which implies that countable subsets of $Q$ don't have limit point, which is a contradiction. Thus
we conclude that any infinite subset of $X$ has a limit point, thus $X$ is limit point compact, as
desired.

\textit{Given the fact that we haven't used $T_1$ in the proof of the converse, we either
  have some sort of a nice property of countable compactness in general, and not just $T_1$
  sets, or I've made a mistake. My money is on the latter.}

\subsection{}

\textit{Show that $X$ is countably compact if and only if every nested sequence
  $$C_1 \supseteq C_2 \subseteq ... $$
  of closed nonempty sets of $X$ has a nonempty intersection.}

Suppose that $X$ is countably compact. We follow that if
$$C_1 \supseteq C_2 \subseteq ... $$
is a nested sequence of closed nonempty sets of $X$, then
$$X \setminus C_n$$
is a sequence of open sets of $X$. Thus $C_1 \supseteq C_2 \subseteq ... $ is empty if and only if
the union of $X \setminus C_n$ is $X$. The latter then constitutes a countable open cover,
which has a finite subcover. Given that the given sequence of open sets is also nested, but
in the other direction (i.e.
$$X \setminus C_1 \subseteq X \setminus C_2 \subseteq ...$$
we follow that there exists $j \in \omega$ such taht $X \setminus C_j = X$. This implies that
$C_j = \emptyset$, which is a contradiction.

Now assume that every nested sequence $C_1 \supseteq C_2 ... $ of closed nonempty sets of $X$
has a nonempty intersection. Let $U_n$ be a countable open subcover of $X$. We follow that
$$X \setminus \bigcup_{j \leq n}{U_j}$$
constitutes a nested sequence of closed nonempty sets of $X$. Thus it's got a nonempty
intersection, which in turn implies that $U_n$ does not cover $X$, which is a contradiction.

\subsection{}

\textit{Let $(X, d)$ be a metric space. If $f: X \to X$ satisfies the condition
  $$d(f(x), f(y)) = d(x, y)$$
  for all $x, y \in X$, then $f$ is called an isometry of $X$. Show that if $f$ is an
  isometry and $X$ is compact, then $f$ is bijective and hence a homeomorphism.}

We follow that if $x \neq y$, then $d(x, y) \neq 0$, thus $d(f(x), f(y)) \neq 0$, which
implies that $f(x) \neq f(y)$, which makes $f$ injective. 

If there exists $a \in X$ such that $a \notin f[X]$. We follow that since $X$ is compact, that
there exists $\epsilon$-neighborhood of $a$ that is located outside of $f[X]$.
Let
$$x_1 = a$$
$$x_2 = f(x_1)$$
$$x_3 = f(x_2)$$
and in general
$$x_n = f(x_{n - 1})$$
We follow that since $x_2 \in f[X]$ that $x_2 \notin B_d(x, \epsilon)$, and thus
$d(x_2, x_1) \geq \epsilon$. By the same logic we can follow that since $x_n \in f[X]$
for all $n > 1$, we follow tha $d(x_n, x_1) \geq \epsilon$.
We can also follow that since
$$d(f(x), f(y)) = d(x, y)$$
that
$$d(x_3, x_2) = d(f(x_2), f(x_1)) = d(x_2, x_1) \geq \epsilon$$
in general we can follow that if $m > n$, then
$$d(x_m, x_n) = d(f(x_{m - 1}), f(x_{n - 1})) = d(x_{m - 1}, x_{n - 1}) = ... = d(x_{m - n + 1}, x_1)
\geq \epsilon$$
thus by commutativity of $d$ we follow that if $m \neq n$, then $d(x_m, x_n) \geq \epsilon$
(although this reasoning is pretty thorough, it's not as rigorous as it can be; more rigorous
proof of this conclusion can be drawn from induction; GOTO set theory course, part on arithmetics
for a more concrete example).

This in turn implies that we can take our initial sequence, take bases around its elements
with radius less than $\epsilon$, and we'll get infinite cover of this subsequence,
which will not have finite subcover. Since $X$ is metrizable, we follow that its compactness
is equivalent to limit point compactness, and we follow that given sequence can
not have a limit point, which implies that we've got a contradiction. Thus we conclude that
there does not exist $a \in X \setminus f[X]$, which implies that $f$ is bijective.
And within given circumstances it is trivial to prove that given function is a homeomorphism.

\subsection{}

\textit{Let $(X, d)$ be a metric space. If $f$ satisfies the condition
  $$d(f(x), f(y)) < d(x, y)$$
  for all $x, y \in X$ with $x \neq y$, then $f$ is called a shrinking map. If there's a number
  $\alpha < 1$ such that
  $$d(f(x), f(y)) < \alpha d(x, y)$$
  for all $x, y \in X$ then $f$ is called a contraction. A fixed point of $f$ is a point $x$ such
  that $f(x) = x$. }

\textit{(a) If $f$ is a contraction and $X$ is compact, show $f$ has a unique fixed point.}

Define
$$C_0 = X$$
$$C_1 = f[X]$$
$$C_2 = f[f[X]]$$
and in general
$$C_n = f^n[X]]$$
Since $X$ is compact and nonempty, we follow that $f[X]$ is compact and
nonempty, and in general $f^n[X]$ is compact and nonempty
(more rigorously this thing can be proven once again by induction).
Since every compact subspace in a Hausforff space is closed, we follow that every $C_n$ is closed.
Thus we follow that $\bigcap{C_n}$ is nonempty. Let $x \in \bigcap{C_n}$. Suppose that $x \neq f(x)$.
We follow that $d(x, f(x)) = \epsilon$ for some $\epsilon \in R$. We also follow that
$f(x) \in \bigcap{C_n}$. We can follow that the diameter of all the $C_n$'s converging to zero,
which implies that there couldn't be two points in $\bigcap{C_n}$. This in turn implies that we've got
a contradiction, and thus $x = f(x)$. We also follow for the same reason that given $x$ is
unique, as desired.

\textit{(b) Show more generally that if $f$ is a shrinking map and $X$ is compact, then $f$
  has a unique fixed point.}

We can define $C_n$ in the same way as in the previous point, and also conclude the same
things as before. Define sequence $x_n$ such that $x_n \in C_n$. We follow that if $a$ is the
limit of some subsequence, then every neighborhood of $a$ intersects $x_n$. Thus we follow that
a ball with any given radius intersects $x_n$ at some point. Thus we follow that if
$a \notin \bigcap{C_n}$, then we follow that there exists $j \in \omega$ such that
$a \notin C_j$. Given that any of the $C_j$ is the closed set, we follow that $x \setminus C_j$
is open, and thus there exists a neighborhood of $a$ that does not intersect $C_j$. Given
that all the $C_n$'s are nested, we follow that given neighborhood of  $a$ does not intersect
any of the $C_n$'s for $n \geq j$, and thus does not intersect any of the $x_n$'s for the
same $n$'s. Given that there are finitely many of $x_n$'s such that $n < j$, we follow that
$a$ is not a limit point of $x_n$, which is a contradiction. Thus we conclude that
$a \in \bigcap{C_n}$. We follow that if $f(a) \neq a$, then TODO


\textit{(c) Let $X = [0, 1]$. Show that $f(x) = x - x^2/2$ maps $X$ into $X$ and is a shrinking map
  that is not a contraction.}

We follow that $f'(x) = 1 - x$ and thus $f'(x) = 0 \lra x = 1$ and $x \in [0, 1) \ra f'(x) > 0$.
Thus we can follow that minimal point of $f$ on $X$ happens in $0$ and maximal point happens at
$1$. We follow that $f(0) = 0$ and $f(1) = 1/2$, thus $f[X] = [0, 1/2] \subseteq [0, 1]$.
For the discussion on the extremums and whatnot goto any given calculus book and/or
real analysis course. BTW, given "into" notation is defined as "$f$ maps $A$ into $B$
if and only if $f[A] \subseteq B$"

Now we need to follow that this thing is a shrinking map. Assume that $x > y$. Our discussion
concerning $f'(x)$ implies that $f$ is strictly increasing on $X$, and thus $f(x) > f(y)$. Therefore
$$d(f(x), f(y)) = |x - x^2/2 - (y - y^2/2)| = x - x^2/2 - (y - y^2/2) = x - x^2/2 - y + y^2/2 = $$
$$ = (x - y) - (x^2/2 - y^2/2)$$
We can follow with pretty much the same arguments that $g(x) = x^2/2$ is incresing on
$X$, thus $x^2/2 - y^2/2 > 0$, and therefore
$$d(f(x), f(y)) = (x - y) - (x^2/2 - y^2/2) < x - y = |x - y| = d(x, y)$$
if $x > y$. If $x < y$, then we follow the same result since $d$ is commutative. Case of $x = y$
is obvious, and doesn't require our attention sice definition of a shrinking map excludes this
case for obvious reasons. Thus we conclude that $f$ is shrinking.

Now MVT implies that if $b > a$, then there exists $c \in [a, b]$ such that
$$f'(c) = \frac{f(b) - f(a)}{b - a}$$
thus
$$f'(c)(b - a) = f(b) - f(a)$$
and given that we can set $b = 1$ and $a$ to be arbirtarily close to $b$,  we follow that
there's no $\alpha < 1$ such that
$$\alpha(b - a) = f(b) - f(a)$$
for all $b > a \in X$.  Thus we can use the fact that $f$ is increasing on a given set and
use commutativity of $d$ to follow that there's no $\alpha \in [0, 1)$ such that 
$$\alpha d(a, b) \geq d(f(a),f(b))$$
which implies that $f$ is not a shrinking map on $X$. It's (probably) important to note
here that although $f$ is not shrinking on $X$, it might be shrinking on some subset of $X$,
thus concluding that domain of the function playes the role in whether or not the function
is shrinking (or a contraction for that matter).

\textit{The rest (i.e. part b and d) of this exercise is left for better times}

\section{Local Compactness}

\subsection*{Notes}

We can follow pretty easily that a space $X$ is locally compact at $a$ if and only if some compact
subspace $C$ of $X$ that contains a basis neighborhood of $X$.

\subsection{}

\textit{Show that the rationals $Q$ are not locally compact}

We can follow that given $a, b \in R$ such that $a \neq b$ the set $[a, b] \cap Q$ is not comact. 
Since there exists $i_1, i_2 \in [a, b]$ such that $i_1, i_2 \in I$ and $i_1 \neq i_2$
we follow that $i_1, i_2 \notin [a, b] \cap Q$ and thus $([a, i_1) \cup (i_1, i_2) \cup (i_2, b])
\cap Q = [a, b] \cap Q$. We can follow
pretty easily that $(i_1, i_2)$ is not compact that thus $[a, b] \cap Q$ is not compact.

Let $a \in Q$. Let $U$ be a basis neighborhood of $a$. We follow that there exist $a, b \in R$
such that $U = (a, b) \cap Q$. We then follow that there exists an interval $[a', b']$
such that $[a', b'] \subseteq (a, b)$ and $a \in [a', b']$, which implies that no
subspace $C$ that contains $U$ is compact. Since $U$ and $a$ is arbitrary, we follow that
at no point the space $Q$ is locally compact, as desired.

\subsection{}

\textit{Let $\set{X_\alpha}$ be an indexed family of nonempty spaces.}

\textit{(a) Show that if $\prod{X_\alpha}$ is locally compact, then each $X_\alpha$ is locally
  compact and $X_\alpha$ is compact for all but finitely many values of $\alpha$.}

We can follow that if $\set{X_\alpha}$ is a finite set, then our implication doesn't say anything.
Thus assume that $\set{X_\alpha}$ is infinite.

Suppose that there exist infinitely many $n$'s such that $X_n$ is not locally compact.
Suppose that $b \in \prod{X_\alpha}$ and $U$ is a basis neighborhood of $b$. We follow that
there exists a collection of open sets $A_n $ such that
$$U = \prod{A_n}$$
and such that $A_n \neq X_n$ for finitely many $n$ since we assume the product topology here.
We follow that there exists $n$ such that $X_n$ is not locally compact and $A_n = X_n$.

Since $X_n$ is not locally compact, we follow that there exists $j \in X_n$ such that
there's no compact subspace of $X_n$ that contains a neighborhood of $j$.

Suppose that there's a compact subspace of $\prod{X_\alpha}$ that contains
$b$ and is a superset of $U$. We then can follow pretty easily
that this would give us that $X_n$ is compact, which is a contradiction.

\textit{(b) Prove the converse, assuming the Tychonoff theorem}

To refresh the memory: Tychonoff theorem states that product of infinite compact spaces
is compact.

Now suppose that each $X_\alpha$ is locally compact and $X_\alpha$ is compact for all but
finitely many values of $\alpha$.

Let $\set{X_\alpha}$ consist of two locally compact spaces, namely $X_1$ and $X_2$.
Let $\eangle{q, w} \in X_1 \times X_2$. We follow that $q \in X_1$, $w \in X_2$ and
thus there exist neighborhoods $U$ and $W$ of $q$ and $w$ respectively in $X_1$ and
$X_2$ respectively such that there exist compact subspaces $C_1$ and $C_2$
such that $U \subseteq C_1$, $W \subseteq C_2$. Product of compact spaces is compact, thus
we follow that $C_1 \times C_2$ is compact. Thus we follow that $U \times W$ is an open
subset of $X_1 \times X_2$ that is a neighborhood of $\eangle{q, w}$, which implies that
$X_1 \times X_2$ is locally compact at $\eangle{q, w}$. Since points $q$ and $w$ are
arbitrary, we follow that the space $X_1 \times X_2$ is locally compact, which in turns
implies by induction that finite product of locally compact spaces is also compact.

Now let's emabark on a bit of a tangent: I want to define and prove some helpful things.

Firstly, let us define a notion of \textbf{reordering}.  Although it might have been proven before
somewhere in this text (or in the book itself), I can't recall encountering this notion before.
Let $\set{X_\alpha}$ be a set of topological spaces. Then reordering of $\prod{X_\alpha}$ is
the product of spaces $X_\alpha$ indexed under (maybe a) different index. This "(maybe a)"
insures that initial product is itself a reordering. As an example there are
spaces $X_1 \times X_2$ and $X_2 \times X_1$. We can follow pretty easily that those
two spaces are homeomorphic, and not only that, we can also prove by induction on the case
of two spaces that product of any finite family of spaces is homeomorphic to any of its
reorderings.
If $\set{X_\alpha}$ is an infinite family, then we can follow that under product, uniform,
and box topologies the reorderings of any given family are homeomorphic as well, which can 
be followed from definitions of reordering and respective topologies.
Some exposure to set theory lets us also follow that if $\set{X_\alpha}$ is indexed on a set $J$
(i.e. $\alpha \in J$ for arbitrary set $J$), then product of those spaces is
homeomorphic to $\set{X_\gamma}$ such
that $\gamma \in \Gamma$, where $\Gamma$ is a cardinal. Let us henseforth name such a reordering
a \textbf{cardinal reordering}. This will give us some nice well-orders and therefore
some nice notation, but otherwise will be pretty useless.

Second definition is a particular case of reordering: given that reorderings are
homeomorphic, we follow that we can define a notion of \textbf{homeomorphic product commutativity}.
Homeomorphic product commutativity is the fact that $X_1 \times X_2$ and $X_2 \times X_1$
are homeomorphic.

Lastly, I want to define a notion of \textbf{homeomorphic product associativity}. This
notion comes from the fact that strictly speaking, cartesian product is not associative.
Namely, with set-theoretic definitions we've got that 
$$(X_1 \times X_2) \times X_3 \neq X_1 \times (X_2 \times X_3)$$
but one can pretty easily follow that although those sets aren't equal, they are homeomorphic.
This notion can also be extended to product, uniform and box topologies, when we look
at the definitions of those topologies.

Basically the point of this tangent and all those definitions is simple: as long as the sets
are the same, we can mix and match them under product sign however we like, and the resulting
products will be homeomorphic.

Now we can proceed with our initial task: assume that $\set{X_\alpha}$ consists
of locally compact spaces, which are compact for all but the finite number of spaces.
We follow that we can reorder the given family and put all the non-compact spaces at the front.
Thus $\set{X_\beta}$ is a cardinal reordering of $\set{X_\alpha}$ such that there exists
$n \in \omega$ such that $X_m$ is locally compact if and only if $m \leq n$. We thus
can follow that product of $X_\beta$'s up to and including $n$ is locally compact by
the fact that the set of those spaces is finite. Tychonoff theorem in turn implies that
the product of the rest of the given family is also compact, and thus locally compact. This
in turn implise that
$$\prod_{j \leq n}{X_j} \times \prod_{\gamma > n}{X_\gamma}$$
is a product of two locally compact spaces and thus itself locally compact. 
Now homeomorphic product associativity and  implies that
$$\left(\prod_{j \leq n}{X_j}\right) \times \left(\prod_{\gamma > n}{X_\gamma}\right)$$
is homeomorphic to the original product $\prod{X_\alpha}$. One can prove pretty easily that
if some space is homeomorphic to a locally compact space, then it's locally compact. Thus
we conclude that $\prod{X_\alpha}$ is locally compact, as desired.

\subsection{}

\textit{Let $X$ be a locally compact space. If $f: X \to Y$ is continous, does it follow that
  $f[X]$ is locally compact? What if $f$ is both continous and open? Justify your answers.}

Let us firstly answer the last question: if $f$ is both continous and open, then we follow that
for some $y \in f[X]$ there exists $x \in X$ such that $f(x) = y$. Around $x$ there exists
a neighborhood $U$, for which we follow that $y \in f[U]$, and since $f$ is open, we follow that
$f[U]$ is a neighborhood of $y$. We also follow that for this particular neighborhood there
exists compact subspace  $C$ such that $U \subseteq C$, and since $f$ is continous, we follow that
$f[C]$ is also compact, thus for $y$ there exists a neighborhood $f[U]$, that is contained
in compact $f[C]$, which means that $f[X]$ is locally compact in $y$. Since $y$ is arbitrary,
we conclude that $f[X]$ is locally compact in general, as desired.

The problem with ordinaly (i.e. not necessarily open) $f: X \to Y$ comes from the fact
that there might not exist a neighborhood around a particular $y$ that will satisfy the desired
conditions.

We can follow that $\omega$, as any given totally ordered space is locally compact. We can also
follow that order topology on $\omega$ is the same as discrete topology. Since $Q$ is
infinitely countable, we follow that there's a bijection between $Q$ and $\omega$.
Let us denote this bijection by $h$. Since $\omega$ has discrete topology, we follow that
any given function (such as $h$) is continous. But we can follow that $h[\omega] = Q$ is
not locally compact, which proves that image of a locally compact space under plain
continuity is not necessarily locally compact, which proves my answer to the first question.

\subsection{}

\textit{Show that $[0, 1]^\omega$ is not locally compact in the uniform topology.}

We need to prove two things here: firstly that the closure of any given
basis element in uniform topology in this sense is equal to the the
product of the closed intervals. The second is that the product of the closed intervals is not
compact by the fact that it's not limit point compact, as proven before.

Let $W$ be a basis element in $[0, 1]^\omega$. We follow that there exists $\epsilon \in R$
such that $\epsilon > 0$ and $W = B(x, \epsilon) \cap [0, 1]^\omega$. From one of our
previous exercises we know that
$$W = \bigcup_{\delta < \epsilon}{U(x, \delta)}$$
where
$$U(x, \delta) = (x_1 - \delta, x_1 + \delta) \times (x_2 - \delta, x_2 + \delta) ...$$
Let
$$K = [x_1 - \epsilon, x_1 + \epsilon] \times [x_2 - \epsilon, x_2 + \epsilon] ...$$
we follow that if $y \in K$ and $J$ is a neighborhood of $y$, then there exists $\gamma \in R$
such thaht
$$J = \bigcup_{\delta < \gamma}{U(y, \delta)}$$
Let $0 < \alpha < \gamma$. We follow that $U(y, \alpha) \subseteq J$. Thus let us look at
$$U(y, \alpha) = \prod{(y_i - \alpha, y_i + \alpha)}$$
and
$$K = \prod{[x_i - \epsilon, x_i + \epsilon]}$$
Since $x_i - \epsilon \leq y_i \leq x_i + \epsilon$  we follow that $y_i - \alpha < x_i + \epsilon$
and $y_i + \alpha > x_i - \epsilon$ for all $i$'s. Moreover, we can follow that there
exists fixed $\theta \in R$ such that $0 < \theta < \epsilon$ (althought I don't provide
rigorous proof here, a simple picture can do wonders in this partuicular case) such that
$y_i - \alpha < x_i + \theta$ and $y_i + \alpha > x_i - \theta$. Since $i$ is arbitrary,
we follow that $U(y, \alpha)$ and $U(x, \theta)$ intersect, which in turn implies that
$U(y, \alpha)$ and $B(x, \epsilon)$ intersect by the virtue that
$U(x, \theta) \subseteq B(x, \epsilon)$, which in turns imply that $J$ and $B(x, \epsilon)$
intersect. Given that $y$ and $J$ are arbitrary we can conclude that
$K \subseteq \overline{B(x, \epsilon)}$.

Now assume that $q \notin K$. We follow that there exists $i \in \omega$ such that
$q_i < x_i - \epsilon$ or $q_i > x_i + \epsilon$. Assume the former. Then we follow that
there exsits a nonzero $\mu \in R$ such that $\mu < x - \epsilon - q_i$. Density of reals
and some basic implication will show that $B(q, \mu)$ will not intersect $B(x, \epsilon)$.
This now implies that $q \notin K \ra q \notin \overline{B(x, \epsilon)}$, which implies that
$K = \overline{B(x, \epsilon)}$, as desired. Thus we follow that
$$\overline{W} = \overline{B(x, \epsilon)} \cap [0, 1]^\omega = K \cap [0, 1]^\omega$$

Now let $x \in [0, 1]^\omega$. Assume that $[0, 1]^\omega$ is locally compact.
Since $[0, 1]^\omega$ is Hausdorff (as can be proven pretty easily,
if not already), we follow that given arbitrary neighborhood $U$
there must exist $\overline{B(x, \epsilon)}$ such that
$\overline{B(x, \epsilon)} \cap [0, 1]^\omega$ is compact. We follow that
$$\overline{B(x, \epsilon)} = \prod{[x_i - \epsilon, x_i + \epsilon]}$$
and thus
$$\overline{B(x, \epsilon)} \cap [0, 1]^\omega = \prod{[a_i, b_i]}$$
for some $a_i$'s and $b_i$'s such that $a_i \neq b_i$ for all $i$'s, as can be easily checked.
Moreover, we can follow that $|a_i - b_i| > \theta$ for some $\theta \in R_+$ and all $i$'s
on the account that closure contains $B(x, \epsilon)$
We now can define a sequence $f_i$ by
$$f_i(j) =
\begin{cases}
  j < i \ra a_i \\
  b_i \text{ otherwise}
\end{cases}
$$
and then follow that if $q \in f_i$, then by original definition of uniform topology we will
have that $B(q, \theta)$ (or some value less than $\theta$, which exsists by the density of reals)
will contain only $q$ itself, and given $q \notin f_i$, one can also prove (athough
thruogh a pretty tedious process) that there's a neighborhood of $q$ that does not intersect $f_i$
(this is the same idea as in the first exercise in the previous chapter). Thus we can conclude that
$$\overline{B(x, \epsilon)} \cap [0, 1]^\omega = \prod{[a_i, b_i]}$$
does not have limit points, and thus is not limit point compact, which implies that
given set is also not compact, which gives us a contradiction, as desired.

\subsection{}

\textit{if $f: X_1 \to X_2$ is a homeomorphism of locally compact Hausdorff spaces,
  show that $f$ extends to a homeomorphism of their one-point compactification.}


Let $f$ be such a homeomorphism. Let us denote one-point compactifications of $X_1$ and $X_2$
by $X_1'$ and $X_2'$ respectively, and by extent, let $f'$ be an extension of $f$ such that
it maps new point to a new point. We want to prove that $f'$ is a homeomorphism as well.

We firsly follow some obvious things: $f'$ is obviously a bijection, if $U \subseteq X_1$
or $U \subseteq X_2$, then we follow taht $f'\inv[U]$ or $f'[U]$ are open since original $f$ is
a homeomorphism.

Let us now denote $p \in X_1' \setminus X$. Let $U$ be an open set such that $p \in U$.
We follow that $U$ is not in $X_1$, and thus there's a compact subspace $F \subseteq X$ such
that $U = X_1' \setminus F$. We then follow that
$$f'[U] = f'[X_1 \setminus F] = f'[X_1] \setminus f'[F] = X_2' \setminus f'[F]$$
where we can derive all the preceding stuff by either basic properties of functions in general,
or basic properties of bijections in particular. 
Since $F \subseteq X_1$, we follow that $f'[F] = f[F]$, and since $F$ is compact and $f$
is continous, we follow that $f[F]$ is compact as well. Thus $f'[U] = X_2' \setminus f'[F]$
is a type 2 set of topology of one-point compactification of $X_2$, and thus it is itself open.

Since all open sets of $X_1'$ either have or don't have $p$ in them, we follow that $f'$
maps open sets to open, which implies that $f'\inv$ is continous. The same logic
can be applied to $f'\inv$ and $X_2'$, which means that $f'$ is a bijection, that is itself
continous and whose reverse functtion is also continous, which means that $f'$ is a homeomorphism,
as desired.

\subsection{}

\textit{Show that the one-point compactification of $R$ is homeomorphic with the circle $S^1$.}

Let $S' = S^1 \setminus \set{\eangle{0, 1}}$. We then follow that $S'$ is locally compact
Hausdorff space since it's a subspace of $R^2$, which is locally compact and Hausdorff. We
now want to prove that $S'$ is homeomorphic to $R$. We know that there's a homeomorphism
from $R$ to any given open interval, and we thus there's a homeomorphism to $(0, 2\pi)$.
There's a function $g: (0, 2\pi) \to S'$
$$g(x) = \eangle{\sin(x), \cos(x)}$$
which is continous, since $\sin$ and $\cos$ are continous. We can also follow from calculus that
this function is a bijection. Some geometry (or its variarion from calculus) can also give us that
a given function is an open map, which implies that $(0, 2\pi)$ and $S'$ are homeomorphic. Thus
$S'$ and $R$ are homeomorphic.

Since $S^1$ is compact we follow that $S^1$ is one-point
compactification of $S'$, and since there's a homeomorphism between $R$ and $S'$ and
both of those spaceas are locally compact Hausdorff, we follow by previous exercise that
one-point compactification of $R$ and $S^1$ are homeomorphic, as desired.

\subsection{}

\textit{Show that the one point compactification of $S_\Omega$ is homeomorphic
  with $\overline{S_\Omega}$}

We know that $S_\Omega$ is Hausdorff since it's got the order topology, and the same
applies to $\overline{S_\Omega}$. 
We also know that $S_\Omega$ is a subset of $\overline{S_\Omega}$ and $\overline{S_\Omega} \setminus
S_\Omega = \set{\Omega}$. If we can prove that $\overline{S_\Omega}$ is compact, then we can
follow that $S_\Omega$ is locally compact, in which case $\overline{S_\Omega}$ is homeomorphic
to the one-point compactification of $S_\Omega$. We can also prove that $S_\Omega$ is
locally compact, which will give us pretty much the same conclusion.

Since $S_\Omega$ is a woset, it is a toset, and thus it's simply ordered. We know that if $A$ is a
bounded subset of $S_\Omega$, then the set of upper bounds of $S_\Omega$ is a woset, thus we follow
that $A$ has a least upper bound. Thus we conclude that $S_\Omega$ is locally compact, as it was
proven to us in the chapter, and thus it's got a one-point compactification.

We can also follow (from 27.1) that $\overline{S_\Omega}$ is compact, and thus we conclude by
the first theorem in the chapter that $\overline{S_\Omega}$ is homeomoprhic with a one
point compactification of $S_\Omega$.

\chapter{Countability and Separation Axioms}

\section{The Countability Axioms}

\subsection{}

\textit{(a) A $G_\sigma$ set in a space $X$ is a set $A$ that equals a countable intersection
  of open sets of $X$. Show that in a first-countable $T_1$ space, every one-point set is a
  $G_\sigma$ set}

Assume that $Q$ is a first-countable $T_1$ space and let $q \in Q$. We follow that there's a
countable collection of open sets $B$ such that every neighborhood $U$ contains an element of $B$.
Since $B$ is a collection of neighborhoods of $q$ we conclude that
$q \in \bigcup{B}$. Now let $j \in \bigcap{B}$. We conclude that if $j \neq q$, then $\set{j}$
is closed and thus there's an open set $Q \setminus \set{j}$ that contains
an element of $B$ and thus $j \notin \bigcap{B}$, which gives us contradiction. Thus we
conclude that
$$\bigcap{B} = \set{q}$$
which proves that singletons are $G_\sigma$, as desired.

\textit{(b) There's a familiar space in which every one-point set is a $G_\sigma$ set, which
  nevertheless does not satisfy the first countability axiom. What is it?}

Since metrizability implies first countability, we need to look at some non-metrizable spaces.

$\overline{S_\Omega}$ won't do since every collection of basis neighborhoods of $\set{\Omega}$
got a point other than $\Omega$, which means that the sequence of those points have an upper
bound other than $\Omega$, which means that $\set{\Omega}$ isn't $G_\sigma$, which sucks.

The other option is a space $R^\omega$ in box topology. We follow that we can create a
collections of concentric balls (i.e. product of $U$'s with diameter $1/n$)
around an arbitrary $x \in R^\omega$, whose intersection
will be $x$, thus provind that singletons in $R^\omega$ in box topology are $G_\sigma$. We
can also follow the same thing from the fact that uniform topology is coarser than box
topology. Thus we need to prove that $R^\omega$ is not first countable (if it is, obviously).

We can follow that if $B$ is a set of countable neighborhoods of $x$, then if it's finite, then
intersection of those sets is an infinite open set, and since $R^\omega$ is Hausdorff,
we can create some open set, that does not contain none of the elements of $B$.
If $B$ is infinite however, then we can create a bijection $f$ from $\omega$ to $B$.
Then for each $n \in \omega$ we can take an open neighborhood $C_n$ of $x_n$ such that is
properly contained in $n$'th projection of $f(n)$. Then we can conclude that
$\prod{C_n}$ is a neighborhood of $x$, since for every $n \in \omega$ we've got that
$x_n \in C_n$ and that this neighborhood  does not contain any of $B$ because for every
$n \in \omega$ we've got that $n$'th projection of $B$ is  a proper superset of $C_n$, thus
proving that $\prod{C_n}$ does not contain $B$. Thus we conclude that for every countable
collection of neighborhoods $B$ of arbitrary $x \in X$ we've got that there exists a neighborhood
of $x$ that does not contain any of $B$'s, thus implying that $R^\omega$ in box topology isn't
first countable, as desired.

\subsection{}

\textit{Show that if $X$ has a countable basis $\set{B_n}$, then every basis $C$ for $X$ contains
  a countable basis for $X$.}

Let us try to use the hint and collect a set $C_{n, m}$ such that
$B_n \subseteq C_{n, m} \subseteq B_m$. If there's several sets in $C$ that are in between
two sets in $B$, then let us just pick one. Then we follow that $\set{C_{n, m}}$ is countable,
since it's indexed by a countable set (set of pairs of elements of $\omega$ is coubtable).
If the set $\set{C_{n, m}}$ happens to be a basis, then definition of $B_n$, together with
lemma 13.3 (and some omitted magic for the second clause of the theorem) will imply that
those two sets have same topologies.

Assume that it isn't and there's $x \in X$ such that
there are no $n, m \in \omega$ such that $B_n \subseteq C_{n, m} \subseteq B_m$ and
such that $x \in C_{n, m}$. Since $\set{B_n}$ is a basis, we conclude that there is
$n \in \omega$ such that $x \in B_n$. 
We then follow that $B_n$ is an open set, and thus there's a collection $Q$ of sets in $C$
such that $B_n = \bigcup{Q}$. Since $B_n = \bigcup{Q}$, we follow that there's a set $U \in Q$
such that $x \in U$, and since $U$ is an $Q$ and $Q$ is a subset of a basis we conclude that
$U$ is also an open set, which implies that there's a collection $D \subseteq \set{B_n}$ such that
$U = \bigcup{D}$. We then conclude that there's a set $B_m \in D$ such that $x \in B_m$,
which implies that $B_m \subseteq U \subseteq B_n$, and since $U$ is in $Q$, which is a subset
of $C$, we conclude that there exist an element of $C$ that satisfies the constrains
of $\set{C_{n, m}}$ and thus  there's a  $C_{m, n}$ such that $B_m \subseteq C_{m, n} \subseteq B_n$
for which $x \in B_m \subseteq C_{m, n}$, which gives us a contradiction.

Now suppose that $x \in X$ is such that there exist $m, n \in \omega$ such that
$x \in C_n \cap C_m$ (we've re-indexed the set $\set{C_{n, m}}$ here to simplify the notation a bit;
since we've proven already that the  set is  countable, we can index it however we like).
Since $C_n$ and $C_m$ are open, we conclude that there's a $k \in \omega$ such that
$x \in B_k$ and $B_k \subseteq C_n \cap C_m$. We then follow that there's a set $Q \in C$
such that $x \in Q \subseteq B_k$, and then we follow that there's a set $B_l$ such that 
$x \in B_l \subseteq Q \subseteq B_k$, thus there's $o \in \omega$ such that
$x \in B_l \subseteq C_o \subseteq B_k \subseteq C_m \cap C_n$, which simplifies to
$$x \in C_o \subseteq C_m \cap C_n$$
which gives us the second constraint of the basis, thus proving that $\set{C_{n, m}}$ is a basis.

\subsection{}

\textit{Let $X$ have a countable basis; let $A$ be an uncountable subset of $X$. Show that
  uncountably many points of $A$ are limit points of $A$.}

We firstly follow that since $A$ is a subset of $X$ that it is itself second-countable. Now we
need to prove that if $A$ is an uncountable second-countable space, then it's got an
uncountably many limit points. Since $A$ is second-countable, we follow that there exists
a countable subset $Q$ of $A$ that is dense in $A$. Since $A$ is uncountable, $Q$ is
countable and $Q \subseteq A$, we follow that $A \setminus Q$ is uncountable. We now wanna show
that every point of $A \setminus Q$ is a limit point of $A$. Let $j \in A \setminus Q$. We
follow that since $j \in A$ that $j \in \overline{Q}$. Thus we follow that every neighborhood of $j$
intersects $Q$ and since $j \notin Q$ and $Q \subseteq A$ we follow that every neighborhood
of $j$ intersects $A$ at some point other than $j$ itself. Thus we follow that a set of limit points
of $A$ has an uncountable subset, which implies that the set of limit points
of $A$ is uncountable as well. Therefore now we can conclude that every uncountable second-countable
spaces have uncountable set of limit points, and since $A$ is such a set, we follow the desired
conclusion.

\end{document}