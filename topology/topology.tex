\documentclass[11pt,oneside,titlepage]{book}
\title{My topology exercises}
\usepackage{amsmath, amssymb}
\usepackage{geometry}
\usepackage{hyperref}
\author{Evgeny Markin}
\date{2023}

\DeclareMathOperator \map {\mathcal {L}}
\DeclareMathOperator \pow {\mathcal {P}}
\DeclareMathOperator \topol {\mathcal {T}}
\DeclareMathOperator \basis {\mathcal {B}}
\DeclareMathOperator \ns {null}
\DeclareMathOperator \range {range}
\DeclareMathOperator \fld {fld}
\DeclareMathOperator \inv {^{-1}}
\DeclareMathOperator \Span {span}
\DeclareMathOperator \lra {\Leftrightarrow}
\DeclareMathOperator \eqv {\Leftrightarrow}
\DeclareMathOperator \la {\Leftarrow}
\DeclareMathOperator \ra {\Rightarrow}
\DeclareMathOperator \imp {\Rightarrow}
\DeclareMathOperator \true {true}
\DeclareMathOperator \false {false}
\DeclareMathOperator \dom {dom}
\DeclareMathOperator \ran {ran}
\newcommand{\eangle}[1]{\langle #1 \rangle}
\newcommand{\set}[1]{\{ #1 \}}



\begin{document}
\maketitle
\tableofcontents

\chapter*{Preface}

Those are my solutions for the James Munkres' "Topology", 2nd edition.

Majority of the notation that is used here migrated from my course on the set theory. In
my very personal opinion, notation that is used there is far superior that whatever is
happening in Munkres' book. Sometimes I use some abusive notation when it is painfully
clear what's going on.

If you decide to persue the study of topology yourself, then I highly recommend firstly
to go through a course on axiomatic set theory and logic, because first chapter of this
book is highly insufficient in this regard. My personal recommendations are the
combo by Cunningham, which includes "Set theory: A first course" and
"A Logical Introduction to Proof", or 
"A first course in Mathematical Logic and Set Theory" by Michael L. O’Leary for both subjects.

\part{General Topology}

\chapter{Set Theory and Logic}


\section{Fundamental Concepts}

\subsection{}

\textit{Check distributive and DML laws}

\textit{GOTO set theory book}

\subsection{}

\textit{Determine which of the following are true.}

(a) - impl

(b) - impl

(c) - true

(d) - rimpl

(e) - $\subseteq$, true if $B \subseteq A$.

(f) - $\supseteq$;   $A - (B - A) = A$.

(g) - true

(h) - $\supseteq$

(i) - true

(j) - true

(k) - false

(l) - true

(m) - $\subseteq$

(n) - true

(o) - true

(p) - true

(q) - $\supseteq$

\subsection{}

\textit{(a) Write a contrapositive and converse of the following statement:
  "If $x < 0$, then $x^2 - x > 0$" and determine which ones are true}

Contrapositive:
$$x^2 - x \leq 0 \ra x \geq 0$$
Converse
$$x^2 - x > 0 \ra x < 0$$

Contrapositive is correct, converse is incorrect ($2^2 - 2 > 0$)

\textit{(b) Do the same for the statement $x > 0 \ra x^2 - x > 0$}

Contrapositive:
$$x^2 - x \leq 0 \ra x \leq 0$$
Converse
$$ x^2 - x > 0 \ra x > 0 $$

Contrapositive is false ($1^2 - 1 = 0$); Converse is also false ($(-2)^2 - (-2) = 6$).

\subsection{}

\textit{Let $A$ and $B$ be the sets of real numbers. Write the negation of each of the
  following statements: }

\textit{(a)}
$$ (\exists a \in A)(a^2 \notin B)$$
\textit{(b)}
$$ (\forall a \in A)(a^2 \notin B)$$
\textit{(c)}
$$ (\exists a \in A)(a^2 \in B)$$
\textit{(d) }
$$ (\forall a)(a \notin A \ra a^2 \notin B)$$

\subsection{}

\textit{Let $A$ be a nonempty collection of sets. Determine the truths of each of the
  following and their converses}

\textit{(a)
$$x \in \bigcup{A} \lra (\exists B \in A)(x \in B)$$}
\textit{(b)
$$x \in \bigcup{A} \la (\forall B \in A)(x \in B)$$}
\textit{(c)
$$x \in \bigcap{A} \ra (\exists B \in A)(x \in B)$$}
\textit{(d)
$$x \in \bigcap{A} \lra (\forall B \in A)(x \in B)$$}

\subsection{}

Skip

\subsection{}

skip

\subsection{}

GOTO set theory book

\subsection{}

\textit{Formulate DML for arbitrary unions and intersections}

$$A \setminus \bigcap{(B)} = \bigcup{(A \setminus B)} $$
$$A \setminus \bigcup{(B)} = \bigcap{(A \setminus B)} $$

For the proof goto set theory or real analisys book

\subsection{}

(a, b, d) are true

\section{Functions}

\subsection{}

\textit{Let $f: A \to B$. Let $A_0 \subseteq A$ and $B_0 \subseteq B$.}

\textit{(a) Show that $A_0 \subseteq f\inv[f[A_0]]$ and that equality holds
  if $f$ is injective.}

Suppose that $x \in A_0$. We follow that there exists $\eangle{x, y} \in f$ for some
$y \in f[A_0]$. Therefore there exists $\eangle{y, x} \in f\inv$. Because $y \in f[A_0]$,
we follow that $x \in f\inv[f[A_0]]$. Therefore $A_0 \subseteq f\inv[f[A_0]]$.

Suppose that $f$ is injective. Suppose that there exists $x_0 \in f\inv[f[A_0]]$ such that
$x_0 \notin A_0$. We follow that $\eangle{y, x_0}, \eangle{y, x}, \in f\inv$,
therefore $\eangle{x_0, y}, \eangle{x, y} \in f$, and because $x_0 \neq x$ we follow
that we've got a contradiction.

\textit{((b) }

pretty simular to $(a)$

\textit{This chapter practicly mirrors the content of my set theory course
  . Gonna skip it for now, and will come back if the need arises.}

\chapter{Topological Spaces and Continous Functions}

\section{Topological Spaces}

I want to state here that if $\topol \subseteq \pow(X)$ satisfies
properties
$$\set{X, \emptyset} \subseteq \topol$$
$$(\forall Y \in \pow(\topol))( \bigcup{U} \in \topol)$$
$$(\forall Y \in \pow(\topol))(Y \neq \emptyset \land |Y| <_c
|\omega|  \to \bigcap{U} \in \topol)$$
then $\topol$ is a topology on $X$.

\section{Basis for a Topology}

Let $Y \subseteq \pow(X)$. If
$$(\forall x \in X)(\exists y \in Y)(x \in y)$$
and
$$(\forall x \in X)(\exists y_1, y_2, y_3 \in Y)(x \in y_1 \cap y_3 \to
x \in y_3 \land y_3 \subseteq y_1 \cap y_2)$$
then $Y$ is a basis for a topology on $X$.

\subsection{}

\textit{Let $X$ be a topological space; Let $A$ be a subset of $X$. Suppose that for each
  $x \in A$ there is an open set $U$ containing $x$ such that $U \subseteq A$. Show that $A$ is
  open in $X$.}

Let $U: A \to \pow(A)$ be an indexed function such that 
$$x \in U(x) \land U(x) \subseteq A \land U(x) \in \topol(X)$$
We want to show that $A = \bigcup{\ran(U)}$. Suppose that $x \in A$. We follow that
$x \in U(x)$. Thus $x \in \bigcup{\ran(U)}$. Therefore $A \subseteq \bigcup{\ran(U)}$.

Suppose that $z \in \bigcup{\ran(U)}$. We follow that
$$(\exists Y \in \ran(U))(z \in Y) \ra
(\exists x \in A)(z \in U(x))$$
Since $(\forall x \in A)(U(x) \subseteq A)$, we follow that $z \in A$. Thus
$\bigcup{\ran(U)} = A$.

Because $(\forall x \in A)(U(x) \in \topol(X))$, we follow that
$$\ran(U) \subseteq \topol(A)$$, therefore by definition of topology we follow that
$$\bigcup{\ran(U)} \in \topol(X)$$
as desired.

\subsection{}

Too tedious, skip

\subsection{}

\textit{Show that the collection $\topol_c$ given in Example 4 of p. 12 is a topology on the
  set $X$. Is the collection
  $$\topol_\infty = \{U \in \pow(X):
  |X \setminus U| \geq_c |\omega| \lor X \setminus U = \emptyset \lor
  X \setminus U = X\}$$
  a topology on $X$?
}

We firstly state that
$$\topol_c = \{U \in \pow(X): |X \setminus U| \leq_c |\omega| \lor X \setminus U = X\}$$

We can follow that $X \setminus X = \emptyset$, which is countable, thus $X \in \topol_c$.
$X \setminus \emptyset = X$, therefore $\emptyset \in \topol_c$.

Suppose that $U' \subseteq \topol_c$. If $U' = \{\emptyset\}$, then $
X \setminus \bigcap{U'} = X$ and $X \setminus \bigcup{U'} = X$.
Thus assume that $U' \neq \{\emptyset\}$.

We follow that
$$(\forall u \in U')(|X \setminus u| \leq_c |\omega| \lor X \setminus u = X)$$
We follow that if $\emptyset \in U'$, then $\bigcup{U'} = \bigcup{(U' \setminus \{\emptyset\})}$.
Then we follow by DML that
$$X \setminus \bigcup\{U'\} = X \setminus \bigcup\{U' \setminus \{\emptyset\}\} =
\bigcap_{U' \setminus \{\emptyset\}}{X \setminus u}$$
we know that $(\forall u \in U')(|X \setminus u| \leq_c |\omega|)$. For any $u \in U'$ we
follow that
$$\bigcap_{u \in U' \setminus \{\emptyset\}}{X \setminus u} \subseteq X \setminus u'$$
and given that $X \setminus u'$ is countable, we follow that $\bigcap_{u \in U'}{X \setminus u}$
is countable as well, thus $\bigcup{U'} \in \topol_c$.

Now let $U' \subseteq \topol_c$ and $|U'| < |\omega|$ and $U' \neq \{\emptyset\}$.
We follow that if $\emptyset \in U'$, then $\bigcap{U'} = \emptyset$, and therefore
$X \setminus \bigcap{U'} = X$. Therefore assume that $\emptyset \notin U'$.

Then we can follow that
$$X \setminus \bigcap{U'} = \bigcup_{u \in U'}{X \setminus u}$$
Given that $U'$ is countable and $X \setminus u$ is countable we follow that
$\bigcup_{u \in U'}{X \setminus u}$ is countable, thus $X \setminus \bigcap{U'}$ is countable.

Therefore we conclude that $\topol_c$ is a topology on $X$.

Now let us consider $T_\infty$. We can state that $X \in T_\infty$ because
$X \setminus X = \emptyset$. Because $X \setminus \emptyset = X$, we follow that
$\emptyset \in T_\infty$.

Suppose that $X$ is not infinite and $T_\infty \neq \{\emptyset, X\}$. Then there exists
$u \in T_\infty$ such that $u \neq \emptyset$ and $u \neq X$. Therefore $X - u$ is
nonempty finite set, therefore $u \notin T_\infty$, which is a contradiction.
Therefore we conclude that if $X$ is finite, then $T_\infty$ is a trivial topology.

If $X$ is infinite, then we follow that we can have an injection $f: \omega \to X$.
Let $O$ be the set of odd naturals and $E$ be the set of evens. Then we follow that
$$|X \setminus f[O]| = |f[E]| \geq_c |\omega|$$
and
$$|X \setminus f[E]| =_c |f[O]| \geq_c |\omega|$$
which tells us that $f[O]$ and $f[E]$ are both in $X$. We can also follow that
$$|X \setminus f[O \cup \{0\}]| \geq |\omega|$$
thus $ f[O \cup \{0\}] \in \topol_\infty$. This gives us that
$$f[E] \cap f[O \cup \{0\}] = \{f(0)\} \in \topol_\infty$$
but $\{f(0)\}$ is a finite nonempty set for which none of the conditions of $\topol_\infty$
hold. Therefore we conclude that if $X$ is infinite, then $\topol_\infty$ is not a topology.

Therefore we conclude that if $X$ is a finite set, then $T_\infty$ is equal to a
trivial topology; if $X$ is infinite, then $T_\infty$ is not a topology at all, since
it is not closed under finite intersections.

\subsection{}

\textit{(a) if $\{\topol_\alpha\}$ is a family of topologies on $X$, show that
  $\bigcap{\topol_\alpha}$ is a topology on $X$. Is $\bigcup{\topol_\alpha}$ a topology on $X$?}

Since every topology on $X$ has $X$ and $\emptyset$ as elements, we follow that
$$\{X, \emptyset\} \subseteq \bigcap{\topol_\alpha}$$
If $Y \subseteq \bigcap{\topol_\alpha}$, then we follow that
$$(\forall Z \in \{\topol_\alpha\})(\bigcap{\topol_\alpha} \subseteq Z)$$
$$(\forall Z \in \{\topol_\alpha\})(Y \subseteq Z)$$
since every $Z$ is a topology, we follow that
$$(\forall Z \in \{\topol_\alpha\})(\bigcup{Y} \in Z)$$
$$\bigcup{Y} \in \bigcap{\topol_\alpha}$$
If $Y$ is finite and nonempty, we can also follow that
$$(\forall Z \in \{\topol_\alpha\})(Y \in Z) \ra
(\forall Z \in \{\topol_\alpha\})(\bigcap{Y} \in Z) \ra \bigcap{Y} \in \bigcap{\topol_\alpha}$$
thus we conclude that $\bigcap{\topol_\alpha}$ is a topology.

$\bigcup{\topol_\alpha}$ is not necessarily a topology. Although
$\set{X, \emptyset} \in \bigcup{\topol_\alpha}$, we cannot follow that the topology is
closed under unions. Case in point: Let $X = \set{a, b, c}$ and
$$\topol_1 = \set{\emptyset, X, \set{a}}, \topol_1 = \set{\emptyset, X, \set{b}}$$
then $Y = \topol_1 \cup \topol_2$ does not contain $\set{a, b}$, which would be necessary
for this case. Thus we conclude that in general we can't have implications for
$\bigcup{\topol_\alpha}$.

\textit{(b) Let $\set{\topol_\alpha}$ be a family of topologies on $X$. Show that there is a
  unique smallest topology on $X$ containing all the collections $\topol_\alpha$ and
  a unique largest topology contained in all $\topol_\alpha$.}

Let us take $\bigcup{\set{\topol_\alpha}}$. We cannot follow that presented
set is a topology on $X$, nor can we state that it is a basis of a topology. Former
is followed from the discussion in the previous section of this exercise, and the latter
cannot be followed because we don't necessarily satisfy the
second point of the definition of the basis. Namely, we don't have that
$$(\forall x \in X)(\exists y_1, y_2, y_3 \in \bigcup{\set{\topol_\alpha}})(x \in y_1 \cap y_3 \to
x \in y_3 \land y_3 \subseteq y_1 \cap y_2)$$
Let $Q$ be a set of all of the intersections of finite nonempty subsets of
$\bigcup{\set{\topol_\alpha}}$. We follow that $(\forall x \in \bigcup{\set{\topol_\alpha}})
(x = \bigcap{\{x\}})$, therefore $\bigcup{\set{\topol_\alpha}} \subseteq Q$. 
Thus we follow that $Q$ satisfies
the first requirement for the basis of $X$. Now let $x \in X$ be such that there
exist $y_1, y_2 \in Q$ such that $x \in y_1 \cap y_2$. We follow that there exist
finite subsets $Y_1, Y_2 \subseteq \bigcup{\set{\topol_\alpha}}$ such that 
$$y_1 = \bigcap{Y_1} \land y_2 = \bigcap{Y_2}$$
therefore
$$y_1 \cap y_2  = \bigcap{Y_1} \cap \bigcap{Y_2}$$
which is an intersection of a finite subset of $\bigcup{\topol_\alpha}$. Thus we follow that there
exists $y_3 \in Q$ such that $x \in y_3 \land y_3 \subseteq y_1 \cap y_2$. 
Therefore we can follow that the set $Q$ is indeed a basis for a topology on $X$.
Let us name the topology generated by this set as $\topol_q$.

Suppose that there is a topology, which contains
all of the topologies $\set{\topol_\alpha}$. Then we follow that it contains
$\bigcup{\set{\topol_\alpha}}$, therefore we follow that it contains all of the unions
of $\bigcup{\set{\topol_\alpha}}$, and finite intersections of subsets of
$\bigcup{\set{\topol_\alpha}}$, and thus it contains $\topol_q$. Therefore
we follow that $\topol_q$ is the smallest topology, which contains all
the topologies of $\set{\topol_\alpha}$.

Suppose that $\topol_p$ is a topology, which is contained in all of the $\set{\topol_\alpha}$.
Then we follow that $\topol_p \subseteq \bigcap{\topol_\alpha}$. Because $\bigcap{\topol_\alpha}$
is a topology itself, we follow that it is the largest topology, which is contained
in all of the $\set{\topol_\alpha}$.

\textit{(c) If $X = \set{a, b, c}$, let
  $$\topol_1 = \set{\emptyset, X, \set{a}, \set{a, b}}$$
  $$\topol_2 = \set{\emptyset, X, \set{a}, \set{b, c}}$$
  Find the smallest topology containing $\topol_1$ and $\topol_2$, and the largest topology
  contained in $\topol_1, \topol_2$.
}

We can follow from previous discussions that largest contained topology is
$$\set{\emptyset, X, \set{a}}$$
and the smallest containing topology is
$$\set{\emptyset, X, \set{a}, \set{b}, \set{a, b}, \set{b, c}}$$

\subsection{}

\textit{Show that if $A$ is a basis for a topology on $X$, then the topology generated by $A$
  equals the intersection of all topologies on $X$ that contains $A$. Prove the same
  if $A$ is a subbasis.}

Let $A$ be a subbasis.
Let $\set{\topol_\alpha}$ be a set of topologies, that contain $A$ and  $\topol_A$ is
a topology generated by $A$. We can follow that $\topol_A \in \set{\topol_\alpha}$,
therefore $\bigcap{\set{\topol_\alpha}} \subseteq {\topol_A}$. If $x \in \topol_A$, then we
follow that there exists a subset $B \subseteq A$ such that $x$ is equal to some
union of some finite intersections of $B$. Since
$B \subseteq A$, we follow that $(\forall y \in \topol_\alpha)(B \subseteq y)$. Therefore
all of the finite intersections of $B$ are in any topology of $\topol_\alpha$.  Therefore
all of the unions of those intersections are in any $\topol_\alpha$. Therefore
we conclude that $(\forall y \in \topol_\alpha)(x \in y)$.
and thus $x \in \bigcap{\topol_\alpha}$.
Therefore we conclude that $\topol_A \subseteq  \bigcap{\topol_\alpha}$, and by double
inclusion we get that $\topol_A =  \bigcap{\topol_\alpha}$, as desired.

Since every basis of a topology is a subbasis by first clause of the definition, we follow
that the desired result holds for bases as well.

\subsection{}

\textit{Show that the topologies of $R_l$ and $R_k$ are not comparable.}

Let $[0, 1)$ be an element of a basis of topology $R_l$. Then we follow that
there are no elements of basis of standart topology on $R$ that contains $0$ and lies inside
$[0, 1)$. We can follow this by contradiction

Suppose that $0 \in (x, y)$ and $(x, y) \subseteq [0, 1)$. Since $0 \in (x, y) $,
we follow that $x < 0$. Thus we conclude that there exists $n \in Z_+$ such that
$1/n < |x|$. Therefore $-1/n \in (x, y)$ and $-1/n \notin [0, 1)$ which gives us
that $(x, y) \not \subseteq [0, 1)$, which is a contradiction.
The same logic applies to any element of basis of $R_k$.

Now let us look at the basis element $(-1, 1) \setminus K$ and the point $0$. We can
follow that $0 \in (-1, 1) \setminus K$ and suppose that there exists basis element of
$R_l$  $[a, b)$ that has point $0$ and is contained within $(-1, 1) \setminus K$.
Since $0 \in [a, b)$, we follow that $a \leq 0 < b$. Thus we conclude that there exists
$n \in Z_+$ such that $0 < 1/n < b$. Thus we conclude that $1/n \in [a, b)$ and
$1/n \notin (-1, 1) \setminus K$, since $1/n \in K$ for all $n \in Z_+$. Thus we
conclude that $R_k$ and $R_l$ are not comparable, as desired.

\subsection{}

\textit{Consider the following topologies on $R$:
  $$\topol_1 = \text{the standart topology on $R$}$$
  $$\topol_2 = \text{the topology of $R_k$}$$
  $$\topol_3 = \text{the finite complement topology}$$
  $$\topol_4 = \text{the upper limit topology, having all sets $(a, b]$ as basis}$$
  $$\topol_5 = \text{the topology having all sets
    $(-\infty, a) = \set{x: x < a}$ as a basis}$$
    Determine, for each of these topologies, which of the others it contains
}

We can follow that $T_2$ contains $T_1$, since it's finer, as proven in the chapter. The
reverse is not true, as proven in the chapter.

We can follow that $T_3$ does not contain $T_1$, because if it is, then we follow that
$(-\infty, a] \cup [b, \infty)$ has finite number of points. The revese is true, since
we can divide each element of a finite complement into a union of
open intervals. For example, if $x \in T_3$ is such that $x = R \setminus \set{x_1, x_2, x_3}$
and $x_1 < x_2 < x_3$,
then we can state that $x = (-\infty, x_1) \cup (x_1, x_2) \cup (x_2, x_3) \cup (x_3, \infty)$.
We can follow that middle 2 intervals are in the basis of standart topology, and two infinite
intervals are unions of infinite set of intervals of basis.
Thus $\topol_1$ is strictly finer than $\topol_3$.

We can follow that the same logic, that worked with lower limit, works with upper limit as well.
thus we conclude that $T_4$ is strictly finer than $T_1$.

We can follow that for $(-\infty, a) \in T_5$ we can get a sequence $(x_n) = a - n$, then
get a set of intervals $\set{(a, a - 1), (x_{n + 1}, x_n)}$, all of which are in the basis of
standart topology, get another set$\set{V_{0.1}(x_n)}$ to path the holes in this set,
and take union of unions of both sets to get that $(-\infty, a) \in T_1$.

For $(a, b)$ - a set in the basis of standard topology we follow that every set in the
basis of $T_5$ contains $a - 1$, thus we conclude that $(a, b) \notin T_5$. Thus we
conclude that $T_1$ is strictly finer than $T_5$.

Topology $\topol_2$ is strictly finer than $\topol_1$, therefore we follow that
topologies that are finer than $\topol_1$ are a subset of $\topol_2$. This includes
$\topol_3$ and $\topol_5$. (Almost) the same reasoning that worked with $R_k$ and $R_l$
can be applied to show that $\topol_2$ is not finer than $\topol_4$. On the other hand,
suppose that $x \in X$ and $y \in \topol_2$ is such that $x \in y$. We follow that
if $y \in \topol_1$, then there exists an element of $\topol_4$ that is finer than
$y$. Thus assume taht $y \notin \topol_1$ and therefore is in the form
$y = (a, b) \setminus K$ for some $a, b \in R$. If $x \leq 0$, then we can have
set $(a, x] \subseteq y$ that will satisfy. Thus assume that $x > 0$. We follow that there
exists $n \in Z_+$ such that $1/n < x$. By well-ordering properties of $Z_+$ we
follow that there exists lowest $n \in Z_+$ such that $1/n  x$. Therefore we follow that
there are no elements $z \in K$ such that $1/n < z < x$. Since $x \in (a, b) \setminus K$,
we follow that $x \notin K$, therefore
$(\forall y \in (1/n, x])(y \in x \in (a, b) \setminus K)$. Therefore we conclude that
$\topol_4$ is strictly finer than $\topol_2$, which is neat.

$\topol_3$ is strictly coarser than $\topol_1$, $\topol_2$. Since $\topol_4$
is strictly finer than $\topol_2$, we follow that $\topol_3$ is coarser than $\topol_4$.
Suppose that $a < x < b$ and let $y = R \setminus \set{a, b}$ be an element of $\topol_3$.
Then we follow that no element of basis of $\topol_5$ has $x$ and does not have $a$.
If $(-\infty, a)$ is an element of $\topol_5$, then we follow that every element of topology
$\topol_3$ has numbers greater than $a$ in it (since there are infinitly  many of them).
Thus we conclude that no element of $\topol_3$ is a subset of $(-\infty, a)$. Thus we
conclude that $\topol_3$ and $\topol_5$ are not comparable.

And after all of the discussion, we can conclude that
$$[\topol_3 | \topol_5] \subset \topol_1 \subset \topol_2 \subset \topol_4$$
is the desired conclusion.

\subsection{}

\textit{(a) Apply Lemma 13.2 to show tha the countable collection
  $$B = \set{(a, b): a < b \land a, b \in Q}$$
  is a basis that generates the standard topology on $R$.}

Denote $\topol$ as a standard topology on $R$. Let $x \in \topol$. We follow that
there exists an interval $(a, b)$ in basis of standard topology such that
$x \in (a, b)$. We can follow that there exist $a', b' \in Q$ such that
$a < a' < x < b' < b$ (otherwise we run into some problem with density of rationals
in reals). Therefore we follow that $x \in (a', b')$. Lemma 13.2 tells us that
the presented result implies that $B$ is a basis for standard topology, as desired.

\textit{(b) Show that the collection
  $$C = \set{[a, b): a < b \land a, b \in Q}$$
  is a basis that genenrates a topology different from the lower limit topology on $R$.}

Proof that $C$ is a basis is trivial.
Let us look at $[\sqrt{2}, 2)$ - an element of $R_l$. Suppose that $c = [a, b) \in C$ is
such that $\sqrt{2} \in c$. Because $\sqrt{2} \notin Q$, we follow that
$a \neq \sqrt{2}$, therefore $a < \sqrt{2} < b$. Therefore we can conclude that $C$ is
not finer than $R_l$. Proving that $C$ is a subset of $R_l$ is trivial, thus
we conclude that $R_l$ is strictly finer than $C$, and thus $C$ generates a topology
different than $R_l$, as desired.

\section{The Order Topology}

\section{The Product Topology on $X \times Y$}

\section{The Subspace Topology}

\subsection{}

\textit{Show that if $Y$ is a subspace of $X$, and $A$ is a subspace of $Y$, then the topology
  $A$ inherits as a subspace of $Y$ is the same as the topology it
  inherits as a subspace of $X$.}



Suppose that $Q$ is an open set in $A$ with respect to topology, inherited from $X$.
We follow that there exists an open set in $X$  $Q_x \subseteq X$ such that $Q = Q_x \cap A$
by definition of a subspace topology. We follow that there exists open in $Y$ set $Q_y \subseteq Y$
such that $Q_y = Q_x \cap Y$. With respect to $Q_y$ there exists an open in $A$ set
$Q' = Q_y \cap A$. Thus
$$Q' = Q_y \cap A$$
$$Q' = Q_x \cap Y \cap A$$
Since $A \subseteq Y$, we follow that $Y \cap A = A$. THus
$$Q' = Q_x \cap (Y \cap A)$$
$$Q' = Q_x \cap A$$
$$Q' = Q$$
Therefore we conclude that if $Q$ is in topology of $A$ inherited from $X$, then $Q$ is also
in a topology of $A$ inherited from $Y$. Proof of the converse is pretty much the same
proof

Here's another, more logical and rigorous proof.
Denote topology of $A$ inherited from $Y$ by $\topol_A$ and topology of $A$ inherited from $X$
by $\topol_A'$. Also denote topology of $X$ by $\topol_X$ and topology of $Y$ inherited from $X$
by $\topol_Y$. Then we can state that 
$$Q \in \topol_A \lra (\exists Q_y \in \topol_Y)(Q = Q_y \cap A) \lra
(\exists Q_X \in \topol_X)(Q_y = Q_x \cap Y \land Q = Q_y \cap A) \lra$$
$$ \lra 
(\exists Q_X \in \topol_X)(Q = Q_x \cap Y  \cap A) \lra
(\exists Q_X \in \topol_X)(Q = Q_x \cap (Y  \cap A)) \lra
$$
$$ \lra 
(\exists Q_X \in \topol_X)(Q = Q_x \cap A) \lra
Q \in \topol_A'$$
thus $\topol_A' = \topol_A$ by extensionality axiom.

\subsection{}

\textit{if $\topol$ and $\topol'$ are topologies on $X$ and $\topol'$ is strictly finer
  than $\topol$, what can you cay about the corresponding topologies on the subset $Y$ of $X$.}

Denote corresponding topologies by $\topol_Y'$ and $\topol_Y$.
There're three plausible cases:

1 - we can't say nothing

2 - $\topol_Y' \supset \topol_Y$

3 - $\topol_Y' \supseteq \topol_Y$

I'm betting on the second case, so let us try to prove that. In order to do that, let us firstly
prove the third case, which is a "subcase" of the second.

Suppose that $Q \in \topol_Y$. We follow that there exists $Q_X \in \topol$ such that
$Q = Q_X \cap Y$. Since $Q_X \in \topol$, we follow by $\topol \subset \topol'$ that
$Q_x \in \topol'$. Thus $Q = Q_X \cap Y$ implies that $Q \in \topol_Y'$. Therefore we follow that
$\topol_Y' \supseteq \topol_Y$.

Although I'm betting on the second case, it seems that I'm not getting my money back.
We can follow that second case is not always true, if we substitute $\emptyset$ for $Y$.
Then $\topol_Y = \topol_Y' = \emptyset$. If we look into topologies of some
almost-trivial set, such as $X = \set{a, b, c}$, then I think that we can come up with a more
persuasive case as well. Therefore we conclude that
presented conditions imply that $\topol_Y \subseteq \topol_Y'$.

\subsection{}

\textit{Considet the set $Y = [1, 1]$ as a subspace of $R$. Which of the following sets are
  open in $Y$? Which are open in $R$?
  $$A = \set{x : \frac{1}{2} < |x| < 1 }$$
  $$B = \set{x : \frac{1}{2} < |x| \leq 1 }$$
  $$C = \set{x : \frac{1}{2} \leq |x| < 1 }$$
  $$D = \set{x : \frac{1}{2} \leq |x| \leq 1 }$$
  $$E = \set{x : 0 < |x| < 1 \land 1/x \notin Z_+ }$$
}

We can follow that
$A = (-1, -1/2) \cup (1/2, 1)$
is open in both $Y$ and $R$.

$B = [-1, -1/2) \cup (1/2, 1]$
is a union of two rays in $Y$, therefore we follow that it is open in $Y$. For $R$ we've got that
there is no open interval, that contains a point $1$ and does not contain anything larger
then $1$. Therefore we conclude that given set is not a union of open intervals, and therefore
it is not open in $R$.

We can follow pretty easily that $C$ and $D$ are not open in both $Y$ and $R$ since there is
no open interval/ray that contains $1/2$ and does not contain anything in the interval
$(-1/2, 1/2)$.

We can represent $E$ as
$$E = (-1, 0) \cup ((0, 1) \setminus K)$$
We follow that $(-1, 0)$ is an element of a basis of both $Y$ and $R$. Suppose
that $x \in (0, 1) \setminus K$. Then we follow that there exist lowest $n_1 \in Z_+$ such that
$1/n_1 < x < 1/(n_1 + 1)$. Therefore we can conclude that if $x \in E$, then there exist
a basis element $Q$ of both $Y$ and $R$ such that $x \in Q \subseteq Y, R$. Therefore
we follow that $E$ is an open set in both $Y$ and $R$.

\subsection{}

\textit{A map $f: X \to Y$ is said to be an open map if for every open set $U$ of $X$, the
  set $f(U)$ is open in $Y$. Show that $\pi_1: X \times Y \to X$ and
  $\pi_2 : X \times Y \to Y$ are open maps.}

Suppose that $Q \in X \times Y$ is an open set. Therefore we follow that it is a union of
some element of a basis of $X \times Y$, therefore there exist a subset $R$ of a basis of
$X \times Y$ such that $Q = \bigcap{R}$. From a set theory course we know that
$$U[\bigcup{G}] = \bigcup{\set{R[C]: C \in G}}$$
for any relation $U$. Therefore we can follow that the same result holds for functions
$\pi_1, \pi_2$. We can follow that for any $r \in R$ we've got that both $\pi_1(r)$ and
$\pi_2(r)$ are open by the definition of a basis for the product topology. Therefore we
conclude that $\pi_1[Q] = \pi_1[\bigcup{R}] = \bigcup{\set{\pi_1[\bigcup{r}]: r \in R}}$.
Therefore we conclude that $\pi_1[Q]$ is a union of open sets of $X$, therefore we conclude that
it is in topology of $X$. We can follow the simular result for $\pi_2$ using simular logic.

\subsection{}

\textit{Let $X$ and $X'$ denote a single set in the topologies $\topol$ and $\topol'$
  respectively; let $Y$ and $Y'$ denote a single set in the topologies $U$ and $U'$,
  respectively. Asuume that these sets are nonempty. }

There're a couple of ways to deconstruct the text of this exercise: 
we can assume that  $X = X'$,
$Y = Y'$, $X \in \topol$, $X' \in \topol'$, $Y \in U$ and $Y' \in U'$,
or we can assume that $X \in \topol$, $X' \in \topol'$, $Y \in U$ and $Y' \in U'$
without $X = X'$ and $Y = Y'$. The latter case will obviously present some problems
in the proofs, therefore we will assume that the author intended to use the former case.

\textit{(a) Show that if $\topol' \supseteq \topol$ and $U' \supseteq U$, then the product
  topology on $X' \times Y'$ is finer than the product topology on $X \times Y$}

Let $\basis$ denote the basis for $\topol_{X \times Y}$. Let $q \in X \times Y$.
Because $\basis$ is a basis for $\topol_{X \times Y}$ we follow that there exists
$b \in \basis$ such that $q \in b$. Since $b \in \basis$, we follow that there
exist $x \in \topol$ and $y \in U$ such thaht $b = x \times y$.
Since $\topol \subseteq \topol'$ and $U \subseteq U'$,
we follow that $x \in \topol'$ and $y \in U'$. Therefore $x \times y \in \basis'$
where $\basis'$ denotes the basis for $\topol_{X' \times Y'}$. Therefore we conclude that
for every $x \in X \times Y$ and every basis element $q \in \basis$ there exists
$q' \in \basis'$ such that $q' \subseteq q$ and $x \in q'$. Therefore we conclude that
$\topol_{X \times Y} \subseteq \topol_{X' \times Y'}$, as desired.

\textit{(b) Does the converse of (a) hold? Justify your answer.}

Let $\topol$ and $\topol'$ be defined on a set $Q = \set{a, b}$ and
$U$ and $U'$ be defined on $W = \set{c, d}$.
Let $X = X' = \set{a}$, $Y = Y' = \set{c}$, $\topol = \set{\emptyset, \set{a}, \set{b}, Q}$,
$\topol' = \set{\emptyset, \set{a}, Q}$, and $U = U'$. Then we follow that
topology defined on $X \times Y$ is finer than the topology defined $X' \times Y'$ (and vice versa),
but $\topol'$ is not finer than $\topol$.



\subsection{}

\textit{Show that the countable collection
  $$\set{(a, b) \times (c, d): a < b \land c < d \land a, b, c, d \in Q}$$
  is a basis for $R^2$.}

Let us denote this set by $L$.
Suppose that $x \in R^2$. We follow that there exist $x_1, x_2, y_1, y_2 \in Q$ such that
$x_1 < x < x_2$ and $y_1 < y < y_2$, therefore $(\exists l \in L)(x \in l)$. Thus we follow that
the first condition of a definition of a basis is sastisfied. The last condition can be satisfied
by through the argument about the density of rationals in reals.


We can follow that topology, that is presented by given basis is a subset of the standard topology
on $R^2$, and we can follow though pretty much the same argument that given topology is
finer than the standard topology. Therefore I'm pretty sure that we can state that given basis
generates the standard topology (I'll not provide any proof of that, just stating what I think).


\subsection{}

\textit{Let $X$ be an ordered set. If $Y$ is a proper subset of $X$ that is
  convex in $X$, does it follow that $Y$ is an interval or a ray in $X$?}

Don't think so. I think that the author tries to give us a hint to what's going to come afterwards
(probably something about the completeness and whatnot).

Pretty sure, that we don't need to prove that $Q$ is a totally ordered set, so we're going to
take it as a given. Let
$$M = \set{x \in Q: x^2 < 2 \land x \geq 0}$$
(I've added the latter condition in order not to be bogged down  by
several cases, depending on the sign).
Let $x < y \in M$. Suppose that $z \in Q$ is such that  $x < z < y$. Then
we follow that $z > x  \geq  0$, thus $z > 0$. Since all of the numbers are positive,
we're justified to square them and get that
$x^2 < z^2 < y^2$. Given that $y^2 < 2$, we conclude that $z^2 < 2$ as well. Therefore
we follow that $z \in M$. Thus we can follow that $z \in (x, y) \ra z \in M$. Therefore
we can state that presented set is convex.

Given that $M$ is bounded above and below, we follow that it is not a ray. Suppose that
it is an interval. Then we follow that there exists $k \in Q$ such that $M = [0, k)$. Therefore
we follow that $k$ is a least lower bound of $M$, which is not the case, as proven in numerous
real analysis books. Thus we conclude that $M$ is not an interval.

\subsection{}

\textit{If $L$ is a straight line in the plane, describe the topology $L$ inherits as
  a subspace of $R_l \times R$ and as a subspace of $R_l \times R_l$. In each case it is a
  familiar topology.}

Let $\basis$ be the basis for $R_l \times R$ and $\basis'$ be the basis for $R_l \times R$.
Let $q \in \basis$  and suppoe that $b = L \cap q \neq \emptyset$. From plotting elements
of the basis and the line itself on the graph, we can conclude that $b$ is some sort of an interval
on the plane (either closed or open), and it might as well be a ray (once again, open or closed).
In case with $R_l \times R_l$ we conclude that the topology here is once again open or closed
intervals on the plane.

\subsection{}

\textit{Show that the dictionary order topology on the set $R \times R$ is the same
  as the product topology $R_d \times R$, where $R_d$ denotes $r$ in the discrete topology.
  Compare this topology with the standard topology on $R^2$.}


\end{document}
%%% Local Variables:
%%% mode: latex
%%% TeX-master: t
%%% End:
