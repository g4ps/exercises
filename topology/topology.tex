\documentclass[11pt,oneside,titlepage]{book}
\title{My topology exercises}
\usepackage{amsmath, amssymb}
\usepackage{geometry}
\usepackage{hyperref}
\author{Evgeny Markin}
\date{2023}

\DeclareMathOperator \map {\mathcal {L}}
\DeclareMathOperator \pow {\mathcal {P}}
\DeclareMathOperator \topol {\mathcal {T}}
\DeclareMathOperator \ns {null}
\DeclareMathOperator \range {range}
\DeclareMathOperator \fld {fld}
\DeclareMathOperator \inv {^{-1}}
\DeclareMathOperator \Span {span}
\DeclareMathOperator \lra {\Leftrightarrow}
\DeclareMathOperator \eqv {\Leftrightarrow}
\DeclareMathOperator \la {\Leftarrow}
\DeclareMathOperator \ra {\Rightarrow}
\DeclareMathOperator \imp {\Rightarrow}
\DeclareMathOperator \true {true}
\DeclareMathOperator \false {false}
\DeclareMathOperator \dom {dom}
\DeclareMathOperator \ran {ran}
\newcommand{\eangle}[1]{\langle #1 \rangle}
\newcommand{\set}[1]{\{ #1 \}}



\begin{document}
\maketitle
\tableofcontents

\chapter*{Preface}

Those are my solutions for the James Munkres' "Topology", 2nd edition.

Majority of the notation that is used here migrated from my course on the set theory. In
my very personal opinion, notation that is used there is far superior that whatever is
happening in Munkres' book.

If you decide to persue the study of topology yourself, then I highly recommend firstly
to go through a course on axiomatic set theory and logic, because first chapter of this
book is highly insufficient in this regard. My personal recommendations are the
Cunningham's "Set theory: A first course" book on set theory and
"A first course in Mathematical Logic and Set Theory" by Michael L. O’Leary on both subjects.
Although I've persued the completion of the first book, I don't think that it was
a wise decision. 

\part{General Topology}

\chapter{Set Theory and Logic}


\section{Fundamental Concepts}

\subsection{}

\textit{Check distributive and DML laws}

\textit{GOTO set theory book}

\subsection{}

\textit{Determine which of the following are true.}

(a) - impl

(b) - impl

(c) - true

(d) - rimpl

(e) - $\subseteq$, true if $B \subseteq A$.

(f) - $\supseteq$;   $A - (B - A) = A$.

(g) - true

(h) - $\supseteq$

(i) - true

(j) - true

(k) - false

(l) - true

(m) - $\subseteq$

(n) - true

(o) - true

(p) - true

(q) - $\supseteq$

\subsection{}

\textit{(a) Write a contrapositive and converse of the following statement:
  "If $x < 0$, then $x^2 - x > 0$" and determine which ones are true}

Contrapositive:
$$x^2 - x \leq 0 \ra x \geq 0$$
Converse
$$x^2 - x > 0 \ra x < 0$$

Contrapositive is correct, converse is incorrect ($2^2 - 2 > 0$)

\textit{(b) Do the same for the statement $x > 0 \ra x^2 - x > 0$}

Contrapositive:
$$x^2 - x \leq 0 \ra x \leq 0$$
Converse
$$ x^2 - x > 0 \ra x > 0 $$

Contrapositive is false ($1^2 - 1 = 0$); Converse is also false ($(-2)^2 - (-2) = 6$).

\subsection{}

\textit{Let $A$ and $B$ be the sets of real numbers. Write the negation of each of the
  following statements: }

\textit{(a)}
$$ (\exists a \in A)(a^2 \notin B)$$
\textit{(b)}
$$ (\forall a \in A)(a^2 \notin B)$$
\textit{(c)}
$$ (\exists a \in A)(a^2 \in B)$$
\textit{(d) }
$$ (\forall a)(a \notin A \ra a^2 \notin B)$$

\subsection{}

\textit{Let $A$ be a nonempty collection of sets. Determine the truths of each of the
  following and their converses}

\textit{(a)
$$x \in \bigcup{A} \lra (\exists B \in A)(x \in B)$$}
\textit{(b)
$$x \in \bigcup{A} \la (\forall B \in A)(x \in B)$$}
\textit{(c)
$$x \in \bigcap{A} \ra (\exists B \in A)(x \in B)$$}
\textit{(d)
$$x \in \bigcap{A} \lra (\forall B \in A)(x \in B)$$}

\subsection{}

Skip

\subsection{}

skip

\subsection{}

GOTO set theory book

\subsection{}

\textit{Formulate DML for arbitrary unions and intersections}

$$A \setminus \bigcap{(B)} = \bigcup{(A \setminus B)} $$
$$A \setminus \bigcup{(B)} = \bigcap{(A \setminus B)} $$

For the proof goto set theory or real analisys book

\subsection{}

(a, b, d) are true

\section{Functions}

\subsection{}

\textit{Let $f: A \to B$. Let $A_0 \subseteq A$ and $B_0 \subseteq B$.}

\textit{(a) Show that $A_0 \subseteq f\inv[f[A_0]]$ and that equality holds
  if $f$ is injective.}

Suppose that $x \in A_0$. We follow that there exists $\eangle{x, y} \in f$ for some
$y \in f[A_0]$. Therefore there exists $\eangle{y, x} \in f\inv$. Because $y \in f[A_0]$,
we follow that $x \in f\inv[f[A_0]]$. Therefore $A_0 \subseteq f\inv[f[A_0]]$.

Suppose that $f$ is injective. Suppose that there exists $x_0 \in f\inv[f[A_0]]$ such that
$x_0 \notin A_0$. We follow that $\eangle{y, x_0}, \eangle{y, x}, \in f\inv$,
therefore $\eangle{x_0, y}, \eangle{x, y} \in f$, and because $x_0 \neq x$ we follow
that we've got a contradiction.

\textit{((b) }

pretty simular to $(a)$

\textit{This chapter practicly mirrors the content of my set theory course
  . Gonna skip it for now, and will come back if the need arises.}

\chapter{Topological Spaces and Continous Functions}

\section{Topological Spaces}

I want to state here that if $\topol \subseteq \pow(X)$ satisfies
properties
$$\set{X, \emptyset} \subseteq \topol$$
$$(\forall Y \in \pow(\topol))( \bigcup{U} \in \topol)$$
$$(\forall Y \in \pow(\topol))(Y \neq \emptyset \land |Y| <_c
|\omega|  \to \bigcap{U} \in \topol)$$
then $\topol$ is a topology on $X$.

\section{Basis for a Topology}

Let $Y \subseteq \pow(X)$. If
$$(\forall x \in X)(\exists y \in Y)(x \in y)$$
and
$$(\forall x \in X)(\exists y_1, y_2, y_3 \in Y)(x \in y_1 \cap y_3 \to
x \in y_3 \land y_3 \subseteq y_1 \cap y_2)$$
then $Y$ is a basis for a topology on $X$.

\subsection{}

\textit{Let $X$ be a topological space; Let $A$ be a subset of $X$. Suppose that for each
  $x \in A$ there is an open set $U$ containing $x$ such that $U \subseteq A$. Show that $A$ is
  open in $X$.}

Let $U: A \to \pow(A)$ be an indexed function such that 
$$x \in U(x) \land U(x) \subseteq A \land U(x) \in \topol(X)$$
We want to show that $A = \bigcup{\ran(U)}$. Suppose that $x \in A$. We follow that
$x \in U(x)$. Thus $x \in \bigcup{\ran(U)}$. Therefore $A \subseteq \bigcup{\ran(U)}$.

Suppose that $z \in \bigcup{\ran(U)}$. We follow that
$$(\exists Y \in \ran(U))(z \in Y) \ra
(\exists x \in A)(z \in U(x))$$
Since $(\forall x \in A)(U(x) \subseteq A)$, we follow that $z \in A$. Thus
$\bigcup{\ran(U)} = A$.

Because $(\forall x \in A)(U(x) \in \topol(X))$, we follow that
$$\ran(U) \subseteq \topol(A)$$, therefore by definition of topology we follow that
$$\bigcup{\ran(U)} \in \topol(X)$$
as desired.

\subsection{}

Too tedious, skip

\subsection{}

\textit{Show that the collection $\topol_c$ given in Example 4 of p. 12 is a topology on the
  set $X$. Is the collection
  $$\topol_\infty = \{U \in \pow(X):
  |X \setminus U| \geq_c |\omega| \lor X \setminus U = \emptyset \lor
  X \setminus U = X\}$$
  a topology on $X$?
}

We firstly state that
$$\topol_c = \{U \in \pow(X): |X \setminus U| \leq_c |\omega| \lor X \setminus U = X\}$$

We can follow that $X \setminus X = \emptyset$, which is countable, thus $X \in \topol_c$.
$X \setminus \emptyset = X$, therefore $\emptyset \in \topol_c$.

Suppose that $U' \subseteq \topol_c$. If $U' = \{\emptyset\}$, then $
X \setminus \bigcap{U'} = X$ and $X \setminus \bigcup{U'} = X$.
Thus assume that $U' \neq \{\emptyset\}$.

We follow that
$$(\forall u \in U')(|X \setminus u| \leq_c |\omega| \lor X \setminus u = X)$$
We follow that if $\emptyset \in U'$, then $\bigcup{U'} = \bigcup{(U' \setminus \{\emptyset\})}$.
Then we follow by DML that
$$X \setminus \bigcup\{U'\} = X \setminus \bigcup\{U' \setminus \{\emptyset\}\} =
\bigcap_{U' \setminus \{\emptyset\}}{X \setminus u}$$
we know that $(\forall u \in U')(|X \setminus u| \leq_c |\omega|)$. For any $u \in U'$ we
follow that
$$\bigcap_{u \in U' \setminus \{\emptyset\}}{X \setminus u} \subseteq X \setminus u'$$
and given that $X \setminus u'$ is countable, we follow that $\bigcap_{u \in U'}{X \setminus u}$
is countable as well, thus $\bigcup{U'} \in \topol_c$.

Now let $U' \subseteq \topol_c$ and $|U'| < |\omega|$ and $U' \neq \{\emptyset\}$.
We follow that if $\emptyset \in U'$, then $\bigcap{U'} = \emptyset$, and therefore
$X \setminus \bigcap{U'} = X$. Therefore assume that $\emptyset \notin U'$.

Then we can follow that
$$X \setminus \bigcap{U'} = \bigcup_{u \in U'}{X \setminus u}$$
Given that $U'$ is countable and $X \setminus u$ is countable we follow that
$\bigcup_{u \in U'}{X \setminus u}$ is countable, thus $X \setminus \bigcap{U'}$ is countable.

Therefore we conclude that $\topol_c$ is a topology on $X$.

Now let us consider $T_\infty$. We can state that $X \in T_\infty$ because
$X \setminus X = \emptyset$. Because $X \setminus \emptyset = X$, we follow that
$\emptyset \in T_\infty$.

Suppose that $X$ is not infinite and $T_\infty \neq \{\emptyset, X\}$. Then there exists
$u \in T_\infty$ such that $u \neq \emptyset$ and $u \neq X$. Therefore $X - u$ is
nonempty finite set, therefore $u \notin T_\infty$, which is a contradiction.
Therefore we conclude that if $X$ is finite, then $T_\infty$ is a trivial topology.

If $X$ is infinite, then we follow that we can have an injection $f: \omega \to X$.
Let $O$ be the set of odd naturals and $E$ be the set of evens. Then we follow that
$$|X \setminus f[O]| = |f[E]| \geq_c |\omega|$$
and
$$|X \setminus f[E]| =_c |f[O]| \geq_c |\omega|$$
which tells us that $f[O]$ and $f[E]$ are both in $X$. We can also follow that
$$|X \setminus f[O \cup \{0\}]| \geq |\omega|$$
thus $ f[O \cup \{0\}] \in \topol_\infty$. This gives us that
$$f[E] \cap f[O \cup \{0\}] = \{f(0)\} \in \topol_\infty$$
but $\{f(0)\}$ is a finite nonempty set for which none of the conditions of $\topol_\infty$
hold. Therefore we conclude that if $X$ is infinite, then $\topol_\infty$ is not a topology.

Therefore we conclude that if $X$ is a finite set, then $T_\infty$ is equal to a
trivial topology; if $X$ is infinite, then $T_\infty$ is not a topology at all, since
it is not closed under finite intersections.

\subsection{}

\textit{(a) if $\{\topol_\alpha\}$ is a family of topologies on $X$, show that
  $\bigcap{\topol_\alpha}$ is a topology on $X$. Is $\bigcup{\topol_\alpha}$ a topology on $X$?}

Since every topology on $X$ has $X$ and $\emptyset$ as elements, we follow that
$$\{X, \emptyset\} \subseteq \bigcap{\topol_\alpha}$$
If $Y \subseteq \bigcap{\topol_\alpha}$, then we follow that
$$(\forall Z \in \{\topol_\alpha\})(\bigcap{\topol_\alpha} \subseteq Z)$$
$$(\forall Z \in \{\topol_\alpha\})(Y \subseteq Z)$$
since every $Z$ is a topology, we follow that
$$(\forall Z \in \{\topol_\alpha\})(\bigcup{Y} \in Z)$$
$$\bigcup{Y} \in \bigcap{\topol_\alpha}$$
If $Y$ is finite and nonempty, we can also follow that
$$(\forall Z \in \{\topol_\alpha\})(Y \in Z) \ra
(\forall Z \in \{\topol_\alpha\})(\bigcap{Y} \in Z) \ra \bigcap{Y} \in \bigcap{\topol_\alpha}$$
thus we conclude that $\bigcap{\topol_\alpha}$ is a topology.

$\bigcup{\topol_\alpha}$ is not necessarily a topology. Although
$\set{X, \emptyset} \in \bigcup{\topol_\alpha}$, we cannot follow that the topology is
closed under unions. Case in point: Let $X = \set{a, b, c}$ and
$$\topol_1 = \set{\emptyset, X, \set{a}}, \topol_1 = \set{\emptyset, X, \set{b}}$$
then $Y = \topol_1 \cup \topol_2$ does not contain $\set{a, b}$, which would be necessary
for this case. Thus we conclude that in general we can't have implications for
$\bigcup{\topol_\alpha}$.

\textit{(b) Let $\set{\topol_\alpha}$ be a family of topologies on $X$. Show that there is a
  unique smallest topology on $X$ containing all the collections $\topol_\alpha$ and
  a unique largest topology contained in all $\topol_\alpha$.}

Let us take $\bigcup{\set{\topol_\alpha}}$. We cannot follow that presented
set is a topology on $X$, nor can we state that it is a basis of a topology. Former
is followed from the discussion in the previous section of this exercise, and the latter
cannot be followed because we don't necessarily satisfy the
second point of the definition of the basis. Namely, we don't have that
$$(\forall x \in X)(\exists y_1, y_2, y_3 \in \bigcup{\set{\topol_\alpha}})(x \in y_1 \cap y_3 \to
x \in y_3 \land y_3 \subseteq y_1 \cap y_2)$$
Let $Q$ be a set of all of the intersections of finite nonempty subsets of
$\bigcup{\set{\topol_\alpha}}$. We follow that $(\forall x \in \bigcup{\set{\topol_\alpha}})
(x = \bigcap{\{x\}})$, therefore $\bigcup{\set{\topol_\alpha}} \subseteq Q$. 
Thus we follow that $Q$ satisfies
the first requirement for the basis of $X$. Now let $x \in X$ be such that there
exist $y_1, y_2 \in Q$ such that $x \in y_1 \cap y_2$. We follow that there exist
finite subsets $Y_1, Y_2 \subseteq \bigcup{\set{\topol_\alpha}}$ such that 
$$y_1 = \bigcap{Y_1} \land y_2 = \bigcap{Y_2}$$
therefore
$$y_1 \cap y_2  = \bigcap{Y_1} \cap \bigcap{Y_2}$$
which is an intersection of a finite subset of $\bigcup{\topol_\alpha}$. Thus we follow that there
exists $y_3 \in Q$ such that $x \in y_3 \land y_3 \subseteq y_1 \cap y_2$. 
Therefore we can follow that the set $Q$ is indeed a basis for a topology on $X$.
Let us name the topology generated by this set as $\topol_q$.

Suppose that there is a topology, which contains
all of the topologies $\set{\topol_\alpha}$. Then we follow that it contains
$\bigcup{\set{\topol_\alpha}}$, therefore we follow that it contains all of the unions
of $\bigcup{\set{\topol_\alpha}}$, and finite intersections of subsets of
$\bigcup{\set{\topol_\alpha}}$, and thus it contains $\topol_q$. Therefore
we follow that $\topol_q$ is the smallest topology, which contains all
the topologies of $\set{\topol_\alpha}$.

Suppose that $\topol_p$ is a topology, which is contained in all of the $\set{\topol_\alpha}$.
Then we follow that $\topol_p \subseteq \bigcap{\topol_\alpha}$. Because $\bigcap{\topol_\alpha}$
is a topology itself, we follow that it is the largest topology, which is contained
in all of the $\set{\topol_\alpha}$.

\textit{(c) If $X = \set{a, b, c}$, let
  $$\topol_1 = \set{\emptyset, X, \set{a}, \set{a, b}}$$
  $$\topol_2 = \set{\emptyset, X, \set{a}, \set{b, c}}$$
  Find the smallest topology containing $\topol_1$ and $\topol_2$, and the largest topology
  contained in $\topol_1, \topol_2$.
}

We can follow from previous discussions that largest contained topology is
$$\set{\emptyset, X, \set{a}}$$
and the smallest containing topology is
$$\set{\emptyset, X, \set{a}, \set{b}, \set{a, b}, \set{b, c}}$$

\subsection{}

\textit{Show that if $A$ is a basis for a topology on $X$, then the topology generated by $A$
  equals the intersection of all topologies on $X$ that contains $A$. Prove the same
  if $A$ is a subbasis.}

Let $A$ be a subbasis.
Let $\set{\topol_\alpha}$ be a set of topologies, that contain $A$ and  $\topol_A$ is
a topology generated by $A$. We can follow that $\topol_A \in \set{\topol_\alpha}$,
therefore $\bigcap{\set{\topol_\alpha}} \subseteq {\topol_A}$. If $x \in \topol_A$, then we
follow that there exists a subset $B \subseteq A$ such that $x$ is equal to some
union of some finite intersections of $B$. Since
$B \subseteq A$, we follow that $(\forall y \in \topol_\alpha)(B \subseteq y)$. Therefore
all of the finite intersections of $B$ are in any topology of $\topol_\alpha$.  Therefore
all of the unions of those intersections are in any $\topol_\alpha$. Therefore
we conclude that $(\forall y \in \topol_\alpha)(x \in y)$.
and thus $x \in \bigcap{\topol_\alpha}$.
Therefore we conclude that $\topol_A \subseteq  \bigcap{\topol_\alpha}$, and by double
inclusion we get that $\topol_A =  \bigcap{\topol_\alpha}$, as desired.

Since every basis of a topology is a subbasis by first clause of the definition, we follow
that the desired result holds for bases as well.

\subsection{}

\textit{Show that the topologies of $R_l$ and $R_k$ are not comparable.}

TODO



\end{document}
%%% Local Variables:
%%% mode: latex
%%% TeX-master: t
%%% End:
