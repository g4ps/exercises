\documentclass[11pt,oneside,titlepage]{book}
\title{My set theory exercises}
\usepackage{amsmath, amssymb}
\usepackage{geometry}
\usepackage{hyperref}
\author{Evgeny Markin}
\date{2023}

\DeclareMathOperator \map {\mathcal {L}}
\DeclareMathOperator \pow {\mathcal {P}}
\DeclareMathOperator \ns {null}
\DeclareMathOperator \range {range}
\DeclareMathOperator \fld {fld}
\DeclareMathOperator \inv {^{-1}}
\DeclareMathOperator \Span {span}
\DeclareMathOperator \lra {\Leftrightarrow}
\DeclareMathOperator \eqv {\Leftrightarrow}
\DeclareMathOperator \la {\Leftarrow}
\DeclareMathOperator \ra {\Rightarrow}
\DeclareMathOperator \imp {\Rightarrow}
\DeclareMathOperator \true {true}
\DeclareMathOperator \false {false}
\DeclareMathOperator \dom {dom}
\DeclareMathOperator \card {card}
\DeclareMathOperator \ran {ran}
\DeclareMathOperator \On {On}
\DeclareMathOperator \Crd {Crd}
\DeclareMathOperator \ineq {\underline{\in}}
\newcommand{\eangle}[1]{\langle #1 \rangle}
\newcommand{\set}[1]{\{ #1 \}}

\begin{document}
\maketitle
\tableofcontents

\chapter*{Preface}

Exercises are from the "Set Theory: A First Course" by Daniel W. Cunningham.

\chapter*{Useful things}

I think that it is pretty straightforward to define some class function based on
axioms that we get.
For example pairing axiom allows us to define $PA: S \times S \to S$ by
$$PA(u, v) = \{u, v\}$$
same goes for union axiom
$$UA(u) = \{\text{elements of elements of U}\}$$
Later some other function might be defined in the same manner.

In logic notation, I denote tautology as '$\true$' and contradiction as '$\false$'

There is a rule that I've used
$$a \land (b \lor \neg a) \lra (a \land b) \lor (a \land \neg a))
\lra (a \land b) \lor (\false) \lra a \land b$$
which I don't remember seeing in the book, but it's pretty useful.

Sometimes I use $\oplus$ to denote xor.

I also denote $^BA$ as $A^B$, because it's more convinient.

Sometimes I omit parantheses when it's clear what's going on.

I firmly believe that using "injective" and "surjective" instead of "one-to-one" and
"onto" makes much more sense, and thus I use the former notation here.

\chapter*{On Classes and Collections}

None of the courses in set theory that I've seen so far have proper
definitions of classes and collections, despite the fact that those words
occur frequently within texts and definitions. This little fact of life
generated a lot of confusion when I've first started the course. 
After some time spent with this book
I've came up with either a definition or just a useful mneumonic which allowes me
to be a bit more rigorous. 
When we say that $C$ is a class and $\alpha \in C$, what we mean is that $C$ is essentially
just a formula, and $\alpha$ satisfies this formula. For example, when we define
$\On$ as a class of ordinals, we mean that $\On$ is just a formula
$$\On(x): x \text{ is transitive} \land x \text{ is well-ordered by } \ineq$$
thus when we say that $\alpha \in \On$ we actually mean is that
$$\alpha \text{ is transitive} \land \alpha \text{ is well-ordered by } \ineq$$
In the language of programming languages, we essentially "overload" the operation $\in$
to mean "is in the set" in case if we've got a set to the right of that symbol and
"satisfies the formula" in case if we've got a class to the right of that symbol.

With this definition of class we can define "collection" to be equivalent to a class.
When we say that a class $C$ is a set, we mean that there exists a set $Q$ such that
$$x \in Q \iff x \text{ satisfies C}$$
good example of such a class is a class of natural numbers. In the book we've firstly defined
them by a logical formula (i.e. a set is a natural number if and only if it is an element of
every inductive set), and then proved that this class is a set. 
Thus we can define a proper class to be a formula such that there is no set $Q$ for which
$$x \in Q \iff x \text{ satisfies C}$$
this definition gives us a (somewhat useless) equivalence that $C$ is
a proper class if and only if for all sets $X$ there exists $\alpha \notin X$
such that $C(\alpha)$.

I'm pretty sure that there's a good reason for why we don't define classes and collections
in courses of set theory. So far my approach haven't presented any problems during the course
of this book, and I'm planning to continue to use this definition untill I'm proven wrong.


\chapter{Introduction}

\section{Elementary Set Theory}

Let A, B, C be sets

\subsection*{1.1.1}

\textit{If $a \notin A \setminus B$ and $a \in A$, show that $a \in B$}

Because $a \notin A \setminus B$, we follow that $x \in B$ or $x \notin A$. Since $x \in A$, we
follow that $x \in B$, as desired.

\subsection*{1.1.2}

\textit{Show that if $A \subseteq B$, then $C \setminus B \subseteq C \setminus A$}

Let $c \in C \setminus B$. Then we follow that $c \in C$ or $c \notin B$. Since $A \subseteq B$,
we follow that $c \notin B$ implies that $c \notin A$. Thus we follow that
$c \in C \setminus B$ implies that $c \in C \setminus A$. Therefore $C \setminus B \subseteq
C \setminus A$.

\subsection*{1.1.3}

\textit{Suppose $A \setminus B \subseteq C$. Show that $A \setminus C \subseteq B$.}

Suppose that $a \in A \setminus C$. Then we follow that $a \in A$ and $a \notin C$.

Given that $A \setminus B \subseteq C$ and $A \notin C$, we follow that $a \notin A \setminus B$.
Thus $a \notin A$ or $a \in B$. Since $a \in A$, we follow that $a \in B$. Thus
$$a \in A \setminus C \to a \in B$$
$$A \setminus C \subseteq  B$$
as desired.

\subsection*{1.1.4}

\textit{Suppose $A \subseteq B$ and $A \subseteq C$. Show that $A \subseteq B \cap C$}

Suppose that $a \in A$. Then we follow that $a \in B$ and $a \in C$. Thus $a \in B \cap C$.
Therefore we follow that $A \subseteq B \cap C$.

\subsection*{1.1.5}

\textit{Suppose $A \subseteq B$ and $B \cap C = \emptyset$. Show that $A \in B \setminus C$}

Suppose that $a \in A$. Then we follow that $a \in B$ and since $B \cap C = \emptyset$, we
follow that $a \notin C$. Thus $a \in B \setminus C$ by definition. Therefore
$A \subseteq B \setminus C$.

\subsection*{1.1.6}

\textit{Show that $A \setminus (B \setminus C) \subseteq (A \setminus B) \cup C$.}

Suppose that $a \in A \setminus (B \setminus C)$. Then we follow that
$a \in A$ and $a \notin B \setminus C$. Thus $a \notin B$ and $a \in C$. Thus we
follow that $a \in A \setminus B$ or $a \in C$. Thus
$A \setminus (B \setminus C) \subseteq (A \setminus B) \cup C$
as desired.

\subsection*{1.1.7}

\textit{Show that $A \setminus B \subseteq C$ and $A \not \subseteq C$ then
  $A \cap B = \emptyset$}

Suppose that $A \cap B = \emptyset$. We follow that $A \setminus B = A$, thus
$A \setminus B = A \subseteq C$, which is a contradiction.

\subsection*{1.1.8}

\textit{Let $P(x)$ be the property $x > \frac 1 x$. Are the assertions $P(2)$, $P(-2)$,
  $P(\frac 1 2)$ $P( \frac{-1}{2})$ true or false .}

$$2 > \frac 1 2 \to P(2) = true$$
$$-2 < \frac{-1}{2} \to P(-2) = false$$
last two are reversed.

\subsection*{1.1.9}

\textit{Show that each of the following sets can be expressed as an interval}

$$a) (-3, 3)$$
$$b) (-3, \infty)$$
$$c) (-3, 3)$$

all of them follow from order properties of real numbers.

\subsection*{1.1.10}

\textit{Express the following sets as truth sets}

$$A = \{1, 4, 9, 16, 25, ...\} \iff A = \{x \in N: x = y^2 \textit{ for some } y \in N\}$$
$$B = \{..., -15, -10, -5, 0, 5, ... \} \iff A = \{x \in N: x = 5y  \textit{ for some } y \in N\}$$

\section{Logical Notation}

\subsection*{1.2.1}

\textit{Using truth tables, show that $\neg(P \ra Q) \lra (P \land \neg Q)$}

\begin{center}
  \begin{tabular}{c| c| c| c| c| c|}
    P & Q & $P \ra Q$ & $\neg(P \ra Q)$ & $\neg Q$ & $P \land \neg Q$ \\
    false & false & true & false & true & false \\
    false & true & true & false & false & false \\
    true & false & false & true & true & true \\
    true & true & true & false & false & false \\
  \end{tabular}  
\end{center}

from this we can see that they are equqivalent.

\textit{Following 4 exercises are the same as this one, so I'm skipping them}

\subsection*{1.2.5}

\textit{Show that $(P \imp Q) \land (P \imp R) \eqv P \imp (Q \land R)$, using logic laws}

$$(P \imp Q) \land (P \imp R) \eqv (\neg P \lor Q) \land (\neg P \lor R) \eqv
\neg P  \lor (R \land Q) \eqv  P  \imp (R \land Q) $$
Laws used: 
$$CL \to DIST \to CL$$

\subsection*{1.2.6}

\textit{Show that $(P \imp R) \lor (Q \imp R) \eqv (P \land Q) \imp R$, using logic laws}

$$(P \imp R) \lor (Q \imp R)  \eqv (\neg P \lor R) \lor (\neg Q \lor R) \eqv
\neg P \lor R \lor \neg Q \lor R \eqv (\neg Q \lor \neg P) \lor R \eqv$$
$$ \eqv \neg (Q \land P) \lor R
\eqv (Q \land R) \imp R$$
Laws used:
$$CL \to ASC \to ID, ASC \to DML \to CL$$

\subsection*{1.2.7}

\textit{Show that $P \imp (Q \imp R) \eqv (P \land Q) \imp R$, using logic laws}

$$P \imp (Q \imp R) \eqv \neg P \lor (Q \imp R) \eqv \neg P \lor (\neg Q \lor R) \eqv
(\neg P \lor \neg Q) \lor R \eqv \neg (P \land Q) \lor R \eqv (P \land Q) \imp R$$
Laws used:
$$CL \to CL \to ASC \to DML \to CL$$

\subsection*{1.2.8}

\textit{Show that $(P \imp Q) \imp R$ and $P \imp (Q \imp R)$ are not logically equivalent}

We're gonna show that $q \land w \eqv false$
$$((P \imp Q) \imp R) \land (P \imp (Q \imp R)) \eqv (\neg (\neg P \lor Q) \lor R) \land
(\neg P \lor (\neg Q \lor R)) \eqv $$
$$\eqv ((P \land \neg Q) \lor R) \land (\neg P \lor \neg Q \lor R) \eqv
((P \land Q) \land (\neg P \lor \neg Q)) \lor R  \eqv$$
$$ \eqv ((P \land Q) \land \neg ( P \land  Q)) \lor R  \eqv false \lor R \eqv false$$


\section{Predicates and Quantifiers}

\section{A Formal Language for Set Theory}

\subsection{}

\textit{What does the formula $\exists x \forall y (x \notin y)$ say in English? }

There exists $x$ such that for every $y$ we've got that x is not in y. In other ways, there
exists an empty set.

\subsection{}

\textit{What does the formula $\forall y \exists x (y \notin x)$ say in English?}

For every y there exists set x such that y is not in x.

\subsection{}

\textit{What does the formula $\forall y \exists x (x \notin y)$ say in English?}

For every y there exists x such that x is not in y.

\subsection{}

\textit{What does the formula $\forall y \neg \exists x (x \notin y)$ say in English?}

For every y there does not exist an x such that x is not in y.

\subsection{}

\textit{What does the formula $\forall z \exists x \exists y (x \in y \land y \in z)$
  say in English?}

For every $z$ there exists x and y such that x is in y and y is in z

\subsection{}

\textit{Let $\phi(x)$ be a formula. What does $\forall z \forall y((\phi(x) \land \phi(y))
  \to z = y$}

For every z and y, $\phi(x)$ and $\phi(y)$ implies that z = y.

\subsection{}

\textit{Translate each of the following into the language of set theory.}

\textit{(a) x is the union of a and b}

$$\forall (y \in x) (y \in a \land y \in b)$$

\textit{(b) x is not a subset of y}

$$\exists (z \in x) (\neg z \in y)$$

\textit{(c) x is the intersection of a and b}

$$\forall (y \in x) (y \in a \lor y \in b)$$

\textit{(d) a and b have no elements in common}

$$\forall (x \in a) \forall (y \in b) (\neg x = y)$$

\subsection{}

\textit{Let a, b, C and D be sets. Show that the relationship}
$$y =
\begin{cases}
  a \text{ if } x \in C \setminus D \\
  b \text{ if } x \notin C \setminus D \\
\end{cases}
$$

$$((x \in C \land \neg x \in D) \to (y = a)) \land ((\neg x \in C \land \neg x \in D) \to (y = a))$$


\section{The Zermelo-Fraenkel Axioms}

\subsection{}

\textit{Let $u, v, w$ be sets. By pairing axiom, the sets $\{u\}$ and $\{v, w\}$
  exist. Using the pairing and union axioms, show that the set $\{u, v, w\}$
  exists.}

By pairing axiom we've got that
$$PA(u, u) = \{u\}$$
$$PA(v, w) = \{v, w\}$$
thus
$$PA(\{u\}, \{v, w\}) = \{\{u\}, \{v, w\}\}$$
and therefore by union axiom we follow that
$$UA(\{\{u\}, \{v, w\}\}) = \{u, v, w\}$$
as desired.

\subsection{}

\textit{Let $A$ be a set. Show that the pairing axiom implies that the set $\{A\}$ exists}

$$PA(A, A) = \{A, A\}$$
which by extension axiom is equal to $\{A\}$, as desired.

\subsection{}

\textit{Let $A$ be a set. The pairing axiom implies that the set $\{A\}$ exists. Using the
  regularity axiom, show that $A \cap \{A\} = 0$. Conclude that $A \notin A$.}

Since $\{A\} \neq \emptyset$, we follow that there exists $x$ such that $x \in \{A\}$ and
$x \cap \{A\} = \emptyset$. Since $A$ is the only element of $\{A\}$, we follow that
$A \cap \{A\} = \emptyset$, as desired.

\subsection{}

\textit{For sets $A, B$, the set $\{A, B\}$ exists by the pairing axiom. Let $A \in B$.
  Using the regularity axiom, show that $A \cap \{A, B\} = \emptyset$, and thus $B \notin A$.}

$\{A, B\}$ consists of sets $A$ and $B$, thus it is not empty and therefore
there exists $x \in \{A, B\}$ such that $x \in \{A, B\} \land x \cap \{A, B\} = \emptyset$.
For $B$ we've got that $B \in \{A, B\}$. Since $A \in B$ and $A \in \{A, B\}$, we can follow that
$A \in (B \cap \{A, B\})$. By pairing axiom we follow that the element with desired
property must exists, and given that the only other choice is $A$,
we conclude that $A \cap \{A, B\} = \emptyset$. Therefore we can follow that $B \notin A$, as
desired.

\subsection{}

\textit{Let $A, B, C$ be sets. Suppoes that $A \in B$ and $B \in C$. Using the regularity axiom,
  show that $C \notin A$.}

This is an expantion of previous exercise. We can follow that
$$B \in \{A, B, C\} \land B \in C \imp B \in C \cap \{A, B, C\} \imp C \cap \{A, B, C\}
\neq \emptyset$$
$$A \in \{A, B, C\} \land A \in B \imp A \in B \cap \{A, B, C\} \imp B \cap \{A, B, C\}
\neq \emptyset$$
thus the only other choice is $A$, and therefore $A \cap  \{A, B, C\} = \emptyset$. Therefore
$C \notin A$, as desired.


\subsection{}

\textit{Let $A, B$ be sets. Using the subset and power set axioms, show that the set
  $\pow(A) \cap B$ exists.}

Because set $A$ exists we follow that $\pow(A)$ exists. By setting $\phi(x): x \in B$ and
subset axiom we follow that there exists a subset of $\pow(A)$ such that
$x \in S \lra x \in \pow(A) \land x \in B$. Thus we follow by Extensionality axiom
that $S = \pow(A) \cap B$. Thus $\pow(A) \cap B$ exists.

\subsection{}

\textit{Let $A, B$ be sets. Using the subset axiom, show that the set $A \setminus B$ exists.}

$$\phi(x): \neg x \in B$$
thus by subset axiom
$$x \in S \lra x \in A \land \neg x \in B$$
thus $A \setminus B$ exists.

\subsection{}

\textit{Show that no two of the sets $\emptyset, \{\emptyset\}, \{\emptyset, \{\emptyset\}\}$
  are equal to each other.}

I had a little confusion with this one at first because I thought that every set has
empty set in it, which is false. Every set has an empty set as a subset, but it
might be so that empty set is not in the set itself.
$$\emptyset \notin \emptyset \land \emptyset \in \{\emptyset\} \imp \emptyset \neq \{\emptyset\}$$
$$\emptyset \notin \emptyset \land \emptyset \in \{\emptyset, \{\emptyset\}\} \imp
\emptyset \neq \{\emptyset, \{\emptyset\}\}$$
$$\{\emptyset\} \notin \{\emptyset\} \land \{\emptyset\} \in \{\emptyset, \{\emptyset\}\} \imp
\{\emptyset\} \neq \{\emptyset, \{\emptyset\}\}$$
all of the implication follow from extensionality axiom.

\subsection{}

\textit{Let $A$ be a set with no elements. Show that for all $x$, we have that $x \in A$ if
  and only if $x \in \emptyset$. Using the extensionality axiom, conclude that $A = \emptyset$.}

Suppose that $\neg x \in A$. Then we follow that $x$ is an element, therefore $\neg x \in \emptyset$.
Thus
$$\neg x \in A \imp \neg x \in \emptyset \iff x \in \emptyset \imp x \in A$$
Suppose that $\neg x \in \emptyset$. Then we follow that $x$ is an element. Thus $\neg x \in A$.
Thus
$$\neg x \in \emptyset \imp \neg x \in A \iff x \in A \imp x \in  \emptyset$$
thus we follow that
$$x \in \emptyset \lra x \in A$$
thus by extensionality axiom we follow that
$$\emptyset = A$$
which gives us nice follow-up that
$$\emptyset = \{\}$$

\subsection{}

\textit{Let $\phi(x, y)$ be the formula $\forall z(z \in y \lra z = x)$ which asserts that
  $y = \{x\}$. For all x the set $\{x\}$ exists. So $\forall x  \exists! y \phi(x, y)$.
  Let $A$ be a set. Show that the collection $\{\{x\}: x \in A\}$ is a set.}

We know that $A$ is a set and therefore $\pow(A)$ is also a set. Thus by subset axiom
there exists a set 
$$\exists S (x \in S \lra x \in \pow(A) \land \exists(y \in A)(\phi(x, y)))$$
which is precisely our collection.

\chapter{Basic Set-Building Axioms and Operations}

\section{The First Six Axioms}

Prove the following theorems, where $A, B, C, D$ are sets.

\subsection*{2.1.1}

$$A \subseteq B \to (A \subseteq A \cup B \land A \cap B \subseteq A )$$

$$ \forall x (x \in A \to x \in B) \to ((\forall x (x \in A \imp x \in A \lor x \in B)) \land
(\forall (x \in A \land x \in B \imp x \in A))) \lra $$
$$ \lra
\forall x (x \in A \to x \in B) \to ((\forall x (\neg x \in A \lor x \in A \lor x \in B)) \land
(\forall (\neg (x \in A \land x \in B) \lor x \in A))) \lra $$
$$ \lra \forall x (x \in A \to x \in B) \to ((\forall x ( \true  \lor x \in B)) \land
(\forall ( \neg x \in A \lor \neg x \in B \lor x \in A))) \lra $$
$$ \lra \forall x (x \in A \to x \in B) \to (\true \land
(\forall ( true \lor \neg x \in B))) \lra $$
$$ \lra \neg \forall x (x \in A \to x \in B) \lor (\true \land \true) \lra $$
$$ \lra \neg \forall x (x \in A \to x \in B) \lor \true \lra $$
$$\true$$

\subsection*{2.1.2}

$$A \subseteq B \land B \subseteq C \to A \subseteq C$$

$$(\forall x (x \in A \imp x \in B)) \land (\forall x (x \in B \imp x \in C)) \to
\forall x (x \in A \imp x \in C) \lra$$
$$\lra (\forall x (\neg x \in A \lor x \in B)) \land (\forall x (\neg x \in B \lor x \in C)) \to
\forall x (\neg x \in A \lor x \in C) \lra $$
$$\lra (\forall x ((\neg x \in A \lor x \in B) \land  (\neg x \in B \lor x \in C))) \to
\forall x (\neg x \in A \lor x \in C) \lra $$
$$\lra (\forall x ((\neg x \in A \land  (\neg x \in B \lor x \in C))
\lor( x \in B  \land  (\neg x \in B \lor x \in C)))) \to
\forall x (\neg x \in A \lor x \in C) \lra $$
$$\lra (\forall x (\neg x \in A \land  (\neg x \in B \lor x \in C))
\lor( (x \in B  \land  \neg x \in B)  \lor (x \in B  \land x \in C)))) \to
\forall x (\neg x \in A \lor x \in C) \lra $$
$$\lra (\forall x ((\neg x \in A \land  \neg x \in B) \lor (\neg x \in A \land x \in C) 
\lor(  x \in B  \land x \in C)) \to \forall x (\neg x \in A \lor x \in C) \lra  ... $$

So this thing is tedious as hell and should be left to computers.

Suppose that $x \in A$. Then we follow by $A \subseteq B$ that $x \in B$. Thus by $B \subseteq C$
we follow that$x \in C$. Therefore $x \in A \to x \in C$. Therefore $A \subseteq C$, as desired.

\subsection*{2.1.3}

$$B \subseteq C \imp A \setminus C \subseteq A \setminus B$$

Suppose that $x \in A \setminus C$. Then we follow that $x \in A$ and $x \notin C$. Therefore
$x \in A$ and $x \notin B$ since $B \subseteq C$. Thus $x \in A \setminus B$. Therefore
we follow that $B \subseteq C$ implies that  $A \setminus C \subseteq A \setminus B$,
as desired.

\subsection*{2.1.4}

$$C \subseteq A \land C \subseteq B \iff C \subseteq A \cap B$$

Suppose that $x \in C$. Then we follow that $x \in A$ and $x \in B$. Thus $x \in A \cap B$.
Therefore $C \subseteq A \cap B$. Thus we follow that
$C \subseteq A \land C \subseteq B \imp C \subseteq A \cap B$

Suppose that $x \in C$. Then we follow that $x \in A \cap B$. Thus $x \in A$ and $x \in B$.
Therefore $C \subseteq A \land C \subseteq B$. Therefore
$C \subseteq A \cap B \imp C \subseteq A \land C \subseteq B $
thus we follow that
$$C \subseteq A \land C \subseteq B \iff C \subseteq A \cap B$$
as desired.

\subsection*{2.1.5}

\textit{There exists an $x$ such that $x \notin A$}

Suppose that there does not exist $x$ such that $x \notin A$. Then we follow that every set is a
member of $A$, which is impossible.

\subsection*{2.1.6}

$$A \cap B = B \cap A$$

$$x \in A \cap B \iff x \in A \land x \in B \iff x \in B \land x \in A \iff x \in B \cap A$$

\subsection*{2.1.7}

$$A \cup B = B \cup A$$

$$x \in A \cup B \iff x \in A \lor x \in B \iff x \in B \lor x \in A \iff x \in B \cup A$$

\subsection*{2.1.8}

$$A \cap (B \cup C) = (A \cup C) \cap (A \cup B)$$

$$x \in A \cap (B \cup C) \lra  x \in A \land x \in (B \cup C) \lra
x \in A \land (x \in B \lor x \in C) \lra
$$
$$ \lra (x \in A \lor x \in C) \land (x \in A \lor x \in C)
\lra (x \in A \cup B) \lor (x \in A \cup C) \lra x \in ((A \cup B) \cap (A \cup C))$$

\subsection*{2.1.31}

$$A \subseteq \pow(\cup(A))$$

Let $x \in A$. Then we follow that $x \subseteq \cup(A)$. Thus $x \in \pow(A)$. Thus
$A \subseteq \pow(\cup(A))$.

\subsection*{2.1.32}

\textit{Let $C \in F$. Then $\pow(C) \in \pow(\pow(\cup F))$}

Suppose that $C \in F$. Then we follow that $C \subseteq \cup F$. Therefore $C \in \pow (\cup F)$.
Thus $\pow(C) \in \pow(\pow (\cup F))$.


\textit{the rest of the exercises for this section are more of the same.}

\section{Operations on Sets}

Prove the following theorems

\subsection{}

\textit{Let $A$ be a set and $F \neq \emptyset$. Then}
$$A \setminus \cap F = \cup\{A \setminus C: C \in F\}$$

$$x \in A \setminus \cap F \lra
x \in A \land x \notin \cap F \lra
x \in A \land \neg x \in \cap F \lra
x \in A \land \neg (\forall(C \in F)(x \in C)) \lra
$$
$$ \lra
x \in A \land \exists(C \in F)(x \notin C) \lra
\exists(C \in F)(x \notin C \land x \in A) \lra
\exists(C \in F)(x \in A \setminus C) \lra
x \in \cup\{A \setminus C: C \in F\}
$$
which seems to hold.

\subsection{}

\textit{Let $A, F$ be sets. Then $A \cup (\cup F) = \cup\{A \cup C: C \in F\}$}

$$x \in A \cup (\cup F) \lra x \in A \lor x \in \cup F \lra
x \in A \lor  (\exists C \in F)(x \in C) \lra$$
$$
\lra (\exists C \in F)(x \in A) \lor \exists(C \in F)( x \in C) \lra
$$
$$ \lra 
\exists(C \in F)(x \in A \lor x \in C) \lra
\exists(C \in F)(x \in A \cup C) \lra x \in \cup\{A \cup C: C \in F\}
$$

Where we've used the fact that
$$x \in A \lra x \in A \land \true \lra x \in A \land (\exists C \in F)(\true) \lra
(\exists C \in F)(x \in A  \land \true) \lra  (\exists C \in F)(x \in A)$$
don't know if we can use it, but I used it anyways.




\subsection{}

\textit{Let $A, F$ be sets. Then $A \cap (\cup F) = \cup\{A \cap C: C \in F\}$}

$$x \in A \cap (\cup F) \lra x \in A \land x \in \cup F \lra
x \in A \land  (\exists C \in F)(x \in C) \lra$$
$$ \lra 
\exists(C \in F)(x \in A \land x \in C) \lra
\exists(C \in F)(x \in A \cap C) \lra x \in \cup\{A \cap C: C \in F\}$$


\subsection*{2.2.5}

\textit{Let $A$ and $F$ be sets. Then there exists a unique set $\epsilon$ such that for all
  $Y$ we have that $Y \in \epsilon$ if and only if $Y = A \cap C$ for some $C \in F$.}

$\cup F$ exists by union axiom, $A \cap (\cup F)$ exists by subset axiom. Thus $\pow(A \cap (\cup F))$
exists by power axiom. Since $Y = A \cap C \imp Y \subseteq A \cap (\cup F)$, we follow that
$Y$ is a subset of $\pow(A \cap (\cup F))$, which exists by subset axiom. By extensionality axiom we follow that
the set is unique.

\subsection*{2.2.12}

\textit{If $F$ and $G$ are nonempty sets, then }
$$\cap(F \cup G) = \cap(F) \cap  \cap(G)$$

$$x \in \cap(F \cup G) \lra (\forall C \in F \cup G)(x \in C) \lra (\forall C \in F)(x \in C) \land
(\forall C \in G)(x \in C) \lra $$
$$ \lra x \in \cap(F) \land x \in \cap(G) \lra x \in (\cap(F)) \cap (\cap(G))$$


\subsection*{2.2.14}

\textit{Let $F$ be a nonempty set. Then}
$$\pow(\cap (F)) = \cap\{\pow(C): C \in F\}$$

$$x \in \pow(\cap (F)) \lra x \subseteq \cap (F) \lra (\forall y \in x)(y \in \cap (F)) \lra
(\forall y \in x)(\forall (C \in F) (y \in F)) \lra $$
$$
\forall (C \in F) ( (\forall y \in x) y \in F) \lra
\lra \forall(C \in F)(x \subseteq C) \lra \forall(C \in F)(x \in \pow(C)) \lra x \in  \cap\{\pow(C): C \in F\}$$

\chapter{Relations and Functions}

\section{Ordered Pairs in Set Theory}

\subsection{}

\textit{Define $\eangle{a, b, c} = \eangle{\eangle{a, b}, c}$ for any sets $a, b, c$. Prove that this yields an
  ordered tuple; that is, prove tahat if $\eangle{x, y, z} = \eangle{a, b, c}$, then $x = a$, $y = b$, $z = c$.}

Suppose that
$$\eangle{x_1, x_2, x_3} = \eangle{y_1, y_2, y_3}$$
then we follow that
$$\eangle{\eangle{x_1, x_2}, x_3} = \eangle{\eangle{y_1, y_2}, y_3}$$
from which we get that $\eangle{x_1, x_2} = \eangle{y_1, y_2}$ and $x_3 = y_3$. From
$\eangle{x_1, x_2} = \eangle{y_1, y_2}$ we get that $x_1  = y_1$ and $x_2 = y_2$. In
total we get that
$$\eangle{\eangle{x_1, x_2}, x_3} = \eangle{\eangle{y_1, y_2}, y_3}
\imp x_1 = y_1 \land x_2 = y_2 \land x_3 = y_3$$
Thus we follow that given construction defines an ordered tuple, as desired.

\subsection{}

\textit{Prove that $(A \cup B) \times C = (A \times C) \cup (B \times C)$}

$$x \in (A \cup B) \times C \lra x = \eangle{y, z} \land y \in A \cup B \land z \in C
\lra  x = \eangle{y, z} \land (y \in A \lor y \in  B) \land z \in C $$
$$ \lra  (x = \eangle{y, z} \land z \in C ) \land (y \in A \lor y \in  B) \lra $$
$$\lra 
(x = \eangle{y, z} \land z \in C \land y \in A ) \lor
(x = \eangle{y, z} \land z \in C \land y \in B ) \lra$$
$$\lra 
(x \in A \times C ) \lor (x \in B \times C) \lra x \in (A \times C) \cup (B \times C)$$
as desired.

\subsection{}

\textit{Prove that $(A \setminus B) \times C = (A \times C) \setminus (B \times C)$}

$$x \in (A \setminus B) \times C \lra x = \eangle{y, z} \land y \in A \setminus B \land z \in C
\lra  x = \eangle{y, z} \land (y \in A \land y \notin  B) \land z \in C $$
$$ \lra  (x = \eangle{y, z} \land z \in C ) \land (y \in A \land y \notin  B) \lra $$
$$ \lra  x = \eangle{y, z} \land z \in C \land y \in A \land y \notin  B \lra $$
$$ \lra  (x = \eangle{y, z} \land y \in A \land z \in C) \land
 (x \neq \eangle{y, z} \lor y \notin B \lor z \notin C)  \lra $$
$$ \lra  (x = \eangle{y, z} \land y \in A \land z \in C) \land
 (x \neq \eangle{y, z} \lor y \notin B \lor z \notin C)  \lra $$
$$ \lra  (x = \eangle{y, z} \land y \in A \land z \in C) \land
\neg (x = \eangle{y, z} \land y \in B \land z \in C))  \lra $$
$$\lra 
(x \in A \times C) \land \neg (x \in B \times C) \lra x \in (A \times C) \setminus (B \times C)$$

Used a biconditional defined in "useful things"


\subsection{}

\textit{Prove that $$(\cup F) \times C = \cup\{A \times C: A \in F\}$$}

$$x \in (\cup F) \times C \lra x = \eangle{y, z} \land y \in (\cup F) \land z \in C \lra
x = \eangle{y, z} \land (\exists A \in F)(y \in A) \land z \in C \lra
$$
$$ \lra   (\exists A \in F)(y \in A \land x = \eangle{y, z} \land z \in C) \lra
(\exists A \in F)(x \in A \times C) \lra$$
$$\lra x \in  \cup\{A \times C: A \in F\}$$

\section{Relations}

\subsection{}

\textit{Explain why the empty set is a relation}

Relation is defined to be a set of ordered pairs. That is, for every $x \in R$, $x$ is an ordered pair.
Since we haven't got any elements in the emptyset, we follow that the logical statement is true and  therefore
emptyset is a relation.

Other way to see it is to assume that it is not a relation. Then we follow that emptyset has an element that
is not an ordered pair. Since emptyset does not have any elements, we follow that we have
a contradiction.

\subsection{}

\textit{Prove items 1-3 of Theorem 3.2.7}

$$x \in \dom(R \inv) \lra \exists y (\eangle{x, y} \in R \inv)
\lra \exists y (\eangle{y, x} \in R) \lra x \in \ran(R)$$

$$x \in \ran(R \inv) \lra \exists y (\eangle{y, x} \in R \inv)
\lra \exists y (\eangle{x, y} \in R) \lra x \in \dom(R)$$

$$x \in (R \inv) \inv \lra \exists y \exists z (\eangle{y, z} \in (R \inv) \inv ) \land x = \eangle{y, z}
\lra \exists y \exists z (\eangle{z, y} \in (R \inv) ) \land x = \eangle{y, z} \lra $$
$$ \lra  \exists y \exists z (\eangle{y, z} \in R ) \land x = \eangle{y, z} \lra x \in R$$


\subsection*{3.2.4}

$$\dom(R) = \{0, 1, 2, 3, 4\}$$
$$\ran(R) = \{0, 1, 2, 3, 4\}$$
$$R \circ R = \{\eangle{0, 2}, \eangle{0, 3}, \eangle{0, 0}, \eangle{0, 3}, \eangle{0, 4},
\eangle{1, 0}, \eangle{1, 3}, \eangle{1, 4}, \eangle{1, 3}, \eangle{1, 2}, \eangle{2, 1},$$
$$ 
\eangle{2, 2}, \eangle{2, 3}, \eangle{2, 2}, \eangle{2, 4}, \eangle{3, 3}, \eangle{3, 2},
\eangle{4, 4}\}$$
$$R|\{1\} = \{\eangle{1, 2}, \eangle{1, 3}\}$$
$$R\inv|\{1\} = \{\eangle{1, 0}\}$$
$$R[\{1\}] = \{2, 3\}$$
$$R\inv[\{1\}] = \{0\}$$

\subsection*{3.2.5}

\textit{Suppose that $R$ is a relation. Prove that $R|(A \cup B) = (R | A) \cup (R | B)$
  for any sets $A, B$}

$$x \in R|(A \cup B) \lra (\exists y \in A \cup B) (\exists z \in  \ran(R)) (\eangle{y, z}
\in R \land x = \eangle{y, z}) \lra $$
$$ 
\lra (\exists y \in A) (\exists z \in  \ran(R)) (\eangle{y, z}
\in R \land x = \eangle{y, z}) \lor
 (\exists y \in B) (\exists z \in  \ran(R)) (\eangle{y, z}
\in R \land x = \eangle{y, z}) \lra $$
$$ \lra x \in R|A \lor x \in R|B \lra x \in (R|A \cup R|B)$$
thus
$$R|(A \cup B) = (R | A) \cup (R | B)$$
as desired.

\subsection*{3.2.6}

\textit{Let $R$ be a relation. Prove that $\fld(R) = \bigcup \bigcup R$.}

$$x \in \fld(R) \lra x \in \dom(R) \lor x \in \ran(R) \lra
(\exists z \in R)(\exists y)(z = \eangle{x, y} \lor z = \eangle{y, x}) \lra$$
$$ \lra
(\exists z \in R)(\exists y)(z = \{\{x\}, \{x, y\}\} \lor z = \{\{y\}, \{y, x\}\})
\lra
(\exists z \in R)(x \in \bigcup z) \lra x \in \bigcup \bigcup R$$

\subsection*{3.2.7}

\textit{Let $R$ and $S$ be two relations and let $A, B, C$ be sets. Prove that $R|A$,
  $R\inv [B]$, $R[C]$ and $R \circ S$ are sets.}

Given that $R$ and $S$ are relation, we follow that both of them are sets,
$\bigcup \bigcup R$ and  $\bigcup \bigcup S$ are sets and $\dom (R), \ran(R), \dom(S), \ran(S)$
are sets. Thus we follow that $R|A$ is a subset of
$R$, which is a set; $R\inv [B]$  and $R[C]$ are subsets of $\bigcup \bigcup R$,
and $R \circ S$ are subsets of a set $\dom (R) \times \ran(S)$, which is a set.


\subsection*{3.2.8}

\textit{Let $R$ be a relation and $G$ be a set. Prove that $\{R[C]: C \in G\}$ is a set. Prove
  that if $G$ is nonempty, then $\{R[C]: C \in G\}$ is also nonempty}

If $R$ is a relation, then $\ran(R)$ is a set. Therefore $\pow(\ran(R))$ is a set. Thus
for any set $C$, $R[C] \subseteq \ran(R)$, therefore $R[C] \in \pow(\ran(R))$. Thus
$\{R[C]: C \in G\}$ is a subset of $\pow(\pow(\ran(R)))$, which is a set.

Suppose that $G$ is nonempty. Then we follow that there exists $C \in G$. Thus
$R[C]$ is a set. Thus $R[C] \in \{R[C]: C \in G\}$. Therefore $\{R[C]: C \in G\}$ is
nonempty.

\subsection*{3.2.10}

\textit{Let $R$ be a relation on $A$. Prove that $R$ is symmetric if and only if
  $$R\inv \subseteq R$$}

\textbf{In forward direction: }
Suppose that $R$ is symmetric. 
Let $y \in R \inv$. We follow that there exists $u, v$ such that $y = \eangle{u, v}$.
Thus $\eangle{v, u} \in R$ by the definition. Since $R$ is symmetric, we follow that
$\eangle{u, v} \in R$. Therefore $y \in R\inv \ra y \in R$, as desired.

\textbf{In reverse direction: }
Suppose that $R\inv \subseteq R$. Let $y \in R$. Then we follow that there exists
$u, v$ such that $y = \eangle{u, v}$. Thus $\eangle{v, u} \in R\inv$. Since $R\inv \subseteq R$,
we follow that $\eangle{v, u} \in R$. Thus we follow that $\eangle{u, v} \in R \ra
\eangle{u, v} \in R$. Thus $R$ is symmetric by definition, as desired.

\subsection*{3.2.19}

\textit{Prove item (2) of Theorem 3.2.8}

$$R[\bigcup G] = \bigcup{R[C]: C \in G}$$

$$x \in R[\bigcup G] \lra (\exists y \in \bigcup G)(\eangle{y, x} \in R) \lra
(\exists C \in G)(y \in C \land \eangle{y, x} \in R) \lra$$
$$\lra (\exists C \in G)(x \in R[C]) \lra x \in \bigcup{R[C]: C \in G}$$

\subsection*{3.2.20}

\textit{Prove item (4) fo Theorem 3.2.8}

$$x \in R[\bigcap G] \lra (\exists y \in \bigcap G)(\eangle{y, x} \in R) \lra
\exists y (\forall C \in  G)( y \in C \land \eangle{y, x} \in R) \ra$$
$$ \ra
(\forall C \in G) (\exists y \in C) ( \eangle{y, x} \in R) \lra
(\forall C \in G) (x \in R[C]) \lra x \in \bigcap{\{R[C]: C \in G\}}$$

\subsection*{3.2.22}

\textit{Let $R$ and $S$ be single-rooted relations. Prove that $R \circ S$ is
  single-rooted}

Suppose that $\eangle{x, y} \in R \circ S$. Then we follow that there exists $z \in
fld(R)$ such that
$\eangle{x, z} \in S$ and $\eangle{z, y} \in R$. Suppose that there exists  $\eangle{x', y} \in
R \circ S$ such that $x \neq x'$. Then we follow that there exists $j \in \fld(R)$ such that
$\eangle{x', j} \in S$ and $\eangle{j, y} \in R$. If $j = z$, then we follow that
$\eangle{x', z} \in S \land \eangle{x, z} \in S \land x \neq x'$. Thus we conclude that $S$ is not
single rooted, which is a contradiction. If $j \neq z$, then we follow that
$$\eangle{j, y} \in R \land \eangle{z, y} \in R \land j \neq z$$
thus $R$ is not single-rooted, which is also a contradiction. Therefore we conclude that
if $\eangle{x', y} \in R \circ S$, then $x' = x$. Therefore we follow that $R \circ S$ is
single rooted, as desired.

\section{Functions}

\subsection*{3.3.1}

\textit{Prove Lemma 3.3.5 and Lemma 3.3.13}

Suppose that $F$ and $G$ are functions such that $\dom(F) = \dom(G)$. 
Lemma 3.3.5 states that $F = G$ iff $F(x) = G(x)$ for every $x \in \dom(F)$
If $F = G$, then
$$F(x) = y \lra \eangle{x, y} \in F \lra \eangle{x, y} \in G \lra G(x) = y$$
thus $F(x) = G(x)$ for every $x \in \dom(F)$.

Now suppose that $F(x) = G(x)$ for every $x \in \dom(F)$. Then we follow that
$$z \in F \lra z = \eangle{x, y} \land F(x) = y \lra
z = \eangle{x, y} \land G(x) = y \lra z \in G$$
as desired.

Lemma 3.3.13 states that a function $F$ is one-to-one if and only if $F$ is single-rooted.

Suppose that $F$ is one-to-one and $F$ is not single rooted. Then we follow that
there exists $x, y \in F$ such that $x = \eangle{u, w} \in F, y = \eangle{j, w} \in F$.
Then we follow that $F(u) = w = F(j)$, which is a contradiction.

Proof of converse is extremely simular.

\subsection*{3.3.2}

\textit{Let $F$ be a function and let $A \subseteq B \subseteq \dom(F)$. Prove that
  $F[A] \subseteq F[B]$.}

$$x \in F[A] \lra x = \eangle{u, v} \land u \in A \land \eangle{u, v} \in F \ra
x = \eangle{u, v} \land u \in B \land \eangle{u, v} \in F \lra x \in F[B]$$

\subsection*{3.3.5}

\textit{Let $g: C \to D$ be a one-to-one function, $A \subseteq C$ and $B \subseteq C$. Prove
  that if $A \cup B = \emptyset$, then $g[A] \cap g[B] = \emptyset$.}

Suppose that $A \cap B = \emptyset$ and $g[A] \cap g[B] \neq \emptyset$. Then we follow that
there exists $x \in g[A] \cap g[B]$. Thus
$$x \in g[A] \land x \in g[B] \lra (\exists y \in A)(\eangle{y, x} \in g) \land
(\exists z \in B)(\eangle{z, x} \in g)$$
since $g$ is one-to-one, we follow that $z = y$. Thus there exists $z \in A \cap B$, therefore
$A \cap B \neq \emptyset$, which is a contradiction.

\subsection*{3.3.9}

\textit{Suppose that $F: X \to Y$ is a function. Prove that if $C \subseteq Y$ and
  $D \subseteq Y$, then $F\inv[C \cap D] = F\inv[C] \cap F\inv[D]$.}

Since $F$ is a function, we follow that $F\inv$ is a single-rooted relation. Thus we follow that
$$F\inv[C \cap D] = F\inv[C] \cap F\inv[D]$$
as desired.

\subsection*{3.3.10}

\textit{Let $F, G$ be functions from $A$ to $B$. Suppose $F \subseteq G$. Prove that $F = G$.}

Suppose that $x \in A$. Then we follow that
$$(\exists y \in B)(\eangle{x, y} \in F) \ra (\exists y \in B)(\eangle{x, y} \in G)$$
Thus we follow that for every $x \in A$ (where $\dom (F) = A = \dom(G)$)
$$F(x) = G(x)$$
thus by the lemma 3.3.5 we follow that
$$F = G$$
as desired.


\subsection*{3.3.11}

\textit{Let $C$ be a set of functions. Suppose that for all $f$ and $g$ in $C$, we have either
  $f \subseteq g$ or $g \subseteq f$. }

\textit{(a) Prove that $\cup C$ is a function}

Firstly, since $C$ is a set of sets of ordered pairs, we follow that $\cup C$ is a set of
ordered pairs, and therefore it is a relation. Suppose that $x \in \cup C$. Then we follow that
there exist  $f \in C$ such that $x \in f$ and 
$x = \eangle{u, v}$. Suppose that there exists $y \in \cup C$, such that $y = \eangle{u, v'}$,
where $u' \neq u$. Since $y \in \cup C$, we follow that there exists $g \in C$ such that
$y \in C$. Because $u' \neq u$, we follow that $g \neq f$. Therefore either $g \subset f$,
or $f \subset g$. In both cases we follow that we can't have the case that $u' \neq u$. Thus
we follow that for $x, y \in \cup C$, whenever the first part of the $x$ is equal to the first
part of $y$, we follow that the last parts are also equal. Thus we follow that $\cup C$ is
a single-valued relation, and therefore it is a function.

\subsection*{3.3.13}

\textit{Assume $f: A \to B$ is onto $B$. Let $C \subseteq B$. Prove that $f[f\inv[C]] = C$}

$$x \in f[f\inv[C]] \lra (\exists y \in f\inv[C])(f(y) = x) \lra
(\exists z \in C)(\eangle{y, z} \in f \land f(y) = x) \lra$$
$$ \lra (\exists z \in C)(f(y) = z \land f(y) = x) \lra (\exists z \in C)(x = z) \lra x \in C$$
this notation may be a bit sloppy, but the result is derived faithfully.

\subsection*{3.3.15}

\textit{Let $f: A \to B$ be a one-to-one function. Define $G: \pow(A) \to \pow(B)$ by
  $G(X) = f[X]$, for each $X \in \pow(A)$. Prove that $G$ is one-to-one.}

Let $X_1, X_2 \in \pow(A)$ be such that $G(X_1) = G(X_2)$. Then we follow that
$$f[X_1] = f[X_2]$$
thus
$$x \in X_1 \lra (\exists y \in f[X_1])(\eangle{x, y} \in f) \lra
(\exists y \in f[X_2])(\eangle{x, y} \in f) \lra x \in X_2$$
thus we follow that $X_1 = X_2$. Therefore $G(X_1) = G(X_2) \to X_1 = X_2$, thus $G$ is
one-to-one, as desired.

\subsection*{3.3.21}

\textit{Let $\eangle{A_i: i \in I}$ be an indexed function with nonempty terms. Prove that there
  is an indexed function ${x_i: i \in I}$ is that $x_i \in A_i$ for all $i \in I$, using
  theorem 3.3.24}

Let $C$ be defined as
$$C = ran(A)$$
then by theorem 3.3.24 we follow that there exists a function $H: C \to \cup C$ such that
$$H(A_i) \in A_i$$
define $x = H \circ A $. Then we follow that
$$x(i) = H(A(i)) = H(A_i) \in A_i$$
thus we have the desired function.

\section{Order Relations}

\subsection*{3.4.1}

\textit{Define a relation $\preceq$ on the set of intezers Z by $x \preceq y$ if and only if
  $x \leq y$ and $x + y$ is even for all $x, y \in Z$. Prove that $\preceq$ is a partial order
  on $Z$}

Suppose that $x, y \in Z$. Poset requirements of $\leq$ are going to be ommited.

Then we follow that $$x + x = 2x$$, therefore we've got symmetry.
If $x \leq y$, $y \leq z$, $x + y$ is even and $y + z$ is even, then we follow that
$x + y + y + z$ is even, therefore $x + z - 2y$ is also even.
Antisymmetry follows from $\leq$.

\textit{Then answer the following questions about the poset $(Z, \preceq)$}

\textit{(a) Is $S = \{1, 2, 3, 4, 5, 6, ..., \}$ a chain in $Z$}

No, since $1 + 2 = 3$ is not even, we follow that $1 \not \preceq 2 \land 2 \not \preceq 1$.

\textit{(b) Is $S = \{1, 3, 5, 7 ...,\}$ a chain in $Z$?}

Suppose that $x, y \in Z$. Then we follow that there exist $m, n \in Z$ such that
$$x = 2m + 1$$
$$y = 2n + 1$$
thus
$$x + y = 2m + 2n + 2$$
thus $x + y$ is even for all cases. Since $\leq$ is a total order, we follow that $S$ is
indeed a chain in $Z$.

\textit{(c) Does the  set $S = \{1, 2, 3, 4, 5\}$ have a lower bound or an upper bound?}

No, since we've got both odd and even numbers in $S$, we follow that there is no $x \in Z$ such that
$x + s$ is even for all $s \in S$, thus there does not exist an element such taht
$x \preceq s$ or $s \preceq x$ for all $s \in S$

\textit{(d) Does $S = \{1, 2, 3, 4, 5\}$ have any maximal or minimal elements?}

Yes, $1, 2$ are minimal elements and $4, 5$ are maximal elements.


\subsection*{3.4.2}

\textit{Prove Lemma 3.4.5}

Suppose that $(A, \preceq)$ is a poset and let $\prec$ be the strict order corresponding to
$\preceq$. Let $x, y, z \in A$. Then we follow that:

\textit{(1)}

Since $x = x$, we follow that $x \not \prec x$ by definition of a strict order

\textit{(2)}

Suppose that $x \prec y$. Then we follow that $x \preceq y$ and $y \neq x$. Thus by antisymmetry
of $\preceq$ we follow that $y \not \preceq x$, and therefore $y \not \prec x$ by definition.

\textit{(3)}

Suppose that $x \prec y$ and $y \prec z$. Therefore we follow that $x \preceq y$ and $y \preceq z$,
which gives us that $x \preceq z$. Suppose that $x = z$. Then we follow that $z \preceq y$
and thus $z = y$, which is a contradiction. Thus we follow that $x \preceq z$ and $x \neq z$,
therefore $x \prec z$, as desired.

\textit{(4)}

If $x = y$ then $x \not \prec y$ and $y \not \prec x$ by definition of strict order

Both $x \prec y$ and $y \prec x$ imply that $x \neq y$, and by case (2) we follow that
for given $x, y \in A$ only one of $x = y$, $x \prec y$ or $y \prec x$ hold.

\subsection*{3.4.3}

\textit{Find the greatest lower bound of the set $S = \{15, 20, 30\}$ in the poset
  $(A, |)$, where $A = \{n \in N: n > 0\}$. Now find the least upper bound of $S$.}

$5$ and $60$ (gcd and lcm) respectively.

\subsection*{3.4.4}

\textit{Let $(A, \preceq)$ be a poset and let $S \subseteq A$. Suppose that $b$ is the
  largest element of $S$. Prove that $b$ is also the least upper bound of $S$.}

Suppose that $s$ is an upper bound of $S$. Then from the definition of the lower bound we
follow that $x \prec s$ for every $x \in A$. Since $b \in s$ we follow that $b \prec s$.
Thus $b$ is the least upper bound (can't we just call it supremum?) of $A$ by definition.

\subsection*{3.4.11}

\textit{Let $(B, \preceq')$ be a poset. Suppose $h: A \to B$ is an injective function.
  Define the retalion $\preceq$ on $A$ by $x \preceq y$ if an only if $h(x) \preceq' h(y)$,
  for all $x, y \in A$. Prove that $\preceq$ is a parial order.}

Let $x, y, z \in A$. Since $h$ is a function, we follow that $x = x$ implies that $h(x) = h(x)$,
therefore $h(x) \preceq' h(x)$, thus $x \preceq x$. Thus we've got reflexivity.

If $x \preceq y$ and $y \preceq z$ we follow that $h(x) \preceq' h(y)$ and $h(y) \preceq h(z)$,
from which we follow that $h(x) \preceq' h(z)$ and therefore $x \preceq z$.

Suppose that $x \preceq y$ and $y \preceq x$. Then we follow that $h(x) \preceq' h(y)$
and $h(y) \preceq' h(x)$. Thus $h(y) = h(x)$. Because $h$ is one-to-one, we follow that
this implies that $x = y$. Thus we've got antisymmetry. Therefore $\preceq$ is indeed a poset.

\subsection*{3.4.12}

\textit{Let $(B, \preceq')$ be a totally ordered. Suppose $h: A \to B$ is an injective function.
  Define the retalion $\preceq$ on $A$ by $x \preceq y$ if an only if $h(x) \preceq' h(y)$,
  for all $x, y \in A$. Prove that $\preceq$ is a parial order.}

We follow that $\preceq$ is a poset from previous exercise. Suppose that $x, y \in A$.
then we follow that $h(x) \preceq' h(y) \lor h(y) \preceq' h(x)$. Thus
$x \preceq y \lor y \preceq x$. Therefore $\preceq$ is a total order.

\subsection*{3.4.14}

\textit{Analogous to 4}

\subsection*{3.4.14}

\textit{Let $\preceq$ be a partial order of $A$. Let $C \subseteq A$. Show that
  $\preceq_C$ is a partial order on $C$. Show that if $\preceq$ is a total order on $A$,
  then $\preceq_C$ is a total order on $C$.}

Everything follows directly from definitions.

\subsection*{3.4.17}

\textit{Let $\preceq$ be a partial order on $A$. Show that $fld(\preceq) = A$.}

For every $x \in A$ we've got that 
$x \preceq x$, thus we follow that $\eangle{x, x} \in \preceq$, thus $x \in fld(\preceq)$.
Thus $A \subseteq fld(\preceq)$. Since $fld(\preceq) \subseteq A$
we follow that $A = fld(\preceq)$, as desired.

\subsection*{3.4.18}

\textit{Let $C$ be a set where each $\preceq \in C$ is a partial order on its field.
  Suppose that for any $\preceq, \preceq'$ in $C$ either $\preceq \subseteq \preceq'$
  or $\preceq' \subseteq \preceq$}

\textit{(a) Show that for every $\preceq, \preceq'$ in $C$ if $\preceq \subseteq \preceq'$,
  then $\fld(\preceq) \subseteq \fld(\preceq')$}

Suppose that $\preceq \subseteq \preceq'$. Then we follow that
$\fld(\preceq) = \bigcup \bigcup \preceq$ and
$\fld(\preceq') = \bigcup \bigcup \preceq'$. Suppose that
$$x \in \fld(\preceq)$$
then we follow that there exists $y$ such that  $\eangle{x, y} \in \preceq$ or
$\eangle{y, x} \in \preceq$. Since $\preceq \subseteq \preceq'$, we follow
that $\eangle{x, y} \in \preceq'$ or $\eangle{y, x} \in \preceq'$. Thus
$x \in \fld(\preceq')$. Thus $x \in \fld(\preceq) \ra x \in \fld(\preceq')$,
as desired.

\textit{(b) Let $\preceq^C$ be the relation $\bigcup C$ with field
  $\bigcup\{\fld(\preceq): \preceq \in C\}$. Prove taht $\preceq^C$ is a partial order
  on its field}

Since $\preceq^C = \bigcup C$ we follow that if $x \in \bigcup\{\fld(\preceq): \preceq \in C\}$,
then there exists $\preceq' \in C$ such that $x \in \fld(\preceq')$. Thus, by
the fact that $\preceq'$ is a partial order, we follow that $\eangle{x, x} \in \preceq'$. Thus
$\eangle{x, x} \in \bigcup C$, and therefore $\eangle{x, x} \in \preceq^C$. Thus
$\preceq^C$ is reflexive.

Suppose that $x, y, z \in \bigcup\{\fld(\preceq): \preceq \in C\}$. Then we follow that
there exist $\preceq_x, \preceq_y, \preceq_z$ such that
$$x \in \fld(\preceq_x)$$
and so on. From the description of $C$ we follow that $\preceq_x \subseteq \preceq_y$
or $\preceq_y \subseteq \preceq_x$, which in any case means that there exists
$\preceq_{xy}$ (equal to either of them) such that
$$\{x, y\} \subseteq \fld(\preceq_{xy})$$
same goes for $\preceq_z$ and $\preceq_{xy}$, which means that there exists $\preceq'$ such that
$$\{x, y, z\} \subseteq \fld(\preceq')$$

Since $\preceq'$ is a partial order, we follow that if $x \preceq' y$ and $y \preceq' z$
implies that $x \preceq' z$, therefore $\eangle{x, z} \in \preceq'$, and therefore
$\eangle{x, z} \in \preceq^C$, thus the relation is transitive.

Now let's go back a bit and focus on $\preceq_{xy}$. We follow that if $x \preceq_{xy} y$
and $y \preceq_{xy} x$, then by antisymmetry of $\preceq_{xy}$ we follow that $x = y$. Thus
we've got the antisymmetry of $\preceq^C$.

Thus $\preceq^C$ satisfies all requirements of a poset, as desired.

\subsection*{3.4.19}

\textit{Let $(A, \preceq)$ and $(B, \preceq')$ be posets. Define the relation $\preceq_l$ on
  $A \times B$ by
  $$ \eangle{a, b} \preceq_l \eangle{x, y} \iff a \prec x \lor (a = x \land b \preceq' y) $$
  for all $\eangle{a, b}$ and $\eangle{x, y}$ in $A \times B$. }

\textit{(a) Prove that $\preceq_l$ is a partial order on $A \times B$}

Let $a \in A$ and $b \in B$. Then we follow that $b \preceq' b$ and $a = a$, thus
$$\eangle{a, b} \preceq_l \eangle{a, b}$$

If for some $a_1, a_2, a_3 \in A$ and $b_1, b_2, b_3 \in B$ it is true that
$$\eangle{a_1, b_1} \preceq_l \eangle{a_2, b_2}$$
and
$$ \eangle{a_2, b_2} \preceq_l \eangle{a_3, b_3} $$
then we follow that
$$a_1 \prec a_2 \lor (a_1 = a_2 \land b_1 \preceq' b_2)$$
and
$$a_2 \prec a_3 \lor (a_2 = a_3 \land b_2 \preceq' b_3)$$

If $a_1 = a_2 = a_3$, then we follow that $b_1 \preceq' b_2$ and $b_2 \preceq' b_3$, from which
we follow that $b_1 \preceq b_3$ and therefore $\eangle{a_1 b_1} \preceq \eangle{a_3, b_3}$.

If $a_1 \neq a_2$ and $a_2 = a_3$ then we follow that $a_1 \prec a_2$, thus $a_1 \prec a_3$,
therefore $\eangle{a_1, b_1} \preceq_l \eangle{a_3, b_3}$. Similar case holds if
$a_1 = a_2$ and $a_2 \neq a_3$

If $a_1 \neq a_2$ and $a_2 \neq a_3$, then we follow that $a_1 \prec a_2$ and $a_2 \prec a_3$,
from which we follow that $a_1 \preceq a_3$. If $a_1 = a_3$, then we follow that
$a_1 \not \prec a_2$, which is a contradiction. Thus $a_1 \neq a_3$. Thus $a_1 \prec a_3$.
Therefore $\eangle{a_1, b_1} \preceq_l \eangle{a_3, b_3}$

Thus we follow that $\eangle{a_1, b_1} \preceq_l \eangle{a_3, b_3}$ for all $a_1, a_3$, thus
we've got the transitive condition.

Now suppose that $\eangle{a_1, b_1} \preceq_l \eangle{a_2, b_2}$ and
$\eangle{a_2, b_2} \preceq_l \eangle{a_1, b_1}$. If $a_1 \neq a_2$, then we follow that
$a_1 \prec a_2$ and $a_2 \prec a_1$, which is not possible, thus we conclude that $a_1 = a_2$.
Thus we follow that $b_1 \preceq' b_2$ and $b_2 \preceq' b_1$, from which we follow that
$b_1 = b_2$. Thus $\eangle{a_1, b_1} \preceq_l \eangle{a_2, b_2}$ and
$\eangle{a_2, b_2} \preceq_l \eangle{a_1, b_1}$, then we follow that
$\eangle{a_1, b_1} = \eangle{a_2, b_2}$. Thus we've got antisymmetry.

Now we can follow that $\preceq_l$ satisfies all of the conditions of poset, as desired.

\textit{(b) Suppose that both $\preceq$ and $\preceq'$ are total orders on their respective sets.
  Prove that $\preceq_l$ is a total order on $A \times B$.}

Suppose that $\eangle{a_1, b_1}, \eangle{a_2, b_2}  \in A \times B$ are arbitrary. It follow that
$a_1 \preceq a_2$ or $a_2 \preceq a_1$. Thus $a_1 \prec a_2 \lor a_2 \prec a_1 \prec a_1 = a_2$.
We also follow that $b_1 \preceq' b_2$ or $b_2 \preceq' b_1$. This it can be shown that
$$\eangle{a_1, b_1} \preceq_l \eangle{a_2, b_2}$$
or
$$\eangle{a_2, b_2} \preceq_l \eangle{a_1, b_1}$$
as desired.

\section{Congruence and Preorder}

\subsection*{3.5.1}

\textit{Let $\sim$ be an equivalence relation on $A$. Let $f: A \to A/_\sim$ be defined by
  $$f(x) = [x]$$
  for all $x \in A$. Show that the function $f$ is one-to-one if and only if each equivalence
  classs in $A/_\sim$ is a singleton.}

\textbf{In forward direction: }
Suppose that
$$f(x) = [x]$$
is one-to-one. Suppose that $x, y \in [x]$. Then we follow that $f(x) = f(y)$. Thus by the fact
that $f$ is one-to-one we follow that $x = y$. Thus $[x]$ is a singleton.

\textbf{In reverse direction: }
Suppose that each $[x]$ is a singleton. Then we follow that if $x \neq y$, then $f(x) \neq f(y)$.
Therefore $f$ is one-to-one, as desired.

\subsection*{3.5.2}

\textit{Prove theorem 3.5.6}

Let $\sim$ be an equivalence relation on $A$ and let $f: A \times A \to A$.

Suppose that there exisxst a function $\hat{f}: A/_\sim \times A/_\sim \to A/_\sim$ such that
$\hat{f}([x], [y]) = [f(x, y)]$
then we follow that if $x \sim w$ and $y \sim z$, then $[x] = [w]$ and $[y] = [z]$, thus
$$\hat{f}([x], [y]) = [f(x, y)] = \hat{f}([w], [z]) = [f(w, z)]$$
thus we follow that $[f(x, y)] = [f(w, z)]$, therefore $f(x, y) \sim f(w, z)$.
Thus we follow that $x \sim w$ and $y \sim z$ implies than $f(x, y) \sim f(w, z)$, therefore
$f$ is congruent with $\sim$.

If $x \sim w$ and $y \sim z$ implies that $f(x, y) \sim f(w, z)$, then we follow that
we can construct a relation
$\hat{f}$  by
$$\hat{f} = \{ \eangle{\eangle{[x], [y]}, [f(x, y)]}, x, y \in A\}$$
if $[x] = [w]$ and $[y] = [z]$, then we follow that $x \sim w$ and $y \sim z$, therefore
$f(x, y) \sim f(w, z)$, therefore  $[f(x, y)] = [f(w, z)]$. Thus this relation is a function, as
desired.

\subsection*{3.5.3}

\textit{Prove theorem 3.5.7}

Suppose that $\sim$ is an equivalence relation on $A$ and $R$ is a relation on $A$.

If there exists a relation $\hat{R}$ on $A/_\sim$ such that
$\hat{R}([x], [y]) \iff R(x, y)$
then we follow that if  $xRy$ for some $x, y \in A$ then $\hat{R}([x], [y])$
If there exist $w, z \in A$ such that
$x \sim w$ and $y \sim z$, then $[x] = [w]$ and $[y] = [z]$, therefore $\hat{R}([w], [z])$
and thus $wRz$. Thus we follow that $xRy$ iff $wRz$. Thus $R$ is congruent with $\sim$.

If $R$ is congruent with $\sim$, then we can create a relation $\hat{R}$ by
$$\hat{R} = \{\eangle{[x], [y]} : \eangle{x, y} \in R\}$$
that will satisfy the desired conditions.

\subsection*{3.5.4}

\textit{Prove lemma 3.5.9}

Let $(A, \preceq)$ be a preordered set and let $\sim$ be the relation on $A$ defined by
$$a \sim b \iff a \preceq b \land b \preceq a$$
Then we follow that for $a \in A$ $a \preceq a$ by reflexivity of $\preceq$, therefore
$a \sim a$ and thus $\sim$ is reflexive.

If $a, b, c \in A$ such that $a \sim b$ and $b \sim c$, then we follow that
$$a \preceq b \land b \preceq c \ra a \preceq c$$
$$c \preceq b \land b \preceq a \ra c \preceq a$$
thus $a \sim c$. Therefore $\sim$ is transitive.

Suppose that $x \sim y$. Then we follow that $x \preceq y $ and $y \preceq x$. Thus we follow that
$y \sim x$, therefore $\sim$ is symmetric. Thus $\sim$ is an equivalence relation, as desired.

\subsection*{3.5.5}

\textit{Prove lemma 3.5.10}

Suppose taht $(A, \preceq)$ is a preordered set. Let $\sim$ be the derived equivalence
relation on $A$. Suppose that $a, b, c, d \in A$ are such that $a \sim c$ and $b \sim d$.

If $a \preceq b$ then we follow that $b \preceq d$ by $b \sim d$, therefore $a \preceq d$
by transitive property. From $a \sim c$ we derive that $c \preceq a$, therefore we
follow that $c \preceq d$. Similar implication hold for reverse case. Thus we follow that
$\preceq$ is congruent with $\sim$.


\chapter{The Natural Numbers}

\section{Inductive Sets}

\subsection*{4.1.1}

\textit{Let $I$ and $J$ be inductive sets. Prove that $I \cap J$ is also inductive.}

Since $I$ and $J$ are both inductive, we follow that $\emptyset \in I$ and $\emptyset \in J$,
thus $\emptyset \in I \cap J$.

Suppose that $x \in I \cap J$. Then we follow that $x \in I$ and $x \in J$. Since both of them
are inductive, we follow that $x^+ \in I$ and $x^+ \in J$, therefore $x^+ \in I \cap J$. Thus
we've got that
$$\emptyset \in I \cap J \land (\forall x \in I \cap J)(x^+ \in I \cap J)$$
thus we follow tha $I \cap J$ is inductive, as desired.

\subsection*{4.1.2}

\textit{Prove that if $A$ is a transitive set, then $A^+$ is also a transitive set.}

Suppose that $A$ is transitive. That means that $(\forall x \in A)(x \subseteq A)$.
By definition, $A^+ = A \cup \{A\}$
Now let $y \in A^+$. We follow that $y = A \lor y \in A$. If $y = A$, then we follow that
$y \subseteq A^+$. If $y \in A$, then we follow that $y \subseteq A$ by the fact that
$A$ is transitive, therefore $y \subseteq A^+$. Thus we've got that
$$(\forall y \in A^+)(y \subseteq A^+)$$
thus we follow that $A^+$ is a transitive set, as desired.

\subsection*{4.1.3}

\textit{Prove that if $A$ is a transitive set if and only if $A \subseteq \pow(A)$.}

Suppose that $A$ is transitive. Then we follow that $(\forall x \in A)(x \subseteq A)$. Thus
if $x \in A$, then $x \subseteq A$ and therefore $x \in \pow(A)$. Thus we follow that
$A \subseteq \pow(A)$.

If $A \subseteq \pow(A)$, then we follow that if $x \in A$, then $x \in \pow(A)$ and therefore
$x \subseteq A$. Thus $A$ is transitive.


\subsection*{4.1.6}

\textit{Prove Proposition 4.1.9}

4.1.9 states that
$$\bigcup(A \cup B) = (\bigcup A) \cup (\bigcup B)$$

$$x \in \bigcup(A \cup B) \lra (\exists C \in A \cup B)(x \in C) \lra
(\exists C \in A)(x \in C) \lor (\exists C \in B)(x \in C) \lra$$
$$ \lra \in \bigcup A \lor x \in \bigcup B \lra x \in (\bigcup A) \cup (\bigcup B)$$


\subsection*{4.1.7}

\textit{Prove that $n \neq n^+$ for all $n \in \omega$}

This is true in general by regularity axiom, but let's apply some induction.
Let
$$I =  \{n \in \omega: n \neq n^+\}$$

$0 \neq \{0\}$, since $0 \in  \{0\}$ and $0 \notin 0$ (because it is an empty set). Thus $0 \in I$.

Suppose that $n \in I$. Then we follow that $n \neq n^+$.
Suppose that $n^+ = (n^+)^+$. Then we follow by 4.1.12 that $n = n^+$, which is a contradiction,
because $n \in I$. Thus we follow that $n \in I \ra n^+ \in I$, therefore $I$ is inductive,
therefore $I = \omega$, as desired.


\subsection*{4.1.8}

\textit{Prove that $n^+ \not \subseteq n$ for all $n \in \omega$}

Suppose that $n^+ \subseteq n$. Since $n^+ = n \cup \{n\}$, we follow that $n \subseteq n^+$.
Thus by double inclusion we've got that $n^+ = n$, which contradicts previous exercise.


\subsection*{4.1.9}

\textit{Prove that for all $m \in \omega$ and all $n \in \omega$, if $m \in n$, then
  $n \not \subseteq m$.}

Let
$$I = \{n \in \omega: m \in n \ra n \not \subseteq m\}$$

Since $m \in 0 \lra m \in \emptyset$ is always false, we follow that $0$ is vacuously in $I$.

Let $n \in I$. Let $m \in \omega$ be such that $m \in n^+$.
Thus $m \in n \cup \{n\}$. Thus $m \in n \lor m = n$.

If $m \in n$, then we follow that $n \not \subseteq m$, thus $(\exists y \in n)(y \notin m)$.
Thus $(\exists y \in n \cup \{n\})(y \notin m)$. Therefore $n^+ \not \subseteq m$.

If $m = n$, then we follow that $n^+ \not \subseteq n$ by exercise 4.1.8.
Thus we conclude that $n \in I \ra n^+ \in I$. Therefore $I$ is inductive and thus $I = \omega$,
as desired.

\subsection*{4.1.10}

\textit{Conclude from exercise 9 that $n \notin n$ for all $n \in \omega$.}

Suppose that $n \in n$. Then we follow by previous exercise
that $n \not \subseteq n$, which is bonkers.

\subsection*{4.1.11}

\textit{Let $A$ be a set and suppose that $\bigcup A = A$. Prove that $A$ is transitive
  and for all $x \in A$ there is a $y \in A$ such that $x \in y$}

Let $x \in A$ be arbitrary.
Then we follow that $x \subseteq \bigcup A$ and thus $x \subseteq A$. Thus we follow that
$A$ is transitive, as desired.

Suppoes that $x \in A$ and suppose that for all $y \in A$ we've got that $x \notin y$.
Then we follow that $(\forall y \in A)(x \notin y)$. Thus
$$(\forall y \in A)(x \notin y) \lra \neg (\exists y \in A)(x \in y) \lra x \notin \bigcup A \lra
x \notin A$$
which is a contradiction.

\section{The Recursion Theorem on $\omega$}

\textit{How would one modify the proof of Theorem 4.2.1 in order to establish the
  following: }

\textit{Theorem. Let $A$ be a set and let $a \in A$. Suppose that $f: \omega \times A \to A$
  is a function. Then there exists a unique function $h: \omega \to A$ such that
  $$h(0) = a$$
  $$h(n^+) = f(n, h(n))$$}

We can define $S$ and $h$ in normal fashion, then prove that $\eangle{0, a} \in h$ and
$\eangle{n, u} \in h$ will imply that $\eangle{n^+, f(n^+, h(n)} \in h$, thus concluding the
proof of claim 1.

Proof of claim 2 for the base case will not be altered. (IH) Will be modified in notation.


Proof of uniqueness does not depend on the domain of the functions and is true in general.

\subsection*{4.2.2}

\textit{Let $a \in A$ and $f: A \to A$ be a function such that $a \notin \ran(f)$. Suppose that
  $h: \omega \to A$ satisfies
  $$(1) h(0) = a$$
  $$(2) h(n^+) = f(h(n))$$
  Assume taht $h$ is one-to-one and onto A. Prove that $f$ is one-to-one.
}

Suppose that $f$ is not one-to-one. Then we follow that there exist $x \neq y \in A$
such that $f(x) = f(y)$. Since $h$ is onto, we follow that there exist $x', y' \in \omega$
such thar $h(x') = x$ and $h(y') = y$. Since $x \neq y$ we follow that $h(x') \neq h(y')$
and therefore $x' \neq y'$ because $h$ is a function.

Thus we follow that $x'^+ \neq y'^+$. Because $h$ is one-to-one we follow that
$$h(x'^+) \neq  h(y'^+)$$
thus
$$f(h(x')) \neq  f(h(y'))$$
$$f(x) \neq  f(y)$$
which is a contradiction. Thus we follow that $f$ is one-to-one.

\subsection*{4.2.3}

\textit{Let $A$ be a set and let $a \in A$. Suppose that $f: A \to A$ satisfies 
$$a \in f(a)$$
$$(\forall x, y \in A)(x \in y \to f(x) \in f(y))$$
Let $h: \omega \to A$ be states as in theorem 4.2.1. Prove that $h(n) \in h(n^+)$ for
all $n \in \omega$.}

We follow that $h(0) = a$ and $h(n^+) = f(h(n))$. For base case we've got that
$h(0) = a \in f(a) = h(0^+) = h(1)$.
Let $I$ be defined as a subset of $\omega$  such that $m \in I$ implies
that $h(m) \in h(m^+)$. Base case gives us that $0 \in I$. Suppose that
$n \in I$. Then we follow that
$$h(n) \in h(n^+)$$
Since $h(n), h(n^+) \in A$, we follow that
$$f(h(n)) \in f(h(n^+))$$
by definition of $h$ we follow that
$$h(n^+) \in h((n^+)^+)$$
thus $n^+ \in I$. Thus we follow that $I$ is an inductive set. Since $I \subseteq \omega$,
we follow that $I = \omega$, as desired.

\subsection*{4.2.4}

\textit{Let $F: A \to A$ be a function and let $y \in A$.}

\textit{(a) Prove that the class $S = \{B: B \subseteq A, y \in B, F[B] \subseteq B\}$}

From this definition we follow that $B \in S \ra B \subseteq A$. Thus $B \in S \ra
B \in \pow(A)$. Thus $S \subseteq \pow(A)$.Since $A$ is a set we follow that its power set is a set,
therefore the subset of a power set is also a set.

\textit{(b) Show that $S$ is nonempty}

We follow that $A \subseteq A$ by basic properties, $y \in B$ by the given restraints,
and $F[B] \subseteq A$ since $F: A \to A$, and therefore $\ran(F) \subseteq A$. Thus
we follow that $A \in S$, therefore $S$ is nonempty.

\textit{(c) Let $C = \bigcap S$. Prove that $y \in C$ and $F[C] \subseteq C$.}

We follow that
$$(\forall B \in S)(y \in B)$$
from definition of $S$. Suppose that $y \in F[C]$. Then we follow that
because $R[\bigcap G] \subseteq \bigcap\{R[C]: C \in G\}$ for any relation, we follow that
$$y \in \bigcap\{F[B]: B \in S\}$$
$$(\forall B \in S)(y \in F[B])$$
Since for every $B$ we've got that $F[B] \subseteq B$, therefore we can follow that
$$(\forall B \in S)(y \in B)$$
thus $y \in \bigcap S$. Thus $y \in F[C] \ra y \in C$. Thus $F[C] \subseteq C$, as desired.

\textit{(d) Prove that for all $B \subseteq A$, if $y \in B$ and $F[B] \subseteq B$, then
  $C \subseteq B$}

If $y \in B$ and $F[B] \subseteq B$, then we follow that $B \in S$. Since $C = \bigcap S$,
we follow that $(\forall x \in C) (x \in C \to x \in B)$. Thus $C \subseteq B$, as desired.

\textit{(e) Prove that $y \in F[C]$ if and only if $F[C] = C$}

If $F[C] = C$, then we follow that since $y \in C$, then $y \in F[C]$.

For $C$ we've got that $F[C] \subseteq C$. Thus $F[F[C]] \subseteq C$. Thus if $y \in F[C]$, then
$F[C]$ satisfies requirements presented in the previous point, and thus $C \subseteq F[C]$.
By double inclusion we follow that $F[C] = C$, as desired.

\section{Arithmetic on $\omega$}

\subsection*{Note}

Although not explicitly, but we've started to use the associative and commutative
properties of addition pretty liberaly at this point. To be more precise,
we explicitly define the notation of several pluses in a row to be left-associative (e.g.
$$a_1 + a_2 + a_3 = (a_1 + a_2) + a_3 = A(A(a_1, a_2), a_3) $$
is one such example). Associative and distributive properties guarantee us that the order in
which the functions are applied does not matter, and thus the notation is justified.

The order of operations of addition and multiplication is followed from the properties,
that are proven in the chapter, and we follow that
$$a \cdot b + c = (a \cdot b) + c = A(M(a, b), c)$$
thus the usual rules of order of operations applies.

In general, all of the operations are defined in terms of functions, and the usual notation
of addition and multiplication is nothing but a syntatic sugar. If in doubt, one can
prove that the usual notation makes sense pretty easily.

\subsection*{4.3.1}

\textit{Let $m \in \omega$. Suppose that $m + n = 0$. Prove that $m = n = 0$.}

Suppose that $n \neq 0$. Then we follow that there exists $j$ such that $j^+ = n$. Thus
$$m + n = m + j^+ = (m + j)^+ \neq 0$$
thus we get the contradiction.

\subsection*{4.3.2}

\textit{Let $m \in \omega$ and $n \in \omega$. Show that if $m \cdot n = 0$, then $m = 0$ or
  $n = 0$.}

Suppose that $m \neq 0$ and $n \neq 0$. Then we follow that there exist $n_p \in \omega$
such that $n = n_p^+$. Thus
$$m \cdot n = m \cdot n_p^+ = m \cdot n_p +  m$$
Since $m \neq 0 \in \omega$ and $m \cdot n_p^+ \in \omega$, we follow from previous exercise
that $m \cdot n_p = 0$ and $m = 0$, which contradicts the assumtions that $m \neq 0$.

\subsection*{4.3.3}

\textit{Let $m \in \omega$ and $n \in \omega$. Prove that for all $p \in \omega$, if
  $m + p = n + p$, then $m = n$.}

Let
$$I = \{a \in \omega: (\forall m, n \in \omega)(m + a = n + a \ra m = n)\}$$
If $p = 0$, then $m + 0 = m$, $n + 0 = n$, thus $m + 0 = n + 0$ implies that $m = n$.
Thus $0 \in I$.

(IH) Suppose that $p$ is such that $m + p = n + p$. Suppose that $m + p^+ = n + p^+$. Then
$$m + p^+ = (m + p)^+ = (n + p)^+ = n + p^+$$
thus we follow that $p \in I$ implies that  $p^+ \in I$. Thus $I$ is inductive, and therefore
$I = \omega$.

\subsection*{4.3.4}

\textit{Prove that for all $n \in \omega$, the inequality $0 \cdot n = 0$ holds.}

Suppose that $I$ is the set such taht
$I = \{n \in \omega: 0 \cdot n = 0\}$.
Since $0 \cdot 0 = 0$, we follow that $0 \in I$. Suppose that $0 \cdot n = 0$. Then we follow that
$$0 \cdot n^+ = 0 \cdot n + 0 = 0 + 0 = 0$$
Thus $n \in I \to n^+ \in I$. Thus $I = \omega$, as desired.


\subsection*{4.3.5}

\textit{Prove that for all $m \in \omega$ and $n \in \omega$, we have that
  $m^+ \cdot n  = m \cdot n + n$}

Suppose that
$$I = \{n \in \omega: (\forall m \in \omega)(m^+ \cdot n  = (m \cdot n) + n)\}$$

We follow from $M1$ that $m^+ \cdot 0 = 0$  and $(m \cdot 0) + 0 = m \cdot 0 = 0$
from $A1$ and $M1$. Thus $0 \in I$

Suppose that $n \in I$. Then we follow that
$$m^+ \cdot n^+ = m^+ \cdot n + m^+ = m \cdot n + n + m^+ = m \cdot n + n + m + 1 = $$
$$= m \cdot n + m + n + 1  = m \cdot n + m + n^+ = m \cdot n^+ + n^+$$
where theorems used are
$$M2 \ra IH \ra 4.3.4 \ra 4.3.4 \ra  4.3.10 \ra 4.3.4 \ra M2$$

\subsection*{4.3.6}

\textit{Let $n \in \omega$. Prove theorem 4.3.13}

Let
$$I = \{n \in \omega: n \cdot m = m \cdot n\}$$
We follow that $0 \cdot n = n \cdot 0 = 0$ from the previous exercises/theorems in the
chapter. Thus $0 \in I$.

Suppose that $n \cdot m = m \cdot n$. Then for $n^+$ we've got that
$$n^+ \cdot m = n \cdot m + m = m \cdot n + m = m \cdot n^+$$
where we've used previous exercise, IH, and $M2$ respectively. Thus we follow that
$n \in I \ra n^+ \in I$, thus $I = \omega$, as desired.

\subsection*{4.3.7}

\textit{Prove that if $n \in \omega$, then $n$ is either even or odd (details of definition
  of evenness in the exercise)}

Let
$$I: \{n \in \omega: (\exists k \in \omega: n = 2 \cdot k + 1 \lor n = 2 \cdot k\}$$

Since $0 = 2 \cdot 0$, we follow that $0 \in I$.

Suppose that $n \in I$. Then we follow that $n$ is even or odd (or both, we don't know yet).

If $n$ is even, then we follow that $n = 2 \cdot k$. Thus $n^+ = n + 1 = 2 \cdot k + 1$. Thus
$n^+$ is odd, therefore $n^+ \in I$.

If $n$ is odd, then we follow that $n = 2 \cdot k + 1$. Thus $n^+ = n + 1 = 2 \cdot k + 1 + 1 =
2 \cdot k + 2 = 2 \cdot k^+$. Thus we follow that $n^+ \in I$.

Thus we follow that $n \in I \ra n^+ \in I$, thus $I = \omega$, as desired.

\subsection*{4.3.8}

\textit{Let $I = \{n \in \omega: \neg((\exists k, j \in \omega)(n = 2 \cdot k \land
  n = 2 \cdot j + 1)\}$
  Prove that $I$ is inductive. Conclude that no natural number is both odd and even.}

Let
$I = \{n \in \omega: \neg((\exists k, j \in \omega)(n = 2 \cdot k \land n = 2 \cdot j + 1)\}$

We've shown earlier that $0$ is even. Suppose that $0$ is odd. Then we follow that
$$0 = 2 \cdot k + 1 = (2 \cdot k)^+$$
which is a contradiction.

Suppose that $n \in I$. Then we follow that
$$\neg((\exists k, j \in \omega)(n = 2 \cdot k \land n = 2 \cdot j + 1)$$
thus
$$\neg((\exists k, j \in \omega)(n^+ = (2 \cdot k)^+ \land n^+ = (2 \cdot j + 1)^+) \lra$$
$$\lra \neg((\exists k, j \in \omega)(n^+ = 2 \cdot k + 1 \land n^+ = 2 \cdot j + 2))\lra $$
$$\lra \neg((\exists k, j \in \omega)(n^+ = 2 \cdot k + 1 \land n^+ = 2 \cdot j^+))\lra$$
$$\lra \neg((\exists k, l \in \omega)(n^+ = 2 \cdot k + 1 \land n^+ = 2 \cdot l))$$
(i've changed a variable undere the quantifier for avoid confusion, which does not change the
meaning)
thus $n \in I \ra n^+ \in I$. Thus $I$ is inductive, and $I = \omega$. Therefore
every natural number is either odd or even, but not both.


\subsection*{4.3.9}

\textit{Prove for all $m, n, k \in \omega$ that $m^{n + k} = m^n \cdot m^k$}'

Let
$$I = \{k \in \omega: (\forall m, n \in \omega)(m^{n + k} = m^n \cdot m^k)\}$$

We follow that if $k = 0$ then
$$E(m, n + k) = E(m, n + 0) = E(m, n) = E(m, n) \cdot 1 = E(m, n) \cdot E(m, k)$$
thus we follow that $0 \in I$.

Suppose that $k \in I$. We follow that for $k^+$ we've got
$$E(m, n + k^+) = E(m, n + k) \cdot m = E(m, n) \cdot E(m, k) \cdot m =
E(m, n) \cdot E(m, k^+)$$
thus we follow that $k \in I \ra k^+ \in I$. Thus we follow that $I = \omega$, as desired.


\section{Order on $\omega$}

\subsection*{Note}

I'm using the symbol $\leq$ instead of "in or equal" because it's more convinient.
Theorems in the chapter provide us with the equivalence of two, and therefore the justification
for this notation.

\subsection*{4.4.1}

\textit{Let $n \in \omega$. Show that $1 \leq n^+$}

We know that for every $n \in \omega$, $0 \leq n$.
Thus $0 \leq n \ra 0^+ \leq n^+ \lra 1 \leq n^+$, as desired.

\subsection*{4.4.2}

\textit{Let $m, n \in \omega$. Show that if $m \in n^+$, then $m \leq n$.}

Suppose that $m \in n^+$. We follow that
$$m \in n^+ \lra m \in n \cup \{n\} \lra m \in n \lor m = n \lra m \leq n$$


\subsection*{4.4.3}

\textit{Let $n, a \in \omega$. Show that if $n \in a$, then $n^+ \leq a$}

Suppose that $n \in a$. We know that one of  $n^+ = a$, $n^+ \in a$ or $a \in n^+$ holds.
Suppose that $a \in n^+$. We follow that
$$a \in n \cup \{n\} \lra a \in n \lor a = n$$
If $a \in n$, then we've got that $n \in a$ and $a \in n$, which is impossible. If $a = n$,
then $n \in n$, which is also not good. Thus we follow that $a \in n^+$ is impossible.
Thus we follow that $n^+ \in a$ or $n^+ = a$. In other words, $n \leq a$, as desired.


\subsection*{4.4.4}

\textit{Let $I$ be inductive and let $a \in \omega$. Prove that
  $\{n \in I: n \in a \lor a \subseteq n\}$ is inductive }

Let
$$ Q = \{n \in I: n \in a \lor a \subseteq n\}$$
We follow that $0 \in Q$ since $0$ is in every inductive set (by virtue of being a natural number),
and by trichotomy property $0 \in a$ or $a \leq 0 \lra a \subseteq 0$.

Suppose that $q \in Q$. Since $Q \subseteq I$, we follow that $q \in I$, and
because $I$ is inductive, we follow that $q^+ \in I$ as well.
We also follow that $q \in a \lor a \subseteq q$.

If $a \subseteq q$, then we follow that $a \subseteq q \cup \{q\}$, therefore $a \subseteq q^+$,
and thus $q \in Q$.

Suppose that $q \in a$. Then we follow that since $a \in \omega \to a \subseteq \omega$,
we follow that $q \in \omega$. Therefore $q^+ \in \omega$, and
we've got by trichotomy that either $q^+ \in a$, $q^+ = a$ or $a \in q^+ \lra a \subset q^+$.
Thus we follow that $q^+ \in a \lor a \subseteq q^+$, therefore $q^+ \in Q$.

Thus we can conclude that $0 = \emptyset \in Q$ and $q \in Q \ra q^+ \in Q$. Thus $Q$ is
inductive, as desired.


\subsection*{4.4.5}

\textit{Let $m, n \in \omega$. Suppose that $m \leq n$. Prove that $\max(m, n) = n$.}

From the chapter we follow that $\max(m, n) = m \cup n$. Since $m \leq n$, we follow that
$m \subseteq n$. Thus $\max(m, n) = m \cup n = n$, as desired.

\subsection*{4.4.6}

\textit{Let $m, n \in \omega$. Prove that for all $p \in \omega$ if $m \in n$, then
  $m + p \in n + p$}

Let
$$I = \{p \in \omega: (\forall m, n \in \omega)(m \in n \ra m + p \in n + p\}$$
Since $m + 0 = m$ and $n = n + 0$, we follow that $m \in n \lra m + 0 \in n + 0$.

Suppose that $p \in I$. Then we follow that
$$m \in n \lra m + p \in n + p \lra (m + p)^+ \in (n + p)^+ \lra m + p^+ \in n + p^+$$
thus $p^+ \in I$. Thus $I = \omega$, as desired.

\subsection*{4.4.7}

\textit{Let $m, n \in \omega$. Prove that for all $p \in \omega$, if $m + p \in n + p$, then
  $m \in n$}

In previous exercise we've got only biconditionals, thus we get the reverse case as well

\subsection*{4.4.8, 4.4.9}

\textit{let $m, n \in \omega$. Prove that for all $p \in \omega$ $m \in n \lra
  m \cdot p^+ \in n \cdot p^+$}

Let
$$I = \{p \in \omega: (\forall m, n \in \omega)( m \in n \lra  m \cdot p^+ \in n \cdot p^+)\}$$

We first follow that if $p = 0$, then
$$m \cdot 0^+ \in n \cdot 0^+ \lra (m \cdot 0 + m) \in (n \cdot 0 + n) \lra (0 + m) \in (0 + n)
\lra m \in n$$
thus $0 \in I$.

Suppose that $p \in I$. Then we follow that
$$m \in n \lra m \cdot p^+ \in n \cdot p^+$$

$m \in n$ if and only if $m + a \in n + a$ by previous exercise. Thus
$$m \in n \lra m + m \cdot p^+ \in n + m \cdot p^+$$
and also
$$ m \cdot p^+ \in n \cdot p^+ \lra  m \cdot p^+ + n \in n \cdot p^+  + n$$
by commutativity we follow that
$$ m \cdot p^+ \in n \cdot p^+ \lra  n + m \cdot p^+ \in n \cdot p^+  + n$$
thus
$$m \in n \lra (m + m \cdot p^+ \in n + m \cdot p^+) \land (n + m \cdot p^+ \in n \cdot p^+  + n)
\lra$$
$$\lra (m + m \cdot p^+ \in n + m \cdot p^+) \land
(n + m \cdot p^+ \subset n \cdot p^+  + n) \ra$$
$$ m + m \cdot p^+ \in n \cdot p^+  + n \lra m \cdot (p^+)^+ \in n \cdot (p^+)^+$$
thus $m \in n \ra m \cdot (p^+)^+ \in n \cdot (p^+)^+$.

Let $m, n \in \omega$ be such that
$m \cdot (p^+)^+ \in n \cdot (p^+)^+$. By trichotomy we
follow that $m \in n$, $n \in m$ or $m = n$.

If $m = n$, then we follow that
$$n \cdot (p^+)^+ = m \cdot (p^+)^+ \in n \cdot (p^+)^+$$
which is a contradiction.

If $n \in m$, then we follow by implication that 
$$n \cdot (p^+)^+ \in m \cdot (p^+)^+$$
and since $m \cdot (p^+)^+ \in n \cdot (p^+)^+$, we follow that
$n \cdot (p^+)^+ \in n \cdot (p^+)^+$, which is also bad. Thus we follow that
$m \cdot (p^+)^+ \in n \cdot (p^+)^+ \ra m \in n$. By including the implication we follow that
$$m \cdot (p^+)^+ \in n \cdot (p^+)^+ \lra m \in n$$
and therefore $p^+ \in I$. Thus $I = \omega$, as desired.

\subsection*{4.4.10}

\textit{Use Theorem 4.4.11 and Theorem 4.4.9 to prove Corollary 4.4.12}

Suppose that $m + p = n + p$. If $m \in n$, then we follow that $m + p \in n + p$, which
is not the case.
If $n \in m$, then we follow that $n + p \in m + p$, which is also not the casee.
Thus we follow that $m = p$. Thus $m + p = n + p \lra m = n$ (reverse case is trivial)

Suppose that $p \neq 0$ and we've got that $m \cdot p = n \cdot p$. Since $p \neq 0$, we
follow that $p = j^+$ for some $j \in \omega$.

We follow that $m \in n$ and $n \in m$ are both impossible, because they would imply
that $m \cdot p \in n \cdot p$ and $n \cdot p \in m \cdot p$, which are not the case. Thus
we follow that
$$(\forall p \in (\omega \setminus 0))(m \cdot p = n \cdot p \lra m = n)$$
where reverse case is once again trivial.

\subsection*{4.4.11}

\textit{Let $m \in \omega$. Prove that $m \in m + p^+$ for all $p \in \omega$.}

We know that $m \in m^+$ for all $m \in \omega$. Thus we follow that for any $p \in \omega$
we've got that $m + p \in m^+ + p \lra m + p \in m + p^+$. We also can follow that
for every $p \in \omega$ we've got that $m \leq m + p$ (can't find the exact reference,
but it is easily followed by induction). Thus by transitive property we can follow
that  $m \in m + p^+$, as desired.

\subsection*{4.4.12}

\textit{Let $m \in \omega$. Prove that for all $n \in \omega$, if $m \in n$ then
  $m + p^+ = n$ for some $p \in \omega$.}

Let
$$I = \{j \in \omega: j \in m \lor (\exists p \in \omega)(j = m + p^+) \}$$
We follow that $0 \in m$, therefore $0 \in I$.

Suppose that $n \in I$. Then we follow that $n \in m \lor (\exists p \in \omega) (j = m + p^+)$.
If $n \in m$, then we follow that $n^+ \in m \lor n^+ = m \lra
n^+ \in m \lor n^+ = m + 0 $. Thus we follow that if $n \in m$, then $n^+ \in I$.

If $(\exists p \in \omega)(n = m + p^+)$, then we follow that $n^+ = (m + p^+)^+ = m + (p^+)^+$.
Since $p^+ \in \omega$, we follow $n^+ \in I$ as well.

Thus we can conclude that $n \in I \ra n^+ \in I$. Thus $I$ is inductive, and since it is
a subset of $\omega$, we follow that $I = \omega$. Thus we follow that our initial $n \in I$.
Thus $n \in m \lor (\exists p \in \omega)(n = m + p^+)$. Because $m \in n$, we follow
by trichotomy that $n \notin m$, thus $ (\exists p \in \omega)(n = m + p^+)$, as desired.

\subsection*{4.4.13}

Combination of previous two exercises into one biconditional.

\subsection*{4.4.18}

\textit{Prove for all $n \in \omega$, if $f: n \to \omega$, then $\ran(f) \subseteq k$ for
  some $k \in \omega$.}


Let
$$I = \{n \in \omega: (\forall f)
(\exists k \in \omega)(f: n \to \omega \ra \ran(f) \subseteq k)\}$$

We follow that if $f: 0 \to \omega$, then $\dom(f) = \emptyset$, thus $f = \emptyset$, therefore
$\ran(f) = \emptyset = 0 \subseteq 0$. Thus $0 \in I$.

Suppose that $n \in I$. Let $f: n^+ \to \omega$. Then we follow that
there exists $f|n: n \to \omega$, and thus $(\exists k \in \omega)(\ran(f|n) \subseteq k)$.
Since $n^+ = n \cup \{n\}$, we follow that
$$\ran(f) = f[n^+] = f[n] \cup f[\{n\}] = f[n] \cup f(n) \subseteq k \cup f(n)$$
Since $f: n^+ \to \omega$, we follow that $f(n) \in \omega$. Thus $k \cup f(n) \in \omega$,
thus there exists $j = k \cup f(n) \in \omega$ such that
$f: n^+ \to \omega \ra \ran(f) \subseteq j$. Thus we follow that $n^+ \in I$.
Thus $n \in I \ra n^+ \in I$, therefore $I = \omega$, as desired.


\chapter{On the Size of Sets}

Because of chapter 4'th construction of natural numbers and their ordering I think
that we're justified to use basic number theory and some properties of the
natural numbers, since it follows directly from the proven facts.
One can argue even that we're justified to use $Z$ in general, but until
the need arises I will refrain from doing so.

\section{Finite Sets}

\subsection*{5.1.1}

\textit{In our proof of Theorem 5.1.3, under the case that $k \in m$, we did not show
that the function g is one-to-one. Complete the proof by proving that g is
one-to-one.}

We need to prove that
$$g(a) =
\begin{cases}
  k \text{ if } f(a) = m \\
  f(a) \text{ if } f(a) \neq m \\
\end{cases}
$$
is one-to-one.

Suppose that $a_1, a_2 \in A$ are such that $a_1 \neq a_2$.
Suppose that $f(a_1) \neq m$. Then we follow that if $f(a_2) \neq m$, then
we've got that $f(a_1) \neq f(a_2)$ by the fact that $f$ is injective, thus
$g(a_1) = f(a_1) \neq f(a_2) = g(a_2)$ as well.
If $f(a_2) = m$, then we follow that $g(a_2) = k$. Thus we follow that
$f(a_1) = g(a_1) \in \ran(f)$ and $g(a_2) = k \notin \ran(f)$. Thus we follow that
$g(a_1) \neq g(a_2)$.

Since $f$ is injective, we can follow that $a_1 \neq  a_2 \ra f(a_1) \neq f(a_2)$, thus
we follow that the case when $f(a_1) = f(a_2) = m$ can't happen. Thus we follow that
$a_1 \neq a_2 \ra g(a_1) \neq g(a_2)$, therefore $g$ is injective, as desired.

\subsection*{5.1.2}

\textit{Suppose that the set $B$ is finite and $A \subseteq B$. Prove that $A$ is finite.}

We follow that there exists $n \in \omega$ $|B| = n$. Thus we follow that
there exists $f: B \to n$, that is injective. Since $f$ is injective and $A \subseteq B$,
we follow that $f|A$ is also one-to-one. Thus we follow that $A$ is finite, as desired.

\subsection*{5.1.3}

\textit{Let $A$ and $B$ be two finite sets, and let $f: A \to B$ be one-to-one. Prove
  that $|A| \leq |B|$. }

Given that $A$ and $B$ are finite, we follow that there exist natural numbers $n = |A|$ and
$m = |B|$. We can also follow that there exist bijections $h: A \to n$ and $g: B \to m$.
Because $h$ is a bijection, we follow that $h \inv: n \to A$ is also a bijection.
Thus we follow that $g \circ f \circ h \inv: n \to m $ is a composition of injective functions,
and therefore it's injective as well. Thus we follow that $|n| \leq m$, therefore
$$||A|| \leq |B|$$
$$|A| \leq |B|$$
as desired.

\subsection*{5.1.4}

\textit{Prove theorem 5.1.7 }

Let $A$ and $B$ be finite. Let $n = |A|$ and $m = |B|$. We follow that there exist bijections
$f: A \to n$ and $g: B \to m$.

Thus we follow that if there exists a bijection $h: A \to B$,
then we follow that $g \circ h \circ f \inv: n \to m$ is a bijection. Thus
$g \circ h \circ f \inv: n \to m$ is injective and
$(g \circ h \circ f \inv: n \to m)\inv$ is injective, thus we follow that $n \geq m$ and
$m \geq n$, thus by antisymmetry we've got that $n = m$ and $|A| = |B|$.

If $|A| = |B|$, then we follow that $f: A \to |A|$ and $g: B \to |B|$ are bijections,
and thus $g \inv \circ f: A \to B$ is also a bijection.

\subsection*{5.1.5}

\textit{Let $n \in \omega$ and let $f: n \to A$. Prove that $f[n]$ is finite.}

Let us firstly prove a theorem (which I'm sure that it was proven in the book before,
but I can't find it) assuming for a moment that $n$ is a general set.

\textbf{If $f: n \to A$ is a function, then there exists an injective function $g$
such that $g \subseteq f\inv$.}

Suppose that $y \in f[n]$. Then we follow that there exists a nonempty set
$$A_y = \{x \in n : \eangle{x, y} \in f\}$$
Thus we can define an indexed function $\eangle{A_y: y \in f[n]}$, from which by axiom of
choice we follow that there exists a function $g$ such that 
$$g(y) \in A_y$$
Thus we follow that $g: f[n] \to n$. From a definition of $A_y$ we can follow that
$g \subseteq f\inv$. Suppose that $n_1 \neq n_2 \in f[n]$. We follow that because $f$ is a
function, there does not exist $x \in n$ such that $f(x) = n_1$ and $f(x) = n_2$. Thus
$g(n_1) \neq g(n_2)$.
Thus we follow that $n_1 \neq n_2 \ra g(n_1) \neq g(n_2)$. Therefore $g$ is injective, as desired.

By provided theorem we can follow that there exists an injective function $g$ from $f[n]$ to $n$.
Thus we follow that $f[n]$ is finite by definition.

\subsection*{5.1.6}

\textit{Let $A$ be finite and let $f: A \to \omega$. Prove that $\ran(f) \subseteq k$, for
  some $k \in \omega$.}


Since $A$ is finite, we follow that there exists a bijection $g: A \to n$
for some $n \in \omega$. Thus we follow that $g \inv: n \to A$ is also a bijection.
We now follow that $f \circ g\inv: n \to \omega $ is a function with a finite domain, and
because $g\inv$ is a bijection, we follow that $\ran(f \circ g\inv) = ran(f)$.
By exercise 4.4.18 we follow that there exists $k \in \omega$ such that
$\ran(f \circ g\inv) \subseteq k$. Thus we follow that $\ran(f) \subseteq k$ for some
$k \in \omega$, as desired.

\subsection*{5.1.7}

\textit{Suppose that $f: A \to n$ is a bijection, where $n \in \omega$. Ptove that $|A| = n$.}

Since $f$ is a bijection, we follow that it is injective. Thus we follow that $A$ is finite.
Let $|A| = m$. We follow that there exists a bijection $g: A \to m$ and thus $g \inv: m \to A$
is also a bijection. Thus
$$f \circ g \inv: m \to n$$
is also a bijection. Thus we follow that $|n| = |m|$. Therefore $n = m$, and thus $|A| = n$,
as desired.

\subsection*{5.1.8}

\textit{Let $f: \omega \times \omega \to \omega$ be defined by $f(i, j) = 2^i \cdot 3^j$.
  Prove that $f$ is one-to-one. Prove that if $i < m$ and $j < n$, then $f(i, j) \leq f(m, n)$.}

I think that at this point we're kind of justified to use the normal stuff from the number
theory, such as divisibility and whatnot (given the fact that we've justified all of the
normal things such as ), and deduce from it that given function is indeed
one-to-one. The fact that $i < m$ and $j < n$ follows easily from induction on $i$ and then
by induction on $j$.

\subsection*{5.1.9}

\textit{Prove that $p: \omega \times \omega \to \omega$ defined by
  $$p(i, j) = 2^i(2j + 1) - 1$$
  is one-to-one and onto $\omega$. Prove that if $i \in M$ and $j \in n$,
  then $p(i, j) < p(m, n)$.}

Suppose that $i, j, i', j' \in \omega$ and $i > i', j > j'$. Then we follow that
$$p(i, j) - p(i', j') = 2^i(2j + 1) - 2^{i'}(2j' + 1) =
2^i 2j + 2^i - 2^{i'} 2j' - 2^{i'} = $$
from which we follow that $2^i > 2^{i'}$ because exponentiation function is strictly increasing
for bases greater then $1$.
We can also follow that $j > j'$ implies that $2j > 2j'$, and becase we've seen earlier that
$2^i > 2^{i'}$, we follow that $2^i 2j  > 2^{i'} 2j'$. Thus we follow that
$$i > i' \land j > j' \ra p(i, j) > p(i', j')$$
and
$$i = i' \land j > j' \ra p(i, j) > p(i', j')$$
$$i > i' \land j = j' \ra p(i, j) > p(i', j')$$
thus we follow that
$$\eangle{i, j} \neq \eangle{i', j'} \ra p(i, j) \neq p(i', j')$$
thus the function is injective.

Now suppose that $n \in \omega$. Then we follow that
$$2^i(2j + 1) - 1 = n$$
$$2^i(2j + 1)  = n + 1$$
we follow that since $n  + 1 > 0 \ra n \in Z^+$,
there exist a finite set of prime divisors $P = \{p_1, p_2, ..., p_q\}$ and set of powers
$K = \{k_1, ..., k_l\}$ such that
$$n + 1 = \prod{p_l^{k_l}}$$
from which we follow that we can set $i$ to be the power of $2$ in case that $2 \in P$, or
$0$ in case that $2 \notin P$ to get 
$$n + 1 = 2^i\prod_{P \setminus 2}{p_j^{k_j}}$$
since $2 \notin P \setminus 2$, we follow that $\prod_{P \setminus 2}{p_j^{k_j}}$ is an odd
number, and thus there exists $j$ such that $2j + 1 = \prod_{P \setminus 2}{p_j^{k_j}}$. Thus
we follow that
$$n + 1 = 2^i(2j + 1)$$
$$n = 2^i(2j + 1) - 1$$
thus we follow that $p(i, j)$ is a surjective function. Thus $p$ is bijective, as desired.

\subsection*{5.1.10}

\textit{Prove that there exists a one-to-one function $f: \omega \to \omega \times \omega$
  that is onto $\omega \times \omega$}

Take a function from previous exercise and take its inverese. We follow that because it is a
bijection, then its inverse is a function and a bijection, thus we've got the desired
function, as desired.

\subsection*{5.1.11}

\textit{Suppose that $A$ and $B$ are finite disjoint sets. Prove that $A \cup B$ is finite.}

We follow that since both of them are finite, we've got that there exist bijections
$f: |A| \to A$ and $g: |B| \to B$. Thus we follow that we can relation $h : |A| + |B| \to A \cup B$
by
$$h(x) = 
\begin{cases}
  x < |A| \ra f(x) \\
  x \geq |A| \ra g(x - |A|)
\end{cases}
$$
(I've used a bit of a notation abuse here, but the principle is pretty clear)
We follow that if $x \in |A| + |B|$, then $x < |A| \oplus x \geq |A|$. Thus we follow that
$h(x)$ is a well-defined funcion. Now suppose that $x, y \in |A| + |B| \land x \neq y$.
If $x, y < |A|$ or $x, y \geq |A|$, then we follow that
$$\eangle{h(x), h(y)} = \eangle{f(x), f(y)} \oplus  \eangle{h(x), h(y)} = \eangle{g(x), g(y)}$$
from which we follow that $h(x) \neq h(y)$. If $x < |A|$ and $y \geq |A|$ (or vice-versa),
then we follow that $x \in A$ and $y \in B$, and since $A$ and $B$ are disjoint,
we follow that $x \neq y$.
Thus we can conclude that $x \neq y \ra h(x) \neq h(y)$, and therefore the function is injective.

Now suppose that $x \in A \cup B$. If $x \in A$, we follow that there exists
$n \in |A| \ra n \in |A| + |B|$ such
that $h(n) = f(n) = x$.
If $x \in B$, we follow that there exists $n \in |B|$ such that $g(n) = x$.
$n \in |B|$ implies that $n + |A| \in |A| + |B|$, and thus there exists
$k = n + |A| \in |A| + |B|$ such that
$$h(k) = g(n + |A| - |A|) = g(n) = x$$
thus we follow that $x \in A \cup B \ra (\exists k \in |A| + |B|)(h(k) = x)$. Thus we follow that
$h$ is surjective.

Thus we follow that $h: |A| + |B| \to A \cup B$ is bijective, and since $|A| + |B| \in \omega$,
we follow that $A \cup B$ is finite, as desired.

By setting $A = A'$ and $B = B' \setminus A$, we can conclude that union of two arbitraty
finite sets $A', B'$ is finite as well.

\subsection*{5.1.12 }

\textit{Let $A$ be an infinite set abd $B$ be a finite set. Prove that $A \setminus B$ is
  infinite.}

Suppose that $A \setminus B$ is finite.
Then we follow that $A \setminus B$ and $B$ are disjoint sets,
thus $(A \setminus B) \cup B$ is finite by previous exercise. Since
$$(A \setminus B) \cup B = A \cup B$$
we follow that $A \cup B$ is finite. Thus $A \subseteq A \cup B$ is finite as well, which is
a contradiction.

\subsection*{5.1.13}

\textit{Let $k \in \omega$. Suppose that $|A| = k^+$ and $a \in A$. Prove that
  $|A \setminus \{a\}| = k$.}

One of the implications of exercise 5.1.12 is that if $A$ and $B$ are finite and disjoint,
then $|A \cup B| = |A| + |B|$. Applying this thing to our exercise we follow that
$$|A| = |A \setminus \{a\} \cup \{a\}| = k^+$$
Without providing trivial proof we gonna state here that 
$$|\{a\}| = 1$$
thus we conclude that
$$|A \setminus \{a\}| + |\{a\}| = k^+$$
$$|A \setminus \{a\}| + 1 = k + 1$$
$$|A \setminus \{a\}| = k$$
as desired.

\subsection*{5.1.14}

\textit{Suppose taht $|K| = k$, $|N| = n$, and $f: K \to N$ is onto $N$ where $k, n \in \omega$.
  Show that there is a function $g: k \to n$ that is onto}

We follow that there exist bijections $h: k \to K$ and $j': n \to N$. Since $j'$ is a bijection
we're justified to define bijection $j: N \to n; j = j'\inv$. Thus we can define $g: k \to n$
by
$$g = j \circ f \circ h $$
since all of the $j, f, h$ are surjective, we follow that $g$ is surjective as well, as desired.

\subsection*{5.1.15}

\textit{Suppose that $A$ and $B$ are finite. Prove that $A \cup B$ is finite.}

GOTO last paragraph of 5.1.12

\subsection*{5.1.16}

\textit{Let $n \in \omega$. Suppose that $f: n \to A$ is onto $A$. Prove that $A$ is finite.}

We follow that since $f$ is surjective, there exists injective $g: A \to n$ such that
$g \subseteq f\inv$. Thus we follow that $A$ is finite, as desired.

\subsection*{5.1.17}

\textit{Let $A$ be finite set. Prove that there exists an $f: n \to A$ that is onto $A$
  for some $n \in \omega$.}

We know that there exists a bijection $f: |A| \to A$, thus concluding the exercise.

\subsection*{5.1.18}

\textit{Suppose that $f: A \to B$ is onto $B$ where $A$ is finite. Prove that $B$ is finite.}

Since $A$ is finite, we follow that there exists a bijection $g: |A| \to A$. Thus there
exists a surjective function $f \circ g : |A| \to B$, which by 5.1.16 means that $B$ is finite,
as desired.

\subsection*{5.1.19}

\textit{Let $n \in \omega$ and $\{A_i: i \in n\}$ be such that $A_i$ is finite for all
  $i \in n$. Prove, by induction, that $\bigcup_{i \in n}{A_i}$ is finite.}

Let
$$I = \{n \in \omega: \bigcup_{i \in n}{A_i} \text{ is finite}\}$$
(we can make a more formal definition here, but it'll be a bit coumbersome).

We follow that if $n = 0$, then $\{A_i: i \in n\} = \emptyset$, thus we follow that
$$\bigcup_{i \in n}{A_i: i \in n} = \emptyset$$
thus
$$|\bigcup_{i \in n}{A_i: i \in n}| = 0$$
thus we follow that $0 \in I$.

Suppose now that $n \in I$. Then we follow that $\bigcup_{i \in n}{A_i}$ and $A_{n^+}$ are finite,
thus $\bigcup_{i \in n^+}{A_i}$ is finite as well. Thus we've got the desired conclusion.

\subsection*{5.1.20}

\textit{Suppose that $A$ and $B$ are finite sets. Prove that $A \times B$ is finite.}

We follow that $\pow(A)$ is finite, $\pow(B)$ is also finite, thus $\pow(A) \cup \pow(B)$
is finite, therefore $\pow(\pow(A) \cup \pow(B))$ is finite, and since
$A \times B \subseteq \pow(\pow(A) \cup \pow(B))$, we follow that $A \times B$ is finite
as well.

\subsection*{5.1.21 }

\textit{Prove Corollary 5.1.11}

Suppose that $G: \omega \to A$ is one-to-one. Suppose that $A$ is finite. Then we follow that
there exists a bijection $f: A \to |A|$. Therefore we follow that
$f \circ G: \omega \to |A|$ is injective, and thus $\omega$ is a finite, which is a contradiction.
Thus we follow that $A$ is not finite (i.e. infinite), as desired.

\subsection*{5.1.22}

\textit{Let $\eangle{A_i: i \in I}$ be an indexed function where $I$ is a finite set and
  $A_i \neq \emptyset$. Without using AoC prove by induction on $|I|$ that there is
  an indexed function $\eangle{x_i: i \in I}$ such that $x_i \in A_i$ for all $i \in I$.}

Define $W$ to be the set, for which the conditions hold.

We follow that if $|I| = 0$, then $I = \emptyset$, thus $\eangle{A_i: i \in I} = \emptyset$,
and therefore $\emptyset$ is an indexed function $\eangle{x_i: i \in I}$ as well, thus
$i \in W$.

Suppose that $n \in W$. We follow that we can have  $I'$ such that
$|I'| = n + 1$ and indexed function $\eangle{A_i: i \in I'}$. Since $I'$ is finite, we follow that
there exists a bijection $f: n + 1 \to I'$, therefore there exists an element $f(0) \in I'$.
Define $a = f(0)$. We can follow that $|I' \setminus \{a\}| = n$ We follow that for
$$\eangle{A_i: i \in I' \setminus \{a\}}$$
there exists
$$\eangle{x_i: i \in I' \setminus \{a\}}$$
Let us now look at $A_{\{a\}}$. We follow that since $A_{\{a\}} \neq \emptyset$ we can follow that
there exists $b \in A_{\{a\}}$. Define a function $h: I' \to \bigcup_{i \in I'}{A_i}$ by
$$h(x) = 
\begin{cases}
  x = a \ra b \\
  x \neq a \ra \eangle{x_i: i \in I' \setminus \{a\}}
\end{cases}
$$
then we follow that $h$ is the desired indexed function, and thus $n^+ \in W$. Therefore
$W = \omega$, as desired.

\subsection*{5.1.23}

\textit{Modify the proof of Theorem 5.1.14 to show that if $|A| = n$, then $|\pow(A)| = 2^n$.}

Not gonna provide it, but we can follow that by making definition of $I$ a bit more strict and
following the same logic for the argument.

\section{Countable Sets}

\subsection*{5.2.1}

\textit{Let $A = \{4, 8, 12, 16, ...\}$ and let $B = \{n \in Z: n < -25\}$. Define a bijection
  $f: A \to \omega$ and define a bijection $g: B \to \omega$.}

$$f(x) = x/4 - 1$$
$$g(x) = -n + 26$$

\subsection*{5.2.2}

\textit{Let $A$ and $B$ be as in Exercise 1. Define an injection $h: A \cup B \to \omega$.}

Becase $A$ and $B$ are disjoint we can set
$$h(x) =
\begin{cases}
  x = 2x \ra f\inv(x / 2) \\
  x = 2x + 1 \ra g\inv((x - 1) / 2) 
\end{cases}
$$

\subsection*{5.2.3}

\textit{Show that the set of negative integers $Z^-$ is countable}

We can define $f: Z^- \to \omega$ by $f(x) = -x - 1$, which will be a bijection between
$Z^-$ and $\omega$. Thus we conclude that $Z^-$ is countable.

\subsection*{5.2.4}

\textit{Conclude from Exercise 3 that set of integers $Z$ is countable.}

We can follow that sets $Z^-, N$ are both countable, thus $Z^- \cup N = Z$ is countable as well.

\subsection*{5.2.5}

\textit{Let $p: \omega \times \omega \to \omega$ be one-to-one. Using $p$, show that
  the set of positive rationals is countable. Conclude that the set of negative rationals
  is countable and the set of rationals is countable.}

We can define an injection from $Q^+$ to $\omega \to \omega$ by
$$h(m/n) = \eangle{m, n}$$
where $m/n$ is in the lowest possible terms.
From exerise 5.1.9 we follow that there exists an injection $p$ from $\omega \times \omega$
to $\omega$. Thus we conclude that $p \circ h$ is an injection from $Q^+$ to $\omega$, thus
$Q^+$ is countable.

Because we can define a bijection from $Q^+$ to $Q^-$ by $f(x) = -x$ we follow that $Q^+$
is countable as well. Thus we follow that $Q^+ \cup Q^+ \cup \{0\} = Q$ is a countable union
of countable sets, and therefore is is countable as well, as desired.

\subsection*{5.2.6}

\textit{Let $A$ and $B$ be two countably infinite sets. Prove that there exists a bijection
  $f: A \to B$.}

Since $A$ and $B$ are both countably infitite we follow that there exist bijections
$h: A \to \omega$ and $g: B \to \omega$. Thus if we set
$f = g\inv \circ h $ we get the desired bijection.

\subsection*{5.2.7}

\textit{Let $A$ and $B$ be countable sets. Prove that $A \times B$ is countable.}

We follow that there exist injections $f: A \to \omega$ and $g: B \to \omega$. Thus
we can define an injection $p': A \times B \to \omega \to \omega$ by
$$p'(x, y) = \eangle{f(x), g(y)}$$
which we can plus into bijective $p: \omega \times \omega \to \omega$ to get injective
$p \circ p': A \times B \to \omega$, which proves that $A \times B$ is countable.

\subsection*{5.2.8}

\textit{Let $A$ be a set. Suppose that $f: \omega \to A$ is onto $A$. Prove that $A$ is
  countable}

We follow that there exists an injective function  $g \subseteq f\inv: A \to \omega$,
which proves that $A$ is countable, as desired.

\subsection*{5.2.9}

\textit{Let $A$ be a nonempty countable set. Prove that there exists an $f: \omega \to A$
  that is onto $A$.}

Because $A$ is nonempty, we follow that there exists $a \in A$. 
We follow that there exists an injection $f: A \to \omega$, therefore we can define a
surjective function $h: \omega \to A$
$$h(x) =
\begin{cases}
  x \in \ran(f) \ra f\inv[\{x\}] \\
  x \notin \ran(f) \ra a 
\end{cases}
$$
as desired.

\subsection*{5.2.10}

\textit{Suppose that $f: A \to B$ is onto $B$ where $A$ is countable. Prove that $B$ is
  also countable.}

If $A = \emptyset$, then we follow that $B$ is also an emptyset, therefore $B$ is countable.

If $A$ is not empty, then we follow by previous exercise that there exists a surjective
$g: \omega \to A$. Thus we follow that
$$f \circ g: \omega \to B$$
is surjective, therefore by exercise 5.2.8 we follow that $B$ is countable, as desired.


\subsection*{5.2.11}

\textit{Let $A$ be an infinite set and let $C = \pow(A) \setminus \{\emptyset\}$.
  By AoC there exists a function $g: C \to A$ such that $g(B) \in B$ for all $B \in C$.
  One can show that there is a function $h: \omega \to A$ such that
  $$(\forall n \in \omega )(h(n) = g(A \setminus h[n]))$$
  Prove that $h: \omega \to A$ is one-to-one}

Suppose that $x < y \in \omega$ and $h(x) = h(y)$. We follow that $x \in y$,
therefore
$$h(x) = h(y) = g(A \setminus h[y]) \in A \setminus h[y]$$
Since $x \in y$ we follow that $h(x) \in h[y]$, therefore we have a contradiction.


\section{Uncountable Sets}

\subsection*{5.3.1}

\textit{Let $A$ be uncountable. Prove that $A \times B$ is uncountable for any nonempty
  set $B$.}

Let $b \in B$. Define a function $f: A \to A \times B$ by
$$f(a) = \eangle{a, b}$$
We follow that for $x, y \in A$
$$f(x) = f(y) \ra \eangle{x, b}  = \eangle{y, b} \ra x = y$$
thus we follow that $f$ is injective. Therefore we conclude that $A \times B$ is
uncountable, as desired.

\subsection*{5.3.2}

\textit{Let $A$ be uncountable and $B$ be countable. Prove that $A \setminus B$ is
  uncountable.}

Suppose that $A \setminus B$ is countable. Then we follow that $A = A \setminus B \cup B$
is a countable union of countable sets, therefore it is countable as well, which
is a contradiction.

\subsection*{5.3.3}

\textit{Prove that the set of irrationals $R \setminus Q$ is uncountable.}

$R$ is uncountable, $Q$ is countable, thus by previous exercise we follow this
conclusion.

\subsection*{5.3.4}

\textit{Suppose that $A$ is uncountable and $A \subseteq B$. Prove that $B$ is
  uncountable.}

We follow that since $A \subseteq B$ we've got an injection $f: A \to B$
which is a restricted identity on $A$. Thus we follow that $B$ is uncountable, as desired.

\subsection*{5.3.5}

\textit{Prove that $\omega^\omega$ is uncountable}

We follow that $\{0, 1\}^\omega$ is a subset of $\omega^{\omega}$ and because the former
is uncountable, we follow by previous exercise that $\omega^{\omega}$ is uncountable.

\subsection*{5.3.6}

\textit{Suppose that $A$ is uncountable. Prove that $\pow(A)$ is uncountable.}

Let us define a function $f: A \to \pow(A)$
$$f(x) = \{x\}$$
this function is clearly injective, thus $\pow(A)$ is uncountable, as desired.

\subsection*{5.3.7}

\textit{Let $A \neq \emptyset$. Prove that if $B$ is uncountable, then $B^A$ is uncountable.}

Let $a \in A$ and define $f: B \to B^A$ to be so that $f(b)$ is a function such that
$$(\forall a'\in A)((f(a))(a') = b$$
(i.e. it sends every $a$ to value $b$). We follow that $f$ is injective, therefore
we can follow that $B^A$ is uncountable, as desired.

\subsection*{5.3.8}

\textit{Let $F$ be as in Theorem 5.3.2, and let $(*)$ $f_1, f_2, ..., f_n$ be a finite
  list of functions in $F$. Using the argument in the proof of Theorem 5.3.2
  define a new function $g \in F$ that is not in the list $(*)$. Conclude that $F$ is infinite}

$$
g(i) =
\begin{cases}
  i \leq n \ra \text{ if $f_i(i) = 0$ then 1, otherwise 0 } \\
  i > n \ra f_0(i)
\end{cases}
$$
will do


\textit{The rest of the exercises were handled in my real analysis course, specifically in
  Chapter 1}

\section{Cardinality}

\subsection*{5.4.1}

\textit{Let $A = \{x \in R: 0 < x < 1\}$ and let $B = \{x \in R: 2 < x < 5\}$. Prove that
  $|A| =_c |B|$}

Let $a < b \in R$. Define $h = (b - a)/2$. We follow that we can define function $f: (a, b) \to R$
% $$f_0(x) = x - h$$
% $$f_1(x) = \frac{h}{h - |f_0(x)|}$$
$$f(x) = \frac{x - h}{|x - h|} \frac{h}{h - |x - h|}$$
where $|\cdot|$ denotes absolute value. We can prove that given function
is a bijection (concrete proof given in real analysis book, but the
idea is to put center the interval at the origin and then strech it to the set of reals).
Thus we follow that for any nonempty open interval we've got a bijection from it to $R$. Therefore
$$a \neq b \in R \ra |(a, b)| =_c |R|$$
thus we follow from it that two sets, that are given in the
text of the exercise (which are open intervals) have the same
cardinality (there are easier ways to show that there exists a bijection from one
open interval to another, but I wanted to show off a bit).

\subsection{5.4.2}

\textit{Prove that $|A^{\{0, 1\}}| =_c |A \times A|$.}

Define $f: A^{\{0, 1\}} \to A \times A$ by
$$f(g) = \eangle{g(0), g(1)}$$
obviously $f$ is bijective, therefore we've got the necessary result.

\subsection*{5..4.3}

\textit{Let $A$ and $B$ be countably infinite sets. Prove that $|A| =_c |B|$. Conclude
  that $|\omega \times \omega| =_c |omega|$}

We know that there exists a bijection $f: \omega \times \omega \to \omega$ from some previous
exercises. Thus we follow the desired result.

\subsection*{5.4.4}

\textit{Let $A$ and $B$ be sets. Prove that $|A| =_c |A|$ and prove that if $|A| =_c |B|$,
  then $|B| =_c |A|$}

We follow that the identity function is bijective, thus $|A| =_c |A|$. Inverse of a bijection
is a bijection, therefore if there exists  a bijection $f: A \to B$, then there
exists a bijection $f\inv: B \to A$, thus we follow the symmetric property.

\subsection*{5.4.5}

Composition of bijections,

\subsection*{5.4.6}

\textit{Prove that if $|A| <_c |B|$ and $|B|  = |C|$<, then $|A| <_c |C|$.}

We follow that there exists an injection $f: A \to B$ and bijection $g: B \to C$, thus
there exists an injection $g \circ f: A \to C$, thus $|A| \leq_C |C|$.

Suppose that there exists a bijection $h: A \to C$. Then we follow that $g\inv \circ h : A \to B$ is
a bijection, which contradicts $|A| <_c |B|$

\subsection*{5.4.7}

\textit{Let $B$ be countable. Prove that if $|A| \leq_c |B|$, then $A$ is countable.}

We follow that there exist injections $f: A \to B$ and $g: B \to \omega$, thus there
exists an injection $g \circ f: A \to \omega$, as desired.

\subsection*{5.4.8}

\textit{Let $A$ be uncountable. Prove that if $|A| \leq |B|$, then $B$ is uncountable.}

Suppose that it isn't. Then by previous exercise we follow that $A$ is countable, which
is a contadiction.

\subsection*{5.4.9}

\textit{Prove that if $|A| <_c |B|$ and $|B| <_c |C|$, then $|A| <_c |C|$.}

We can follow that $|A| \leq_c |C|$ by composition of injections. Suppose that
$|A| =_c |C|$. Then we follow that there exists a bijection $f: C \to A$, thus
we follow that $|C| <_c |B|$, which is a contradiction.

\subsection*{5.4.10}

\textit{Prove that the function $G$ in the proof of Theorem 5.4.3 is a bijection.}

Firstly I want to state that the check for the fact that $G$ is indeed a function is pretty
trivial.

Let $G$ be defined as in the theorem. Suppose that $f \neq g \in \{0, 1\}^A$. We follow that
there exists $a \in A$ such that $f(a) \neq g(a)$. Since $a = 0 \lor a = 1$ we
follow that $f(a) \neq g(a) \ra f(a) = 1 \lor g(a) = 1$. Assume that $f(a) = 1$ and thus
$g(a) = 0$. Now we can follow that $a \in G(f)$ and $a \notin G(g)$. Thus we conclude that
$G(f) \neq G(g)$. Thus $f \neq g \ra G(f) \neq G(g)$, thus implying that $G$ is injejctive.

The fact that $G$ is surjective follows from the definition of $X_f$ and $G$.

\subsection*{5.4.11}

\textit{Prove that the set $\pow(\omega)$ is uncountable.}

Follows from Cantor's theorem.

\subsection*{5.4.12}

\textit{Using Theorem 5.4.18 prove that $|A| =_c |R|$, where $A = [-\pi, \pi)$.}

We follow that $|A| \leq_c |R|$ because of the identity. By the proof in the first exercise
of this chapter we follow that there exists an injection from  $R$ to $(-\pi, \pi)$. Thus
by Shroder-Bernstein Theorem (5.4.18) we follow that $|A| =_c |R|$, as desired.

\subsection*{5.4.13}

$a, y$

\subsection*{5.4.14}

\textit{Let $f: A \to B$ be onto $B$, and let $C = \{x \in A: x \notin f(x)\}$. Show that
  $C \notin B$.}

Suppose that $C \in B$. Since $f$ is surjective we follow that there exists $a \in A$
such that $f(a) = C$. If $a \in C$, then $a \notin f(a) \ra a \notin C$, which is not possible.
Suppose that $a \notin C$. Then we follow that $a \notin f(a)$, therefore $a \in C$ by definition,
which is also a contradiction. THus we follow that it is impossible to have $C \in B$. Thus
$C \notin B$, as desired.

\subsection*{5.4.15}

\textit{Suppose $A \subseteq B$ and $|B| \leq_c |A|$. Prove that $|A| =_c |B|$.}

We follow that since $A \subseteq B$, then there exists an injection (namely identity)
from $A$ to $B$, thus we follow that $|A| \leq_c |B|$. THus we follow from
Shroder-Bernstein Theorem that $|A| =_c |B|$, as desired.

\subsection*{5.4.16}

\textit{Suppose $A \subseteq B \subseteq C$ and $|A| =_c |C|$. Prove that $|B| =_c |C|$.}

We follow that there exists a injections $f: B \to C$ and $g: A \to B$, which are
just the identities with modified codomain. We also follow that
there exists a bijection $h: A \to C$. From this we can follow that 
$g \circ h\inv: C \to B$ is an injection. Thus we follow that there exists an injection
from $B$ to $C$ and vice-versa, thus implying that $|B| =_c |C|$, as desired.

\subsection*{5.4.17}

\textit{Prove that $|A \times B| =_c |B \times A|$}

We can define a bijection $f(\eangle{a, b}) = \eangle{b, a}$, which will give us desired result.

\subsection*{5.4.18}

\textit{Suppose that $|A| \leq_c |B|$, $|B| \leq_c |C|$ and $|C| \leq_c |A|$. Prove
  that $|A| =_c |B| =_c |C|$}

We can use the same argument as in 5.4.19.

\subsection*{5.4.19}

\textit{Let $f: A \to K$ and $g: B \to L$ be bijections. Let $h: A \times B \to K \times L$
  be defined by $h(\eangle{a, b}) = \eangle{f(a), g(b)}$ for all $\eangle{a, b} \in A \times B$.
  Prove that $h$ is a bijection.}

Suppose that $\eangle{k, l} \in K \times L$. We follow that $k \in K$ and $l \in L$, thus
there exist $a \in A, b \in B$ such that $f(a) = k \land g(b) = l$ by the fact
that $f, g$ are both bijective and therefore surjective. Thus there exists
$\eangle{a, b} \in A \times B$ such that $h(\eangle{a, b}) \in \eangle{k, l}$. Thus $h$
is surjective.

Suppose that $\eangle{a, b} \neq \eangle{a', b'}$. We follow that $a \neq a'$ or $b \neq b'$.
Thus by injectiveness of $f, g$ we follow that $f(a) \neq f(a')$ or $g(b) \neq g(b')$, which imply
that $\eangle{f(a), g(b)} \neq \eangle{f(a'), g(b')}$. Thus we follow that $h$ is injective,
as desired.

\subsection*{5.4.20}

\textit{ Prove that $G(g\inv \circ h \circ f) = h$ where the functions $G, g\inv, h, f$ are
  as in the proof of Theorem 5.4.6(3) }

We follow that
$$G(g\inv \circ h \circ f) = g \circ g\inv \circ h \circ f \circ f\inv $$
We can follow that both $g \circ g\inv$ and $f \circ f\inv$ are identity functions on
$L$ and $K$ respectively. Thus we follow that since $h: K \to L$
$$G(g\inv \circ h \circ f) = g \circ g\inv \circ h \circ f \circ f\inv = h$$
as desired.

\subsection*{5.4.21}

\textit{Suppose $A$ is nonempty. Let $c \in B$ and $d \in B$ be distinct. Using a diagonal argument,
  show that there is no function $F: A \to B^A$ that is onto $B^A$. Now find a one-to-one
  function $G: A \to B^A$. Conclude that $|A| <_c |B^A|$.}

Define a function $f: \pow(A) \to B^A$ by the rule
$$(f(X))(a) =
\begin{cases}
  a \in X \ra (f(X))(a) = c \\
  a \notin X \ra (f(X))(a) = d
\end{cases}
$$
Suppose that $X_1 \neq X_2 \in \pow(A)$. Then we follow that there exists $a \in X_1$ such that
$a \notin X_2$ (or vice versa, both cases are equivalent).
Thus $(f(X_1))(a) \neq (f(X_2))(a)$. Now we can follow that $X_1 \neq X_2 \ra f(X_1) \neq f(X_2)$.
Thus $f$ is injective. Therefore we can follow that
$$|\pow(A)| \leq_c |B^A|$$
Since $|A| <_c |\pow(A)|$, we conclude that there could be no surjections between $A$ and $B^A$,
as desired. From this argument alone we can follow that there exists an injection
$h: A \to B^A$, but I haven't used the diagonal argument, because I can't figure out where
I'm supposed to put it here, so I'll prodive a direct injection anyways.

Since $A$ is nonempty, we follow that there exists $a \in A$. Thus we follow that
for each $a \in A$ we can define a function $f: A \to B$ by
$$f(x) =
\begin{cases}
  x = a \ra c\\
  x \neq a \ra d
\end{cases}
$$
we can follow that the map which sends $A$ to produced function is injective, thus we
can conclude that $|A| \leq_C |B^A|$. Keeping in mind out previous result we can follow that
$$|A| <_c |B^A|$$, as desired.

\subsection*{5.4.22}

\textit{Prove that $|\omega^\omega| =_c |\pow(\omega)|$. Conclude that $|\omega^\omega| =_c |R|$.}

We know that there exists bijective function $p: \omega \times \omega \to \omega$. Thus we
follow that there exists $p \inv: \omega \to \omega \times \omega$. For $f \in
\omega^\omega$ define $A: \omega^\omega \to \pow(\omega)$
$$A_f = \{p(i, j): \eangle{i, j} \in f\}$$
Then we follow that
$$f_1 \neq f_2 \ra \eangle{i, j} \in f_1 \land \eangle{i, j} \notin f_2 \ra
p(i, j) \in A_{f_1} \land p(i, j) \notin  A_{f_2} \ra A_{f_1} \neq A_{f_2}$$
Thus given function is injective. Thus $ |\omega^\omega|  \leq_c |\pow(\omega)|$.

We know that $|\pow(\omega)| =_c |\{0, 1\}^{\omega}|$, and since
$\{0, 1\}^{\omega} \subseteq \omega^\omega$, we follow that $|\pow(\omega)| \leq_c |\omega^\omega|$.

Thus we follow by Schroder-Bernstein that $|\pow(\omega)| =_c |\omega^\omega|$, as desired.

\subsection*{5.4.23}

\textit{Prove Theorem 5.4.7(1)}

Suppose that $A \cap B = \emptyset$. Then we follow that for $\eangle{f, g} \in K^A \times K^B$
we can define a function $h_{f, g}: A \cup B \to K$ by
$$
h(x) =
\begin{cases}
  x \in A \ra f(x) \\
  x \in B  \ra g(x) \\
\end{cases}
$$
Since $A \cap B = \emptyset$ we follow that the function is well defined.
Thus we can define $H: K^A \times K^B \to K^{A \cup B}$ by this rule.
We can follow that the function is injective and surjective pretty easily, thus we follow that
$$|K^A \times K^B| =_c |K^{A \cup B}|$$
as desired.

\subsection*{5.4.24}

\textit{Prove Theorem 5.4.7(2)}

Let $f \in K^A \times L^A$. We follow that $f = \eangle{g, h}$. Thus we can define
$k: A \times (K \times L)$ by
$$k(x) = \eangle{g(x), h(x)}$$
We can follow that this function is bijective pretty easily.

\subsection*{5.4.25}

skip.

\subsection*{5.4.26}

\textit{Prove that $|(\omega^\omega)^\omega| =_c |\omega^\omega|$}

We follow that $|\omega \times \omega| =_c |\omega|$. Thus
$|\omega^{\omega \times \omega}| =_c |\omega^\omega|$. Since
$|(\omega^\omega)^\omega|  = |\omega^{\omega \times \omega}|$, we follow that
$|(\omega^\omega)^\omega| = |\omega^\omega|$, as desired

\subsection*{5.4.27}

\textit{Prove that
$$|(\omega^\omega)^{(\omega^\omega)}| =_c|\omega^{(\omega^\omega)}|$$
and
$$|(\{0, 1\}^\omega)^{\omega^\omega}| =_c|\{0, 1\}^{(\omega^\omega)}|$$
}

We can follow that $$|\omega^{\omega} \times \omega| =_c |\omega^\omega|$$
thus we follow that since $|A| =_c |A|$ for any set $A$
$$|A^{(\omega^{\omega} \times \omega)}| =_c |A^{(\omega^\omega)}|$$
and since
$$|A^{(\omega^{\omega} \times \omega)}| =_c  |(A^\omega)^{(\omega^{\omega})}|$$
we follow that
$$|A^{(\omega^\omega)}| =_c |(A^\omega)^{(\omega^{\omega})}|$$

Substituting $\omega$ or $\{0, 1\}$ for $A$ we get the desired result.

\subsection*{5.4.31}

\textit{Let $A$ and $B$ be nonempty sets. Using the AC, prove that there
  exists a one-to-one $f: A \to B$ iff there exists $g: B \to A$ that is
  onto $A$}

Suppose that $f: A \to B$ is injective. Because $A$ is nonempty, we follow that
there exists $a \in A$. We follow that
a function $f': A \to \ran(A)$ defined by $f'(x) = f(x)$ is bijective, therefore
we can define $g: B \to A$ by
$$g(x) =
\begin{cases}
  x \in \ran(A) \ra f\inv(x) \\
  x \notin \ran(A) \ra a
\end{cases}
$$
Suppose that $x \in A$. We follow that $$g(f\inv(x)) = x$$. Thus we follow that
$g$ is surjective.

Conversely,
suppose that there exists surjective  $g: B \to A$. 
Let $\eangle{G_i: i \in A}$ be indexed function, where
$$G(i) = g\inv[\{i\}] $$
Because $g$ is surjective, we follow that for every $a \in A$ there exists
$y \in B$ such that $g(y) = a$. 
Thus we follow that $g\inv[\{a\}] \neq 0$,
and therefore $\eangle{B_i: i \in A}$ has nonempty terms.
We can also follow that since $g$ is a function,
$$(\forall i_1, i_2 \in A)(i_1 \neq i_2 \ra g\inv[\{i_1\}] \cap g\inv[\{i_2\}] = \emptyset$$
We follow from AC
that there exists a function $\eangle{h_i: i \in A}$ such that $h_i \in G(i)$.
Since $G(i) \subseteq A$ we can state that $\ran(h) \subseteq A$, thus we're
justified to define a function $f: A \to B$ by $f(x) = h(x)$.
Let  $x_1, x_2 \in A$ be such that $x_1 \neq x_2$. Suppose that $f(x_1) = f(x_2)$.
Then we follow that $g\inv[\{x_1\}] \cap g\inv[\{x_2\}] \neq \emptyset$, which is a
contradiction. Therefore we can follow that $f$ is injective, as desired.

\subsection*{5.4.39}

\textit{Let $A$ be a set and let $H: \pow(A) \to \pow(A)$. Suppose for all $C$ and $D$
  in $\pow(A)$ we have that $H(C) \subseteq H(D)$ whenever $C \subseteq D$.
  Let
  $$S = \{D: D \subseteq A \land H(D) \subseteq D\}$$
  and let $X = \bigcap{S}$. Prove that $H(X) = X$. Moreover, suppose $Y \subseteq X$
  satisfies $H(Y) \subseteq Y$. Prove that $X = Y$.}

We follow that
$$(\forall D \in S)(\bigcap{S} \subseteq D)$$
Thus we follow that
$$(\forall D \in S)(X \subseteq D)$$
therefore by definition of $H$ we follow that 
$$(\forall D \in S)(H(X) \subseteq H(D))$$
And since $D \in S$, we follow that $H(D) \subseteq D$, tehrefore
$$(\forall D \in S)(H(X) \subseteq D)$$
thus
$$H(X) \subseteq \bigcap{S}$$
thus $H(X) \subseteq X$. Thus we follow that $X \in S$. By the definition of $H$
we follow that $H(X) \subseteq X$ implies that $H(H(X)) \subseteq H(X)$. Thus
we follow that $H(X) \in S$ by definition of $S$. Thus we follow that
$\bigcap{S} \subseteq H(X)$. Therefore $X \subseteq H(X)$. Therefore by double inclusion
we follow that $H(X) = X$.

Suppose that $Y \subseteq X$ is such that $H(Y) \subseteq Y$. We follow that since
$X \subseteq A$ and by given conditions $Y \in S$. Thus $\bigcap{S} \subseteq Y$.
Therefore $X \subseteq Y$. Therefore by double inclusion $X = Y$, as desired.

\chapter{Transfinite Recursion}

\section{Well-Ordering}

\subsection*{6.1.1}

\textit{Let $R$ be the set of real number and let $\leq$ be the standart order on $R$.
  Suppose $f: \omega \to R$ is such that for all $n, m \in \omega$, if $n \in m$,
  then $f(n) < f(m)$. Let $A = \{f(i): i \in \omega\}$.}

\textit{(a) Let $i$ and $j$ be in $\omega$. Prove that if $f(i) < f(j)$, then
  $i \in j$.}

We follow that if $j \in i$, then $f(j) < f(i)$, which is not the case.
We also follow that if $i = j$, then $f(j) = f(i)$, therefore we follow by trichotomy that
$i \in j$, as desired.


\textit{(b) Prove that $\leq$ is a well-ordering on $A$.}

Suppose that $Q \subseteq A$ and $Q \neq \emptyset$. Let $Q_i = f\inv[Q]$. We
follow that $Q_i \subseteq \omega$ and $Q_i \neq \emptyset$, therefore by
well-ordering of $\omega$ we follow that there exists least element $q \in Q$.
Let $a_m = f(q)$ and let there be arbitrary $a' \in A$.
For $a'$ there exists $j \in Q_i$  such that $f(j) = a'$.
Because $q$ is the least element of $Q_i$ we follow that $q \leq a'$, therefore
by definition of $f$, $f(q) \leq f(a')$. Thus we follow that
$f(q)$ is the least element of $Q$. Therefore we follow that
every nonempty subset of $A$ has a least element, therefore $\leq$ is
a well-ordering on $A$, as desired.

\subsection*{6.1.2}

\textit{Let $A = \omega \cup \{\omega\}$, and for all $x,y \in A$ define
  the relation $\leq$ on $A$ by $x \leq y$ if and only if $x \in y$ or $x = y$.
  Show that $\leq$ is a well-ordering on $A$.}

Suppose that $x, y, z \in A$. 

We can follow that $x = x$, therefore $x \leq x$.

Suppose that $x \leq y$ and $y \leq z$.
If none of $x, y, z$ are equal to $\omega$, then we follow transitive properties
from ordering on $\omega$, therefore $x \leq z$.
If $z = \omega \land x, y \neq \omega$, then we follow that $x \in z \ra x \leq z$.
If $z = y = \omega$ and $x \neq \omega$, then we follow that $z \in z \ra x \leq z$.
If $x = y = z = \omega$, then we follow that $x \leq \omega$.
All of the other cases are impossible, therefore we follow transitive property.


Suppose that $x \leq y$ and $y \leq x$. Suppose that $y \in x$. Then we follow that
if $x = y$, then $y \in y$, which is impossible. If $x \in y$, then we also have impossible
case. Thus we conclude that $x = y$, which gives us antisymmetry.

Suppose that $x, y \in A$. If $x, y \neq \omega$, then we follow that $x \leq y$
or $y \leq x$ by standart order on $\omega$. If $y = \omega$ and $x \neq \omega$, then
we follow that $x \in y$, therefore $x \leq y$. If $y = x = \omega$, then we follow that
$y \leq x$. This gives us total ordering.

Suppose that $B \subseteq A$ and $B \neq \emptyset$. If $B = \{omega\}$, then
we follow that $\omega$ is the least (and only) element of $B$. If
$B \neq \{omega\}$ then we follow that $B' = B \setminus \{omega\}$ is nonempty. We follow that
$B' \subseteq \omega$, therefore there exists a least element $b$ of $B'$. Therefore
we follow that
$$(\forall x \in B')(b \leq x)$$
Since $B' \subseteq \omega$, we follow that $b \in \omega$. Thus
$$(\forall x \in B)(b \leq x)$$
therefore $b$ is the least element of $B$. Therefore we follow that
for all subsets of $A$ we've got that it has a least element. Therefore
$\leq$ is a well-ordering on $A$, as desired.

\subsection*{6.1.3}

\textit{Suppoes that $\preceq$ is a well-ordering on $A$ abd that $f: A \to A$ is
  such that if $x \prec y$ then $f(x) \prec f(y)$ whenever $x, y \in A$. }

\textit{(a) Show that $f$ is injective.}

Suppose that $x \neq y \in A$. We follow that $x \prec y$ or $y \prec x$ (by trichotomy).
THus we follow that $f(x) \prec f(y)$ or $f(y) \prec f(x)$. Therefore $f(x) \neq f(y)$.
Therefore $f$ is injective, as desired.

\textit{(b) Prove that $x \preceq f(x)$ for all $x \in A$.}

Suppose that
$$f(x) \prec x$$
then we follow that
$$f(f(x)) \prec f(x)$$
by definition of $f$. This gives us the idea to define a function $p: \omega \to A$ by
$$p(0) = x$$
$$p(n^+) = f(p(n))$$
Now define
$$I = \{n \in \omega: p(n) \prec p(n^+)$$
we follow that $0 \in I$ by our assumption that $f(x) \prec x$. Suppose that
$n \in I$. Then we follow that $p(n) \prec(n^+)$. Therefore by definition of
$f$
$$f(p(n)) \prec f(p(n^+))$$
$$p(n^+) \prec p((n^+)^+)$$
thus we follow that $n \in I \ra n^+ \in I$. Therefore $I = \omega$, and therfore
we've got a function $p: \omega \to A$ such that $p(n) \prec p(n^+)$, which
gives us a contradiction by 6.1.2.

\subsection*{6.1.4}

\textit{Let $A$ be an infinite set with the well-ordering $\preceq$. Now let
  $h: \omega \to A$ be the function defined in the proof of Theorem 6.1.3. Let $y \in A$.
  Prove that for all $n \in \omega$, if $y \prec h(n)$, then $y = h(i)$ for
  some $i \in n$.}

Let
$$I = \{n \in \omega: \underline{L}_{h(n)} \subseteq \ran(h)\}$$
we follow that for $h(0) = p$ and
$$\underline{L}_{h(0)} = \underline{L}_{p} = \{p\}$$
thus $0 \in I$. Suppose that $j \in I$. We follow that $\underline{L}_{h(j)} \subseteq \ran(h)$.
From the inductive proof in the theorem we follow that
$$\underline{L}_{h(j^+)}  = \underline{L}_{h(j)} \cup \{h(j^+)\}$$
Since $\{h(j^+)\}, \underline{L}_{h(j)} \subseteq \ran(h)$,
we follow that $\underline{L}_{h(j^+)} \subseteq \ran(h)$, therefore $I = \omega$.

From this we follow that if $y \prec h(n)$, then $y \in \underline{L}_{h(j^+)}$, therefore
$(\exists i \in \omega)(y = h(i))$, as desired.

\subsection*{6.1.5}

\textit{Suppose that $\preceq$ is a well-ordering on $W$ and let $C \subseteq W$.
  Show that $\preceq_C$ is a well-ordering on $C$.}

We can follow that $\preceq_C$ is a total order from one of the exercises in the
section 3.4. Suppose that $X \subseteq C$. Then we follow that $X \subseteq W$.
Therefore it has a $\preceq$-least element $x$. Since $x \in X$, we follow that
$x \in C$. Suppose that $y \in X$. Then we follow that $x \preceq y$. Because $y \in X$
we follow that $y \in C$. Therefore we follow that $x \preceq_C y$. Therefore
$x$ is a $\preceq_C$-least element of $X$. Thus we conclude that every
nonempty subset of $C$ has a $\preceq_C$-least element. Therefore $\preceq_C$ is a
well-order on $C$, as desired.

\subsection*{6.1.6}

\textit{Prove Theorem 6.1.4}

Let $\preceq'$ be a well-ordering on the set $B$. Suppose that $h: A \to B$ is
injective. Define the relation $\preceq$ on $A$ by $x \preceq y$ iff $h(x) \preceq' h(y)$
for all $x, y \in A$. Then we follow that $\preceq$ is a total order by
some exercise (or theorem) in chapter 3 (if not, then it's pretty evident).

Suppose that $X$ is nonempty and $X \subseteq A$. We follow that
$h[X] \subseteq B$ and it is nonempty as well. By well-ordering of $\preceq'$ we follow that
there exists $x \in h[X]$ that is $\preceq'$-least element of $h[X]$. Because
$h$ is injective, we follow that there exists only one $y \in X$ such that
$h(x) = y$. Let $z \in X$ be arbitrary. Then we follow that
$$h(x) \preceq' h(z)$$
therefore
$$x \preceq z$$
thus we follow that $x$ is the $\preceq$-least element of $X$. Thus we follow that
every nonempty subset of $A$ has a $\preceq$-least element, thus we can conclude that
$\preceq$ is a well-order on $A$, as desired.

\subsection*{6.1.7}

\textit{Prove that every finite set has a well-ordering}

We follow that by 6.1.4 and the fact that $\leq$ is a well-order
for any $n \in \omega$.

\subsection*{6.1.8}

\textit{Prove that every countably infinite set has a well-ordering}

We follow that by 6.1.4 and the fact that $\leq$ is a well-order
for $\omega$.

\section{Transfinite Recursion Theorem}

\subsection*{6.2.1}

\textit{Complete the proof of Theorem 6.2.1 by showing that the function $H$ satisfies
  $H(u) = F(H|s(u))$ for all $u \in W$. Also, show that $H$ is the only such function.}

Assume that 
$$H(u) \neq F(H|s(u))$$
for some $u \in W$.

Let $w$ is the $\preceq$-lowest element of $W$ such that
$$H(w) \neq F(H|s(w))$$
We know that $g_w = G(w)$ is such that
$$g_w(w) = F(g_w|s(w))$$
From claim 3 we follow that
$$(\forall x \in s(w))(g_w(x) = g_x(x))$$
By using definition of $H$  we get
$$(\forall x \in s(w))(g_w(x) = H(x))$$
therefore
$$g_w|s(w) = H|s(w)$$
thus
$$g_w(w) = F(g_w|s(w))$$
$$g_w(w) = F(H|s(w))$$
$$H(w) = F(H|s(w))$$
which is a contradiction.

Now suppose that there exist two functions $H_1 \neq H_2$  such that
$$H_1(w) = F(H_1|s(w))$$
$$H_2(w) = F(H_2|s(w))$$
we follow that there exists lowest $w \in W$ such that $H_1(w) \neq H_2(w)$. Thus
$$H_1|s(w) = H_2|s(w)$$
And since $F$ is a function we follow that 
$$F(H_1|s(w)) = F(H_2|s(w))$$
thus
$$H_1(w) = H_2(w)$$
which is a contradiction.

\subsection*{6.2.2}

\textit{Show that Theorem 6.2.5 implies Theorem 6.2.1}

Let $F: A^{\prec W} \to A$. If $A = \emptyset$, then we follow that $A^{\prec W} = \emptyset$,
therefore $H = \emptyset$, which presents a trivial case. Because of it, suppose that
$A$ is nonempty. Because $A$ is nonempty, we follow that there exists certain $a \in A$.
Define
$$\phi(x, y) = (x \in A^{\prec W} \land F(x) = y) \lor(x \notin A^{\prec W} \land y = a)$$
For every set $x$ we follow that $x \in A^{\prec W}$ or $x \notin A^{\prec W}$. Thus
we follow that two clauses of $\phi$ are mutually exclusive. 
If $x \in A^{\prec W}$, then we follow that there exists a unique $y \in A$ such that
$F(x) = y$ by the virtue of the fact that $F$ is a function.
If $x \notin A^{\prec W}$, we follow that $y = A$ is the only possibility.

Applying 6.2.5 to set $W$ with corresponding well-order $\prec$, we follow that there
exits a unique function $H$ with domain $W$ such that
$$\phi(H|s(u), H(u))$$
From the definition of $\phi$ we follow that
$$\forall x(\phi(x, y) \ra y \in A)$$
therefore for all $u \in W$ we've got that $H(u) \in A$. Therefore we're justified
to state that $H: W \to A$.

Therefore we conclude that there exists $H: W \to A$ such that
$$H(u) = F(H|s(u))$$
which gives us the result of theorem 6.2.1, as desired.

\subsection*{6.2.3}

\textit{Let $\preceq$ be a well-ordering on a set $W$ and $A$ be a set.
  Define $l: A^{prec W} \to W$ by $l(g)$ equals the $\preceq$-least $u \in W$
  such that $u \notin \dom(g)$. Let $G: A^{\prec W} \times W \to A$ be a function.
  Show that Theorem 6.2.1 implies there exists a function $H: W \to A$ such that
  $$H(u) = G(H|s(u), u)$$
  for all $u \in W$.}

Define a relations $F$ by
$$\eangle{u, y} \in F \lra G(\eangle{u, l(u)}) = y$$
We can follow that for arbitrary $u \in A^{\prec W}$ there exists a unique tuple $\eangle{u, l(u)}$,
and for this tuple theree exists a unique $y \in A$ such that $G(\eangle{u, l(u)}) = y$.
Thus we follow that $F$ is a function from $A^{\prec W}$ to $A$.
Thus we follow that there exists a unique $H: W \to A$ such that for all $u \in W$
$$H(u) = F(H|s(u))$$
by definition of $F$ we folllow that
$$H(u) = G(H|s(u), l(H|s(u)))$$
We follow that $H|s(u): s(u) \to A$. Therefore definition of $l$ gives us that
$$l(H|s(u)) = u$$
thus
$$H(u) = G(H|s(u), u)$$
as desired.

\subsection*{6.2.4}

\textit{Suppose that $\preceq$ is a well-ordering on a set $W$ and $F: A^{\prec W} \to A$ is
  a function. Let $H: W \to A$ be as in Theorem 6.2.1}

\textit{(a) Let $u, v \in W$. Prove that if $s(u) = s(v)$, then $u = v$.}

Suppose that $u \neq v$. Then we follow that $u \prec v$ or $v \prec u$. Assume the former
(both cases are simular). Then we follow that $u \in s(v)$. Thus
$u \in s(u)$, thus $u \prec u$, therefore $u \neq u$, which is a contradiction.

\textit{(b) Suppose that $F$ is one-to-one. Prove that $H$ is also one-to-one.}

Suppose that $u \neq v \in W$. We follow that $s(u) \neq s(v)$. Thus $H|s(v) \neq H|s(u)$
(we're talking about restrictions and not ranges, thus this implication is valid). Now we
can follow that injectivity of $F$ implies that  $F(H|s(v)) \neq F(H|s(u))$. Therefore
we conclude by applying definition of $H$ that $H(v) \neq H(u)$. Therefore we can follow that
$H$ is injective, as desired.

\subsection*{6.2.5}

\textit{Let $\preceq$ be a well-ordering on $W$ and $\preceq^*$ be a well-ordering on $A$.
  Assume for each $g \in A^{\prec W}$, there is a $y \in A$ such that $g(x) \prec^* y$ for
  all $x \in \dom(g)$. Show that there is an $H: W \to A$ so that for all $v \in W$ and
  $u \in W$, if $v \prec u$, then $H(v) \prec^* H(u)$.}

Define $F: A^{\prec W} \to A$ by
$$F(g) = \prec^*\text{- lowest element of
  $\{y \in A: (\forall x \in \dom(g))(g(x) \prec^* y)\}$}$$
We can follow pretty easily that this thing is a function. 6.2.1 gives us that there exists
$H: W \to A$ such that
$$H(u) = F(H|s(u))$$
Now let $v, u$ be such that $v \prec u$. We follow that $v \in s(u)$. Therefore
we can follow that $H(u) = F(H|s(u))$. By definition of $F$ we follow that
$(\forall x \in s(u))(H(x) \prec H(u))$. Since $v \in s(u)$ we follow that
$H(y) \prec H(u)$, as desired.

\subsection*{6.2.6}

\textit{Let $A$ be a set. Suppose that $A \subseteq C$ where $C$ is a transitive set.
  Prove that $\overline{A} \subseteq C$. Thus, $\overline{A}$ is the "smallest"
  transitive set for which $A$ is a subset.}

Suppose that $A \subseteq C$ and $C$ is a transitive set.
Let
$$I = \{n \in \omega: F(n) \subseteq C\}$$
we follow that $F(0) = A \subseteq C$ by premise of the exercise. Let $n \in I$.
We follow that $F(n) \subseteq C$. We follow that
$$(\forall x \in F(n))(x \in C)$$
because $C$ is transitive, we follow
$$(\forall x \in F(n))(x \subseteq C)$$
let $y \in F(n^+)$. We follow that $y \in \bigcup{F(n)}$. Thus
$$(\exists X \in F(n))(y \in X)$$
thus $y \in C$. Therefore $\bigcup{F(n)} \subseteq C$. Therefore $F(n^+) \subseteq C$.
Therefore $n \in I \ra n^+ \in I$. Therefore $I = \omega$. Thus we follow that
$$(\forall n \in \omega)(F(n) \subseteq C)$$

Suppose that $x \in \overline{A}$. We follow that there exists $n \in \omega$ such that
$x \in F(n)$. Because $F(n) \subseteq C$, we follow that $x \in C$. Thus we follow that
$\overline{A} \subseteq C$, as desired.

Given that $\overline{A}$ is transitive, we follow that $\overline{A}$ is the smallest
transitive set for which $A$ is a subset.

\subsection*{6.2.7}

\textit{Let $B$ be a set. Conclude from Exercise 6 that $\overline{\{B\}}$ is the
  smallest transitive set that contains $B$ as an element.}

Suppose that $C$ is a transitive set that contains $B$ as an element. We follow that
$C$ contains $\{B\}$ as a subset. Therefore we follow that $\overline{\{B\}} \subseteq C$
by previous exercise. Therefore for all
transitive set $C$
$$B \in C \ra \overline{\{B\}} \subseteq C$$

We follow that  $B \in \overline{\{B\}}$ and $\overline{\{B\}}$ imply that
$\overline{\{B\}}$ is the smallest transitive subset such that $B$ is an element this set.

\subsection*{6.2.8}

\textit{Let $A$ be a set. Using the replacement axiom, one can show that
  $\{\overline{x}: x \in A\}$ is a set. Establish the following:}

\textit{(a) Prove that if $x \in A$, then $\overline{x} \subseteq \overline{A}$.}

We follow that $x \in A$ and $A \subseteq \overline{A}$ implies that
$x \in \overline{A}$. Since $\overline{A}$ is transitive, we follow that
$x \subseteq \overline{A}$. Since $\overline{A}$ is transitive, exercise 6 implies that
$\overline{x} \subseteq \overline{A}$.

\textit{(b) Prove that $A \cup \bigcup{\{\overline{x}: x \in A\}} \subseteq \overline{A}$}

We can follow that $A \subseteq \overline{A}$. For every $x \in A$ we follow that
$\overline{x} \subseteq \overline{A}$. Suppose that $y \in \bigcup{\{\overline{x}: x \in A\}}$.
We follow that there exsits $x \in A$ such that $y \in \overline{x}$. Thus
$y \in \overline{A}$. Therefore $\bigcup{\{\overline{x}: x \in A\}} \subseteq \overline{A}$
Therefore we follow that $A \cup \bigcup{\{\overline{x}: x \in A\}} \subseteq \overline{A}$.

\textit{(c) Prove that $A \cup \bigcup{\{\overline{x}: x \in A\}}$ is a transitie set.}

Suppose that $y \in A \cup \bigcup{\{\overline{x}: x \in A\}}$.
If $y \in A$, then $\overline{y} \in \{\overline{x}: x \in A\}$. Therefore
$\overline{y} \subseteq \bigcup{\overline{y} \in \{\overline{x}: x \in A\}}$.
Since $\overline{y}$ is transitive and $y \in \overline{y}$, we follow that
$y \subseteq \bigcup{\{\overline{x}: x \in A\}}$.

If $y \in \bigcup{\{\overline{x}: x \in A\}}$. We follow that
there exists $\overline{x'} \in A$ such that $y \in \overline{x'}$. Thus
$y \subseteq \overline{x'}$. Therefore $\bigcup{\overline{y} \in \{\overline{x}: x \in A\}}$.
Thus we follow that $y \subseteq  A \cup \bigcup{\{\overline{x}: x \in A\}}$.

From those conclusion we follow that presented set is transitive.

\textit{(d) Prove that $A \cup \bigcup{\{\overline{x}: x \in A\}} = \overline{A}$}

We follow that $A \subseteq A \cup \bigcup{\{\overline{x}: x \in A\}}$ and
$A \cup \bigcup{\{\overline{x}: x \in A\}}$ is transitive, therefore we follow that
$\overline{A} \subseteq A \cup \bigcup{\{\overline{x}: x \in A\}}$. From part
(a) we get double inclusion, which gives us equality.

\subsection*{6.2.9}

\textit{Let $A$ be a set. Using the proof of Theorem 6.2.8 as a model, prove that
  there is a function $F$ with domain $\omega$ such that}

\textit{1. $F(0) = A$}

\textit{2. $F(n^+) = \{F(n)\}$ for all $n \in \omega$.}

\textit{Let $C = F[\omega]$. Prove thta $A \in C$ and for all $X$, if $X \in C$,
  then $\{X\} \in C$.}

We can define
$$y =
\begin{cases}
  A \text{ if } f = \emptyset \\
  \{F(n)\} \text{ if $f$ is a function with domain $k^+$ for some $k \in \omega$} \\
  \emptyset \text{ otherwise}
\end{cases}
$$
and define $\phi(f, y)$ based on this definition. Then we follow that there exsits
a function $F$ with domain $\omega$ such that the required conditions hold.

Let $C = F[\omega]$. We follow that $F(0) \in C$, therefore $A \in C$.

If $X \in C$, then we follow that there exists $n \in \omega$ such that
$F(n) = X$. Thus $F(n^+) = \{X\}$. Since $n^+ \in \omega$, we follow that
$F(n^+) \in F[\omega]$, therefore $\{X\} \in F[\omega]$, as desired.

\subsection*{6.2.10}

\textit{Let $A$ be a set. Using the proof of Theorem 6.2.8 as a model, prove that
  there is a function $F$ with domain $\omega$ such that}

\textit{1. $F(0) = A$}

\textit{2. $F(n^+) = \pow(F(n))$ for all $n \in \omega$.}

\textit{Let $C = F[\omega]$. Prove thta $A \in C$ and for all $X$, if $X \in C$,
  then $\pow(X) \in C$.}

We can define
$$y =
\begin{cases}
  A \text{ if } f = \emptyset \\
  \pow(F(n)) \text{ if $f$ is a function with domain $k^+$ for some $k \in \omega$} \\
  \emptyset \text{ otherwise}
\end{cases}
$$
and define $\phi(f, y)$ based on this definition. Then we follow that there exsits
a function $F$ with domain $\omega$ such that the required conditions hold.

Let $C = F[\omega]$. We follow that $F(0) \in C$, therefore $A \in C$.

If $X \in C$, then we follow that there exists $n \in \omega$ such that
$F(n) = X$. Thus $F(n^+) = \pow(X)$. Since $n^+ \in \omega$, we follow that
$F(n^+) \in F[\omega]$, therefore $\pow(X) \in F[\omega]$, as desired.

\chapter{The Axiom of Choice (Revisited)}

\section{Zorn's Lemma}

\subsection*{7.1.1}

\textit{Let $\eangle{A_i: i \in I}$ be an indexed function so that $A_i \neq \emptyset$ for
  all $i \in I$. Without appealing to the axiom of choice, prove that if $\bigcup_{i \in I}{A_i}$
  is countably infinite, then there is a function $f: I \to \bigcup_{i \in I}{A_i}$,
  such that $f(i) \in A_i$, for all $i \in I$.}

Because $\bigcup_{i \in I}{A_i}$ is countably infinite and all of $A_i$ are nonempty,
we can follow that $|I| \leq_c |\omega|$, $|A_i| \leq_c |\omega|$
and $|\bigcup_{i \in I}{A_i}| =_c |\omega|$ (we can follow that if one of those is not true,
then we've got a contradiction of the fact that $|\bigcup_{i \in I}{A_i}| =_c |\omega|$).

Thus we follow that there exists a bijection $g: \omega \to \bigcup_{i \in I}{A_i}$.
We can follow that for every $i$ there exists $g\inv[A_i] \subseteq \omega$.
Because each $A_i$ is nonempty and $g$ is a bijection, we follow that $g\inv[A_i]$ is
nonempty for all $i \in I$. Define $f': I \to \omega$ by
$$f'(i) = \text{$\leq$-least element of $g\inv[A_i]$}$$
since the least element of a subset is unique, we follow that the function is well-defined.
thus we follow that for $f = g \circ f': I \to \bigcup_{i \in I}{A_i}$ we've got
$$f(i) \in A_i$$
for all $i \in I$, as desired.

\subsection*{7.1.2}

\textit{Let $R \subseteq A \times B$ and let $R(x, y)$ denote $\eangle{x, y} \in R$. Now
  suppose that for all $x \in A$, there is a $y \in B$ such that $R(x, y)$. Using the
  axiom of choice, show that there is a function $f: A \to B$ such that $R(x, f(x))$
  for all $x \in A$}

Because for every $x \in A$ there exists $y \in B$ such that $R(x, y)$, we follow that
for $x \in A$ we can define a function $Q: A \to \pow(B)$ by
$$Q(x) = \set{y \in B: R(x, y)}$$
Becase for every $x$ there exists $y$ such that $R(x, y)$ we follow that
$$(\forall x \in A)(Q(x) \neq \emptyset)$$
thus we can follow that $Q_x$ is an indexed function with nonempty terms. Thus AC implies that
there exists $q: A \to \bigcup{\pow(B)}$ such that $q(x) \in Q(x)$. Since
$\bigcup{\pow(B)} = B$, we follow that $q: A \to B$ and bu the fact that $q(x) \in Q(x)$, we
follow that $R(x, q(x))$, as desired.

\subsection*{7.1.3}

\textit{RW: Let $R \subseteq A \times A$ and let $R(x, y)$ denote $\eangle{x, y} \in R$.
  Suppose that  $A \neq \emptyset$ and for all $x \in A$, there is a $y \in A$ such that $R(x, y)$.
  Show that there is a function $h: \omega \to A$ such that $R(h(n), h(n^+))$ for all $n \in
  \omega$ }

By previous exercise we follow that there exists $f: A \to A$ such that
$R(x, f(x))$ for all $x \in A$. Let $a \in A$ be arbitrary (it exists because $A \neq \emptyset$).
We follow by 4.2.1 that we can define
$h: \omega \to A$ by
$$h(0) = a$$
$$h(n^+) = f(h(n))$$
Suppose that $k \in \omega$. Let $b = h(k)$. By proven properties of $f$ and the fact that
$b \in A$ we  follow that
$$R(b, f(b))$$
by substituting $b$ for its definition we follow that
$$R(h(k), f(h(k)))$$
and by definition of $k$ we follow that
$$R(h(k), h(k^+))$$
Since $k \in \omega$  is arbitrary, we follow the desired conclustion.

\subsection*{7.1.4}

\textit{Prove 7.1.4}

Let $\preceq$ be a partial order on $A$, and let $C$ be a chain in $A$. Suppose that
$C' \subseteq C$ is a cofinal chain.

Suppose that $x \in C_p$. We follow that
$$(\forall c \in C)(c \prec x)$$
Since $C' \subseteq C$, we follow that
$$(\forall c \in C')(c \prec x)$$
thus $x \in C'_p$ by definition. Therefore we conclude that  $C'_p \subseteq C_p$.

Suppose that $x \in C'_p$. We follow that
$$(\forall c \in C')(c \prec x)$$
Let $y \in C$. By the fact that $C'$ is cofinal we follow that there exists $q \in C'$ such that
$y \prec q$. Because $q \in C'$, we follow that $q \prec x$. Thus we follow by transitivity
that $y \prec x$. Because $y \in C$ is arbitrary, we conclude that
$$(\forall c \in C)(c \prec x)$$
therefore $x \in C_p$. Therefore we follow that $C'_p \subseteq C_p$. Thus by double
inclusion we have $C_p = C'_p$, as desired.

\subsection*{7.1.5}

\textit{Let $(F, \subseteq)$ be a partially ordered set where $F$ is a set (of sets).
  Suppose that for every chain $C \subseteq F$ we have that $\bigcup{C} \in F$. 
}

\textit{(a) Let $C \subseteq F$ be a chain.
  Prove that $\bigcup{C}$ is an upper bound for $C$.}

Suppose $X \in C$. Suppose that $x \in X$. We follow that $x \in \bigcup{C}$ and
thus $X \subseteq \bigcup{C}$. Therefore
$$(\forall X \in C)(X \subseteq \bigcup{C})$$
and by the properties of $F$ we follow that $\bigcup{C}$. Thus we conclude that $\bigcup{C}$
is an upper bound for $C$ by definition.

\textit{(b) Using Zorn's Lemma, show that there exists an $M \in F$ that is
  maximal, that is, $M$ is not the proper subset of any $A \in F$.}

By previous point we can follow that every chain $C$ of $F$ has an upper bound. Thus
Zorn's Lemma implies that $F$ has a maximal element $M$. Definition of a maximal
element implies the desired result.

\subsection*{7.1.6}

\textit{Let $f: A \to B$ and $F = \set{X \subseteq A: (f|X): X \to B \text{ is one-to-one}}$.
  Consider the  partially ordered set $(F, \subseteq)$.}

The fact that $F$ is nonempty is followed from the fact that $(f|\emptyset)$ is injective.
Also, if $f$ is restricted to a singleton, then it is also injective.

\textit{(a) Using Zorn's Lemma, show that there is an $M \in F$ that is maximal}

Suppose that $C$ is a chain in $F$. We want to show that $C$ has an upper bound.
If $f: A \to B$ is injective, then we follow that $A$ is an upper bound for any subset of $F$.

Suppose that $C$ is a chain in $F$. Suppose that $x \neq y \in \bigcup{C}$. We follow that
there exist two sets $X, Y \in C$ such that $x \in X$ and $y \in Y$. Because $C$ is a
chain we follow that $X \subseteq Y$ or $Y \subseteq X$. Assume the former. Then we follow that
$x, y \in Y$, and by the fact that $Y \in F$, we follow that $f(x) \neq f(y)$. Therefore
$f|\bigcup{C}$ is injective. Therefore $\bigcup{C} \in F$.
Thus we follow that if $C$ is a chain in $F$, then $\bigcup{C} \in F$. Using previous exercise
(which uses Zorn's Lemma) we can conclude that $F$ has a maximal element.

\textit{(probably a typo in the text of the exrecise)
  (b) Prove that if $f$ is onto $B$, then $(f|M): M \to B$ is a bijection.}

Suppose that $(f|M): X \to B$ is not a bijection. Since $M \in F$, we follow that
$(f|M): M \to B$ is injective. Thus we follow that $(f|M): M \to B$ is not surjective.
Therefore there exists $b \in B$ such that $b \notin f[M]$. Because $f: A \to B$ is
surjjective, we can conclude that there exists $x \in A$ such that $f(x) = b$. Therefore
$(f|M \cup \{x\}): M \cup \{x\} \to B$ is injective and thus $M \cup \{x\} \in F$.
Because $M \subseteq M \cup \{x\}$, we follow that $M$ is not a maximal element, which is
a contradiction.

\subsection*{7.1.7}

\textit{Let $(A, \preceq)$ be a poset. Let $F = \set{D \in \pow(A): D \text{ is a chain in A}}$.
  Consider the partially ordered set $(F, \subseteq)$.}

\textit{(a) Let $C$ be a chain in $F$. Show that $\bigcup{C}$ is a chain in $A$.}

Suppose that $x, y \in \bigcup{C}$. We follow that there exist sets $X, Y \in C$ such that
$x \in X$ and $y \in Y$. Since $C$ is a chain we follow that $X \subseteq Y$ or $Y \subseteq X$.
Assume the former. Then we follow that $x, y \in Y$. Thus $x \preceq y$ or $y \preceq x$
by the fact that $Y$ is a chain. Therefore we conclude that $\bigcup{C}$ is a chain in $A$.

The remaining part (and this part as well) is practicly identical to 7.1.5, therefore
I'll skip the details.

\subsection*{7.1.8}

\textit{Let $P$ be the set of nonzero natural numbers. Then $(P, |)$ is a poset
  where $|$ is a divisibility relation on $P$. Define a partial order $\preceq$ on $P$
  such that $| \subseteq \preceq$ and $2 \preceq 3$.}

I think that standart $\leq$ will do. It is a partial order
(every total order is a partial order), if $x | y$, then $x \leq y$, therefore $| \subseteq \leq$,
and we've got that $2 \leq 3$. Don't know if I'm missing something obvious, but I think
that's it.

\subsection*{7.1.9}

\textit{Using Theorem 7.1.10, show there is a total order $\preceq$ on $\pow(\omega)$
  such that for all $x, y \in \pow(\omega)$, if $x \subseteq y$, then $x \preceq y$.
  Show that $\preceq \neq \subseteq$ on $\pow(\omega)$.}

By straightforward application of 7.1.10 on a partial order $\subseteq$ we get
$\preceq \subseteq \pow(\omega) \times \pow(\omega)$ such that $(\subseteq) \subseteq (\preceq)$
(paretheses are used for clarity). This gives us the desired result.

We can follow that $\set{2} \not \subseteq \set{3}$ and  $\set{3} \not \subseteq \set{2}$,
but given that $\preceq$ is a total order we follow that
$\set{2} \preceq \set{3}$ or  $\set{3} \preceq \set{2}$. Assume the former. Then we follow that
$\eangle{\set{2}, \set{3}} \in \preceq$ and $\eangle{\set{2}, \set{3}} \notin (\subseteq)$,
which tells us that $\subseteq \neq \preceq$, as desired.

\subsection*{7.1.10}

\textit{Let $\eangle{A_i: i \in I}$ be an indexed function with nonempty terms, and let
  $$F = \set{g: g \text{ is a function} \land \dom(g) \subseteq I
    \land (\forall i \in \dom(g))(g(i) \in A_i)}$$
  Thus, $(F, \subseteq)$ is a partially ordered set. Assuming Zorn's Lemma,
  prove the AC as follows: }

\textit{Let $C \subseteq F$ be a chain. Prove that $\bigcup{C}$ is in $F$ and is an upper
  bound for $C$.}

We can follow that since $C \subseteq F$, we can state that $\bigcup{C}
\subseteq I \times \ran(A)$. 
Suppose that $x, y \in \bigcup{C}$ are such that $x = \eangle{u, v}$, $y = \eangle{o, w}$.
We follow that there exist $X, Y \in C$ such that
$x \in X$ and $y \in Y$. By the fact that $C$ is a chain we follow that $X \subseteq Y$
or $Y \subseteq X$. Assume the former. Then we follow that $x, y \in Y$. Since
$Y \in C \subseteq F$,
we follow that it is a function, and thus if $u = o$, then $v = w$. Thus we follow
that $\bigcap{C}$ is a function. We can also follow that because $Y \in F$ we've got that
$\dom(Y) \subseteq I$, and thus $u, o \in I$, therefore $\dom{\bigcup{C}} \subseteq I$.
By the same token we've got the last condition of inclusion in $F$, therefore
we can conclude that $\bigcup{C} \in F$.

Now exercise 7.1.5 can be applied to state that provided result implies that $\bigcup{C}$ is
an upper bound of $C$ whenever $C$ is a chain.

\textit{(b) Conclude, via Zorn's Lemma, that $F$ has a maximal element $h$.}

Conditions that are proven in previous point are sufficient to apply Zorn's Lemma and get the
desired result.

\textit{(c) Show that $\dom(h) = I$ and $(\forall i \in I)(h(i) \in A_i)$. Thus
  $\eangle{h_i: i \in I}$ is a choice function.}

Because $h \in F$, we follow that $\dom(h) \subseteq I$. Suppose that $I \not \subseteq \dom(h)$.
Let  $i \in I \setminus \dom(h)$.
Now let $a \in A_i$. We can follow that $f: \set{\eangle{i, a}}$ is a function, that satisfies
all of the conditions of $F$, and therefore $f \in F$. We can follow  that because
$i \in I \setminus \dom(h)$, we can state that $h \cup f \in F$. Thus we follow that
$h \subseteq (h \cup f)$ and thus $h$ is not a maximal element, which is a contradiction.

Therefore we conclude that $\dom(h) = I$. The latter requirement is satisfied by the fact that
$\dom(h) = I$ and the last requirement for inclusion for $F$.

\section{Filters and Ultrafilters}

\subsection*{7.2.1}

\textit{Let $S \subseteq X$ where $S$ is nonempty. Let $F = \set{A \subseteq X: S \subseteq A}$.
  Prove that $F$ is a filter on $A$.}

Because $F$ is a set of subsets of $X$ we follow that $F \subseteq \pow(X)$. 

Since $S \subseteq X$, we follow that $X \in F$. Because $S$ is nonempty, we follow that
$S \not \subseteq \emptyset$, and thus $\emptyset \notin F$.

Suppose that $q, w \in F$. We follow that $S \subseteq q$ and $S \subseteq w$. Thus
$S \subseteq q \cap w$. Therefore we follow that $q \cap w \in F$.

Suppose that $Y \in F$ and there exists $Z \subseteq X$ such that $Y \subseteq Z$
(i.e. $Y \subseteq Z \subseteq X$). We follow that since $S \subseteq Y$ by
transitivity of $\subseteq$ we get that $S \subseteq Z \subseteq X$. Thus we follow that
$Z \in F$.

Therefore we conclude that $F$ satisfies all of the requirements of a filter and thus is a
filter on $X$, as desired.

\subsection*{7.2.2}

\textit{Let $F$ be a filter on $X$. Prove that $F$ is closed under finite intersections.
  Conclude that $F$ has the finite intersection property.}

Let
$$I = \set{n \in \omega: S \subseteq F \land |S| = n^+ \ra \bigcap{S} \in F}$$
Suppose that $S \subseteq F$ and $|S| = 1 = (0)^+$. We follow that $S$ is a singleton,
whose element is an element of $F$, and thus $\bigcap{S} \in F$. Thus we conclude that
$0 \in I$.

Suppose that $n \in I$. Let $S \subseteq F$ be such that $|S| = (n^+)^+$.
We follow that there exists $a \in S$ and thus
$$S = (S \setminus \set{a}) \cup \set{a}$$
since $|S \setminus \set{a}| = n^+$ we follow that $\bigcap{S \setminus \set{a}} \in F$.
Thus we follow that
$$\bigcap{S} = \bigcap{S \setminus \set{a}} \cap \set{a}$$
where on the right hand side we've got an intersection of two elements of $F$. Thus we
follow that this intersection is in $F$ (by the second property of a filter). Thus we
conclude that $n^+ \in I$. Therefore we follow that $n \in I \ra n^+ \in I$, thus
$I = \omega$. Therefore we can conclude that if $S$ is a finite nonempty subset of $F$,
then its intersection is also in $F$, and thus $F$ is closed under finite intersections.

Given that $F$ is closed under finite intersections, we follow that intersection of
a finite subset of $F$ is in $F$. Given that $\emptyset \notin F$, we conclude that
intersection of a finite subset of $F$ is nonempty, and thus $F$ posseses finite
intersection property, as desired.

\subsection*{7.2.3}

\textit{Let $a \in X$ and $b \in X$ where $a \neq b$. Prove tha the fiter generated by
  $\set{a, b}$ is not maximal.}

Let us denote the filter generated by $\set{a, b}$ as $S$.
Let $S'$ be the principal filter, generated by $\set{a}$. We follow that $\set{a} \in S'$,
but $\set{a} \notin S$, thus $S' \neq S$. Suppose that $x \in S$. We follow that
$\set{a, b} \subseteq x$. Thus $\set{a} \subseteq x$. Thus $x \in S'$. Therefore we follow that
$S \subseteq S'$. Provided that $S \neq S'$, we conclude that $S$ is not a maximal filter, as
desired.

\subsection*{7.2.4}

\textit{Let $F$ be a filter on a nonempty set $X$. Let $A \in F$. Prove that $\pow(A) \cap F$
  is a filter on $A$.}

Since $A \in F$ and $A \in \pow(A)$, we follow that $A \in F \cap \pow(A)$.

Suppose that $Z \in F \cap \pow(A)$ and there exists $Y$ such that
$Z \subseteq Y \subseteq A$. Since $A \in F$, we follow
that $A \in \pow(X)$, and thus $A \subseteq X$. Thus $Z \subseteq Y \subseteq X$.
Because $Z \in F$, we follow that $Y \in F$, and given that $Y \subseteq A$ we conclude that
$Y \in F \cap \pow(A)$.

Suppose that $X, Y \in F \cap \pow(A)$. We follow that $X \cap Y \in F$. We also follow that
$$X \in \pow(A) \land Y \in \pow(A) \ra
X \subseteq A \land Y \subseteq A \land X \cap Y \subseteq A \ra
X \cap Y \in \pow(A)$$
Thus we follow that $X, Y \in F \cap \pow(A) \ra X \cap Y \in F \cap \pow(A)$. Therefore we
conclude that $F \cap \pow(A)$ is a filter on $A$, as desired.

\subsection*{7.2.5}

\textit{Suppose that $X$ is an infinite set, and let $A$ be a cofinite subset of $X$.
  Prove the following two items: }

\textit{(a) If $A \subseteq B \subseteq X$, then $B$ is a cofinite subset of $X$.}

Lemme expand this exercise a bit and prove that set of cofinite sets constitutes a filter.

Let us prove that
$$F = \set{S \subseteq X: X \setminus S\text{ is finite}}$$
is a filter. We follow that $X \in F$ since $X \setminus X = \emptyset$. We've proven in the
chapter the second property of the filter.
We can also follow that if $A \subseteq B \subseteq X$, then
$X \setminus B \subseteq X \setminus A$, and since all of the subsets of finite sets are finite,
we follow that $X \setminus B$ is finite and thus $B$ is cofinite as well. Thus $F$ is a filter.

\textit{(b) If $C$ is a cofinite subset of $A$, then $C$ is a cofinite subset of $X$.}

Because $C$ is a subset of $A$ we follow that
$$A \setminus (A \setminus C ) = C$$

If $C$ is a cofinite subset of $A$, then $A \setminus C$ is finite, thus
$((A \setminus C) \cap C)$ is finite and thus
$((A \setminus C) \cap C) \cup (X \setminus A)$ is finite. Thus we conclude that 
$$((A \setminus C) \cap C) \cup (X \setminus A)  = X \setminus (A \setminus (A \setminus C)) =
X \setminus C$$
therefore $X \setminus C$ is finite and thus $C$ is cofinite subset of $X$, as desired.

\subsection*{7.2.6}

\textit{Let $F$ be the filter of cofinite subsets of $\omega$. Show that $F$ is not
  an ultrafilter.}

Let $o$ be the set of odd numbers. Then we follow that $\omega \setminus o$ is infinite, thus
$o \notin F$. We can also follow that $\omega \setminus (\omega \setminus o)$
is also infinite, and thus neither $o$ nor $\omega \setminus o$ are in $F$. Thus we conclude that
$F$ is not an ultrafilter.

\subsection*{7.2.[7,8]}

\textit{Prove Lemma 7.2.4}

There's probably a typo in the text of the execise.

Lemma 7.2.4 states that \textit{if $U$ is an ultrafilter on $X$, then}

\textit{(a) For all $A \subseteq X$ and $B \subseteq X$, if $A \cup B \in U$, then either $A \in U$
or $B \in U$.}

\textit{(b) Let $F$ be a filter on $X$. If $U \subseteq F$, then $U = F$.}

Suppose that $A \subseteq X$, $B \subseteq X$ and $A \cup B \in U$. Suppose that neither
$A$ not $B$ are in $U$. We follow that since $U$ is an ultrafilter, that
$X \setminus A \in U$ and $X \setminus B \in U$. Thus
$(X \setminus A) \cap (X \setminus B) \in U$. Therefore by DML
$$(X \setminus A) \cap (X \setminus B) = X \setminus (A \cup B) \in U$$
thus we follow that both $X \setminus (A \cup B)$ and $A \cup B$ are in $U$. Thus
$(X \setminus (A \cup B)) \cap (A \cup B) = \emptyset \in U$, which is a contradiction
of the fact that $U$ is a filter. Thus we conclude that $A \in U$ or $B \in U$.

Suppose that $F$ is a filter on $X$ such that $U \subseteq F$. Suppose that $x \in F$.
We follow that $x \subseteq X$, and thus $x \in U$ or $X \setminus x \in U$.
If latter holds, then $X \setminus x \in U \subseteq F$, thus $X \setminus x, x \in F$
and therefore $(X \setminus x) \cap x = \emptyset \in F$, which is a contradiction. Thus we
conclude that former holds and thus we've got that $F = U$ by double inclusion.

\subsection*{7.2.9}

\textit{Let $X$ be an infinite set and let $F$ be the filter of all cofinite subsets of $X$.
  Suppose $G$ is a filter on $X$ such that $F \subseteq G$.}

\textit{(a) Prove that every $A \in G$ is infinite.}

Suppose that $A$ is not infinite (i.e. finite). Then we follow that 
$X \setminus (X \setminus A) = A$ is finite, thus $X \setminus A \in F$. Thus
$X \setminus A \in G$. Therefore $(X \setminus A) \cap A = \emptyset \in G$, which is a
contradiction.

\textit{(b) Prove that $G$ is nonprincipal.}

Suppose that it is principal. Then we follow that there exists $a \in X$ such that
$\set{a} \in G$. Thus $G$ has finite elements, which is a contradiction.

\textit{(c) Conclude that there is a nonprincipal ultrafilter on $X$.}

7.2.9 implies that there exists ultrafilter $U$ on $X$ such that $F \subseteq U$.

\subsection*{7.2.10}

\textit{Let $F$ be a nonempty set of filters on $X$. Prove that $\bigcap{F}$ is a filter on $X$.}

For completeness' sake let us state that because
$$(\forall f \in F)(f \subseteq \pow(X))$$
thus we follow that $\bigcap{F} \subseteq \pow(X)$.

We can follow that since
$$(\forall f \in F)(X \in F \land \emptyset \notin F)$$
that $X \in \bigcap{F}$ and $\emptyset \notin \bigcap{F}$.

Suppose that $Y \in \bigcap{F}$ and there exists $Z$  such that $Y \subseteq Z \subseteq X$.
We follow that since $Y \in \bigcap{F}$ that
$$(\forall f \in F)(Y \in f)$$
thus $Y \subseteq Z \subseteq X$ implies that
$$(\forall f \in F)(Y \in f \land Y \subseteq Z \subseteq X)$$
thus
$$(\forall f \in F)(Z \in f)$$
therefore $Z \in \bigcap{F}$.

Suppose that $Y, Z \in \bigcap{F}$ are arbitrary. We follow that
$$(\forall f \in F)(Y, Z \in f)$$
since every $f$ is a filter we follow that
$$(\forall f \in F)(Y \cap Z \in f)$$
thus $Y \cap Z \in \bigcap{F}$. Therefore we can conclude that $\bigcap{F}$ satisfies
all of the properties of a filter, and thus it is a filter, as desired.

\subsection*{7.2.11}

\textit{Let $F$ be a filter on $X$ and let $a \in X$.}

\textit{(a) Prove that if $\set{a} \in F$, then $a \in A$ for all $A \in F$}

Suppose that $A \in F$ is arbitrary. We follow that since $\set{a} \in F$ that
$$\set{a} \cap A \in F$$
if $a \notin A$, we follow that $\set{a} \cap A = \emptyset \in F$, which is a contradiction.
Therefore we conclude the desired result.

\textit{(b) Suppose that $F$ is an ultrafilter on $X$ and $\set{a} \in F$. Conclude that
  $F$ is the principal filter on $X$}

Suppose that $S \subseteq X$ is such that $\set{a} \subseteq S$. We follow that
since $F$ is an ultrafilter that $S \in F$ or $X \setminus S \in F$. The latter is impossible
by the fact that $a \notin X \setminus S$, and previous point states otherwise. Thus we
conclude that
$$(\forall S \in \pow(X))(\set{a} \subseteq S \ra S \in F)$$
thus $F$ contains a principal filter of $a$.

Previous point proves that principal filter of $a$ contains $F$. Therefore we conclude that
$F$ is a principal filter of $a$ on $X$ by double inclusion.

\subsection*{7.2.12}

\textit{Let $U$ be an ultrafilter on $X$.}

\textit{(a) Suppose there is a finite set in $U$. Prove that $\set{a} \in U$ for some
  $a \in X$.}

Let $G$ be a finite set in $U$. Suppose that there is no $a \in X$ such that
$\set{a} \in U$. We follow that
$$(\forall a \in X)(\set{a} \notin U)$$
thus
$$(\forall a \in X)(X \setminus \set{a} \notin U)$$
therefore
$$(\forall a \in G)(X \setminus \set{a} \notin U)$$
and since $G$ is finite, we follow that $\set{X \setminus \set{a}: a \in G}$ is finite,
thus
$$\bigcup{\set{X \setminus \set{a}: a \in G}} = X \setminus G$$
is also in $U$. Thus we follow that $(X \setminus G) \cap G = \emptyset \in U$, which is
a contradiction.

\textit{(b) Conclude that if $U$ is nonprincipal, then every $B \in U$ is infinite.}

Previous point implies that if there exists finite $B \in U$, then $U$ has $\set{a}$.
From previous exercise we follow that because $U$ is an ultrafilter that $U$ is a
principal filter. Thus we've got the desired result by the contrapositive.

\subsection*{7.2.13}

\textit{Show that $\pow(X) \setminus U = \set{B \in \pow(X): X \setminus B \in U}$, whenever
  $U$ is an ultrafilter.}

This thing directly follows from the definition of an ultrafilter:

$$y \in \pow(X) \setminus U \lra y \in \pow(X) \land y \notin U \lra
y \in \pow(X) \land X \setminus y \in U$$


\subsection*{7.2.14}

\textit{Let $F$ be a filter on $X$. Let
  $S = \set{E \subseteq \pow(X): E \text{ is a filter on X}}$. Note that $F \in S$. Consider
  the partially ordered set $(S, \subseteq)$.}

\textit{(a) Let $C \subseteq S$ be a chain. Prove that $\bigcup{C} \in S$ and that
  $\bigcup{C}$ is an upper bound for $C$.}

Because $C \subseteq S$ and $S$ is a set of filters, we follow that
$$(\forall c \in C)(X \in c \land \emptyset \notin c)$$
thus we follow that $\emptyset \notin \bigcup{C}$ and $X \in \bigcup{C}$.

Suppose that $Z \in \bigcup{C}$ and there exist $Y \subseteq X$ such that
$Z \subseteq Y \subseteq X$. We follow that since $Z \in \bigcup{C}$ we've
got that there exists $c \in C$ such that $Z \in c$. Thus if $Z \subseteq Y \subseteq X$
we follow that  $Y \in c$ as well, since $c$ is a chain, and thus $Y \in \bigcup{C}$.

Suppose that $Y, Z \in \bigcup{C}$. We follow that there exist $c_1, c_2 \in C$ such that
$Y \in c_1 \land Z \in c_2$. Since $C$ is a chain under $\subseteq$, we follow that
$c_1 \subseteq c_2$ or $c_2 \subseteq c_1$. Assume the former. Then we conclude that
$Y, Z \in c_2$ and thus $Y \cap Z \in c_2$, thus $Y \cap Z \in \bigcup{C}$.

Thus we conclude that $\bigcup{C}$ is a filter, as desired. Suppose that $c \in C$.
Then we folow $c \subseteq \bigcup{C}$, and thus $\bigcup{C}$ is an upper bound for $C$.

\textit{(b) Show, via Corollary 7.1.6 that $S$ has a maximal
  element $U$ such that $F \subseteq U$.}

This result is directly implied by 7.1.6.

\textit{(c) Using Lemma 7.2.6 and Lemma 7.2.8, prove that $U$ is an ultrafilter.}

Suppose that $U$ is not an ultrafilter. We follow that there exists
$y \in \pow(X)$ such that $y \notin U$ and $X \setminus y \notin U$. Lemma 7.2.8
implies that $U \cup \set{y}$ has FIP. 7.2.6 implies that there exists a filter $H$ on $X$
such that $U \cup \set{y} \subseteq H$. Since $y \notin U$ we follow that
$U \neq U \cup \set{y}$ and thus
$U \subset U \cup \set{y} \subseteq H$, thus $U \subset H$, therefore $U$ is not a maximal element,
which is a contradiction.

\subsection*{7.2.15}

\textit{Let $F$ be a filter on $X$. Prove that $F$ is an ultrafilter if and only if $F$ is
  maximal}

Lemma 7.2.4(20) implies that every ultrafilter is maximal. Part (c) of previous exercise provides
a proof that every maximal filter is an ultrafilter.

\subsection*{7.2.16}

\textit{Let $F$ be a filter on $A$. Let $B^A = \set{f: f \text{ is a function from A to B}}$.
Define the relation $\sim$ on $B^A$ by
$$f \sim g \iff \set{x \in A: f(x) = g(x)} \in F$$
for all $f, g \in B^A$. Prove that $\sim$ is an equivalence relation in $B^A$.}

Suppose that $f\in B^A$. Since $f$ is a function, we follow that
$$(\forall x \in X)(f(x) = f(x))$$
thus
$$\set{x \in A: f(x) = f(x)} = A$$
therefore we follow that $A \in F$ since $F$ is a filter on $A$. Thus we follow that
$$(\forall f \in B^A)(f \sim f)$$

Suppose that $f, g, h$ are such that $f \sim g$ and $g \sim h$. We follow that
$$\set{x \in A: f(x) = g(x)} \in F \land \set{x \in A: g(x) = h(x)} \in F $$
Since $F$ is a filter we follow that their intersection is in $F$. Thus 
$$\set{x \in A: g(x) = h(x)} \cap \set{x \in A: f(x) = g(x)} \in F \lra $$
$$\lra \set{x \in A: f(x) = g(x) \land g(x) = h(x)}\in F \lra$$
$$ \lra \set{x \in A: f(x) = h(x)}\in F \lra f \sim h$$
thus we've got that
$$(\forall f, g, h \in B^A)(f \sim g \land g \sim h \ra f \sim h)$$

Suppose that $f, g \in B^A$ are such that $f \sim g$. We follow that
$$f \sim g \lra \set{x \in A: f(x) = g(x)} \in F \lra
\set{x \in A: g(x) = f(x)} \in F \lra g \sim f$$
thus
$$(\forall f, g \in B^A)(f \sim g \ra g \sim f)$$
thus we conclude that $\sim$ is reflexive, transitive and symmetric, and thus it constitutes an
equivalence relation, as desired.

\section{Well-Orderiing Theorem}

\subsection*{7.3.1}

\textit{Let $\eangle{A_i: i \in I}$ be an indexed function such that $A_i \neq \emptyset$ for
  all $i \in I$. By Theorem 7.3.1, the set $\bigcup_{i \in I}{A_i}$ has a well-ordering
  $\preceq$. Define a function $f$ such that $\dom(f) = I$ and
  $(\forall i \in I)(f(i) \in A_i)$. Thus, $\eangle{f(i): i \in I}$ is a choice
  function.}

We follow that $A_j \subseteq \bigcup_{i \in I}{A_i}$, and thus there
exists $\preceq$-least element of $A_j$.

Define $f: I \to \bigcup_{i \in I}{A_i}$ by
$$f(i) = \preceq-\text{least element of } A_i$$
we follow that $(\forall i \in I)(f(i) \in A_i)$, as desired.

\subsection*{7.3.2}

\textit{Let $\preceq$ be a total order on $A$. For each $u \in A$, let
  $s(u) = \set{x \in A: x \prec u}$. Suppose that for every $u \in A$ and $X$,
  if $X \subseteq s(u)$  is nonempty, then $X$ has a $\preceq$-least element.
  Prove that $\preceq$ is a well-ordering on $A$.}

Let $Y \subseteq A$ be nonempty. Because it is nonempty we follow that there exsits $u \in Y$.
If $s(u) = \emptyset$, then we follow that for all $a \in A$ that $a = u$ or $u \prec a$, thus
$u$ is a $\preceq$-least element of both $A$ and $Y$. Suppose that $s(u) \neq \emptyset$.
If $Y \cap s(u) = \emptyset$, then we follow by trichotomy that for all $y \in Y$
$$u = y \lor u \prec y$$
thus once again, $u$ is the $\preceq$-least element of $Y$. If $s(u) \cap Y$ is
nonempty, then we follow that since $s(u) \cap Y \subseteq s(u)$ that
it has a $\preceq$-least element $x$, and thus for all $y \in Y$ we've got that
$x \preceq y$. Therefore we conclude that if $X$ is a subset of $A$, then it has a
$\preceq$-least element, which means that $\preceq$ is a well-order on $A$, as desired.

\subsection*{7.3.3}

\textit{Let $\preceq$ be a well-order on $C$ and let $\preceq'$ be a well-ordering
  on $D$. Suppose that $\preceq \subseteq \preceq'$ and whenever $x \preceq y'$ and
  $y \preceq y$, then $x \preceq y$.}

\textit{(a) Prove that $C \subseteq D$}

Suppose that it isn't. Then we follow that there exists $c \in C$ such that $c \notin D$. Thus
$c \preceq c$, therefore $c \preceq' c$, which contradicts the fact that $\preceq'$ is
a relation on $D$.

\textit{(b) Prove that $\preceq = \preceq' \cap (C \times C)$.}

We follow that $\preceq \subseteq C \times C$ and $\preceq \subseteq \preceq'$, thus
$\preceq \subseteq \preceq' \cap (C \times C)$. Suppose that $\eangle{x, y} \in
\preceq' \cap (C \times C)$ for some $x, y \in C$. We follow that
$x \preceq' y$, and because $\preceq$ is a total order we follow that
$x \prec y$, $x = y$ or $y \prec x$. We follow that $y \prec x$ contradicts the
$x \preceq' y$ since $\preceq \subseteq \preceq'$. Thus we follow that
$x \preceq y$. Therefore we conclude that
$\preceq' \cap (C \times C) \subseteq \preceq$. Double inclusion gives us the desired equality.

\textit{Prove that if $C \neq D$, then $C = \set{z: z \prec' l}$ for some $l \in D$.}

Suppose that $C \neq D$. We follow that $D \setminus C$ is nonempty. Suppose
that there exists $d \in D \setminus C$ such that $d \prec' c$ for some $c \in C$.
We follow that $d \preceq' c$ and $c \preceq c$ by reflexivity of $c$. Thus by restrictions
in the text of the exercise, we follow that $d \preceq c$. Therefore $d \in C$, which is a
contradiction.

Thus we conclude that for all $d \in D \setminus C$, $c \preceq' d$ for all $c \in C$.
Since $D \setminus C$ is nonempty, we follow that it's got a $\preceq'$-least element $l$.
Since $l \in D \setminus C$, we conclude that $l \neq c$ for all $c \in C$.
Thus we follow that $C \subseteq \set{z: z \prec' l}$. Suppose that
$x \in \set{z: z \prec' l}$. We follow that if $x \in D \setminus C$, then both $l \preceq' x$
and $x \prec' l$ are true, which is a contradiction. Thus we follow that $x \in D \cap C$.
Therefore we  follow that $x \in \set{z: z \prec' l} \ra x \in C$, therefore
$\set{z: z \prec' l} \subseteq C$. Double inclusion gives us that
$\set{z: z \prec' l} = C$, as desired.


\subsection*{7.3.4}

\textit{Let $\trianglelefteq$ be the relation defined on $W$ in the proof of 7.3.1. Prove that
  $\trianglelefteq$ is a partial order on $W$.}

We follow thta $\preceq \subseteq \preceq$ and for all $x \in \fld(\preceq)$ we follow that
$x \preceq y \land y \preceq y \ra x \preceq y$. Thus we follow that
$\preceq \trianglelefteq \preceq$.

Suppose that $\preceq, \preceq', \preceq'' \in W$ are such that
$\preceq \trianglelefteq \preceq'$ and $\preceq' \trianglelefteq \preceq''$.
We follow that $\preceq \subseteq \preceq''$ by the transitivity of $\subseteq$.
Suppose that $x, y \in A$ are such that $x \preceq'' y$ and $y \preceq y$.
By the fact that $\preceq \trianglelefteq \preceq'$ we follow that $y \preceq' y$.
By the fact that $\preceq' \trianglelefteq \preceq''$ we conclude that
$x \preceq' y$. By the $\preceq \trianglelefteq \preceq'$ we conclude that
$x \preceq y$. Therefore if $x, y \in A$ are such that
$x \preceq'' y$ and $y \preceq y$, we follow that $x \preceq y$. Therefore
$\preceq \trianglelefteq \preceq''$ by definition. Therefore we conclude that
$\trianglelefteq$ is transitive.

We follow antisymmetry by double inclusion. Therefore we conclude that
$\trianglelefteq$ is a partial order on $W$, as desired.

\subsection*{7.3.5}

\textit{Let $\preceq$ be a well-ordering on a set $X$ and let $a \notin X$. Define
  a relation $\preceq'$ on $X \cup \set{a}$ by
  $$\preceq' =  \preceq \cup \set{\eangle{x, a}: x \in X \lor x = a}$$
}

\textit{(a) Show that $\preceq'$ is well-ordering on $X \cup \set{a}$}

Let $A \subseteq X \cup \set{a}$ and $A \neq \emptyset$.
If $a \notin A$, then $A \subseteq X$, and thus
$A$ has the $\preceq$-least element $l$. Since $l \neq A$, we follow that $l$ is
a $\preceq'$-least element of $A$.

If $a \in A$, then we've got two cases: $A = \set{a}$ or $A \neq \set{a}$. In the former
case we follow that $a$ is the only element of $A$ and thus it is $\preceq'$-least
element of $A$. If $A \neq \set{a}$, then we follow that there exists $q \in A$ such that
$q$ is the $\preceq$-least element of $A \setminus \set{a}$, and because
$\eangle{q, a} \in \preceq' \lra q \preceq' a$, we follow that $q$ is the $\preceq'$-least
element of $A$. Therefore we conclude that every nonempty subset of $X \cup \set{a}$ contains
a $\preceq'$-least element, and thus $\preceq'$ is a well-order, as desired.

\textit{(b) Prove that $\preceq \trianglelefteq \preceq'$, where $\trianglelefteq$
  is the relation defined in (7.8.) and in the exercise 4.}

We follow that
$\preceq \subseteq \preceq'$ by definition of $\preceq'$. Suppose that
$x, y \in X \cup \set{a}$ are such that $x \preceq' y$ and $y \preceq y$.
We follow that $y \neq a$ since $a \notin \fld(\preceq)$. If $x = a$, then we follow that
$y \preceq' a$ by definition, and since $y \neq a$, we follow that $y \prec' a$, which
is a contradiction of trichotomy of $\prec'$ since $x \preceq' y$. Thus we follwo that
$x \neq a$ as well. Therefore we follow that $x, y \in \fld(\preceq)$, and thus
$x \preceq' y$ implies that $x \preceq y$. Therefore $\preceq \trianglelefteq \preceq'$,
as desired.

\subsection*{7.3.6}

\textit{Supposet hat $\preceq$ is a well-ordering on $A$ and that $\preceq'$ is a well-ordering
  on $B$. Let $\preceq_l$ be the lexicographic ordering on $A \times B$. Prove that $\preceq_l$
  is a well-ordering.}

I think that it was proven before (in 3.4.19 to be precise) that
if $\preceq$ and $\preceq'$ are total orders, then $\preceq_l$ is a total order as well.

Let $Q \subseteq A \times B$ is such that $Q \neq \emptyset$. Let
$$A' = \set{a \in A: (\exists b \in B)(\eangle{a, b} \in Q)}$$
Since $A' \subseteq A$, and $A'$ is nonempty by the fact that $Q$ is nonempty,
we follow that it's got a $\preceq$-least element $a'$ in $A'$.
Let
$$B' = \set{b \in B: \eangle{a', b} \in Q}$$
since $B' \subseteq B$ and there exists $\eangle{a', b} \in Q$ by the construction of $A'$,
we conclude that there exists $\preceq'$-least elenent $b'$ in $B'$.

Construction of those sets imply that $\eangle{a', b'} \in Q$ and
$$\forall(q \in Q)(\eangle{a', b'} \preceq_l q)$$
thus $\eangle{a', b'}$ is the $\preceq_l$-least element of $Q$. Since $Q$ is arbitrary,
we conclude that every nonempty subset of $A \times B$ has a $\preceq_l$-least element,
thus making $\preceq_l$ a well-order.

\chapter{Ordinals}

\section{Ordinal Numbers}

\subsection*{8.1.1}

\textit{Suppose that $(A, \preceq)$, $(B, \preceq^*)$, and $(C, \preceq^\#)$ are all
  well-ordered structured. Prove the following: }

\textit{(a) $(A, \preceq) \cong (A, \preceq)$}

We can follow that the identity function is the desired isomorphism

\textit{(b) If $(A, \preceq) \cong (B, \preceq^*)$, then $(B, \preceq^*) \cong (A, \preceq)$}

We can take the isomorphism, for which it is true that
$(A, \preceq) \cong (B, \preceq^*)$, and prove that the inverse of this isomorphism is
an isomorphism that gives us
$(B, \preceq^*) \cong (A, \preceq)$.

\textit{(c) If $(A, \preceq) \cong (B, \preceq^*)$ and $(B, \preceq^*) \cong (C, \preceq^\#)$ ,
  then $(A, \preceq) \cong (C, \preceq^\#)$}

Composition of two isomorphisms, for which it is true that
$(A, \preceq) \cong (B, \preceq^*)$ and $(B, \preceq^*) \cong (C, \preceq^\#)$
is an isomorphism as well, for which it is true that
$(A, \preceq) \cong (C, \preceq^\#)$

\textit{Conclude that two well-ordered structures have the same order type if and only if
  the structures are isomorphic.}

Suppose that two well-ordered structures $(A, \preceq)$ and $(B, \preceq^*)$ have
the same order type $\delta$. We follow by remark 8.1.8 (which comes from a proof of 8.1.4)
that there exist isomorphisms
$H: A \to \delta$ and $H': B \to \delta$ to a well-ordered structure $(\delta, \ineq)$.
Point (b) implies that $H'\inv: \delta \to B$ is an isomorphism as well. Point (c)
shows that $H'\inv \circ H: A \to B$ is an isomorphism from $A$ to $B$. Therefore we
follow that $(A, \preceq) \cong (B, \preceq^*)$, as desired.

Reverse implication was proven in the chapter (see Theorem 8.1.10)

\subsection*{8.1.2}

\textit{Let $\alpha$ be an ordinal, and so $(\alpha, \ineq)$ is a well-ordered structure.
  By the proof of Theorem 8.1.4, there is a function $H$ with domain $\alpha$
  satisfying 
  $$H(u) = \set{H(x): x \in u} \text{ for all } u \in \alpha$$
  Show that $H(u) = u$ for all $u \in \alpha$ and conclude that $\ran(H) = \alpha$}

Assume that $C = \set{u \in \alpha: u \neq H(u)}$ is nonempty. Let $a \in \alpha$ be the
$\ineq$-least element in $C$. Thus we follow that
$$(\forall x \in a)(H(x) = x)$$
Thus
$$H(a) = \set{H(x): x \in a} = \set{x: x \in a} = a$$
thus $H(a) = a$, which is a contradiction of our assumption that $a \in C$. Thus we follow that
$C$ is empty. Thus we've got that
$$(\forall u \in A)(H(u) = u)$$
thus we conclude that $H$ is an identity relation, and thus $\ran(H) = \alpha)$, as desired.

\subsection*{8.1.3}

\textit{Let $\alpha$ and $\beta$ be ordinals. Prove that either $\alpha \subseteq \beta$ or
  $\beta \subseteq \alpha$.}

Since $\alpha$ and $\beta$ are both oridnals, we follow that either $\alpha \ineq \beta$
or $\beta \ineq \alpha$. Assume the former. If $\alpha = \beta$, then we're done.
If $\alpha \in \beta$, then we follow that $\alpha \subseteq \beta$ by transitivity of
ordinals. Therefore we've got the desired conclusion.

\subsection*{8.1.4}

\textit{Suppose $\omega \ineq \alpha$, where $\alpha$ is an ordinal. Show that $\alpha$
  is an infinite set.}

We follow that since $\omega \ineq \alpha$ that $\omega \subseteq \alpha$. Thus we follow
that there exists an injection from $\omega$ to $\alpha$, which implies that $\alpha$ is
infinite, as desired.

\subsection*{8.1.5}

\textit{Let $\gamma$ be an ordinal and let $\beta \in \gamma$. Since $\gamma$ is well-ordered
  by $\ineq$, the initial segment up to $\beta$ is the set
  $s^\in(\beta) = \set{x \in \gamma: x \in \beta}$. Show that $s^\in(\beta) = \beta$.}

Since $\beta \in \gamma$, we follow that $\beta \subseteq \gamma$. Thus we follow that
$$\set{x \in \gamma: x \in \beta} = \set{x \in \beta} = \beta$$
as desired.

\subsection*{8.1.6}

\textit{Suppose $(A, \preceq)$ is a well-ordered structure, and let $H$ and $G$ be
  functions with domain $A$ satisfying
  $$H(u) = \set{H(x): x \prec u} \text{ for all } u \in A$$
  $$G(u) = \set{G(x): x \prec u} \text{ for all } u \in A$$
  Prove, without appealing to Theorem 6.2.5 that $H = G$
}

I don't know if appealing to 8.1.4 counts as appealing to 6.2.5, but we can follow that
both of those functions are isometries from $(A, \preceq)$ to $(\delta, \ineq)$,
where $\delta$ is an order type of $(A, \preceq)$. Suppose that $H \neq G$.

Let $C = \set{a \in A: H(a) \neq G(a)}$. We follow that there exists $\preceq$-least element
$c \in C$. Thus we follow that
$$(\forall y \prec c)(H(a) = G(a))$$
We follow that
$$H(c) = \set{H(x): x \prec c} = \set{G(x): x \prec c} = G(c)$$
which is a contradiction of the fact that $c \in C$. Thus we follow that
$(\forall a \in A)(H(a) = G(a))$, as desired.

\subsection*{8.1.7}

\textit{Let $(A, \preceq)$ and $(B, \preceq^*)$ be well-ordered structures with order type
  $\delta$ and $\delta^*$ respectively. Suppose that $f: A \to B$ is a one-to-one
  function satisfying $x \preceq y$ if and only if $f(x) \preceq^* f(y)$, for all
  $x$ and $y$ in $A$. Let $H$ and $H^*$ be as in the proof of Theorem 8.1.10. Thus,
  $\delta = \ran(H)$ and $\delta = \ran(H^*)$.}

\textit{(a) Prove that $H(u) \ineq H^*(f(u))$ for all $u \in A$.}

Define $C = \set{x \in A: H^*(f(x)) \in H(x)}$. Assume that $C$ is
nonempty so that there exists
$\preceq$-least element of $C$, which we denote by $c$. We follow that
$$(\forall a \in c)(H(a) \ineq H^*(f(a)))$$
Therefore we follow that
$$H(c) = \set{H(x): x \prec c}$$
Let $y \in H(c)$. We follow that there exists $k \prec c$ such that $y = H(k)$. Thus
we follow that $y \in H^*(f(k))$ and by definition of $H^*$ we follow that $y \in H^*(f(c))$.
Therefore we follow that $H(c) \subseteq H^*(f(c))$.
Therefore we can follow that $H(c) \ineq H^*(f(c))$, which contradicts our assumtion
that $c \in C$. Thus we follow that $C$ is empty and thus
$$(\forall a \in A)(H(a) \ineq H^*(f(a)))$$
as desired.

\textit{(b) Prove that $\ran(H) \subseteq \ran(H^*)$}

Suppose that $x \in \ran(H)$. We follow that there exists $a \in A$ such that
$x = H(a)$. From previous point we follow that $x \ineq H^*(f(a))$.
If $x = H^*(f(a))$, then we're done. Otherwise if $x \in H^*(f(a))$, then we follow that
$x \in H^*(f(a))$ and $H^*(f(a)) \in \ran(H^*)$, which gives us that $x \in \ran(H^*)$
by the fact that $\ran(H)$ is a set of ordinals.

\textit{(c) Prove that $\delta \ineq \delta^*$}

We follow that by previous point.

\subsection*{8.1.8}

\textit{Let $(A, \preceq)$ be a well-ordered structure with order type $\alpha$. Let
  $\gamma$ be the order type of $(C, \preceq_C)$ where $C \subseteq A$ and $\preceq_C$
  is the induced relation. Using exercise 7, show that $\gamma \ineq \alpha$.}

We follow that the cropped identity function is an injection that gives us the desired result.

\subsection*{8.1.9}

\textit{Let $\alpha$ and $\beta$ be ordinals.}

\textit{(a) Prove that there is no ordinal $\gamma$ such that $\alpha \in \gamma \in \alpha^+$}

Suppose that such a set exists. We follow that since $\gamma \in \alpha^+$ that
$\gamma \in \alpha$ or $\gamma = \alpha$. Former case violates the fact that
$\alpha \in \gamma$, the latter case gives us that $\alpha \in \alpha$, which is also
a contradiction. Thus we follow that there is no such set. Given that ordinals are sets,
we follow that no such ordinals exist.

\textit{(b) Suppose that $\alpha \in \beta$. Prove that $\alpha^+ \ineq \beta$.}

Because $\alpha$ is an ordinal, we follow that $\alpha^+$ is an ordinal as well.
Therefore we need to only prove that the case $\beta \in \alpha^+$ is impossible.
We follow that this case is impossible by previous point (i.e. we've got that
$\alpha \in \beta \in \alpha^+$).

\textit{(c) Prove that if $\alpha \in \beta$, then $\alpha^+ \in \beta^+$}

We can state that both $\alpha^+$ and $\beta^+$ are ordinals.
Suppose that $\alpha^+ = \beta^+$. Then we follow that
$$\alpha \in \beta \in \beta^+ \lra \alpha \in \beta \in \alpha^+ $$
which is a contradiction by point (a).

Suppose that $\beta^+ \in \alpha^+$. By the fact that $\alpha \in \beta \in \beta^+$ we've got
that $\alpha \in \beta^+$. Thus we've got thaht $\alpha \in \beta \in \alpha^+$, which is
a contradiction by point (a).

\textit{(d) Prove that if $\alpha^+ = \beta^+$, then $\alpha = \beta$}

Suppose that $\alpha^+ = \beta^+$ and $\alpha \neq \beta$. We follow that
$\alpha \in \beta$ or $\beta \in \alpha$. Former case implies that $\alpha^+ \in \beta^+$,
and latter case implies that $\beta^+ \in \alpha^+$ by previous point. Therefore
we conclude that $\alpha = \beta$ implies that $\alpha^+ = \beta^+$.

\textit{(e) Prove that if $\alpha \in \beta$, then $\alpha \cap \beta = \alpha$.}

If $\alpha  \in \beta$, then we follow that $\alpha \subseteq \beta$ by transitivity of
ordinals. This implies that $\alpha \cap \beta = \alpha$, as desired.

\subsection*{8.1.10}

\textit{Let $\alpha$ be a nonzero ordinal}

\textit{(a) Prove that $\alpha$ is a limit ordinal iff
  $(\forall \lambda \in \alpha)(\exists \gamma)(\lambda \in \gamma \in \alpha)$.}

Suppose that $\alpha$ is a limit ordinal. This means that
$$(\forall \lambda \in \alpha) (\lambda^+ \in \alpha)$$
thus we follow that
$$(\forall \lambda \in \alpha) (\lambda \in \lambda^+ \in \alpha) \ra
(\forall \lambda \in \alpha)(\exists \gamma \in \alpha)(\lambda \in \gamma \in \alpha)$$

Suppose that
$$(\forall \lambda \in \alpha)(\exists \gamma \in \alpha)(\lambda \in \gamma \in \alpha)$$
Suppose that $\lambda \in \alpha$ and $\gamma \in \alpha$ is such an ordinal
that $\lambda \in \gamma \in \alpha$. Suppose that $\lambda^+ \notin \alpha$.
Then we follow that $\alpha \ineq \lambda^+$. If $\alpha = \lambda^+$, then
$\lambda \in \gamma \in \lambda^+$, which is a contradiction.
If $\alpha \in \lambda^+$, then $\lambda \in \alpha \in \lambda^+$, which is also
a contradiction. Thus we conclude that $\lambda^+ \in \alpha$, as desired.

\textit{(b) Prove that $\alpha$ is a limit ordinal if and only if $\alpha = \sup \alpha$.}

If $\alpha$ is a any ordinal and $x \in \sup{\alpha}$ then we follow that there
exists $a \in \alpha$ such that $x \in a \in \alpha$. Thus $x \in \alpha$. Therefore
if $\alpha$ is an ordinal, then $\sup \alpha \subseteq \alpha$. 

Suppose that $x \in \alpha$. Then we follow that
$x \subseteq \bigcup \alpha$. Thus $x \ineq \bigcup{\alpha}$. If $x = \bigcup{\alpha}$,
then we follow that $\bigcup{\alpha} \in x^+$. We know that $x \in \alpha$ implies that
$x^+ \in \alpha$, which gives us that $x^+ \subseteq \bigcup{\alpha}$, and thus
$x^+ \ineq \bigcup{\alpha}$, which violates our previous
conclusion that $\bigcup{\alpha} \in x^+$. Therefore we follow that $x \neq \bigcup{\alpha}$,
thus $x \in \bigcup{\alpha}$, and thus we follow that if $\alpha$ is a limit ordinal,
then $\alpha = \sup\alpha$ by double inclusion.

Suppose that $\beta$ is a successor ordinal. Then we follow that there exists $\gamma$
such that $beta = \gamma \cap \set{\gamma}$. Given that
$(\forall x \in \gamma)(x \subseteq \gamma)$, we follow that $\sup \beta =
\bigcup{\gamma \cap \set{\gamma}} = \gamma \in \beta$. Thus we follow that
if $\beta$ is a successor ordinal, then $\sup \beta \in \beta$ and thus
$\sup \beta \subset \beta$. Thus we follow that if $\alpha = \sup \alpha$, then
$\alpha$ is not a successor ordinal, therefore it is a limit ordinal, as desired.

\subsection*{8.1.11}

\textit{Suppose that $(A, \preceq)$ is a well-ordered structure and $a \in A$.
  Let $s(a) \subseteq A$ be defined by $s(a) = \set{x \in A: x \prec a}$.
  Thus, $(s(a), \preceq_{s(a)})$ is a well-ordered structure. Let $\delta$ be the
  order type of $(A, \preceq)$ and let $H: A \to \delta$ be as in the proof of 8.1.4.
  Show that $\eta$ is the order type of $(s(a), \preceq_{s(a)})$ if and only if
  $H(a) = \eta$.}

Suppose that order type of $(s(a), \preceq_{s(a)})$ is $\eta$. We follow that for it
there exists $H^*: s(a) \to \eta$ from theorem 8.1.4 such that $\eta = \ran(H^*)$. From the
proof of 8.1.4 we follow that $(\forall x \prec x)H^*(x) = H(x)$. Given that
$(\forall x, y \in s(a))(x \preceq_{s(a)} y \iff x \preceq y)$,
we follow that for $x \in s(a)$
$$H^*(x) = \set{H^*(y): y \prec_{s(a)} x} = \set{H^*(y): y \prec x} =
\set{H(y): y \prec x} = H(x)$$
thus we follow that $\eta = \ran(H^*) = \set{H^*(y): y \prec_{s(a)} x} = \set{H(y): y \prec x} =
H(x)$.

Suppose that $\eta = H(x)$. By pretty much the same definitions we can follow the desired
result.

\subsection*{8.1.12}

\textit{Let $(A, \preceq)$ and $(C, \preceq^*)$ be well-ordered structures. Let $\alpha$
  be the order type of $(A, \preceq)$ and let $\eta$ be the order type of $(C, \preceq^*)$.
  Suppose that $\eta \in \alpha$. Prove that there exists an $a \in A$ such that
  $(s(a), \preceq_{s(a)} \cong (C, \preceq^*)$, where $s(a) = \set{x \in A: x \prec a}$}

Define function $H$ and $H^*$ as in theorem 8.1.4 (those things really need a name)
for $(A, \preceq)$ and $(C, \preceq^*)$
respectively. We follow that $H$ is an isomorphism between $(A, \preceq)$ and
$(\alpha, \ineq)$. Since $H$ is an isomorphism, we follow that it is bijective,
and thus for $\eta \in \alpha$, there exists $a \in A$ such that
$H(a) = \eta$. Previous exercise implies that order of  $(s(a), \preceq_{s(a)})$ is
$\eta$. Thus we follow that two structures $(s(a), \preceq_{s(a)})$ and
$(C, \preceq^*)$ have the same order type and thus they are isomorphic by exercise 1.


\subsection*{8.1.13}

\textit{Let $\delta = \sup{C}$ where $C$ is a set of ordinals. }

\textit{(a) Show that if $\alpha \in C$, then $\alpha \ineq \delta$}

If $\alpha \in C$, then $\alpha \subseteq \bigcup{C}$, thus $\alpha \subseteq \sup{C}$
and therefore $\alpha \subseteq \delta$, which implies that $\alpha \ineq \delta$.

\textit{(b) Let $\gamma$ be an ordinal. Show that if $\alpha \ineq \gamma$ for
  all $\alpha \in C$, then $\delta \ineq \gamma$.}

Suppose that $x \in \delta$. Thus $x \in \bigcup{C}$, therefore
there exists $c \in C$ such taht $x \in c$. Thus $x \in c \ineq \gamma$. Therefore
$x \in \gamma$. Thus we follow that $\delta \subseteq \gamma$, which implies that
$\delta \ineq \gamma$. Therefore $\delta$ is indeed a least upper bound for $C$
(with respect to $\ineq$), as desired.

\subsection*{8.1.14}

\textit{Let $C$ be a nonempty set of ordinals and let $\gamma$ be the $\ineq$-least
  element in $C$. Prove that $\gamma = \inf C$.}

Suppose that $x \in \gamma$. Since $\gamma$ is the $\ineq$-least
element in $C$, we follow that $(\forall c \in C)(\gamma \ineq c)$. Thus
$(\forall c \in C)(x \in \gamma \ineq c)$ and thus $(\forall c \in C)(x \in c)$. Thus
$x \in \inf(C)$.

Suppose that $x \in \inf C = \bigcap{C}$. We follow that $(\forall c \in C)(x \in c)$
Since $\gamma \in C$, we follow that $x \in \gamma$. Thus $\inf(C) \subseteq \gamma$.
Therefore we follow that $\gamma = \inf(C)$ by double inclusion.

\subsection*{8.1.15}

\textit{Prove that there is no set that contains every successor ordinal.}

Suppose that set $B$ contains every successor ordinal. Suppose that $\alpha$ is an
arbitrary ordinal. Then we follow that $\alpha^+$ is a successor ordinal. Thus
$\alpha^+ \in B$. Therefore $\alpha \in \bigcup{B}$. Therefore $\bigcup{B}$
is a set that contains every ordinal, which is a contradiction.

\section{Ordinal Recursion and Class Functions}

\subsection*{8.2.1}

\textit{Let $C$ be a set of ordinals with largest element $\gamma$. Show that $\sup{C} = \gamma$}

Largest element in this case means that
$$(\forall \beta \in C)(\beta \ineq \gamma)$$

Suppose that $x \in \sup{C}$. We follow that $x \in \bigcup{C}$ and therefore
$(\exists \delta \in C)(x \in \delta)$. Since $\delta \in C$, we follow that
$\delta \ineq \gamma$. Thus $x \in \delta \ineq \gamma$, therefore
$\sup{C} \subseteq \gamma$.

Suppose that $x \in \gamma$. Since $\gamma \in C$, we follow that $\gamma \subseteq \bigcup{C}$.
Thus $\gamma \ineq \bigcup{C}$, and therefore $x \in \gamma \ineq \bigcup{C}$. Thus
$x \in \bigcup{C}$, therefore $x \in \sup{C}$, and thus $\gamma \subseteq \sup{C}$.
Double inclusion gives us the desired equality.

\subsection*{8.2.2}

\textit{Let $F: \On \to \On$ be strictly increasing. Prove that $F$ is one-to-one.}

Suppose that it isn't. We follow that there exist $\alpha, \beta$
such that $\alpha \neq \beta$ and  $F(\alpha) = F(\beta)$. Since
$\alpha, \beta \in \On$ and $\alpha \neq \beta$, we follow that $\alpha \in \beta$
or $\beta \in \alpha$. Assume the former. Then we follow that $\alpha \in \beta$
and $F(\alpha) = F(\beta)$, which contradicts our assumption that $F$ is strictly
increasing. THus we conclude that $F$ is injective, as desired.

\subsection*{8.2.3}

\textit{Prove Lemma 8.2.12}

8.2.12 states that
\textit{Suppose that $C$ is a limit set. Then }

\textit{(a) $\sup C$ is a limit ordinal.}

\textit{(b) C is cofinal in $\sup C$}

Now let us assume that $C$ is a limit set. Let $x \in \sup{C}$. We follow that
$(\exists c \in C)(x \in c)$. Since $C$ is a limit set, we follow that there exists $\beta \in C$
such that $c \in \beta$. Therefore we've got that $x \in c \in \beta$. Since $\beta \in C$,
we follow that $\beta \subseteq \sup C$, and thus $\beta \ineq \sup C$. Therefore
$x \in c \in \beta \ineq \sup C$, thus $x \in c \in \sup C$. Therefore we follow that
$$(\forall x \in \sup C)(\exists c \in \sup C)(x \in c \in \sup C)$$
exercise 10(a) in previous section implies that $\sup C$ is a limit ordinal.

We've followed in previous section's exercises that for every ordinal we've got that
$C \subseteq \sup{C}$. Suppose that $\alpha \in \sup{C}$.
We follow that $(\exists \beta \in C)(\alpha \in \beta)$ by definition. Thus we follow
that $C$ is a cofinal in $D$ by definition.

\subsection*{8.2.4}

\textit{Prove Lemma 8.2.13}

Lemma 8.2.13 states that
\textit{Let $F: \On \to \On$ be normal and let $D$ be a limit set. If $C$ is
  cofinal in $D$, then $F[D]$ is a limit set and $F[C]$ is cofinal in $F[D]$.}

Assume the premises of 8.2.13. Suppose that $C$ is cofinal in $D$.

Suppose that $\alpha \in F[D]$. We follow that there exists $\eta \in D$ such that
$F(\eta) = \alpha$. Because $D$ is a limit set, we follow that there exists $\mu \in D$
such that $\eta \in \mu$. Because $F$ is normal we follow that it is strictly increasing,
and thus $F(\eta) \in F(\mu)$. Since $\mu \in D$, we follow that there exists $\beta \in F[D]$
such that $F(\mu) = \beta$. Therefore we follow that $F(\eta) \in F(\mu) \ra \alpha \in \beta$.
Thus we follow that $F[D]$ is a limit set, as desired.

Because $C \subseteq D$ we follow that $F[C] \subseteq F[D]$ (this is not only true
for set functions, the proof is trivial). Suppose that $\alpha \in F[D]$.
We follow that there exists $\gamma \in D$ such that $\alpha = F(\gamma)$. Since
$C$ is cofinal in $D$, we follow that there exsits $\delta \in C$ such that
$\gamma \in \delta$. This implies that $F(\gamma) \in F(\delta)$ by the fact that
$F$ is strictly increasing. Let $\beta = F(\delta)$. Then we follow that
for all $\alpha \in F[D]$, we've got $\beta \in F[C]$ such that $\alpha \in \beta$.
Therefore we conlcude that $F[C]$ is cofinal in $F[D]$.

For some reason we haven't used the continuity of $F$, which leads me to beleve that
I've fudged somemthing up, although I don't see no problems with the proof.

\subsection*{8.2.5}

\textit{Prove Lemma 8.2.14}

Lemma 8.2.14 states
\textit{Let $D$ be a limit set. If $C$ is cofinal in $D$, then $\sup C = \sup D$.}

Suppose that $x \in \sup{D}$. We follow that $(\exists d \in D)(x \in d)$.
Since $C$ is cofinal in $D$, we follow that there exists $c \in C$ such that
$d \in c$. Therefore $x \in d \in c$. Thus $x \in c$. Therefore
$x \in \sup C$. Thus $\sup {D} \subseteq \sup{C}$. 

Since $C \subseteq D$, we follow that $\bigcup{C} \subseteq \bigcup{D}$, and thus
$\sup C \subseteq \sup D$. Thus by double inclusion we've got that $\sup C = \sup D$,
as desired.

\subsection*{8.2.6}

\textit{Let $C$ be a limit set. Using Theorem 4.2.1, define a one-to-one function
  $h: \omega \to C$. Conclude that $C$ is infinite.}

Since $C$ is a limit set, we follow that there exists a $\ineq$-least element of $C$.
Denote this element by $q$.

Since $C$ is a limit set we follow that for every $c \in C$ there exists $d \in C$
such that $c \in d$. Define $f: C \to C$ by
$$f(c) = \ineq\text{-least element of } C \setminus s^{\ineq}(c)$$
the fact that $C$ is a limit set tells us that $C \setminus s^{\ineq}(c)$ is
nonempty for all $c \in C$.

Thus, by theorem 4.2.1 we can follow that there exists a function $h: \omega \to C$
which is defined by
$$h(0) = q$$
$$h(n^+) = f(h(n))$$

Suppose that $c_1, c_2 \in C$ is such that $c_1 \in c_2$. Suppose that
$f(c_1) = f(c_2)$. We follow that $c_2 \in C \setminus s^{\ineq}(c_1)$ and $c_2 \in f(c_2)$.
and $f(c_1) \in c_2$. Thus we follow that $f(c_1) = f(c_2)$ presents a contradiction.
Therefore we conclude that $f$ is injective and thus $h$ is injective as well, as desired.

The fact that we've got injective $h: \omega \to C$ implies that $C$ is infinite.

\subsection*{8.2.7}

\textit{Let $F: \On \to On$ be normal and let $C$ be a limit set. Prove that $F[C]$
  is also a limit set.}

I think that this was proven in the proof of 8.2.13 in previous exercises.

\subsection*{8.2.8}

\textit{Using exericse 7, prove Corollary 8.2.16}

8.2.15 implies that $\sup F[C] = F(\sup C)$. If $\gamma$ is a limit ordinal, then
$\sup \gamma = \gamma$, and thus $\sup F[\gamma] = F(\gamma)$. Therefore
we follow that $F(\gamma)$ is an ordinal. Exercise 7 implies that $F[\gamma]$ is
a limit set. Lemma 8.2.12 implies that $\sup F[\gamma]$ is a limit ordinal, and thus
$F(\gamma)$ is a limit ordinal, as desired.

\subsection*{8.2.9}

\textit{Suppose that $F: \On \to \On$ is strictly increasing. Prove that
  $\alpha \ineq F(\alpha)$ for all ordinals $\alpha$.}

We're going to employ the transfinite recursion on this one.
Our IH is that
$$(\forall \eta \in \beta)(\eta \ineq F(\eta))$$

We follow that $\beta \in \On \ra \emptyset \ineq \beta$. Thus $\emptyset \ineq F(\emptyset)$.

Assume that $\beta$ is a nonempty ordinal (we're going to do both limit and successor
cases here). If $\alpha \in \beta$,
then $\alpha \ineq F(\alpha)$. Since $\alpha \in \beta$, we follow that $F(\alpha) \in F(\beta)$,
therefore $\alpha \in F(\beta)$. Thus $\beta \subseteq F(\beta)$, and therefore
$\beta \ineq F(\beta)$. Thus we follow that if $\beta$ is an ordinal, then $\beta \in F(\beta)$,
as desired.

\subsection*{8.2.10}

\textit{Suppose $F: \On \to \On$ and $G: \On \to On$ are normal.}

\textit{(a) Prove that the composition $(F \circ G): \On \to \On$ is normal.}

Suppose that $\alpha, \beta \in \On$ and $\alpha \in \beta$. Then we follow that
$G(\alpha) \in G(\beta)$, and thus $F(G(\alpha)) \in F(G(\beta))$,
therefore $F \circ G$ is strictly increasing.

Let $\beta$ be a limit ordinal. Then we follow that $\beta = \sup \beta$, and thus 
$$F(G(\beta)) = F(G(\sup \beta)) = F(\sup\set{G(\gamma): \gamma \in \beta}) =
\sup \set{F(\theta): \theta \in \set{G(\gamma): \gamma \in \beta}} =$$
$$ =
\sup \set{F(G(\gamma)): \gamma \in \beta} $$
therefore concluding that $F \circ G$ is continous.

Thus we follow that $F \circ G: \On \to \On$ is a normal function, as desired.

\textit{(b) Show that if $F(G(\alpha)) = \alpha$, then $G(\alpha) = \alpha$ and
  $F(\alpha) = \alpha$.}

From exercise 9 we know that if $F$ is strictly increasing, then $\alpha \in F(\alpha)$.

Assume that $G(\alpha) \neq \alpha$. Then we follow that $\alpha \in G(\alpha)$.
From this we conclude that $G(\alpha) \ineq F(G(\alpha ))$, thus
$\alpha \in F(G(\alpha))$, and thus $\alpha \neq F(G(\alpha))$. Simular thing holds for
$F(\alpha) \neq \alpha$. Therefore we conclude that $F(\alpha) = G(\alpha) = \alpha$, as desired.

\subsection*{8.2.11}

\textit{Let $F, C$, and $h$ be as in proof of Theorem 8.2.19}

\textit{(a) Prove that $h(n) \in h(n^+)$ for all $n \in \omega$. Conclude that $C$ is a
  limit set.}

We've got that $h(n): \omega \to \On$, and thus we've got that
for a particular $n \in \omega$ there are three cases:  $h(n) \in h(n^+)$,
$h(n) = h(n^+)$ and $h(n^+) \in h(n)$.
Since $F$ is normal, we follow that it is strictly increasing, and thus we've got that
$\alpha \ineq F(\alpha)$. Thus we follow that
$$h(n) \ineq F(h(n)) \ra h(n)  \ineq h(n^+)$$
therefore the last case is impossible.

I think that at that point in the proof it is assumed that $F(\gamma) \neq \gamma$.
Thus we follow that $h(0) \in F(h(0))$. By the fact that $F$ is strictly increasing
we follow that $F(h(0)) \in F(F(h(0)))$, thus $h(0^+) \in h((0^+)^+)$. Thus we can
draw up a pretty simple induction proof.

Let
$$I = \set{n \in \omega: h(n) \in h(n^+)}$$
we follow that $0 \in I$ by the stuff that we've proven before. Assume that $n \in I$.
Then $h(n) \in h(n^+)$, and therefore $F(h(n)) \in F(h(n^+))$, thus $h(n^+) \in h((n^+)^+)$,
thus following that $n \in I \ra n^+ \in I$. Thus $I = \omega$, and therefore we've got our
proof.

$C$ is defined to be a range of $h$ (pretty sure that we can use set notation in this case
since domain of $h$ is a set). Assume that $c \in C$. Then $c = h(n)$ for some $n \in \omega$.
Thus there exists $h(n^+) \ in C$ such that $c = h(n) \in h(n^+)$, therefore proving that
$C$ is a limit set, as desired.

\textit{(b) Prove that $F[C]$ is cofinal in $C$.}

Let $x \in F[C]$. Therefore there exsits $c \in C$ such that $x = F(c)$. Since
$c \in C$, we follow that there exists $n \in \omega$ such that $c = h(n)$.
Therefore $x = F(h(n))$. Thus $x = h(n^+)$. Therefore $x \in C$.
Thus $F[C] \subseteq C$.

Assume that $c \in C$. We follow that $c = h(n)$ for some $n \in \omega$. Therefore
$F(c) \in F[C]$ as well. $F(c) = h(n^+)$ and also $c = h(n) \in h(n^+)$,
thus we follow that for all $c \in C$ we've got $h(n^+) \in F[C]$ such that
$c \in h(n^+)$. Therefore $F[C]$ is cofinal in $C$, as desired.

\subsection*{8.2.12}

\textit{Let $F: \On \to \On$ be normal. Suppose $F(\beta) = \beta$. Show that
  for all $\alpha$, if $\alpha \in \beta$, then $F(\alpha) \in F(\beta)$}

Since $F$ is normal and therefore strictly increasing, we follow
that $\alpha \in \beta \ra F(\alpha) \in F(\beta)$. Since $F(\beta) = \beta$, we follow that
$\alpha \in \beta \ra F(\alpha) \in \beta$, as desired.

\subsection*{8.2.13}

\textit{Let $F: \On \to \On$ be normal, and suppose that $C$ is a limit set such that the
  set $D = \set{\alpha \in C: F(\alpha) = \alpha}$ is cofinal in $C$. Let $\gamma = \sup C$.
  Prove that $F(\gamma) = \gamma$. }

Since $D$ is cofinal in $C$, we follow that $\sup C = \sup D$ by 8.2.14. This implies that
$\gamma = \sup C = \sup D$.

$$F(\gamma) = F(\sup C) = F(\sup D) = \sup\set{F(\xi): \xi \in D}$$
Since $D = \set{\alpha \in C: F(\alpha) = \alpha}$, we follow that $\xi \in D \ra F(\xi) = \xi$.
Thus
$$\sup\set{F(\xi): \xi \in D} = \sup\set{\xi: \xi \in D} = \sup D = \gamma$$
as desired.

\subsection*{8.2.14}

\textit{Let $(A, \preceq)$ be a poset in which every chain $C \subseteq A$ has an upper
  bound in $A$. Let $n_0 \notin A$. For each chain $C \subseteq A$, let $C_p$ be the set
  of proper upper bounds for $C$. Let
  $D  = \set{C_p: C_p \neq \emptyset \land C \subseteq A \text{ is a chain}}$. By AC,
  there is a function $G$ such that $G(C_p) \in C_p$ for all $C_p \in D$.
  Let $\phi(f, y)$ be a formula that asserts the following:
  $$y =
  \begin{cases}
    G(f[\gamma]_p) \text { if } f: \gamma \to A \land \gamma \in \On \land f[\gamma] \subseteq A
    \text{ is a chain } \land f[\gamma]_p \in D \\
    n_0 \text{ otherwise}
  \end{cases}
  $$
  Hence, for all $f$ there exists a unique $y$ such that $\phi(f, y)$. Theorem 8.2.4 implies that
  there exists a class function $F: \On \to A \cup \set{n_0}$ such that
  $\phi(F|\gamma, F(\gamma))$, for all $\gamma \in \On$. Hence, for each ordinal $\gamma$
  $$F(\gamma) =
  \begin{cases}
    G(F([\gamma]_p) \text{ if } F[\gamma] \subseteq A \text{ is a chain }
    \land F[\gamma]_p \in D \\
    n_0 \text{ otherwise}
  \end{cases}
  $$
}

\textit{(a) Using Lemma 8.2.5, show that $F(\gamma) = n_0$ for some ordinal $\gamma$.}

8.2.5 implies that since $F(\gamma)$ is a class function from a class of ordinals to
a set we can follow that $F$ is not injective.

Given that $F(\gamma)$ is not injective, we follow that there exists
$\alpha, \beta \in \On$ such that $\alpha \neq \beta$ and $F(\alpha) = F(\beta)$.
Suppose that $F(\alpha)  = F(\beta) \neq n_0$. 

Since $\alpha, \beta \in \On$ are such that $\alpha \neq \beta$, we follow that
$\alpha \in \beta$ or $\beta \in \alpha$. Assume the former.
Then we follow that $\alpha \in \beta$. Thus $F(\alpha) \in F[\beta]$. Since $F(\beta) \neq n_0$
we follow that 
$$F(\beta) = G(F[\beta]_p) \in F[\beta]_p$$
Thus $F(\alpha) \in F[\beta]$ and $F(\alpha) \in F[\beta]_p$. Since $F[\beta]$ and $F[\beta]_p$
are disjoint (althought it wasn't proven directly, it follows easily from definition
of a proper upper bound and common sense), we follow that we've got a contradiction.

\textit{(b) Let $\alpha$ be the least ordinal such that $F(\alpha) = n_0$. Show that
  $F[\alpha] \subseteq A$ is a chain with no proper upper bound.}

Let $x, y \in F[\alpha]$. We follow that there exist $\mu, \nu \in \alpha$ such that
$x = F(\mu) \land y = F(\nu)$. Since $\mu, \nu \in \alpha$ and $\alpha$
is the lowest ordinal such that $F(\alpha) = n_0$, we follow that
$F(\mu), F(\nu) \neq n_0$, and thus we follow that $F[\mu], F[\nu]$ are chains.

If $\mu \in \nu$, then we follow that $F(\mu) \in F[\nu]$, and thus $F(\mu) \prec F(\nu)$
by the fact that $F(\nu) \in F[\nu]_p$ which implies that $x \prec y$.
Case when $\nu \in \mu$ follows the same logic.

If $\mu = \nu$, then we follow that $F(\mu) = F(\nu) \ra x = y$
by the fact that $F$ is a class function.
Thus we conclude that $x, y \in F[\alpha] \ra x \preceq y \lor y \preceq x$ and thus $F[\alpha]$
is a chain.

Assume that $F[\alpha]$ has got a proper upper bound. Then we follow that
$F[\alpha]_p \neq \emptyset$ and thus $F[\alpha]_p \in D$. Thus
$F(\alpha) \in \ran(G) \subseteq A$, which contradicts our assumtion that $n_0 \notin A$.

\textit{(c) By assumption, the chain $F[\alpha]$ has an upper bound m. Show that $m$
  is a maximal element.}

Assume that it's not a maximal element. This implies that there exists $b \in A$
such that $m \prec b$. Since $m$ is an upper bound of $F[\alpha]$, we follow that
$(\forall j \in F[\alpha])(j \prec b )$ by properties of strict partial order.
Thus we follow that $b$ is a proper upper bound of $F[\alpha]$, which is a contradiction.


\subsection*{8.2.15}

\textit{Let $A$ be a noempty set and let $n_0 \notin A$. Let
  $C = \pow(A) \setminus \set{\emptyset}$. Note that $\bigcup{C} = A$. By Theorem 3.3.24,
  let $G: C \to A$ be a choice function for $C$. Let $\phi(f, y)$ be a formula that
  asserts the following:
  $$y =
  \begin{cases}
    G(A \setminus f[\gamma]) \text{ if } f: \gamma \to A, \gamma \in \On,
    A \setminus f[\gamma] \neq \emptyset \\
    n_0 \text{ otherwise}
  \end{cases}
  $$
  So for all $f$, there exists a unique $y$ such that $\phi(f, y)$. Theorem 8.2.4
  implies that there exists a class function $F: \On \to A \cup \set{n_0}$ such that
  $\phi(F | \gamma, F(\gamma))$ for all $\gamma \in \On$. Thus
  $$F(\gamma) =
  \begin{cases}
    G(A \setminus F[\gamma]) \text { if } F[\gamma \subseteq A \land A \setminus F[\gamma] \neq
    \emptyset \\
    n_0 \text{ otherwise}
  \end{cases}
  $$
}

\textit{(a) Using Lemma 8.2.5, show that $F(\gamma) = n_0$ for sime ordinal $\gamma$. }

8.2.5 implies that $F$ is not injective.

Assume that $\alpha, \beta \in \On$ are such that $\alpha \neq \beta$. Then we follow by
trichotomy of ordinals that $\alpha \in \beta$ or $\beta \in \alpha$.
Assume the former. Then we follow that $F(\alpha) \in F[\beta]$.
Therefore we follow that $F(\alpha) \notin A \setminus F[\beta]$.

If we assume that there is no $\gamma$ such that $F(\gamma) = n_0$, then we can follow that
$\alpha \neq \beta \ra F(\beta) \neq F(\alpha)$, thus making $F$ injective, which is a
contradiction. THerefore we conclude that there exists $\gamma \in \On$ such that
$F(\gamma) = n_0$.

\textit{(b) Let $\gamma$ be the least ordinal such that $F(\gamma) = n_0$. Show that the set
function $(F|\gamma): \gamma \to A$ is a bijection.}

We can follow that $\alpha \in \gamma \ra F(\alpha) \neq n_0$, therefore we follow that
$(F|\gamma)$ is injective by the stuff that we've proven earlier.

Let $a \in A$. Suppose that $a \notin \ran(F|\gamma)$. Then we follow that
$A \setminus F[\gamma]$ is a nonempty set and also $F[\gamma] \subseteq A$ by the
fact that $(F|\gamma): \gamma \to A$. Thus we follow that $F(\gamma) \neq n_0$,
which is a contradiction. Thus we follow that
$$(\forall a \in A)(\exists \alpha \in \gamma)(F(\alpha) = a)$$
therefore $F|\gamma$ is surjective. 
Taking into account previous paragraph we conclude that $F|\gamma$ a bijection, as desired.

\textit{(c) Show that $A$ has a well-ordering}

Define $\preceq \subseteq A \times A$  by
$$(\forall a, b \in A)(a \preceq b \lra F\inv(a) \ineq F\inv(b))$$
We follow that since $\ineq$ is a well-ordering on $\gamma$ that $\preceq$ is a well-ordering
on $A$, as desired.

\section{Ordinal Arithmetic}

\subsection*{8.3.1}

\textit{Let $\alpha$ be an ordinal and let $\phi(\alpha, f, y)$ be a formula that expresses
  8.14. Let $A$ be an ordinal class function satisfying $\phi(\alpha, A|\beta, A(\beta))$
  for all $\beta \in \On$. Prove, by transfinite induction that $A(\beta) \in \On$ for
  all $\beta \in \On$.}

Our transfinite IH is that
$(\forall \gamma \in beta)(A(\gamma) \in \gamma)$

If $\beta = \emptyset$, then we follow that $A(\emptyset) = \alpha \in \On$

If $\beta$ is a limit ordinal, then we follow that $\beta = \gamma^+$ for some
$\gamma \in \On$. Thus $A(\beta) = A(\gamma)^+$. Since $\gamma \in \beta$, we follow
by IH that $A(\gamma) \in \On$, and thus $A(\beta) = A(\gamma)^+ \in \On$.

If $\beta$ is a limit ordinal, then $A(\beta) = \sup A[\beta]$.  By IH we've
got that $A[\beta]$ is a set of ordinals, thus $A(\beta)$ is a supremum of a set of
ordinals, and thus itself is an ordinal.

Thus by transfinite induction we follow that $A(\beta) \in \On$ for any $\beta \in \On$,
as desired.

\subsection*{8.3.2}

\textit{Show that $\omega \in \omega + 1$ and $2 * \omega \neq \omega * 2$.}

We follow that $\omega + 1 = (\omega + 0)^+ = \omega^+$, thus
we follow that $\omega \in \omega + 1$ by definition of succcessor.

We've proven somewhere already that $\omega$ is a limit ordinal. Thus
$$ 2 * \omega = \sup\set{2 * n: n \in \omega}$$
by basic arithmetic we follow that $n \ineq 2 * n$, thus
$$(\forall m \in \omega)(\exists j \in \set{2 * n: n \in \omega})(m \in j)$$
thus we follow that $\set{2 * n: n \in \omega}$ is cofinite in $\omega$, and thus
we've got that
$$2 * \omega = \sup\set{2 * n: n \in \omega} = \sup \omega = \omega$$
since $\omega$ is a limit ordinal.

We've got that
$$\omega * 2 = (\omega * 1) + \omega = (\omega * 0) + \omega + \omega = \omega + \omega =
\sup\set{\omega + n: n \in \omega}$$
we've got that $\omega + 1 = \omega^+ \in \set{\omega + n: n \in \omega}$,
and since $\omega \in \omega^+$, we follow that $\omega \in \sup\set{\omega + n: n \in \omega}$,
therefore $\omega \neq \sup\set{\omega + n: n \in \omega}$ and
thus $2 * \omega \neq \omega * 2$, as desired.

\subsection*{8.3.3}

\textit{Show that $(\omega + 1) * 2 \neq \omega * 2 + 1 * 2$. and
  $(2 * 2)^\omega \neq 2^\omega * 2^\omega$}

We follow that
$$(\omega + 1) * 2 = \omega * 2$$
and
$$\omega * 2 + 1 * 2 = \omega * 2 + 2 = ((\omega * 2)^+)^+$$
Since no ordinal is equal to its successor, we follow that
$(\omega + 1) * 2 \neq \omega * 2 + 1 * 2$, as desired.

We can follow that
$$(2 * 2)^\omega = 4^\omega = \sup\set{4^n: n \in \omega}$$
we can prove that $\set{4^n: n \in \omega}$ is cofinal in $\omega$, thus
$(2 * 2)^\omega = \omega$.

We can follow the same logic to show $n^\omega = \omega$ for all $n \in \omega$.
Thus we follow that $2^\omega * 2^\omega = \omega * \omega \neq \omega$, as proven
before.

\subsection*{8.3.4}

\textit{Show that $2 + \omega \notin 3 + \omega$, $2 * \omega \notin 3 * \omega$,
  and $2^\omega \notin 3^\omega$. }

Assume that $n \in \omega$ is such that $n \neq 0$. We follow that
$$n + \omega = \sup\set{n + m: m \in \omega}$$
$$n * \omega = \sup\set{n * m: m \in \omega}$$
$$n^\omega = \sup\set{n^m: m \in \omega}$$
we can follow easily that all of the sets under supremum are cofinal in $\omega$,
and thus we follow that $n + \omega = n * \omega = n^\omega = \omega$ for all
$n \in \omega \setminus \set{0}$. Since $\omega \notin \omega$, we follow the result
in the text of the exercise.

\subsection*{8.3.5}

\textit{Show that $\omega^2 = \sup{\omega * n: n \in \omega}$}

We follow that
$$\omega^2 = (\omega^1) * \omega = \omega^0 * \omega * \omega = \omega * \omega =
\sup\set{\omega * n: n \in \omega}$$
where we follow everything above from respective definitions and the fact that
$\omega$ is a limit supremum.

\subsection*{8.3.6}

\textit{Let $\alpha$ be an ordinal such that $1 + \alpha = \alpha$. }

I don't see how can we proceed with the completion of this exercise without
using Theorem 8.3.6, therefore I assume for now, despite the fact that the thing wasn't
proven fully, that implications of that theorem are given.

\textit{(a) Prove that $1 + \alpha * (1 + n) = \alpha * (1 + n)$ for all $n \in \omega$.}

We've got that
$$1 + \alpha * (1 + n) = 1 + (\alpha * 1 + \alpha * n) = 1 + \alpha * 1 + \alpha * n = $$
$$ = 
1 + \alpha + \alpha * n = \alpha + \alpha * n = \alpha * 1 + \alpha * n = \alpha * (1 + n)$$

\textit{(b) Show that $1 + \alpha * \omega = \alpha * \omega$}

From our discussion in the chapter we follow that $\omega = 1 + \omega$, and thus 
$$\alpha * \omega = \alpha * (1 + \omega) = \alpha * 1 + \alpha * \omega =
\alpha + \alpha * \omega$$
From this we follow that 
$$1 + \alpha * \omega = 1 + \alpha + \alpha * \omega = \alpha + \alpha * \omega$$
thus we conclude that
$$1 + \alpha * \omega = \alpha + \alpha * \omega = \alpha * \omega$$
as desired.

\textit{(c) Show that $1 + \omega^{n + 1} = \omega^{n + 1}$ for each $n \in \omega$.}

Let
$$I = \set{n \in \omega: 1 + \omega^{n + 1} = \omega^{n + 1}}$$
discussion in the chapter implies that $0 \in I$. Assume that $n \in I$. Then we follow that
$$1 + \omega{n + 2} = 1 + \omega^{n  + 1} * \omega$$
previous point of this exercise implies that if $alpha = 1 + \alpha$, then
$1 + \alpha * \omega = \alpha$. IH states that $\omega^{n + 1} = 1 + \omega^{n + 1}$.
Thus we follow that
$$1 + \omega^{n + 1}  * \omega = \omega^n * \omega = \omega^{n + 1}$$
therefore we've got that $0 \in  I \land n \in I \to n^+ \in I$, thus $I = \omega$.

\textit{(d) Let $i \in \omega$ and $n \in \omega$. Show that
  $\omega^i + \omega^{i + (n + 1)} = \omega^{i + n + 1}$}

$$\omega^{i + n + 1} = \omega^i * \omega^{n + 1} = \omega^i * (1 + \omega^{n + 1}) =
\omega^i + \omega^i \omega^{n + 1} = \omega^i +  \omega^{i + n + 1}$$
Pretty sure that every equality is justified somewhere.

\subsection*{8.3.7}

\textit{Prove that for all ordinals $\alpha$, if $\omega \ineq \alpha$,
  then $1 + \alpha = \alpha$}

Let our transfinite IH be
$$(\forall \beta \in \alpha)(\omega \ineq \beta \ra \beta = 1 + \beta)$$

We follow that $\emptyset$ is vacuously true.

If $\alpha$ is a successor ordinal, then we've got two cases: $\alpha = \gamma^+$ such that
$\omega \ineq \gamma$, or $\alpha \in \omega$. Let us handle the former case first:  
$$1 + \alpha = 1 + \gamma^+ = (1 + \gamma)^+ = \gamma^+ = \alpha$$
The latter case is also true, since $\beta \in \alpha \ra \omega \not \ineq \beta$,
thus evaluating conditional inside to a true value.

If $\alpha$ is a limit ordinal, then we follow that $\omega \ineq \alpha$ by
some theorem in previous section. Thus we've got that 
$$1 + \alpha = \sup\set{1 + \beta: \beta \in \alpha} = \sup\set{\beta: \beta \in \alpha} =
\alpha$$

Thus we can conclude by transfinite induction that $\beta \in \On \ra \omega \ineq \beta \to
\beta = 1 + \beta$, as desired.

\subsection*{8.3.8}

\textit{Suppose $\omega \ineq \alpha$. Prove for all ordinals $\beta$, if
  $\alpha^2 \ineq \beta$, then $\alpha + \beta = \beta$.}

Let the conclusion of the exercise be tranfinite IH.
We follow that this is true for $\emptyset$.

Assume that $\beta = \gamma^+$ is a successor ordinal. Assume that $\alpha^2 \ineq \beta$.
If $\alpha^2 \in \beta$, then
$$\alpha^2 \in \beta$$
$$\alpha^2 \in \gamma^+$$
$$\alpha^2 \ineq \gamma$$
Thus we follow that $\alpha + \gamma = \gamma$, and thus $\beta = (\alpha + \gamma)^+ =
\alpha + \gamma + 1 = \alpha + \beta$.
Since $\omega \ineq \alpha$, we follow that $1 + \alpha = \alpha$, and thus
$$\beta = \alpha^2 = \alpha * \alpha = \alpha * (1 + \alpha) =
\alpha + \alpha^2  = \alpha + \beta$$

If $\beta$ is a successor ordinal, then
$$\alpha + \beta = \sup\set{\alpha + \delta: \delta \in \beta} =
\sup\set{\delta: \delta \in \beta} = \beta$$
as desired.

\subsection*{8.3.[9, 10, 11]}

\textit{Prove Lemma 8.3.2, 8.3.4 and 8.3.9}

Lemma 8.3.2 states that the class function $A_\alpha: \On \to \On$ is normal.

Let $\beta$ be a limit ordinal. We follow that
$$A_\alpha(\beta) = \sup A_\alpha[\beta]$$
by definition, therefore it is continous.

Let $\gamma \in \On$. We follow that $\gamma^+$ is a successor ordinal and thus
$$A_\alpha(\gamma^+) = A_\alpha(\gamma)^+$$
by definition of the function. Therefore we can follow that $\gamma \in \On \ra
A_\alpha(\gamma) \in A_\alpha(\gamma^+)$, and therefore it is strictly increasing
by one of the lemmas in the previous chapter. Therefore we conclude that
$A_\alpha$ is continous and strictly increasing, and thus it is a normal function,
as desired.

Lemma 8.3.4 states that if $\alpha$ is a nonzero ordinal, then $M_\alpha$ is normal.

Assume that $\alpha$ is a nonzero ordinal.

If $\beta$ is a limit ordinal, then
$$M_\alpha = \sup{M_\alpha[\beta]}$$
by definition, therefore it is continous. Since $\alpha$ is nonzero, we follow that
$\emptyset \in \alpha$, and thus we've got that for $\gamma \in \On$
$$M_\alpha(\gamma^+) = M_\alpha(\gamma) + \alpha$$
Since $A_\delta$ is a strictly increasing function for any $\delta \in \On$, we follow that
$A_{M_\alpha(\gamma)}$ is also an increasing function, thus $\emptyset \in \alpha$
implies that
$$A_{M_\alpha(\gamma)}(\emptyset) \in A_{M_\alpha(\gamma)}(\alpha)$$
thus
$$M_\alpha(\gamma) = M_\alpha(\gamma) + \emptyset \in
M_\alpha(\gamma) + \alpha = M_\alpha(\gamma^+)$$
thus we follow that $\gamma \in \On \ra M_\alpha(\gamma) \in M_\alpha(\gamma^+)$, and
thus $M_\alpha$ is a strictly increasing function for any $\alpha \neq \emptyset$.
Thus we conclude that for $\alpha \neq \emptyset$  $M_\alpha$ is normal, as desired.

Lemma 8.3.9 states that for $2 \ineq \alpha$ we've got that $E_\alpha$ is normal.

We once again follow that if $\beta$ is a limit ordinal, then
$E_\alpha(\beta) = \sup{E_\alpha[\beta]}$, therefore making it continous.

If $2 \ineq \alpha$, then we follow that for $\gamma \in \On$ we've got that
$$E_\alpha(\gamma^+) = E_\alpha(\gamma) * \alpha = M_{E_\alpha(\gamma)}(\alpha)$$
We also have that 
$$E_\alpha(\gamma) = M_{E_\alpha(\gamma)}(1)$$
Since $2 \ineq \alpha$, we follow that $1 \in \alpha$, thus if $E_\alpha(\gamma) \neq \emptyset$,
then we follow that $E_\alpha$ is strictly increasing by the fact that $M$ is strictly increasing.
We can also follow that if $2 \ineq \alpha$, then $\emptyset \in E_\alpha(\gamma)$, which
means that we're done.

\subsection*{8.3.12}

\textit{For every ordinal $\alpha$, prove that there exists an ordinal $\beta \neq 0$ such that
  $\alpha + \beta = \beta$.}

We've got two cases here: $\alpha \in \omega$ or $\omega \ineq \alpha$. If former is the
case, then we can set $\beta = \omega$ and we'll get the desired result (easily follows
from the fact that $1 + \omega = \omega$ and induction; that case was proven before
by me earlier in this chapter's exercise, but scrapped afterwards for some reason).
If $\omega \ineq \alpha$, then we follow that setting $\beta = \alpha^2$ will get us
the desired result by exercise 8 in this chapter.

\subsection*{8.3.13}

\textit{Let $\alpha$ be a nonzero ordinal. Prove that $\beta + \alpha \neq \beta$ for
  all ordinals $\beta$}

The fact that $\alpha$ is nonzero implies that $\emptyset \in \alpha$.
Lemma 8.3.2 implies that $A_\beta$ is strictly increasing, and thus 
$$\beta = A_\beta(0) \in A_\beta(\alpha) = \beta + \alpha$$
thus $\beta \in \beta + \alpha$ implies the desired result.

\subsection*{8.3.14}

\textit{Prove Lemma 8.3.5(b)}

That lemma states that class function $G: \On \to \On$ defined by
$G(\xi) = \alpha * \beta + \alpha * \xi$ is normal for any $\alpha, \beta \in \On$.
There's also a typo in the text of the exercise, there should be an assumption that
$\alpha \neq 0$.

Assuming that we follow that we an follow that $G(\xi) = A_{\alpha * \beta}(M_\alpha(\xi))$,
which is a composition of normal functions and therefore it is itself normal, as desired.

\subsection*{8.3.[15, 16]}

\textit{Prove Lemma 8.3.6}

Firstly, let us prove that
$\alpha + 0 = 0 + \alpha = \alpha$. 
We've got that $\alpha + 0 = \alpha$ by definition.

Let our transfinite IH be that
$$(\forall \mu \in \alpha)(0 + \mu = \mu )$$
we follow that $0 + 0 = 0$ by definition. If $\alpha = \gamma^+$, then
$$0 + \alpha = 0 + \gamma^+ = (0 + \gamma)^+ = \gamma^+ = \alpha$$
where we've got third equality from IH. If $\alpha$ is a limit ordinal, then
$$0 + \alpha = \sup\set{0 + \beta: \beta \in \alpha} = \sup\set{\beta : \beta \in \alpha} =
\sup \alpha = \alpha$$
thus we follow that $\alpha \in \On \ra 0 + \alpha = \alpha$. Therefore
we've got the desired result.

Proof of the  statements
$$\alpha * 1 = 1 * \alpha = 0$$
and
$$\alpha * 0 = 0 * \alpha = 0$$
have the same logic as the previous one, so they are skipped.

Thus let us do
$$(\alpha + \beta) + \gamma = \alpha + (\beta + \gamma)$$
We're going to do this by transfinite induction.
Our transfinite IH is that
$$(\forall \xi \in \gamma)((\alpha + \beta) + \xi = \alpha + (\beta + \xi))$$
It also can be expressed as
$$(\forall \in \in \gamma)(A_{A_\alpha(\beta)} (\xi) = A_\alpha (A_\beta(\xi)))$$

If $\gamma = 0$, then
$$(\alpha + \beta) + \gamma = \alpha + \beta = \alpha + (\beta + \gamma)$$
by definitions and later results.
If $\gamma = \delta^+$ for some $\delta \in \On$, then
$$(\alpha + \beta) + \gamma = (\alpha + \beta) + \delta^+ = 
((\alpha + \beta) + \delta)^+ = (\alpha + (\beta + \delta))^+ = \alpha + (\beta + \delta)^+ =
\alpha + (\beta + \gamma)$$
where we follow everything from definitions and IH.


We can follow that since $A_\alpha$ and $A_\beta$ are normal functions that
their commposition is a normal function as well. Therefore we've got that
$$\alpha + (\beta + \gamma) = A_\alpha(A_\beta( \gamma)) =
A_\alpha(A_\beta( \sup \gamma)) = \sup  \set{A_\alpha(A_\beta(\xi)): \xi \in \gamma} =$$
$$ = 
\sup  \set{A_{A_\alpha(\beta)} (\xi): \xi \in \gamma} = A_{A_\alpha(\beta)} (\sup \gamma) =
A_{A_\alpha(\beta)} (\gamma) = (\alpha + \beta) + \gamma$$
therefore we've got that $(\alpha + \beta) + \gamma = \alpha + (\beta + \gamma)$
by transfinite induction.

For 
$$(\alpha * \beta) * \gamma = \alpha * (\beta * \gamma)$$
we've got that if $\alpha = 0$ or $\beta = 0$, then
$\alpha * (\beta * \gamma) = (\alpha * \beta) * \gamma = 0$
by the 6th point of this theorem. Thus let us assume that $\alpha, \beta \neq \emptyset$.
This point is needed because we need to assume that $M_\alpha$ and $M_\beta$ are normal.
Then we can follow the same logic as in proof of associativity of addition in order to get
the desired result.

\subsection*{8.3.17}

\textit{Prove theorem 8.3.7(1)}

Theorem 8.3.7(1) states that for $\alpha, \beta, \gamma \in \On$
$$\beta \in \gamma \lra \alpha + \beta \in \alpha + \gamma$$
Forward direction is followes from the fact thta $A_\alpha$ is a strictly increasing function.

Assume that $\gamma \ineq \beta$. Then we follow that
$\alpha + \gamma \ineq \alpha + \beta$ by the facts that $A_\alpha$ is a function in case
of equality, and by the fact that $A_\alpha$ is strictly increasing in case if $\gamma \in \beta$.
Thus we follow that the desired result by contrapositive.

\subsection*{8.3.18}

\textit{Prove the following: }

\textit{(a) Let $\alpha, \gamma \in \On$. If $\gamma \ineq \alpha \in \gamma + \omega$,
ten $\alpha = \gamma + n$ for some $n \in \omega$. }

If $\gamma = \alpha$, then we follow that $\alpha = \gamma + \emptyset$.

If $\gamma \in \alpha$, then we follow that
$$\alpha \in \gamma + \omega \lra \alpha \in \sup{\gamma + n: n \in \omega} \lra
(\exists m \in \omega)(\alpha \in \gamma + m)$$
Since $\gamma \neq \alpha$, we follow that
$$(\exists m \in \omega \setminus \set{0})(\alpha \in \gamma + m)$$
thus
$$(\exists j^+ \in \omega )(\alpha \in \gamma + j^+)$$
Let $k$ be the $\ineq$-lowest ordinal such that $\alpha \in \gamma + k^+$.
Assume that $\alpha \neq \gamma + k$. Then we follow that $\alpha \in \gamma + k$
or $\gamma + k \in \alpha$. Former implies that $k^+$ is not lowest number such that
$\alpha \in \gamma + k^+$ and latter implies that $\gamma + k \in \gamma$,
both of which are false. Therefore we follow that there exists $k^+ \in \omega$
such that $\alpha = \gamma + k^+$.

\textit{(b) For all $\zeta \in \On$ we've got that $\zeta + \omega$ is a limit
  ordinal.}

We know that $A_\zeta$ is a normal function for every $\zeta \in \On$, therefore
$\zeta + \omega = A_\zeta(\omega)$ is a limit ordinal by corollary 8.2.16, as desired.

\textit{(c) For all $\zeta \in \On$, if $\zeta \neq 0$, then $\omega * \zeta$ is a limit
  ordinal.}

Let our transfinite IH be
$$(\forall \xi \in \zeta)(\xi \neq 0 \to \omega * \xi \text{ is a limit ordinal})$$

For $\zeta = 0$ we follow that our case is vacuously true.

If $\zeta = \delta^+$, then
$$\omega * \zeta = \omega * \delta^+ = (\omega * \delta) + \omega = A_{(\omega * \delta)}(\omega)$$
thus it is a limit ordinal by the fact that $A_\alpha$ is normal for $\alpha \in \On$.

If $\zeta$ is a limit ordinal, then we follow that since $\omega \neq \emptyset$
we've got that $M_\omega$ is normal, and thus $\omega * \zeta = M_\omega(\zeta)$ is a
limit ordinal as well. Therefore we follow desired result by transfinite induction, as desired.

\textit{(d) If $\alpha$ is a limit ordinal, then $\alpha = \omega * \zeta$ for
  some nonzero ordinal $\zeta$.}

We follow that $\alpha \ineq \omega * \alpha$ by the fact that $\omega \neq \emptyset$,
thus making $M_\omega$ normal and implications of exercise 8.2.9.

Thus we follow that there exists a set
$\set{\nu \in \On \land \alpha \ineq \omega * \nu: \nu \in (\omega * \alpha)^+}$.
We follow that $\omega * \alpha$ is in this set, therefore
making it nonempty.
Since it's a set of
ordinals, we follow that we've got the $\ineq$-lowest element $\nu$ in the bunch.
We also follow that since $\omega * 0 = 0$ and $0 \in \alpha$ that $\nu \neq 0$.
Now assume that $\alpha \neq \omega * \nu$ and thus $\alpha \in \omega * \nu$.

If $\nu$ is a successor ordinal, then $\nu = \delta^+$ for some $\delta \in \On$.
We follow that
$$\omega * \nu = \omega * \delta^+ = \omega * \delta + \omega$$
We follow that $\omega * \delta \ineq \alpha \in \omega * \delta + \omega$.
Point (1) implies that $\alpha = \omega * \delta + n$ for some $n \in \omega$. Since
$\omega * \delta \in \alpha$, we follow that $n \neq 0$, and thus $n = k^+$ for some
$k \in \omega$. Therefore $\alpha = \omega * \delta + k^+ = (\omega * \delta + k)^+$,
which means that $\alpha$ is a successor ordinal, which is a contradiction.

If $\nu$ is a limit ordinal, then we follow that
$$\omega * \nu = \sup\set{\omega * \mu: \mu \in \nu}$$.
we follow that $(\forall \mu \in \nu)(\omega * \mu \in \alpha)$.
If $\epsilon \in \sup\set{\omega * \mu: \mu \in \nu}$, we follow that there
exists $\mu \in \nu$ such that $\epsilon \in \omega * \mu$. This implies that
$\epsilon \in \alpha$. Thus we follow that $\sup\set{\omega * \mu: \mu \in \nu} \subseteq
\alpha$ and thus $\sup\set{\omega * \mu: \mu \in \nu} \ineq \alpha$, which is a
contradiction.

Therefore we follow that our assumption that $\alpha \neq \omega * \nu$ was incorrect and
thus $\alpha = \omega * \nu$, as desired.

\textit{(d) If $\alpha$ is a successor ordinal, then $\alpha = \omega * \zeta + n$
  for some $\zeta \in \On$ and $n \in \omega$.}

We are once again justified to use set
$\set{\nu \in \On \land \alpha \ineq \omega * \nu: \nu \in (\omega * \alpha)^+}$
and take the $\ineq$-lowest element $\nu$ from it. Now we've got three cases:
$\nu = 0$, $\nu$ is a successor ordinal and $\nu$ is a limit ordinal.

If $\nu = 0$, then we follow that $\alpha \ineq \omega * 0 = 0$, thus making
$\alpha = 0$. Therefore there exists $\zeta = 0, n = 0$ such that
$\alpha = \omega * \zeta + n$.

If $\nu$ is a successor ordinal, then we follow that $\nu = \delta^+$, thus
$\omega * \delta \ineq \alpha \in \omega * \delta + \omega$. Thus we follow that
$\alpha = \omega * \delta + n$ for some $n \in \omega$ by point (1), as desired.

If $\nu$ is a limit ordinal, then we follow that $\omega * \nu \ineq \alpha$ by the
in the previous point in this exrcise. Neither
$\omega * \nu = \alpha$ nor $\omega * \nu \in \alpha$ are possible (former by the fact that
$\omega * \nu$ is a limit ordinal and $\alpha$ isn't; latter by definition of $\nu$),
thus we follow that this case is impossible.

Therefore we conclude that if $\alpha$ is an ordinal, then there indeed exist $\zeta \in \On$
and $n \in \omega$ such that $\alpha = \omega * \zeta + n$. Previous point also
implies that this is the case for all $\alpha \in \On$ (just set $n = 0$.)

\textit{(f) There exists an ordinal $\alpha$ such that $\alpha = \omega * \alpha$}

Easy: $0 = \omega * 0$.

There probably should be some sort of restriction added to the text of this exercise,
so that we need to remember that Theorem  8.2.19 exists, that proves the result for
a more general case.

\subsection*{8.3.[19, 20]}

\textit{Prove Theorem 8.3.10}

8.3.10(1) states that for all $\alpha, \beta, \gamma$, if $2 \ineq \alpha$, then 
$$\alpha^{\beta + \gamma} = \alpha^\beta * \alpha^\gamma$$
in other words it can be expressed as
$$E_\alpha(A_\beta(\gamma)) = M_{E_\alpha(\beta)}(E_\alpha(\gamma))$$

Let our transfinite IH be that for all $\xi \in \gamma$
$$E_\alpha(A_\beta(\xi)) = M_{E_\alpha(\beta)}(E_\alpha(\xi))$$

If $\gamma = 0$, then we follow that
$$\alpha^{\beta + \gamma} = \alpha^\beta = \alpha^\beta * 1 = \alpha^\beta * \alpha^\gamma\alpha$$

If $\gamma$ is a successor ordinal, we follow that $\gamma = \delta^+$ for some $\delta \in \On$,
and thus
$$\alpha^{\beta + \gamma} = \alpha^{\beta + \delta^+} =
\alpha^{(\beta + \delta)^+} = \alpha^{\beta + \delta} * \alpha =
\alpha^\beta * \alpha * \delta * \alpha = $$
$$ = \alpha^\beta * \alpha^{\delta^+} = \alpha^\beta * \alpha^\gamma$$
where in order to get the desired  we've used nothing but definitions and rules of associativity
of multiplication.

If $\gamma$ is a supremum ordinal, then we follow that $E_\alpha(\beta) \neq 0$,
and thus both $E_\alpha \circ A_\beta$ and $M_{E_\alpha(\beta)} \circ E_\alpha$ are normal
functions. Therefore we follow that 
$$E_\alpha(A_\beta(\gamma)) = E_\alpha(A_\beta(\sup \gamma)) = \sup E_\alpha \circ A_\beta [\gamma]
\sup M_{E_\alpha(\beta)} \circ E_\alpha[\gamma]
= M_{E_\alpha(\beta)}(E_\alpha(\sup \gamma)) = $$
$$ = M_{E_\alpha(\beta)}(E_\alpha(\gamma))$$
therefore we conclude that the desired result holds by transfinite induction.

For point (2) we want to prove that
$$(\alpha^\beta)^\gamma = \alpha^{\beta * \gamma}$$
we're going to use transfinite induction here as well. We need to expand definitions here a bit
in order to accomodate for $\alpha = 1$ and $\alpha = 0$, since if we set $\beta = 0$, then
$$(\alpha^\beta)^\gamma = 1^\gamma$$
which is undefined.
I don't see the reason on why we
can't state that $E_\alpha$ can be definned on $\alpha \in \set{0, 1}$, so let's do that.
Definitions are preetty straghtforwad: $1^\alpha = 1$ for every $\alpha \in \On$ and
we've got that $0^0 = 1$ and $0^\alpha = 0$ for all nonzero $\alpha \in \On$.

Thus we follow that if $\gamma = 0$, then
$$(\alpha^\beta)^\gamma = 1 = \alpha^0 = \alpha^{\beta * \gamma}$$
If $\gamma = \delta^+$, then
$$(\alpha^\beta)^\gamma = (\alpha^\beta)^{\delta^+} = (\alpha^\beta)^{\delta} \alpha^\beta =
\alpha^{\beta * \delta} * \alpha^\beta = \alpha^{\beta * \delta + \beta} = \alpha^{\beta * \delta^+} =
\alpha^{\beta * \gamma}$$

And if $\gamma$ is a limit ordinal then we follow that we've got two cases: $\beta = 0$
and $\beta \neq 0$. In the former case we've got that
$$(\alpha^\beta)^\gamma = 1^\gamma = 1 = \alpha^0 = \alpha^{\beta * \gamma}$$
and in the latter case we follow the desired result by the fact that both
$E_\alpha \circ E_\beta$ and $E_\alpha \circ M_\beta$ are both normal.
Thus we follow the desired result by transfinite induction.

We can expand point (3) a bit to get
$$\alpha^\beta = \alpha^\gamma \iff \beta = \gamma$$

Assume that $\beta \neq \gamma$. Then we follow that $\beta \in \gamma$ or $\gamma  \in \beta$.
Assume the former. Then we follow that $\alpha^\beta \in \alpha^\gamma$ by the fact that
$E_\alpha$ is strictly increasing. Thus we follow that $\beta \neq \gamma \ra \alpha^\beta
\neq \alpha^\gamma$, which is equivalent to forward direction. Reverse direction is
followed by the fact that $E_\alpha$ is a class function.

This case also encompasses the fouth point of this theorem.

\subsection*{8.3.21}

\textit{(AC) Let $\alpha$ be a countable ordinal. Prove that for all ordinals $\beta$,
  if $\beta$ is a countable ordinal, then $\alpha + \beta$, $\alpha * \beta$
  and $\alpha^\beta$ are countable.}

If $\beta = 0$, then $\alpha * \beta = 0$, $\alpha + \beta = \alpha$ and $\alpha^\beta = 1$,
thus we can follow that all of them are countable.

If $\beta$ is a successor ordinal, then we follow that
$$\alpha + \beta = \alpha + \delta^+ = (\alpha + \delta)^+$$.
We follow that $\alpha + \delta$ is a countable ordinal by IH and we know that
successors of countable ordinals are countable ordinals as well.

We also follow that if $\beta$ is a successor ordinal, then
$$\alpha + \beta = \sup\set{\alpha + \delta: \delta \in \beta} =
\bigcup\set{\alpha + \delta: \delta \in \beta} $$
we follow that this set is countable by the fact that $\alpha + \delta$ is countable
by IH and Theorem 5.2.9. This logic also works for $E_\alpha$ and $M_\alpha$ as well.

Thus we can conclude that if $\alpha, \beta \in \On$ are countable, then $\alpha + \beta$
is also countable.

We've skipped before cases of successor ordinals for $M_\alpha$ and $E_\alpha$,
because case for $M_\alpha$ depends on $A_\alpha$, and $E_\alpha$ depends on $M_\alpha$.
Thus let us proceed. 

If $\beta$ is a successor ordinal, then we follow that $\beta = \delta^+$ for
some $\delta \in \On$. Thus
$$\alpha * \beta = \alpha * \delta^+ = \alpha * \delta + \alpha$$
which is a sum of two countable cardinals and therefore it is countable as well.
Thus we follow that $\alpha * \beta$ is countable.
We also follow that
$$\alpha ^\beta = \alpha^{\delta^+} = \alpha^\delta * \alpha$$
thus we follow that it is a product of two countable ordinals, and therefore it is
countable as well, as desired.

\textit{Last two exercises and last section are skipped for time reasons. Maybe I will
  return to them at some other time.}

\chapter{Cardinals}

\section{Cardinal Numbers}

\subsection*{Notes}

Although it wasn't stated in the book explicitly, I want to add here that
$|A|$ is called cardinality of $A$. That ties up some loose ends which concern the
definition of "have the same cardinality".

I'm a big fan of functions, and therefore want to add some notation, that concerns cardinality.
Let us state that 
$$\card(A) = |A|$$
where $\card$ is a class function from a class of sets to a class of cardinals.

For some reason there is no shorthand for a class of cardinals. Thus I denote it as
$\Crd$ untill I'm presented with a more widely-used notation.

\subsection*{9.1.1}

\textit{Let $\alpha$ be an ordinal. Show that $|\alpha| \ineq \alpha$.}

Let $\beta = |\alpha|$. We follow that $\beta$ is the least ordinal such that there
is a injective function $f: \alpha \to \beta$. 

Assume that $\alpha \in \beta$. Then we follow that $\alpha \subseteq \beta$.
Let $f: \alpha \to \alpha$ be defined by $f(x) = x$ (i.e. it's the identity function).
Since $f$ is injective and $\alpha \in \beta$ we follow that $\beta$ is not the lowest
ordinal such that there is an injective function $f: \alpha \to \beta$, which is a contradiction.

By trichotomy of ordinals we conclude that $\beta \ineq \alpha$,
and thus $|\alpha| \ineq \alpha$, as desired.

\subsection*{9.1.2}

\textit{Let $\kappa$ be a cardinal. Prove that $|\kappa| = \kappa$.}

From previous exercise we know that $|\kappa| \ineq \kappa$. Thus it will suffice to show that
$|\kappa| \in \kappa$ is impossible.

Assume that $|\kappa| \in \kappa$. Let $\beta = |\kappa|$. Definition of cardinality
implies that there exists an injective function $f: \kappa \to \beta$. Thus
we've got $\beta \in \kappa$ and an injective function $f: \kappa \to \beta$,
which contradicts the definition of a cardinal.

We can also follow the converse (i.e. $|\kappa| = \kappa \ra \kappa \in Crd$)
by the fact that codomain of cardinality class function
if a class of cardinals.

\subsection*{9.1.3}

\textit{Prove that $|\omega^+| = \omega$.}

We can define a function $f: \omega^+ \to \omega$ by
$$f(x) =
\begin{cases}
  x = \omega \to 0 \\
  x \in \omega \to x^+
\end{cases}
$$
We can follow that this function is injective by common sense. Assume that there
exists $\alpha \in \omega$ such that $f: \omega^+ \to \alpha$ is injective. Then we'll
follow that $\omega$ is finite, which is a contradiction. Therefore we conclude that
$|\omega^+| = \omega$, as desired.

\subsection*{9.1.4}

\textit{Prove that the class $\set{k: k \text{ is a cardinal}}$ is not a set.}

Let  us denote $C = \set{k: k \text{ is a cardinal}}$. We want to show that this thing is a
proper class.

We follow that if $C$ is a set, then it is a set of ordinals. Thus
$\sup C$ is a supremum ordinal of $C$. Therefore by Hartogs' Theorem and the lemma
that follows it, there exists
a cardinal $\kappa$ that is greater than $\sup C$. Therefore for all $\alpha \in C$
we've got that $\alpha \in \kappa$. Therefore
$$(\forall \sigma \in C)(\kappa \neq \sigma)$$
which gives us that $\kappa \notin C$, and therefore it is not a cardinal,
which is a contradiction.

\subsection*{9.1.5}

\textit{Prove Theorem 9.1.10}

Theorem 9.1.10 states that if $C$ is a  set of cardinals, then
$\sup C$ is a cardinal and if $C$ is a limit set, then $\delta \in \sup C$ for all $\delta \in C$.

Let $C$ be an arbitrary set of cardinals.

We follow that if $C = \emptyset$, then
$\sup C = \bigcup C = \emptyset = 0$, thus the former case holds. Latter case requires $C$ to
be a limit set, and a limit set is nonempty by definition, thus that implication holds vacuously
as well.

Now assume that $C$ is nonempty. Let us denote $\sup C$ by $\kappa$. Assume that $\kappa$
is not a cardinal and thus there is an injection $f$ from $\kappa$ to some $\beta \in \kappa$.
Since $\beta \in \kappa$, we follow that $\beta \in \bigcup C$ and thus
$$(\exists \sigma  \in C)(\beta \in \sigma)$$
We follow that since $\sigma \in C$ that $\sigma \ineq \kappa$ by the fact that $\kappa$
is a supremum of $C$. Since $\sigma \in C$ we follow that $\sigma$ is a cardinal,
and thus there are no injections from $\sigma \to \beta$. But since $\sigma \in \kappa \ra
\sigma \subseteq \kappa$ and there is an injection $f$  from $\kappa \to \beta$, we conclude that
there is an injection $f|\sigma: \sigma \to \beta$, which is a contradiction. Thus we've got
the first case.

Now assume that $C$ is nonempty and a limit set. We follow that
$\delta \in C \ra \delta \ineq \sup{C}$ by the properties of supremum. Thus we need
to show that $\delta \neq \sup C$. Assume that $\delta = \sup C$. Since $C$ is a limit
set, we follow that there exists $\gamma \in C$ such that $\delta \in \gamma$.
Thus we follow that $\sup C \ineq \gamma$ and $\gamma \in C$, which contradicts the
fact that $\sup C$ is an upper bound. Therefore we follow that our assumption that
$\delta = \sup C$ was false, and thus we've got that $\delta \in C \ra \delta \in \sup{C}$,
as desired. Don't want to spend my time further here, but maybe latter implication holds in
general for sets of ordinals.

\subsection*{9.1.6}

\textit{let $A$ be a set and $\alpha$ be an ordinal. Suppose that $f: A \to \alpha$ is
  injective. Define a relation $\preceq$ on $A$ by $x \preceq y$ if and only if $f(x) \ineq f(y)$
  for all $x, y \in A$. Prove that $\preceq$ is a well-ordering on $A$. If $f: A \to \alpha$
  is a bijection, then preove that $\alpha$ is the order type of $(A, \preceq)$.}

We follow that $\ran(f) \subseteq \alpha$. Thus we follow that restricted $\preceq$ is a
well-ordering on $\ran(f)$. We follow that if we restrict codomain of $f$ to
get a bijection $f': A \to \ran(f)$, then we follow that $\preceq$ is
indeed a well-ordering by some exercise/lemma that was proven a couple of chapters ago.

If $f$ is bijective, then we follow that $(A, \preceq)$ and $(\alpha, \ineq)$ are isomorphic
and therefore $\alpha$ is an order type of $A$ by one of the lemmas as well.\

\subsection*{9.1.7}

\textit{Let $\alpha$ be an ordinal. Suppose $g: \alpha \to X$ is a function.
  Let $A \subseteq \alpha$. Prove that $g[A]$ has a well-ordering and that $|g[A]| \ineq |A|$. }

Any set has a well-ordering if we assume AC, therefore I think that we need not to use
it in this exercise.

Let $x \in g[A]$. We follow that $g\inv[\set{x}] \subseteq \alpha$ and thus
there is a unique $\ineq$-least element of $g\inv[\set{x}]$. Thus we can define a function from
$f: g[A] \to \alpha$ by this construction.
Since for all $x, y \in g[A]$ such that $x \neq y$ we follow that
$g\inv[\set{x}] \cap  g\inv[\set{y}]$, we can follow that the function is injective. By previous
exercise we conclude that there is a well-order $\preceq$ that can be defined on $g[A]$.

We can note that $f$ can be thought as a funtion from $g[A]$ to $A$ instead of generic $\alpha$.
Thus we follow that there is an injection from $g[A]$ to $A$ and thus $|g[A]| \leq_c |A|$.
Exercise 11 and Lemma 9.1.7  do not depend on this exercise, thus I think that we're justified
to use taht fact that $|g[A]| \leq_c |A| \iff |g[A]| \ineq |A|$.

\subsection*{9.1.8}

\textit{Assume that $A$ has a well-ordering and let $B \subseteq A$. Prove that $|B| \ineq |A|$
  using Definition 9.1.5 }

Let $\alpha = |A|$. We follow that there exists an injection $f: A \to \alpha$ and $\alpha$ is the
least such ordinal. We follow that since $B \subseteq A$ that $f|B: B \to \alpha$
is also an injection. Therefore we follow that if $|B| \neq \alpha$ that $|B| \in \alpha$.
Thus we conclude that $|B| \ineq |A|$, as desired.

\subsection*{9.1.9}

\textit{Prove that if $A$ has a well-ordering and $\kappa = |A|$, then $\kappa$ is a
  cardinal.}

Assume that $\alpha = |A|$ is not a cardinal. By definition of cardinality we follow that
there exists an injection $g: A \to \alpha$ and $\alpha$ is the lowest ordinal such that
an injection exists. 
Since $\alpha$ is not a cardinal, we follow that there exists $\beta \in \alpha$
such that there is an injection $f: \alpha \to \beta$. Thus we follow that
$f \circ g: A \to \beta$ is an injection. Therefore we conclude that $|A| \neq \alpha$,
which is a contradiction.

Just a quick point: solution of this exercise depends only on definitions, therefore
our initial discussion about the class functions and whatnot is valid.
Thus proof of Lemma 9.1.6 is dependent on the stuff in the previous chapter and definitions
in this section.

\subsection*{9.1.10}

\textit{Let $\kappa$ be a cardinal. Prove that if $f: A \to \kappa$ is a bijection,
  then $\kappa = |A|$.}

Assume that $|A| \neq \kappa$. Since $f$ is a bijection and therefore
injection, we follow that $|A| \in \kappa$. Thus we follow that there is an injection
$g: A \to |A|$. Thus we follow that there is an injection $g \circ f\inv: \kappa \to |A|$.
Since $|A| \in \kappa$, we conclude that $\kappa$ is not a cardinal, which is
a contradiction.

\subsection*{9.1.11}

\textit{Prove Lemma 9.1.7}

Lemma 9.1.7 states that
$$|A| \leq_c |B| \lra |A| \ineq |B|$$
$$|A| =_c |B| \lra |A| = |B|$$
$$|A| <_c |B| \lra |A| \in |B|$$

If $|A| \ineq |B|$, then we follow that there are bijections $f_1 : A \to |A|$, $f_2 B \to |B|$,
and since $|A| \ineq |B|$, we follow that $|A| \subseteq |B|$, thus we follow that there is
an injection $ f_2\inv \circ f_1$. Thus we follow that $|A| \leq_c |B|$. Reverse point and
the rest is pretty trivial and depends exclusively on composition of functions and Lemma 9.1.6.

\subsection*{9.1.12}

\textit{Show that the collection $W$ in the proof of Hartogs' Theorem is a set.}

We can follow that since $X$ is a set that $\pow(X)$ is a set. We also follow that $X \times X$
is a set, thus $\pow(X \times X)$ is a set of all relations on $X$. Thus we follow that
$\pow(X) \times \pow(X \times X)$ is the set of all pairs between all subsets of $X$ and
all relations on $X$, and thus $W$ is a subset of this set.

\subsection*{9.1.13}

\textit{Prove Theorem 9.1.13}

9.1.13 states that if $\gamma$ is a limit ordinal, then $cf(\gamma)$ is a cardinal.
Assume that $cf(\gamma)$ is not a cardinal. Let us denote $\kappa = cf(\gamma)$.
Firstly let us state that $\kappa \neq 0$ by the fact that if it is, then $f: \kappa \to \gamma$
is an empty function, thus $f[\kappa] = \emptyset$, and $\emptyset$ is not cofinal in any
limit ordinal by definition of cofinality and common sense.
Since $\kappa$ is not a cardinal, we follow that there is $\beta \in \kappa$
and $f: \kappa \to \beta$ such that $f$ is injective. Since $f$ is injective, we follow that
we can restrict its codomain to get the bijective function $f': \kappa \to \ran(f)$, where
we note that $\ran(f) \subseteq \beta$. Since $f'$ is a bijection, we follow that
$f'\inv: \ran(f) \to \kappa$ is also a bijection. Thus we can define a function
$g: \beta \to \kappa$ by
$$g(x) =
\begin{cases}
  x \in \ran(f) \to f'\inv(x) \\
  x \notin \ran(f) \to \emptyset \\
\end{cases}
$$
and since $f'\inv: \ran(f) \to  \kappa$ was a bijection, we follow that it was sujective,
thus we follow that a function that is constructed by expanding domain of $f'\inv$
is also surjective. 

Since $\kappa = cf(\gamma)$, we follow that there is a function $h: \kappa \to \gamma$
such that $h[\kappa]$ is cofinal in $\gamma$. We can follow that $h[\kappa] = (h \circ g)[\beta]$
by the fact that $g$ is surjective, and thus we follow that there is an oridnal $\beta \in \kappa$
for which there exists a function $(h \circ g)$ such that $(h \circ g)[\beta] = h[\kappa]$ is
cofinal in $\gamma$. Therefore we conclude that $\kappa$ is not the lowest such ordinal,
and thus $\kappa \neq cf(\gamma)$, which is a contradiction.

We don't explicitly use here the fact that $\gamma$ is a limit cardinal, but we use it
implicitly by the fact that $cf$ is defined only on limit ordinal. 

\subsection*{9.1.14}

\textit{Prove Theorem 9.1.15}

9.1.15 states that if $\gamma$ is a limit ordinal, then $cf(\gamma)$ is a regular cardinal.
Translating this lemma to a more functional language, we need to prove that $cf(cf(\gamma)) =
cf(\gamma)$.

Firstly we've got to state some heavily implied things in writing.
Let $\kappa$ be an infinite cardinal. We follow that we
can take an identity function, and since $\kappa$ is cofinal in $\kappa$ we follow that
$cf(\kappa) \ineq \kappa$. Thus by trichotomy of $\ineq$ in ordinals we follow that any
given infinite cardinal is either regular or singular (and not both).

% Suppose that $cf(\gamma)$ is not regular. We follow that there is $\alpha \in cf(\gamma)$
% and $f: \alpha \to cf(\gamma)$ such that there is no $\theta \in cf(\gamma)$ such that
% $\ran(f) \subseteq \theta$. We follow that for  all $\mu \in cf(\gamma)$ there
% exists $\beta \in \ran(f)$ and by extension $\nu \in \alpha$ such that $\beta = f(\nu)$
% and $\mu \in f(\nu)$. This implies that $\ran(f) = f[\alpha]$ is cofinal in $\gamma$ and thus 

Now let's get back to our proof. Let us denote $\beta = cf(\gamma)$.
Assume that $\beta$ is not a regular cardinal. We follow that $\beta$ is singular and thus
$cf(\beta) \in \beta$. Let $\kappa = cf(\beta)$. By definition of cofinality we follow that
there is a function $f: \kappa \to \beta$ such that $f[\kappa] = \ran(f)$ is cofinal in $\beta$.
Definition of $\beta$ implies that there is a function $g: \beta  \to \gamma$ such that
$g[\beta]$ is cofinal in $\gamma$.

Now we need to justify the fact that for well-ordered sets $A$ and $B$,
if there is a function $f: A \to B$, then there is
a function $g: A \to B$ such that $x \preceq y \lra g(x) \preceq g(y)$. This is a crude
method, and there's probably a better proof of this lemma, but I can't seem to find one.

We follow that  $g: \beta \to \gamma$ is a function such that $\ran(g)$ is cofinal in $\gamma$.
We follow that there is a set $B \subseteq \beta$ such that 
$$B = \set{\mu \in \beta: (\forall \delta \in \mu)( g(\mu) \neq g(\delta))}$$
and by definition of it we follow that $g|B$ is injective and it has the same range as $g$.
Thus we follow that we can define $g': B \to \ran(g) = g|B$, which will be bijective.

Since $g'$ is bijective, we follow that $|B| = |\ran(g)|$. We also follow that
$B$ and $\ran(g)$ both have well-ordering by $\ineq$, and they also have the same
order type, which we're gonna name $\alpha$. Thus we follow that there is a bijection
$l: B \to \ran(g)$ such that $x \in y \lra l(x) \in l(y)$.

We follow that we can create a funtion $p: \beta \to \gamma$ by
$$p(x) = l(\ineq-\text{least element of $B \setminus x$})$$
Thus we follow that $\ran(p \circ f)$ is cofinite in $\gamma$, and thus we conclude that
there is $\kappa \in \beta$ and a function $(p \circ f): \kappa \to \gamma$ such that
$\ran(p \circ f)$ is cofinite in $\gamma$, which means that $\beta \neq cf(\gamma)$, which
is a contradiction.

\subsection*{9.1.15}

\textit{Prove that $\aleph: \On \to \On$ is strictly increasing.}

We follow that $\aleph$ is continous by definition. Thus we follow that we need
to prove that $\aleph(\gamma) \in \aleph(\gamma^+)$, and my Theorem 8.2.9
we'll get that the function is strictly increasing (and by extension normal).

Let $\gamma \in \On$. We follow that $\aleph(\gamma^+) = S(\aleph(\gamma))$ by definition.
Since $S$ is defined to be the least cardinal such that $\aleph(\gamma) \in S(\aleph(\gamma))$,
we follow that $\aleph(\gamma) \in S(\aleph(\gamma)) \lra \aleph(\gamma) \in \aleph(\gamma^+)$.
Thus we follow by theorem 8.2.9 that $\aleph$ is strictly increasing, as desired.

\subsection*{9.1.16}

\textit{Let $\alpha$ be a limit ordinal. Prove that $cf(\aleph_\alpha) = cf(\alpha)$}

We follow that since $\alpha$ is a limit ordinal that
$$\aleph_\alpha = \sup \aleph[\alpha]$$

Let $\beta = cf(\alpha)$. We follow that there is a function $f: \beta \to \alpha$
such that $f[\alpha]$ is cofinal by definition.

Now define a function $g = \aleph \circ f: \beta \to \alpha$. 
Suppose that $\gamma \in \aleph_\alpha$. We follow that $\gamma \in \sup \aleph[\alpha]$,
and thus $\gamma \in \bigcup{ \aleph[\alpha]}$. Thus there is $\theta \in  \alpha$
such that $\gamma \in \aleph(\theta) $. Thus we follow that there is an element $\mu \in \beta$
such that $\theta \in f(\mu)$, and thus $\aleph(\theta) \in \aleph(f(\mu))$ by the
fact that $\aleph$ is strictly increasing. Thus we follow that $g[\beta]$ is cofinal in
$\aleph_\alpha$ and thus we follow that
$cf(\aleph_\alpha) \ineq cf(\alpha)$.

Now assume that $\nu = cf(\aleph_\alpha)$. We follow that there is a function
$f: \nu \to \aleph_\alpha$ such that $f[\nu]$ is cofinal in $\aleph_\alpha$.
Let $\beta \in \alpha$. We follow that there is unique $\aleph_\beta \in \aleph_\alpha$,
and thus there is $\mu \in \nu$ such that $f(\mu) \in \aleph_\alpha$ such that
$\aleph_\beta \in f(\mu)$. We follow that $S(f(\mu)) = \aleph_\theta$ for some $\theta \in \alpha$,
and thus we can define a function $\aleph\inv \circ S \circ f: \nu \to \alpha$
such that $(\aleph\inv \circ S \circ f)[\nu]$ is cofinal in $\alpha$. Thus
$\nu = cf(\aleph_\alpha) \ineq cf(\alpha)$, and thus we conclude that
$cf(\aleph_\alpha) = cf(\alpha)$, as desired.

\subsection*{9.1.17}

\textit{Prove that for all ordinals $\beta$ we have that $\beta \ineq \aleph_\beta$.}

Since $\aleph$ is strictly increasing, we follow by exercise 8.2.9 that $\beta \in \aleph(\beta)$.

\subsection*{9.1.18}

\textit{Prove for every ordinal $\alpha$ there is an ordinal $\beta$ such that
  $\alpha \ineq \beta$ and $\aleph_\beta = \beta$.}

Follows from the fact that $\aleph$ is normal and theorem 8.2.19

\subsection*{9.1.19}

\textit{Prove there is an ordinal $\xi$ such that $cf(\xi) = \omega$ and $\aleph_\xi = \xi$}

From the proof of 8.2.19 we follow that $cf(C) = cf(\sup C) = \omega$, and
$aleph_{\sup C} = \sup C$.

\subsection*{9.1.20}

\textit{Let $\kappa$ be an infinite cardinal. Prove that $\aleph_\alpha = \kappa$
  for some ordinal $\alpha$.}

There is a least ordinal $\alpha$ such that $\kappa \ineq \aleph_\alpha$.

If $\alpha = 0$, then we follow that $\kappa = \omega$, and we're done.
If $\alpha$ is a successor ordinal, then we follow that $\alpha = \gamma^+$, and
thus $\aleph(\alpha) = \aleph(\gamma^+) = S(\aleph(\gamma))$. Since $\alpha$ is the lowest
ordinal such that $\kappa \ineq \aleph_\alpha$, we follow that $\aleph_\gamma \in \kappa$.
By definition of $S$ and $\aleph_\alpha$ we  follow that both $\aleph_\alpha$ and $\kappa$
are the lowest cardinals such that $\aleph_\alpha$ is in it. Thus we conclude that
$\aleph_\alpha = \kappa$.

If $\alpha$ is a limit ordinal, then we follow that
$$\aleph_\alpha = \sup \aleph[\alpha] = \bigcup \aleph[\alpha]$$
If $\kappa \in \aleph_\alpha$, we follow that there's $\beta \in \alpha$ is such that
$\kappa \in \aleph_\beta$, which would contradict the fact that $\alpha$ is the least
ordinal such that $\kappa \ineq \aleph_\alpha$. Thus we conclude that $\kappa = \aleph_\alpha$.

Therefore we conclude that if $\kappa$ is an arbitrary infinite arcdinal, then there's
an ordinal $\alpha$ such that $\kappa = \aleph_\alpha$.

\subsection*{9.1.21}

\textit{Let $\alpha$ be an ordinal so that $\aleph_0 \in \alpha \in \aleph_1$.
  Prove that $\alpha$ is countable.}

Firstly, we follow that $\alpha$ is not a cardinal by definition and following stuff of $\aleph$.

Assume that $\alpha$ is not countable. Since $\alpha \in \aleph_1$, we follow that
$\alpha \subseteq \aleph_1$, and thus there is an injection from $\alpha$ to $\aleph_1$.
We follow that there's no injection from $\aleph_1$ to $\alpha$ by the fact that $\aleph_1$
is a cardinal. Since $\alpha$ is a well-ordered set, we follow that it's got cardinality.
Thus there's a bijection from $\alpha$ to some cardinal $\beta$. We follow that
$\beta \notin \aleph_0$ by the fact that $\omega \subseteq \alpha$ and thus $\alpha$
is infinite. Thus we follow that the only other choice is if $\alpha$ has a bijectiton
with $\aleph_0$, and thus it is countable, as desired.

\subsection*{9.1.22}

\textit{Let $\alpha \ineq \gamma$ where $\alpha$ is an ordinal and $\gamma$ is a limit
  ordinal. Let $f: \alpha \to \gamma$ be such that $f[\alpha]$ is cofinal in $\gamma$.
  Prove that $\bigcup{\set{f(\xi): \xi \in \alpha}} = \gamma$.}

We follow that $f[\alpha]$ is a cofinal set in $\gamma$, thus
$$\gamma = \sup \gamma = \sup f[\alpha] = \sup \set{f(\xi): \xi \in \alpha} =
\bigcup \set{f(\xi): \xi \in \alpha} $$
by the properties of sets of ordinals  in the previos chapter.

\subsection*{9.1.23}

\textit{Assume that for any two sets $X$ and $Y$, either $|X| \leq_c |Y|$ or
  $|Y| \leq_c |X|$. Now, using Theorem 9.1.8, prove that every set can be well-ordered.}

Let $X$ be a set.  We follow that there is a cardinal $\kappa$ such that  there
is no injection $f: \kappa \to X$. Thus we follow that $|X| \leq_c |\kappa|$,
therefore there's an injection from $X$ to $\kappa$. Theorem 9.1.4 implies that
$X$ has a well-ordering.\

\subsection*{9.1.24}

\textit{Let $\gamma$ be an ordinal. Show that $\preceq_\gamma$ is a total order on
  $\gamma \times \gamma$}

Let $\eangle{\alpha, \beta}, \eangle{\delta, \theta}, \eangle{\mu, \nu} \in \gamma \times \gamma$.
We follow that
$$\max(\eangle{\alpha, \beta}) = \max(\eangle{\alpha, \beta}) \land \alpha = \alpha
\land \beta \ineq \beta$$
thus we follow that $\preceq_\gamma$ has reflexivity.

If
$$\eangle{\alpha, \beta} \preceq_\gamma \eangle{\delta, \theta}
\land
\eangle{\delta, \theta} \preceq_\gamma \eangle{\mu, \nu}$$
then we've got several cases.

Assume that $\max(\alpha, \beta) \in \max(\delta, \theta)$.
If $\max(\delta, \theta) \in \max(\mu, \nu)$, then $\max(\alpha, \beta) \in \max(\mu, \nu)$
by well-ordering of $\ineq$ in ordinals.
If $\max(\delta, \theta) = \max(\mu, \nu)$, then $\max(\alpha, \beta) \in \max(\mu, \nu)$

Assume that $\max(\alpha, \beta) = \max(\delta, \theta)$ and $\alpha \in \delta$.
If $\max(\delta, \theta) \in \max(\mu, \nu)$, then $\max(\alpha, \beta) \in \max(\mu, \nu)$.
If $\max(\delta, \theta) = \max(\mu, \nu)$ and $\delta \in \mu$, then $\alpha \in \mu$.
If $\max(\delta, \theta) = \max(\mu, \nu)$ and $\delta = \mu$, then $\alpha \in \mu$.

Assume that $\max(\alpha, \beta) = \max(\delta, \theta)$, $\alpha = \delta$ and
$\beta \ineq \theta$.
If $\max(\delta, \theta) \in \max(\mu, \nu)$, then $\max(\alpha, \beta) \in \max(\mu, \nu)$.
If $\max(\delta, \theta) = \max(\mu, \nu)$ and $\delta \in \mu$, then $\alpha \in \mu$.
If $\max(\delta, \theta) = \max(\mu, \nu)$, $\delta = \mu$ and $\theta \ineq \nu$, then
$\beta \ineq \nu$.
Thus we conclude that $\preceq_\gamma$ is transitive (phew).

Assume that $\eangle{\alpha, \beta} \preceq_\gamma \eangle{\delta, \theta}$ and
$ \eangle{\delta, \theta} \preceq_\gamma \eangle{\alpha, \beta}$.
We follow that if $max(\alpha, \beta) \in \max(\delta, \theta)$,
then $ \max(\delta, \theta) \notin \max(\alpha, \beta)$ and 
$ \max(\delta, \theta) \neq \max(\alpha, \beta)$, thus this case is impossible.
Thus $\max(\alpha, \beta) = \max(\delta, \theta)$. Case with $\alpha \in \delta$ is
also impossible for a simular reason. Thus we conclude that
$\alpha = \delta$ and $\beta \ineq \theta$ and $\theta \ineq \beta$. Thus
$\alpha = \delta \land \beta = \gamma$, thus $\eangle{\alpha, \beta} = \eangle{\delta, \theta}$.
Therefore we conclude that $\preceq_\gamma$ has antisymmetry. Therefore $\preceq_\gamma$ is
a preorder.

If $\eangle{\alpha, \beta} =  \eangle{\delta, \theta}$, then
$\eangle{\alpha, \beta} \preceq_\gamma \eangle{\delta, \theta}$ by reflexivity.
If $\eangle{\alpha, \beta} \neq \eangle{\delta, \theta}$, then we follow that
$\alpha \neq \delta \lor \theta \neq \beta$.
If $\alpha \neq \delta$, then $\alpha \in \delta$ or $\delta \in \alpha$.
If $\alpha \in \delta$, then we can follow that $\beta \in \theta$, $\theta \in \beta$
or $\beta = \theta$.

\textit{Rest of this exercise and the next exercise are skipped. Not sure if I'll ever come
  back to them}

\section{Cardinal Arithmetic}

\end{document}
%%% Local Variables:
%%% mode: latex
%%% TeX-master: t
%%% End:



