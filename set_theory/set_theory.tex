\documentclass[11pt,oneside,titlepage]{book}
\title{My set theory exercises}
\usepackage{amsmath, amssymb}
\usepackage{geometry}
\usepackage{hyperref}
\author{Evgeny Markin}
\date{2023}

\DeclareMathOperator \map {\mathcal {L}}
\DeclareMathOperator \ns {null}
\DeclareMathOperator \range {range}
\DeclareMathOperator \inv {^{-1}}
\DeclareMathOperator \Span {span}
\newcommand{\eangle}[1]{\langle #1 \rangle}


\begin{document}
\maketitle
\tableofcontents

\chapter{Introduction}

\section{Elementary Set Theory}

Let A, B, C be sets

\subsection{}

\textit{If $a \notin A \setminus B$ and $a \in A$, show that $a \in B$}

Because $a \notin A \setminus B$, we follow that $x \in B$ or $x \notin A$. Since $x \in A$, we
follow that $x \in B$, as desired.

\subsection{}

\textit{Show that if $A \subseteq B$, then $C \setminus B \subseteq C \setminus A$}

Let $c \in C \setminus B$. Then we follow that $c \in C$ or $c \notin B$. Since $A \subseteq B$,
we follow that $c \notin B$ implies that $c \notin A$. Thus we follow that
$c \in C \setminus B$ implies that $c \in C \setminus A$. Therefore $C \setminus B \subseteq
C \setminus A$.

\subsection{}

\textit{Suppose $A \setminus B \subseteq C$. Show that $A \setminus C \subseteq B$.}

Suppose that $a \in A \setminus C$. Then we follow that $a \in A$ and $a \notin C$.

Given that $A \setminus B \subseteq C$ and $A \notin C$, we follow that $a \notin A \setminus B$.
Thus $a \notin A$ or $a \in B$. Since $a \in A$, we follow that $a \in B$. Thus
$$a \in A \setminus C \to a \in B$$
$$A \setminus C \subseteq  B$$
as desired.

\subsection{}

\textit{Suppose $A \subseteq B$ and $A \subseteq C$. Show that $A \subseteq B \cap C$}

Suppose that $a \in A$. Then we follow that $a \in B$ and $a \in C$. Thus $a \in B \cap C$.
Therefore we follow that $A \subseteq B \cap C$.

\subsection{}

\textit{Suppose $A \subseteq B$ and $B \cap C = \emptyset$. Show that $A \in B \setminus C$}

Suppose that $a \in A$. Then we follow that $a \in B$ and since $B \cap C = \emptyset$, we
follow that $a \notin C$. Thus $a \in B \setminus C$ by definition. Therefore
$A \subseteq B \setminus C$.

\subsection{}

\textit{Show that $A \setminus (B \setminus C) \subseteq (A \setminus B) \cup C$.}
Suppose that $a \in A \setminus (B \setminus C)$. Then we follow that
$a \in A$ and $a \notin B \setminus C$. Thus $a \notin B$ and $a \in C$. Thus we
follow that $a \in A \setminus B$ or $a \in C$. Thus
$A \setminus (B \setminus C) \subseteq (A \setminus B) \cup C$
as desired.

\subsection{}

\textit{Let $P(x)$ be the property $x > \frac 1 x$. Are the assertions $P(2)$, $P(-2)$,
  $P(\frac 1 2)$ $P( \frac{-1}{2})$ true or false .}

$$2 > \frac 1 2 \to P(2) = true$$
$$-2 < \frac{-1}{2} \to P(-2) = false$$
last two are reversed.

\subsection{}

\textit{Sow that each of the following sets can be expressed as an interval}

$$a) (-3, 3)$$
$$b) (-3, \infty)$$
$$c) (-3, 3)$$

all of them follow from order properties of real numbers.

\subsection{}

\textit{Express the following sets as truth sets}

$$A = \{1, 4, 9, 16, 25, ...\} \iff A = \{x \in N: x = y^2 \textit{ for some } y \in N\}$$
$$B = \{..., -15, -10, -5, 0, 5, ... \} \iff A = \{x \in N: x = 5y  \textit{ for some } y \in N\}$$

\textit{Rest are also trivial, not gonna go deep here}

\end{document}
%%% Local Variables:
%%% mode: latex
%%% TeX-master: t
%%% End:
