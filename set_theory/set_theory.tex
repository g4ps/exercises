\documentclass[11pt,oneside,titlepage]{book}
\title{My set theory exercises}
\usepackage{amsmath, amssymb}
\usepackage{geometry}
\usepackage{hyperref}
\author{Evgeny Markin}
\date{2023}

\DeclareMathOperator \map {\mathcal {L}}
\DeclareMathOperator \pow {\mathcal {P}}
\DeclareMathOperator \ns {null}
\DeclareMathOperator \range {range}
\DeclareMathOperator \inv {^{-1}}
\DeclareMathOperator \Span {span}
\DeclareMathOperator \lra {\Leftrightarrow}
\DeclareMathOperator \eqv {\Leftrightarrow}
\DeclareMathOperator \la {\Leftarrow}
\DeclareMathOperator \ra {\Rightarrow}
\DeclareMathOperator \imp {\Rightarrow}
\DeclareMathOperator \true {true}
\DeclareMathOperator \false {false}
\newcommand{\eangle}[1]{\langle #1 \rangle}


\begin{document}
\maketitle
\tableofcontents

\chapter*{Useful things}

I think that it is pretty straightforward to define some function based on axioms that we get.
For example pairing axiom allows us to define $PA: S \times S \to S$ by
$$PA(u, v) = \{u, v\}$$
same goes for union axiom
$$UA(u) = \{\text{elements of elements of U}\}$$
Later some other function might be defined in the same manner.

In logic notation, I denote tautology as '$\true$' and contradiction as '$\false$'

\chapter{Introduction}

\section{Elementary Set Theory}

Let A, B, C be sets

\subsection{}

\textit{If $a \notin A \setminus B$ and $a \in A$, show that $a \in B$}

Because $a \notin A \setminus B$, we follow that $x \in B$ or $x \notin A$. Since $x \in A$, we
follow that $x \in B$, as desired.

\subsection{}

\textit{Show that if $A \subseteq B$, then $C \setminus B \subseteq C \setminus A$}

Let $c \in C \setminus B$. Then we follow that $c \in C$ or $c \notin B$. Since $A \subseteq B$,
we follow that $c \notin B$ implies that $c \notin A$. Thus we follow that
$c \in C \setminus B$ implies that $c \in C \setminus A$. Therefore $C \setminus B \subseteq
C \setminus A$.

\subsection{}

\textit{Suppose $A \setminus B \subseteq C$. Show that $A \setminus C \subseteq B$.}

Suppose that $a \in A \setminus C$. Then we follow that $a \in A$ and $a \notin C$.

Given that $A \setminus B \subseteq C$ and $A \notin C$, we follow that $a \notin A \setminus B$.
Thus $a \notin A$ or $a \in B$. Since $a \in A$, we follow that $a \in B$. Thus
$$a \in A \setminus C \to a \in B$$
$$A \setminus C \subseteq  B$$
as desired.

\subsection{}

\textit{Suppose $A \subseteq B$ and $A \subseteq C$. Show that $A \subseteq B \cap C$}

Suppose that $a \in A$. Then we follow that $a \in B$ and $a \in C$. Thus $a \in B \cap C$.
Therefore we follow that $A \subseteq B \cap C$.

\subsection{}

\textit{Suppose $A \subseteq B$ and $B \cap C = \emptyset$. Show that $A \in B \setminus C$}

Suppose that $a \in A$. Then we follow that $a \in B$ and since $B \cap C = \emptyset$, we
follow that $a \notin C$. Thus $a \in B \setminus C$ by definition. Therefore
$A \subseteq B \setminus C$.

\subsection{}

\textit{Show that $A \setminus (B \setminus C) \subseteq (A \setminus B) \cup C$.}
Suppose that $a \in A \setminus (B \setminus C)$. Then we follow that
$a \in A$ and $a \notin B \setminus C$. Thus $a \notin B$ and $a \in C$. Thus we
follow that $a \in A \setminus B$ or $a \in C$. Thus
$A \setminus (B \setminus C) \subseteq (A \setminus B) \cup C$
as desired.

\subsection{}

\textit{Let $P(x)$ be the property $x > \frac 1 x$. Are the assertions $P(2)$, $P(-2)$,
  $P(\frac 1 2)$ $P( \frac{-1}{2})$ true or false .}

$$2 > \frac 1 2 \to P(2) = true$$
$$-2 < \frac{-1}{2} \to P(-2) = false$$
last two are reversed.

\subsection{}

\textit{Sow that each of the following sets can be expressed as an interval}

$$a) (-3, 3)$$
$$b) (-3, \infty)$$
$$c) (-3, 3)$$

all of them follow from order properties of real numbers.

\subsection{}

\textit{Express the following sets as truth sets}

$$A = \{1, 4, 9, 16, 25, ...\} \iff A = \{x \in N: x = y^2 \textit{ for some } y \in N\}$$
$$B = \{..., -15, -10, -5, 0, 5, ... \} \iff A = \{x \in N: x = 5y  \textit{ for some } y \in N\}$$

\textit{Rest are also trivial, not gonna go deep here}

\section{Logical Notation}

\subsection*{1.2.1}

\textit{Using truth tables, show that $\neg(P \ra Q) \lra (P \land \neg Q)$}

\begin{center}
  \begin{tabular}{c| c| c| c| c| c|}
    P & Q & $P \ra Q$ & $\neg(P \ra Q)$ & $\neg Q$ & $P \land \neg Q$ \\
    false & false & true & false & true & false \\
    false & true & true & false & false & false \\
    true & false & false & true & true & true \\
    true & true & true & false & false & false \\
  \end{tabular}  
\end{center}

from this we can see that they are equqivalent.

\textit{Following 4 exercises are the same as this one, so I'm skipping them}

\subsection*{1.2.5}

\textit{Show that $(P \imp Q) \land (P \imp R) \eqv P \imp (Q \land R)$, using logic laws}

$$(P \imp Q) \land (P \imp R) \eqv (\neg P \lor Q) \land (\neg P \lor R) \eqv
\neg P  \lor (R \land Q) \eqv  P  \imp (R \land Q) $$
Laws used: 
$$CL \to DIST \to CL$$

\subsection*{1.2.6}

\textit{Show that $(P \imp R) \lor (Q \imp R) \eqv (P \land Q) \imp R$, using logic laws}

$$(P \imp R) \lor (Q \imp R)  \eqv (\neg P \lor R) \lor (\neg Q \lor R) \eqv
\neg P \lor R \lor \neg Q \lor R \eqv (\neg Q \lor \neg P) \lor R \eqv$$
$$ \eqv \neg (Q \land P) \lor R
\eqv (Q \land R) \imp R$$
Laws used:
$$CL \to ASC \to ID, ASC \to DML \to CL$$

\subsection*{1.2.7}

\textit{Show that $P \imp (Q \imp R) \eqv (P \land Q) \imp R$, using logic laws}

$$P \imp (Q \imp R) \eqv \neg P \lor (Q \imp R) \eqv \neg P \lor (\neg Q \lor R) \eqv
(\neg P \lor \neg Q) \lor R \eqv \neg (P \land Q) \lor R \eqv (P \land Q) \imp R$$
Laws used:
$$CL \to CL \to ASC \to DML \to CL$$

\subsection*{1.2.8}

\textit{Show that $(P \imp Q) \imp R$ and $P \imp (Q \imp R)$ are not logically equivalent}

We're gonna show that $q \land w \eqv false$
$$((P \imp Q) \imp R) \land (P \imp (Q \imp R)) \eqv (\neg (\neg P \lor Q) \lor R) \land
(\neg P \lor (\neg Q \lor R)) \eqv $$
$$\eqv ((P \land \neg Q) \lor R) \land (\neg P \lor \neg Q \lor R) \eqv
((P \land Q) \land (\neg P \lor \neg Q)) \lor R  \eqv$$
$$ \eqv ((P \land Q) \land \neg ( P \land  Q)) \lor R  \eqv false \lor R \eqv false$$


\section{Predicates and Quantifiers}

\section{A Formal Language for Set Theory}

\subsection{}

\textit{What does the formula $\exists x \forall y (x \notin y)$ say in English? }

There exists $x$ such that for every $y$ we've got that x is not in y. In other ways, there
exists an empty set.

\subsection{}

\textit{What does the formula $\forall y \exists x (y \notin x)$ say in English?}

For every y there exists set x such that y is not in x.

\subsection{}

\textit{What does the formula $\forall y \exists x (x \notin y)$ say in English?}

For every y there exists x such that x is not in y.

\subsection{}

\textit{What does the formula $\forall y \neg \exists x (x \notin y)$ say in English?}

For every y there does not exist an x such that x is not in y.

\subsection{}

\textit{What does the formula $\forall z \exists x \exists y (x \in y \land y \in z)$
  say in English?}

For every $z$ there exists x and y such that x is in y and y is in z

\subsection{}

\textit{Let $\phi(x)$ be a formula. What does $\forall z \forall y((\phi(x) \land \phi(y))
  \to z = y$}

For every z and y, $\phi(x)$ and $\phi(y)$ implies that z = y.

\subsection{}

\textit{Translate each of the following into the language of set theory.}

\textit{(a) x is the union of a and b}

$$\forall (y \in x) (y \in a \land y \in b)$$

\textit{(b) x is not a subset of y}

$$\exists (z \in x) (\neg z \in y)$$

\textit{(c) x is the intersection of a and b}

$$\forall (y \in x) (y \in a \lor y \in b)$$

\textit{(d) a and b have no elements in common}

$$\forall (x \in a) \forall (y \in b) (\neg x = y)$$

\subsection{}

\textit{Let a, b, C and D be sets. Show that the relationship}
$$y =
\begin{cases}
  a \text{ if } x \in C \setminus D \\
  b \text{ if } x \notin C \setminus D \\
\end{cases}
$$

$$((x \in C \land \neg x \in D) \to (y = a)) \land ((\neg x \in C \land \neg x \in D) \to (y = a))$$


\section{The Zermelo-Fraenkel Axioms}

\subsection{}

\textit{Let $u, v, w$ be sets. By pairing axiom, the sets $\{u\}$ and $\{v, w\}$
  exist. Using the pairing and union axioms, show that the set $\{u, v, w\}$
  exists.}

By pairing axiom we've got that
$$PA(u, u) = \{u\}$$
$$PA(v, w) = \{v, w\}$$
thus
$$PA(\{u\}, \{v, w\}) = \{\{u\}, \{v, w\}\}$$
and therefore by union axiom we follow that
$$UA(\{\{u\}, \{v, w\}\}) = \{u, v, w\}$$
as desired.

\subsection{}

\textit{Let $A$ be a set. Show that the pairing axiom implies that the set $\{A\}$ exists}

$$PA(A, A) = \{A, A\}$$
which by extension axiom is equal to $\{A\}$, as desired.

\subsection{}

\textit{Let $A$ be a set. The pairing axiom implies that the set $\{A\}$ exists. Using the
  regularity axiom, show that $A \cap \{A\} = 0$. Conclude that $A \notin A$.}

Since $\{A\} \neq \emptyset$, we follow that there exists $x$ such that $x \in \{A\}$ and
$x \cap \{A\} = \emptyset$. Since $A$ is the only element of $\{A\}$, we follow that
$A \cap \{A\} = \emptyset$, as desired.

\subsection{}

\textit{For sets $A, B$, the set $\{A, B\}$ exists by the pairing axiom. Let $A \in B$.
  Using the regularity axiom, show that $A \cap \{A, B\} = \emptyset$, and thus $B \notin A$.}

$\{A, B\}$ consists of sets $A$ and $B$, thus it is not empty and therefore
there exists $x \in \{A, B\}$ such that $x \in \{A, B\} \land x \cap \{A, B\} = \emptyset$.
For $B$ we've got that $B \in \{A, B\}$. Since $A \in B$ and $A \in \{A, B\}$, we can follow that
$A \in (B \cap \{A, B\})$. By pairing axiom we follow that the element with desired
property must exists, and given that the only other choice is $A$,
we conclude that $A \cap \{A, B\} = \emptyset$. Therefore we can follow that $B \notin A$, as
desired.

\subsection{}

\textit{Let $A, B, C$ be sets. Suppoes that $A \in B$ and $B \in C$. Using the regularity axiom,
  show that $C \notin A$.}

This is an expantion of previous exercise. We can follow that
$$B \in \{A, B, C\} \land B \in C \imp B \in C \cap \{A, B, C\} \imp C \cap \{A, B, C\}
\neq \emptyset$$
$$A \in \{A, B, C\} \land A \in B \imp A \in B \cap \{A, B, C\} \imp B \cap \{A, B, C\}
\neq \emptyset$$
thus the only other choice is $A$, and therefore $A \cap  \{A, B, C\} = \emptyset$. Therefore
$C \notin A$, as desired.


\subsection{}

\textit{Let $A, B$ be sets. Using the subset and power set axioms, show that the set
  $\pow(A) \cap B$ exists.}

Because set $A$ exists we follow that $\pow(A)$ exists. By setting $\phi(x): x \in B$ and
subset axiom we follow that there exists a subset of $\pow(A)$ such that
$x \in S \lra x \in \pow(A) \land x \in B$. Thus we follow by Extensionality axiom
that $S = \pow(A) \cap B$. Thus $\pow(A) \cap B$ exists.

\subsection{}

\textit{Let $A, B$ be sets. Using the subset axiom, show that the set $A \setminus B$ exists.}

$$\phi(x): \neg x \in B$$
thus by subset axiom
$$x \in S \lra x \in A \land \neg x \in B$$
thus $A \setminus B$ exists.

\subsection{}

\textit{Show that no two of the sets $\emptyset, \{\emptyset\}, \{\emptyset, \{\emptyset\}\}$
  are equal to each other.}

I had a little confusion with this one at first because I thought that every set has
empty set in it, which is false. Every set has an empty set as a subset, but it
might be so that empty set is not in the set itself.
$$\emptyset \notin \emptyset \land \emptyset \in \{\emptyset\} \imp \emptyset \neq \{\emptyset\}$$
$$\emptyset \notin \emptyset \land \emptyset \in \{\emptyset, \{\emptyset\}\} \imp
\emptyset \neq \{\emptyset, \{\emptyset\}\}$$
$$\{\emptyset\} \notin \{\emptyset\} \land \{\emptyset\} \in \{\emptyset, \{\emptyset\}\} \imp
\{\emptyset\} \neq \{\emptyset, \{\emptyset\}\}$$
all of the implication follow from extensionality axiom.

\subsection{}

\textit{Let $A$ be a set with no elements. Show that for all $x$, we have that $x \in A$ if
  and only if $x \in \emptyset$. Using the extensionality axiom, conclude that $A = \emptyset$.}

Suppose that $\neg x \in A$. Then we follow that $x$ is an element, therefore $\neg x \in \emptyset$.
Thus
$$\neg x \in A \imp \neg x \in \emptyset \iff x \in \emptyset \imp x \in A$$
Suppose that $\neg x \in \emptyset$. Then we follow that $x$ is an element. Thus $\neg x \in A$.
Thus
$$\neg x \in \emptyset \imp \neg x \in A \iff x \in A \imp x \in  \emptyset$$
thus we follow that
$$x \in \emptyset \lra x \in A$$
thus by extensionality axiom we follow that
$$\emptyset = A$$
which gives us nice follow-up that
$$\emptyset = \{\}$$

\subsection{}

\textit{Let $\phi(x, y)$ be the formula $\forall z(z \in y \lra z = x)$ which asserts that
  $y = \{x\}$. For all x the set $\{x\}$ exists. So $\forall x  \exists! y \phi(x, y)$.
  Let $A$ be a set. Show that the collection $\{\{x\}: x \in A\}$ is a set.}

We know that $A$ is a set and therefore $\pow(A)$ is also a set. Thus by subset axiom
there exists a set 
$$\exists S (x \in S \lra x \in \pow(A) \land \exists(y \in A)(\phi(x, y)))$$
which is precisely our collection.

\chapter{Basic Set-Building Axioms and Operations}

\section{The First Six Axioms}

Prove the following theorems, where $A, B, C, D$ are sets.

\subsection*{2.1.1}

$$A \subseteq B \to (A \subseteq A \cup B \land A \cap B \subseteq A )$$

$$ \forall x (x \in A \to x \in B) \to ((\forall x (x \in A \imp x \in A \lor x \in B)) \land
(\forall (x \in A \land x \in B \imp x \in A))) \lra $$
$$ \lra
\forall x (x \in A \to x \in B) \to ((\forall x (\neg x \in A \lor x \in A \lor x \in B)) \land
(\forall (\neg (x \in A \land x \in B) \lor x \in A))) \lra $$
$$ \lra \forall x (x \in A \to x \in B) \to ((\forall x ( \true  \lor x \in B)) \land
(\forall ( \neg x \in A \lor \neg x \in B \lor x \in A))) \lra $$
$$ \lra \forall x (x \in A \to x \in B) \to (\true \land
(\forall ( true \lor \neg x \in B))) \lra $$
$$ \lra \neg \forall x (x \in A \to x \in B) \lor (\true \land \true) \lra $$
$$ \lra \neg \forall x (x \in A \to x \in B) \lor \true \lra $$
$$\true$$

\subsection*{2.1.2}

$$A \subseteq B \land B \subseteq C \to A \subseteq C$$

$$(\forall x (x \in A \imp x \in B)) \land (\forall x (x \in B \imp x \in C)) \to
\forall x (x \in A \imp x \in C) \lra$$
$$\lra (\forall x (\neg x \in A \lor x \in B)) \land (\forall x (\neg x \in B \lor x \in C)) \to
\forall x (\neg x \in A \lor x \in C) \lra $$
$$\lra (\forall x ((\neg x \in A \lor x \in B) \land  (\neg x \in B \lor x \in C))) \to
\forall x (\neg x \in A \lor x \in C) \lra $$
$$\lra (\forall x ((\neg x \in A \land  (\neg x \in B \lor x \in C))
\lor( x \in B  \land  (\neg x \in B \lor x \in C)))) \to
\forall x (\neg x \in A \lor x \in C) \lra $$
$$\lra (\forall x (\neg x \in A \land  (\neg x \in B \lor x \in C))
\lor( (x \in B  \land  \neg x \in B)  \lor (x \in B  \land x \in C)))) \to
\forall x (\neg x \in A \lor x \in C) \lra $$
$$\lra (\forall x ((\neg x \in A \land  \neg x \in B) \lor (\neg x \in A \land x \in C) 
\lor(  x \in B  \land x \in C)) \to \forall x (\neg x \in A \lor x \in C) \lra  ... $$

So this thing is tedious as hell and should be left to computers.

Suppose that $x \in A$. Then we follow by $A \subseteq B$ that $x \in B$. Thus by $B \subseteq C$
we follow that$x \in C$. Therefore $x \in A \to x \in C$. Therefore $A \subseteq C$, as desired.

\subsection*{2.1.3}

$$B \subseteq C \imp A \setminus C \subseteq A \setminus B$$

Suppose that $x \in A \setminus C$. Then we follow that $x \in A$ and $x \notin C$. Therefore
$x \in A$ and $x \notin B$ since $B \subseteq C$. Thus $x \in A \setminus B$. Therefore
we follow that $B \subseteq C$ implies that  $A \setminus C \subseteq A \setminus B$,
as desired.

\subsection*{2.1.4}

$$C \subseteq A \land C \subseteq B \iff C \subseteq A \cap B$$

Suppose that $x \in C$. Then we follow that $x \in A$ and $x \in B$. Thus $x \in A \cap B$.
Therefore $C \subseteq A \cap B$. Thus we follow that
$C \subseteq A \land C \subseteq B \imp C \subseteq A \cap B$

Suppose that $x \in C$. Then we follow that $x \in A \cap B$. Thus $x \in A$ and $x \in B$.
Therefore $C \subseteq A \land C \subseteq B$. Therefore
$C \subseteq A \cap B \imp C \subseteq A \land C \subseteq B $
thus we follow that
$$C \subseteq A \land C \subseteq B \iff C \subseteq A \cap B$$
as desired.

\subsection*{2.1.5}

\textit{There exists an $x$ such that $x \notin A$}

Suppose that there does not exist $x$ such that $x \notin A$. Then we follow that every set is a
member of $A$, which is impossible.

\subsection*{2.1.6}

$$A \cap B = B \cap A$$

$$x \in A \cap B \iff x \in A \land x \in B \iff x \in B \land x \in A \iff x \in B \cap A$$

\subsection*{2.1.7}

$$A \cup B = B \cup A$$

$$x \in A \cup B \iff x \in A \lor x \in B \iff x \in B \lor x \in A \iff x \in B \cup A$$

\subsection*{2.1.8}

$$A \cap (B \cup C) = (A \cup C) \cap (A \cup B)$$

$$x \in A \cap (B \cup C) \lra  x \in A \land x \in (B \cup C) \lra
x \in A \land (x \in B \lor x \in C) \lra
$$
$$ \lra (x \in A \lor x \in C) \land (x \in A \lor x \in C)
\lra (x \in A \cup B) \lor (x \in A \cup C) \lra x \in ((A \cup B) \cap (A \cup C))$$

\subsection*{2.1.31}

$$A \subseteq \pow(\cup(A))$$

Let $x \in A$. Then we follow that $x \subseteq \cup(A)$. Thus $x \in \pow(A)$. Thus
$A \subseteq \pow(\cup(A))$.

\subsection*{2.1.32}

\textit{Let $C \in F$. Then $\pow(C) \in \pow(\pow(\cup F))$}

Suppose that $C \in F$. Then we follow that $C \subseteq \cup F$. Therefore $C \in \pow (\cup F)$.
Thus $\pow(C) \in \pow(\pow (\cup F))$.


\textit{the rest of the exercises for this section are more of the same.}

\section{Operations on Sets}

Prove the following theorems

\subsection{}

\textit{Let $A$ be a set and $F \neq \emptyset$. Then}
$$A \setminus \cap F = \cup\{A \setminus C: C \in F\}$$

$$x \in A \setminus \cap F \lra
x \in A \land x \notin \cap F \lra
x \in A \land \neg x \in \cap F \lra
x \in A \land \neg (\forall(C \in F)(x \in C)) \lra
$$
$$ \lra
x \in A \land \exists(C \in F)(x \notin C) \lra
\exists(C \in F)(x \notin C \land x \in A) \lra
\exists(C \in F)(x \in A \setminus C) \lra
x \in \cup\{A \setminus C: C \in F\}
$$
which seems to hold.




\end{document}
%%% Local Variables:
%%% mode: latex
%%% TeX-master: t
%%% End:
