\documentclass[11pt,oneside,titlepage]{book}
\title{My set theory exercises}
\usepackage{amsmath, amssymb}
\usepackage{geometry}
\usepackage{hyperref}
\author{Evgeny Markin}
\date{2023}

\DeclareMathOperator \map {\mathcal {L}}
\DeclareMathOperator \pow {\mathcal {P}}
\DeclareMathOperator \ns {null}
\DeclareMathOperator \range {range}
\DeclareMathOperator \fld {fld}
\DeclareMathOperator \inv {^{-1}}
\DeclareMathOperator \Span {span}
\DeclareMathOperator \lra {\Leftrightarrow}
\DeclareMathOperator \eqv {\Leftrightarrow}
\DeclareMathOperator \la {\Leftarrow}
\DeclareMathOperator \ra {\Rightarrow}
\DeclareMathOperator \imp {\Rightarrow}
\DeclareMathOperator \true {true}
\DeclareMathOperator \false {false}
\DeclareMathOperator \dom {dom}
\DeclareMathOperator \ran {ran}
\newcommand{\eangle}[1]{\langle #1 \rangle}


\begin{document}
\maketitle
\tableofcontents

\chapter*{Useful things}

I think that it is pretty straightforward to define some function based on axioms that we get.
For example pairing axiom allows us to define $PA: S \times S \to S$ by
$$PA(u, v) = \{u, v\}$$
same goes for union axiom
$$UA(u) = \{\text{elements of elements of U}\}$$
Later some other function might be defined in the same manner.

In logic notation, I denote tautology as '$\true$' and contradiction as '$\false$'

There is a rule that I've used
$$a \land (b \lor \neg a) \lra (a \land b) \lor (a \land \neg a))
\lra (a \land b) \lor (\false) \lra a \land b$$
which I don't remember seeing in the book, but it's pretty useful.

\chapter{Introduction}

\section{Elementary Set Theory}

Let A, B, C be \subsection{}

\textit{If $a \notin A \setminus B$ and $a \in A$, show that $a \in B$}

Because $a \notin A \setminus B$, we follow that $x \in B$ or $x \notin A$. Since $x \in A$, we
follow that $x \in B$, as desired.

\subsection{}

\textit{Show that if $A \subseteq B$, then $C \setminus B \subseteq C \setminus A$}

Let $c \in C \setminus B$. Then we follow that $c \in C$ or $c \notin B$. Since $A \subseteq B$,
we follow that $c \notin B$ implies that $c \notin A$. Thus we follow that
$c \in C \setminus B$ implies that $c \in C \setminus A$. Therefore $C \setminus B \subseteq
C \setminus A$.

\subsection{}

\textit{Suppose $A \setminus B \subseteq C$. Show that $A \setminus C \subseteq B$.}

Suppose that $a \in A \setminus C$. Then we follow that $a \in A$ and $a \notin C$.

Given that $A \setminus B \subseteq C$ and $A \notin C$, we follow that $a \notin A \setminus B$.
Thus $a \notin A$ or $a \in B$. Since $a \in A$, we follow that $a \in B$. Thus
$$a \in A \setminus C \to a \in B$$
$$A \setminus C \subseteq  B$$
as desired.

\subsection{}

\textit{Suppose $A \subseteq B$ and $A \subseteq C$. Show that $A \subseteq B \cap C$}

Suppose that $a \in A$. Then we follow that $a \in B$ and $a \in C$. Thus $a \in B \cap C$.
Therefore we follow that $A \subseteq B \cap C$.

\subsection{}

\textit{Suppose $A \subseteq B$ and $B \cap C = \emptyset$. Show that $A \in B \setminus C$}

Suppose that $a \in A$. Then we follow that $a \in B$ and since $B \cap C = \emptyset$, we
follow that $a \notin C$. Thus $a \in B \setminus C$ by definition. Therefore
$A \subseteq B \setminus C$.

\subsection{}

\textit{Show that $A \setminus (B \setminus C) \subseteq (A \setminus B) \cup C$.}
Suppose that $a \in A \setminus (B \setminus C)$. Then we follow that
$a \in A$ and $a \notin B \setminus C$. Thus $a \notin B$ and $a \in C$. Thus we
follow that $a \in A \setminus B$ or $a \in C$. Thus
$A \setminus (B \setminus C) \subseteq (A \setminus B) \cup C$
as desired.

\subsection{}

\textit{Let $P(x)$ be the property $x > \frac 1 x$. Are the assertions $P(2)$, $P(-2)$,
  $P(\frac 1 2)$ $P( \frac{-1}{2})$ true or false .}

$$2 > \frac 1 2 \to P(2) = true$$
$$-2 < \frac{-1}{2} \to P(-2) = false$$
last two are reversed.

\subsection{}

\textit{Sow that each of the following sets can be expressed as an interval}

$$a) (-3, 3)$$
$$b) (-3, \infty)$$
$$c) (-3, 3)$$

all of them follow from order properties of real numbers.

\subsection{}

\textit{Express the following sets as truth sets}

$$A = \{1, 4, 9, 16, 25, ...\} \iff A = \{x \in N: x = y^2 \textit{ for some } y \in N\}$$
$$B = \{..., -15, -10, -5, 0, 5, ... \} \iff A = \{x \in N: x = 5y  \textit{ for some } y \in N\}$$

\textit{Rest are also trivial, not gonna go deep here}

\section{Logical Notation}

\subsection*{1.2.1}

\textit{Using truth tables, show that $\neg(P \ra Q) \lra (P \land \neg Q)$}

\begin{center}
  \begin{tabular}{c| c| c| c| c| c|}
    P & Q & $P \ra Q$ & $\neg(P \ra Q)$ & $\neg Q$ & $P \land \neg Q$ \\
    false & false & true & false & true & false \\
    false & true & true & false & false & false \\
    true & false & false & true & true & true \\
    true & true & true & false & false & false \\
  \end{tabular}  
\end{center}

from this we can see that they are equqivalent.

\textit{Following 4 exercises are the same as this one, so I'm skipping them}

\subsection*{1.2.5}

\textit{Show that $(P \imp Q) \land (P \imp R) \eqv P \imp (Q \land R)$, using logic laws}

$$(P \imp Q) \land (P \imp R) \eqv (\neg P \lor Q) \land (\neg P \lor R) \eqv
\neg P  \lor (R \land Q) \eqv  P  \imp (R \land Q) $$
Laws used: 
$$CL \to DIST \to CL$$

\subsection*{1.2.6}

\textit{Show that $(P \imp R) \lor (Q \imp R) \eqv (P \land Q) \imp R$, using logic laws}

$$(P \imp R) \lor (Q \imp R)  \eqv (\neg P \lor R) \lor (\neg Q \lor R) \eqv
\neg P \lor R \lor \neg Q \lor R \eqv (\neg Q \lor \neg P) \lor R \eqv$$
$$ \eqv \neg (Q \land P) \lor R
\eqv (Q \land R) \imp R$$
Laws used:
$$CL \to ASC \to ID, ASC \to DML \to CL$$

\subsection*{1.2.7}

\textit{Show that $P \imp (Q \imp R) \eqv (P \land Q) \imp R$, using logic laws}

$$P \imp (Q \imp R) \eqv \neg P \lor (Q \imp R) \eqv \neg P \lor (\neg Q \lor R) \eqv
(\neg P \lor \neg Q) \lor R \eqv \neg (P \land Q) \lor R \eqv (P \land Q) \imp R$$
Laws used:
$$CL \to CL \to ASC \to DML \to CL$$

\subsection*{1.2.8}

\textit{Show that $(P \imp Q) \imp R$ and $P \imp (Q \imp R)$ are not logically equivalent}

We're gonna show that $q \land w \eqv false$
$$((P \imp Q) \imp R) \land (P \imp (Q \imp R)) \eqv (\neg (\neg P \lor Q) \lor R) \land
(\neg P \lor (\neg Q \lor R)) \eqv $$
$$\eqv ((P \land \neg Q) \lor R) \land (\neg P \lor \neg Q \lor R) \eqv
((P \land Q) \land (\neg P \lor \neg Q)) \lor R  \eqv$$
$$ \eqv ((P \land Q) \land \neg ( P \land  Q)) \lor R  \eqv false \lor R \eqv false$$


\section{Predicates and Quantifiers}

\section{A Formal Language for Set Theory}

\subsection{}

\textit{What does the formula $\exists x \forall y (x \notin y)$ say in English? }

There exists $x$ such that for every $y$ we've got that x is not in y. In other ways, there
exists an empty set.

\subsection{}

\textit{What does the formula $\forall y \exists x (y \notin x)$ say in English?}

For every y there exists set x such that y is not in x.

\subsection{}

\textit{What does the formula $\forall y \exists x (x \notin y)$ say in English?}

For every y there exists x such that x is not in y.

\subsection{}

\textit{What does the formula $\forall y \neg \exists x (x \notin y)$ say in English?}

For every y there does not exist an x such that x is not in y.

\subsection{}

\textit{What does the formula $\forall z \exists x \exists y (x \in y \land y \in z)$
  say in English?}

For every $z$ there exists x and y such that x is in y and y is in z

\subsection{}

\textit{Let $\phi(x)$ be a formula. What does $\forall z \forall y((\phi(x) \land \phi(y))
  \to z = y$}

For every z and y, $\phi(x)$ and $\phi(y)$ implies that z = y.

\subsection{}

\textit{Translate each of the following into the language of set theory.}

\textit{(a) x is the union of a and b}

$$\forall (y \in x) (y \in a \land y \in b)$$

\textit{(b) x is not a subset of y}

$$\exists (z \in x) (\neg z \in y)$$

\textit{(c) x is the intersection of a and b}

$$\forall (y \in x) (y \in a \lor y \in b)$$

\textit{(d) a and b have no elements in common}

$$\forall (x \in a) \forall (y \in b) (\neg x = y)$$

\subsection{}

\textit{Let a, b, C and D be sets. Show that the relationship}
$$y =
\begin{cases}
  a \text{ if } x \in C \setminus D \\
  b \text{ if } x \notin C \setminus D \\
\end{cases}
$$

$$((x \in C \land \neg x \in D) \to (y = a)) \land ((\neg x \in C \land \neg x \in D) \to (y = a))$$


\section{The Zermelo-Fraenkel Axioms}

\subsection{}

\textit{Let $u, v, w$ be sets. By pairing axiom, the sets $\{u\}$ and $\{v, w\}$
  exist. Using the pairing and union axioms, show that the set $\{u, v, w\}$
  exists.}

By pairing axiom we've got that
$$PA(u, u) = \{u\}$$
$$PA(v, w) = \{v, w\}$$
thus
$$PA(\{u\}, \{v, w\}) = \{\{u\}, \{v, w\}\}$$
and therefore by union axiom we follow that
$$UA(\{\{u\}, \{v, w\}\}) = \{u, v, w\}$$
as desired.

\subsection{}

\textit{Let $A$ be a set. Show that the pairing axiom implies that the set $\{A\}$ exists}

$$PA(A, A) = \{A, A\}$$
which by extension axiom is equal to $\{A\}$, as desired.

\subsection{}

\textit{Let $A$ be a set. The pairing axiom implies that the set $\{A\}$ exists. Using the
  regularity axiom, show that $A \cap \{A\} = 0$. Conclude that $A \notin A$.}

Since $\{A\} \neq \emptyset$, we follow that there exists $x$ such that $x \in \{A\}$ and
$x \cap \{A\} = \emptyset$. Since $A$ is the only element of $\{A\}$, we follow that
$A \cap \{A\} = \emptyset$, as desired.

\subsection{}

\textit{For sets $A, B$, the set $\{A, B\}$ exists by the pairing axiom. Let $A \in B$.
  Using the regularity axiom, show that $A \cap \{A, B\} = \emptyset$, and thus $B \notin A$.}

$\{A, B\}$ consists of sets $A$ and $B$, thus it is not empty and therefore
there exists $x \in \{A, B\}$ such that $x \in \{A, B\} \land x \cap \{A, B\} = \emptyset$.
For $B$ we've got that $B \in \{A, B\}$. Since $A \in B$ and $A \in \{A, B\}$, we can follow that
$A \in (B \cap \{A, B\})$. By pairing axiom we follow that the element with desired
property must exists, and given that the only other choice is $A$,
we conclude that $A \cap \{A, B\} = \emptyset$. Therefore we can follow that $B \notin A$, as
desired.

\subsection{}

\textit{Let $A, B, C$ be sets. Suppoes that $A \in B$ and $B \in C$. Using the regularity axiom,
  show that $C \notin A$.}

This is an expantion of previous exercise. We can follow that
$$B \in \{A, B, C\} \land B \in C \imp B \in C \cap \{A, B, C\} \imp C \cap \{A, B, C\}
\neq \emptyset$$
$$A \in \{A, B, C\} \land A \in B \imp A \in B \cap \{A, B, C\} \imp B \cap \{A, B, C\}
\neq \emptyset$$
thus the only other choice is $A$, and therefore $A \cap  \{A, B, C\} = \emptyset$. Therefore
$C \notin A$, as desired.


\subsection{}

\textit{Let $A, B$ be sets. Using the subset and power set axioms, show that the set
  $\pow(A) \cap B$ exists.}

Because set $A$ exists we follow that $\pow(A)$ exists. By setting $\phi(x): x \in B$ and
subset axiom we follow that there exists a subset of $\pow(A)$ such that
$x \in S \lra x \in \pow(A) \land x \in B$. Thus we follow by Extensionality axiom
that $S = \pow(A) \cap B$. Thus $\pow(A) \cap B$ exists.

\subsection{}

\textit{Let $A, B$ be sets. Using the subset axiom, show that the set $A \setminus B$ exists.}

$$\phi(x): \neg x \in B$$
thus by subset axiom
$$x \in S \lra x \in A \land \neg x \in B$$
thus $A \setminus B$ exists.

\subsection{}

\textit{Show that no two of the sets $\emptyset, \{\emptyset\}, \{\emptyset, \{\emptyset\}\}$
  are equal to each other.}

I had a little confusion with this one at first because I thought that every set has
empty set in it, which is false. Every set has an empty set as a subset, but it
might be so that empty set is not in the set itself.
$$\emptyset \notin \emptyset \land \emptyset \in \{\emptyset\} \imp \emptyset \neq \{\emptyset\}$$
$$\emptyset \notin \emptyset \land \emptyset \in \{\emptyset, \{\emptyset\}\} \imp
\emptyset \neq \{\emptyset, \{\emptyset\}\}$$
$$\{\emptyset\} \notin \{\emptyset\} \land \{\emptyset\} \in \{\emptyset, \{\emptyset\}\} \imp
\{\emptyset\} \neq \{\emptyset, \{\emptyset\}\}$$
all of the implication follow from extensionality axiom.

\subsection{}

\textit{Let $A$ be a set with no elements. Show that for all $x$, we have that $x \in A$ if
  and only if $x \in \emptyset$. Using the extensionality axiom, conclude that $A = \emptyset$.}

Suppose that $\neg x \in A$. Then we follow that $x$ is an element, therefore $\neg x \in \emptyset$.
Thus
$$\neg x \in A \imp \neg x \in \emptyset \iff x \in \emptyset \imp x \in A$$
Suppose that $\neg x \in \emptyset$. Then we follow that $x$ is an element. Thus $\neg x \in A$.
Thus
$$\neg x \in \emptyset \imp \neg x \in A \iff x \in A \imp x \in  \emptyset$$
thus we follow that
$$x \in \emptyset \lra x \in A$$
thus by extensionality axiom we follow that
$$\emptyset = A$$
which gives us nice follow-up that
$$\emptyset = \{\}$$

\subsection{}

\textit{Let $\phi(x, y)$ be the formula $\forall z(z \in y \lra z = x)$ which asserts that
  $y = \{x\}$. For all x the set $\{x\}$ exists. So $\forall x  \exists! y \phi(x, y)$.
  Let $A$ be a set. Show that the collection $\{\{x\}: x \in A\}$ is a set.}

We know that $A$ is a set and therefore $\pow(A)$ is also a set. Thus by subset axiom
there exists a set 
$$\exists S (x \in S \lra x \in \pow(A) \land \exists(y \in A)(\phi(x, y)))$$
which is precisely our collection.

\chapter{Basic Set-Building Axioms and Operations}

\section{The First Six Axioms}

Prove the following theorems, where $A, B, C, D$ are sets.

\subsection*{2.1.1}

$$A \subseteq B \to (A \subseteq A \cup B \land A \cap B \subseteq A )$$

$$ \forall x (x \in A \to x \in B) \to ((\forall x (x \in A \imp x \in A \lor x \in B)) \land
(\forall (x \in A \land x \in B \imp x \in A))) \lra $$
$$ \lra
\forall x (x \in A \to x \in B) \to ((\forall x (\neg x \in A \lor x \in A \lor x \in B)) \land
(\forall (\neg (x \in A \land x \in B) \lor x \in A))) \lra $$
$$ \lra \forall x (x \in A \to x \in B) \to ((\forall x ( \true  \lor x \in B)) \land
(\forall ( \neg x \in A \lor \neg x \in B \lor x \in A))) \lra $$
$$ \lra \forall x (x \in A \to x \in B) \to (\true \land
(\forall ( true \lor \neg x \in B))) \lra $$
$$ \lra \neg \forall x (x \in A \to x \in B) \lor (\true \land \true) \lra $$
$$ \lra \neg \forall x (x \in A \to x \in B) \lor \true \lra $$
$$\true$$

\subsection*{2.1.2}

$$A \subseteq B \land B \subseteq C \to A \subseteq C$$

$$(\forall x (x \in A \imp x \in B)) \land (\forall x (x \in B \imp x \in C)) \to
\forall x (x \in A \imp x \in C) \lra$$
$$\lra (\forall x (\neg x \in A \lor x \in B)) \land (\forall x (\neg x \in B \lor x \in C)) \to
\forall x (\neg x \in A \lor x \in C) \lra $$
$$\lra (\forall x ((\neg x \in A \lor x \in B) \land  (\neg x \in B \lor x \in C))) \to
\forall x (\neg x \in A \lor x \in C) \lra $$
$$\lra (\forall x ((\neg x \in A \land  (\neg x \in B \lor x \in C))
\lor( x \in B  \land  (\neg x \in B \lor x \in C)))) \to
\forall x (\neg x \in A \lor x \in C) \lra $$
$$\lra (\forall x (\neg x \in A \land  (\neg x \in B \lor x \in C))
\lor( (x \in B  \land  \neg x \in B)  \lor (x \in B  \land x \in C)))) \to
\forall x (\neg x \in A \lor x \in C) \lra $$
$$\lra (\forall x ((\neg x \in A \land  \neg x \in B) \lor (\neg x \in A \land x \in C) 
\lor(  x \in B  \land x \in C)) \to \forall x (\neg x \in A \lor x \in C) \lra  ... $$

So this thing is tedious as hell and should be left to computers.

Suppose that $x \in A$. Then we follow by $A \subseteq B$ that $x \in B$. Thus by $B \subseteq C$
we follow that$x \in C$. Therefore $x \in A \to x \in C$. Therefore $A \subseteq C$, as desired.

\subsection*{2.1.3}

$$B \subseteq C \imp A \setminus C \subseteq A \setminus B$$

Suppose that $x \in A \setminus C$. Then we follow that $x \in A$ and $x \notin C$. Therefore
$x \in A$ and $x \notin B$ since $B \subseteq C$. Thus $x \in A \setminus B$. Therefore
we follow that $B \subseteq C$ implies that  $A \setminus C \subseteq A \setminus B$,
as desired.

\subsection*{2.1.4}

$$C \subseteq A \land C \subseteq B \iff C \subseteq A \cap B$$

Suppose that $x \in C$. Then we follow that $x \in A$ and $x \in B$. Thus $x \in A \cap B$.
Therefore $C \subseteq A \cap B$. Thus we follow that
$C \subseteq A \land C \subseteq B \imp C \subseteq A \cap B$

Suppose that $x \in C$. Then we follow that $x \in A \cap B$. Thus $x \in A$ and $x \in B$.
Therefore $C \subseteq A \land C \subseteq B$. Therefore
$C \subseteq A \cap B \imp C \subseteq A \land C \subseteq B $
thus we follow that
$$C \subseteq A \land C \subseteq B \iff C \subseteq A \cap B$$
as desired.

\subsection*{2.1.5}

\textit{There exists an $x$ such that $x \notin A$}

Suppose that there does not exist $x$ such that $x \notin A$. Then we follow that every set is a
member of $A$, which is impossible.

\subsection*{2.1.6}

$$A \cap B = B \cap A$$

$$x \in A \cap B \iff x \in A \land x \in B \iff x \in B \land x \in A \iff x \in B \cap A$$

\subsection*{2.1.7}

$$A \cup B = B \cup A$$

$$x \in A \cup B \iff x \in A \lor x \in B \iff x \in B \lor x \in A \iff x \in B \cup A$$

\subsection*{2.1.8}

$$A \cap (B \cup C) = (A \cup C) \cap (A \cup B)$$

$$x \in A \cap (B \cup C) \lra  x \in A \land x \in (B \cup C) \lra
x \in A \land (x \in B \lor x \in C) \lra
$$
$$ \lra (x \in A \lor x \in C) \land (x \in A \lor x \in C)
\lra (x \in A \cup B) \lor (x \in A \cup C) \lra x \in ((A \cup B) \cap (A \cup C))$$

\subsection*{2.1.31}

$$A \subseteq \pow(\cup(A))$$

Let $x \in A$. Then we follow that $x \subseteq \cup(A)$. Thus $x \in \pow(A)$. Thus
$A \subseteq \pow(\cup(A))$.

\subsection*{2.1.32}

\textit{Let $C \in F$. Then $\pow(C) \in \pow(\pow(\cup F))$}

Suppose that $C \in F$. Then we follow that $C \subseteq \cup F$. Therefore $C \in \pow (\cup F)$.
Thus $\pow(C) \in \pow(\pow (\cup F))$.


\textit{the rest of the exercises for this section are more of the same.}

\section{Operations on Sets}

Prove the following theorems

\subsection{}

\textit{Let $A$ be a set and $F \neq \emptyset$. Then}
$$A \setminus \cap F = \cup\{A \setminus C: C \in F\}$$

$$x \in A \setminus \cap F \lra
x \in A \land x \notin \cap F \lra
x \in A \land \neg x \in \cap F \lra
x \in A \land \neg (\forall(C \in F)(x \in C)) \lra
$$
$$ \lra
x \in A \land \exists(C \in F)(x \notin C) \lra
\exists(C \in F)(x \notin C \land x \in A) \lra
\exists(C \in F)(x \in A \setminus C) \lra
x \in \cup\{A \setminus C: C \in F\}
$$
which seems to hold.

\subsection{}

\textit{Let $A, F$ be sets. Then $A \cup (\cup F) = \cup\{A \cup C: C \in F\}$}

$$x \in A \cup (\cup F) \lra x \in A \lor x \in \cup F \lra
x \in A \lor  (\exists C \in F)(x \in C) \lra$$
$$
\lra (\exists C \in F)(x \in A) \lor \exists(C \in F)( x \in C) \lra
$$
$$ \lra 
\exists(C \in F)(x \in A \lor x \in C) \lra
\exists(C \in F)(x \in A \cup C) \lra x \in \cup\{A \cup C: C \in F\}
$$

Where we've used the fact that
$$x \in A \lra x \in A \land \true \lra x \in A \land (\exists C \in F)(\true) \lra
(\exists C \in F)(x \in A  \land \true) \lra  (\exists C \in F)(x \in A)$$
don't know if we can use it, but I used it anyways.




\subsection{}

\textit{Let $A, F$ be sets. Then $A \cap (\cup F) = \cup\{A \cap C: C \in F\}$}

$$x \in A \cap (\cup F) \lra x \in A \land x \in \cup F \lra
x \in A \land  (\exists C \in F)(x \in C) \lra$$
$$ \lra 
\exists(C \in F)(x \in A \land x \in C) \lra
\exists(C \in F)(x \in A \cap C) \lra x \in \cup\{A \cap C: C \in F\}$$


\subsection*{2.2.5}

\textit{Let $A$ and $F$ be sets. Then there exists a unique set $\epsilon$ such that for all
  $Y$ we have that $Y \in \epsilon$ if and only if $Y = A \cap C$ for some $C \in F$.}

$\cup F$ exists by union axiom, $A \cap (\cup F)$ exists by subset axiom. Thus $\pow(A \cap (\cup F))$
exists by power axiom. Since $Y = A \cap C \imp Y \subseteq A \cap (\cup F)$, we follow that
$Y$ is a subset of $\pow(A \cap (\cup F))$, which exists by subset axiom. By extensionality axiom we follow that
the set is unique.

\subsection*{2.2.12}

\textit{If $F$ and $G$ are nonempty sets, then }
$$\cap(F \cup G) = \cap(F) \cap  \cap(G)$$

$$x \in \cap(F \cup G) \lra (\forall C \in F \cup G)(x \in C) \lra (\forall C \in F)(x \in C) \land
(\forall C \in G)(x \in C) \lra $$
$$ \lra x \in \cap(F) \land x \in \cap(G) \lra x \in (\cap(F)) \cap (\cap(G))$$


\subsection*{2.2.14}

\textit{Let $F$ be a nonempty set. Then}
$$\pow(\cap (F)) = \cap\{\pow(C): C \in F\}$$

$$x \in \pow(\cap (F)) \lra x \subseteq \cap (F) \lra (\forall y \in x)(y \in \cap (F)) \lra
(\forall y \in x)(\forall (C \in F) (y \in F)) \lra $$
$$
\forall (C \in F) ( (\forall y \in x) y \in F) \lra
\lra \forall(C \in F)(x \subseteq C) \lra \forall(C \in F)(x \in \pow(C)) \lra x \in  \cap\{\pow(C): C \in F\}$$

\chapter{Relations and Functions}

\section{Ordered Pairs in Set Theory}

\subsection{}

\textit{Define $\eangle{a, b, c} = \eangle{\eangle{a, b}, c}$ for any sets $a, b, c$. Prove that this yields an
  ordered tuple; that is, prove tahat if $\eangle{x, y, z} = \eangle{a, b, c}$, then $x = a$, $y = b$, $z = c$.}

Suppose that
$$\eangle{x_1, x_2, x_3} = \eangle{y_1, y_2, y_3}$$
then we follow that
$$\eangle{\eangle{x_1, x_2}, x_3} = \eangle{\eangle{y_1, y_2}, y_3}$$
from which we get that $\eangle{x_1, x_2} = \eangle{y_1, y_2}$ and $x_3 = y_3$. From
$\eangle{x_1, x_2} = \eangle{y_1, y_2}$ we get that $x_1  = y_1$ and $x_2 = y_2$. In
total we get that
$$\eangle{\eangle{x_1, x_2}, x_3} = \eangle{\eangle{y_1, y_2}, y_3}
\imp x_1 = y_1 \land x_2 = y_2 \land x_3 = y_3$$
Thus we follow that given construction defines an ordered tuple, as desired.

\subsection{}

\textit{Prove that $(A \cup B) \times C = (A \times C) \cup (B \times C)$}

$$x \in (A \cup B) \times C \lra x = \eangle{y, z} \land y \in A \cup B \land z \in C
\lra  x = \eangle{y, z} \land (y \in A \lor y \in  B) \land z \in C $$
$$ \lra  (x = \eangle{y, z} \land z \in C ) \land (y \in A \lor y \in  B) \lra $$
$$\lra 
(x = \eangle{y, z} \land z \in C \land y \in A ) \lor
(x = \eangle{y, z} \land z \in C \land y \in B ) \lra$$
$$\lra 
(x \in A \times C ) \lor (x \in B \times C) \lra x \in (A \times C) \cup (B \times C)$$
as desired.

\subsection{}

\textit{Prove that $(A \setminus B) \times C = (A \times C) \setminus (B \times C)$}

$$x \in (A \setminus B) \times C \lra x = \eangle{y, z} \land y \in A \setminus B \land z \in C
\lra  x = \eangle{y, z} \land (y \in A \land y \notin  B) \land z \in C $$
$$ \lra  (x = \eangle{y, z} \land z \in C ) \land (y \in A \land y \notin  B) \lra $$
$$ \lra  x = \eangle{y, z} \land z \in C \land y \in A \land y \notin  B \lra $$
$$ \lra  (x = \eangle{y, z} \land y \in A \land z \in C) \land
 (x \neq \eangle{y, z} \lor y \notin B \lor z \notin C)  \lra $$
$$ \lra  (x = \eangle{y, z} \land y \in A \land z \in C) \land
 (x \neq \eangle{y, z} \lor y \notin B \lor z \notin C)  \lra $$
$$ \lra  (x = \eangle{y, z} \land y \in A \land z \in C) \land
\neg (x = \eangle{y, z} \land y \in B \land z \in C))  \lra $$
$$\lra 
(x \in A \times C) \land \neg (x \in B \times C) \lra x \in (A \times C) \setminus (B \times C)$$

Used a biconditional defined in "useful things"


\subsection{}

\textit{Prove that $$(\cup F) \times C = \cup\{A \times C: A \in F\}$$}

$$x \in (\cup F) \times C \lra x = \eangle{y, z} \land y \in (\cup F) \land z \in C \lra
x = \eangle{y, z} \land (\exists A \in F)(y \in A) \land z \in C \lra
$$
$$ \lra   (\exists A \in F)(y \in A \land x = \eangle{y, z} \land z \in C) \lra
(\exists A \in F)(x \in A \times C) \lra$$
$$\lra x \in  \cup\{A \times C: A \in F\}$$

\section{Relations}

\subsection{}

\textit{Explain why the empty set is a relation}

Relation is defined to be a set of ordered pairs. That is, for every $x \in R$, $x$ is an ordered pair.
Since we haven't got any elements in the emptyset, we follow that the logical statement is true and  therefore
emptyset is a relation.

Other way to see it is to assume that it is not a relation. Then we follow that emptyset has an element that
is not an ordered pair. Since emptyset does not have any elements, we follow that we have
a contradiction.

\subsection{}

\textit{Prove items 1-3 of Theorem 3.2.7}

$$x \in \dom(R \inv) \lra \exists y (\eangle{x, y} \in R \inv)
\lra \exists y (\eangle{y, x} \in R) \lra x \in \ran(R)$$

$$x \in \ran(R \inv) \lra \exists y (\eangle{y, x} \in R \inv)
\lra \exists y (\eangle{x, y} \in R) \lra x \in \dom(R)$$

$$x \in (R \inv) \inv \lra \exists y \exists z (\eangle{y, z} \in (R \inv) \inv ) \land x = \eangle{y, z}
\lra \exists y \exists z (\eangle{z, y} \in (R \inv) ) \land x = \eangle{y, z} \lra $$
$$ \lra  \exists y \exists z (\eangle{y, z} \in R ) \land x = \eangle{y, z} \lra x \in R$$


\subsection*{3.2.4}

$$\dom(R) = \{0, 1, 2, 3, 4\}$$
$$\ran(R) = \{0, 1, 2, 3, 4\}$$
$$R \circ R = \{\eangle{0, 2}, \eangle{0, 3}, \eangle{0, 0}, \eangle{0, 3}, \eangle{0, 4},
\eangle{1, 0}, \eangle{1, 3}, \eangle{1, 4}, \eangle{1, 3}, \eangle{1, 2}, \eangle{2, 1},$$
$$ 
\eangle{2, 2}, \eangle{2, 3}, \eangle{2, 2}, \eangle{2, 4}, \eangle{3, 3}, \eangle{3, 2},
\eangle{4, 4}\}$$
$$R|\{1\} = \{\eangle{1, 2}, \eangle{1, 3}\}$$
$$R\inv|\{1\} = \{\eangle{1, 0}\}$$
$$R[\{1\}] = \{2, 3\}$$
$$R\inv[\{1\}] = \{0\}$$

\subsection*{3.2.5}

\textit{Suppose that $R$ is a relation. Prove that $R|(A \cup B) = (R | A) \cup (R | B)$
  for any sets $A, B$}

$$x \in R|(A \cup B) \lra (\exists y \in A \cup B) (\exists z \in  \ran(R)) (\eangle{y, z}
\in R \land x = \eangle{y, z}) \lra $$
$$ 
\lra (\exists y \in A) (\exists z \in  \ran(R)) (\eangle{y, z}
\in R \land x = \eangle{y, z}) \lor
 (\exists y \in B) (\exists z \in  \ran(R)) (\eangle{y, z}
\in R \land x = \eangle{y, z}) \lra $$
$$ \lra x \in R|A \lor x \in R|B \lra x \in (R|A \cup R|B)$$
thus
$$R|(A \cup B) = (R | A) \cup (R | B)$$
as desired.

\subsection*{3.2.6}

\textit{Let $R$ be a relation. Prove that $\fld(R) = \bigcup \bigcup R$.}

$$x \in \fld(R) \lra x \in \dom(R) \lor x \in \ran(R) \lra
(\exists z \in R)(\exists y)(z = \eangle{x, y} \lor z = \eangle{y, x}) \lra$$
$$ \lra
(\exists z \in R)(\exists y)(z = \{\{x\}, \{x, y\}\} \lor z = \{\{y\}, \{y, x\}\})
\lra
(\exists z \in R)(x \in \bigcup z) \lra x \in \bigcup \bigcup R$$

\subsection*{3.2.7}

\textit{Let $R$ and $S$ be two relations and let $A, B, C$ be sets. Prove that $R|A$,
  $R\inv [B]$, $R[C]$ and $R \circ S$ are sets.}

Given that $R$ and $S$ are relation, we follow that both of them are sets,
$\bigcup \bigcup R$ and  $\bigcup \bigcup S$ are sets and $\dom (R), \ran(R), \dom(S), \ran(S)$
are sets. Thus we follow that $R|A$ is a subset of
$R$, which is a set; $R\inv [B]$  and $R[C]$ are subsets of $\bigcup \bigcup R$,
and $R \circ S$ are subsets of a set $\dom (R) \times \ran(S)$, which is a set.


\subsection*{3.2.8}

\textit{Let $R$ be a relation and $G$ be a set. Prove that $\{R[C]: C \in G\}$ is a set. Prove
  that if $G$ is nonempty, then $\{R[C]: C \in G\}$ is also nonempty}

If $R$ is a relation, then $\ran(R)$ is a set. Therefore $\pow(\ran(R))$ is a set. Thus
for any set $C$, $R[C] \subseteq \ran(R)$, therefore $R[C] \in \pow(\ran(R))$. Thus
$\{R[C]: C \in G\}$ is a subset of $\pow(\pow(\ran(R)))$, which is a set.

Suppose that $G$ is nonempty. Then we follow that there exists $C \in G$. Thus
$R[C]$ is a set. Thus $R[C] \in \{R[C]: C \in G\}$. Therefore $\{R[C]: C \in G\}$ is
nonempty.

\subsection*{3.2.10}

\textit{Let $R$ be a relation on $A$. Prove that $R$ is symmetric if and only if
  $$R\inv \subseteq R$$}

\textbf{In forward direction: }
Suppose that $R$ is symmetric. 
Let $y \in R \inv$. We follow that there exists $u, v$ such that $y = \eangle{u, v}$.
Thus $\eangle{v, u} \in R$ by the definition. Since $R$ is symmetric, we follow that
$\eangle{u, v} \in R$. Therefore $y \in R\inv \ra y \in R$, as desired.

\textbf{In reverse direction: }
Suppose that $R\inv \subseteq R$. Let $y \in R$. Then we follow that there exists
$u, v$ such that $y = \eangle{u, v}$. Thus $\eangle{v, u} \in R\inv$. Since $R\inv \subseteq R$,
we follow that $\eangle{v, u} \in R$. Thus we follow that $\eangle{u, v} \in R \ra
\eangle{u, v} \in R$. Thus $R$ is symmetric by definition, as desired.

\subsection*{3.2.19}

\textit{Prove item (2) of Theorem 3.2.8}

$$R[\bigcup G] = \bigcup{R[C]: C \in G}$$

$$x \in R[\bigcup G] \lra (\exists y \in \bigcup G)(\eangle{y, x} \in R) \lra
(\exists C \in G)(y \in C \land \eangle{y, x} \in R) \lra$$
$$\lra (\exists C \in G)(x \in R[C]) \lra x \in \bigcup{R[C]: C \in G}$$

\subsection*{3.2.20}

\textit{Prove item (4) fo Theorem 3.2.8}

$$x \in R[\bigcap G] \lra (\exists y \in \bigcap G)(\eangle{y, x} \in R) \lra
\exists y (\forall C \in  G)( y \in C \land \eangle{y, x} \in R) \ra$$
$$ \ra
(\forall C \in G) (\exists y \in C) ( \eangle{y, x} \in R) \lra
(\forall C \in G) (x \in R[C]) \lra x \in \bigcap{\{R[C]: C \in G\}}$$

\subsection*{3.2.22}

\textit{Let $R$ and $S$ be single-rooted relations. Prove that $R \circ S$ is
  single-rooted}

Suppose that $\eangle{x, y} \in R \circ S$. Then we follow that there exists $z \in
fld(R)$ such that
$\eangle{x, z} \in S$ and $\eangle{z, y} \in R$. Suppose that there exists  $\eangle{x', y} \in
R \circ S$ such that $x \neq x'$. Then we follow that there exists $j \in \fld(R)$ such that
$\eangle{x', j} \in S$ and $\eangle{j, y} \in R$. If $j = z$, then we follow that
$\eangle{x', z} \in S \land \eangle{x, z} \in S \land x \neq x'$. Thus we conclude that $S$ is not
single rooted, which is a contradiction. If $j \neq z$, then we follow that
$$\eangle{j, y} \in R \land \eangle{z, y} \in R \land j \neq z$$
thus $R$ is not single-rooted, which is also a contradiction. Therefore we conclude that
if $\eangle{x', y} \in R \circ S$, then $x' = x$. Therefore we follow that $R \circ S$ is
single rooted, as desired.

\section{Functions}

\subsection*{3.3.1}

\textit{Prove Lemma 3.3.5 and Lemma 3.3.13}

Suppose that $F$ and $G$ are functions such that $\dom(F) = \dom(G)$. 
Lemma 3.3.5 states that $F = G$ iff $F(x) = G(x)$ for every $x \in \dom(F)$
If $F = G$, then
$$F(x) = y \lra \eangle{x, y} \in F \lra \eangle{x, y} \in G \lra G(x) = y$$
thus $F(x) = G(x)$ for every $x \in \dom(F)$.

Now suppose that $F(x) = G(x)$ for every $x \in \dom(F)$. Then we follow that
$$z \in F \lra z = \eangle{x, y} \land F(x) = y \lra
z = \eangle{x, y} \land G(x) = y \lra z \in G$$
as desired.

Lemma 3.3.13 states that a function $F$ is one-to-one if and only if $F$ is single-rooted.

Suppose that $F$ is one-to-one and $F$ is not single rooted. Then we follow that
there exists $x, y \in F$ such that $x = \eangle{u, w} \in F, y = \eangle{j, w} \in F$.
Then we follow that $F(u) = w = F(j)$, which is a contradiction.

Proof of converse is extremely simular.

\subsection*{3.3.2}

\textit{Let $F$ be a function and let $A \subseteq B \subseteq \dom(F)$. Prove that
  $F[A] \subseteq F[B]$.}

$$x \in F[A] \lra x = \eangle{u, v} \land u \in A \land \eangle{u, v} \in F \ra
x = \eangle{u, v} \land u \in B \land \eangle{u, v} \in F \lra x \in F[B]$$

\subsection*{3.3.5}

\textit{Let $g: C \to D$ be a one-to-one function, $A \subseteq C$ and $B \subseteq C$. Prove
  that if $A \cup B = \emptyset$, then $g[A] \cap g[B] = \emptyset$.}

Suppose that $A \cap B = \emptyset$ and $g[A] \cap g[B] \neq \emptyset$. Then we follow that
there exists $x \in g[A] \cap g[B]$. Thus
$$x \in g[A] \land x \in g[B] \lra (\exists y \in A)(\eangle{y, x} \in g) \land
(\exists z \in B)(\eangle{z, x} \in g)$$
since $g$ is one-to-one, we follow that $z = y$. Thus there exists $z \in A \cap B$, therefore
$A \cap B \neq \emptyset$, which is a contradiction.

\subsection*{3.3.9}

\textit{Suppose that $F: X \to Y$ is a function. Prove that if $C \subseteq Y$ and
  $D \subseteq Y$, then $F\inv[C \cap D] = F\inv[C] \cap F\inv[D]$.}

Since $F$ is a function, we follow that $F\inv$ is a single-rooted relation. Thus we follow that
$$F\inv[C \cap D] = F\inv[C] \cap F\inv[D]$$
as desired.

\subsection*{3.3.10}

\textit{Let $F, G$ be functions from $A$ to $B$. Suppose $F \subseteq G$. Prove that $F = G$.}

Suppose that $x \in A$. Then we follow that
$$(\exists y \in B)(\eangle{x, y} \in F) \ra (\exists y \in B)(\eangle{x, y} \in G)$$
Thus we follow that for every $x \in A$ (where $\dom (F) = A = \dom(G)$)
$$F(x) = G(x)$$
thus by the lemma 3.3.5 we follow that
$$F = G$$
as desired.


\subsection*{3.3.11}

\textit{Let $C$ be a set of functions. Suppose that for all $f$ and $g$ in $C$, we have either
  $f \subseteq g$ or $g \subseteq f$. }

\textit{(a) Prove that $\cup C$ is a function}

Firstly, since $C$ is a set of sets of ordered pairs, we follow that $\cup C$ is a set of
ordered pairs, and therefore it is a relation. Suppose that $x \in \cup C$. Then we follow that
there exist  $f \in C$ such that $x \in f$ and 
$x = \eangle{u, v}$. Suppose that there exists $y \in \cup C$, such that $y = \eangle{u, v'}$,
where $u' \neq u$. Since $y \in \cup C$, we follow that there exists $g \in C$ such that
$y \in C$. Because $u' \neq u$, we follow that $g \neq f$. Therefore either $g \subset f$,
or $f \subset g$. In both cases we follow that we can't have the case that $u' \neq u$. Thus
we follow that for $x, y \in \cup C$, whenever the first part of the $x$ is equal to the first
part of $y$, we follow that the last parts are also equal. Thus we follow that $\cup C$ is
a single-valued relation, and therefore it is a function.

\subsection*{3.3.13}

\textit{Assume $f: A \to B$ is onto $B$. Let $C \subseteq B$. Prove that $f[f\inv[C]] = C$}

$$x \in f[f\inv[C]] \lra (\exists y \in f\inv[C])(f(y) = x) \lra
(\exists z \in C)(\eangle{y, z} \in f \land f(y) = x) \lra$$
$$ \lra (\exists z \in C)(f(y) = z \land f(y) = x) \lra (\exists z \in C)(x = z) \lra x \in C$$
this notation may be a bit sloppy, but the result is derived faithfully.

\subsection*{3.3.15}

\textit{Let $f: A \to B$ be a one-to-one function. Define $G: \pow(A) \to \pow(B)$ by
  $G(X) = f[X]$, for each $X \in \pow(A)$. Prove that $G$ is one-to-one.}

Let $X_1, X_2 \in \pow(A)$ be such that $G(X_1) = G(X_2)$. Then we follow that
$$f[X_1] = f[X_2]$$
thus
$$x \in X_1 \lra (\exists y \in f[X_1])(\eangle{x, y} \in f) \lra
(\exists y \in f[X_2])(\eangle{x, y} \in f) \lra x \in X_2$$
thus we follow that $X_1 = X_2$. Therefore $G(X_1) = G(X_2) \to X_1 = X_2$, thus $G$ is
one-to-one, as desired.

\subsection*{3.3.21}

\textit{Let $\eangle{A_i: i \in I}$ be an indexed function with nonempty terms. Prove that there
  is an indexed function ${x_i: i \in I}$ is that $x_i \in A_i$ for all $i \in I$, using
  theorem 3.3.24}

Let $C$ be defined as
$$C = ran(A)$$
then by theorem 3.3.24 we follow that there exists a function $H: C \to \cup C$ such that
$$H(A_i) \in A_i$$
define $x = H \circ A $. Then we follow that
$$x(i) = H(A(i)) = H(A_i) \in A_i$$
thus we have the desired function.

\section{Order Relations}

\subsection*{3.4.1}

\textit{Define a relation $\preceq$ on the set of intezers Z by $x \preceq y$ if and only if
  $x \leq y$ and $x + y$ is even for all $x, y \in Z$. Prove that $\preceq$ is a partial order
  on $Z$}

Suppose that $x, y \in Z$. Poset requirements of $\leq$ are going to be ommited.

Then we follow that $$x + x = 2x$$, therefore we've got symmetry.
If $x \leq y$, $y \leq z$, $x + y$ is even and $y + z$ is even, then we follow that
$x + y + y + z$ is even, therefore $x + z - 2y$ is also even.
Antisymmetry follows from $\leq$.

\textit{Then answer the following questions about the poset $(Z, \preceq)$}

\textit{(a) Is $S = \{1, 2, 3, 4, 5, 6, ..., \}$ a chain in $Z$}

No, since $1 + 2 = 3$ is not even, we follow that $1 \not \preceq 2 \land 2 \not \preceq 1$.

\textit{(b) Is $S = \{1, 3, 5, 7 ...,\}$ a chain in $Z$?}

Suppose that $x, y \in Z$. Then we follow that there exist $m, n \in Z$ such that
$$x = 2m + 1$$
$$y = 2n + 1$$
thus
$$x + y = 2m + 2n + 2$$
thus $x + y$ is even for all cases. Since $\leq$ is a total order, we follow that $S$ is
indeed a chain in $Z$.

\textit{(c) Does the  set $S = \{1, 2, 3, 4, 5\}$ have a lower bound or an upper bound?}

No, since we've got both odd and even numbers in $S$, we follow that there is no $x \in Z$ such that
$x + s$ is even for all $s \in S$, thus there does not exist an element such taht
$x \preceq s$ or $s \preceq x$ for all $s \in S$

\textit{(d) Does $S = \{1, 2, 3, 4, 5\}$ have any maximal or minimal elements?}

Yes, $1, 2$ are minimal elements and $4, 5$ are maximal elements.


\subsection*{3.4.2}

\textit{Prove Lemma 3.4.5}

Suppose that $(A, \preceq)$ is a poset and let $\prec$ be the strict order corresponding to
$\preceq$. Let $x, y, z \in A$. Then we follow that:

\textit{(1)}

Since $x = x$, we follow that $x \not \prec x$ by definition of a strict order

\textit{(2)}

Suppose that $x \prec y$. Then we follow that $x \preceq y$ and $y \neq x$. Thus by antisymmetry
of $\preceq$ we follow that $y \not \preceq x$, and therefore $y \not \prec x$ by definition.

\textit{(3)}

Suppose that $x \prec y$ and $y \prec z$. Therefore we follow that $x \preceq y$ and $y \preceq z$,
which gives us that $x \preceq z$. Suppose that $x = z$. Then we follow that $z \preceq y$
and thus $z = y$, which is a contradiction. Thus we follow that $x \preceq z$ and $x \neq z$,
therefore $x \prec z$, as desired.

\textit{(4)}

If $x = y$ then $x \not \prec y$ and $y \not \prec x$ by definition of strict order

Both $x \prec y$ and $y \prec x$ imply that $x \neq y$, and by case (2) we follow that
for given $x, y \in A$ only one of $x = y$, $x \prec y$ or $y \prec x$ hold.

\subsection*{3.4.3}

\textit{Find the greatest lower bound of the set $S = \{15, 20, 30\}$ in the poset
  $(A, |)$, where $A = \{n \in N: n > 0\}$. Now find the least upper bound of $S$.}

$5$ and $60$ (gcd and lcm) respectively.

\subsection*{3.4.4}

\textit{Let $(A, \preceq)$ be a poset and let $S \subseteq A$. Suppose that $b$ is the
  largest element of $S$. Prove that $b$ is also the least upper bound of $S$.}

Suppose that $s$ is an upper bound of $S$. Then from the definition of the lower bound we
follow that $x \prec s$ for every $x \in A$. Since $b \in s$ we follow that $b \prec s$.
Thus $b$ is the least upper bound (can't we just call it supremum?) of $A$ by definition.

\subsection*{3.4.11}

\textit{Let $(B, \preceq')$ be a poset. Suppose $h: A \to B$ is an injective function.
  Define the retalion $\preceq$ on $A$ by $x \preceq y$ if an only if $h(x) \preceq' h(y)$,
  for all $x, y \in A$. Prove that $\preceq$ is a parial order.}

Let $x, y, z \in A$. Since $h$ is a function, we follow that $x = x$ implies that $h(x) = h(x)$,
therefore $h(x) \preceq' h(x)$, thus $x \preceq x$. Thus we've got reflexivity.

If $x \preceq y$ and $y \preceq z$ we follow that $h(x) \preceq' h(y)$ and $h(y) \preceq h(z)$,
from which we follow that $h(x) \preceq' h(z)$ and therefore $x \preceq z$.

Suppose that $x \preceq y$ and $y \preceq x$. Then we follow that $h(x) \preceq' h(y)$
and $h(y) \preceq' h(x)$. Thus $h(y) = h(x)$. Because $h$ is one-to-one, we follow that
this implies that $x = y$. Thus we've got antisymmetry. Therefore $\preceq$ is indeed a poset.

\subsection*{3.4.12}

\textit{Let $(B, \preceq')$ be a totally ordered. Suppose $h: A \to B$ is an injective function.
  Define the retalion $\preceq$ on $A$ by $x \preceq y$ if an only if $h(x) \preceq' h(y)$,
  for all $x, y \in A$. Prove that $\preceq$ is a parial order.}

We follow that $\preceq$ is a poset from previous exercise. Suppose that $x, y \in A$.
then we follow that $h(x) \preceq' h(y) \lor h(y) \preceq' h(x)$. Thus
$x \preceq y \lor y \preceq x$. Therefore $\preceq$ is a total order.

\subsection*{3.4.14}

\textit{Analogous to 4}

\subsection*{3.4.14}

\textit{Let $\preceq$ be a partial order of $A$. Let $C \subseteq A$. Show that
  $\preceq_C$ is a partial order on $C$. Show that if $\preceq$ is a total order on $A$,
  then $\preceq_C$ is a total order on $C$.}

Everything follows directly from definitions.

\subsection*{3.4.17}

\textit{Let $\preceq$ be a partial order on $A$. Show that $fld(\preceq) = A$.}

For every $x \in A$ we've got that 
$x \preceq x$, thus we follow that $\eangle{x, x} \in \preceq$, thus $x \in fld(\preceq)$.
Thus $A \subseteq fld(\preceq)$. Since $fld(\preceq) \subseteq A$
we follow that $A = fld(\preceq)$, as desired.

\subsection*{3.4.18}

\textit{Let $C$ be a set where each $\preceq \in C$ is a partial order on its field.
  Suppose that for any $\preceq, \preceq'$ in $C$ either $\preceq \subseteq \preceq'$
  or $\preceq' \subseteq \preceq$}

\textit{(a) Show that for every $\preceq, \preceq'$ in $C$ if $\preceq \subseteq \preceq'$,
  then $\fld(\preceq) \subseteq \fld(\preceq')$}

Suppose that $\preceq \subseteq \preceq'$. Then we follow that
$\fld(\preceq) = \bigcup \bigcup \preceq$ and
$\fld(\preceq') = \bigcup \bigcup \preceq'$. Suppose that
$$x \in \fld(\preceq)$$
then we follow that there exists $y$ such that  $\eangle{x, y} \in \preceq$ or
$\eangle{y, x} \in \preceq$. Since $\preceq \subseteq \preceq'$, we follow
that $\eangle{x, y} \in \preceq'$ or $\eangle{y, x} \in \preceq'$. Thus
$x \in \fld(\preceq')$. Thus $x \in \fld(\preceq) \ra x \in \fld(\preceq')$,
as desired.

\textit{(b) Let $\preceq^C$ be the relation $\bigcup C$ with field
  $\bigcup\{\fld(\preceq): \preceq \in C\}$. Prove taht $\preceq^C$ is a partial order
  on its field}

Since $\preceq^C = \bigcup C$ we follow that if $x \in \bigcup\{\fld(\preceq): \preceq \in C\}$,
then there exists $\preceq' \in C$ such that $x \in \fld(\preceq')$. Thus, by
the fact that $\preceq'$ is a partial order, we follow that $\eangle{x, x} \in \preceq'$. Thus
$\eangle{x, x} \in \bigcup C$, and therefore $\eangle{x, x} \in \preceq^C$. Thus
$\preceq^C$ is reflexive.

Suppose that $x, y, z \in \bigcup\{\fld(\preceq): \preceq \in C\}$. Then we follow that
there exist $\preceq_x, \preceq_y, \preceq_z$ such that
$$x \in \fld(\preceq_x)$$
and so on. From the description of $C$ we follow that $\preceq_x \subseteq \preceq_y$
or $\preceq_y \subseteq \preceq_x$, which in any case means that there exists
$\preceq_{xy}$ (equal to either of them) such that
$$\{x, y\} \subseteq \fld(\preceq_{xy})$$
same goes for $\preceq_z$ and $\preceq_{xy}$, which means that there exists $\preceq'$ such that
$$\{x, y, z\} \subseteq \fld(\preceq')$$

Since $\preceq'$ is a partial order, we follow that if $x \preceq' y$ and $y \preceq' z$
implies that $x \preceq' z$, therefore $\eangle{x, z} \in \preceq'$, and therefore
$\eangle{x, z} \in \preceq^C$, thus the relation is transitive.

Now let's go back a bit and focus on $\preceq_{xy}$. We follow that if $x \preceq_{xy} y$
and $y \preceq_{xy} x$, then by antisymmetry of $\preceq_{xy}$ we follow that $x = y$. Thus
we've got the antisymmetry of $\preceq^C$.

Thus $\preceq^C$ satisfies all requirements of a poset, as desired.

\subsection*{3.4.19}

\textit{Let $(A, \preceq)$ and $(B, \preceq')$ be posets. Define the relation $\preceq_l$ on
  $A \times B$ by
  $$ \eangle{a, b} \preceq_l \eangle{x, y} \iff a \prec x \lor (a = x \land b \preceq' y) $$
  for all $\eangle{a, b}$ and $\eangle{x, y}$ in $A \times B$. }

\textit{(a) Prove that $\preceq_l$ is a partial order on $A \times B$}

Let $a \in A$ and $b \in B$. Then we follow that $b \preceq' b$ and $a = a$, thus
$$\eangle{a, b} \preceq_l \eangle{a, b}$$

If for some $a_1, a_2, a_3 \in A$ and $b_1, b_2, b_3 \in B$ it is true that
$$\eangle{a_1, b_1} \preceq_l \eangle{a_2, b_2}$$
and
$$ \eangle{a_2, b_2} \preceq_l \eangle{a_3, b_3} $$
then we follow that
$$a_1 \prec a_2 \lor (a_1 = a_2 \land b_1 \preceq' b_2)$$
and
$$a_2 \prec a_3 \lor (a_2 = a_3 \land b_2 \preceq' b_3)$$

If $a_1 = a_2 = a_3$, then we follow that $b_1 \preceq' b_2$ and $b_2 \preceq' b_3$, from which
we follow that $b_1 \preceq b_3$ and therefore $\eangle{a_1 b_1} \preceq \eangle{a_3, b_3}$.

If $a_1 \neq a_2$ and $a_2 = a_3$ then we follow that $a_1 \prec a_2$, thus $a_1 \prec a_3$,
therefore $\eangle{a_1, b_1} \preceq_l \eangle{a_3, b_3}$. Similar case holds if
$a_1 = a_2$ and $a_2 \neq a_3$

If $a_1 \neq a_2$ and $a_2 \neq a_3$, then we follow that $a_1 \prec a_2$ and $a_2 \prec a_3$,
from which we follow that $a_1 \preceq a_3$. If $a_1 = a_3$, then we follow that
$a_1 \not \prec a_2$, which is a contradiction. Thus $a_1 \neq a_3$. Thus $a_1 \prec a_3$.
Therefore $\eangle{a_1, b_1} \preceq_l \eangle{a_3, b_3}$

Thus we follow that $\eangle{a_1, b_1} \preceq_l \eangle{a_3, b_3}$ for all $a_1, a_3$, thus
we've got the transitive condition.

Now suppose that $\eangle{a_1, b_1} \preceq_l \eangle{a_2, b_2}$ and
$\eangle{a_2, b_2} \preceq_l \eangle{a_1, b_1}$. If $a_1 \neq a_2$, then we follow that
$a_1 \prec a_2$ and $a_2 \prec a_1$, which is not possible, thus we conclude that $a_1 = a_2$.
Thus we follow that $b_1 \preceq' b_2$ and $b_2 \preceq' b_1$, from which we follow that
$b_1 = b_2$. Thus $\eangle{a_1, b_1} \preceq_l \eangle{a_2, b_2}$ and
$\eangle{a_2, b_2} \preceq_l \eangle{a_1, b_1}$, then we follow that
$\eangle{a_1, b_1} = \eangle{a_2, b_2}$. Thus we've got antisymmetry.

Now we can follow that $\preceq_l$ satisfies all of the conditions of poset, as desired.

\textit{(b) Suppose that both $\preceq$ and $\preceq'$ are total orders on their respective sets.
  Prove that $\preceq_l$ is a total order on $A \times B$.}

Suppose that $\eangle{a_1, b_1}, \eangle{a_2, b_2}  \in A \times B$ are arbitrary. It follow that
$a_1 \preceq a_2$ or $a_2 \preceq a_1$. Thus $a_1 \prec a_2 \lor a_2 \prec a_1 \prec a_1 = a_2$.
We also follow that $b_1 \preceq' b_2$ or $b_2 \preceq' b_1$. This it can be shown that
$$\eangle{a_1, b_1} \preceq_l \eangle{a_2, b_2}$$
or
$$\eangle{a_2, b_2} \preceq_l \eangle{a_1, b_1}$$
as desired.

\section{Congruence and Preorder}

\subsection*{3.5.1}

\textit{Let $\sim$ be an equivalence relation on $A$. Let $f: A \to A/_\sim$ be defined by
  $$f(x) = [x]$$
  for all $x \in A$. Show that the function $f$ is one-to-one if and only if each equivalence
  classs in $A/_\sim$ is a singleton.}

\textbf{In forward direction: }
Suppose that
$$f(x) = [x]$$
is one-to-one. Suppose that $x, y \in [x]$. Then we follow that $f(x) = f(y)$. Thus by the fact
that $f$ is one-to-one we follow that $x = y$. Thus $[x]$ is a singleton.

\textbf{In reverse direction: }
Suppose that each $[x]$ is a singleton. Then we follow that if $x \neq y$, then $f(x) \neq f(y)$.
Therefore $f$ is one-to-one, as desired.

\subsection*{3.5.2}

\textit{Prove theorem 3.5.6}

Let $\sim$ be an equivalence relation on $A$ and let $f: A \times A \to A$.

Suppose that there exisxst a function $\hat{f}: A/_\sim \times A/_\sim \to A/_\sim$ such that
$\hat{f}([x], [y]) = [f(x, y)]$
then we follow that if $x \sim w$ and $y \sim z$, then $[x] = [w]$ and $[y] = [z]$, thus
$$\hat{f}([x], [y]) = [f(x, y)] = \hat{f}([w], [z]) = [f(w, z)]$$
thus we follow that $[f(x, y)] = [f(w, z)]$, therefore $f(x, y) \sim f(w, z)$.
Thus we follow that $x \sim w$ and $y \sim z$ implies than $f(x, y) \sim f(w, z)$, therefore
$f$ is congruent with $\sim$.

If $x \sim w$ and $y \sim z$ implies that $f(x, y) \sim f(w, z)$, then we follow that
we can construct a relation
$\hat{f}$  by
$$\hat{f} = \{ \eangle{\eangle{[x], [y]}, [f(x, y)]}, x, y \in A\}$$
if $[x] = [w]$ and $[y] = [z]$, then we follow that $x \sim w$ and $y \sim z$, therefore
$f(x, y) \sim f(w, z)$, therefore  $[f(x, y)] = [f(w, z)]$. Thus this relation is a function, as
desired.

\subsection*{3.5.3}

\textit{Prove theorem 3.5.7}

Suppose that $\sim$ is an equivalence relation on $A$ and $R$ is a relation on $A$.

If there exists a relation $\hat{R}$ on $A/_\sim$ such that
$\hat{R}([x], [y]) \iff R(x, y)$
then we follow that if  $xRy$ for some $x, y \in A$ then $\hat{R}([x], [y])$
If there exist $w, z \in A$ such that
$x \sim w$ and $y \sim z$, then $[x] = [w]$ and $[y] = [z]$, therefore $\hat{R}([w], [z])$
and thus $wRz$. Thus we follow that $xRy$ iff $wRz$. Thus $R$ is congruent with $\sim$.

If $R$ is congruent with $\sim$, then we can create a relation $\hat{R}$ by
$$\hat{R} = \{\eangle{[x], [y]} : \eangle{x, y} \in R\}$$
that will satisfy the desired conditions.

\subsection*{3.5.4}

\textit{Prove lemma 3.5.9}

Let $(A, \preceq)$ be a preordered set and let $\sim$ be the relation on $A$ defined by
$$a \sim b \iff a \preceq b \land b \preceq a$$
Then we follow that for $a \in A$ $a \preceq a$ by reflexivity of $\preceq$, therefore
$a \sim a$ and thus $\sim$ is reflexive.

If $a, b, c \in A$ such that $a \sim b$ and $b \sim c$, then we follow that
$$a \preceq b \land b \preceq c \ra a \preceq c$$
$$c \preceq b \land b \preceq a \ra c \preceq a$$
thus $a \sim c$. Therefore $\sim$ is transitive.

Suppose that $x \sim y$. Then we follow that $x \preceq y $ and $y \preceq x$. Thus we follow that
$y \sim x$, therefore $\sim$ is symmetric. Thus $\sim$ is an equivalence relation, as desired.

\subsection*{3.5.5}

\textit{Prove lemma 3.5.10}

Suppose taht $(A, \preceq)$ is a preordered set. Let $\sim$ be the derived equivalence
relation on $A$. Suppose that $a, b, c, d \in A$ are such that $a \sim c$ and $b \sim d$.

If $a \preceq b$ then we follow that $b \preceq d$ by $b \sim d$, therefore $a \preceq d$
by transitive property. From $a \sim c$ we derive that $c \preceq a$, therefore we
follow that $c \preceq d$. Similar implication hold for reverse case. Thus we follow that
$\preceq$ is congruent with $\sim$.


\chapter{The Natural Numbers}

\section{Inductive Sets}

\subsection*{4.1.1}

\textit{Let $I$ and $J$ be inductive sets. Prove that $I \cap J$ is also inductive.}

Since $I$ and $J$ are both inductive, we follow that $\emptyset \in I$ and $\emptyset \in J$,
thus $\emptyset \in I \cap J$.

Suppose that $x \in I \cap J$. Then we follow that $x \in I$ and $x \in J$. Since both of them
are inductive, we follow that $x^+ \in I$ and $x^+ \in J$, therefore $x^+ \in I \cap J$. Thus
we've got that
$$\emptyset \in I \cap J \land (\forall x \in I \cap J)(x^+ \in I \cap J)$$
thus we follow tha $I \cap J$ is inductive, as desired.

\subsection*{4.1.2}

\textit{Prove that if $A$ is a transitive set, then $A^+$ is also a transitive set.}

Suppose that $A$ is transitive. That means that $(\forall x \in A)(x \subseteq A)$.
By definition, $A^+ = A \cup \{A\}$
Now let $y \in A^+$. We follow that $y = A \lor y \in A$. If $y = A$, then we follow that
$y \subseteq A^+$. If $y \in A$, then we follow that $y \subseteq A$ by the fact that
$A$ is transitive, therefore $y \subseteq A^+$. Thus we've got that
$$(\forall y \in A^+)(y \subseteq A^+)$$
thus we follow that $A^+$ is a transitive set, as desired.

\subsection*{4.1.3}

\textit{Prove that if $A$ is a transitive set if and only if $A \subseteq \pow(A)$.}

Suppose that $A$ is transitive. Then we follow that $(\forall x \in A)(x \subseteq A)$. Thus
if $x \in A$, then $x \subseteq A$ and therefore $x \in \pow(A)$. Thus we follow that
$A \subseteq \pow(A)$.

If $A \subseteq \pow(A)$, then we follow that if $x \in A$, then $x \in \pow(A)$ and therefore
$x \subseteq A$. Thus $A$ is transitive.


\subsection*{4.1.6}

\textit{Prove Proposition 4.1.9}

4.1.9 states that
$$\bigcup(A \cup B) = (\bigcup A) \cup (\bigcup B)$$

$$x \in \bigcup(A \cup B) \lra (\exists C \in A \cup B)(x \in C) \lra
(\exists C \in A)(x \in C) \lor (\exists C \in B)(x \in C) \lra$$
$$ \lra \in \bigcup A \lor x \in \bigcup B \lra x \in (\bigcup A) \cup (\bigcup B)$$


\subsection*{4.1.7}

\textit{Prove that $n \neq n^+$ for all $n \in \omega$}

This is true in general by regularity axiom, but let's apply some induction.
Let
$$I =  \{n \in \omega: n \neq n^+\}$$

$0 \neq \{0\}$, since $0 \in  \{0\}$ and $0 \notin 0$ (because it is an empty set). Thus $0 \in I$.

Suppose that $n \in I$. Then we follow that $n \neq n^+$.
Suppose that $n^+ = (n^+)^+$. Then we follow by 4.1.12 that $n = n^+$, which is a contradiction,
because $n \in I$. Thus we follow that $n \in I \ra n^+ \in I$, therefore $I$ is inductive,
therefore $I = \omega$, as desired.


\subsection*{4.1.8}

\textit{Prove that $n^+ \not \subseteq n$ for all $n \in \omega$}

Suppose that $n^+ \subseteq n$. Since $n^+ = n \cup \{n\}$, we follow that $n \subseteq n^+$.
Thus by double inclusion we've got that $n^+ = n$, which contradicts previous exercise.


\subsection*{4.1.9}

\textit{Prove that for all $m \in \omega$ and all $n \in \omega$, if $m \in n$, then
  $n \not \subseteq m$.}

Let
$$I = \{n \in \omega: m \in n \ra n \not \subseteq m\}$$

Since $m \in 0 \lra m \in \emptyset$ is always false, we follow that $0$ is vacuously in $I$.

Let $n \in I$. Let $m \in \omega$ be such that $m \in n^+$.
Thus $m \in n \cup \{n\}$. Thus $m \in n \lor m = n$.

If $m \in n$, then we follow that $n \not \subseteq m$, thus $(\exists y \in n)(y \notin m)$.
Thus $(\exists y \in n \cup \{n\})(y \notin m)$. Therefore $n^+ \not \subseteq m$.

If $m = n$, then we follow that $n^+ \not \subseteq n$ by exercise 4.1.8.
Thus we conclude that $n \in I \ra n^+ \in I$. Therefore $I$ is inductive and thus $I = \omega$,
as desired.

\subsection*{4.1.10}

\textit{Conclude from exercise 9 that $n \notin n$ for all $n \in \omega$.}

Suppose that $n \in n$. Then we follow by previous exercise
that $n \not \subseteq n$, which is bonkers.

\subsection*{4.1.11}

\textit{Let $A$ be a set and suppose that $\bigcup A = A$. Prove that $A$ is transitive
  and for all $x \in A$ there is a $y \in A$ such that $x \in y$}

Let $x \in A$ be arbitrary.
Then we follow that $x \subseteq \bigcup A$ and thus $x \subseteq A$. Thus we follow that
$A$ is transitive, as desired.

Suppoes that $x \in A$ and suppose that for all $y \in A$ we've got that $x \notin y$.
Then we follow that $(\forall y \in A)(x \notin y)$. Thus
$$(\forall y \in A)(x \notin y) \lra \neg (\exists y \in A)(x \in y) \lra x \notin \bigcup A \lra
x \notin A$$
which is a contradiction.

\section{The Recursion Theorem on $\omega$}

\textit{How would one modify the proof of Theorem 4.2.1 in order to establish the
  following: }

\textit{Theorem. Let $A$ be a set and let $a \in A$. Suppose that $f: \omega \times A \to A$
  is a function. Then there exists a unique function $h: \omega \to A$ such that
  $$h(0) = a$$
  $$h(n^+) = f(n, h(n))$$}

We can define $S$ and $h$ in normal fashion, then prove that $\eangle{0, a} \in h$ and
$\eangle{n, u} \in h$ will imply that $\eangle{n^+, f(n^+, h(n)} \in h$, thus concluding the
proof of claim 1.

Proof of claim 2 for the base case will not be altered. (IH) Will be modified in notation.


Proof of uniqueness does not depend on the domain of the functions and is true in general.

\subsection*{4.2.2}

\textit{Let $a \in A$ and $f: A \to A$ be a function such that $a \notin \ran(f)$. Suppose that
  $h: \omega \to A$ satisfies
  $$(1) h(0) = a$$
  $$(2) h(n^+) = f(h(n))$$
  Assume taht $h$ is one-to-one and onto A. Prove that $f$ is one-to-one.
}

Suppose that $f$ is not one-to-one. Then we follow that there exist $x \neq y \in A$
such that $f(x) = f(y)$. Since $h$ is onto, we follow that there exist $x', y' \in \omega$
such thar $h(x') = x$ and $h(y') = y$. Since $x \neq y$ we follow that $h(x') \neq h(y')$
and therefore $x' \neq y'$ because $h$ is a function.

Thus we follow that $x'^+ \neq y'^+$. Because $h$ is one-to-one we follow that
$$h(x'^+) \neq  h(y'^+)$$
thus
$$f(h(x')) \neq  f(h(y'))$$
$$f(x) \neq  f(y)$$
which is a contradiction. Thus we follow that $f$ is one-to-one.

\subsection*{4.2.3}

\textit{Let $A$ be a set and let $a \in A$. Suppose that $f: A \to A$ satisfies 
$$a \in f(a)$$
$$(\forall x, y \in A)(x \in y \to f(x) \in f(y))$$
Let $h: \omega \to A$ be states as in theorem 4.2.1. Prove that $h(n) \in h(n^+)$ for
all $n \in \omega$.}

We follow that $h(0) = a$ and $h(n^+) = f(h(n))$. For base case we've got that
$h(0) = a \in f(a) = h(0^+) = h(1)$.
Let $I$ be defined as a subset of $\omega$  such that $m \in I$ implies
that $h(m) \in h(m^+)$. Base case gives us that $0 \in I$. Suppose that
$n \in I$. Then we follow that
$$h(n) \in h(n^+)$$
Since $h(n), h(n^+) \in A$, we follow that
$$f(h(n)) \in f(h(n^+))$$
by definition of $h$ we follow that
$$h(n^+) \in h((n^+)^+)$$
thus $n^+ \in I$. Thus we follow that $I$ is an inductive set. Since $I \subseteq \omega$,
we follow that $I = \omega$, as desired.

\subsection*{4.2.4}

\textit{Let $F: A \to A$ be a function and let $y \in A$.}

\textit{(a) Prove that the class $S = \{B: B \subseteq A, y \in B, F[B] \subseteq B\}$}

From this definition we follow that $B \in S \ra B \subseteq A$. Thus $B \in S \ra
B \in \pow(A)$. Thus $S \subseteq \pow(A)$.Since $A$ is a set we follow that its power set is a set,
therefore the subset of a power set is also a set.

\textit{(b) Show that $S$ is nonempty}

We follow that $A \subseteq A$ by basic properties, $y \in B$ by the given restraints,
and $F[B] \subseteq A$ since $F: A \to A$, and therefore $\ran(F) \subseteq A$. Thus
we follow that $A \in S$, therefore $S$ is nonempty.

\textit{(c) Let $C = \bigcap S$. Prove that $y \in C$ and $F[C] \subseteq C$.}

We follow that
$$(\forall B \in S)(y \in B)$$
from definition of $S$. Suppose that $y \in F[C]$. Then we follow that
because $R[\bigcap G] \subseteq \bigcap\{R[C]: C \in G\}$ for any relation, we follow that
$$y \in \bigcap\{F[B]: B \in S\}$$
$$(\forall B \in S)(y \in F[B])$$
Since for every $B$ we've got that $F[B] \subseteq B$, therefore we can follow that
$$(\forall B \in S)(y \in B)$$
thus $y \in \bigcap S$. Thus $y \in F[C] \ra y \in C$. Thus $F[C] \subseteq C$, as desired.

\textit{(d) Prove that for all $B \subseteq A$, if $y \in B$ and $F[B] \subseteq B$, then
  $C \subseteq B$}

If $y \in B$ and $F[B] \subseteq B$, then we follow that $B \in S$. Since $C = \bigcap S$,
we follow that $(\forall x \in C) (x \in C \to x \in B)$. Thus $C \subseteq B$, as desired.

\textit{(e) Prove that $y \in F[C]$ if and only if $F[C] = C$}

If $F[C] = C$, then we follow that since $y \in C$, then $y \in F[C]$.

For $C$ we've got that $F[C] \subseteq C$. Thus $F[F[C]] \subseteq C$. Thus if $y \in F[C]$, then
$F[C]$ satisfies requirements presented in the previous point, and thus $C \subseteq F[C]$.
By double inclusion we follow that $F[C] = C$, as desired.

\section{Arithmetic on $\omega$}

\subsection*{Note}

Although not explicitly, but we've started to use the associative and commutative
properties of addition pretty liberaly at this point. To be more precise,
we explicitly define the notation of several pluses in a row to be left-associative (e.g.
$$a_1 + a_2 + a_3 = (a_1 + a_2) + a_3 = A(A(a_1, a_2), a_3) $$
is one such example). Associative and distributive properties guarantee us that the order in
which the functions are applied does not matter, and thus the notation is justified.

The order of operations of addition and multiplication is followed from the properties,
that are proven in the chapter, and we follow that
$$a \cdot b + c = (a \cdot b) + c = A(M(a, b), c)$$
thus the usual rules of order of operations applies.

In general, all of the operations are defined in terms of functions, and the usual notation
of addition and multiplication is nothing but a syntatic sugar. If in doubt, one can
prove that the usual notation makes sense pretty easily.

\subsection*{4.3.1}

\textit{Let $m \in \omega$. Suppose that $m + n = 0$. Prove that $m = n = 0$.}

Suppose that $n \neq 0$. Then we follow that there exists $j$ such that $j^+ = n$. Thus
$$m + n = m + j^+ = (m + j)^+ \neq 0$$
thus we get the contradiction.

\subsection*{4.3.2}

\textit{Let $m \in \omega$ and $n \in \omega$. Show that if $m \cdot n = 0$, then $m = 0$ or
  $n = 0$.}

Suppose that $m \neq 0$ and $n \neq 0$. Then we follow that there exist $n_p \in \omega$
such that $n = n_p^+$. Thus
$$m \cdot n = m \cdot n_p^+ = m \cdot n_p +  m$$
Since $m \neq 0 \in \omega$ and $m \cdot n_p^+ \in \omega$, we follow from previous exercise
that $m \cdot n_p = 0$ and $m = 0$, which contradicts the assumtions that $m \neq 0$.

\subsection*{4.3.3}

\textit{Let $m \in \omega$ and $n \in \omega$. Prove that for all $p \in \omega$, if
  $m + p = n + p$, then $m = n$.}

Let
$$I = \{a \in \omega: (\forall m, n \in \omega)(m + a = n + a \ra m = n)\}$$
If $p = 0$, then $m + 0 = m$, $n + 0 = n$, thus $m + 0 = n + 0$ implies that $m = n$.
Thus $0 \in I$.

(IH) Suppose that $p$ is such that $m + p = n + p$. Suppose that $m + p^+ = n + p^+$. Then
$$m + p^+ = (m + p)^+ = (n + p)^+ = n + p^+$$
thus we follow that $p \in I$ implies that  $p^+ \in I$. Thus $I$ is inductive, and therefore
$I = \omega$.

\subsection*{4.3.4}

\textit{Prove that for all $n \in \omega$, the inequality $0 \cdot n = 0$ holds.}

Suppose that $I$ is the set such taht
$I = \{n \in \omega: 0 \cdot n = 0\}$.
Since $0 \cdot 0 = 0$, we follow that $0 \in I$. Suppose that $0 \cdot n = 0$. Then we follow that
$$0 \cdot n^+ = 0 \cdot n + 0 = 0 + 0 = 0$$
Thus $n \in I \to n^+ \in I$. Thus $I = \omega$, as desired.


\subsection*{4.3.5}

\textit{Prove that for all $m \in \omega$ and $n \in \omega$, we have that
  $m^+ \cdot n  = m \cdot n + n$}

Suppose that
$$I = \{n \in \omega: (\forall m \in \omega)(m^+ \cdot n  = (m \cdot n) + n)\}$$

We follow from $M1$ that $m^+ \cdot 0 = 0$  and $(m \cdot 0) + 0 = m \cdot 0 = 0$
from $A1$ and $M1$. Thus $0 \in I$

Suppose that $n \in I$. Then we follow that
$$m^+ \cdot n^+ = m^+ \cdot n + m^+ = m \cdot n + n + m^+ = m \cdot n + n + m + 1 = $$
$$= m \cdot n + m + n + 1  = m \cdot n + m + n^+ = m \cdot n^+ + n^+$$
where theorems used are
$$M2 \ra IH \ra 4.3.4 \ra 4.3.4 \ra  4.3.10 \ra 4.3.4 \ra M2$$

\subsection*{4.3.6}

\textit{Let $n \in \omega$. Prove theorem 4.3.13}

Let
$$I = \{n \in \omega: n \cdot m = m \cdot n\}$$
We follow that $0 \cdot n = n \cdot 0 = 0$ from the previous exercises/theorems in the
chapter. Thus $0 \in I$.

Suppose that $n \cdot m = m \cdot n$. Then for $n^+$ we've got that
$$n^+ \cdot m = n \cdot m + m = m \cdot n + m = m \cdot n^+$$
where we've used previous exercise, IH, and $M2$ respectively. Thus we follow that
$n \in I \ra n^+ \in I$, thus $I = \omega$, as desired.

\subsection*{4.3.7}

\textit{Prove that if $n \in \omega$, then $n$ is either even or odd (details of definition
  of evenness in the exercise)}

Let
$$I: \{n \in \omega: (\exists k \in \omega: n = 2 \cdot k + 1 \lor n = 2 \cdot k\}$$

Since $0 = 2 \cdot 0$, we follow that $0 \in I$.

Suppose that $n \in I$. Then we follow that $n$ is even or odd (or both, we don't know yet).

If $n$ is even, then we follow that $n = 2 \cdot k$. Thus $n^+ = n + 1 = 2 \cdot k + 1$. Thus
$n^+$ is odd, therefore $n^+ \in I$.

If $n$ is odd, then we follow that $n = 2 \cdot k + 1$. Thus $n^+ = n + 1 = 2 \cdot k + 1 + 1 =
2 \cdot k + 2 = 2 \cdot k^+$. Thus we follow that $n^+ \in I$.

Thus we follow that $n \in I \ra n^+ \in I$, thus $I = \omega$, as desired.

\subsection*{4.3.8}

\textit{Let $I = \{n \in \omega: \neg((\exists k, j \in \omega)(n = 2 \cdot k \land
  n = 2 \cdot j + 1)\}$
  Prove that $I$ is inductive. Conclude that no natural number is both odd and even.}

Let
$I = \{n \in \omega: \neg((\exists k, j \in \omega)(n = 2 \cdot k \land n = 2 \cdot j + 1)\}$

We've shown earlier that $0$ is even. Suppose that $0$ is odd. Then we follow that
$$0 = 2 \cdot k + 1 = (2 \cdot k)^+$$
which is a contradiction.

Suppose that $n \in I$. Then we follow that
$$\neg((\exists k, j \in \omega)(n = 2 \cdot k \land n = 2 \cdot j + 1)$$
thus
$$\neg((\exists k, j \in \omega)(n^+ = (2 \cdot k)^+ \land n^+ = (2 \cdot j + 1)^+) \lra$$
$$\lra \neg((\exists k, j \in \omega)(n^+ = 2 \cdot k + 1 \land n^+ = 2 \cdot j + 2))\lra $$
$$\lra \neg((\exists k, j \in \omega)(n^+ = 2 \cdot k + 1 \land n^+ = 2 \cdot j^+))\lra$$
$$\lra \neg((\exists k, l \in \omega)(n^+ = 2 \cdot k + 1 \land n^+ = 2 \cdot l))$$
(i've changed a variable undere the quantifier for avoid confusion, which does not change the
meaning)
thus $n \in I \ra n^+ \in I$. Thus $I$ is inductive, and $I = \omega$. Therefore
every natural number is either odd or even, but not both.


\subsection*{4.3.9}

\textit{Prove for all $m, n, k \in \omega$ that $m^{n + k} = m^n \cdot m^k$}'

Let
$$I = \{k \in \omega: (\forall m, n \in \omega)(m^{n + k} = m^n \cdot m^k)\}$$

We follow that if $k = 0$ then
$$E(m, n + k) = E(m, n + 0) = E(m, n) = E(m, n) \cdot 1 = E(m, n) \cdot E(m, k)$$
thus we follow that $0 \in I$.

Suppose that $k \in I$. We follow that for $k^+$ we've got
$$E(m, n + k^+) = E(m, n + k) \cdot m = E(m, n) \cdot E(m, k) \cdot m =
E(m, n) \cdot E(m, k^+)$$
thus we follow that $k \in I \ra k^+ \in I$. Thus we follow that $I = \omega$, as desired.


\section{Order on $\omega$}

\subsection*{Note}

I'm using the symbol $\leq$ instead of "in or equal" because it's more convinient.
Theorems in the chapter provide us with the equivalence of two, and therefore the justification
for this notation.

\subsection*{4.4.1}

\textit{Let $n \in \omega$. Show that $1 \leq n^+$}

We know that for every $n \in \omega$, $0 \leq n$.
Thus $0 \leq n \ra 0^+ \leq n^+ \lra 1 \leq n^+$, as desired.

\subsection*{4.4.2}

\textit{Let $m, n \in \omega$. Show that if $m \in n^+$, then $m \leq n$.}

Suppose that $m \in n^+$. We follow that
$$m \in n^+ \lra m \in n \cup \{n\} \lra m \in n \lor m = n \lra m \leq n$$


\subsection*{4.4.3}

\textit{Let $n, a \in \omega$. Show that if $n \in a$, then $n^+ \leq a$}

Suppose that $n \in a$. We know that one of  $n^+ = a$, $n^+ \in a$ or $a \in n^+$ holds.
Suppose that $a \in n^+$. We follow that
$$a \in n \cup \{n\} \lra a \in n \lor a = n$$
If $a \in n$, then we've got that $n \in a$ and $a \in n$, which is impossible. If $a = n$,
then $n \in n$, which is also not good. Thus we follow that $a \in n^+$ is impossible.
Thus we follow that $n^+ \in a$ or $n^+ = a$. In other words, $n \leq a$, as desired.


\subsection*{4.4.4}

\textit{Let $I$ be inductive and let $a \in \omega$. Prove that
  $\{n \in I: n \in a \lor a \subseteq n\}$ is inductive }

Let
$$ Q = \{n \in I: n \in a \lor a \subseteq n\}$$
We follow that $0 \in Q$ since $0$ is in every inductive set (by virtue of being a natural number),
and by trichotomy property $0 \in a$ or $a \leq 0 \lra a \subseteq 0$.

Suppose that $q \in Q$. Since $Q \subseteq I$, we follow that $q \in I$, and
because $I$ is inductive, we follow that $q^+ \in I$ as well.
We also follow that $q \in a \lor a \subseteq q$.

If $a \subseteq q$, then we follow that $a \subseteq q \cup \{q\}$, therefore $a \subseteq q^+$,
and thus $q \in Q$.

Suppose that $q \in a$. Then we follow that since $a \in \omega \to a \subseteq \omega$,
we follow that $q \in \omega$. Therefore $q^+ \in \omega$, and
we've got by trichotomy that either $q^+ \in a$, $q^+ = a$ or $a \in q^+ \lra a \subset q^+$.
Thus we follow that $q^+ \in a \lor a \subseteq q^+$, therefore $q^+ \in Q$.

Thus we can conclude that $0 = \emptyset \in Q$ and $q \in Q \ra q^+ \in Q$. Thus $Q$ is
inductive, as desired.


\subsection*{4.4.5}

\textit{Let $m, n \in \omega$. Suppose that $m \leq n$. Prove that $\max(m, n) = n$.}

From the chapter we follow that $\max(m, n) = m \cup n$. Since $m \leq n$, we follow that
$m \subseteq n$. Thus $\max(m, n) = m \cup n = n$, as desired.

\subsection*{4.4.6}

\textit{Let $m, n \in \omega$. Prove that for all $p \in \omega$ if $m \in n$, then
  $m + p \in n + p$}

Let
$$I = \{p \in \omega: (\forall m, n \in \omega)(m \in n \ra m + p \in n + p\}$$
Since $m + 0 = m$ and $n = n + 0$, we follow that $m \in n \lra m + 0 \in n + 0$.

Suppose that $p \in I$. Then we follow that
$$m \in n \lra m + p \in n + p \lra (m + p)^+ \in (n + p)^+ \lra m + p^+ \in n + p^+$$
thus $p^+ \in I$. Thus $I = \omega$, as desired.

\subsection*{4.4.7}

\textit{Let $m, n \in \omega$. Prove that for all $p \in \omega$, if $m + p \in n + p$, then
  $m \in n$}

In previous exercise we've got only biconditionals, thus we get the reverse case as well

\subsection*{4.4.8, 4.4.9}

\textit{let $m, n \in \omega$. Prove that for all $p \in \omega$ $m \in n \lra
  m \cdot p^+ \in n \cdot p^+$}

Let
$$I = \{p \in \omega: (\forall m, n \in \omega)( m \in n \lra  m \cdot p^+ \in n \cdot p^+)\}$$

We first follow that if $p = 0$, then
$$m \cdot 0^+ \in n \cdot 0^+ \lra (m \cdot 0 + m) \in (n \cdot 0 + n) \lra (0 + m) \in (0 + n)
\lra m \in n$$
thus $0 \in I$.

Suppose that $p \in I$. Then we follow that
$$m \in n \lra m \cdot p^+ \in n \cdot p^+$$

$m \in n$ if and only if $m + a \in n + a$ by previous exercise. Thus
$$m \in n \lra m + m \cdot p^+ \in n + m \cdot p^+$$
and also
$$ m \cdot p^+ \in n \cdot p^+ \lra  m \cdot p^+ + n \in n \cdot p^+  + n$$
by commutativity we follow that
$$ m \cdot p^+ \in n \cdot p^+ \lra  n + m \cdot p^+ \in n \cdot p^+  + n$$
thus
$$m \in n \lra (m + m \cdot p^+ \in n + m \cdot p^+) \land (n + m \cdot p^+ \in n \cdot p^+  + n)
\lra$$
$$\lra (m + m \cdot p^+ \in n + m \cdot p^+) \land
(n + m \cdot p^+ \subset n \cdot p^+  + n) \ra$$
$$ m + m \cdot p^+ \in n \cdot p^+  + n \lra m \cdot (p^+)^+ \in n \cdot (p^+)^+$$
thus $m \in n \ra m \cdot (p^+)^+ \in n \cdot (p^+)^+$.

Let $m, n \in \omega$ be such that
$m \cdot (p^+)^+ \in n \cdot (p^+)^+$. By trichotomy we
follow that $m \in n$, $n \in m$ or $m = n$.

If $m = n$, then we follow that
$$n \cdot (p^+)^+ = m \cdot (p^+)^+ \in n \cdot (p^+)^+$$
which is a contradiction.

If $n \in m$, then we follow by implication that 
$$n \cdot (p^+)^+ \in m \cdot (p^+)^+$$
and since $m \cdot (p^+)^+ \in n \cdot (p^+)^+$, we follow that
$n \cdot (p^+)^+ \in n \cdot (p^+)^+$, which is also bad. Thus we follow that
$m \cdot (p^+)^+ \in n \cdot (p^+)^+ \ra m \in n$. By including the implication we follow that
$$m \cdot (p^+)^+ \in n \cdot (p^+)^+ \lra m \in n$$
and therefore $p^+ \in I$. Thus $I = \omega$, as desired.

\subsection*{4.4.10}

\textit{Use Theorem 4.4.11 and Theorem 4.4.9 to prove Corollary 4.4.12}

Suppose that $m + p = n + p$. If $m \in n$, then we follow that $m + p \in n + p$, which
is not the case.
If $n \in m$, then we follow that $n + p \in m + p$, which is also not the casee.
Thus we follow that $m = p$. Thus $m + p = n + p \lra m = n$ (reverse case is trivial)

Suppose that $p \neq 0$ and we've got that $m \cdot p = n \cdot p$. Since $p \neq 0$, we
follow that $p = j^+$ for some $j \in \omega$.

We follow that $m \in n$ and $n \in m$ are both impossible, because they would imply
that $m \cdot p \in n \cdot p$ and $n \cdot p \in m \cdot p$, which are not the case. Thus
we follow that
$$(\forall p \in (\omega \setminus 0))(m \cdot p = n \cdot p \lra m = n)$$
where reverse case is once again trivial.

\subsection*{4.4.11}

\textit{Let $m \in \omega$. Prove that $m \in m + p^+$ for all $p \in \omega$.}

We know that $m \in m^+$ for all $m \in \omega$. Thus we follow that for any $p \in \omega$
we've got that $m + p \in m^+ + p \lra m + p \in m + p^+$. We also can follow that
for every $p \in \omega$ we've got that $m \leq m + p$ (can't find the exact reference,
but it is easily followed by induction). Thus by transitive property we can follow
that  $m \in m + p^+$, as desired.

\subsection*{4.4.12}

\textit{Let $m \in \omega$. Prove that for all $n \in \omega$, if $m \in n$ then
  $m + p^+ = n$ for some $p \in \omega$.}

Let
$$I = \{j \in \omega: j \in m \lor (\exists p \in \omega)(j = m + p^+) \}$$
We follow that $0 \in m$, therefore $0 \in I$.

Suppose that $n \in I$. Then we follow that $n \in m \lor (\exists p \in \omega) (j = m + p^+)$.
If $n \in m$, then we follow that $n^+ \in m \lor n^+ = m \lra
n^+ \in m \lor n^+ = m + 0 $. Thus we follow that if $n \in m$, then $n^+ \in I$.

If $(\exists p \in \omega)(n = m + p^+)$, then we follow that $n^+ = (m + p^+)^+ = m + (p^+)^+$.
Since $p^+ \in \omega$, we follow $n^+ \in I$ as well.

Thus we can conclude that $n \in I \ra n^+ \in I$. Thus $I$ is inductive, and since it is
a subset of $\omega$, we follow that $I = \omega$. Thus we follow that our initial $n \in I$.
Thus $n \in m \lor (\exists p \in \omega)(n = m + p^+)$. Because $m \in n$, we follow
by trichotomy that $n \notin m$, thus $ (\exists p \in \omega)(n = m + p^+)$, as desired.

\subsection*{4.4.13}

Combination of previous two exercises into one biconditional.

\subsection*{4.4.18}

\textit{Prove for all $n \in \omega$, if $f: n \to \omega$, then $\ran(f) \subseteq k$ for
  some $k \in \omega$.}


Let
$$I = \{n \in \omega: (\forall f)
(\exists k \in \omega)(f: n \to \omega \ra \ran(f) \subseteq k)\}$$

We follow that if $f: 0 \to \omega$, then $\dom(f) = \emptyset$, thus $f = \emptyset$, therefore
$\ran(f) = \emptyset = 0 \subseteq 0$. Thus $0 \in I$.

Suppose that $n \in I$. Let $f: n^+ \to \omega$. Then we follow that
there exists $f|n: n \to \omega$, and thus $(\exists k \in \omega)(\ran(f|n) \subseteq k)$.
Since $n^+ = n \cup \{n\}$, we follow that
$$\ran(f) = f[n^+] = f[n] \cup f[\{n\}] = f[n] \cup f(n) \subseteq k \cup f(n)$$
Since $f: n^+ \to \omega$, we follow that $f(n) \in \omega$. Thus $k \cup f(n) \in \omega$,
thus there exists $j = k \cup f(n) \in \omega$ such that
$f: n^+ \to \omega \ra \ran(f) \subseteq j$. Thus we follow that $n^+ \in I$.
Thus $n \in I \ra n^+ \in I$, therefore $I = \omega$, as desired.


\chapter{On the Size of Sets}

\section{Finite Sets}

\subsection*{5.1.1}

\textit{In our proof of Theorem 5.1.3, under the case that $k \in m$, we did not show
that the function g is one-to-one. Complete the proof by proving that g is
one-to-one.}

We need to prove that
$$g(a) =
\begin{cases}
  k \text{ if } f(a) = m \\
  f(a) \text{ if } f(a) \neq m \\
\end{cases}
$$
is one-to-one.

Suppose that $a_1, a_2 \in A$ are such that $a_1 \neq a_2$.
Suppose that $f(a_1) \neq m$. Then we follow that if $f(a_2) \neq m$, then
we've got that $f(a_1) \neq f(a_2)$ by the fact that $f$ is injective, thus
$g(a_1) = f(a_1) \neq f(a_2) = g(a_2)$ as well.
If $f(a_2) = m$, then we follow that $g(a_2) = k$. Thus we follow that
$f(a_1) = g(a_1) \in \ran(f)$ and $g(a_2) = k \notin \ran(f)$. Thus we follow that
$g(a_1) \neq g(a_2)$.

Since $f$ is injective, we can follow that $a_1 \neq  a_2 \ra f(a_1) \neq f(a_2)$, thus
we follow that the case when $f(a_1) = f(a_2) = m$ can't happen. Thus we follow that
$a_1 \neq a_2 \ra g(a_1) \neq g(a_2)$, therefore $g$ is injective, as desired.

\subsection*{5.1.2}

\textit{Suppose that the set $B$ is finite and $A \subseteq B$. Prove that $A$ is finite.}

We follow that there exists $n \in \omega$ $|B| = n$. Thus we follow that
there exists $f: B \to n$, that is injective. Since $f$ is injective and $A \subseteq B$,
we follow that $f|A$ is also one-to-one. Thus we follow that $A$ is finite, as desired.

\subsection*{5.1.3}

\textit{Let $A$ and $B$ be two finite sets, and let $f: A \to B$ be one-to-one. Prove
  that $|A| \leq |B|$. }

Given that $A$ and $B$ are finite, we follow that there exist natural numbers $n = |A|$ and
$m = |B|$. We can also follow that there exist bijections $h: A \to n$ and $g: B \to m$.
Because $h$ is a bijection, we follow that $h \inv: n \to A$ is also a bijection.
Thus we follow that $g \circ f \circ h \inv: n \to m $ is a composition of injective functions,
and therefore it's injective as well. Thus we follow that $|n| \leq m$, therefore
$$||A|| \leq |B|$$
$$|A| \leq |B|$$
as desired.

\subsection*{5.1.4}

\textit{Prove theorem 5.1.7 }

Let $A$ and $B$ be finite. Let $n = |A|$ and $m = |B|$. We follow that there exist bijections
$f: A \to n$ and $g: B \to m$.

Thus we follow that if there exists a bijection $h: A \to B$,
then we follow that $g \circ h \circ f \inv: n \to m$ is a bijection. Thus
$g \circ h \circ f \inv: n \to m$ is injective and
$(g \circ h \circ f \inv: n \to m)\inv$ is injective, thus we follow that $n \geq m$ and
$m \geq n$, thus by antisymmetry we've got that $n = m$ and $|A| = |B|$.

If $|A| = |B|$, then we follow that $f: A \to |A|$ and $g: B \to |B|$ are bijections,
and thus $g \inv \circ f: A \to B$ is also a bijection.

\subsection*{5.1.5}

\textit{Let $n \in \omega$ and let $f: n \to A$. Prove that $f[n]$ is finite.}

Let us firstly prove a theorem (which I'm sure that it was proven in the book before,
but I can't find it) assuming for a moment that $n$ is a general set.

\textbf{If $f: n \to A$ is a function, then there exists an injective function $g$
such that $g \subseteq f\inv$.}

Suppose that $y \in f[n]$. Then we follow that there exists a nonempty set
$$A_y = \{x \in n : \eangle{x, y} \in f\}$$
Thus we can define an indexed function $\eangle{A_y: y \in f[n]}$, from which by axiom of
choice we follow that there exists a function $g$ such that 
$$g(y) \in A_y$$
Thus we follow that $g: f[n] \to n$. From a definition of $A_y$ we can follow that
$g \subseteq f\inv$. Suppose that $n_1 \neq n_2 \in f[n]$. We follow that because $f$ is a
function, there does not exist $x \in n$ such that $f(x) = n_1$ and $f(x) = n_2$. Thus
$g(n_1) \neq g(n_2)$.
Thus we follow that $n_1 \neq n_2 \ra g(n_1) \neq g(n_2)$. Therefore $g$ is injective, as desired.

By provided theorem we can follow that there exists an injective function $g$ from $f[n]$ to $n$.
Thus we follow that $f[n]$ is finite by definition.

\subsection*{5.1.6}

\textit{Let $A$ be finite and let $f: A \to \omega$. Prove that $\ran(f) \subseteq k$, for
  some $k \in \omega$.}


Since $A$ is finite, we follow that there exists a bijection $g: A \to n$
for some $n \in \omega$. Thus we follow that $g \inv: n \to A$ is also a bijection.
We now follow that $f \circ g\inv: n \to \omega $ is a function with a finite domain, and
because $g\inv$ is a bijection, we follow that $\ran(f \circ g\inv) = ran(f)$.
By exercise 4.4.18 we follow that there exists $k \in \omega$ such that
$\ran(f \circ g\inv) \subseteq k$. Thus we follow that $\ran(f) \subseteq k$ for some
$k \in \omega$, as desired.

\subsection*{5.1.7}

\textit{Suppose that $f: A \to n$ is a bijection, where $n \in \omega$. Ptove that $|A| = n$.}

Since $f$ is a bijection, we follow that it is injective. Thus we follow that $A$ is finite.
Let $|A| = m$. We follow that there exists a bijection $g: A \to m$ and thus $g \inv: m \to A$
is also a bijection. Thus
$$f \circ g \inv: m \to n$$
is also a bijection. Thus we follow that $|n| = |m|$. Therefore $n = m$, and thus $|A| = n$,
as desired.

\end{document}
%%% Local Variables:
%%% mode: latex
%%% TeX-master: t
%%% End:
