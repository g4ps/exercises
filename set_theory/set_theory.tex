\documentclass[11pt,oneside,titlepage]{book}
\title{My set theory exercises}
\usepackage{amsmath, amssymb}
\usepackage{geometry}
\usepackage{hyperref}
\author{Evgeny Markin}
\date{2023}

\DeclareMathOperator \map {\mathcal {L}}
\DeclareMathOperator \ns {null}
\DeclareMathOperator \range {range}
\DeclareMathOperator \inv {^{-1}}
\DeclareMathOperator \Span {span}
\DeclareMathOperator \lra {\Leftrightarrow}
\DeclareMathOperator \eqv {\Leftrightarrow}
\DeclareMathOperator \la {\Leftarrow}
\DeclareMathOperator \ra {\Rightarrow}
\DeclareMathOperator \imp {\Rightarrow}
\newcommand{\eangle}[1]{\langle #1 \rangle}


\begin{document}
\maketitle
\tableofcontents

\chapter{Introduction}

\section{Elementary Set Theory}

Let A, B, C be sets

\subsection{}

\textit{If $a \notin A \setminus B$ and $a \in A$, show that $a \in B$}

Because $a \notin A \setminus B$, we follow that $x \in B$ or $x \notin A$. Since $x \in A$, we
follow that $x \in B$, as desired.

\subsection{}

\textit{Show that if $A \subseteq B$, then $C \setminus B \subseteq C \setminus A$}

Let $c \in C \setminus B$. Then we follow that $c \in C$ or $c \notin B$. Since $A \subseteq B$,
we follow that $c \notin B$ implies that $c \notin A$. Thus we follow that
$c \in C \setminus B$ implies that $c \in C \setminus A$. Therefore $C \setminus B \subseteq
C \setminus A$.

\subsection{}

\textit{Suppose $A \setminus B \subseteq C$. Show that $A \setminus C \subseteq B$.}

Suppose that $a \in A \setminus C$. Then we follow that $a \in A$ and $a \notin C$.

Given that $A \setminus B \subseteq C$ and $A \notin C$, we follow that $a \notin A \setminus B$.
Thus $a \notin A$ or $a \in B$. Since $a \in A$, we follow that $a \in B$. Thus
$$a \in A \setminus C \to a \in B$$
$$A \setminus C \subseteq  B$$
as desired.

\subsection{}

\textit{Suppose $A \subseteq B$ and $A \subseteq C$. Show that $A \subseteq B \cap C$}

Suppose that $a \in A$. Then we follow that $a \in B$ and $a \in C$. Thus $a \in B \cap C$.
Therefore we follow that $A \subseteq B \cap C$.

\subsection{}

\textit{Suppose $A \subseteq B$ and $B \cap C = \emptyset$. Show that $A \in B \setminus C$}

Suppose that $a \in A$. Then we follow that $a \in B$ and since $B \cap C = \emptyset$, we
follow that $a \notin C$. Thus $a \in B \setminus C$ by definition. Therefore
$A \subseteq B \setminus C$.

\subsection{}

\textit{Show that $A \setminus (B \setminus C) \subseteq (A \setminus B) \cup C$.}
Suppose that $a \in A \setminus (B \setminus C)$. Then we follow that
$a \in A$ and $a \notin B \setminus C$. Thus $a \notin B$ and $a \in C$. Thus we
follow that $a \in A \setminus B$ or $a \in C$. Thus
$A \setminus (B \setminus C) \subseteq (A \setminus B) \cup C$
as desired.

\subsection{}

\textit{Let $P(x)$ be the property $x > \frac 1 x$. Are the assertions $P(2)$, $P(-2)$,
  $P(\frac 1 2)$ $P( \frac{-1}{2})$ true or false .}

$$2 > \frac 1 2 \to P(2) = true$$
$$-2 < \frac{-1}{2} \to P(-2) = false$$
last two are reversed.

\subsection{}

\textit{Sow that each of the following sets can be expressed as an interval}

$$a) (-3, 3)$$
$$b) (-3, \infty)$$
$$c) (-3, 3)$$

all of them follow from order properties of real numbers.

\subsection{}

\textit{Express the following sets as truth sets}

$$A = \{1, 4, 9, 16, 25, ...\} \iff A = \{x \in N: x = y^2 \textit{ for some } y \in N\}$$
$$B = \{..., -15, -10, -5, 0, 5, ... \} \iff A = \{x \in N: x = 5y  \textit{ for some } y \in N\}$$

\textit{Rest are also trivial, not gonna go deep here}

\section{Logical Notation}

\subsection*{1.2.1}

\textit{Using truth tables, show that $\neg(P \ra Q) \lra (P \land \neg Q)$}

\begin{center}
  \begin{tabular}{c| c| c| c| c| c|}
    P & Q & $P \ra Q$ & $\neg(P \ra Q)$ & $\neg Q$ & $P \land \neg Q$ \\
    false & false & true & false & true & false \\
    false & true & true & false & false & false \\
    true & false & false & true & true & true \\
    true & true & true & false & false & false \\
  \end{tabular}  
\end{center}

from this we can see that they are equqivalent.

\textit{Following 4 exercises are the same as this one, so I'm skipping them}

\subsection*{1.2.5}

\textit{Show that $(P \imp Q) \land (P \imp R) \eqv P \imp (Q \land R)$, using logic laws}

$$(P \imp Q) \land (P \imp R) \eqv (\neg P \lor Q) \land (\neg P \lor R) \eqv
\neg P  \lor (R \land Q) \eqv  P  \imp (R \land Q) $$
Laws used: 
$$CL \to DIST \to CL$$

\subsection*{1.2.6}

\textit{Show that $(P \imp R) \lor (Q \imp R) \eqv (P \land Q) \imp R$, using logic laws}

$$(P \imp R) \lor (Q \imp R)  \eqv (\neg P \lor R) \lor (\neg Q \lor R) \eqv
\neg P \lor R \lor \neg Q \lor R \eqv (\neg Q \lor \neg P) \lor R \eqv$$
$$ \eqv \neg (Q \land P) \lor R
\eqv (Q \land R) \imp R$$
Laws used:
$$CL \to ASC \to ID, ASC \to DML \to CL$$

\subsection*{1.2.7}

\textit{Show that $P \imp (Q \imp R) \eqv (P \land Q) \imp R$, using logic laws}

$$P \imp (Q \imp R) \eqv \neg P \lor (Q \imp R) \eqv \neg P \lor (\neg Q \lor R) \eqv
(\neg P \lor \neg Q) \lor R \eqv \neg (P \land Q) \lor R \eqv (P \land Q) \imp R$$
Laws used:
$$CL \to CL \to ASC \to DML \to CL$$

\subsection*{1.2.8}

\textit{Show that $(P \imp Q) \imp R$ and $P \imp (Q \imp R)$ are not logically equivalent}

We're gonna show that $q \land w \eqv false$
$$((P \imp Q) \imp R) \land (P \imp (Q \imp R)) \eqv (\neg (\neg P \lor Q) \lor R) \land
(\neg P \lor (\neg Q \lor R)) \eqv $$
$$\eqv ((P \land \neg Q) \lor R) \land (\neg P \lor \neg Q \lor R) \eqv
((P \land Q) \land (\neg P \lor \neg Q)) \lor R  \eqv$$
$$ \eqv ((P \land Q) \land \neg ( P \land  Q)) \lor R  \eqv false \lor R \eqv false$$


\section{Predicates and Quantifiers}



\end{document}
%%% Local Variables:
%%% mode: latex
%%% TeX-master: t
%%% End:
