\documentclass[10pt,oneside,titlepage]{book}
\title{My linear algebra exercises}
\usepackage{amsmath,amssymb}
% \usepackage{geometry}
% \usepackage{pdfpages}
% \usepackage{tocloft}
\usepackage{hyperref}
\author{Evgeny Markin}
\date{2022}

\begin{document}
\maketitle
\tableofcontents

\chapter*{Preface}

Exercises are from "Linear algebra done right" by Sheldon Axler, 3rd ed.
I've already read this book before and completed some exercises from it.
Right now I want to brush up the material once again, put all the
proofs on a more durable material than paper and to prepare myself to
what's gonna happen afterwards.

\chapter{1.A $R^n$ and $C^n$}

\section*{1}
\textit{Suppose $a$ and $b$ are real numbers, not both $0$. Find real nuber
  $c$ and $d$ such that }
$$1/(a + bi) = c + di$$

$$\frac{1}{a + bi} = c + di$$
$$\frac{1}{a + bi} - c - di = 0$$
$$\frac{a - bi}{(a + bi)(a - bi)} = c + di$$
$$\frac{a - bi}{(a^2 + b^2)} = c + di$$
$$\frac{a}{a^2 + b^2} - \frac{b}{a^2 + b^2}i = c + di$$
Thus $c = \frac{a}{a^2 + b^2}$ and $d = -\frac{b}{a^2 + b^2}$

\section*{2}
\textit{Show that }
$$\frac{-1  + \sqrt{3}i}{2}$$
\textit{is a cube root of $1$ (meaning that its cube equals 1)}
$$(\frac{-1  + \sqrt{3}i}{2})^3 =
\frac{(-1  + \sqrt{3}i)^3}{8} =
\frac{(-1  + \sqrt{3}i)(-1  + \sqrt{3}i)^2}{8} =
\frac{(-1  + \sqrt{3}i)(1  - 2\sqrt{3}i - 3)}{8} =
$$
$$
=\frac{(-1  + \sqrt{3}i)(-2  - 2\sqrt{3}i)}{8} =
\frac{2 + 2\sqrt{3}i - 2\sqrt{3}i + 6}{8} =
\frac{8}{8} = 1
$$
as desired.

\section*{3}
\textit{Find two distinct square roots of $i$}

Square root of $i$, I assume, is a number, whose square is equal to $i$.
Suppose that $(a + bi)^2 = i$. It follows that
$$(a + bi)^2 = a^2 + 2abi - b^2$$
So if we set $$a = b = 1/\sqrt{2}$$ this equation holds. Also it holds for
$$a = b = -1/\sqrt{2}$$
maxima seems to agree with me on this one

\section*{4}
\textit{Show that $\alpha + \beta = \beta + \alpha$ for all
  $\alpha, \beta \in \textbf{C}$}

Let $\alpha = a_1 + b_1 i$ and $\beta = a_2 + b_2 i$. It follows
$$\alpha + \beta = a_1 + b_1 i + a_2 + b_2 i = a_2 + b_2 i + a_1 + b_1 i =
\beta  + \alpha$$
as desired.

\section*{5}
\textit{Show that $(\alpha + \beta) + \lambda = \alpha + (\beta + \lambda)$
  for all  $\alpha, \beta, \lambda \in \textbf{C}$}

Let $\alpha = a_1 + b_1 i$ , $\beta = a_2 + b_2 i$, $\lambda = a_3 + b_3 i$.
It follows that 
$$\alpha + (\beta + \lambda)  = a_1 + b_1 i + (a_2 + b_2 i + a_3 + b_3 i) =
(a_1 + b_1 i + a_2 + b_2 i) + a_3 + b_3 i = 
(\alpha + \beta) + \lambda$$

\section*{6}
\textit{Show that $(\alpha \beta) \lambda = \alpha( \beta \lambda)$}

$$\alpha + (\beta + \lambda)  = (a_1 + b_1 i) ((a_2 + b_2 i) + (a_3 + b_3 i)) =
((a_1 + b_1 i)(a_2 + b_2 i)) + (a_3 + b_3 i) = 
(\alpha \beta) \lambda$$

\section*{7}
\textit{Show that for every $\alpha \in \textbf{C}$ there exists a unique
  $\beta \in \textbf{C}$ such that $\alpha + \beta = 0$}

Suppose that there exist two different $\beta_1 \neq \beta_2$ such that
$\alpha + \beta_1 = 0$ and $\alpha + \beta_2 = 0$. It follows that
$$ \beta_1 = \beta_1 + 0 =  \beta_1 + \alpha + \beta_2 = \alpha + \beta_1  + \beta_2 = 0  + \beta_2 = \beta_2$$
which is a contradiction. Therefore there exists only one unique $\beta$.

\section*{8}
\textit{Show that for every $\alpha \in \textbf{C}$ with $\alpha \neq 0$
  there exists a unique $\beta \in \textbf{C}$ such that $\alpha \beta = 1$}

Suppose that it is not true and there exist two different
$\beta_1 \neq \beta_2$ such that
$$\alpha \beta_1 = 1 \textit{ and } \alpha \beta_2 = 1$$
it follows then that 
$$\beta_1 = 1 *  \beta_1 = \alpha \beta_2 \beta_1  =
\alpha \beta_1 \beta_2 = 1 * \beta_2 = \beta_2$$
which is a contradiction. Therefore there exists only one unique $\beta$.

\section*{etc}
The rest of the section is the repetition of this kind of stuff.
That is a lot of writing, and not a lot of thinking, so I'll skip it.
I don't ususally like to skip sections, but I have  aa feeling, that I've
completed this thing on paper somewhere, and there is not much reason to
rewrite it here.

\chapter{1.B Definition of Vector Space}

\section*{1}
\textit{Prove that $-(-v)= v$ for every $v \in V$.}

For $v$ there exists only one $-v$. For $-v$ there exists only one $-(-v))$.

Thus
$$v = v + 0 = v + (-v) + (-(-v)) = 0 + (-(-v)) = -(-v)$$
as desired (idk if it's true, I'm not good at axioms and stuff)

\section*{2}
\textit{Suppose $a \in F, v \in V$, and $av = 0$. Prove that
  $a = 0$ or $v = 0$.}

Suppose that $a \neq 0$, $v \neq 0$ but $av = 0$. It follows that there
exist $1/a$ - multiplicative inverse of $a$. It follows that
$$1/a * av = 1/a * 0$$
$$1v = 0$$
$$v = 0$$
which is a contradiction. Thus either $a = 0$ or $v = 0$.

\section*{3}
\textit{Suppose $v, w \in V$. Explain why there exists a unique $x \in V$
  such that $v + 3x = w$.}

Suppose that there exists $x_1 \neq x_2$ such that
$v + 3x_1 = w$ and $v + 3x_2 = w$. Thus
$$3x_1 = w - v = 3x_2$$
$$x_1 = \frac{1}{3}(w - v) = x_2$$
which is a contradiction.

Same can be stated from the fact that $x$ is a unique additive inverse of
$\frac{1}{3}(v - w)$.

\section*{4}
\textit{The empty set is not a vector space. The empty set fails to satisfy
  only one of the requirements listed in 1.19. Which one?}

Additive indentity. Empty set does not have zero element in it.
BTW $\{0\}$ is a vector space.

\section*{5}
\textit{Show that n the defintition of a vector space (1.19), the additive
  inverse condition can be replaced with the condition that}
$$0v = 0 \textit{ for all } v \in V$$
\textit{Here the 0 on the left side is the number 0, and the 0 on the right
  side is the additive identity of $V$.}

$$0v = 0$$
$$(1 - 1)v = 0$$
$$1v - 1v = 0$$
$$v - v= 0$$
$$v + (- v)= 0$$

\section*{6}
\textit{Let $\infty$ and $-\infty$ denote two distinct object, neither of
  which is in $R$. Define an addition and multiplication on
  $R \cup \{\infty\} \cup \{-\infty\}$ as you could guess from the notation.
  Specifically, the sum and the product of two real numbers is as usual,
  and for $t \in R$ define
}

$$
t\infty =
\begin{cases}
  -\infty \text{ if } t < 0 \\
  0 \text{ if } t = 0 \\
  \infty \text{ if } t > 0 \\
\end{cases}
$$

$$
t(-\infty) =
\begin{cases}
  \infty \text{ if } t < 0 \\
  0 \text{ if } t = 0 \\
  -\infty \text{ if } t > 0 \\
\end{cases}
$$

$$t + \infty = \infty + t = \infty$$
$$t + (-\infty) = (-\infty) + t = (-\infty)$$
$$\infty + \infty = \infty$$
$$(-\infty) + (-\infty) = (-\infty)$$
$$\infty + (-\infty) = 0$$

\textit{Is $R \cup \{\infty\} \cup \{-\infty\}$ a vector space over
  $R$? Explain.}

I don't think that it is.

$$(t + \infty) - \infty = \infty - \infty = 0$$
$$t + (\infty - \infty) = t + 0 = t$$
thus
$$t + (\infty - \infty) \neq (t + \infty) - \infty$$
thus 
$R \cup \{\infty\} \cup \{-\infty\}$ is not associative, therefore it is
not a vector space.

\chapter{1.C Subspaces}

\section*{1}
\textit{For each of the following subsets of $F^3$, determine whether it is a
  subspace of $F^3$:}

\textit{(a) $\{(x_1, x_2, x_3) \in F^3: x_1 + 2x_2 + 3x_3 = 0\}$}

Yes, it is. $0$ is contained within it.
$$(x_1, x_2, x_3) + (y_1, y_2, y_3) = (x_1 + y_1, x_2 + y_2, x_3 + y_3)$$
therefore
$$x_1 + y_1 + 2(x_2 + y_2) +  3(x_3 + y_3) =
x_1 + 2x_2 + 3x_3 + y_1 + 2y_2 + 3y_3 = 0+ 0 = 0$$
therefore it is closed under addition
$$n(x_1, x_2, x_3) = (nx_1, nx_2, nx_3)$$
$$nx_1 + 2nx_2 + 3nx_3 = n(x_1 + 2x_2 + 3x_3 ) = 0n = 0$$
therefore it is closed under multiplication.

\textit{(b) $\{(x_1, x_2, x_3) \in F^3: x_1 + 2x_2 + 3x_3 = 4\}$}

It's not a subspace, because it does not contain zero.


\textit{(c) $\{(x_1, x_2, x_3) \in F^3: x_1 x_2 x_3 = 0\}$}

It's not a subspace, because
$$(0, 1, 1) + (1, 0, 0) = (1, 1, 1)$$
therefore it's not closed under addition.

\textit{(d) $\{(x_1, x_2, x_3) \in F^3: x_1  = 5x_3\}$}

It's a subspace, proof is the same as in (a), can be seen more clearly when we
rewrite constraint as
$$x_1 = 5x_3 \to x_1 + 0x_2 -5x_3 = 0$$

\end{document}