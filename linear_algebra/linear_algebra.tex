\documentclass[10pt,oneside,titlepage]{book}
\title{My linear algebra exercises}
\usepackage{amsmath,amssymb}
% \usepackage{geometry}
% \usepackage{pdfpages}
% \usepackage{tocloft}
\usepackage{hyperref}
\author{Evgeny Markin}
\date{2022}

\begin{document}
\maketitle
\tableofcontents

\chapter*{Preface}

Exercises are from "Linear algebra done right" by Sheldon Axler, 3rd ed.
I've already read this book before and completed some exercises from it.
Right now I want to brush up the material once again, put all the
proofs on a more durable material than paper and to prepare myself to
what's gonna happen afterwards.

\chapter{Vector Spaces}
\section{$R^n$ and $C^n$}

\subsection{}
\textit{Suppose $a$ and $b$ are real numbers, not both $0$. Find real nuber
  $c$ and $d$ such that }
$$1/(a + bi) = c + di$$

$$\frac{1}{a + bi} = c + di$$
$$\frac{1}{a + bi} - c - di = 0$$
$$\frac{a - bi}{(a + bi)(a - bi)} = c + di$$
$$\frac{a - bi}{(a^2 + b^2)} = c + di$$
$$\frac{a}{a^2 + b^2} - \frac{b}{a^2 + b^2}i = c + di$$
Thus $c = \frac{a}{a^2 + b^2}$ and $d = -\frac{b}{a^2 + b^2}$

\subsection{}
\textit{Show that }
$$\frac{-1  + \sqrt{3}i}{2}$$
\textit{is a cube root of $1$ (meaning that its cube equals 1)}
$$(\frac{-1  + \sqrt{3}i}{2})^3 =
\frac{(-1  + \sqrt{3}i)^3}{8} =
\frac{(-1  + \sqrt{3}i)(-1  + \sqrt{3}i)^2}{8} =
\frac{(-1  + \sqrt{3}i)(1  - 2\sqrt{3}i - 3)}{8} =
$$
$$
=\frac{(-1  + \sqrt{3}i)(-2  - 2\sqrt{3}i)}{8} =
\frac{2 + 2\sqrt{3}i - 2\sqrt{3}i + 6}{8} =
\frac{8}{8} = 1
$$
as desired.

\subsection{}
\textit{Find two distinct square roots of $i$}

Square root of $i$, I assume, is a number, whose square is equal to $i$.
Suppose that $(a + bi)^2 = i$. It follows that
$$(a + bi)^2 = a^2 + 2abi - b^2$$
So if we set $$a = b = 1/\sqrt{2}$$ this equation holds. Also it holds for
$$a = b = -1/\sqrt{2}$$
maxima seems to agree with me on this one

\subsection{}
\textit{Show that $\alpha + \beta = \beta + \alpha$ for all
  $\alpha, \beta \in \textbf{C}$}

Let $\alpha = a_1 + b_1 i$ and $\beta = a_2 + b_2 i$. It follows
$$\alpha + \beta = a_1 + b_1 i + a_2 + b_2 i = a_2 + b_2 i + a_1 + b_1 i =
\beta  + \alpha$$
as desired.

\subsection{}
\textit{Show that $(\alpha + \beta) + \lambda = \alpha + (\beta + \lambda)$
  for all  $\alpha, \beta, \lambda \in \textbf{C}$}

Let $\alpha = a_1 + b_1 i$ , $\beta = a_2 + b_2 i$, $\lambda = a_3 + b_3 i$.
It follows that 
$$\alpha + (\beta + \lambda)  = a_1 + b_1 i + (a_2 + b_2 i + a_3 + b_3 i) =
(a_1 + b_1 i + a_2 + b_2 i) + a_3 + b_3 i = 
(\alpha + \beta) + \lambda$$

\subsection{}
\textit{Show that $(\alpha \beta) \lambda = \alpha( \beta \lambda)$}

$$\alpha + (\beta + \lambda)  = (a_1 + b_1 i) ((a_2 + b_2 i) + (a_3 + b_3 i)) =
((a_1 + b_1 i)(a_2 + b_2 i)) + (a_3 + b_3 i) = 
(\alpha \beta) \lambda$$

\subsection{}
\textit{Show that for every $\alpha \in \textbf{C}$ there exists a unique
  $\beta \in \textbf{C}$ such that $\alpha + \beta = 0$}

Suppose that there exist two different $\beta_1 \neq \beta_2$ such that
$\alpha + \beta_1 = 0$ and $\alpha + \beta_2 = 0$. It follows that
$$ \beta_1 = \beta_1 + 0 =  \beta_1 + \alpha + \beta_2 = \alpha + \beta_1  + \beta_2 = 0  + \beta_2 = \beta_2$$
which is a contradiction. Therefore there exists only one unique $\beta$.

\subsection{}
\textit{Show that for every $\alpha \in \textbf{C}$ with $\alpha \neq 0$
  there exists a unique $\beta \in \textbf{C}$ such that $\alpha \beta = 1$}

Suppose that it is not true and there exist two different
$\beta_1 \neq \beta_2$ such that
$$\alpha \beta_1 = 1 \textit{ and } \alpha \beta_2 = 1$$
it follows then that 
$$\beta_1 = 1 *  \beta_1 = \alpha \beta_2 \beta_1  =
\alpha \beta_1 \beta_2 = 1 * \beta_2 = \beta_2$$
which is a contradiction. Therefore there exists only one unique $\beta$.

\subsection{}
The rest of the section is the repetition of this kind of stuff.
That is a lot of writing, and not a lot of thinking, so I'll skip it.
I don't ususally like to skip sections, but I have  aa feeling, that I've
completed this thing on paper somewhere, and there is not much reason to
rewrite it here.

\section{Definition of Vector Space}

\subsection{}
\textit{Prove that $-(-v)= v$ for every $v \in V$.}

For $v$ there exists only one $-v$. For $-v$ there exists only one $-(-v))$.

Thus
$$v = v + 0 = v + (-v) + (-(-v)) = 0 + (-(-v)) = -(-v)$$
as desired (idk if it's true, I'm not good at axioms and stuff)

\subsection{}
\textit{Suppose $a \in F, v \in V$, and $av = 0$. Prove that
  $a = 0$ or $v = 0$.}

Suppose that $a \neq 0$, $v \neq 0$ but $av = 0$. It follows that there
exist $1/a$ - multiplicative inverse of $a$. It follows that
$$1/a * av = 1/a * 0$$
$$1v = 0$$
$$v = 0$$
which is a contradiction. Thus either $a = 0$ or $v = 0$.

\subsection{}
\textit{Suppose $v, w \in V$. Explain why there exists a unique $x \in V$
  such that $v + 3x = w$.}

Suppose that there exists $x_1 \neq x_2$ such that
$v + 3x_1 = w$ and $v + 3x_2 = w$. Thus
$$3x_1 = w - v = 3x_2$$
$$x_1 = \frac{1}{3}(w - v) = x_2$$
which is a contradiction.

Same can be stated from the fact that $x$ is a unique additive inverse of
$\frac{1}{3}(v - w)$.

\subsection{}
\textit{The empty set is not a vector space. The empty set fails to satisfy
  only one of the requirements listed in 1.19. Which one?}

Additive indentity. Empty set does not have zero element in it.
BTW $\{0\}$ is a vector space.

\subsection{}
\textit{Show that n the defintition of a vector space (1.19), the additive
  inverse condition can be replaced with the condition that}
$$0v = 0 \textit{ for all } v \in V$$
\textit{Here the 0 on the left side is the number 0, and the 0 on the right
  side is the additive identity of $V$.}

$$0v = 0$$
$$(1 - 1)v = 0$$
$$1v - 1v = 0$$
$$v - v= 0$$
$$v + (- v)= 0$$

\subsection{}
\textit{Let $\infty$ and $-\infty$ denote two distinct object, neither of
  which is in $R$. Define an addition and multiplication on
  $R \cup \{\infty\} \cup \{-\infty\}$ as you could guess from the notation.
  Specifically, the sum and the product of two real numbers is as usual,
  and for $t \in R$ define
}

$$
t\infty =
\begin{cases}
  -\infty \text{ if } t < 0 \\
  0 \text{ if } t = 0 \\
  \infty \text{ if } t > 0 \\
\end{cases}
$$

$$
t(-\infty) =
\begin{cases}
  \infty \text{ if } t < 0 \\
  0 \text{ if } t = 0 \\
  -\infty \text{ if } t > 0 \\
\end{cases}
$$

$$t + \infty = \infty + t = \infty$$
$$t + (-\infty) = (-\infty) + t = (-\infty)$$
$$\infty + \infty = \infty$$
$$(-\infty) + (-\infty) = (-\infty)$$
$$\infty + (-\infty) = 0$$

\textit{Is $R \cup \{\infty\} \cup \{-\infty\}$ a vector space over
  $R$? Explain.}

I don't think that it is.

$$(t + \infty) - \infty = \infty - \infty = 0$$
$$t + (\infty - \infty) = t + 0 = t$$
thus
$$t + (\infty - \infty) \neq (t + \infty) - \infty$$
thus 
$R \cup \{\infty\} \cup \{-\infty\}$ is not associative, therefore it is
not a vector space.

\section{Subspaces}

\subsection{}
\textit{For each of the following subsets of $F^3$, determine whether it is a
  subspace of $F^3$:}

\textit{(a) $\{(x_1, x_2, x_3) \in F^3: x_1 + 2x_2 + 3x_3 = 0\}$}

Yes, it is. $0$ is contained within it.
$$(x_1, x_2, x_3) + (y_1, y_2, y_3) = (x_1 + y_1, x_2 + y_2, x_3 + y_3)$$
therefore
$$x_1 + y_1 + 2(x_2 + y_2) +  3(x_3 + y_3) =
x_1 + 2x_2 + 3x_3 + y_1 + 2y_2 + 3y_3 = 0+ 0 = 0$$
therefore it is closed under addition
$$n(x_1, x_2, x_3) = (nx_1, nx_2, nx_3)$$
$$nx_1 + 2nx_2 + 3nx_3 = n(x_1 + 2x_2 + 3x_3 ) = 0n = 0$$
therefore it is closed under multiplication.

\textit{(b) $\{(x_1, x_2, x_3) \in F^3: x_1 + 2x_2 + 3x_3 = 4\}$}

It's not a subspace, because it does not contain zero.


\textit{(c) $\{(x_1, x_2, x_3) \in F^3: x_1 x_2 x_3 = 0\}$}

It's not a subspace, because
$$(0, 1, 1) + (1, 0, 0) = (1, 1, 1)$$
therefore it's not closed under addition.

\textit{(d) $\{(x_1, x_2, x_3) \in F^3: x_1  = 5x_3\}$}

It's a subspace, proof is the same as in (a), can be seen more clearly when we
rewrite constraint as
$$x_1 = 5x_3 \to x_1 + 0x_2 -5x_3 = 0$$

\subsection{}
\textit{Verify all the assertions in Example 1.35}

\textit{(a) if $b \in F$, then}
$$\{(x_1, x_2, x_3, x_4) \in F^4: x_3 = 5x_4 + b\}$$
\textit{is a subspace of $F^4$ if and only if $b = 0$}

If $b \neq 0$, then $0$ is not an element of this set.

Proving that it's a subspace when $b = 0$ is trivial

\textit{(b) The set of continous real-valued functions on the interval $[0, 1]$
  is a subspace of $R^{[0, 1]}$.}

$(kf) = kf$ by algebraic properties of continous functions.
If $f$ and $g$ are continous, then $(f + g)$ is continous as well by the same
property.
$f(x) = 0$ is continous because it's a constant functions.

By the way, same (probably) applies to a set of uniformly continous functions.

\textit{(c) The set of differentiable real-valued functions on $R$ is a
  subspace of $R^R$.}

Same deal, algebraic proerties imply linearity, adn zero is included.

\textit{(d) The set of differentiable real-valued functions $f$ on the
  interval $(0, 3)$ such that $f'(2) = b$ is a subspace of $R^{(0, 3)}$
  if and only if $b = 0$.}

Same deal as in previous one, $f'(2)$ needs to be equal to zero in order
to include zero. Previous part does not include it, because it does not
have specific restrictions on derivatives being particular values at particular places.

\textit{(e) The set of all sequences of complex numbers with limit $0$ is a
  subspace of $C^{\infty}$.}

Here we can take zero to be $(x_n) = 0$. Linearity is implied by aldebraic
properties of limits of sequences.

\subsection{}
\textit{Show that the set of differentiable real-valued functions $f$ on the
  interval $(-4, 4)$ such that $f'(1) = 3f(2)$ is a subspace of $R^{[-4, 4]}$.}

Zero is included here. Suppose that $f$ and $g$ are functions in given set.
It follows that
$$f'(1) + g'(1) = 3f(2) + 3g(2)$$
$$f'(1) + g'(1) = 3(f(2) + g(2))$$
$$(f + g)'(1) = 3(f + g)(2)$$
thus it's closed under addition.

$$(kf)'(1) = 3(kf)(2)$$
implies
$$kf'(1) = 3kf(2)$$
therefore it's closed under multiplication by scalar.
Therefore we can state that given subset is a vector subspace.

\subsection{}
analogous to previous

\subsection{}
\textit{Is $R^2$ a subspace of the complex vector space $C^2$?}

No, it's not closed under scalar multiplication.

\subsection{}
\textit{(a) Is }
$$\{(a, b, c) \in R^3: a^3 = b^3\}$$
\textit{a subspace if $R^3$?}

Yes. it is. $a^3 = b^3 \to a = b \to a - b = 0$, the rest of proof is trivial.

\textit{(b) Is }
$$\{(a, b, c) \in C^3: a^3 = b^3\}$$
\textit{a subspace if $C^3$?}

I want to say no to this one, example is
$$(1/2 + i\frac{\sqrt{3}}{2}, -1, 0) +
(1/2 - i\frac{\sqrt{3}}{2}, -1, 0) =
(1, -1, 0)$$
thus it's not closed under additon.

\subsection{}
\textit{Give an example of a nonemplty subset $U$ of $R^2$ such that $U$ is
  closed under addition and under additive inverses (meaning $-u \in U$
  whenever $u \in U$), but $U$ is not a subspace of $R^2$}

$Q^2$. On the other thouhgh, $Z$ will do as well.

\subsection{}
\textit{Give an example of a nonempty subset $U$ of $R^2$ such that $U$ is
  closed under scalar multiplication, but $U$ is not a subspace of $R^2$.}

Two lines through origin.

\subsection{}
\textit{A function is called periodic if there exists a positive number
  $p$ such that $f(x) = f(x + p)$ for all $x \in R$. Is the set of
  periodic functions from $R$ to $R$ a subspace of $R^R$? Explain. }

Zero is a periodic function. Set is
certainly closed under scalar multiplication.

Suppose that $f$ and $g$ are both periodic and $f$ has a period of $p1$
and $g$ has a period of $p2$. Thus if $p2/p1 \in I$,
then functions will be constantly out of phase, therefore the set is not
closed under addition. Thus this subset is not a subspace.

\subsection{}
\textit{Suppose $U_1$ and $U_2$ are subspaces of $V$. Prove that the
  intersection $U_1 \cap U_2$ is a subspace of $V$.}

Zero is included in any subspace, therefore zero is included.

Suppose that $u_1, u_2 \in U_1 \cap U_2$. It follows that for $z \in F$
$zu_1 \in U_1$ and $zu_1 \in U_2$ by closure of those two subspaces.
Therefore $zu_1 \in U_1 \cap U_2$ for any scalar, thus the set is
closed under scalar multiplication.

$u_1 + u_2 \in U_1$ and $u_1 + u_2 \in U_2$ by closure under addition for
both subspaces. Thus $u_1 + u_2 \in U_1 \cap U_2$ for any such vectors.
Therefore the set is closed under addition.

Thus the set satisfies all requirements to be a subspace. Therefore it is a
subspace.

\subsection{}
\textit{Prove that the interseection of every collection of subspace of $V$ is
  a subspace of $V$}

Intersection of two subspaces  is a subspace. Therefore by induction
intersection of any finite collection of subspaces is a subspace.

Suppose that $\Lambda$ is an arbitrary collection of subspaces.
Every subspace contains a zero element, therefore
$$0 \in \cap \Lambda$$

Any vector in $\cap \Lambda$ will be closed under scalar multiplication
for every $U \in \Lambda$. Thus, it will be contained in every
$U \in \Lambda$. Therefore it is contained in $\cap \Lambda$.


Any two  vectors in $\cap \Lambda$ will be closed under addition,
for every $U \in \Lambda$. Thus, their sum  will be contained in every
$U \in \Lambda$. Therefore it is contained in $\cap \Lambda$.

Thus $\cap \Lambda$ is a vector space.

\subsection{}
\textit{Prove that the union of two subspaces of $V$ is a subspace of $V$
  if and only if one of the subspaces is contained in the other.}

Suppose that a union of two subspaces $U_1 \cup U_2$
is a subspace of $V$.

Zero is included in eveery subspace, so in case of the union we don't worry
about it.
Scalar multiplication is also trivial, as we are working only with one vector.

Now for the interesting part: addition. Let $u_1, u_2 \in U_1 \cup U_2$.
In case when $u_1, u_2$ are contained only in one subspace we've got
a trivial case. Interesting part comes when $u_1 \in U_1$ and $u_2 \in U_2$.

What we want to prove is that it is impossible to have
$u_1 \in U_1 \setminus U_2$ and $u_2 \in U_2 \setminus U_1$ and we're going to
use contradiction. Suppose that
$u_1 \in U_1 \setminus U_2$, $u_2 \in U_2 \setminus U_1$ and
$u_1 + u_2 \in U_1 \cup U_2$. Thus it must be the case that
$u_1 + u_2 \in U_1$ or $u_1 + u_2 \in U_2$. Suppose that the former is true;
then it follows that $u_1 + u_2 - u_1 = u_2 \in U_1$, which is a contradiction
(same thing happens if we assume the latter). Thus given case is impossible.
Therefore there cannot exist $u_1 \in U_1 \setminus U_2$ and
$u_2 \in U_2 \setminus U_1$. Thus
$U_1 = U_1 \cup U_2$ or $U_2 = U_1 \cup U_2$.

The reverse case is trivial: if we have two subspaces and
one of it is a subset of another, then larger subspace is is subspace.

\subsection{}
\textit{Prove that the union of three subspaces of $V$ is a subspace of
  $V$ if and only if one of the subspaces contains the other two.}

Same thing applies as in previous exercise: zero and multiplication are
trivial.

We are going to proceed with a proof by contradiction, but firstly we want
to state precisely what we want to prove in a first place. We want to state,
that if a union of three subspaces is a subspace, then this union is equal to
one of the subpsaces. So let us start: suppose that the union of three subspaces is not equal to one of the subspaces.

Firstly, we can eliminate the case, when one of the subspaces is a
subset of another subspace, but third isn't, because it will mean that
union of first two subspaces constitues a subspace, and thus we'll default
to result in the previous exercise.

Thus let us assume that none of the subspaces is a subset of another subspace.
Now we've got two cases to sort out: suppose that if we take $u_2 \in U_2$
and $u_3 \in U_3$ we get that
$$u_2 + u_3 \in U_1$$
for every $u_2 \in U_2$ and $u_3 \in U_3$. Then we can follow, by setting
$u_2 = 0$ to the case that
$$\forall u_3 \in U_3 \to u_3 + u_2 \in U_1 \to u_3 + 0 \in U_1 \to
u_3 \in U_1$$
thus $U_3$ is a subset of $U_1$, which raises a contradiction (in our
assumptions that $U_3$ is not a subset of $U_1$  and by extension
for the default 2-subspace case).

The case when $u_2 \in U_2$, $u_3 \in U_3$ and $u_2 + u_3 \notin
U_1 \cup U_2 \cup U_3$ implies that $U_1 \cup U_2 \cup U_3$ is
not a vector space, thus it cannot happen.

The case when $u_2 \in U_2$, $u_3 \in U_3$ and $u_2 + u_3 \notin U_1$
implies that $u_2 + u_3$ is  in $U_2 \cup U_3$. This raises the case that
$U_2$ is a subspace of $U_3$, which is a contradiction.

Thus we can follow that there exists $u_1 \in U_1$ such that it
cannot be represented in terms of vectors from $U_2$ and $U_3$.
Thus we can follow that analogous  vectors   $u_2 \in U_2$ and $u_3 \in U_3$
also exist.

Because we are still assuming that $U_1 \cup U_2 \cup U_3$ we can follow that
$$u_1 + u_2 + u_3 \in U_1 \cup U_2 \cup U_3$$
Thus this sum is bound to be located in one of the $U_1$, $U_2$ or $U_3$.
Let us assume for simplicity of notation that it is located in $U_1$. Then
we can follow that
$$u_1 + u_2 + u_3 - u_1 = u_2 + u_3 \in U_1$$

Suppose that we take $u_2 \in U_2 \setminus (U_3 \cup U_1)$ and
$u_3 \in U_3 \setminus (U_1 \cup U_2)$. It follows that
$u_2 + u_3$ cannot be in either $U_2$ nor in $U_3$ because in this
case we have that
$$u_2 + u_3 - u_2 = u_3 \in U_2$$
which is a contradiction. Thus
$$u_2 + u_3 \in U_1 \setminus (U_2 \cup U_3)$$
let us call it $u_1'$. In the same fashion we can define $u_2'$ and $u_3'$.

Thus $u_1' + u_2' + u_3' \in U_1 \cup U_2 \cup U_3$. Thus it needs to
be in one of $U_1$, $U_2$ or $U_3$. Suppose that it is included in
$U_1$. Then we can follow that
$$u_1' + u_2' + u_3' \in U_1$$
$$u_2' + u_3' \in U_1$$
$$u_1 + u_3 + u_1 + u_2 \in U_1$$
$$2u_1 + u_3  + u_2 \in U_1$$
$$u_3  + u_2 \in U_1$$

TODO

\subsection{}
\textit{Verify the assertion in Example 1.38}

1.38 states that

\textit{Suppose that $U = \{(x, x, y, y) \in F^4: x, y \in F\}$ and
  $W = \{(x, x, x, y) \in F^4: x, y \in F\}$. Then }
$$U + W = \{(x, x, y, z) \in F^4: x, y, z \in F\}$$
\textit{as you should verify}

Let $u \in U$ and $w \in W$. It follows that
$$u = (x_1, x_1, x_1, y_1)$$
$$w = (x_2, x_2, y_2, y_2)$$

Suppose that $q \in U + W$.
It follows that
$$q = (x_1 + x_2, x_1 + x_2, x_1 + y_2, y_1 + y_2)$$
thus we can set $x = x_1 + x_2$, $y = x_1 + y_2$ and $z = y_1 + y_2$
and call it a day.

\subsection{}
\textit{Suppose $U$ is a subspace of $V$. What is $U + U$.}

By properties of vector space, if we take $u_1, u_2 \in U$ then
$$u_1 + u_2 \in U$$
for every $u_1, u_2 \in U$. Thus we can follow that
$$U + U = U$$

\subsection{}
\textit{Is the operation of addition on the subspaces of $V$ commutative? In
  other words, if $U$ and $W$ are subspaces of $V$, is $U + W = W + U$?}

If $q \in U + W$ it follows that there exists $u \in U$ and $w \i W$
such that
$$q = v + w = w + v = q'$$
where $q' \in W + U$. Thus we can follow that $W + U = U + W$.

\subsection{}

\textit{Is the operation of addition on the subspaces of $V$ associative? In
  other words, if $U_1$, $U_2$, $U_3$ are subspaces of $V$, is}
$$(U_1 + U_2) + U_3 = U_1 + (U_2 + U_3)?$$

Yes it is. We can apply the same logic as in the previous exercise and it'll do
the job.

\subsection{}

\textit{Does the operation of addition on the subspaces of $V$ have an additive
  identity? Which subspace have additive inverces?}

Every subspace contains zero, therefore
$$U + 0 = U$$
thus we've got additive identity.

By adding two subspaces together we get a larger subspace, thus we can follow
that the only way to get 0 vector space as the result of addition of two
subspaces is to add
$$0 + 0 = 0$$
thus the only subspace that contains additive inverse is $0$.

\textit{Prove or give counterexample: if $U_1$, $U_2$, $W$ are subspaces of
  $V$, such that }
$$U_1 + W = U_2 + W$$
\textit{then $U_1 = U_2$}

This is wrong: suppose that $U_2$ is a nonzero subspace of $W$ and $U_1 = 0$.
Then it follows that
$$U_1 + W = 0 + W = W = W + U_2$$
and
$$U_1 \neq U_2$$
as desired.

\subsection{}

\textit{Suppose}
$$U = \{(x, x, y, y) \in F^4: x, y \in F\}$$
\textit{Find a subspace $W$ of $F^4$ such that $F^4 = U \bigoplus W$ }

$$W = \{(0, x, y, 0) \in F^4: x, y \in F\}$$


\subsection{}

\textit{Suppose}
$$U = \{(x, y, x + y, x - y, 2x) \in F^5: x, y \in F\}$$
\textit{Find a subspace $W$ of $F^5$ such that $F^5 = U \oplus W$ }

$$W = \{(0, 0, x, y, z) \in F^5: x, y, z \in F\}$$

\subsection{}

\textit{Suppose}
$$U = \{(x, y, x + y, x - y, 2x) \in F^5: x, y \in F\}$$
\textit{Find a thee subspaces $W_1$, $W_2$, $W_3$ of $F^5$
  such that $F^5 = U \oplus W_1 \oplus W_2 \oplus W_3$ }

$$W_1 = \{(0, 0, x, 0, 0) \in F^5: x \in F\}$$
$$W_2 = \{(0, 0, 0, y, 0) \in F^5: y \in F\}$$
$$W_3 = \{(0, 0, 0, 0, z) \in F^5: z \in F\}$$

\subsection{}

\textit{Prove or give a counterexample: if $U_1$, $U_2$, $W$ are subspaces
  of $V$ such that }
$$ V = U_1 \oplus W \text{ and } V = U_2 \oplus W $$
\textit{then $U_1 = U_2$}

This one is false;
$$U_1 = \{(x, x) \in F^2: x \in F\}$$
$$U_2 = \{(x, 0) \in F^2: x \in F\}$$
$$W = \{(0, y) \in F^2: y \in F\}$$

\subsection{}

\textit{A function $f: R \to R$ is called even if }
$$f(-x) = f(x)$$
\textit{for all $x \in R$. A function $f: R \to R$ is called odd if }
$$f(-x) = -f(x)$$
\textit{for all $x \in R$. Let $U_e$ denote the set of real-valued
  even functions on $R$ and let $U_o$ denote the set of real-valued odd
  functions on $R$. Show that }
$$R^R = U_e \oplus U_o$$



Let $f: R \to R$ be arbitrary. It follows that

$$f_e(x) =
\begin{cases}
  2 f(x) - f(-x) \text{ if } x \geq 0 \\
  f(x) \text{ if } x = 0 \\
  2 f(-x) - f(x) \text{ if } x < 0
\end{cases}
$$

Every odd function satisfies $f(0) = 0$.
Therefore for even function we've got to have $f_e(0) = f(0)$


$$
f_e(x) =
\begin{cases}
  a_1 f(x) + b_1 f(-x) \text{ if } x > 0 \\
  a_1 f(-x) + b_1 f(x) \text{ if } x < 0
\end{cases}
$$

$$
f_o(x) =
\begin{cases}
  a_2 f(x) + b_2 f(-x) \text{ if } x > 0 \\
  -a_2 f(-x) - b_2 f(x)  \text{ if } x < 0
\end{cases}
$$

$$
\begin{cases}
  a_1 + a_2 = 1 \\
  b_1 + b_2 = 0 \\
  a_1 - a_2 = 0 \\
  b_1 - b_2 = 1 \\
\end{cases}
$$

$$
\begin{cases}
  a_1  = 0.5 \\
  b_1 = 0.5 \\
\end{cases}
$$

$$
f_e(x) =
\begin{cases}
  1/2 f(x) + 1/2 f(-x) \text{ if } x > 0 \\
  f(x) \text{ if } x = 0 \\
  1/2 f(x) + 1/2 f(-x)  \text{ if } x < 0
\end{cases}
$$

$$
f_o(x) =
\begin{cases}
  1/2 f(x) - 1/2 f(-x) \text{ if } x > 0 \\
  0 \text{ if } x = 0 \\
  -1/2 f(-x) + 1/2 f(x)  \text{ if } x < 0
\end{cases}
$$

Thus
$$f_e(x) = f_e(-x)$$
$$f_o(-x) = -f_o(x)$$
and
$$f_e(x) + f_o(x) = f(x)$$
as desired.

Also, the only function that is odd and even at the same time is $0$,
therefore we've got a direct sum, as desired.


\chapter{Finite-Dimentional Vector Spaces}

\section{Span and Linear Independence}

\subsection{}
\textit{Suppose $v_1, v_2, v_3, v_4$ spans $V$. Prove that the list }
$$v_1 - v_2, v_2 - v_3, v_3 - v_4, v4$$
\textit{also spans $V$.}

Let $v \in V$ be represented as
$$v = a_1 v_1 + a_2 v_2 + a_3 v_3 + a_4 v_4$$

then we can follow that
$$v = a_1 (v_1 - v_2) + (a_2 + a_1) (v_2 - v_3)+ (a_3 + a_2 + a_1) (v_3 - v_4) + (a_1 + a_2 + a_3 + a_4) v_4$$
therefore any $v \in V$ can be represented using given list, therefore
given list spans $V$, as desired.

\subsection{}
\textit{Verify the assertion in Example 2.18}

Suppose that $v \in V$. Then it follows from some exercise in previous
chapter that $a_1 v = 0$ iff $a_1 = 0$ or $v = 0$. Thus if $v \neq 0$ we
can follow that the only way to represent zero is to set $a_1$ to 0. Thus
list is linearly independent.


Suppose that we've got linearly independent list of two vectors. We therefore
can follow that the only way to represent 0 is to set
$a_1 = 0$ and $a_2 = 0$. Thus vectors are not a scalar multiples of each other.
In other directon we've got a trivial case.

For the list
$$v_1 = (1, 0, 0, 0), v_2 = (0, 1, 0, 0), v_3 = (0, 0 1, 0)$$
we've got that
$$v = a_1 v_1 + a_2 v_2 + a_3 v_3 = (a_1, a_2, a_3, 0)$$
therefore the only way to represent zero is to set all of a's into 0.

Same case applies for the last one.

\subsection{}
\textit{Find a number $t$ such that}
$$(3, 1, 4), (2, -3, 5), (5, 9, t)$$
\textit{is not linearly inependent in $R^3$}

The only way that this list is not linearly independent is if 
we can represent last vector as a linear combiination of the other two. Thus
$$
\begin{cases}
  3 a_1 + 2 a_2 = 5
  1 a_1 - 3 a_2 = 9
\end{cases}
$$
$$
\begin{cases}
  3 a_1 + 2 a_2 = 5
  a_1= 9 +  3 a_2
\end{cases}
$$
$$   3 (9 +  3 a_3) + 2 a_2 = 5 $$
$$   27 +  9 a_2 + 2 a_2 = 5 $$
$$ 11 a_2= -22$$
$$ a_2 = -2$$
thus
$$a_1 = 3$$
therefore
$$3 * 4 - 5 * 2 = t$$
$$t = 2$$

\subsection{}
\textit{Verify the assertion in the second bullet point in Example 2.20}

$c = 8$ is the only solution such that third vector is a
scalar multipe of first vector plus scalal multiple of second. Thus we
can follow that the last vector is not in the span of first two, therefore
the list is linearly independent.

\subsection{}
\textit{(a) Show that if we think of $C$ as a vector space over $R$, then the
  list $(1 + i, 1 - i)$ is linearly independent.}

$$(1 + i  + 1 - i)/2 = 1$$
$$(1 + i - 1 + i)/2 = i$$
thus the only way to represent $0$ is to set all of a's to zero

\textit{(b) Show that if we think of $C$ as a vector space over $C$, then
  the list $(1 + i, 1 - i)$ is linearly dependent}

List $(1)$ spans $C$, and its length is less that
the length of given set. THus given set is linearly dependent.


\subsection{}
\textit{Suppose $v_1, v_2, v_3, v_4$ is linearly independent.
  Prove that the list }
$$v_1 - v_2, v_2 - v_3, v_3 - v_4, v4$$
\textit{is also linearly independent.}

As we've shown before, spans of two sets are equal, therefore the only
way to represent $0$ is to put all a's to 0.



\subsection{}
\textit{Prove or give counterexample: If $v_1, v_2, ... v_m$ is a linearly
  independent list of vectors in $V$, then}
$$5v_1 - 4v_2, v_2, v_3, ... v_m$$
\textit{is linearly independent}

Both sets span the same space and have the same length, therefore they are
both linearly independent.

\subsection{}

Trivial, equivalent to previous

\subsection{}
\textit{Prove or give counterexample: If $v_1, ... v_m$ and $w_1, ..., w_m$ are
  linearly independent lists of vectors in $V$, then
  $v_1 + w_1, ..., v_m + w_m$ is linearly independent.}

False: set $w_1 = - v_1$ and get the desired result.

\subsection{}
\textit{Suppose $v_1, ..., v_m$ is linearly independent in $V$ and $w \in V$.
  Prove that if $v_1 + w, v_2 + w, ... v_m + w$ is linearly dependent,
  then $w \in span(v_1, v_2, ... v_m)$.}

Suppose that resulting list is linearly dependent. It follows that there
exists a way to represent 
$$\sum_{n = 1}^m{a_n (v_n + w)} = 0$$
such that not all a's are zeroes. Thus
$$\sum_{n = 1}^m{a_n (v_1 + w)} = \sum_{n = 1}^m{(a_n w + a_n v_n)} =
\sum_{n = 1}^m{a_n w} + \sum_{n = 1}^m{a_n v_n} =
w \sum_{n = 1}^m{a_n} + \sum_{n = 1}^m{a_n v_n} = 0
$$
$$- w \sum_{n = 1}^m{a_n} =  \sum_{n = 1}^m{a_n v_n}$$
$\sum_{n = 1}^m{a_n} \neq 0$, because otherwise left side is zero and
therefore right side is zero, which is not assumed.
$$w  =  \sum_{n = 1}^m{ - \frac{a_n}{\sum_{j = 1}^m{a_j}} v_n}$$
thus $w \in span(v_1, v_2, ... v_m)$, as desired.


\subsection{}
\textit{Suppose $v_1, ..., v_m$ is linearly independent in $V$ and $w \in V$.
  Show that $v_1, ..., v_m, w$ is linearly independent if and only if }
$$w \notin span(v_1, ..., v_m)$$

Because otherwise we've got a bigger linearly independent list, that spans
$V$.

\subsection{}
\textit{Explain why there does not exist a list of six polinomials that is
  linearly independent of $\mathcal{P_4}(F)$.}

Because the list of length 5 spans this space.

\subsection{}
\textit{Explain why no list of four polynomials spans $\mathcal{P_4}(F)$.}

Because the list of length 5 spans this space.

\subsection{}
\textit{Prove that $V$ is infinite-dimentional if and only if there is a
  sequence $v_1, v_2, ... $ of vectors in $V$ such that $v_1, ... v_m$ is
  linearly independent for every possible integer $m$.}

Forward is coming from the fact that we can always add new vectors
to a given linearly independent list of vectors, that are outside of span
of given list.

Because there always exists list that is bigger than
given list and is linearly independent in $V$ we can follow that
no final list of vectors spans $V$, therefore it is infinite-dimentional.


\subsection{}
\textit{Prove that $F^{\infty}$ is infinite-dimentional.}

Infinite list
$$(1, 0, ....), (0, 1, 0, ...), .... $$
is all linearly indepenent, therefore no finite set spans the space.


\subsection{}
\textit{PRove that the real vector space of all continous real-valued
  functions on the interval $[0, 1]$ is infinite-dimentional.}

We can create a countable sequence $(r_1, r_2, ... )$ of rationals in this
space, and correspod each one of them with some number, thus creating a
infinite linearly inedependent list.

\subsection{}

\textit{Suppose $p_0, p_1, ... p_n$ are
  polynomials in $\mathcal{P_m}(F)$ such that
  $p_j(2) = 0$ for each $j$. Prove tat $p_0, p_1, ... p_m$ is not linearly
  independent in $\mathcal{P_m}(F)$.}

Because it has the same length as $1, x, x^2 ... $, but doesn't span the same
space.

\section{Bases}

There are no challenging exercises in this section, just a recap of
the material. Looked them over, brushed up the material, not gonna waste
my time writing them down.

\section{Dimention}

\subsection{}
\textit{Suppose $V$ is finite-dimentional and $U$ is a subspace of $V$ such
  that $\dim U = \dim V$. Prove that $U = V$}

They have the same length of basis, thus basis of $U$ is a basis of $V$.

\subsection{}
\textit{Show that the subspaces of $R^2$ are precisely $\{0\}, R^2$ and all
  lines through the origin}

For 0 dimention we've got null

For dimention 1 we've got scalar multiple of any vector, which are lines
through the origin

For dimention 2 we've got the space itself

\subsection{}
\textit{Show that the subspaces of $R^3$ are precisely $\{0\}, R^3$, all
  lines through the origin, and all planes through the origin}

Same idea as in previos exericise, but list of length 2 defines a plane
through the origin and 3 defined space itself

\subsection{}
\textit{(a) Let $U = \{p \in P_4(F): p(6) = 0$. Find a basis of $U$.}

$$(x - 6), (x - 6)^2, (x - 6)^3, (x - 6)^4$$

\textit{(b) Extend the basis in part (a) to a basis of $P_4(F)$}

$$1, (x - 6), (x - 6)^2, (x - 6)^3, (x - 6)^4$$

\textit{Find a subspace $W$ of $P_4(F)$ such that $P_4(F) = U \oplus W$}

$$\{c: c \in F\}$$

\subsection{}
\textit{(a) Let $U = \{p \in P_4(F): p''(6) = 0$. Find a basis of $U$.}

$$1, (x - 6), (x - 6)^3, (x - 6)^4$$

\textit{(b) Extend the basis in part (a) to a basis of $P_4(F)$}

$$1, (x - 6), (x - 6)^2, (x - 6)^3, (x - 6)^4$$

\textit{Find a subspace $W$ of $P_4(F)$ such that $P_4(F) = U \oplus W$}

$$ (x - 6)^2$$


\subsection{}
\textit{(a) Let $U = \{p \in P_4(F): p(2) = p(5)$. Find a basis of $U$.}

$$1, (x - 2)(x - 5), (x - 2)^2(x - 5), (x - 2)^2(x - 5)^2$$

\textit{(b) Extend the basis in part (a) to a basis of $P_4(F)$}

$$1, x, (x - 2)(x - 5), (x - 2)^2(x - 5), (x - 2)^2(x - 5)^2$$

\textit{Find a subspace $W$ of $P_4(F)$ such that $P_4(F) = U \oplus W$}

$$ x $$

\subsection{}
\textit{(a) Let $U = \{p \in P_4(F): p(2) = p(5) = p(6)$. Find a basis of $U$.}

$$1, (x - 2)(x - 5)(x - 6), (x - 2)^2(x - 5)(x - 6)$$

\textit{(b) Extend the basis in part (a) to a basis of $P_4(F)$}

$$1, x, x^2,  (x - 2)(x - 5)(x - 6), (x - 2)^2(x - 5)(x - 6)$$

\textit{Find a subspace $W$ of $P_4(F)$ such that $P_4(F) = U \oplus W$}

$$x, x^2$$

\subsection{}
\textit{(a) Let $U = \{p \in P_4(F): \int_-1^1{p} = 0 \}$.
  Find a basis of $U$.}

$$x, x^3$$

\textit{(b) Extend the basis in part (a) to a basis of $P_4(F)$}

$$1, x, x^2, x^3, x^4$$

\textit{Find a subspace $W$ of $P_4(F)$ such that $P_4(F) = U \oplus W$}

$$1, x^2, x^4$$

\subsection{}
\textit{Suppose $v_1, ... v_m$ is linearly independent in $V$ and $w \in V$.
  Prove that}

$$\dim span(v_1 + w, ..., v_m + w) \geq m - 1$$

Because $v_1, ... v_m$ is linearly independent we can follow that $w$ is either
in $span(v_1, ..., v_m)$ or not. In the latter case we've got that the
case that we increase the span. In the former we've got by linear independence
of $v_1, ... v_m$ that the maximum decline of degree is $1$. Thus
$$\dim span(v_1 + w, ..., v_m + w) \geq m - 1$$
as desired.

\subsection{}

\textit{Suppose $p_0, p_1, ..., p_m \in P(F)$ are such taht each $p_j$ has
  degree $j$. Prove that $p_0, ... p_m$ is a basis of $P_m(F)$. }

Suppose that $p \in P_m(f)$. Because each $p_n$ has a degree of $n$ we can
follow that there exists only 1 $a_m \in F$ such that 
of $p_m$ such that
$$p - a_m p_m \in P_{m - 1}(F)$$.

Bu applying the same procedure  again repeatedly  we get unique
$a_m, ..., a_0$ such that
$$\sum{a_n p_m} = p$$
for every $p \in P_m(f)$. Thus we can follow that given list spans $P_m(F)$
and by unique representation we get that this list is linearly independent.
Thus we can follow that given list is a basis of $P_m(F)$, as desired.


\subsection{}
\textit{Suppose that $U$ and $W$ are subspaces of $R^8$ such that $\dim U = 3$,
  $\dim W = 5$, and $U + W = R^8$. Prove that $R^8 = U \oplus W$.}

We know that
$$\dim (U_1 + U_2) = \dim U_1 + \dim U_2 - \dim (U_1 \cap U_2)$$

Thus we can follow that in this particular case
$$\dim (R^8) = \dim U + \dim W - \dim (U \cap W)$$
$$8 = 3 + 5 - \dim (U \cap W)$$
$$\dim (U \cap W) = 0$$
thus we can follow that $U \cap W = \{0\}$. Therefore
$$U + W = U \oplus W = R^8$$
as desired.

\subsection{}
\textit{Suppose that $U$ and $W$ are both five-dimentional subspaces of $R^9$.
  Prove that $U \cap W \neq \{0\}$}

Once again we get that
$$\dim R^9 = \dim U + \dim W - \dim (U \cap W)$$
$$9 = 5 + 5- \dim (U \cap W)$$
$$\dim (U \cap W) = 1$$
thus
$$U \cap W \neq 0$$
as desired.

\subsection{}
\textit{Suppose $U$ and $W$ are both 4-dimentional subspaces of $C^6$. Prove
  that there exists two vectors in $U \cap W$ such that neither of these
  vectors is a scalar multiple of the other}

Goto previous exercise for concretee explanation if needed, but  we can 
conclude that
$$\dim U \cap W = 2$$
thus there exists a linearly independent list of length 2 in $U \cap W$ (basis)
so that neither of them is a scalar multiple of another by some exercise in 2.A

\subsection{}
\textit{Suppose $U_1, ... U_m$ are finite-dimentional subspaces of $V$.
  Prove that $U_1 + ... + U_m$ is finite-dimentional and }
$$\dim(\sum U_n) \leq \sum{\dim U_n}$$

We know that 
$$\dim (U_1 + U_2) = \dim U_1 + \dim U_2 - \dim (U_1 \cap U_2)$$
given that $\dim W \geq 0$ for any vector space $W$ we follow that
$$\dim (U_1 + U_2) \leq \dim U_1 + \dim U_2$$
Thus by induction
$$\dim(\sum U_n) \leq  \sum \dim U_n $$
which in presented case get us desired result.

\subsection{}
\textit{Suppose $V$ is finite-dimentional, with $\dim V = n \geq 1$. Prove
  that there exist 1-dimentional subspaces $U_1, ... U_n$ of $V$ such that  
}
$$V = U_1 \oplus ... \oplus U_n$$

For $V$ there exists a basis of length $n$. Thus by setting
$$U_j = \{c v_j: c \in F\}$$
we get desired result.

\subsection{}
\textit{Suppose $U_1, ..., U_m$ are finite-dimentional subspaces of $V$ such
  that $U_1 + .. + U_m$ is a direct sum. Prove that $U_1 + ... + U_m$ is
  finite dimentional and that}
$$\dim \sum U_n = \sum \dim U_n$$

We can just go by induction on the case that
$$\dim (U \oplus W) = \dim U + \dim W + \dim (U \cap W) =
\dim U + \dim W + 0$$
Or we can use the fact, that we can combine all bases
of subspaces together in one
mega-basis for their sum. Both will suffice.

\subsection{}
\textit{You might guess, by analogy with the formula for the number of elements
  in the union of three subsets of a finite set, that if $U_1, U_2, U_3$ are
  subspaces of finite-dimentional vector space, then}
$$\dim (U_1 + U_2 + U_3) =
\dim U_1 + \dim U_2 + \dim U_3 - \dim (U_1 \cap U_2) - \dim (U_1 \cap U_3) - $$
$$
- \dim (U_2 \cap U_3)
+ \dim (U_1 \cap U_2 \cap U_3)$$

We know that 
$$\dim (U_1 + U_2) = \dim U_1 + \dim U_2 - \dim (U_1 \cap U_2)$$
and
$$U_1 + U_2 + U_3 = (U_1 + U_2) + U_3$$
thus
$$\dim (U_1 + U_2 + U_3) = \dim ((U_1 + U_2) + U_3) =
\dim (U_1 + U_2) + \dim U_3 - \dim ((U_1 + U_2) \cap U_3) =$$
$$ =
\dim U_1 + \dim U_2 - \dim U_1 \cap U_2 + \dim U_3 -
\dim ((U_1 + U_2) \cap U_3) = $$
here we get a little problem because we don't know how to reduce
$(U_1 + U_2) \cap U_3$ to some managable pieces. After this discovery
one might even glance over
the equation once again in order to try to disprove the theorem.
And indeed we've found a counterexample: suppose that $U_1, U_2, U_3$ are
lines through the origin in $R^3$ such that they are located on the same
plane. Then it follows that left-hand side becomes 2, and the right side is
equal to 3. Thus we've got a contradiction (which is a shame, because
the formula looks nice :( ).

\chapter{Linear maps}

\section{The Vector Space of Linear Maps}

\subsection{}
\textit{Suppose $b, c \in R$. Define $T: R^3 \to R^2$ by }
$$T(x, y, z) = (2x - 4y + 3z + b, 6x + cxyz)$$
\textit{Show that $T$ is linear if and only of $b = c = 0$.}

Suppose that $T$ is linear. Then it follows that
$$T(0) = 0 = (0 + b, 0)$$
thus we can follow that $b = 0$.

Also,
$$T((1, 1, 1) + (2, 2, 2)) = (6 - 12 + 9, 18 + 27c) = (3, 18 + 27c) =
$$
$$
= T((1, 1, 1)) + T(2, 2, 2) = (2 - 4 + 3, 6 + c) + (4 - 8 + 6, 12 + 8c) =
(1, 6 + c) + (2, 12 + 8c) = (3, 18 + 9c)$$

Thus
$$27c = 9c$$
$$3c = c$$
$$c = 0$$
as desired.

Reverse implication is trivial, thus we get the desired result.

\subsection{}
\textit{Suppose $b, c \in R$. Define $T: \mathcal(P)(R) \to R^2$ by}
$$Tp = \left(3p(4) + 5p'(6) + bp(1)p(2), \int_{-1}^2{x^3 p(x) dx} + c \sin{p(0)}\right)$$
\textit{Show that $T$ is linear if and only if $b = c = 0$.}

Suppose that $T$ is linear. Then it follows that if $p(0) = \pi/2$, then latter
part of resulting vector has additive property only when $c = 0$. For the former
we've got result that
$$\lambda^2 b = b$$
for all $\lambda \in R$, which happens only if $b = 0$. Thus $b = c = 0$.

Reverse implication is trivial, thus we have the desired result.

\subsection{}
\textit{Suppose $T \in  \mathcal{L}(F^n, F^m)$. Show that there exists scalars
  $A_{j, k} \in F$ for $j = 1, ..., m$ and $K = 1, ..., n$ such that}
$$T(x_1, ..., x_n) = (A_{1, 1}x_1 + ... + A_{1, n} x_n, ..., A_{m, 1} x_1 + ... + A_{m, n} x_n)$$
\textit{for every $(x_1, ..., x_n) \in F^n$.}

Because $(1, 0, ...), (0, 1, ...), ... $ is a basis of $F^n$ we can follow that
there vector in $F^m$, such that $T(v) \in F^m$. Thus let us denote 
$$T(1, 0, ...) = (A_{1, 1}, A_{2, 1}, ..., A_{m, 1})$$
$$T(0, 1, ...) = (A_{1, 2}, A_{2, 2}, ..., A_{m, 2})$$
$$...$$
Thus given given arbitrary vector $v = (x_1, x_2, ..., x_n)\in T^n$ we get that 
$$T(v) = T(x_1, x_2, ...) = T(x_1, 0, 0, ...) + T(0, x_2, 0, ...) + ... =
x_1T(1, 0, 0, ...) + x_2 T(0, 1, 0, ...) + ... = $$
$$ = (x_1 A_{1_1}, x_1 A_{2, 1}, ... ) +
(x_2 A_{1_2}, x_2 A_{2, 2}, ... ) = (x_1 A_{1, 1} + x_2 A_{1, 2} + ..., x_1 A_{2, 1} + x_2 A_{2, 2} + ... )$$
as desired.

\subsection{}
\textit{Suppose $T \in \mathcal{L}(V, W)$ and $v_1, v_2, ... v_m$ is a list of vectors in $V$
  such that $T v_1, ...,, T v_m$ is a linearly inndependent list in $W$. Prove that
$v_1, v_2, ..., v_m$ is linearly independent.}

Suppose that it isn't. Then we can follow that there exist $w_1 \in W$ such that
$$w_1 = \sum a_j v_j = 0$$
and not all of $a_j$'s are zeroes. Thus we can follow that
$$T(w) = T(\sum a_j v_j) = \sum T (a_j v_j) = \sum a_j T (v_j) = 0$$
But $T(v_j)$ is a list of linearly independent vectors, and therefore their sum is
equal to zero iff all $a_j$'s are zeroes, which is false. Thus we've got a contradiction.

\subsection{}
\textit{Prove the assertion in 3.7}

Let $T_1 = T, T_2 = S, T_3 \in L(V, W)$. Then it follows that

(1) $$(T_1 + T_2)(v) = T_1(v) + T_2(v) = T_2(v) + T_1(v) = (T_2 + T_1)(v)$$

(2) $$(T_1 + (T_2 + T_3))(v) = T_1(v) + (T_2 + T_3)(v) = T_1(v) + T_2(v) + T_3(v) =$$
$$ = (T_1 + T_2)(v) + T_3(v) = ((T_1 + T_2) + T_3)(v)$$

(3) $$\lambda((S + T)(v)) = \lambda ( S(v) + T(v)) = \lambda S(v) + \lambda T(v)
= (\lambda S + \lambda T)(v)$$

(4) $$T + 0 = T$$

(5) $$1T = T$$

(6) $$T + -1T = (1 - 1)T = 0T = 0$$

Thus $L(V, W)$ satisfies all reqirements of a vector space, as desired.

\subsection{}
\textit{Prove the assertion in 3.9}

Let $v \in V$.

(1) Then it follows that
$$((T_1 T_2)T_3)(v) = (T_1 T_2)( T_3(v)) =  T_1 (T_2( T_3(v))) = T_1 ((T_2 T_3)(v))
= (T_1(T_2 T_3))(v)$$
directly from definition. (I wonder if it's  true in general for all functions; it probably
is).

(2)
$$ TI v = T(I(v)) = T(v) = I (T(v))$$

(3)

$$(S_1 + S_2)T(v) = (S_1 + S_2)(T(v)) = S_1(T(v)) + S_2(T(v)) = S_1 T v + S_2 T v$$
$$S(T_1 + T_2)(v) = S((T_1 + T_2)(v)) = S(T_1(v) + T_2(v)) = S(T_1(v)) + S(T_2(v)) = S T_1 v +
S T_2 v$$

as desired.

\subsection{}
\textit{Show that every linear map from a 1-dimentional vector space to itself is
  multiplication by some scalar. More precisely, prove that if $\dim V = 1$ and
  $T \in L(V, V)$, then there exists $\lambda \in F$ such that $Tv = \lambda v$ for
  all $v \in V$.}

Because we've got a 1-dimentional space, it follows that there exists a basis of $V$ - $v_1$.
For this vector we've got that
$$T v_1 = v_2 = \lambda v_1$$
Thus we can follow that if $u \in V$ then
$$T u = T \sigma v_1 =  \sigma T v_1 = \sigma \lambda v_1 = \lambda \sigma v_1 = \lambda u$$
as desired.

\subsection{}
\textit{Give an example of a function $\phi: R^2 \to R$ such that }
$$\phi (av) = a\phi(v)$$
\textit{for all $a \in R$ and all $v \in R^2$ but $\phi$ is not linear.}

$$\phi(x, y) =
\begin{cases}
  x \text{ if } x \neq y \\
  0 \text{ otherwise}
\end{cases}
$$

\subsection{}
\textit{Give an example of a function $\phi: C \to C$ such that }
$$\phi (w + z) = \phi(w) + \phi(z)$$
\textit{for all $w, z \in C$ but $\phi$ is not linear.}

Let us define 
$$\phi(a + bi) = b + ai$$
Thus 
$$\phi(a + bi + c + di) = ai + ci + b + d = \phi(a + bi) + \phi(c + di)$$
but
$$i \phi(a + bi) = -a + bi$$
$$\phi(i(a + bi)) = \phi(ai - b) = -bi + a \neq i \phi(a + bi)$$

\subsection{}
\textit{Suppose $U$ is a subspace of $V$ with $U \neq V$. Suppose $S \in L(V, W)$ and
  $S \neq 0$. Define $T: V \to W$ by}
$$Tv =
\begin{cases}
  Sv \text{ if } v \in U \\
  0 \text{ if } v \in V \text{ and } v \notin U
\end{cases}
$$
\textit{Prove that $T$ is not a linear map on $V$.}

Let $u \neq 0 \in U$ such that $Su \neq 0$ and $v \in V \setminus U$. Then it follows that
$$v + u \notin U$$
(because otherwise $-(v + u)$ is in $U$, therefore $u - (v + u) = -v \in U$ and
thus $v \in U$, which is a contradiction)
Thus we can follow that
$$T(v + u) = 0$$
but
$$T(v) + T(u) = 0 + Su = Su \neq 0 = T(v + u)$$
therefore the function is not linear, as desired.

\subsection{}
\textit{Suppose $V$ is finite-dimentional. Prove that every linear map on a subspace of $V$
  can be extended to a lineaer map on $V$. In other words, show that if $U$ is a subspace of $V$
  and $S$ is a subspace of $V$ and $S = L(V, W)$, then there exists $T \in L(V, W)$ such that
  $Tu = Su$ for all $u \in U$.}

Because $V$ is finite-dimentional and $U$ is a subspace of $V$, we can follow that $U$
is finite-dimentional as well. Thus we can follow that there exists
$u_1, ..., u_m$ - basis of $U$. As we know, we can extend this basis to a basis of $V$ -
$u_1, ..., u_m, v_1, ... v_n$. Therefore we can define a map $P \in L(V, U)$ by 
$$P(x_1, x_2, ...) = (x_1, x_2, ... x_m, 0, 0, ...)$$
(basically trim every element of basis that is not in $U$). Thus we can follow that
$P(u) = u$ if $u \in U$. Proof that $P$ is linear is trivial.
Thus if $S \in L(U, W)$, then $T = SP \in (V, W)$ with the
desired properties.

\subsection{}

\textit{Suppose $V$ is finite-dimentional with $\dim V > 0$, and suppose $W$ is
  infinite-dimentional. Prove that $L(V, W)$ is infinite-dimentional.}

Let $v_1, ..., v_m$ be a basis of $V$ and
let $w_1, w_2, ...$ be a list of linearly independent vectors in $W$. Now
let us look at $T_n: V \to W$
$$T_n((x_1, x_2, ...) ) = x_1 w_n$$
Then it follows that by linear independence of $w_n$ there does not exist a linear
combination of $T_m$ such that
$$\sum_{m \neq n} a_m T_m \neq T_n$$
Thus we can follow that list $T_n$ is linearly independent.
Because list is not finite we can follow that the space $L(V, W)$ is infinite-dimentional,
as desired.

\subsection{}

\textit{Suppose $v_1, ..., v_m$ is a linearly dependent list of vectors in $V$. Suppose
  also that $W \neq \{0\}$. Prove that there exist $w_1, ... w_m \in W$ such that no
  $T \in L(V, W)$ satisfies $Tv_k = w_k$ for each $k = 1, ..., m$.}

Because $v_1, ..., v_m$ is linearly dependent we can reduce it to a linearly independent list
$v_1', ..., v_n'$. Thus resulting list will span some subspace of $V$ and will be its basis.

Thus we can take vector $v_j$ from the original list,
that does not appear in basis.
Then take some vectors $w_1, ... w_n$ in $W$. We know that there exists a unique map
$$Tv_n' = w_n$$
thus by adding to list $w_1, ... w_n$ any vectors from $W$, apart from $T(v_j)$ we create desired
list.

\subsection{}
\textit{Suppose $V$ is finite-dimentional with $\dim V \geq 2$. Prove that there
  exists $S, T, \in L(V, V)$ such that $ST \neq TS$ }

Let $v_1, v_2$ be a basis of $V$ and let
$$ S(x, y) = (y, x)$$
$$ T(x, y) = (x, 0)$$
Then
$$ST = (0, x)$$
and
$$TS = (y, 0)$$
as desired.


\section{Null Spaces and Ranges}

\subsection{}

\textit{Give an example of a linear map $T$ such that $\dim \null T = 3$ and
  $\dim range T = 2$.}

$T(x, y, z) = (x, y)$

\subsection{}

\textit{Suppose $V$ is a vector space and $S, T \in L(V, V)$ are such that }
$$range S \subset \null T$$
\textit{Prove that $(ST)^2 = 0$.}

Let $v \in V$. Then it follows that $S(T(v)) \in range S$. Thus $ST(v) \in \null T$.
Therefore $TST(v) = 0$. And thus  $STST = (ST)^2 = 0$, as desired.

\subsection{}
\textit{Suppose $v_1, ...,  v_m$ is a list of vectors in $V$. Define $T \in L(F^m, V)$ by}
$$T(z_1, ..., z_m) = z_1 v_1 + ... + z_m v_m$$

\textit{(a) What property of $T$ corresponds to $v_1, ..., v_m$ spanning $V$?}

Surjectivity

\textit{(b) What property of $T$ corresponds to $v_1, ..., v_m$ being linearly independent?}

Injectivity

\subsection{}

\textit{Show that }
$$ \{T \in L(R^5, R^4): \dim \null T > 2 \}$$
\textit{is not a subspace of $L(R^5, R^4)$.}

We can set

$$T_1(x, y, z, w, q) = (x, 0, 0, 0)$$
$$T_2(x, y, z, w, q) = (0, y, 0, 0)$$
$$T_3(x, y, z, w, q) = (0, 0, z, 0)$$
$$T_4(x, y, z, w, q) = (0, 0, 0, w)$$

all of which are in the desired subset, but their sum is
$$T(x, y, z, w, q) = (x, y, z, w, 0)$$
which has $\dim \null = 1$. Thus this subset is not closed under addition and therefore it
is not a subspace.

\subsection{}

\textit{Give an example of a linear map $T: R^4 \to R^4$ such that }
$$range T = \null T$$

$$T(x, y, z, w) = (z, w, 0, 0)$$.

\subsection{}

\textit{Prove that there does not exist a linear map $T R^5 \to R^5$ such that }
$$range T = \null T$$

$\dim$ is always an integer, therefore for $\dim range T = \dim \null T = n$ and
$$\dim T = 2n = 5$$
which is impossible.

\subsection{}

\textit{Suppose $V$ and $W$ are finite-dimentional with $2 \leq \dim V \leq \dim W$.
  Show that $\{T \in L(V, W): T \text{ is not injective}\}$ is not a subspace of $L(V, W)$.}

Suppose that $v_1, ..., v_m$ is a basis for $V$ and $w_1, ..., w_n$ is a basis of $W$.
We can follow that there exist, which maps $v_1$ to $w_1$ and so on. By adding all of
those maps together we get an injective map. Thus we can follow that given set is not
closed under addition and therefore is not a subspace.

\subsection{}

\textit{Suppose $V$ and $W$ are finite-dimentional with $2 \leq \dim W \leq \dim V$.
  Show that $\{T \in L(V, W): T \text{ is not surjective}\}$ is not a subspace of $L(V, W)$.}

By following the simular logic as in previous exercise, we get a desired result.

\subsection{}

\textit{Suppose $T \in L(V, W)$ is injective and $v_1, ..., v_n$ is linearly independent
  in $V$. Prove that $Tv_1, ..., Tv_n$ is linearly independent in $W$.}

Suppose that it is not the case. Then it follows that there exists $a_1, ... a_n \in F$ such
that not all of them are equal to zero and 
$$\sum a_n T v_n = 0$$
Thus we can follow that
$$T \sum a_n v_n = 0$$
Thus $\sum a_n v_n \in \null T$. Because $T$ is injective we can follow that
$$\sum a_n v_n = 0$$
and some of $a_n$'s are not equal to zero. But $v_1, ..., v_n$ is linearly independent, thus
we get a contradiction.

\subsection{}

\textit{Suppose $v_1, ..., v_n$ spans $V$ and $T \in L(V, W)$. Prove that the list
  $T v_1, .... Tv_n$ spans range $T$.}

Suppose $w \in range T$. Thus we can follow that there exists $v \in V$ such that
$$Tv = w$$
Given that $v_1, ..., v_n$ spans $V$ we can follow that there exists $a_1, ... a_n$ such that
$$v = \sum a_n v_n$$
and thus
$$w = T \sum a_n v_n$$
$$w =  \sum T a_n v_n$$
thus we can follow that $v_1, ..., v_n$ spans the range of $T$, as desired.

\subsection{}
\textit{Suppose $S_1, ..., S_n$ are injective linear maps such that $S_1 S_2 ... S_n$ makes sense.
  Prove that $S_1 S_2 ... S_n$ is injective. }

Suppose that $T$ and $S$ are injective such that $ST$ makes sence. Suppose that
$$STv = 0$$
Then by injectivity of $S$ we get that $Tv \in null S$ and thus $Tv = 0$. Thus, by injectivity of
$T$ we get that $v$ = 0. Therefore $\null ST = 0$. Therefore $ST$ is injective.

The case in the exercise is derived from induction on presented argument.

\subsection{}

\textit{Suppose that $V$ is finite-dimentional and that $T \in L(V, W)$. Prove that there exists
  a subspace $U$ of $V$ such that $U \cap \null T = 0$ and $range T = \{Tu: u \in U\}$.}

Let $N$ be a nullspace of $T$. It follows that it is a subspace of $V$. Now let
$n_1, ..., n_m$ be a basis of $N$ and extend it to a basis of $V$: $n_1, ..., n_m, v_1, ..., v_n$.
Then if follows that $span(v_1, ... v_n) \cap N = 0$ (because otherwise the vector is in nullspace)
and if $w \in rangeT$, then there exists $u \in span(v_1, ... v_n)$ such that
$Tu = w$. Thus $span(v_1, ... v_n)$ is the desired subspace.

\subsection{}

\textit{Suppose $T$ is a linear map from $F^4$ to $F^2$ such that}
$$\null T = \{(x_1, x_2, x_3, x_4) \in F^4: x_1 = 5x_2, x_3 = 7x_4\}$$
\textit{Prove that $T$ is surjective.}

$\dim \null T = 2$, thus $\dim range T = 2$, therefore $T$ is surjective, as desired.

\subsection{}
\textit{Suppose $U$ is a 3-dimentional subspace of $R^8$ and that $T$ is a linear map from
  $R^8$ to $R^5$ such that $\null T = U$. Prove that $T$ is surjcetive.}

We can follow that $\dim range T = 5$, and therefore $T$ is surjective, as desired.

\subsection{}

Very similar to previous one

\subsection{}

Same

\subsection{}

Same

\subsection{}

Same

\subsection{}

Same

\subsection{}

TODO

\end{document}
%%% Local Variables:
%%% mode: latex
%%% TeX-master: t
%%% End:
