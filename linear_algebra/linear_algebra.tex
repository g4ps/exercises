\documentclass[10pt,oneside,titlepage]{book}
\title{My linear algebra exercises}
\usepackage{amsmath,amssymb}
% \usepackage{geometry}
% \usepackage{pdfpages}
% \usepackage{tocloft}
\usepackage{hyperref}
\author{Evgeny Markin}
\date{2022}

\begin{document}
\maketitle
\tableofcontents

\chapter*{Preface}

Exercises are from "Linear algebra done right" by Sheldon Axler, 3rd ed.
I've already read this book before and completed some exercises from it.
Right now I want to brush up the material once again, put all the
proofs on a more durable material than paper and to prepare myself to
what's gonna happen afterwards.

\chapter{1.A $R^n$ and $C^n$}

\section*{1}
\textit{Suppose $a$ and $b$ are real numbers, not both $0$. Find real nuber
  $c$ and $d$ such that }
$$1/(a + bi) = c + di$$

$$\frac{1}{a + bi} = c + di$$
$$\frac{1}{a + bi} - c - di = 0$$
$$\frac{a - bi}{(a + bi)(a - bi)} = c + di$$
$$\frac{a - bi}{(a^2 + b^2)} = c + di$$
$$\frac{a}{a^2 + b^2} - \frac{b}{a^2 + b^2}i = c + di$$
Thus $c = \frac{a}{a^2 + b^2}$ and $d = -\frac{b}{a^2 + b^2}$

\section*{2}
\textit{Show that }
$$\frac{-1  + \sqrt{3}i}{2}$$
\textit{is a cube root of $1$ (meaning that its cube equals 1)}
$$(\frac{-1  + \sqrt{3}i}{2})^3 =
\frac{(-1  + \sqrt{3}i)^3}{8} =
\frac{(-1  + \sqrt{3}i)(-1  + \sqrt{3}i)^2}{8} =
\frac{(-1  + \sqrt{3}i)(1  - 2\sqrt{3}i - 3)}{8} =
$$
$$
=\frac{(-1  + \sqrt{3}i)(-2  - 2\sqrt{3}i)}{8} =
\frac{2 + 2\sqrt{3}i - 2\sqrt{3}i + 6}{8} =
\frac{8}{8} = 1
$$
as desired.

\section*{3}
\textit{Find two distinct square roots of $i$}

Square root of $i$, I assume, is a number, whose square is equal to $i$.
Suppose that $(a + bi)^2 = i$. It follows that
$$(a + bi)^2 = a^2 + 2abi - b^2$$
So if we set $$a = b = 1/\sqrt{2}$$ this equation holds. Also it holds for
$$a = b = -1/\sqrt{2}$$
maxima seems to agree with me on this one

\section*{4}
\textit{Show that $\alpha + \beta = \beta + \alpha$ for all
  $\alpha, \beta \in \textbf{C}$}

Let $\alpha = a_1 + b_1 i$ and $\beta = a_2 + b_2 i$. It follows
$$\alpha + \beta = a_1 + b_1 i + a_2 + b_2 i = a_2 + b_2 i + a_1 + b_1 i =
\beta  + \alpha$$
as desired.

\section*{5}
\textit{Show that $(\alpha + \beta) + \lambda = \alpha + (\beta + \lambda)$
  for all  $\alpha, \beta, \lambda \in \textbf{C}$}

Let $\alpha = a_1 + b_1 i$ , $\beta = a_2 + b_2 i$, $\lambda = a_3 + b_3 i$.
It follows that 
$$\alpha + (\beta + \lambda)  = a_1 + b_1 i + (a_2 + b_2 i + a_3 + b_3 i) =
(a_1 + b_1 i + a_2 + b_2 i) + a_3 + b_3 i = 
(\alpha + \beta) + \lambda$$

\section*{6}
\textit{Show that $(\alpha \beta) \lambda = \alpha( \beta \lambda)$}

$$\alpha + (\beta + \lambda)  = (a_1 + b_1 i) ((a_2 + b_2 i) + (a_3 + b_3 i)) =
((a_1 + b_1 i)(a_2 + b_2 i)) + (a_3 + b_3 i) = 
(\alpha \beta) \lambda$$

\section*{7}
\textit{Show that for every $\alpha \in \textbf{C}$ there exists a unique
  $\beta \in \textbf{C}$ such that $\alpha + \beta = 0$}

Suppose that there exist two different $\beta_1 \neq \beta_2$ such that
$\alpha + \beta_1 = 0$ and $\alpha + \beta_2 = 0$. It follows that
$$ \beta_1 = \beta_1 + 0 =  \beta_1 + \alpha + \beta_2 = \alpha + \beta_1  + \beta_2 = 0  + \beta_2 = \beta_2$$
which is a contradiction. Therefore there exists only one unique $\beta$.

\section*{8}
\textit{Show that for every $\alpha \in \textbf{C}$ with $\alpha \neq 0$
  there exists a unique $\beta \in \textbf{C}$ such that $\alpha \beta = 1$}

Suppose that it is not true and there exist two different
$\beta_1 \neq \beta_2$ such that
$$\alpha \beta_1 = 1 \textit{ and } \alpha \beta_2 = 1$$
it follows then that 
$$\beta_1 = 1 *  \beta_1 = \alpha \beta_2 \beta_1  =
\alpha \beta_1 \beta_2 = 1 * \beta_2 = \beta_2$$
which is a contradiction. Therefore there exists only one unique $\beta$.

\section*{etc}
The rest of the section is the repetition of this kind of stuff.
That is a lot of writing, and not a lot of thinking, so I'll skip it.
I don't ususally like to skip sections, but I have  aa feeling, that I've
completed this thing on paper somewhere, and there is not much reason to
rewrite it here.

\chapter{1.B Definition of Vector Space}

\section*{1}
\textit{Prove that $-(-v)= v$ for every $v \in V$.}

For $v$ there exists only one $-v$. For $-v$ there exists only one $-(-v))$.

Thus
$$v = v + 0 = v + (-v) + (-(-v)) = 0 + (-(-v)) = -(-v)$$
as desired (idk if it's true, I'm not good at axioms and stuff)

\section*{2}
\textit{Suppose $a \in F, v \in V$, and $av = 0$. Prove that
  $a = 0$ or $v = 0$.}

Suppose that $a \neq 0$, $v \neq 0$ but $av = 0$. It follows that there
exist $1/a$ - multiplicative inverse of $a$. It follows that
$$1/a * av = 1/a * 0$$
$$1v = 0$$
$$v = 0$$
which is a contradiction. Thus either $a = 0$ or $v = 0$.

\section*{3}
\textit{Suppose $v, w \in V$. Explain why there exists a unique $x \in V$
  such that $v + 3x = w$.}

Suppose that there exists $x_1 \neq x_2$ such that
$v + 3x_1 = w$ and $v + 3x_2 = w$. Thus
$$3x_1 = w - v = 3x_2$$
$$x_1 = \frac{1}{3}(w - v) = x_2$$
which is a contradiction.

Same can be stated from the fact that $x$ is a unique additive inverse of
$\frac{1}{3}(v - w)$.

\section*{4}
\textit{The empty set is not a vector space. The empty set fails to satisfy
  only one of the requirements listed in 1.19. Which one?}

Additive indentity. Empty set does not have zero element in it.
BTW $\{0\}$ is a vector space.

\section*{5}
\textit{Show that n the defintition of a vector space (1.19), the additive
  inverse condition can be replaced with the condition that}
$$0v = 0 \textit{ for all } v \in V$$
\textit{Here the 0 on the left side is the number 0, and the 0 on the right
  side is the additive identity of $V$.}

$$0v = 0$$
$$(1 - 1)v = 0$$
$$1v - 1v = 0$$
$$v - v= 0$$
$$v + (- v)= 0$$

\section*{6}
\textit{Let $\infty$ and $-\infty$ denote two distinct object, neither of
  which is in $R$. Define an addition and multiplication on
  $R \cup \{\infty\} \cup \{-\infty\}$ as you could guess from the notation.
  Specifically, the sum and the product of two real numbers is as usual,
  and for $t \in R$ define
}

$$
t\infty =
\begin{cases}
  -\infty \text{ if } t < 0 \\
  0 \text{ if } t = 0 \\
  \infty \text{ if } t > 0 \\
\end{cases}
$$

$$
t(-\infty) =
\begin{cases}
  \infty \text{ if } t < 0 \\
  0 \text{ if } t = 0 \\
  -\infty \text{ if } t > 0 \\
\end{cases}
$$

$$t + \infty = \infty + t = \infty$$
$$t + (-\infty) = (-\infty) + t = (-\infty)$$
$$\infty + \infty = \infty$$
$$(-\infty) + (-\infty) = (-\infty)$$
$$\infty + (-\infty) = 0$$

\textit{Is $R \cup \{\infty\} \cup \{-\infty\}$ a vector space over
  $R$? Explain.}

I don't think that it is.

$$(t + \infty) - \infty = \infty - \infty = 0$$
$$t + (\infty - \infty) = t + 0 = t$$
thus
$$t + (\infty - \infty) \neq (t + \infty) - \infty$$
thus 
$R \cup \{\infty\} \cup \{-\infty\}$ is not associative, therefore it is
not a vector space.

\chapter{1.C Subspaces}

\section*{1}
\textit{For each of the following subsets of $F^3$, determine whether it is a
  subspace of $F^3$:}

\textit{(a) $\{(x_1, x_2, x_3) \in F^3: x_1 + 2x_2 + 3x_3 = 0\}$}

Yes, it is. $0$ is contained within it.
$$(x_1, x_2, x_3) + (y_1, y_2, y_3) = (x_1 + y_1, x_2 + y_2, x_3 + y_3)$$
therefore
$$x_1 + y_1 + 2(x_2 + y_2) +  3(x_3 + y_3) =
x_1 + 2x_2 + 3x_3 + y_1 + 2y_2 + 3y_3 = 0+ 0 = 0$$
therefore it is closed under addition
$$n(x_1, x_2, x_3) = (nx_1, nx_2, nx_3)$$
$$nx_1 + 2nx_2 + 3nx_3 = n(x_1 + 2x_2 + 3x_3 ) = 0n = 0$$
therefore it is closed under multiplication.

\textit{(b) $\{(x_1, x_2, x_3) \in F^3: x_1 + 2x_2 + 3x_3 = 4\}$}

It's not a subspace, because it does not contain zero.


\textit{(c) $\{(x_1, x_2, x_3) \in F^3: x_1 x_2 x_3 = 0\}$}

It's not a subspace, because
$$(0, 1, 1) + (1, 0, 0) = (1, 1, 1)$$
therefore it's not closed under addition.

\textit{(d) $\{(x_1, x_2, x_3) \in F^3: x_1  = 5x_3\}$}

It's a subspace, proof is the same as in (a), can be seen more clearly when we
rewrite constraint as
$$x_1 = 5x_3 \to x_1 + 0x_2 -5x_3 = 0$$

\section*{2}
\textit{Verify all the assertions in Example 1.35}

\textit{(a) if $b \in F$, then}
$$\{(x_1, x_2, x_3, x_4) \in F^4: x_3 = 5x_4 + b\}$$
\textit{is a subspace of $F^4$ if and only if $b = 0$}

If $b \neq 0$, then $0$ is not an element of this set.

Proving that it's a subspace when $b = 0$ is trivial

\textit{(b) The set of continous real-valued functions on the interval $[0, 1]$
  is a subspace of $R^{[0, 1]}$.}

$(kf) = kf$ by algebraic properties of continous functions.
If $f$ and $g$ are continous, then $(f + g)$ is continous as well by the same
property.
$f(x) = 0$ is continous because it's a constant functions.

By the way, same (probably) applies to a set of uniformly continous functions.

\textit{(c) The set of differentiable real-valued functions on $R$ is a
  subspace of $R^R$.}

Same deal, algebraic proerties imply linearity, adn zero is included.

\textit{(d) The set of differentiable real-valued functions $f$ on the
  interval $(0, 3)$ such that $f'(2) = b$ is a subspace of $R^{(0, 3)}$
  if and only if $b = 0$.}

Same deal as in previous one, $f'(2)$ needs to be equal to zero in order
to include zero. Previous part does not include it, because it does not
have specific restrictions on derivatives being particular values at particular places.

\textit{(e) The set of all sequences of complex numbers with limit $0$ is a
  subspace of $C^{\infty}$.}

Here we can take zero to be $(x_n) = 0$. Linearity is implied by aldebraic
properties of limits of sequences.

\section*{3}
\textit{Show that the set of differentiable real-valued functions $f$ on the
  interval $(-4, 4)$ such that $f'(1) = 3f(2)$ is a subspace of $R^{[-4, 4]}$.}

Zero is included here. Suppose that $f$ and $g$ are functions in given set.
It follows that
$$f'(1) + g'(1) = 3f(2) + 3g(2)$$
$$f'(1) + g'(1) = 3(f(2) + g(2))$$
$$(f + g)'(1) = 3(f + g)(2)$$
thus it's closed under addition.

$$(kf)'(1) = 3(kf)(2)$$
implies
$$kf'(1) = 3kf(2)$$
therefore it's closed under multiplication by scalar.
Therefore we can state that given subset is a vector subspace.

\section*{4}
analogous to previous

\section*{5}
\textit{Is $R^2$ a subspace of the complex vector space $C^2$?}

No, it's not closed under scalar multiplication.

\section*{6}
\textit{(a) Is }
$$\{(a, b, c) \in R^3: a^3 = b^3\}$$
\textit{a subspace if $R^3$?}

Yes. it is. $a^3 = b^3 \to a = b \to a - b = 0$, the rest of proof is trivial.

\textit{(b) Is }
$$\{(a, b, c) \in C^3: a^3 = b^3\}$$
\textit{a subspace if $C^3$?}

I want to say no to this one, example is
$$(1/2 + i\frac{\sqrt{3}}{2}, -1, 0) +
(1/2 - i\frac{\sqrt{3}}{2}, -1, 0) =
(1, -1, 0)$$
thus it's not closed under additon.

\section*{7}
\textit{Give an example of a nonemplty subset $U$ of $R^2$ such that $U$ is
  closed under addition and under additive inverses (meaning $-u \in U$
  whenever $u \in U$), but $U$ is not a subspace of $R^2$}

$Q^2$. On the other thouhgh, $Z$ will do as well.

\section*{8}
\textit{Give an example of a nonempty subset $U$ of $R^2$ such that $U$ is
  closed under scalar multiplication, but $U$ is not a subspace of $R^2$.}

Two lines through origin.

\section*{9}
\textit{A function is called periodic if there exists a positive number
  $p$ such that $f(x) = f(x + p)$ for all $x \in R$. Is the set of
  periodic functions from $R$ to $R$ a subspace of $R^R$? Explain. }

Zero is a periodic function. Set is
certainly closed under scalar multiplication.

Suppose that $f$ and $g$ are both periodic and $f$ has a period of $p1$
and $g$ has a period of $p2$. Thus if $p2/p1 \in I$,
then functions will be constantly out of phase, therefore the set is not
closed under addition. Thus this subset is not a subspace.

\section*{10}
\textit{Suppose $U_1$ and $U_2$ are subspaces of $V$. Prove that the
  intersection $U_1 \cap U_2$ is a subspace of $V$.}

Zero is included in any subspace, therefore zero is included.

Suppose that $u_1, u_2 \in U_1 \cap U_2$. It follows that for $z \in F$
$zu_1 \in U_1$ and $zu_1 \in U_2$ by closure of those two subspaces.
Therefore $zu_1 \in U_1 \cap U_2$ for any scalar, thus the set is
closed under scalar multiplication.

$u_1 + u_2 \in U_1$ and $u_1 + u_2 \in U_2$ by closure under addition for
both subspaces. Thus $u_1 + u_2 \in U_1 \cap U_2$ for any such vectors.
Therefore the set is closed under addition.

Thus the set satisfies all requirements to be a subspace. Therefore it is a
subspace.

\section*{11}
\textit{Prove that the interseection of every collection of subspace of $V$ is
  a subspace of $V$}

Intersection of two subspaces  is a subspace. Therefore by induction
intersection of any finite collection of subspaces is a subspace.

Suppose that $\Lambda$ is an arbitrary collection of subspaces.
Every subspace contains a zero element, therefore
$$0 \in \cap \Lambda$$

Any vector in $\cap \Lambda$ will be closed under scalar multiplication
for every $U \in \Lambda$. Thus, it will be contained in every
$U \in \Lambda$. Therefore it is contained in $\cap \Lambda$.


Any two  vectors in $\cap \Lambda$ will be closed under addition,
for every $U \in \Lambda$. Thus, their sum  will be contained in every
$U \in \Lambda$. Therefore it is contained in $\cap \Lambda$.

Thus $\cap \Lambda$ is a vector space.

\section*{12}
\textit{Prove that the union of two subspaces of $V$ is a subspace of $V$
  if and only if one of the subspaces is contained in the other.}



\end{document}