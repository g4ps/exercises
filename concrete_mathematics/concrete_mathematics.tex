\documentclass[11pt,oneside,titlepage]{book}
\title{My concrete mathematics exercises}
\usepackage{amsmath, amssymb}
\usepackage{geometry}
\usepackage{hyperref}
\author{Evgeny Markin}
\date{2023}

\begin{document}
\maketitle
\tableofcontents

\chapter*{Preface}

Exercises are from Concrete Mathematics, 2nd ed., by Graham, Knuth and Patashnik.

Probably going to cover warmup, basic and homework exercises, while the rest are
going to be left for the better times.

\chapter{Recurrent Problems}

\section{}

\textit{Too long of a text of exercise}

By saying that 2 through n horses have the same color, we imply that for any given
set of horses of length $n - 1$ we follow that they have the same color. But we've assumed
that for a given set (or even better - list) of horses $[1, n - 1]$, we've got that
if $x, y \in [1, n - 1]$, then $x$ and $y$ have the same color.

\section{}

\textit{Find the sortest sequence of moves that transfers a tower of $n$ disks from the
  left peg $A$ to the right peg $B$, if direct moves between $A$ and $B$ are disallowed. (Each
  move must be to or from the middle peg. As usual, a larger disk must never apper above
  a smaller one.)}

By doing some mental gymnastics we get that
$$n = 1 \to f(n) = 2$$
$$n = 2 \to f(n) = 8$$
$$n = 3 \to f(n) = 26$$
which doesn't give me much of a clue.

Then we conclude, that in order to move  n disks from $A$ to $B$, we need to move $n - 1$ disks
from $A$ to $B$, then move $n$'th disk to the middle peg, then move $n - 1$ disks back from $B$
to $A$, then move the $n$'th disk to its final place at the bottom of $B$, and to finish
it all we need to move $n - 1$ disks again from $A$ to $B$ (During this discuttion I realized,
that good guess for the initial values would be $3^n - 1$ ). Thus we can follow that
$$f(1) = 2$$
$$f(n) = f(n - 1) + 1 + f(n - 1) + 1 + f(n - 1) = 3f(n - 1) + 2$$

Thus let us prove that our guess is correct. We're going to do it by induction. Base case is
covered, therefore we can assume that our guess is true for $n - 1$. Thus
$$3f(n - 1) + 2 = 3* (3^{n - 1} - 1) + 2 = 3^n - 3 + 2 = 3^n - 1$$
as desired.

\section{}

\textit{Show that, in the process of transferring a tower under the restrictions of the preceding
  exercise, we will actually encounter every properly stacked arrangement of $n$ disks
  on three pegs.}

I think that we can even do this one by induction. Base case with 1 disk is clear, we gotta
move it firstly to the middle one, then to the last one, making it all the possible arrangements.
Because we use a language of "bottom disk" in further proof we can probably kick it up a notch
and make the case for 2 disks as base one, just in case that it matters.

Now assume that we get this property for the case of $n - 1$ disks. Then it follows that
there are 3 possible positions for the bottom disk, and for all of them we move all other disks
from first peg to the last peg, or vice versa, making all the possible combinaitons on the way.
Thus we can conclude that disks make all the possible arrangements in the case of $n$ disks,
providing us with the desired iteration.
n
\section{}

\textit{Are there any starting and ending configurations of $n$ disks on three pegs that
  are more than $2^n - 1$ moves apart, under Lucas's original rules?}

I want to say "no" on this one. Maybe some sort of a shift is the counterexample. It seems
like it isn't. Maybe we can somehow show that $2^n - 1$ is the  maximum.
Suppose that we've arranged
disks in order and then we proceed with the fact, that it takes $2T_{n - 1} + 1$ moves to move
$n$'th largest disk from its original place to some other place. Thus we can follow that
$2^n - 1$ is indeed the largest amount of moves, because it moves every disk from one
place to another.

\section{}

\textit{A Venn diagram with three overlapping circles is often used to illustrate the eight
  possible subsets associated with three given sets. Can the sixteen possibilities that arise
  with four given sets be illustrated by four overlapping circles?}

Nope. 

\section{}

\textit{Some of the regions defined by $n$ lines in the plane are infinite, while other are
  bounded. What's the maximum possible number of bounded regions?}

I think that it is the same as with any regions, but shifted by 3, because we need to bound the
first region, and then proceed with its dissection as per normal rules.

\section{}

\textit{Let $H(n) = J(n + 1) - J(n)$. Equation (1.8) tells us that $H(2n) = 2$ and
  $H(2n + 1) = J(2n + 2) - J(2n + 1) = (2J(n + 1) - 1) - (2J(n) + 1) = 2H(n) - 2$
  for all $n \geq 1$. Therefore it seems possible to prove that $H(n) = 2$ for all n,
  by induction on n. What's wrong here?}

$$H(2n) = J(2n + 1) - J(2n) = 2J(n) + 1 - 2J(n) + 1 = 2$$
So everything holds for $H(2n)$.

$$H(2n + 1) = J(2n + 2) - J(2n + 1) = J(2(n + 1)) - J(2n + 1) = $$
$$ = 2J(n + 1) - 1 - 2J(n) - 1 = 2(J(n + 1) - J(n)) - 2 = 2H(n) - 2$$
Thus our math holds for odd numbers, that are greater then 2.

This thing doesn't work on $n = 1$, therefore we need to have our base case to be set at $n = 2$.
Thus we're assuming that it's true for $n \geq 2$ and our induction hypothesis
states that if $n \geq 2$, then $H(n) = 2$. Thus we can't follow anything for the induciton
step, therefore the whole thing is false.


\section{}

\textit{Solve the recurrence }
$$Q_0 = a$$
$$Q_1 = b$$
$$Q_n = (1 + Q_{n - 1})/Q_{n - 2}$$
\textit{Assume that $Q_n \neq 0$ for all $n \geq 0$}

Let's try to extend this thing a bit
$$Q_2 = \frac{1 + b}{a}$$
$$Q_3 = \frac{1 + \frac{1 + b}{a}}{b} =
\frac{\frac{a + 1 + b}{a}}{b} = \frac{a + 1 + b}{ab}
$$
$$Q_4 = \frac{1 + \frac{a + 1 + b}{ab}}{\frac{1 + b}{a}} =
\frac{ab + a + 1 + b}{ab} \frac{1}{\frac{1 + b}{a}} =
\frac{ab + a + 1 + b}{\frac{ab(1 + b)}{a}} =
\frac{ab + a + 1 + b}{b(1 + b)} = \frac{a(b + 1) + 1(b + 1)}{b(1 + b)} =
$$
$$ = 
\frac{(a + 1)(b + 1)}{b(1 + b)} = \frac{a + 1}{b}
$$
$$Q_5 = \frac{1 + \frac{1 + a}{b}}{\frac{a + 1 + b}{ab}} =
\frac{\frac{(b + 1 + a)}{b}}{\frac{a + 1 + b}{ab}} = \frac{1}{b} ab = a
$$
$$Q_6 = \frac{1 + a}{\frac{a + 1}{b}} = b$$
Thus we come a full circle and conclude that
$$Q(n) = Q(n \mod 5)$$
and all of the others are computed. Good ol' brute force, nothing beats that.

\section{}

\textit{Sometimes it's possible to use induction backwards, proving things from $n$
  to $n - 1$ instead of vice versa! For example, consider the statement}
$$P(n): x_1 ...  x_n \leq \left(\frac{x_1 + ... + x_n}{n}\right)^n
\text{ if } x_1 , ...,  x_n \geq 0$$
\textit{This is true when $n = 2$, since $(x_1 + x_2)^2 - 4x_1x_2 = (x_1 - x_2)^2 \geq 0$.}


I'm assuming that $P(n): N \to bool$, which is a little unorthodox, but will do.

\textit{(a) By setting $x_n = (x_1 + ... + x_{n - 1})/(n - 1)$, prove that $P(n)$ implies
  $P(n - 1)$ whenever $n > 1$.}

Suppose that $P(n)$ is true. Then we follow that for any $x_1, ..., x_{n - 1} \geq 0$ and
$x_n = (x_1 + ... + x_{n - 1})/(n - 1)$
we've got that 
$$x_1  ... (x_1 + ... + x_{n - 1})/(n - 1) \leq \left(\frac{x_1 + ... + x_n}{n}\right)^n$$
$$\frac{x_1  ... (x_1 + ... + x_{n - 1})}{n - 1} \leq \left(\frac{x_1 + ... + x_{n - 1} +
    \frac{x_1 + ... + x_{n - 1}}{n - 1}}{n}\right)^n$$
$$\frac{x_1  ... (x_1 + ... + x_{n - 1})}{n - 1} \leq \left(\frac{
    \frac{(n - 1)(x_1 + ... + x_{n - 1}) + x_1 + ... + x_{n - 1}}{n - 1}}{n}\right)^n$$
$$\frac{x_1  ... (x_1 + ... + x_{n - 1})}{n - 1} \leq \left(\frac{
    \frac{(n - 1)(x_1 + ... + x_{n - 1}) + (x_1 + ... + x_{n - 1})}{n - 1}}{n}\right)^n$$
$$\frac{x_1  ... (x_1 + ... + x_{n - 1})}{n - 1} \leq \left(\frac{
    \frac{n(x_1 + ... + x_{n - 1})}{n - 1}}{n}\right)^n$$
$$\frac{x_1  ... (x_1 + ... + x_{n - 1})}{n - 1} \leq \left(
  \frac{x_1 + ... + x_{n - 1}}{n - 1}\right)^n$$
$$x_1  ... x_{n - 1}\frac{(x_1 + ... + x_{n - 1})}{n - 1} \leq \left(
  \frac{x_1 + ... + x_{n - 1}}{n - 1}\right)^n$$
$$x_1  ... x_{n - 1} \leq \left( \frac{x_1 + ... + x_{n - 1}}{n - 1}\right)^{n - 1}$$
which is precisely $P(n - 1)$. Thus we follow that $P(n) \to P(n - 1)$.

\textit{(b) Show that $P(n)$ and $P(2)$ imply $P(2n)$}

Suppose that $P(n)$ is true and $P(2)$ is also true (we've proven $P(2)$ beforehand).
Let $x_1, ..., x_n \geq 0$. Then we can follow that

$$x_1 ... x_n \leq \left(\frac{x_1 + ... + x_n}{n}\right)^n$$

By $P(2)$ we've got that
$$y_1 y_2 \leq \left(\frac{y_1 + y_2}{2}\right)^2$$
Suppose that we take a set of numbers $x_1, ..., x_{2n}$. Then we can divide this set into
two equaly sized halves and get that
$$(x_1 ... x_n) (x_{n + 1} ... x_{2n}) \leq
\left(\frac{(x_1 ... x_n) + (x_{n + 1} ... x_{2n})}{2}\right)^2$$
Since we've got $P(n)$ we can follow that
$$(x_1 ... x_n) \leq \left(\frac{x_1 + ... + x_n}{n}\right)^n$$
and
$$(x_{n + 1} ... x_{2n}) \leq \left(\frac{x_{n + 1} + ... + x_{2n}}{n}\right)^n$$

Thus we follow that
$$ (x_1 ... x_n)(x_{n + 1} ... x_{2n}) \leq \left(\frac{x_1 + ... + x_n}{n}\right)^n
\left(\frac{x_{n + 1} + ... + x_{2n}}{n}\right)^n$$
$$ x_1 ... x_{2n} \leq \frac{(x_1 + ... + x_n)^n}{n^n}
\frac{(x_{n + 1} + ... + x_{2n})^n}{n^n}$$
$$ x_1 ... x_{2n} \leq \frac{(x_1 + ... + x_n)^n(x_{n + 1} + ... + x_{2n})^n}{n^{2n}}$$
$$ x_1 ... x_{2n} \leq \left(\frac{(x_1 + ... + x_n)(x_{n + 1} + ... + x_{2n})}{n^{2}}\right)^n$$
and by applying $P(2)$ on the numerator of the inside fraction we get that 
$$ x_1 ... x_{2n} \leq \left(\frac{\left(\frac{x_1 + ... + x_{2n}}{2}\right)^2}{n^{2}}\right)^n$$
$$ x_1 ... x_{2n} \leq \left(\frac{\left(\frac{x_1 + ... + x_{2n}}{2}\right)}{n}\right)^{2n}$$
$$ x_1 ... x_{2n} \leq \left(\frac{x_1 + ... + x_{2n}}{2n}\right)^{2n}$$
as desired.

\textit{(c) Explain why this implies the truth of $P(n)$ for all $n$}

$P(1)$ and (b) implies that $P(n)$ is true for $n = 2^k$ for any $k \in N$ by continous application
of (b). By (a) we get that
if $n < k$ and $P(k)$ is true,
then $P(n)$ is true as well. Thus we can follow that for any $n \in N$
there exists $k \in N$ such that $n < 2^k$, and therefore $P(n)$ is true, as desired.

\section{}

\textit{Let $Q_n$ be the minimum number of moves needed to transfer a tower of $n$ disks
  from $A$ to $B$ if all moves must be clockwise -- that is, from $A$ to $B$, or from $B$ to
  the other peg, or from the other peg to $A$. Also, let $R_n$ be the minimum number of moves
  needed to go from $B$ back to $A$ under this restriction. Prove that }

$$Q_n =
\begin{cases}
  0 \text{ if } n = 0 \\
  2R_{n - 1}  + 1 \text{ if } n > 0
\end{cases}
$$
and
$$R_n =
\begin{cases}
  0 \text{ if } n = 0 \\
  Q_n + Q_{n - 1} + 1 \text{ if } n > 0
\end{cases}
$$

For the remainder of the exercise let us denote the other peg as C

If we've got no disks, then we don't need to momve nothing, therefore we've got the case for the
zero.

Let's try firstly to handle some of the cases with low number of disks. If we want to move
just 1 disk, then we have got to
$$A_1 \to B_1$$
and that's the case. To move it back to the original peg we've got to
$$B_1 \to C_1 \to A_1$$ 
Thus $Q_1 = 1 = 2R_0 + 1$ and
$R_1 = 2 = Q_0 + Q_1 + 1$. Therefore our assumption seems to be working
for 1 disk.

Now suppose that we've gotta move $n$ disks. Firstly, we have got to follow that
if we  re-label the pegs as such
$$A \to B, B \to C, C \to A$$
then our rules about moving the disks clockwise  still makes sence. Thus we can follow that
it takes $R_{k}$ moves to move $k$ disks counter-clockwise (i.e. $A \to C$, $C \to B$,
$B \to A$.) and it takes $Q_k$ moves to move $k$ disks clockwise ($A \to B$, $B \to C$, $C \to A$).
Thus we can follow that if formulas for $R_n$ and $Q_n$ hold for $n - 1$, then we
can move $n - 1$ disks from $A$ to $C$ in $R_{n - 1}$ moves, then move the last disk from
$A$ to $B$ in one move, and then move $n - 1$ disks from $B$ to $A$ in $R_{n - 1}$ moves.
$$
\begin{cases}
n - 1, A \to C: R_{n - 1}\\
1, A \to B: 1\\
n, C \to B: R_{n - 1}\\  
\end{cases}
$$
Thus we follow that it takes
$$Q_n = 2R_{n - 1} + 1$$
moves to move $n$ disks from $A$ to $B$. Thus we follow that if $Q_n$ and $R_n$ hold for
$n - 1$, then $Q_n$ holds for $n$. 

For $R$ let's try to expand formula a bit:
$$R_n = Q_n + Q_{n - 1} + 1 = 2R_{n - 1} + 1 + Q_{n - 1} + 1$$
therefore suppose that $R_n$ and $Q_n$ holds for $n - 1$, and we've got $n$ disks on $B$.
Then we can use those moves
$$
\begin{cases}
  n - 1, B \to A: R_{n - 1} \\
  1, B \to C: 1 \\
  n - 1, A \to B: Q_{n - 1} \\
  1, C \to A: 1 \\
  n - 1, B \to A: R_{n - 1}
\end{cases}
$$
that sum up to our desired formula.

Thus we can conclude, that if our formulas hold for $n - 1$, then $Q_n$ holds for $n$ and
$R_n$ also holds for $n$, therefore by induction we get that for $n \in N$ our formulas hold, as
desired.


\section{}

\textit{A Double Tower of Hanoi contains 2n disks of n different sizes, two of each size. As
  usual, we're required to move only one disk at a line, without putting a larger one
  over a smaller one.}

\textit{(a) How many moves does it take to transfer a double tower from one peg to
  another, if disks of equal size are indistinguishable from each other?}

Naive guess tells me that it takes a double of moves, that we had in the original problem.
The reasoning is simple: it's the same problem, but we have got to move each "disk" two times
to effectively put it in the other place.

\textit{(b) is a bonus problem, therefore I'll skip it for a while}

\section{}

\textit{Let's generalze exercise 11a even further, by assuming that there are n different
  sizes of disks and exactly $m_k$ disks of size $k$. Determine
  $A(m_1, ..., m_k)$, the minimum number of moves needed to transfer a tower when equal-size
  disks are considered to be indistinguishable.}

The number of disks doesn't affect the overall strategy, therefore the number is determined
recursively with a similar function.


\section{}

\textit{What's the maximum number of regions definable by n zig-zag lines, each of which consists
  of two which consists of two parallel infinite half-lines joined by straight segment?}


Firslty, let's write the first couple of values that we've got 
$$ZZ_0 = 1$$
$$ZZ_1 = 2$$
$$ZZ_2 = 12 \text{ (comes from the picture in the exercise)}$$

One proposition, which was driven by the reasoning in the lines and zigs example in the chapter,
is that number of regions can be connected to the
number of intersections, that those zig-zag lines make.
Given that one zig-zag consists of two parallel half-lines, we follow that parallel half-lines
of two distinct zig-zags can cross each other in at most four different points. We also
can state that cross-lines can cross other zig-zags half-lines in at most two places, and
can cross the other's straight bar as well, bringing the number of intersections of cross bars
to 5. Thus the total number of intersections on any two zig-zags is 9, which isn't really
helpful.

Other idea that popped up is that we can extend two half lines, so that the cross bar becomes
as long as possible. This way in some regions it's gonna look like we've got three parallel
lines instead of the zig-zags, and we can apply the rules for crossing just three parallel lines.

Since this exercise is not an exercise from the topology or geometry book, but an exercise from
counting book and from the chapter on recurrences, we can probably assume that
the number of lines is determined by the values of previous $n$. Since standart lines
produce $n$ new regions in the process, we can  assume that we get the same new regions, as
with lines, and in addition in the points of intersection we get 8 new regions (counted them on
a sketch) in
$(n - 1)$ intersections with previous zig-zags. Thus we can produce a reccurence
$$ZZ_0 = 1$$
$$ZZ_n = ZZ_{n - 1} + n + 8(n - 1) = ZZ_{n - 1} + 9n - 8$$
which gives us values
$$1, 2, 12, 31, 59, 96, 142, 197, 261, 334, 416, 507, 607, 716, 834, ...$$
which looks like an answer.

Idk if we need to produce a closed form for this recurrence, might do it for fun later; TODO

\section{}

\textit{How many pieces of cheese can you obtain from a single thick piece by making five
  straight slices? (The cheese must stay in its original position while you do all the cutting, and
  each slice must correspond to a plane in 3D). Find a recurrence relation for $P_n$, the
  maximum number of three-dimentional that can be defined by $n$ different planes.}

That problem is probably related in the interesting way to a problem with the lines.

Let's try to get the first couple of numbers:
$$P_0 = 1$$
$$P_1 = 2$$
$$P_2 = 4$$
$$P_3 = 8$$
it seems once again that we're doubling the number of regions with every slice, but something
tells me that we can do it for so long. It seems that for the 4th cut we can't double the number
of slices, and it becomes something more complicated.

Can't seem to reason through this one; skipping for now

\section{}

\textit{Josephus has a friend who was saved by getting into the next-to-last position. What
  is $I(n)$, the number of the penultimate survivor when every second person is executed.}

Let's try to reason the same way they did in the chapter.

For $n = 0, 1, 2$ we don't have a reasonable answer, thus let us procced to 3.
$$I(3) = 1; I(4) = 3; I(5) = 5; I(6) = 1$$
and  so on. For the case of 6 we can clearly notice  the same thing that kicked in in the
original Josephusd problem, specifically that
$$I(2n) = 2I(n) - 1$$
and for odd numbers we've got
$$I(2n + 1) = 2I(n) + 1$$
when we plug this recursion in the computer, we get an interesting thing: we get a recursion,
that is simular to the original pattern in Josephus pronblem, but now we get chunks
of $3 * 2^{n - 1}$ of sequential odd numbers.

Thus let us get the desired recursion: we firslty need to go back to some of the more basic cases,
then just 3. Walking backwarkds with our recursion, we get that
$$I(2) = 2; I(1) = 1.5$$
I like integers more then I'd like to admit, therefore I'll ditch the case with $I(1)$; it's
also false if we walk back from 3. Thus I'll define ny recursion as
$$I(2) = 2$$
$$I(3) = 1$$
$$I(2n) = 2I(n) - 1$$
$$I(2n + 1) = 2I(n) + 1$$

\section{}

\textit{Use the repertoire method to solve the general four-parameter recurrence}
$$g(1) = a$$
$$g(2n + j) = 3g(n) + cn + b_j \text{ for } j = 0, 1 \text{ and } n \geq 1$$

TODO

\end{document}

%%% Local Variables:
%%% mode: latex
%%% TeX-master: t
%%% End:
