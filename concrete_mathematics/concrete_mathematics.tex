\documentclass[11pt,oneside,titlepage]{book}
\title{My concrete mathematics exercises}
\usepackage{amsmath, amssymb}
\usepackage{geometry}
\usepackage{hyperref}
\author{Evgeny Markin}
\date{2023}

\begin{document}
\maketitle
\tableofcontents

\chapter{Recurrent Problems}

\section{}

\textit{Too long of a text of exercise}

By saying that 2 through n horses have the same color, we imply that for any given
set of horses of length $n - 1$ we follow that they have the same color. But we've assumed
that for a given set (or even better - list) of horses $[1, n - 1]$, we've got that
if $x, y \in [1, n - 1]$, then $x$ and $y$ have the same color.

\section{}

\textit{Find the sortest sequence of moves that transfers a tower of $n$ disks from the
  left peg $A$ to the right peg $B$, if direct moves between $A$ and $B$ are disallowed. (Each
  move must be to or from the middle peg. As usual, a larger disk must never apper above
  a smaller one.)}

By doing some mental gymnastics we get that
$$n = 1 \to f(n) = 2$$
$$n = 2 \to f(n) = 8$$
$$n = 3 \to f(n) = 26$$
which doesn't give me much of a clue.

Then we conclude, that in order to move  n disks from $A$ to $B$, we need to move $n - 1$ disks
from $A$ to $B$, then move $n$'th disk to the middle peg, then move $n - 1$ disks back from $B$
to $A$, then move the $n$'th disk to its final place at the bottom of $B$, and to finish
it all we need to move $n - 1$ disks again from $A$ to $B$ (During this discuttion I realized,
that good guess for the initial values would be $3^n - 1$ ). Thus we can follow that
$$f(1) = 2$$
$$f(n) = f(n - 1) + 1 + f(n - 1) + 1 + f(n - 1) = 3f(n - 1) + 2$$

Thus let us prove that our guess is correct. We're going to do it by induction. Base case is
covered, therefore we can assume that our guess is true for $n - 1$. Thus
$$3f(n - 1) + 2 = 3* (3^{n - 1} - 1) + 2 = 3^n - 3 + 2 = 3^n - 1$$
as desired.

\section{}

\textit{Show that, in the process of transferring a tower under the restrictions of the preceding
  exercise, we will actually encounter every properly stacked arrangement of $n$ disks
  on three pegs.}

I think that we can even do this one by induction. Base case with 1 disk is clear, we gotta
move it firstly to the middle one, then to the last one, making it all the possible arrangements.
Because we use a language of "bottom disk" in further proof we can probably kick it up a notch
and make the case for 2 disks as base one, just in case that it matters.

Now assume that we get this property for the case of $n - 1$ disks. Then it follows that
there are 3 possible positions for the bottom disk, and for all of them we move all other disks
from first peg to the last peg, or vice versa, making all the possible combinaitons on the way.
Thus we can conclude that disks make all the possible arrangements in the case of $n$ disks,
providing us with the desired iteration.
n
\section{}

\textit{Are there any starting and ending configurations of $n$ disks on three pegs that
  are more than $2^n - 1$ moves apart, under Lucas's original rules?}

I want to say "no" on this one. Maybe some sort of a shift is the counterexample. It seems
like it isn't. Maybe we can somehow show that $2^n - 1$ is the  maximum.
Suppose that we've arranged
disks in order and then we proceed with the fact, that it takes $2T_{n - 1} + 1$ moves to move
$n$'th largest disk from its original place to some other place. Thus we can follow that
$2^n - 1$ is indeed the largest amount of moves, because it moves every disk from one
place to another.

\section{}

\textit{A Venn diagram with three overlapping circles is often used to illustrate the eight
  possible subsets associated with three given sets. Can the sixteen possibilities that arise
  with four given sets be illustrated by four overlapping circles?}

Nope. 

\section{}

\textit{Some of the regions defined by $n$ lines in the plane are infinite, while other are
  bounded. What's the maximum possible number of bounded regions?}

I think that it is the same as with any regions, but shifted by 3, because we need to bound the
first region, and then proceed with its dissection as per normal rules.

\end{document}
