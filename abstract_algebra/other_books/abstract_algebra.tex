\documentclass[11pt,oneside,titlepage]{book}
\title{My abstract algebra exercises}
\usepackage{amsmath, amssymb}
\usepackage{geometry}
\usepackage{hyperref}
\author{Evgeny Markin}
\date{2023}

\DeclareMathOperator \map {\mathcal {L}}
\DeclareMathOperator \pow {\mathcal {P}}
\DeclareMathOperator \ns {null}
\DeclareMathOperator \range {range}
\DeclareMathOperator \inv {^{-1}}
\DeclareMathOperator \Span {span}
\DeclareMathOperator \imp {\Rightarrow}
\DeclareMathOperator \lra {\Leftrightarrow}
\DeclareMathOperator \eqv {\Leftrightarrow}
\DeclareMathOperator \la {\Leftarrow}
\DeclareMathOperator \ra {\Rightarrow}
\DeclareMathOperator \true {true}
\DeclareMathOperator \false {false}
\DeclareMathOperator \dom {dom}
\DeclareMathOperator \ran {ran}
\newcommand{\eangle}[1]{\langle #1 \rangle}
\newcommand{\set}[1]{\{ #1 \}}


\begin{document}
\maketitle
\tableofcontents

\part{Preliminaries}

\chapter{Relations and Functions}

\chapter{The Integers and Modular Arithmetic}

\part{Groups}

\chapter{Introduction to Groups}

\section{An Important Example}

\subsection{}

\textit{In $S_4$, let $\sigma = \
  \begin{pmatrix}
    1 & 2 & 3 & 4 \\
    3 & 1 & 4 & 2
  \end{pmatrix}
  $,
  and $\tau =
  \begin{pmatrix}
    1 & 2 & 3 & 4 \\
    3 & 4 & 1 & 2
  \end{pmatrix}
  $. Calculate $\sigma \tau$, $\tau \sigma$ and $\sigma \inv$.
}

$$\sigma \tau =
\begin{pmatrix}
  1 & 2 & 3 & 4\\
  4 & 2 & 3 & 1 
\end{pmatrix}
$$
$$ \tau \sigma =
\begin{pmatrix}
l  1 & 2 & 3 & 4\\
  1 & 3 & 2 & 4 
\end{pmatrix}
$$
$$\sigma \inv =
\begin{pmatrix}
  1 & 2 & 3 & 4 \\
  2 & 4 & 1 & 3
\end{pmatrix}
$$

\subsection{}

\textit{In $S_5$, let $\sigma =
  \begin{pmatrix}
    1 & 2 & 3 & 4 & 5 \\
    5 & 3 & 2 & 1 & 4
  \end{pmatrix}
  $
  and
  $\tau =
  \begin{pmatrix}
    1 & 2 & 3 & 4 & 5 \\
    2 & 4 & 1 & 3 & 5
  \end{pmatrix}
  $
  calculate $\sigma \tau \sigma$, $\sigma \sigma \tau$, $\sigma \inv$.
}

$$\sigma \tau \sigma =
\begin{pmatrix}
  1 & 2 & 3 & 4 & 5 \\
  4 & 5 & 1 & 3 & 2
\end{pmatrix}
$$
$$\sigma \sigma \tau =
\begin{pmatrix}
  1 & 2 & 3 & 4 & 5 \\
  2 & 5 & 4 & 3 & 1 
\end{pmatrix}
$$
$$\sigma \inv =
\begin{pmatrix}
  1 & 2 & 3 & 4 & 5 \\
  4 & 3 & 2 & 5 & 1
\end{pmatrix}
$$

\subsection{}

\textit{How mamy permutations are there in $S_n$? In $S_5$, how many permutations $\alpha$
  satisfy $\alpha(2) = 2$?}

We can follow that there are $n!$ permutations total, and if we've got a restriction
$\alpha(2) = 2$, then we've got $(n - 1)!$ permutation. For the case $S_5$ it means
that there are $4! = 24$ such permutations.

\subsection{}

\textit{Let $H$ be the set of all permutations $\alpha \in S_5$ satisfying $\alpha(2) = 2$.
  Which of the properties of closure, associativity, identit, inverses does $H$ enjoy
  under composition?}

All of them

\subsection{}

\textit{Consider the set of all functions from $6$ to $6$. Which of the ...}

Everything other then inverse

\subsection{}

\textit{Let $G$ be the set of all ...}

All of them

\section{Groups}

\subsection{}

\textit{Give group tables for following additive grops: $Z_3$, $Z_3 \times Z_2$}

$$
\begin{tabular}{c | c c c}
   &  0 & 1 & 2 \\
  \hline
  0 & 0 & 1 & 2 \\
  1 & 1 & 2 & 0 \\
  2 & 2 & 0 & 1 \\
  \hline
\end{tabular}
$$

last one is ommited

\subsection{}

\textit{Give group tables for the following groups: $U(12), S_3$}

We follow that $U(12) = \set{1, 5, 7, 11}$. THus
$$
\begin{tabular}{c | c c c c}
   &  1 & 5 & 7 & 11 \\
  \hline
  1 & 1 & 5 & 7 & 11 \\
  5 & 5 & 1 & 11 & 7 \\
  7 & 7 & 11 & 1 & 5 \\
  11 & 11 & 7 & 5 & 1 \\
  \hline
\end{tabular}
$$

One of the programs in progs folder produces desired table for $S_3$ (and can
produce one for any $S_n$ for that matter).

\subsection{}

\textit{Show that $G \times H$ is abelian iff $G$ and $H$ are both abelian}

Was proven in dummit and foote, check 1.1.29

\textit{Rest of the exercises in this section were either already proven in D\&F, are trivial, or
could be solved at a later time if I encounter some gaps in the theory.}

\section{}
\section{}
\section{}

\section{Cyclic Groups}

\subsection{}

\textit{Let $G = \eangle{a}$ be a cyclic group of order 12. List every subgroup of $G$.
List every group of $Z_{12}$}

12's divisors are $\set{1, 2, 3, 4, 6, 12}$, therefore subgroups of $G$ are
$\eangle{a^i}$ for $i \in \set{0, 1, 2, 3, 4, 6}$

Since $Z_{12}$ is cyclic, we follow that  $\eangle{[0, 1, 2, 3, 4, 6]}$
are the subgroups of $Z_{12}$.

\subsection{}

\textit{Let $G = \eangle{a}$ be a cyclic group of order 120. List all of the groups of order 120.
  List all of the elements of order 12 in $G$. }

Divisors of $120$ are $\set{1, 2, 3, 4, 5, 6, 8, 10, 12, 24, 60, 120}$, thus
we can state that subgroups of a cyclic group are $a$ to powers of those numbers

According to the theorems, there should be $\phi(12) = 4$ elements of order 12.
All of them lie in a subgroup $\eangle{a^{120 / 12}} = \eangle{a^{10}}$
and are in form $(a^{10})^k$ where $k \in {1, 5, 7, 11}$. 

\textit{How many element of order 12 are there in a cyclic group of order 1200?}

Also 4.

\subsection{}

\textit{Let $p$ be a prime  and $n$ a positive integer. Show that $\phi(p^n) = p^n - p^{n - 1}$}

If $j \in Z_+$ is such that $j = pi$ for some $i \in Z_+$, then we follow that
$(p^n, j) = p$, therefore they are not relatively prime. Suppose that $(p^n, j) = 1$ for some
$j \in Z_+$. Let $S$ be a multiset of prime divisors of $p^nN$ and $T$ be a multiset
of divisors of $j$. Then we follow that $S \cap T = \emptyset$, since otherwise we
would've had that $j$ is a multiple of $p$, which is not relatively prime to $p^n$.
Thus we follow that the set of not relatively prime numbers to  $p^n$ is equal to
the set of multiples of $p$.

We can follow that there are pricicely $p^{n - 1}$ of multiples of $p$ that are
less or equal to $p^n$ (don't think that we need to prove that),
therefore the total amount of numbers that are less or equal to
$p^n$, which are relatively prime to $p^n$ is $p^n - p^{n - 1}$, as desired.

\subsection{}

\textit{Find all positive integers $n$ such that $|U(n)| = 24$.}

We can follow that $\phi(n)$ is an function that tends to infinity
(i.e. for every $n \in Z_+$ there exists $j \in Z_+$ such that
$m > n$ implies that $\phi(m) > j$) since $\phi(n)$ is larger than
the number of prime numbers that is in the set $Z_+ \cap [1, n)$.
Therefore we conclude that there is an upper bound for a number of
numbers $n$ such that $\phi(n) = 24$.

Brute-force shows that those numbers are
$$35, 39, 45, 52, 56, 70, 72, 78, 84, 90$$

Can't come up with a better answer than that, but I'm sure that it's there.

\subsection{}

\textit{Let $G$ be a nonabelian group. If $H$ and $K$ are cyclic subgroups of $G$,
  does it follow that $H \cap K$ is also a cyclyc subgroup? Prove that it does,
  or provide a counterexample.}

We follow that every subgroup has an identity in it, thus $e \in H \cap K$.
Suppose that $j \in H \cap K$. We follow that $j \in H \land j \in K$. Since $H$
and $K$ are both subgroups, we follow that $j \inv \in H \land j\inv \in K$. Thus
$j \inv \in H \cap K$. Therefore $H \cap K$ is closed under inverses.
We can follow also by the same logic that $j, l \in H \cap K$ implies that
$jl \in H \cap K$. Therefore we can conclude that $H \cap K$ is a subgroup.

We can follow that if $H \cap K = \set{e}$, then it's cyclic. We can follow that
$H \cap K$ can be not only a trivial subgroup by setting $H = K$. Suppose that
$H \cap K \neq \set{e}$. By the fact that both $H$ and $K$ are cyclic we follow that
$H \cap K = \set{a^i: i \in \text{ some subset of } Z_+}$.
Since $H \cap K \neq \set{e}$, we follow that there exists an element $a \in G$ and
two sets $H', K' \in \pow(Z_+)$ such that $H = \set{a^i: i \in H'}$ and
$K = \set{a^i: i \in K'}$. Since both $H$ and $K$ are cyclic we follow that
both $H'$ and $K'$ are the sets of multiples of some number. Thus
$H' \cap K'$ is a set of multiples of some number as well (proof ommited). Thus we
follow that $H \cap K = \set{a^i: i \in H' \cap K'}$ is a cyclic group as well.

\subsection{}

\textit{Let $G = \eangle{a}$ be an infinite cyclic. If $m$ and $n$ are positive integers,
find a generator for $\eangle{a^m} \cap \eangle{a^n}$.}

We can follow pretty easily that $\eangle{a^m} \cap \eangle{a^n} =
\eangle{a^{lcm(m, n)}}$

\subsection{}

\textit{Let $n$ be a positive integer and let $T$ be the set of positive integers
  that divide $n$. Show that $\sum_{k \in T}{\phi(k)} = n$.}

For 12 we've got
$$T = \set{1, 2, 3, 4, 6, 12}$$
$$\phi(1) = 1$$
$$\phi(2) = 1$$
$$\phi(3) = 2$$
$$\phi(4) = 2$$
$$\phi(6) = 2$$
$$\phi(12) = 4$$
and we follow that result works.

\section{Cosets and Lagrange's Theorem}

\subsection{}

\textit{For each group $G$ and subgroup $H$, find all the left cosets and right cosets
  of $H$ in $G$.}

\textit{1. $G = Z, H = 4Z$.}

We follow thaht $0 + H = 4Z = H$, $1 + H = \set{1 + x: x \in Z}$, and so on for $3 + H$.
Since the group is abelian, we follow that right cosets are the same.

\subsection{}

\textit{Let $G$ be a group whose order is the product of two (not necessarily distinct)
  primes. Show that every proper subgroup of $G$ is cyclic}

We follow that order of any given proper subgroup is equal to one of those primes, or 1. 
This implies that this subgroup is cyclic, as desired.

\subsection{}

\textit{Let $G$ be a group of order $p^n$ for some prime $p$ and positive integer $n$.
  Show that $G$ has an element of order $p$.}

We follow that any proper subgroup is some power of $p$. By induction we can conclude that
such an element exists.

\subsection{}

\textit{Let $G$ be a group having a subgroup $H$ of order 28 and a
  subgroup $K$ of order $65$. Show that $H \cap K = \set{e}$.}

We follow that
$$28 = 2 * 2 * 7$$
$$65 = 13 * 5$$
since they don't have no commot prime multiples, we follow that $H \cap K$'s only order
as a subgroup of both can be only 1. Since every group has an identity, we
conclude the desired result.

\subsection{}

\textit{blah blah blah}

$$lcm(1, 2, ..., 10)$$

\chapter{Factor Groups and Homomorphisms}

\section{Normal Subgroups}

\subsection{}

\textit{Is each of the following sets a normal subgroup of $GL_2(R)$? }

\textit{1. $H = \set{A \in GL_2(R): \det(A) \in Q}$}

We can follow that this thing is a subgroup by the properties of
determinants of compositions of matrices and the fact that $Q \setminus \set{0}$ is closed
under multiplication.

We follow that for all $a \in GL_2(R)$ and $h \in H$ we've got that
$$\det(a\inv * h * a) = \det(a\inv) * \det(h) * \det(a) =
\det(a\inv) * \det(a) * \det(h) = \det(I) * \det(h) = \det(h)$$
and thus we follow that s$H$ is a normal subgroup by  one of the equivalencees in
of the normal group.

\textit{2. the set of diagonal matrices in $GL_2(R)$.}

We can follow that the thing is a subgroup my matrix identities and whatnot
(identity is diagonal, inverse of diagonal is diagonal and composition of diagonal
is diagonal; for justification GOTO linear algebra course, chapter 6 or 7)
We can't follow that it's a normal group though, since we can set
$$
a =
\begin{pmatrix}
  1 & 2 \\
  1 & 2
\end{pmatrix}
$$
$$
h =
\begin{pmatrix}
  1 & 0 \\
  0 & 2
\end{pmatrix}
$$
which gives us that $a\inv h a$ is not in $H$ (details are ommited put can be supplimented
easily if you feel that you don't have nothing to do).

\subsection{}

\textit{Find every normal subgroup of $S_3$.}

We follow that $S_3$ is itself  a normal subgroup. Every sugroup of index 2 is normal,
thus we follow that in this case that means that every subgroup of size 3 is normal.
Every subgroup of length 1 must contain the identity, and thus we follow  that the
only subgroup of size 1 is the one that contains identity. Therefore by
Lagrange's theorem we follow that the only subgroups left are the ones that have size 2.
Let $U$ be a subgroup of length 2. We follow that it must contain the identity,
and thus they have the form
$$U  = \set{e, s}$$
where $s \in S_3 \setminus \set{e}$. There are 5 such elements, and thus we can check them
by hand.

Since $U$ has to be a subgroup, we follow that $s \in S_3$ must be such that
$s = s \inv$. Thus our search is limited to cycles
$$(2, 3), (1, 2), (1, 3)$$
We follow that
$$(1, 2)\inv \circ (2, 3) \circ (1, 2) = (1, 2) \circ (2, 3) \circ (1, 2) = (1, 3)$$
$$(2, 3)\inv \circ (1, 2) \circ (2, 3) = (2, 3) \circ (1, 2) \circ (2, 3) = (1, 3)$$
$$(2, 3)\inv \circ (1, 3) \circ (2, 3) = (2, 3) \circ (1, 3) \circ (2, 3) = (1, 2)$$
thus we follow that none of those are normal. Therefore we conclude that the only normal
subgroups of $S_3$ are the ones with size $6, 3$ and $1$.

\subsection{}

\textit{If $N$ is a normal subgroup of $G$, and $|N| = 2$, show that $N \leq Z(G)$}

We follow that
$$(\forall a \in G)(\forall h \in H)(a\inv h a \in H)$$
since $|H| = 2$, we follow that $H = \set{e, b}$ for some $a \in G \setminus \set{e}$.
Let $x \in H$.  If $x = e$, then we follow that $x \in Z(G)$. If $x = b$, then we follow that
$$a\inv b a \in H \lra a\inv b a = e \lor a\inv b a = b \lra
b a = a e \lor b a = a b$$
Since $b \neq e$, we follow that $ba \neq a = ae$, thus we follow that for all $a \in G$
we've got that $ba = ab$. Thus we follow that $x \in Z(G)$. Therefore we follow that
$H \subseteq Z(G)$, and thus $H \leq Z(G)$, since the operation is the same, as desired.

\subsection{}

\textit{Let $N$ be a normal subgroup of $G$. Let $H$ be the set of all
  elements $h$ of $G$ such that
  $hn = nh$ for all $n \in N$. Show that $H$ is a normal subgroup of $G$.}

Proof that $H$ is a subgroup is trivial and therefore ommited.

Let $a \in G$. We follow that for all $n \in N$,  $a\inv n a \in N$.
Since $N$ is normal in $G$, we follow that $N = a\inv N a$. Thus 
for given $a \in G$ and  $n \in N$ there exists
$n' \in N$ such that $n = a\inv n' a$. Thus $an = n'a$.

We want to show that if $h \in H$, then $a \inv h a \in H$. We follow that
$a \inv h a \in H$ if and only if for all $n \in N$ we've got that
$$n (a \inv h a) = (a \inv h a) n$$
$$(a \inv h a) = n\inv (a \inv h a) n$$
$$(a \inv h a) = (an) \inv h (a n)$$
$$(a \inv h a) = (n'a)\inv h (n'a)$$
$$(a \inv h a) = a\inv n'\inv h n'a$$
$$(a \inv h a) = a\inv n'\inv n' ha$$
$$(a \inv h a) = a\inv e ha$$
$$a \inv h a = a\inv  ha$$
as desired.

\subsection{}

\textit{Show that the intersection of two normal subgroups of $G$ is also a normal
  subgroup. Then extend this to show that if $N_i$ is a normal subgroup of $G$ for every $i$
  in some set $T$, then $\bigcap_{i \in T}{N_i}$ is a normal subgroup of $G$.}

I think that I've shown earlier (maybe in another document) that $\bigcap_{i \in T}{N_i}$
is a subgroup. If not, then showing that is pretty trivial.

We firsly state here explicitly that $T$ is nonempty, otherwise $\bigcap$ is not defined.
Let $x \in \bigcap_{i \in T}{N_i}$. This means that
$$(\forall i \in T)(x \in N_i)$$
Since $N_i$ is normal for every $i \in T$, we follow that for all $a \in G$ we've got that
$a\inv x a \in N_i$. Thus
$$(\forall a \in G)(\forall i \in T)(a\inv x a \in N_i)$$
which is equivalent to 
$$(\forall a \in G)(a\inv x a \in \bigcap_{i \in T}{N_i})$$
thus we conclude that if $x \in \bigcap_{i \in T}{N_i}$, then $a \inv x a \in \bigcap_{i \in T}{N_i}$.
Therefore for all $x \in \bigcap_{i \in T}{N_i}$ and all $a \in G$ we've got that
$a \inv x a \in \bigcap_{i \in T}{N_i}$. Therefore $\bigcap_{i \in T}{N_i}$ is normal.

\subsection{}

\textit{Let $N_1 \leq N_2 \leq N_3 \leq  ...$ be normal subgroups of $G$. Show that
  $\bigcup_{i = 1}^\infty{N_i}$ is a normal subgroup of $G$.}

This exexrcise is pretty much the same as the previous one, except that
we might have to use a different quantifier.

\subsection{}

\textit{Let $G$ be a group having exactly one subgroup $H$ of order $n$. Show that $H$ is
  normal in $G$.}

Suppose that $H$ is not normal. Then we follow that $a\inv H a \neq H$.
Since $a \inv H a$ is a subgorup with $|a \inv H a| = |H| = n$. Thus we follow that
there are at least two disctinct groups of order $n$, which is a contradiction.

\subsection{}

\textit{Let $G = H \times K$. If $N$ and $L$ are normal subgroups of $H$ and $K$
  respectively, show that $N \times L$ is a normal subgroup of $G$. Is every normal
  subgroup of $G$ of this form?}

Let $a \in G$ be arbitrary.  We follow that $a = \eangle{h', k'}$ for some $h' \in H, k' \in K$.
We follow that $a \inv = \eangle{h'\inv, k'\inv}$. Let $\eangle{n, l} \in N \times L$
be also arbitrary. 
We follow that $h'\inv n h' \in N$ and $k'\inv l k' \in L$, and thus
$\eangle{h'\inv n h', k'\inv l k'} \in N \times L$. Therefore we follow that $N \times L$
is normal, as desired.

Rule of the thumb is that if someone asks you an open question in a math book, then the
answer is no, therefore we want to find a contradiction.
Let $H = K = Z_2$. Let $U = \set{\eangle{1, 1}, \eangle{0, 0}}$. We folloow that since $|U| = 2$,
index of $U$ is $2$, and thus it is normal.

\subsection{}

\textit{Suppose that $H$ is a subgroup of $G$ and $a\inv b\inv a b \in H$ for all $a, b \in G$.
  Show that $H$ is normal.}

Let $c \in G$ and $h \in H$ be arbitrary. We follow that $h, h\inv \in G$ by the
fact that $H$ is a subgroup. Thus
$$c\inv (h\inv) \inv c (h\inv) \in H$$
by property of $H$. Thus
$$c\inv h  c (h\inv) \in H$$
Since $h \in H$, we follow that 
$$c\inv h  c (h\inv) h \in H$$
thus
$$c\inv h  c \in H$$
thus $H$ is normal, as desired.

\subsection{}

\textit{Let $H$ and $K$ be subgroups of $G$. Show that $HK$ is a subgroup if and only if
  $HK = KH$.}

Assume that $HK$ is a subgroup. We follow that since $e \in K$ that $he = h \in HK$ for all
$h \in H$. Thus $H \leq HK$. We also follow by the same logic that $K \leq HK$.
Let $j \in KH$. We follow that $j = k'h'$ for some $h' \in H$ and $k' \in K$. 
Since $H \leq HK$ and $K \leq HK$, we follow that
$k' \in HK$ and $h' \in HK$, thus $k'h' \in HK$. Thus $KH \subseteq HK$.

Let $hk \in HK$. We follow that $(hk)\inv \in HK$. Thus there exist $h' \in H$
and $k' \in K$ such that $h'k' = (hk)\inv$. Thus $hk = (h'k')\inv = k'\inv h'\inv$.
Since $k' \in K$ and $h' \in H$, we follow that $k'\inv \in K$ and $h'\inv \in H$,
and thus $k'\inv h'\inv \in KH$. Thus $hk = k'\inv h'\inv \in KH$. Therefore $HK \subseteq KH$,
as desired.

Suppose that $HK = KH$. Since $e \in H$ and $e \in K$ we follow that $e \in HK$.
Let $hk \in HK$. We follow that
$(hk)\inv = k\inv h\inv \in KH = HK$. Thus $(hk)\inv \in HK$.
Let $a, b \in HK$. We follow that $a = hk$ and $b = h'k'$ for somce $h, h' \in H$
and $k, k' \in K$. Thus
$$ab = hkh'k' = h(kh')k'$$
since $h \in HK$, $k \in HK$ and $kh' \in KH = HK$, we conclude that $ab \in HK$.
Thus $HK$ is a subgroup.


\end{document}

%%% Local Variables:
%%% mode: latex
%%% TeX-master: t
%%% End:
