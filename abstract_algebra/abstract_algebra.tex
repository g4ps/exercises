\documentclass[11pt,oneside,titlepage]{book}
\title{My abstract algebra exercises}
\usepackage{amsmath, amssymb}
\usepackage{geometry}
\usepackage{hyperref}
\author{Evgeny (Gene) Markin}
\date{2024}

\DeclareMathOperator \map {\mathcal {L}}
\DeclareMathOperator \pow {\mathcal {P}}
\DeclareMathOperator \real {\mathbb {R}}
\DeclareMathOperator \topol {\mathcal {T}}
\DeclareMathOperator \basis {\mathcal {B}}
\DeclareMathOperator \ns {null}
\DeclareMathOperator \range {range}
\DeclareMathOperator \fld {fld}
\DeclareMathOperator \inv {^{-1}}
\DeclareMathOperator \Span {span}
\DeclareMathOperator \lra {\Leftrightarrow}
\DeclareMathOperator \eqv {\Leftrightarrow}
\DeclareMathOperator \la {\Leftarrow}
\DeclareMathOperator \ra {\Rightarrow}
\DeclareMathOperator \imp {\Rightarrow}
\DeclareMathOperator \true {true}
\DeclareMathOperator \false {false}
\DeclareMathOperator \dom {dom}
\DeclareMathOperator \ran {ran}
\newcommand{\eangle}[1]{\langle #1 \rangle}
\newcommand{\set}[1]{\{ #1 \}}
\newcommand{\qed}{\hfill $\blacksquare$}



\begin{document}
\maketitle
\tableofcontents

\chapter*{Preface}

This is another yet another attempt at making any progress with
abstract algebra, this time with 'Abstract Algebra: An Integrated
Approach' by Joseph H. Silverman. I really hope that it works out this
time.

So far, it's been most pleasurable journey. This book embodies all the
things, that I like in the mathematics books: lots of rigor and a bit of
lightheartedness, that really lights it up.

After some time with this book my opinion changed a bit in a worse
direction due to an increasing number of typos that I have to deal
with. $\subseteq$ and $\subset$ are often mixed up, I've encountered
an exercise (2.30(c)), whose whole text is one big typo, and some
others. On a brighter note, nothing seems to be as messed up as an
unprovable exercise from the Lovett's book, and every typo is kinda
handled by the context.

Also, the exercise numbering in this book is a tad bit messed up:
it enumerates all the exercises for a given chapter in one sequence
without splitting on the section of a chapter. I don't bother
translating them to the normal format, and by extension some references
in the text of the exercise is a bit messed up.

Some of the notation migrated from my previous endeavours in the
maths. Notably for permutation groups I often use a cyclic
representation, where I do not skip identities.
Some of the notation comes from the book, notable examples of
this notation are:
$$\eangle{g} \text{- a cyclic group that is generated by }g$$
also it is not explicitly defined, but $\cong$ is presumed to
mean isomorphism

I might mix up isometry and isomorphy here and there, but it's
pretty clear from context what exactly do I mean.

Since it's an algebra book, and algebra has rings, that are denoted by
$R$, and those are used in similar context with reals, that are also
denoted by me by $R$, we can see that there's a need to differentiate
between the two. Thus approximately halfway through the chapter about
rings I've started using $\real$ for reals.



\chapter{A Potpourri of Preliminary Topics}

\textit{All of the topics, discussed in this chapter I know already; skip}

\chapter{Groups - Part 1}

\section*{Notes}

I'm gonna use the symbol $*$ as the generic group function and $e$ as
an identity untill stated otherwise since it's most convinient to me.
Sometimes I'll omit $*$ whenever it's clear what's going on. Also,
sometimes I omit parenthesis, but given that we've got associativity,
we can omit them without problems. For rigorousness' sake, I'll
define it to mean left-associative (i.e. $a * b * c = (a * b) * c$).

\section{Introduction to Groups}

\subsection{}

By substituting shapes for numbers we get a trivial exercise


\subsection{}

\textit{Let $n$ be a positive integer, and let $S_n$ be the group of
  permutations of the set $\set{1, 2, ... , n}$ as described in Example
  2.19. Prove that $S_n$ is a finite group, and give a formula for the
  order of $S_n$.}

From the combinatorics we know that the number of permutations is
exacly the factorial of the cardinality of the underlying set.

\subsection{}

\textit{(a) Let $S$ be a finite set, and let $\phi: S \to S$ be a function. Prove that the
  injectivity, surjectivity, and bijectivity of this function are equivatent}

I'm pretty sure that we've proven that rigorously in the set theory
course. If not, then the proof comes from contradiction and theh
cardinality of the codomain of, which proves that injectivity and
surjectivity are equivalent, and bijectivity comes from definition.

\textit{(b) Give an example of an infinite set $S$ and a function
  $\phi: S \to S$ such that $\phi$ is injective but not surjective}

We can let $S = \omega$, $\phi(x) = 2x$, which gives us range of even numbers.

\textit{(c) Give an example of an infinite set $S$ and a function
  $\phi: S \to S$ such that $\phi$ is surjjective but nnot bijective}

We can set $S = \omega$ and
$$\phi(x) =
\begin{cases}
  x = 0 \to 1 \\
  x - 1 \text{ otherwise}
\end{cases}
$$

\subsection{}

\textit{This one involves drawing and is pretty trivial; skip}

\section{Abstract Groups}


\subsection{}

\textit{Let $G$ be a group. In this exercise you will prove the remaining parts of
  Proposition 2.9. Be sure to justify each step using the group axioms or by reference to a
  previously proven fact}

\textit{(a) $G$ has exactly one identity element.}

Suppose that $e_1$ and $e_2$ both satisfy identity axiom. We follow
that both of them are in $G$ and thus
$$e_1 = e_1 e_2 = e_2 e_1 = e_2$$
which comes directly from the Identity Axiom.

\textit{(b) $g, h \in G \ra (g * h)\inv = h\inv g\inv$}

We follow that
$$(g * h)\inv  (g * h) = e$$
by definition of identity. Thus
$$(g * h)\inv  (g * h) * h\inv  = e h\inv$$
$$(g * h)\inv  (g * h) * h\inv  g\inv  = e h\inv g\inv$$
by the fact that $*$ is a binary function. Thus
$$(g * h)\inv  g * (h * h\inv) * g\inv  = e h\inv g\inv$$
$$(g * h)\inv  g * g\inv  = e h\inv g\inv$$
$$(g * h)\inv  (g * g\inv)  = e h\inv g\inv$$
$$(g * h)\inv = e h\inv g\inv$$
by associative laws, and then we've got that
$$(g * h)\inv = h\inv g\inv$$
by the identity axiom, as desired

\textit{(c) $g \in G \ra (g\inv) \inv = g$}

We follow that
$$(g\inv) \inv * (g\inv) = e$$
by the inverse axiom. Thus
$$(g\inv)\inv * (g\inv) * g  = e * g$$
$$(g\inv)\inv * (g\inv) * g  = g$$
by identity and properties of functions. Thus
$$(g\inv)\inv * ((g\inv) * g)  = g$$
$$(g\inv)\inv * e  = g$$
$$(g\inv)\inv  = g$$
by associativity and so on, as desired.

\subsection{}

\textit{Let $G$ be a group, let $g, h \in G$, and suppose that $g$ has
  order $n$ and that $h$ has order $m$.}

\textit{(a) If $G$ is an abelian group and if $gcd(m, n) = 1$, prove that
  the order of $gh$ is $mn$.}

Firstly, I want to follow a couple of things:

If order of $g$ is $m$, and order of $g\inv$ is not $n < m$, then
$$e = e * e = g^m (g\inv)^n = g^{m - n} = e$$
which is a contradiction. Similar case holds for $n > m$.  Thus if
order of $g$ is $n$, then order of $g\inv$ is also $n$.

We also follow that if $g$ has finite order $n$, then $(g^k)^n =
(g^n)^k = e$, and thus $g^k$ has finite order for any $k \in Z$.
Moreover, since $(g^k)^n = e$ we follow that $g^k$'s order divides $n$.

We also follow that if $g$ has finite order $n$, then
$$g * g^{n - 1} = e$$
$$g^{n - 1} = g\inv$$


Now back to our exercise. We follow that
$$(gh)^{mn} = g^{mn} h^{mn} = (g^{n})^m (h^m)^n = e^m e^n = e$$
where we can split it this way since $G$ is abelian. Thus we follow
that order of $gh$ is finite and divides $mn$.

Suppose that the order of $gh$ is $k$. We follow that $k \leq mn$
since it divides $mn$.
$$(gh)^k = e$$
thus
$$(gh)^k = g^k h^k = e$$
if $g^k \neq e$, then we follow that $h^k = (g^k) \inv \neq e$, thus
$h^k = (g\inv)^k = (g^{n - 1})^k$. Thus we follow that $h^k$ is a
multiple of $g$, and thus its order divides order of $g$ $m$. Since
$h^k$ is both multiple of $g$ and $h$, we follow that its order
divides both $m$ and $n$, and since $gcd(m, n) = 1$, we follow that
its order is 1. Thus $h^k = e$, which is a contradiction.

Thus we conclude that $g^k = e$. For $h^k$ we've got a similar
case. Thus we follow that $k$ divides both $m$ and $n$, and thus it's
either $1$ or $mn$. If $k = 1$, then we follow that $g = h\inv$, and
thus $gcd(m, n) = 1$ implies that the order of both $g$ and $h$ cannot
be anything other than $1$, and thus $k = mn = 1$.
If $k \neq 1$, then we follow that $k = mn$, as desired.

\textit{(b) Give an example showing that (a) need not be true of we
allow $gcd(m, n) > 1$}

We can have some group where order of $g$ is $n > 1$ (for example $g =
1$ in $G = Z/5$) and set $h = g\inv$, for which we'll have that order
of $gh$ is 1.

\textit{(c) Give an example of a nonabelian group showing that (a) need not
  be true even if we retain the requirement that $gcd(m, n) = 1$.}

A dihedral group of a triangle with $g = r$ and $h = f$ will do. We'll have that
order of $rf$ is 2:
$$(rf)^2 = (rfrf) = f\inv r r f = e$$
with order of $g$ being $2$ and order of $h$ being $3$

\textit{(d) Again assume that $G$ is an abelian group, and let $l = mn
  / gcd(m, n)$ (i.e.  $l = lcm(m, n)$). Prove that $G$ has an element
  of order $l$.}

We follow that $n$ divides order of $g^m$. We also follow that $m$
divides order of $h^n$.  Since $l = lcm(m, n)$ is divided by both $m$
and $n$ we follow that $g^m h^n$'s order divides $l$.

If there's $k \leq lcm(m, n)$ such that
$$(g^m h^n)^k = e$$
then we follow that
$$g^{mk} h^{nk} = e$$

Here we're gonna employ a more generalized version of an argument in part (a):
If $g^{mk} \neq e$, then $k < lcm(m, n)$, thus  $h^{n}$'s order is a
multiple of both $m$ and $n$. $h^{nk}$ cannot be $1$, and its order
must be then $lcm(m, n)$ or $1$, and it can be neither, thus we've got
a contradiction. Thus we conclude that $g^{mk} = h^{nk} = 1$, which implies that $k$
divides both $m$ and $n$, which implies that $k = lcm(m, n)$, as desired.

\subsection{}

\textit{Definition 2.6 says that a group is a set $G$ with a
composition law satisfying three axioms.  In particular, it says that
there's an identity element $e \in G$ that works on both sides and
that every element $g \in G$ has an inverse that works on both
sides. Suppose that we weaken the requirements to specify that the
identity and inverse work only on one side. In other words, we suppose
that $G$ is a set with a composition law satisfying the following
weaker axioms: }

\textit{(a) (Right-Identity Axiom) There is an element $e \in G$ so
  that $g * e = g$ for all $g \in G$}

\textit{(b) (Right-Inverse Axiom) For all $g \in G$ there is an
  element $h \in G$ so that $g * h = e$}

\textit{(c) (Associative Law) $(g_1 * g_2) * g_3 = g_1 * (g_2 * g_3)$
for all $g_1, g_2, g_3 \in G$.}

\textit{Prove that $G$ is a group.}

We're gonna start with establishing the inverse axiom for $G$,
as suffested in the hint to the exercise.
Suppose that for $g \in G$ there's $h \in G$ such that 
$$g * h = e$$
we also follow that for $h$ there's an element $k$ such that
$h * k = e$ by the same axiom. We thus follow that
$$ h * g = h * g * e = h * g * h * k = h * (g * h) * k = h * e * k = (h * e) * k = h * k = e $$
where we use justification of
$$a \to (\text{h * k = e}) \to c \to b \to c \to a \to (\text{h * k = e})$$
in our equalities (given axioms are presented by letters).
Thus we've got that $h * g = e = g * h$ (i.e. normal inverse axiom)

Now suppose that $g \in G$. We follow that
$$e * g = g * g\inv * g = g * (g\inv * g) =  g * e = g$$
which practially establishes the Identity axiom. The associative law
is unchanged from the stadart definition of the group, and thus we're
following that $G$ is indeed a group, as desired.

\subsection{}

\textit{There are other sorts of algebraic structures that are similar to groups in that
  they are sets $S$ that have a composition law}
$$S \times S \to S, (s_1, s_2) \to s_1 * s_2$$
\textit{but they have fewer or different asiomx than a group. In this exercise we explore two
  of these structures.}

\textit{The set $S$ with its composition law is a monoid if it has an
identity element $e \in S$ and satisfies the associative law, but
elements are not required to have inverses.}

\textit{The set $S$ with its composition law is a semigroup if its
composition law is associative, but it need not have an identity
element or inverses.}

i.e. monoid satisfies associative and identity, semigroup satisfies associative,
and group satisfies all of them. Hence we've got nested classes of structures:

semigroup $\subseteq$ monoid $\subseteq$ group

\textit{For each of the following sets $S$ and composition laws $*$
determine if $(S, *)$ is a group, a monoid, or a semigroup.}

\textit{(a) The set of natural numbers $N = \set{1, 2, 3,
    ...}$ with the composition law being addition.}

I usually do include $0$ into the set of naturals, but in this particular case
it seems that we don't do it.

In this particular case we follow that $N$ has associativity with $+$,
rigorous proof of which comes from the construction of the naturals.
In the course of the set theory I've gone through that proof, but
I'm pretty sure that it's not required here.

If we don't include $0$ into $N$, then we don't have an identity in
$N$, which can be proven by defining order $>$ on $N$ and proceeding
from there (which was also handled in the course on set theory).
Hence it is only a semigroup. If we include it, however, then we get an
identity, and thus this set becomes a monoid.

We can also state that it's not got identities.

\textit{(b) The set of extended  natural numbers $N_0 = \set{0, 1, 2, 3,
    ...}$ with the composition law being addition.}

handled in part (a)

\textit{(c) $(Z, +)$}

We can follow that it's a superset of $N$, which gets its inverses ($a\inv = -a$)
and hence becomes a full-blown group.

\textit{(d) $(N, *)$}

We follow that it's associative, has the identity $1$, and hence is at
least a monoid. It's not got multiplicative inverses (because that
would be $Q_+$), and hence is not a group.

\textit{(e) $(N_0, *)$}

Same as previous, $0$ does not change associativity and identities. Is
not a group for the same reason as the previous case.

\textit{(f) $(Z, *)$}

Same as previous for the same reason.

\textit{(g) The set of integets $Z$ with the composition law $m * n = max\set{m, n}$}

If there's $m \in Z$, then there's $n_0, n_1 \in Z$ such that
$$n_0 < m < n_1$$
and hence
$$n_0 * m = n_0, n_1 * m = m$$
and hence we follow that there's no identity.

We follow that
$$(n * m) * k = max\set{max\set{n, m}, k} = max\set{n, m, k} = max\set{n, max\set{m, k}} = n * (m * k)$$
hence we've got associaativity. Thus the given structure is a semigroup.

\textit{(h) The set of naturals $N$ with the composition law $m * n = max\set{m, n}$}

It's got associativity and hence this thing is a semigroup. We can follow that
$1 \leq m$ for all $m \in N$, and thus $max\set{1, m} = m$. Thus we follow that we also
have a monoid. Inverses are not present, and thus it's not a group.

\textit{(i) The set of naturals $N$ with the composition law $m * n = min\set{m, n}$}

We've got associativity. If $m \in N$, then there's $n \in N$ such that $n > m$, and thus
$min\set{m, n} = m$, which means that there's no identity, and hence this thing is only a semigroup.

\textit{(j) The set of naturals $N$ with the composition law $m * n = mn^2$}

We follow that
$$(m * n) * k = (mn^2) * k = mn^2k^2$$
$$m * (n * k) = m * nk^2 = mn^2k^4$$
setting m, n, k to primes we can follow that $(m * n) * k \neq m * (n * k)$, which implies that
this thing is neither a group, a monoid, nor a semigroup.

\subsection{}

\textit{Look up magmas, Moufang loops, quandles, and matroids}

Magma is just a set with a binanry function. There are some
discussions about closure, but as long as the thing provided with a
set is a binary function, it's a magma.

TODO: look up the rest

\section{Interesting Examples of Groups}

\subsection*{Notes}

Although I'm pretty sure that a the collection of groups is a proper class
(i.e. not a set), I don't have a proof for that. I'll try to change it now.

Empty sets cannot be groups since they've gotta have an element (identity).

Although we can try do to something with cardinals and whatnot, we know that for
any set $S$ there's an injection to the set $S^S$. We then follow that there's
a subset of bijections in $S^S$. We then follow that this subset of bijections
is a group (permutations), and hence we conclude that for each set there's a
group, and thus a collection of groups is a proper class, as desired.

We can also skip all this stuff, and state singleton of any set $S$ is
a trivial group under projection. Trivial group (if I'm not mistaken) is a
group that's got only an identity in it.

Permutation Group practically states that a set of bijections over a
set constitutes a group under composition. There are no further
restrictions, which is pretty neat.

Matrix group leads us to an interesting idea, which is also bleeding
into dihedral groups: as long as a subset of the set of bijections is
closed under composition, includes inverses, and includes an
indentity, given subset is a group (associativity is a property of
composition). Can we lose some restrictions though?

Suppose that there's a set of bijections over a set and it is closed
under composition. If a set is finite, then we can follow that the set of pairs of
bijections is larger than the set of bijections, hence there's got to be a pair
$$S \circ C = S$$
or something like that, which would imply that $C$ is an identity.
If the set is infinite, then this proof will not do.
Finality of the set does not imply the existence of inverses: $\set{C, e}$ will be
closed under composition, will have an identity, and will not have inverses.

Given a set of nonzero naturals $N$, for each $n \in N$ we can have $S_n: \omega \to \omega$
$$S_n(x) = x + n$$
For each $n, m \in N$ we've got that
$$(S_n \circ S_m)(x) = S_n(x + m) = x + m + n = S_{m + n}(x)$$
which gives us a set of bijections, that is closed under a
composition, does not include an identity ($0 \notin N$ by our
restriction), and no element has an inverse.

Thus we conclude that if a set of bijections is finite and is a singleton, then it's a group.
If it's not a singleton and is finite, then it's a monoid (i.e. associativity and identity).
If it's not finite, then it's just a semigroup (i.e. associativity exclusively).

We can't have a group of functions, where the set is not a singleton and
domain and range of the function are distinct, since two functions are supposed
to compose. We can't have non-bijections be present, since we've got to have
both left-hand side and right-hand side inverses of any given element.

\subsection{}

\textit{Let $G$ be a finite cyclic group of order $n$, and let $g$ be
a agenerator of $G$.  Prove that $g^k$ is a generator of $G$ if and
only uf $gcd(k, n) = 1$.}

Suppose that $g^k$ is a generator of $G$. If $k = 1$,
then we follow that $gcd(k, n) = gcd(1, n) = 1$, thus
assume that $k \neq 1$. 

We follow that there's $m \in \omega$ such that $m \neq 0$
and
$$(g^k)^m = g^{km} = g$$
thus
$$g^{km - 1}  = e$$
and hence we follow that $km - 1$ is multiple of order of $G$.  Thus
there exits $j \in Z$ such that $km - 1 = jn$.  Thus $km - jn = 1$,
which implies that $gcd(k, n) = 1$, as desired.

Every implication in the forward direction is pretty much a
biconditional (not exactly though, we need to add some quantifiers to
the mix), so it works in the reverse direction as well.

Also, we haven't gone into a proof that $gcd(m, n) = l$ if and only if
$l$ is the lowest positive number such that there exist $i, j \in Z$
such that
$$mi + jn = l$$
but we've essentially proven the result for $gcd(m, n) = 1$ in 1.35

\subsection{}

\textit{Skip}

\subsection{}

\textit{Prove that the Dihedral group $D_n$, as described in Example
  2.22, has exactly 2n elements}

We can follow that there are $n$ vertices, and each of them gotta go
to one of the other $n$ spaces. There are $n$ ways to put the first vertex
into any of those places. Whenever we put the first vertex $n$ into some place,
there are 2 places where the second vertiex can go, so now we're down to $2n$
possible positions. Whenever we put the second vertex down, positions of the
rest are determined, and hence we can follow that there are total of $2n$
possible positions, as desired.

\subsection{}

\textit{(a) Let $Q^*$ be the set of non-zero rational numbers, with
  the group law being multiplication. Prove that $Q^*$ is a group.}

We follow that $1 * q = q$, and hence we've got an identity.  If $q
\in Q^* \ra q \neq 0 \land q \in Q$, then we follow that $1/q$ exists,
and thus there's $q\inv$ such that $q q\inv = 1$. Multiplication's
associativity does not require a proof (or can be provided, but in
this course it is assumed), and thus we conclude that $Q^*$ is a group
(moreover, an abelian group), as desired.

\textit{the rest is skipped}

\subsection{}

\textit{(b) Let $p$ be a prime number. Prove that the set of non-zero
  elements of $Z/pZ$ is a group using multiplication as the group law}

We fillow that for each $i \in Z/pZ$, $i * 1 = i$, and thus $i$ is our
identity. Associativity of multiplication under $Z$ implies our associativity.

From group theory we know that there's $x$ such that $ax \equiv 1 (mod
m)$ if and only if $gcd(a, m) = 1$.
We then follow that for any given $i \in Z/pZ$ we've got that
$gcd(i, p) = 1$ since $p$ is prime, and thus there's $a \in Z/pZ$
such that $a * i = 1$. Thus we've got the inverse, and $Z/pZ$ is a
group under multiplication, as desired.

We can probably also prove the other direction if we want. If $p$ is
not $1$ and is not a prime, then there's a divisor $q$ of $p$ in
$Z/pZ$, which implies that only multiple of $q$ is its multiple, or
$0$, which implies that it's got no inverse and hence $Z/pZ$ is not a
group, as desired. If $p$ is not a prime, however, then we've got that
$Z/pZ$ is a trivial group. We can probably reword this thing so it
becomes a theorem, but I'm too lazy to do this.

\textit{(c) Heck, I've proven this already and I didnt even read this section}

\textit{(d) blah blah blah}

Got pretty much the same proof as part (a). The only thing of note is
that we've got new notation:
$$(Z/pZ)^*$$
which is a group under multiplication of numbers relatively prime to $p$.

\subsection{}

\textit{Let $C$ be the set of complex numbers, that is, the (... a
crude definition of complex numbers)}

\textit{(a) We make $C$ into ...}

It's a group under addition, yes. Proof is trivial

\textit{(b) $C^* = C \setminus \set{0}$ is a group under multiplication.}

Yes it is. Multiplicative inverse is kinda whacky in complex numbers, let me find it:

$$\frac{1}{a + bi} = \frac{(a - bi)}{(a + bi)(a - bi)} =
\frac{(a - bi)}{a^2 - b^2i^2} = \frac{(a - bi)}{a^2 - b^2 * (-1)}  =
\frac{(a - bi)}{a^2 + b^2}  = \frac{a}{a^2 + b^2}  - \frac{b}{a^2 + b^2}i$$
which I would never ever find on my own, if I wouldn't start a complex
analysis course at some time in the past (probably should look into 
it not, this thing looks pretty fun)

\subsection{}

\textit{(a) Let
$$GL_2(R) = \set{
  \begin{pmatrix}
    a & b \\
    c & d \\
  \end{pmatrix}: a, b, c, d \in R, ad - bc \neq 0}
  $$
  be the indicated set of 2-by-2 matrices, with composition law being matrix multiplication.
  Prove that $GL_2(R)$ is a group.
}

It's a set of bijections, that is closed under composition (see linear
algebra course), each element is invertible by the restriction, and
has identity, thus it's a group.  We can also scale this thing to
complex numbers, and we can also increase dimensions of the underlying
sets and substitute restriction from the determinant-based to just
being invertible (which are equivalent).

\textit{(b)Let
$$SL_2(R) = \set{
  \begin{pmatrix}
    a & b \\
    c & d \\
  \end{pmatrix}: a, b, c, d \in R, ad - bc = 1}
  $$
  be the indicated set of 2-by-2 matrices, with composition law being matrix multiplication.
  Prove that $SL_2(R)$ is a group.
}

Same logic as before. We can scale with determinants here as well.

This group does not include all the isometries though, which is kinda
interesting.  Some isometries (e.g. $1, -1$ on diagonal) will not be
included here. TODO: research this thing a bit more

\textit{(c) ... }

Pretty much is solved by my discussion in previous points. A thing to remeber though:

\textbf{General} linear group is a set of invertible linear functions.

\textbf{Special} linear group is a set of matrices with 1 for the
determinant.

\subsection{}

\textit{Let $GL_2(R)$ be the general linear group. Prove or disprove
  that each of the following subsets of $GL_2(R)$ is a group. In the
  case of non-groups, indicate which of the group conditions fail.
}

\textit{(a)
  $$\set{
  \begin{pmatrix}
    a & b \\
    c & d \\
  \end{pmatrix} \in GL_2(R): a, b, c, d \in R, ad - bc = 2}
  $$
}
Not a group, identity is not in there, and composition is not closed ($|A B | = |A| |B| = 2 * 2 = 4$)

\textit{(b)
  $$\set{
  \begin{pmatrix}
    a & b \\
    c & d \\
  \end{pmatrix} \in GL_2(R): a, b, c, d \in R, ad - bc \in \set{1, -1}}
  $$
}

Is a group, pretty sure that those describe the isometries. 

\textit{(c)
  $$\set{
  \begin{pmatrix}
    a & b \\
    c & d \\
  \end{pmatrix} \in GL_2(R): a, b, c, d \in R, c = 0}
  $$
}

Is a set of upper-triangulars, which are closed under composition, and contain the identity.
Inverses of upper-triangular are also upper-triangular, thus we've got a group.

\textit{(d)
  $$\set{
  \begin{pmatrix}
    a & b \\
    c & d \\
  \end{pmatrix} \in GL_2(R): a, b, c, d \in R, d = 0}
  $$
}

Does not contain the identity, and inverses arent present

\textit{(e)
  $$\set{
  \begin{pmatrix}
    a & b \\
    c & d \\
  \end{pmatrix} \in GL_2(R): a, b, c, d \in R, a = d = 1, c = 0}
  $$
}

Pretty sure that this once is a group as well.

\subsection{}

\textit{Let $Q = \set{\pm 1, \pm i, \pm j, \pm k}$ be the group of quaternions.
  We clamed that the group law for $Q$ is determined by
  $$i^2 = j^k = k^2 = i j k = -1$$
  Use there formulas to prove following formulas, which completely determine the group
  opetrations on $Q$:
}

We firstly follow that $(-1)\inv = -1$. Thus
$$i * i = -1$$
$$(i * i)\inv = (-1)\inv$$
$$i\inv * i\inv = -1$$
$$i\inv = -1 * i $$
$$i\inv = -i$$
where the last part comes from the note that $-1$ commutes with eveything.
We can follow that $k\inv = -k$ and $j\inv = -j$ by the same logic.

We follow now that
$$i * j = i * j * 1 = i * j * (k * -k) = (i * j * k) * -k = -1 * -k = k$$
$$j * k = 1 * j * k = (-i * i) * j * k = -i * -1 = i$$
$$k * i =  1 * k * i = -j * j * k * i = -j * (j * k) * i = -j * i * i = -j * -1 = j$$
we then follow that
$$-k = k\inv = (i * j) \inv = -j * -i = j * i$$
and so on for all the inverses.

\subsection{}

Skip

\subsection{}

\textit{Practically continue with the functions in the notes. Part (a)
is handled}

\textit{(b) If $X$ is a finite set with $n$ elements, Prove that
  $\mathcal{E}_X$ is a finite monoid and compute how many elements it has.}

We follow that the number of functions is $n^n$, the rest is handled in the notes

\textit{(c) If $|X| \geq 3$, prove that $\mathcal{E}_X$ is not commutative, i.e.
  show that there are elements $\phi, \psi \in \mathcal{E}_X$ satisfying
  $\phi \circ \psi \neq \psi \circ \phi$.}

We can map a portion of this set to ${1, 2, 3}$, and then get functions
$$\phi(x) =
\begin{cases}
  1 \to 2 \\
  2 \to 2 \\
  3 \to 2 \\ 
\end{cases}
$$
$$\psi(x) =
\begin{cases}
  1 \to 3 \\
  2 \to 3 \\
  3 \to 3 \\ 
\end{cases}
$$
which will suffice.

\section{Group Homomorphism}

\subsection{}

\textit{Recall that two groups $G_1$, $G_2$ are said to be isomorphic
  if there is a bijective homomorphism
  $$\phi: G_1 \to G_2$$
  The fact that $\phi$ is bijective means that the inverse map $\phi \inv : G_2 \to G_1$ exists.
  Prove that $\phi_1$ is a homomorphism from $G_2$ to $G_1$. 
}

We essentially need to prove that
$$\phi\inv(h_1 * h_2) = \phi \inv (h_1) * \phi \inv (h_2)$$
for all $h_1, h_2 \in G_2$.

$\phi$ is a bijection, and thus for $h_1, h_2 \in G_2$ there are $g_1, g_2 \in G_1$ such that
$\phi(g_1) = h_1, \phi(g_2) = h_2$. By the fact that $\phi$ is a bijection we also follow that
$g_1 = \phi\inv(h_1), g_2 = \phi\inv(h_2)$. Thus we've got that 
$$\phi\inv(h_1 * h_2) = \phi\inv(\phi(g_1) * \phi(g_2)) =
\phi\inv(\phi(g_1 * g_2)) = g_1 * g_2  = \phi\inv(h_1) * \phi\inv(h_2) $$
as desired

\subsection{}

\textit{Let $G$ be a group, and consider the functiion
  $$\phi: G \to G, \phi(g) = g\inv$$
}

\textit{(a) Prove that $\phi(\phi(g)) = g$ for all $g \in G$}

$$\phi(\phi(g)) = \phi(g\inv) = (g\inv)\inv = g$$

\textit{(b) Prove that $\phi$ is a bijection.}

We follow that for each $g \in G$ there's $g\inv \in G$, thus
$\phi(g\inv) = g$, which implies that the range of $\phi$ is $G$.

We follow that if $g_1 \neg g_2$, then $g_1\inv \neq g_2\inv$, and thus
$\phi$ is injective.

\textit{(c) Prove that $\phi$ is a group homomorphism if and ony if
  $G$ is an abelian group.}

\textbf{Forward direction:}
We know that $\phi$ is bijective, and thus for all $h_1, h_2 \in G$
there are $g_1, g_2 \in G$ such that $\phi(g_1) = h_1, \phi(g_2) =
h_2$. If $G$ is a group homomorphism, then we follow that
$$h_1 * h_2 = \phi(g_1) \phi(g_2) = \phi(g_1 g_2) = (g_1 g_2)\inv = g_2\inv g_1 \inv = h_2 * h_1$$
thus the group is abelian.

\textbf{Reverse direction:}

Let us keep assumptions about our variables. If $G$ is abelian, then
we follow that
$$h_2 * h_1 = g_2\inv g_1 \inv  =  (g_1 g_2)\inv  = \phi(g_1 g_2)$$
$$h_1 * h_2 = g_1\inv g_2 \inv  =  \phi(g_1) * \phi(g_2)$$
Since $G$ is abelian we follow that $h_1 h_2 = h_2 h_1$, and thus
we can equate given equalities to get the desired result.

\subsection{}

Skip

\subsection{}

\textit{Let $G_1$ and $G_2$ be groups, and suppose that $\phi: G_1 \to
  G_2$ is an isomorphism}

\textit{(a) Prove that if $G_1$ is finite, then $G_2$ is also finite, and
  that they satify $|G_1| =_c |G_2|$}

Isomorphism is a bijection, thus sets have equal cardinality.

\textit{(b) Suppose that $G_1$ is abelian. Prove that $G_2$ is
abelian}

Let $g_1, g_2 \in G_2$. Since $\phi$ is a bijection we follow that there are
$h_1, h_2 \in G_1$ such that $\phi(h_1) = g_1$ and $\phi(h_2) = g_2$.
We follow that
$$g_1 g_2 = \phi(h_1) \phi(h_2) = \phi(h_1 h_2) = \phi(h_2 h_1) = \phi(h_2) \phi(h_1) = g_2 g_1$$
as desired.

\textit{The solution for the rest of this exercise practically copies
  the structure of part (a) }

\subsection{}

\textit{In this exericse, $C_n$ is a cyclic group of order $n$, $D_n$
  is the $n$h dihedral group, and $S_n$ is the $n$th symmetric group.}

\textit{(a) Prove that $C_2$ and $S_2$ are isomorphic.}

We follow that both $C_2$ and $S_2$ have $2$ and $2! = 2$ elements
respectively. Non-identity elements can interact with identity in a
predictible way, or with themselves, where they've got an order of
$2$. This pretty much concludes the proof.

\textit{(b) Prove that $D_3$ is isomorphic to $S_3$.}

We know that $D_3$ is a subgroup (although we haven't defined at this
point what a subgroup is, it doesn't take a genius to guess) of $S_3$,
with the same cardinality.

\textit{(c) Let $m \geq 3$. Prove that for every $n$, the groups $C_m$ and $S_n$ are not
  isomorphic.}

If $m \neq n!$, then we follow that groups have different cardinalities, thus
there's no bijection between the two, and thus no isomorphisms.

If $m = n!$, then we follow that since $m \geq 3$ that $n \geq 3$ as
well.  We follow that there are elements of $S_n$ that map $(1, 2, 3)$
to $(1, 3, 2)$ and $(2, 1, 3)$ and the rest to identity. Composition
of those elements is not commutative ($(2, 3, 1)$ and $(3, 1, 2)$ in
forward and reverse respectively), and thus $S_n$ is not abelian.
$C_n$ on the other hand is commutative, and thus we follow
that they aren't isomorhic. This also proves that $S_n$ is non-abelian
for any $n \geq 3$.

\textit{(d) Prove that for every $n \geq 4$ the groups $D_n$ and $S_n$
  aren't isomorphic}

They dont have the same cardinalities.

\textit{(e) More generally, let $m \geq 4$, and let $n \geq 4$. Prove
  that the groups $D_m$ and $S_n$ aren't isomorphic.}

We once again look at the case when $2m = n!$. We probably want to
find some irregularities with orders of the elements here.  We follow
that if $m$ is odd, then there is only 1 element of order $2$ ($f$).
If $m$ is even, then we've got 3 such elements ($f, r^{m/2}, fr^{m/2}$).
For $S_n$ for $n \geq 4$ we follow that there are at least 6 of them
(bijections that swap elements, there are $C(4, 2) = 6$ of them at least).

\textit{(f) The cyclic group $C_8$, the dihedral group $D_4$, and the quaternion group
  $Q$ are non-abelian groups of order $8$. Prove that they ren't isomorphic.}

First of all, $C_8$ is abelian (as is any cyclic group for that
matter). If we ignore that, then we can follow that there is one
element of order 4 ($2$) in $C_8$, and there are 6 ($\pm i, \pm j, \pm
k$) in $Q$. There are 3 elements of order $2$ in $D_4$, but only one
in $C_8$. $Q$ has 1 element of order $2$.

\subsection{}

Skip

\section{Subgroups, Cosets, and Lagrange's Theorem}

\subsection*{Notes}

Important definition from the exercises:
\textit{Let $G$ be a group, and let $S \subseteq G$ be a subset of
$G$. The subgroup of $G$ generated by $S$, which we denote by
$\eangle{S}$, is the intersection of all of the subgroups of $G$ that
contains $S$; i.e.
  $$\eangle{S} = \bigcap_{S \subseteq H \subseteq G: H\text{ is a subgroup of }G} H$$}
Exercise proves that $\eangle{S}$ is essentially a subgroup, where
every element is a multiple of some combination (i.e. ordered list in
non-abelian and multiset in abelian) of elements of $S$ or their
inverses, and also the smallest subgroup, that contains the entirety
of $S$.

Another important definition is the definition of centralizer.  The
centralizer of the subset ($Z_G(S)$) is the most general of them, they
are provided below

We also follow that if $H_1, H_2$ are subgroups of $G$, then $H_1 \cap
H_2$ is a subgroup of $G$, which is sort of trivial to prove. Importatnt
to note that union of two subgroups is not necessarily a group.

\subsection{}

\textit{Let $G$ be a group and let $g \in G$ be an element of order
  $n$, and let $k \geq 1$.}

\textit{(a) Prove that $g^k$ has order $n / gcd(n, k)$.}

We follow that $g$ is a generator for a cyclic group, in which
multiples of $g^k$ comprises a subgroup. Since $g$ has order $n$ we
follow that $\eangle{g}$ is a cyclic group of order $n$.

We know that $gcd(n, k)$ is the lowest number that is a positive sum
of multiples of $n$ and $k$ respectively. Thus there are integers
$a, v$ such that
$$an + vk = gcd(n, k)$$
We then follow that
$$(g^n)^a * (g^k)^v = g^{gcd(n, k)}$$
order of $g$ is $n$ and thus
$$(g^n)^a * (g^k)^v = (e)^a * (g^k)^v = e * (g^k)^v = (g^k)^v = g^{gcd(n, k)}$$
We thus follow that multiples of $g^{gcd(n, k)}$ are in cyclic
subgroup of $g^k$. There are exacltly $n / gcd(n, k)$ such elements.

Suppose that $v$ is not a multiple of $gcd(n, k)$ and suppose that
there's $j \in Z_+$ such that $(g^k)^j = g^{k * j} = g^v$. Since $v$ is
not a multiple of $gcd(n, k)$, it is not a multiple of $k$. Thus we
follow that $k * j > n$. We thus conclude that there's a maximal $h \in Z_+$
such that
$$kj - hn = v$$
$$kj + (-h)n = v$$
thus we conclude that $v$ is a multiple of $gcd(n, k)$, which is a
contradiction.

\textit{(b) Use (a) to give a quick proof of Exercise 2.10, which says
  that $G = \eangle{g}$ is a cyclic group of order $n$, then $g^k$
  generates $G$ if and only if $gcd(n, k) = 1$ }

We follow that if $g^k$ generates $G$, then $|\eangle{g^k}| = n / gcd(n,
k) = n$. Thus $gcd(n, k) = 1$. Reverse case is pretty much the same.

\subsection{}

\textit{Let $G$ be a cyclic group of order $n$, and let $d \geq 1$ be
  an integer.}

\textit{(a) Prove that every subgroup of $G$ is cyclic.}

Since $G$ is cyclic we follow that there's $g \in G$ that generates
$G$.

Suppose that $H$ is a subgroup of $G$. If $g \in H$, then $G = H$, and
thus we're done. Thus suppose that $g \notin H$. We then follow that
there is a lowest possible $i \in Z_+$ such that $g^i \in H$. We
follow that every multiple of $i$ is in $H$. Suppose that $k \in Z_+$
is not a multiple of $i$ and such that $g^k \in H$. We follow that
$gcd(k, i) < i$ since $k$ is not a multiple of $i$. Thus we follow
that there are $a, v \in Z$ such that
$$ak + vi = gcd(k, i)$$
and thus
$$g^{ak} + g^{vi} = g^{gcd(k, i)}$$
thus $g^{gcd(k, i)} \in H$, which implies that $i$ is not the lowest element
of $Z_+$ such that $g^i \in H$, which is a contradiction. Thus we
conclude that multiples of $i$ are the only elements of $H$, which implies that
$H$ is cyclic, as desired.

Important note: this argument also works for groups $G$, and so we can follow that
any subgroup of any cyclic group is cyclic.

\textit{(b) If $d$ divides $n$, prove that $G$ has a unique subgroup of order $d$.}

Let $H_1$ and $H_2$ be two subgroups of order $d$. We follow that for
both of there are least elements $i_1, i_2$ of $Z_+$ such that
$g^{i_1} \in H_1$ $g^{i_2} \in H_2$. Suppose that $i_1 \neq i_2$ and
assume that $i_1 < i_2$.  Previous arguments in (a) imply that $g^{i_1}$ and
$g^{i_2}$ generate $H_1$ and $H_2$. We know also that $H_1$ is comrised entirely of
powers of $g^{i_1}$. If $gcd(i_1, d) \neq i$, we follow that $gcd(i_1, d) < i$,
and thus there are $a, v \in Z$ such that
$$(g^{i_1})^a * (g^{d})^v = g^{gcd(i_1, d)}$$
$$(g^{i_1})^a * (e)^v = g^{gcd(i_1, d)}$$
$$(g^{i_1})^a  = g^{gcd(i_1, d)}$$
We thus follow that $g^{gcd(i_1, d)} \in H_1$, and thus $i_1$ is not
the lowest positive integer such that $g^{i_1} \in H$, which is a
contradiction. We thus conclude that $gcd(i_1, d) = i_1$, and thus
$i_1 | d$. This implies that $|H_1| = |\eangle{g^{i_1}}| = d / i_1$.
Thus if $i_1 \neq i_2$ we follow that $|H_1| \neq |H_2|$, which is a
contradiction. Thus we conclude that $i_1 = i_2$, and thus $H_1 = H_2$,
as desired.

\textit{(c) If $d$ does not divide $n$, prove that $G$ does not have a subgroup of order $d$.}

We follow that number of cosets of $G$ of order $d$ is a natural number, and thus
Lagrange's Theorem implies that order of any subgroup of $G$ divides $n$, as desired.

\subsection{}

\textit{Let $G$ be a group, and let $H \subseteq G$. Prove that $H$ is a subgroup
  if and only if it has the following properties:}

\textit{(1) $H \neq \emptyset$}

\textit{(2) For every $h_1, h_2 \in H$, $h_1 * h_2 \inv \in H$}

Forward direction is covered by axioms.

Suppose that $H \neq \emptyset$ and for every $h_1, h_2 \in H$, $h_1 *
h_2 \inv \in H$. We follow that there's $h \in H$, and thus
$h * h\inv = e \in H$. Thus we follow the identity axiom.

Suppose that $k \in H$. We follow that $k \in H$ and $e \in H$, thus
$e * k\inv = k\inv \in H$. Thus we follow the inverse axiom.

Let $h_1, h_2 \in H$. We follow that $h_2 \inv \in H$, and thus
$h_1 * h_2 \inv \inv = h_1 * h_2 \in H$, thus giving us closure,
which completes the definition of the subgroup, as desired.

\subsection{}

\textit{Let $G$ be a group, and let $H \subseteq G$ be a nonempty subset of $G$. Prove that
  $H$ is a subgroup if and only if it has the following property:}

\textit{For every $a \in H$, we have $H = \set{ah: h \in H}$}

If $H$ is a subgroup, then we follow that $a \in H$ implies that
$a\inv \in H$, and thus for every $h \in H$ we follow that $a \inv h
\in H$, and thus $h = e * h = a a\inv h \in \set{ah: h \in H}$, thus
$H \subseteq \set{ah: h \in H}$. Reverse subset argument is trivial, and thus
those two sets are equal, as desired.

qSuppose that for every $a \in H$ we have $H = \set{ah: h \in H}$.
Let $b \in H$ be an arbitrary element. We follow that $a b \in
\set{ah: h \in H} = H$.  We thus follow that all $H = \set{ab * h: h
\in H}$. Thus there's an element $k$ of $h$ such that $abk = a$, which
implies that $bk = e$, and thus $k = b\inv$. Thus we conclude that
$b\inv = k \in H$. Since $b$ is arbitrary, we conclude that $H$
contains inverses of its elements. Since $b\inv \in H$, we follow that
$$H = \set{a h: h \in H}$$
and thus $a b\inv \in H$, which implies that $H$ is a subgroup, as desired.

\subsection{}

\textit{This exercise generalizes the notion of the cyclic subgroup
  generated by an element of a group as described in Example 2.37.
  Let $G$ be a group, and let $S \subseteq G$ be a subset of $G$. The
  subgroup of $G$ generated by $S$, which we denote by $\eangle{S}$,
  is the intersection of all of the subgroups of $G$ that contains $S$;
  i.e.
  $$\eangle{S} = \bigcap_{S \subseteq H \subseteq G: H\text{ is a subgroup of }G} H$$
}

We also note that the set of subgroups of $G$ is a subset of $\pow(G)$,
and thus let us define
$$K = \set{H \in \pow(G): H\text{ is a subgroup of }G \text{ and }S \subseteq H \subseteq G}$$
which is a set and our original definition can be rewritten as
$$\eangle{S} = \bigcap_{H \in K}{H}$$

\textit{(a) Prove that $S$ is not an empty set.}

We follow that $S$ is a subset of $G$, $G \subseteq G$, and $G$ is a
subgroup of $G$, and thus
$$g \in K$$
Thus $K$ is nonempty, which is also important, since there are no
intersections of empty sets.  We follow that every subgroup of $G$
contains $e$, and thus $e \in \eangle{S}$, as desired.

\textit{(b) Prove that $\eangle{S}$ is a subgroup of $G$}

Let $a, b \in \eangle{S}$. We follow that every element of $K$ contains $a, b$, and
since every element of $K$ is a subgroup that $b\inv$ is also in every element of $K$.
Thus $a b\inv$ is also in every element of $K$, and thus $a b\inv \in S$, which implies that
$S$ is a subgroup, as desired.

\textit{(c) Suppose that $L$ is a subgroup of $G$ and that $S \subseteq L$. Prove that
  $\eangle{S} \subseteq L$. (The whole text of this part of the
  exercise is one big typo)
}

We follow that $L \in K$, and thus $\eangle{S} = \bigcap_{H \in K}{H}
\subseteq L$ by definition.

\textit{(d) Let $T$ ve the set of inverses of the elements in $S$; i.e.
  $$T = \set{g\inv: g \in S}$$
  Prove that $\eangle{S}$ is equal to the following set of products
  $$\eangle{S} = \set{e * g_1 g_2 ... g_n: g_{1..n} \in S \cup T}$$
  (I've deviated a tad bit from the definition in the book since we gotta
  take care of the $S = \emptyset$ case))
}

We firstly follow that $S \subseteq \eangle{S}$, and since
$\eangle{S}$ is a subgroup of $G$ we conclude that $T \subseteq
\eangle{S}$ by definition of $T$. We thus follow that for any
$$q \in \set{e * g_1 g_2 ... g_n: g_{1..n} \in S \cup T}$$
$q \in \eangle{S}$ by closure of composition in the subgroup, thus
we've got the $\supseteq$ case.

We now follow that $rhs$ is nonempty by definition (in case when $S =
\emptyset$ we follow that $e \in rhs$). If $a, b \in rhs$, then
there are finite lists $l_1, l_2$ over elements of $S \cap T$ such that
$$\prod{l_1} = a; \prod{l_2} = b$$
we thus follow that $b\inv$ is also in $rhs$ since inverses of $S \cup
T$ is closed under inverses by definition, and thus we've gotta invert
elements of $l_2$ and order them in reverse to get $b\inv$. Thus we conclude
that $a b\inv \in rhs$ since $a b\inv$ will be just a longer list of
elements of $S \cup T$, and thus we follow that $rhs$ is a subgroup of $G$ that
contains $S$. This implies that $\eangle{S} \subseteq rhs$ by definition, which
in tandem with previous paragraph gives us the desired equality.

\subsection{}

\textit{Let $G$ be a finite group, let $m \geq 1$ be an integer, and let
  $$G[m] = \set{g \in G: g^m = e}$$
  be the set of elements of $G$ whose order divides $m$. }

\textit{(a) If $G$ is an abelian group, prove that $G[m]$ is a subgroup of $G$.}

We follow that $e^m = e$, thus $e \in G[m]$, and thus $G[m] \neq
\emptyset$.

We follow that if $b \in G$, then $b^m = e$, thus $(b\inv)^m = e$, and
thus $b\inv \in G[m]$.  Let $a, b \in G[m]$. We follow that $(a b\inv
)^m = a^m b^{-m} = e * e = e$ where we can use the first equality due
to the fact that $G$ is abelian, which implies that $ab\inv \in G[m]$,
which implies that $G[m]$ is a subgroup, as desired.

\textit{(b) Give an example of non-abelian group and $m \geq 1$ such
  that $G[m]$ is not a subgroup of $G$.}

We are aware of several non-abelian groups: permutation group, special
linear group, and dihedral group.

Ler us look at $D_3$ and get all orders of that group:
$$|e| = 1$$
$$|r| = 3$$
$$|r^2| = 3$$
$$|s| = 2$$
$$|sr| = 2$$
$$|sr^2| = 2$$
We thus follow that $D_3[2] = \set{e, s, sr, sr^2}$. We then follow
that $|s sr| = |r| = 3$, and thus we conclude that this set is not
closed under composition, and thus it is not a group, as desired.

\subsection{}

\textit{Skip}

\subsection{}

\textit{Let $G$ be a group, let $A, B \subseteq G$ be subgroups of
  $G$, and let $\phi$ be the map
  $$\phi: A \times B \to G, \phi(a, b) = ab$$}

\textit{(a) Prove that $A \cap B = \set{e}$ if and only if the map $\phi$ is an injection}

Suppose that $A \cap B = \set{e}$. Let $a_1, a_2 \in A$, $b_1, b_2 \in
B$, and suppose that
$$\phi(a_1, b_1) = \phi(a_2, b_2)$$
we follow that
$$a_1 b_1 = a_2 b_2$$
$$a_1  = a_2 b_2 b_1\inv$$
$$a_2 \inv a_1  = b_2 b_1\inv$$
We follow that $a_2 \inv a_1 \in A$ and $b_2 b_1 \inv \in B$, and
since they are equal we conclude that
$$a_2 \inv a_1 = b_2 b_1\inv  = e$$
which implies that $a_1 = a_2$ and $b_1 = b_2$, which in turn implies
that $\phi$ is an injection, as desired.

Now suppose that $\phi$ is an injection. Supposde that there's $q \in
A \cap B$ such that $q \neq e$. Intersection of any two subgroups is a
subgroup, and thus $q\inv \in A \cap B$. And since $q \neq e$ we
follow that $q\inv \neq e$. Thus we follow that $\phi(e, e) = e * e =
e = q q\inv = phi(q, q\inv)$ while $\eangle{e, e} \neq \eangle{q,
q\inv}$, which implies that $\phi$ is not an injection, which is a
contradiction.

\textit{(b) We can turn $A \times B$ into a group by using the group
  operation
  $$\eangle{a_1, b_1} * \eangle{a_2, b_2}  = \eangle{a_1* a_2, b_1 * b_2}$$
  Prove that $\phi$ is a homeomorphism of groups if and only if element of $A$
  commutes with every element of $B$.}

Suppose that $\phi$ is a homeomorphism. This implies that
$$\phi(\eangle{a_1, b_1} * \eangle{a_2, b_2}) =
\phi(\eangle{a_1, b_1}) * \phi(\eangle{a_2, b_2}) =
a_1 b_1 a_2 b_2 $$
it also implies that 
$$\phi(\eangle{a_1, b_1} * \eangle{a_2, b_2}) =
\phi(\eangle{a_1, b_1} * \eangle{a_2, b_2}) =
\phi(\eangle{a_1 * a_2 , b_1 * b_2}) =
a_1 a_2 b_1 b_2 $$
thus
$$a_1 a_2 b_1 b_2 = a_1 b_1 a_2 b_2$$
for all $a_1 a_2 \in A, b_1, b_2 \in B$. By setting $a_1 = b_2 = e$ we
get the desired implication.

By not setting $a_1 = b_1 = e$ and running the argument for the
forward implication (with possible reorder of equalities) in reverse
we can also get the reverse implication, which gives us the desired
result.

\subsection{}

\textit{Let $g$ be a finite group, and let $A, B \subseteq G$ be
subgroup of $G$, and supposet that $gcd(|A|, |B|) = 1$. Prove that $A
\cap B = e$}

$A \cap B$ is a subgroup of both $A$ and $B$, and thus its order gotta
divide orders of both $A$ and $B$. Since $gcd(|A|, |B|) = 1$ we
conclude that order of $A \cap B$ is $1$, which gives us the desired
result.

\subsection{}

\textit{Let $G$ be a group. The center of $G$, denoted by $Z(G)$, is
the seet of elements of $G$ that commute with every other element;
i.e.
$$Z(G) = \set{g \in G: (\forall h \in G)(gh = hg) }$$}

\textit{(a) Prove thtat $Z(G)$ is a subgroup of $G$.}

We follow that for all $h \in H$ we've got that $eh = h = he$, thus
$e \in Z(G)$, and hence $Z(G) \neq \emptyset$

Let $a, b \in Z(G)$. Let $h \in G$ be arbitrary. We follow that
$$bh = hb$$
$$(bh)b\inv = h b b\inv$$
$$(bh)b\inv = h$$
$$b\inv b h b\inv = b\inv h$$
$$h b\inv = b\inv h$$
we thus follow that since $h, b$ are arbitrary that $b \in Z(G)$
implies that $b\inv \in Z(G)$.

With the same assumptions we then follow that
$$(a b\inv) h = a (b\inv h) = a h b\inv = (a h) b\inv = h (a b\inv)$$
thus $a b\inv \in Z(G)$, and thus we conclude that $a, b \in Z(G)$
implies that $ab\inv \in Z(G)$, which with nonemptyness of $Z(G)$
gives us the desired result.

\textit{(b) When does $Z(G)$ equal $G$?}

The obvious case of equality comes when $G$ is abelian. Assume that
$G$ is non-abelian. This implies that there are $a, b \in G$ such that
$ab \neq b a $. We thus conclude that neither $a$ nor $b$ are in
$Z(G)$, and thus $Z(G) \neq G$, which implies that $Z(G) = G$ if and
only if $G$ is abelian.

\textit{(c) Compute the center of the symmetric group $S_n$.}

We follow that for $n \in \set{1, 2}$, $S_n$ is abelian, and thus
$Z(S_2) = S_2$
Let us enumerate elements of $S_3$:
$$S_3 = \set{(1, 2, 3), (1, 3, 2), (2, 1, 3),
  (2, 3, 1), (3, 1, 2), (3, 2, 1) } $$
we follow that $Z(S_3)$ includes at least $e$. Something tells me that
$Z(S_3)$ consists exclusively of $e$. Let us particularly look at the
"shift" operator $(2, 3, 1)$ (the one that maps one element to the
next with mapping the last element to the first). We follow that any
given permutation group contains one. For $n \geq 3$ we can also find
an element, that is distinct from the shift element, that swaps two
elements in place (we're gonna take $(2, 1, 3)$ in this case). We
follow that
$$(2, 3, 1) \circ (2, 1, 3) = (1, 3, 2)$$
$$(2, 1, 3) \circ (2, 3, 1) = (3, 2, 1)$$
thus we follow that shift operator is not in the $Z(S_3)$.

Generally let $a \in \set{1, ..., n}$, let $q \in S_n$ be an element
such that it maps $a$ to itself, and maps $(a + 1) \mod n$ not to
itself, and let $s \in S_n$ be the shift operator. We follow that $(q
\circ s) (a) = q(s(a)) = q(a + 1)$ and $s(q(a)) = s(a) = a + 1$. By
definition $q(a + 1) \neq a + 1$, and thus we conclude that neither
shift operator, nor the element that has partial identity are in
$Z(S_n)$.

With a given example we can generalize even more to create a proof
for our conjecture:

Let $n \geq 3$ and let $q \in S_n$ be such that $q \neq e$. Since $q
\neq e$ we follow that there's an element $a \in \set{1, ..., n}$ such
that $q(a) \neq a$. We then follow that there exists an element $s \in
S_n$ such that $s(a) = a$, and $s(q(a)) \neq q(a)$ (we can do this
because there are at least 3 elements in $\set{1, ..., n}$: we take
the $q(a)$ and some element $b \in \set{1, ..., n} \setminus
\set{q(a), a}$, then map $a$ to itself, and swap $q(a)$ and $b$; the
fact that there's an element that is distinct from both $q(a)$ and $a$
is the restriction that fails for $S_2$). We follow that
$$(s \circ q) (a) = s(q(a)) \neq q(a) = q(s(a)) = (q \circ s) (a)$$
and thus $s \circ q \neq q \circ s$.  Therefore we conclude that $q
\notin Z(S_n)$. Since the only restriction on $q$ is that $q \in S_n
\setminus \set{e}$ we follow that $Z(S_n) = \set{e}$ for all $n \geq
3$, as desired.

\textit{(d) Compute the center of the dihedral group $D_n$.}

We follow that
$$s * r \neq r * s = s * r^{n - 1}$$
thus neither $r$ nor $s$ are in $Z(D_n)$ for $n \geq 3$. We follow
that
$$sr^j * r = sr^{j + 1}$$
$$rsr^j = s r\inv r^j = sr^{j - 1}$$
thus for $n \geq 3$ we follow that $sr^{j + 1} \neq sr^{j - 1}$, and
thus for all $j \in \set{1, ..., n - 1}$ we follow that $sr^j \notin
Z(D_n)$.
We also follow that
$$r^j * s = sr^{n - j}$$
$$s * r^{j} = sr^{j}$$
thus if $j \neq n/2$ we follow that $r^j \notin Z(D_n)$. Thus we
follow that if $n$ is odd that $Z(D_n) = \set{e}$. We follow that
$$r^{n/2} * s r^k = sr^{n - n/2 + k} = sr^{n/2 + k}$$
$$s r^k * r^{n/2} = sr^{n/2 + k}$$
for all $k \in \set{0, ..., n}$. Case with $r^k$ for arbitrary $k$ is
trivial, and thus we conclude that if $n$ is even that $Z(D_n) =
\set{e, r^{n/2}}$, which is kinda surprising.

\textit{(e) Compute the center of the quaternion group $Q$.}

We follow that none of the $i, j, k$ are commuting with each other (by
definition), nor do their inverses, and thus none of them are in
$Z(Q)$. $\pm 1$ on the other hand commute with everything (once again,
by definition), and thus $Z(Q) = \set{\pm 1}$.

\subsection{}

\textit{Let $G$ be a group, and let $g \in G$. The centralizer of $g$,
  denoted $Z_G(g)$ is the set of elements of $G$ that commute with $g$,
  i.e.
  $$Z_G(g) = \set{g' \in G: gg' = g'g}$$
}

\textit{(a) Prove that $Z_G(g)$ is a subgroup of $G$.}

The proof is pretty much the same as with centralizer of the entire
group. We can also note that the definition implies that $Z(G)$ is a
subgroup of $Z_G(g)$ for any given $g \in G$. Also important to note
that $g \in Z_G(g)$.

\textit{(b) Compue the centralizer $Z_G(g)$ for the following groups:}

\textit{(i) $G = D_4$ and $g = r$}

$4$ is even, and thus $\set{e, r^2}$ are in $Z_G(g)$. Any rotation is
also there.  $sr \neq sr^3$, and thus $s$ is not included there. $sr^j
r = sr^j \neq sr{j - 1} = r sr^j$, which implies that none of the
$sr^j$ are in there

\textit{(ii) $G = D_4$ and $g = sr^j$}

$r$ and $r^3$ aren't in there. If $j = 2$, then $s$ is also in there,
otherwise it isn't.  If $j = 2$, then $sr sr^2 = r^{3}$ and $sr^2 sr =
r$, thus $sr$ is not in there.  $sr^3$ is the inverse of $sr$, and
thus it's not in there. The rest can be handled pretty easily.

\textit{(iii) $G = GL_2(R)$ and $g = \
  \begin{pmatrix}
    a & 0 \\
    0 & d
  \end{pmatrix}
  $
}

We follow that this matrix is diagnogal, and thus it commutes with other
diagonal matrices. If matrix is not diagonal then
$$
  \begin{pmatrix}
    a & 0 \\
    0 & d
  \end{pmatrix}
  \begin{pmatrix}
    q & w \\
    e & r
  \end{pmatrix} =
  \begin{pmatrix}
    qa & wa \\
    ed & rd
  \end{pmatrix}
  $$
  $$
  \begin{pmatrix}
    q & w \\
    e & r
  \end{pmatrix}
  \begin{pmatrix}
    a & 0 \\
    0 & d
  \end{pmatrix} =
  \begin{pmatrix}
    qa & wd \\
    ea & rd
  \end{pmatrix}
  $$
  thus we need $wa = wd$ and $ea = ed$. Therefore if $a \neq d$ we
need $w = e = 0$ and this would include only include other diagonal
matrices, that we've already included. If $a = d$, then any matrix
will do.

\textit{(c) Prove that $Z_G(g) = G$ if and only if $g \in Z(G)$.}

IF $g \in Z(G)$, then by definition every element of $G$ commutes
with $G$, and thus $Z_G(g) = G$.

Assume that $Z_G(g) = G$. We follow that every element of $G$ commutes
with $g$, and thus $g \in Z(G)$.

\textit{(d) More generally, if $S \subseteq G$ is any subset of $G$,
then the centralizer of $S$ is the set
  $$Z_G(S) = \set{g \in G: (\forall s \in S)(sg = gs)}$$
  Prove that$ Z_G(S)$ is a subgroup of $G$ }

Proof is borderline identical to the case of centralizer of
group/element.

\subsection{}

\textit{This exercise explains when two elements of $G$ determine the
  same coset of $H$. Let $G$ be a group, let $H$ be a subgroup of $G$,
  and let $g_1, g_2 \in G$. Prove that the following three statements
  are equivalent:}

\textit{(1) $g_1 H = g_2 H$. }

\textit{(2) There are $h \in H$ such that $g_1 = g_2 h$. }

\textit{(3) $g_2\inv g_1 \in H$. }

Assume that $g_1 H = g_2 H$. We follow that $e \in H$ and thus
$g_1 e \in g_2 H$. Thus there's $h \in H$ such that $g_1 e = g_1 = g_2 h$
by definition. Thus
$$g_1 = g_2 h$$
$$g_2 \inv g_1 = h \in H$$
We can run this in reverse to establish equivalence of (2) and (3).
We follow that since $g_1 = g_2 h$ that $g_1 H \cap g_2 H \neq \emptyset$,
and thus they are equal.

\subsection{}

\textit{Let $G$ be a finite group whose only subgroups are $\set{e}$
and $G$.  Prove that either $G = \set{e}$ or else that $G$ is a cyclic
group whose order is prime.}

Set with $G = \set{e}$ is trivial, thus assume that $G \neq \set{e}$.
Let $q \in G \neq e$. We follow that since $G$ is finite that $q$ has
an order, and thus its powers constitute a subgroup
$\eangle{q}$. Since order of $G$ is a prime we follow that order of
$\eangle{q}$ divides the order of $G$ and thus order of  $\eangle{q}$ is
either $1$ or order of $G$. Since $q \neq e$ we conclude the latter,
and thus $\eangle{q} = G$, which implies that $G$ is cyclical, as desired.

\subsection{}

\textit{Let $G$ be a group, and let $K \subseteq H \subseteq G$ be
subgroups. We may thus view $K$ as a subgroup of $G$ or as a subgroup
of $H$. We also recall that the index of a subgroup is its numbder of
distinct cosets. }

\textit{(a) If $G$ is finite, prove the Index Multiplication Rule:
  $$(G:K) = (G:H) (H:K)$$
}

We follow that
$$|H| = |K| (H:K)$$
$$|G| = |H| (G:H)$$
and
$$|G| = |K| (G:K)$$
therefore
$$|G| / |K| =  (G:K)$$
thus
$$|G| = |K| (H:K) (G:H)$$
$$|G| / |K| =  (H:K) (G:H)$$
$$|G| / |K| =  (G:H) (H:K) $$
$$(G:K) =  (G:H) (H:K) $$
as desired.

\textit{(b) Prove that the Index Multiplication Rule is true even if
  $G, H, K$ are allowed to be infinite groups, provided that we assume
  that $(G:K)$ is finite}

Let $Q'$ be the set of distinct cosets of $H$ in $G$.  Let $W'$ be the
set of distinct cosets of $K$ in $H$.  For each $q' \in Q'$ there's an
element $q$ of $G$ such that $qH = q'$. By AoC we can define subset
$Q$ of $G$ that correspond to $Q'$ For each $w' \in W'$ there's an
element $w$ of $H$ such that $wK = w'$. By AoC we can define subset
$W$ of $H$ that correspond to $W'$

Let $j$ be a coset of $K$ in $G$. We follow that there's $g \in G$
such that $gK = j$. We follow that $gH$ is a coset of $H$, and thus
there's a unique $q \in Q$ such that $qH = gH$. We follow then that
$g\inv q \in H$ (see a couple of exercises above), and thus $(q\inv g)
K$ is a coset of $K$ in $H$. Thus there's a unique $w \in W$ such that
$$w K = (q\inv g ) K$$
thus
$$q w K = gK$$
therefore for each coset of $K$ in $G$ there's a pair $\eangle{q, w}
\in Q \times W$. By uniqueness of $q$ and $w$ in $Q$ and $W$,
and the fact that $j$ is an arbitrary coset we
conclude that there's an injection from a set
of cosets of $K$ in $G$ to $Q \times W$.

Let $\eangle{q_1, w_1}, \eangle{q_2, w_2} \in Q \times W$ are such that
$\eangle{q_1, w_1} \neq \eangle{q_2, w_2}$. We follow that
if $q_1 \neq q_2$, and thus $q_1 H \neq q_2 H$ by definition.
$w_1 w_2 \in H$, and thus $w_1 w_2 H = H$. Since $K \subseteq H$ we follow that
$$q_1 w_1 K \subseteq q_1 H$$
$$q_2 w_2 K \subseteq q_2 H$$
and thus
$$q_1 w_1 K \neq q_2 w_2 K$$
If $q_1 = q_2$ we follow that $w_1 \neq w_2$. This implies that
$$w_1 K \neq w_2 K$$
and thus
$$q_1 w_1 K \neq q_1 w_2 K$$
and since $q_1 = q_2$ we follow that
$$q_1 w_1 K \neq q_2 w_2 K$$
This implies that if $\eangle{q_1, w_1}, \eangle{q_2, w_2} \in Q \times W$ then
$q_1 w_1 K \neq q_2 w_2 K$, and thus we conclude that there's an injection from
$Q \times W$ to a set of cosets of $K$. Shroder-Bernstein now implies that
the set of cosets of $K$ in $G$ is equal to $|Q \times W|$. If the latter is finite,
then we follow that $(G:K) = |Q| * |W|$. Definition of $Q$ and $W$ respectively
imply that $|Q| = (G:H)$ and $|W| = (H:K)$, which implies that
$$(G:K) = (G:H) (H:K)$$
as desired.

\subsection{}

\textit{Let $p$ be a prime and let $a \in Z$ be an inteeger satisfying
  $p \not | a$. Use Exercise 2.13(b) and Lagrange's theorem to prove}

\textit{\textbf{Fermat's Little Theorem}: $a^{p - 1} \equiv 1 (\mod p)$}

We follow that if $a \not | p$, then $a \not \equiv 0 \mod p$. We thus follow that
there's $r \in Z_+$ such that $r < p$ and there's an integer $k$ such that
$a = kp + r$. This implies that
$$a^{n} = (kp + r)^n = \sum_{i \in \set{0, ...,  n}}{bc(n, i)(kp)^{i} r^{n - i}}$$
by the binomial theorem ($bc$ is the binomial coefficient). We then
follow that the only element of the sum that does not have $p$ as its
multiple is the one where $i = 0$. Thus we follow that
$$a^{n} \equiv  \sum_{i \in \set{0, ...,  n}}{bc(n, i)(kp)^{i} r^{n - i}} \equiv r^n \mod p$$
thus we follow that the original theorem is equivalent to the
statement that for all $r \in Z_+$ such that $r < p$ we've got that
$$r^{p - 1} \equiv 1 \mod p$$

We know that $1$ is the identity in the multiplicative group
$(Z/pZ)^*$ We also follow that order of $(Z/pZ)^*$ is $p - 1$. We thus
follow that any cyclic subgroup of $(Z/pZ)^*$ has order that divides
$p - 1$. Therefore we follow that for each provided $r$ we've got that
order of $r$ is some divisor $n$ of $p - 1$, and thus
$$r^{p - 1} = r^{n * k} = e^k = e = 1$$
which gives us the desired result.

\section{Products of groups}

\subsection{}

\textit{Let $G_1, G_2, ..., G_n$ be groups. Generalize definition 2.54
  by explaining why
  $$G_1 \times ... \times G_n$$
  with component-wise operation is a group.}

We can follow this case by induction. Interesting thing is to ask that
if $K$ is an indexed set of groups, then is $\prod{K}$ a group?

We can follow that the product of identities is an identity. Inverses
are also handled.  Component-wise association gives us also
association for the product group. $\prod{K}$ is abelian if and only
if all the constructing groups are abelian.

\subsection{}

\textit{Let $G$ be a group, let $A$ and $B$ be subgroups of $G$ and
  consider the map
  $$\phi: A \times B \to G, \phi(a, b) = ab$$}

\textit{(a) If $G$ is an abelian group, prove that $\phi$ is a
  homomorphism.}

Follows from the exercise that we've already finished (there we've handled
the specific case that $B \subseteq Z_G(A)$)

\textit{(b) If $G$ is an abelian group, prove that
  $$ker(\phi) = \set{\eangle{c, c\inv}: c \in A \cap B}$$}

We can follow that application of $\phi$ to the element of $rhs$ will give us
identity by definition, and thus we've got $\supseteq$.

If $\phi(i, j) = e$, then by definition $i j = e$, and thus $i = j\inv$. Thus
we've got the $\subseteq$.

\textit{(c) Suppose that there are elements $a \in A$, $b \in B$ such
thaht $ab \neq ba$.  Prove that $\phi$ is not a homeomorphism}

Exercise that handled this premise before concludes with
biconditional, which gives us this implication

\textit{the rest of the exercises are skipped}

\chapter{Rings - Part 1}

\section{Abstract Rings and Ring Homomorphisms}

\subsection*{Notes}

Distributive law can be generalized through application of
associativity, which will give us
$$a * (b + c + d) = ab + ac + ad$$
and the same for the other direction. We can also by induction
do the same with a general sum.
$$b \sum{a_i} = \sum{b a_i}$$

The idea of ring homomorphism can be summed up to preserving
functions upon application of the function, and mapping identities
to identities.

\subsection{}

\textit{Let $R$ be a ring, and let $a, b \in R$.}

\textit{(a) Prove that
  $$(-a) * (-b) = a * b$$}

$$(-a) * (-b) = (0 + (- a)) * (-b) = (a + (-a) + (-a)) * (-b) = a * (-b) + (-a) * (-b) + (-a) * (-b)$$
(we've added 0 to the $-a$ and then distibuted over it) thus
$$(-a) * (-b) = a * (-b) + (-a) * (-b) + (-a) * (-b)$$
and cancelation law implies that
$$0 = a * (-b) + (-a) * (-b)$$
this in turn implies that
$$(- (a * (-b))) =  (-a) * (-b)$$
we can do a similar thing for $(-b)$, which will result in
$$(- ((-a) * b)) =  (-a) * (-b)$$
Given that everything has additive inverse, we can follow that
$$a + b = (-((-a) * b)) = (- (a * (-b)))$$
which will imply that
$$(-a) * (-b) = (- (a * (-b))) = (- ((-(-a)) * (-b))) = (- ((-(-a)) * (-b))) = (- (- (-(-a)) * b)) = $$
$$ =  (- (- (a * b))) = a * b$$
as desired.

\textit{(b) Our proof of Proposition 3.2(a) used the multiplicative
  identity element $1_R$ of $R$. Find a proof that works even if $R$ does not
  have a multiplicative identity}

We follow that for all $b \in R$
$$ab = a * (b + 0) = ab + a0$$
thus by cancelation law of the group we follow that
$$-(ab) + ab   = -(ab) + ab   + a0$$
$$0 = 0 + a0$$
$$0 = a0$$
as desired.

\subsection{}

\textit{Let $R$ be a commutative ring}

\textit{(a) Suppose that the map
  $$f: R \to R, f(a) = a^2$$
  is a ring homomorphism. Prove that $1_R + 1_R = 0_R$.}

By definition of ring homeomorphism we follow that
$$1_R = f(1_R)$$
and thus we follow that
$$1_R + 1_R = f(1_R) + f(1_R) = f(1_R + 1_R) = (1_R + 1_R)^2 = (1_R + 1_R) (1_R + 1_R) = $$
$$ =  1_R(1_R + 1_R) + 1_R(1_R + 1_R) = (1_R + 1_R) + (1_R + 1_R)$$
thus
$$(1_R + 1_R) = (1_R + 1_R) + (1_R + 1_R)$$
$+$ is a group, thus we're justified to use cancelation to get that
$$1_R + 1_R = 0$$
as desired


\textit{(b) Conversely, if $2 = 0$ in the ring $R$, prove that
  $f(a)^2$ is a homomorphism from $R$ to $R$}

We follow that
$$0^2 = 0$$
$$1^2 = 1$$
so identities are there.

One thing to point out is that $R$ is commutative, which is used in
deriviations here
$$f(a + b) = (a + b)^2 = (a + b)(a + b) = a(a + b) + b(a + b) = $$
$$ = a^2 + ab + ab + b^2 = f(a) + ab + ab + f(b) = f(a) + f(b) + ab(1 + 1) = $$
$$ = f(a) + f(b) +  * 0 = f(a) + f(b)$$
thus we've got the addition
$$f(a * b) = (a * b)^2 = (a * b) * (a * b) = a * b * a * b = a^2 * b^2 = f(a) * f(b)$$
and multiplication, that now satisfies the definition of the
homomorphism.

\textit{(c) Suppose that the map
  $$f: R \to R, f(a) = a^3$$
  is a ring homomorphism. Prove that $6 = 0$ in the ring $R$}

$$ f(1 + 1) = (1 + 1)^3 = (1 + 1) * (1 + 1) * (1 + 1) = 8$$
$$ f(1 + 1) = f(1) + f(1) = 1 + 1 = 2$$
thus
$$8 = 2$$
$$6 = 0$$
as desired. In general we follow that if $f(a) = a^n$ is a
homomorphism, then $2^n - 2 = 0$

\section{Interesting Examples of Rings}

\subsection*{Notes}

When checking for subrings, we follow that associativity of both operations
and their distributivity gets grandfathered by the subset, and thus
the only things that we need to check for are the identities and closures
of operations 

When we're using generalized rings with polynomials, it is possible
that different lists of coefficients might give the same functions.
This is easily observed, when we remember that there exist finite
rings, and thus the space of functions from this ring to itself is
finite, but the list of possible coefficients is infinite.

When we're talking about polynomial equality, we mean that they've got
the same list of coefficients. This essentially means, that a
polynomial is practically a list of elements of the ring, and not the
function.

It's also important to remember, that coefficients of polynomials are
the elements of the underlying ring, but powers are natural
numbers by definition.

We can also theoretically have polynomials over polynomials, and so on,
and so forth.

In the book, the ring of quaternions is defined over coefficients in
$R$, but the subrings of $R$ can be also used to define appropriate
subrings.

Matrix ring is an important example of the fact that two non-zero
elements when multiplied can produce a zero.

Although we haven't observed any product rings (i.e. rings that are
made up by sticking two rings into cartesian product), it's pretty
obvious that we can define them and everything will line up


\subsection{}

\textit{Let $m \geq 1$ be an integer, and define a map
  $$\phi: Z \to Z/ZmZ, \phi(a) = a \mod m$$
  Prove that $\phi$ is a ring homomorphism.}

We firstly follow that $\phi(0) = 0, \phi(1) = 1$ by definition, and
thus identities are mapped properly. Fact that
$$a \equiv c \mod m, b \equiv d \mod m \ra a + b \equiv c + d \mod m, a * b \equiv c * d \mod m$$
implies that all the function properties are satisfied, and thus $\phi$ is a
homeomorphism, as desired.

\subsection{}

\textit{(a) Let $\alpha = 7$ and $\beta = 11$ be elements of the ring
  $Z/17Z$. Compute $\alpha + \beta$ and $\alpha * \beta$.}

We follow that
$$7 + 11 = 18 \equiv 1 \mod 17$$
$$7 * 11 = 77 \equiv 9 \mod 17$$

\textit{(b) Let $\alpha = 2 + 4x$ and $\beta = 1 + 4x + 3x^2$ be
  elements of the ring $(Z/7Z)[x]$. Compute $\alpha + \beta$ and $\alpha *
  \beta$.}

$$(2 + 4x) + (1 + 4x + 3x^2) = 3 + x + 3x^2$$
$$(2 + 4x) * (1 + 4x + 3x^2) = (2 + 1x + 6x^2) + (4x + 2x^2 + 5x^3) = 2 + 5x + x^2 + 5x^3$$

\textit{(c) Let $\alpha = 3 + 2i$ and $\beta = 2 - 3i$ be
  elements of the ring $Z[i]$. Compute $\alpha + \beta$ and $\alpha *
  \beta$.}

$$3 + 2i + 2 - 3i = 5 - i$$
$$(3 + 2i) * (2 - 3i) = 6 - 9i + 4i + 6 = 12 - 5i$$

\textit{(d) Let $\alpha = 3 + 2x - x^2$ and $\beta = 2 - 3x + x^2$ be
  elements of the ring $Z[x]$. Compute $\alpha + \beta$ and $\alpha *
  \beta$.}

$$(3 + 2x - x^2) + (2 - 3x + x^2) = 5 - x$$
$$(3 + 2x - x^2) * (2 - 3x + x^2) =
(6 - 9x + 3x^2) + (4x - 6x^2 + 2x^3) + (-2x^2 + 3x^3 - x^4) = $$
$$= 6 - 15x - 5x^2 + 5x^3 - x^4$$

\textit{(e) Let $R = Z[i]$ and let $\alpha = (1 + i) + (2 - i)x - x^2$
  and $\beta = (2 + i) + (1 + 3i) x$ be elements of the ring
  $Z[x]$. Compute $\alpha + \beta$ and $\alpha * \beta$.}

$$(1 + i) + (2 - i)x - x^2 + (2 + i) + (1 + 3i) x = (2 + 2i) + (3 + 2i)x - x^2$$
$$((1 + i) + (2 - i)x - x^2) * ((2 + i) + (1 + 3i) x) = $$
$$ = 
(2 + 2i + i - 1) + (4 - 2i + 2i + 1)x + (-2 - i)x^2 +
(1 + i + 3i - 3)x  + (2 - i + 6i + 3)x^2 + (-1 - 3i)x^3 = 
$$
$$ =  (1 + 3i) + 5x + (-2 - i)x^2 + (-2 + 4i)x  + (5 + 5i)x^2 + (-1 - 3i)x^3 = $$
$$ =  (1 + 3i) + (3 + 4i)x + (3 + 4i)x^2  + (-1 - 3i)x^3$$

\textit{(f) Let $\alpha = 1 + 2i - j + k$ and $\beta = 2 - i + 3j - k$ be
  elements of the ring $H$. Compute $\alpha + \beta$ and $\alpha *
  \beta$.}

$$1 + 2i - j + k +  2 - i + 3j - k = 3 + i + 2j$$

$$(1 + 2i - j + k) *  (2 - i + 3j - k) = $$
$$ = (2 - i + 3j - k) + 2i(2 - i + 3j - k) - j(2 - i + 3j - k) + k(2 - i + 3j - k) = $$
$$ = (2 - i + 3j - k) + (4i + 2 + 6k + 2j) - (2j + k - 3 - i) + (2k - j - 3i + 1) = $$
$$ = (2 - i + 3j - k) + (2 + 4i + 2j +  6k) + (3 + i - 2j - k) + (1 - 3i - j + 2k) = $$
$$ = 8 + i + 2j + 6k $$

\subsection{}

\textit{For any integer $D$ that is not the square root of an integer,
we can form a ring
  $$Z[\sqrt{D}] = \set{a + b\sqrt{D}: a, b \in Z}$$
  If $D > 0$, then $Z[\sqrt{D}]$ is a subring of $R$, while if $D < 0$,
  then in any case it is a subring of $C$.}

\textit{(a) Let $\alpha = 2 + 3 \sqrt{5}$ and let $\beta = 1 - 2\sqrt{5}$. Compute
  the quntities
  $$a + b, a * b, a^2$$}

We follow that
$$2 + 3 \sqrt{5} + 1 - 2 \sqrt{5} = 3 + \sqrt{5}$$
$$(2 + 3 \sqrt{5}) * (1 - 2 \sqrt{5}) = 2 - 4\sqrt{5} + 3 \sqrt{5} - 30 = -28 - \sqrt{5}$$
$$(2 + 3 \sqrt{5})^2 = 4 + 12 \sqrt{5} + 45 = 49 + 12 \sqrt{5}$$

\textit{(b) Prove that the map
  $$\phi: Z[\sqrt{D}] \to Z[\sqrt{D}], \phi(a + b\sqrt{D}) = a - b \sqrt{D}$$
  is a ring homomorphism. }

We follow that identities are golden.
$$\phi(a + b\sqrt{D} + c + d\sqrt{D}) =  \phi((a + c) + (b + d)\sqrt{D}) = $$
$$ = 
(a + c) - (b + d)\sqrt{D} = a - b\sqrt{D} + c - d\sqrt{D} = \phi(a + b\sqrt{D}) + \phi(c + d\sqrt{D})$$

$$\phi((a + b\sqrt{D}) * (c + d\sqrt{D})) = \phi(ac + ad\sqrt{D} + bc\sqrt{D} + bdD) =
ac + bdD - ad\sqrt{D} - bc\sqrt{D}$$
$$\phi(a + b\sqrt{D}) * \phi(c + d\sqrt{D}) = (a - b\sqrt{D}) * (c - d \sqrt{D}) = $$
$$ ac  + bdD  - ad\sqrt{D} - bc\sqrt{D}$$
thus everything checks out, as desired.

\textit{(c) Prove that for all $\alpha \in Z[\sqrt{D}]$
  $$\alpha *  \overline{\alpha} \in Z$$}
$$\alpha *  \overline{\alpha} = (a + b\sqrt{D}) * (a - b\sqrt{D}) = $$
$$ = a^2 - ab\sqrt{D} + ab\sqrt{D} + b^2D = a^2 + b^2D$$
given that $a, b, D$ are integers, we follow the desired conclusion.

\subsection{}

\textit{Let $\rho$ be the complex number $\rho = \frac{-1 + i\sqrt{3}}{2} \in C$, and let
  $$Z[\rho] = \set{a + b\rho: a, b \in Z}$$}

\textit{(a) Prove tht $\rho^3 = 1$. Thus $\rho$ is a cube root of unity.}

$$\left(\frac{-1 + i\sqrt{3}}{2}\right)^3 =
\left(\frac{\sqrt{3}}{2} i - \frac{1}{2}\right)^3 =
\left(\frac{\sqrt{3}}{2} i \right)^3 -
3 \left(\frac{\sqrt{3}}{2} i \right)^2 \left(\frac{1}{2} \right) +
3 \left(\frac{\sqrt{3}}{2} i \right) \left(\frac{1}{2}\right)^2 -
\left(\frac{1}{2}\right)^3 = $$
$$ = - \frac{3 \sqrt{3}}{8} i  + \frac{9}{8} + \frac{3 \sqrt{3} i}{8}i  - \frac{1}{8} =
1 + 0 = 1$$
as desired.

\textit{(b) Prove that $\rho^2 + \rho + 1 = 0$}

$$\rho^3 - 1 = (\rho - 1)(\rho^2 + \rho + 1)$$
we follow that lhs is not zero, and thus lhs is zero, as desired.

\textit{(c) Prove that the polynomial $X^3 - 1$ factors as
  $$X^3 - 1 = (X - 1)(X - \rho)(X - \rho^2)$$}


We follow that
$$(X - \rho) (X - \rho^2) = X^2 - \rho^2 X - \rho X + 1 = X^2 - \rho^2 X - \rho X - X + X + 1 =$$
$$ = X^2 - X(\rho^2 X+ \rho - 1) + X + 1 = = X^2 - 0X + X + 1 = X^2 + X + 1$$
thus
$$(X - 1) (X - \rho) (X - \rho^2)  = (X - 1) (X^2 + X + 1)$$
and
$$X^3 - 1 = (X - 1)(X^2 + X + 1)$$
as desired.

\textit{(d) Prove that $Z[\rho] $ is a subring of $C$.}

By setting $a, b$ appropriately we get that identites are present. Additive group is
closed by general properties of $C$.

$$(a + b \rho) * (c + d \rho) = ac + ad \rho + bc \rho + bd \rho^2 =
ac + ad \rho + bc \rho + bd \rho^2 + bd \rho + bd - bd\rho - bd = $$
$$ = ac + ad \rho + bc \rho + bd (\rho^2 + \rho + 1) - bd\rho - bd =
ac + ad \rho + bc \rho + bd 0 - bd\rho - bd =$$
$$ = (ac - bd) + \rho(ad + bc - bd)$$
thus this thing is also closed under multiplication, which makes $Z[\rho]$ a
subring, as desired.

\subsection{}

\textit{This one is reasoned in the notes, and is a bunch of arithmetic}

\subsection{}

\textit{Let $R$ be commutative ring, let $c \in R$, and let $E_c: R[x]
  \to R$ be evaluation map $E_c(f) = f(c)$.}

\textit{(a) Prove that $E_c$ is a ring homomorphism.}

We follow that there's a polynomial $x - c$ and $x - c + 1$ that maps
identities.
We follow that
$$E_c(f + g) = (f + g)(c) = f(c) + g(c) = E_c(f) + E_c(g)$$
$$E_c(f * g) = (f * g)(c) = f(c) * g(c) = E_c(f) * E_c(g)$$
by algebraic properties of functions.

\textit{(b) Prove that $E_c(f) = 0$ if ans only if there's a
  polynomial $g(x) \in R[x]$ satisfying $f(x) = (x - c)g(x)$}

$$f(x) = f(x) - 0 = f(x) - f(c) = \sum{a_i (x^j - c^j)}$$
We know that
$$x^n - c^n = (x - c)\sum_{i = 1}^n{x^{n - i} c^i}$$
Which gives us the desired result (needed to look this one up).
Reverse is trivial

\subsection{}

\textit{Let $R$ be a non-commutative ring, with elements $\alpha,
  \beta \in R$ such that $\alpha \neq \beta$. We can still turn $R[x]$
  into a ring by always writing polynomials in the form
  $$p(x) = \sum{a_i x^i}$$
  and using the usual rule that $x$ commutes with all of the elements
  of $R$.  We can also define an evaluation map $E_\alpha: R[x] \to R$
  by setting
  $$E\alpha(\sum{a_i x^i}) = \sum{a_i \alpha^i}$$
  Prove that if $R$ is non-commutative, then $E_\alpha$ is not a
  homomorphism.}

We follow that
$$E\alpha(0) = 0$$
$$E\alpha(1) = 1$$
thus we've got identities, and so there's nothing there.

We know that both $\alpha$ and $\beta$ are polynomials, but
$$E_\alpha(x * \beta) = E_\alpha(\beta x) = \beta \alpha$$
$$E_\alpha(x) * E_\alpha(\beta) = \alpha * \beta$$
which implies that $E_\alpha(x) * E_\alpha(\beta) \neq E_\alpha(x *
\beta)$, which in turn implies that $E_\alpha$ is not a homomorphism,
as desired.

\subsection{}

\textit{Skip}

\subsection{}

\textit{For any commutative ring $R$, let
  $$M_2(R) = \set{
    \begin{pmatrix}
      a & d \\ c & d
    \end{pmatrix}: a, b, c, d \in R
  }$$
  be the set of 2-by-2 matrices with entries in $R$. Define addition
  and multiplication in an obvious way }

\textit{(a) Prove that $M_2(R)$ is a ring.}

Additive identity is the matrix with zeros in it. Multiplicative
identity is the one with multiplicative identites on the diagonal, and
the rest being zeroes.

Properties that this thing is an abelian group
under addition are following from the fact that $R$ is an abelian
group.

Closure under multiplication is derived from the definition of
multiplication.  Detiviations of associativity and distributivity take
a lok of paper, but are pretty uneventful.

\textit{(b) Prove that $M_2(R)$ is non-commutative}

This is where our assumtions that $0 \neq 1$ are kicking in: if
a trivial ring is a singleton, then this thing is commutative.
But we don't assume that, and thus
$$
\begin{pmatrix}
  0 & 1 \\
  1 & 0
\end{pmatrix}
\begin{pmatrix}
  1 & 1 \\
  1 & 0
\end{pmatrix}
=
\begin{pmatrix}
  1 & 0 \\
  1 & 1
\end{pmatrix}
$$
$$
\begin{pmatrix}
  1 & 1 \\
  1 & 0
\end{pmatrix}
\begin{pmatrix}
  0 & 1 \\
  1 & 0
\end{pmatrix}
=
\begin{pmatrix}
  1 & 1 \\
  0 & 1
\end{pmatrix}
$$
which given that $1 \neq 0$ implies the desired result.

\textit{(c) Find non-zero elements $A, B \in M_2(R)$ such that $AB = 0$}
$$
\begin{pmatrix}
  1 & 0 \\
  0 & 0
\end{pmatrix}
\begin{pmatrix}
  0 & 0 \\
  0 & 1
\end{pmatrix} = 0
$$
as desired.

\textit{(d) Let $A \in M_2(R)$. Prove that there's $B \in M_2(R)$
satisfying $AB = I$ if and only if $ad - bc$ has a multiplicative
inverse in $R$.}

Let's start with the forward direction, and assume the appropriate
things. We're gonna implicitly derive the determinant here. We follow that
$$
\begin{pmatrix}
  a & b \\
  c & d
\end{pmatrix}
\begin{pmatrix}
  p & q \\
  r & s
\end{pmatrix} =
\begin{pmatrix}
  pa + rb & qa + sb \\
  pc + rd & qc + sd  
\end{pmatrix} = I
$$
thus we follow that
$$
\begin{cases}
  pa + rb = qc + sd = 1 \\
  qa + sb = pc + rd = 0 \\
\end{cases}
$$
thus

$$
\begin{cases}
  (pa + rb) * (qc + sd) = 1  \\
  (qa + sb) * (pc + rd) = 0
\end{cases}
$$
$$
\begin{cases}
  paqc + pasd + rbqc + rbsd = 1 \\
  qapc + qard + sbpc + sbrd = 0
\end{cases}
$$
$$
\begin{cases}
  paqc + pasd + rbqc + rbsd = 1 \\
  paqc + qard + sbpc + rbsd = 0
\end{cases}
$$
thus
$$ paqc + pasd + rbqc + rbsd -   (paqc + qard + sbpc + rbsd) = 1 $$
$$ pasd + rbqc - (qard + sbpc) = 1 $$
$$ psad + rqbc - rqad - psbc = 1 $$
$$ ps(ad - bc) + rq(bc - ad) = 1 $$
$$ ps(ad - bc) - rq(ad - bc) = 1 $$
$$ (ps - rq) (ad - bc) = 1 $$
which implies that there's $ps - rq \in R$ that is a multiplicative
inverse of $(ad - bc)$, as desired.

Assume that $ad - bc$ has a multiplicative inverse. The following is
just a simple formula for computing inverse of the matrix in $R^2$. We
follow that
$$
\begin{pmatrix}
  a & b \\
  c & d
\end{pmatrix}
\begin{pmatrix}
  d & -b \\
  -c & a
\end{pmatrix} =
\begin{pmatrix}
  da - cb & -ba + ba \\
  -dc + cd & -bc + ad 
\end{pmatrix} =
\begin{pmatrix}
  ad - bc & 0 \\
  0 & ad - bc 
\end{pmatrix}
$$
We then follow that if we would've changed every entry $e$ in the
right matrix for $e / (ad - bc)$, then we would've gotten identity as
the result. Thus we conclude that there's a matrix $B$ such that
$AB = I$, as desired

\textit{(e) Let $A, B \in M_2(R)$. Prove that $AB = I$ if and onyl if $BA = I$}

Given previous point and lots of notation, we can conclude the desired result.



\end{document}

