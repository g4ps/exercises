\documentclass[11pt,oneside,titlepage]{book}
\title{My abstract algebra exercises}
\usepackage{amsmath, amssymb}
\usepackage{geometry}
\usepackage{hyperref}
\author{Evgeny (Gene) Markin}
\date{2024}

\DeclareMathOperator \map {\mathcal {L}}
\DeclareMathOperator \pow {\mathcal {P}}
\DeclareMathOperator \real {\mathbb {R}}
\DeclareMathOperator \topol {\mathcal {T}}
\DeclareMathOperator \basis {\mathcal {B}}
\DeclareMathOperator \ns {null}
\DeclareMathOperator \range {range}
\DeclareMathOperator \fld {fld}
\DeclareMathOperator \inv {^{-1}}
\DeclareMathOperator \Span {span}
\DeclareMathOperator \lra {\Leftrightarrow}
\DeclareMathOperator \eqv {\Leftrightarrow}
\DeclareMathOperator \la {\Leftarrow}
\DeclareMathOperator \ra {\Rightarrow}
\DeclareMathOperator \imp {\Rightarrow}
\DeclareMathOperator \true {true}
\DeclareMathOperator \false {false}
\DeclareMathOperator \dom {dom}
\DeclareMathOperator \ran {ran}
\newcommand{\eangle}[1]{\langle #1 \rangle}
\newcommand{\set}[1]{\{ #1 \}}
\newcommand{\qed}{\hfill $\blacksquare$}



\begin{document}
\maketitle
\tableofcontents

\chapter*{Preface}

This is another yet another attempt at making any progress with
abstract algebra, this time with 'Abstract Algebra: An Integrated
Approach' by Joseph H. Silverman. I really hope that it works out this
time.

So far, it's been most pleasurable journey. This book embodies all the
things, that I like in the mathematics books: lots of rigor and a bit of
lightheartedness, that really lights it up.

After some time with this book my opinion changed a bit in a worse
direction due to an increasing number of typos that I have to deal
with. $\subseteq$ and $\subset$ are often mixed up, I've encountered
an exercise (2.30(c)), whose whole text is one big typo, and some
others. On a brighter note, nothing seems to be as messed up as an
unprovable exercise from the Lovett's book, and every typo is kinda
handled by the context.

Also, the exercise numbering in this book is a tad bit messed up:
it enumerates all the exercises for a given chapter in one sequence
without splitting on the section of a chapter. I don't bother
translating them to the normal format, and by extension some references
in the text of the exercise is a bit messed up.

Some of the notation migrated from my previous endeavours in the
maths. Notably for permutation groups I often use a cyclic
representation, where I do not skip identities.
Some of the notation comes from the book, notable examples of
this notation are:
$$\eangle{g} \text{- a cyclic group that is generated by }g$$
also it is not explicitly defined, but $\cong$ is presumed to
mean isomorphism

I might mix up isometry and isomorphy here and there, but it's
pretty clear from context what exactly do I mean.

Since it's an algebra book, and algebra has rings, that are denoted by
$R$, and those are used in similar context with reals, that are also
denoted by me by $R$, we can see that there's a need to differentiate
between the two. Thus approximately halfway through the chapter about
rings I've started using $\real$ for reals.



\chapter{A Potpourri of Preliminary Topics}

\textit{All of the topics, discussed in this chapter I know already; skip}

\chapter{Groups - Part 1}

\section*{Notes}

I'm gonna use the symbol $*$ as the generic group function and $e$ as
an identity untill stated otherwise since it's most convinient to me.
Sometimes I'll omit $*$ whenever it's clear what's going on. Also,
sometimes I omit parenthesis, but given that we've got associativity,
we can omit them without problems. For rigorousness' sake, I'll
define it to mean left-associative (i.e. $a * b * c = (a * b) * c$).

\section{Introduction to Groups}

\subsection{}

By substituting shapes for numbers we get a trivial exercise


\subsection{}

\textit{Let $n$ be a positive integer, and let $S_n$ be the group of
  permutations of the set $\set{1, 2, ... , n}$ as described in Example
  2.19. Prove that $S_n$ is a finite group, and give a formula for the
  order of $S_n$.}

From the combinatorics we know that the number of permutations is
exacly the factorial of the cardinality of the underlying set.

\subsection{}

\textit{(a) Let $S$ be a finite set, and let $\phi: S \to S$ be a function. Prove that the
  injectivity, surjectivity, and bijectivity of this function are equivatent}

I'm pretty sure that we've proven that rigorously in the set theory
course. If not, then the proof comes from contradiction and theh
cardinality of the codomain of, which proves that injectivity and
surjectivity are equivalent, and bijectivity comes from definition.

\textit{(b) Give an example of an infinite set $S$ and a function
  $\phi: S \to S$ such that $\phi$ is injective but not surjective}

We can let $S = \omega$, $\phi(x) = 2x$, which gives us range of even numbers.

\textit{(c) Give an example of an infinite set $S$ and a function
  $\phi: S \to S$ such that $\phi$ is surjjective but nnot bijective}

We can set $S = \omega$ and
$$\phi(x) =
\begin{cases}
  x = 0 \to 1 \\
  x - 1 \text{ otherwise}
\end{cases}
$$

\subsection{}

\textit{This one involves drawing and is pretty trivial; skip}

\section{Abstract Groups}


\subsection{}

\textit{Let $G$ be a group. In this exercise you will prove the remaining parts of
  Proposition 2.9. Be sure to justify each step using the group axioms or by reference to a
  previously proven fact}

\textit{(a) $G$ has exactly one identity element.}

Suppose that $e_1$ and $e_2$ both satisfy identity axiom. We follow
that both of them are in $G$ and thus
$$e_1 = e_1 e_2 = e_2 e_1 = e_2$$
which comes directly from the Identity Axiom.

\textit{(b) $g, h \in G \ra (g * h)\inv = h\inv g\inv$}

We follow that
$$(g * h)\inv  (g * h) = e$$
by definition of identity. Thus
$$(g * h)\inv  (g * h) * h\inv  = e h\inv$$
$$(g * h)\inv  (g * h) * h\inv  g\inv  = e h\inv g\inv$$
by the fact that $*$ is a binary function. Thus
$$(g * h)\inv  g * (h * h\inv) * g\inv  = e h\inv g\inv$$
$$(g * h)\inv  g * g\inv  = e h\inv g\inv$$
$$(g * h)\inv  (g * g\inv)  = e h\inv g\inv$$
$$(g * h)\inv = e h\inv g\inv$$
by associative laws, and then we've got that
$$(g * h)\inv = h\inv g\inv$$
by the identity axiom, as desired

\textit{(c) $g \in G \ra (g\inv) \inv = g$}

We follow that
$$(g\inv) \inv * (g\inv) = e$$
by the inverse axiom. Thus
$$(g\inv)\inv * (g\inv) * g  = e * g$$
$$(g\inv)\inv * (g\inv) * g  = g$$
by identity and properties of functions. Thus
$$(g\inv)\inv * ((g\inv) * g)  = g$$
$$(g\inv)\inv * e  = g$$
$$(g\inv)\inv  = g$$
by associativity and so on, as desired.

\subsection{}

\textit{Let $G$ be a group, let $g, h \in G$, and suppose that $g$ has
  order $n$ and that $h$ has order $m$.}

\textit{(a) If $G$ is an abelian group and if $gcd(m, n) = 1$, prove that
  the order of $gh$ is $mn$.}

Firstly, I want to follow a couple of things:

If order of $g$ is $m$, and order of $g\inv$ is not $n < m$, then
$$e = e * e = g^m (g\inv)^n = g^{m - n} = e$$
which is a contradiction. Similar case holds for $n > m$.  Thus if
order of $g$ is $n$, then order of $g\inv$ is also $n$.

We also follow that if $g$ has finite order $n$, then $(g^k)^n =
(g^n)^k = e$, and thus $g^k$ has finite order for any $k \in Z$.
Moreover, since $(g^k)^n = e$ we follow that $g^k$'s order divides $n$.

We also follow that if $g$ has finite order $n$, then
$$g * g^{n - 1} = e$$
$$g^{n - 1} = g\inv$$


Now back to our exercise. We follow that
$$(gh)^{mn} = g^{mn} h^{mn} = (g^{n})^m (h^m)^n = e^m e^n = e$$
where we can split it this way since $G$ is abelian. Thus we follow
that order of $gh$ is finite and divides $mn$.

Suppose that the order of $gh$ is $k$. We follow that $k \leq mn$
since it divides $mn$.
$$(gh)^k = e$$
thus
$$(gh)^k = g^k h^k = e$$
if $g^k \neq e$, then we follow that $h^k = (g^k) \inv \neq e$, thus
$h^k = (g\inv)^k = (g^{n - 1})^k$. Thus we follow that $h^k$ is a
multiple of $g$, and thus its order divides order of $g$ $m$. Since
$h^k$ is both multiple of $g$ and $h$, we follow that its order
divides both $m$ and $n$, and since $gcd(m, n) = 1$, we follow that
its order is 1. Thus $h^k = e$, which is a contradiction.

Thus we conclude that $g^k = e$. For $h^k$ we've got a similar
case. Thus we follow that $k$ divides both $m$ and $n$, and thus it's
either $1$ or $mn$. If $k = 1$, then we follow that $g = h\inv$, and
thus $gcd(m, n) = 1$ implies that the order of both $g$ and $h$ cannot
be anything other than $1$, and thus $k = mn = 1$.
If $k \neq 1$, then we follow that $k = mn$, as desired.

\textit{(b) Give an example showing that (a) need not be true of we
allow $gcd(m, n) > 1$}

We can have some group where order of $g$ is $n > 1$ (for example $g =
1$ in $G = Z/5$) and set $h = g\inv$, for which we'll have that order
of $gh$ is 1.

\textit{(c) Give an example of a nonabelian group showing that (a) need not
  be true even if we retain the requirement that $gcd(m, n) = 1$.}

A dihedral group of a triangle with $g = r$ and $h = f$ will do. We'll have that
order of $rf$ is 2:
$$(rf)^2 = (rfrf) = f\inv r r f = e$$
with order of $g$ being $2$ and order of $h$ being $3$

\textit{(d) Again assume that $G$ is an abelian group, and let $l = mn
  / gcd(m, n)$ (i.e.  $l = lcm(m, n)$). Prove that $G$ has an element
  of order $l$.}

We follow that $n$ divides order of $g^m$. We also follow that $m$
divides order of $h^n$.  Since $l = lcm(m, n)$ is divided by both $m$
and $n$ we follow that $g^m h^n$'s order divides $l$.

If there's $k \leq lcm(m, n)$ such that
$$(g^m h^n)^k = e$$
then we follow that
$$g^{mk} h^{nk} = e$$

Here we're gonna employ a more generalized version of an argument in part (a):
If $g^{mk} \neq e$, then $k < lcm(m, n)$, thus  $h^{n}$'s order is a
multiple of both $m$ and $n$. $h^{nk}$ cannot be $1$, and its order
must be then $lcm(m, n)$ or $1$, and it can be neither, thus we've got
a contradiction. Thus we conclude that $g^{mk} = h^{nk} = 1$, which implies that $k$
divides both $m$ and $n$, which implies that $k = lcm(m, n)$, as desired.

\subsection{}

\textit{Definition 2.6 says that a group is a set $G$ with a
composition law satisfying three axioms.  In particular, it says that
there's an identity element $e \in G$ that works on both sides and
that every element $g \in G$ has an inverse that works on both
sides. Suppose that we weaken the requirements to specify that the
identity and inverse work only on one side. In other words, we suppose
that $G$ is a set with a composition law satisfying the following
weaker axioms: }

\textit{(a) (Right-Identity Axiom) There is an element $e \in G$ so
  that $g * e = g$ for all $g \in G$}

\textit{(b) (Right-Inverse Axiom) For all $g \in G$ there is an
  element $h \in G$ so that $g * h = e$}

\textit{(c) (Associative Law) $(g_1 * g_2) * g_3 = g_1 * (g_2 * g_3)$
for all $g_1, g_2, g_3 \in G$.}

\textit{Prove that $G$ is a group.}

We're gonna start with establishing the inverse axiom for $G$,
as suffested in the hint to the exercise.
Suppose that for $g \in G$ there's $h \in G$ such that 
$$g * h = e$$
we also follow that for $h$ there's an element $k$ such that
$h * k = e$ by the same axiom. We thus follow that
$$ h * g = h * g * e = h * g * h * k = h * (g * h) * k = h * e * k = (h * e) * k = h * k = e $$
where we use justification of
$$a \to (\text{h * k = e}) \to c \to b \to c \to a \to (\text{h * k = e})$$
in our equalities (given axioms are presented by letters).
Thus we've got that $h * g = e = g * h$ (i.e. normal inverse axiom)

Now suppose that $g \in G$. We follow that
$$e * g = g * g\inv * g = g * (g\inv * g) =  g * e = g$$
which practially establishes the Identity axiom. The associative law
is unchanged from the stadart definition of the group, and thus we're
following that $G$ is indeed a group, as desired.

\subsection{}

\textit{There are other sorts of algebraic structures that are similar to groups in that
  they are sets $S$ that have a composition law}
$$S \times S \to S, (s_1, s_2) \to s_1 * s_2$$
\textit{but they have fewer or different asiomx than a group. In this exercise we explore two
  of these structures.}

\textit{The set $S$ with its composition law is a monoid if it has an
identity element $e \in S$ and satisfies the associative law, but
elements are not required to have inverses.}

\textit{The set $S$ with its composition law is a semigroup if its
composition law is associative, but it need not have an identity
element or inverses.}

i.e. monoid satisfies associative and identity, semigroup satisfies associative,
and group satisfies all of them. Hence we've got nested classes of structures:

semigroup $\subseteq$ monoid $\subseteq$ group

\textit{For each of the following sets $S$ and composition laws $*$
determine if $(S, *)$ is a group, a monoid, or a semigroup.}

\textit{(a) The set of natural numbers $N = \set{1, 2, 3,
    ...}$ with the composition law being addition.}

I usually do include $0$ into the set of naturals, but in this particular case
it seems that we don't do it.

In this particular case we follow that $N$ has associativity with $+$,
rigorous proof of which comes from the construction of the naturals.
In the course of the set theory I've gone through that proof, but
I'm pretty sure that it's not required here.

If we don't include $0$ into $N$, then we don't have an identity in
$N$, which can be proven by defining order $>$ on $N$ and proceeding
from there (which was also handled in the course on set theory).
Hence it is only a semigroup. If we include it, however, then we get an
identity, and thus this set becomes a monoid.

We can also state that it's not got identities.

\textit{(b) The set of extended  natural numbers $N_0 = \set{0, 1, 2, 3,
    ...}$ with the composition law being addition.}

handled in part (a)

\textit{(c) $(Z, +)$}

We can follow that it's a superset of $N$, which gets its inverses ($a\inv = -a$)
and hence becomes a full-blown group.

\textit{(d) $(N, *)$}

We follow that it's associative, has the identity $1$, and hence is at
least a monoid. It's not got multiplicative inverses (because that
would be $Q_+$), and hence is not a group.

\textit{(e) $(N_0, *)$}

Same as previous, $0$ does not change associativity and identities. Is
not a group for the same reason as the previous case.

\textit{(f) $(Z, *)$}

Same as previous for the same reason.

\textit{(g) The set of integets $Z$ with the composition law $m * n = max\set{m, n}$}

If there's $m \in Z$, then there's $n_0, n_1 \in Z$ such that
$$n_0 < m < n_1$$
and hence
$$n_0 * m = n_0, n_1 * m = m$$
and hence we follow that there's no identity.

We follow that
$$(n * m) * k = max\set{max\set{n, m}, k} = max\set{n, m, k} = max\set{n, max\set{m, k}} = n * (m * k)$$
hence we've got associaativity. Thus the given structure is a semigroup.

\textit{(h) The set of naturals $N$ with the composition law $m * n = max\set{m, n}$}

It's got associativity and hence this thing is a semigroup. We can follow that
$1 \leq m$ for all $m \in N$, and thus $max\set{1, m} = m$. Thus we follow that we also
have a monoid. Inverses are not present, and thus it's not a group.

\textit{(i) The set of naturals $N$ with the composition law $m * n = min\set{m, n}$}

We've got associativity. If $m \in N$, then there's $n \in N$ such that $n > m$, and thus
$min\set{m, n} = m$, which means that there's no identity, and hence this thing is only a semigroup.

\textit{(j) The set of naturals $N$ with the composition law $m * n = mn^2$}

We follow that
$$(m * n) * k = (mn^2) * k = mn^2k^2$$
$$m * (n * k) = m * nk^2 = mn^2k^4$$
setting m, n, k to primes we can follow that $(m * n) * k \neq m * (n * k)$, which implies that
this thing is neither a group, a monoid, nor a semigroup.

\subsection{}

\textit{Look up magmas, Moufang loops, quandles, and matroids}

Magma is just a set with a binanry function. There are some
discussions about closure, but as long as the thing provided with a
set is a binary function, it's a magma.

TODO: look up the rest

\section{Interesting Examples of Groups}

\subsection*{Notes}

Although I'm pretty sure that a the collection of groups is a proper class
(i.e. not a set), I don't have a proof for that. I'll try to change it now.

Empty sets cannot be groups since they've gotta have an element (identity).

Although we can try do to something with cardinals and whatnot, we know that for
any set $S$ there's an injection to the set $S^S$. We then follow that there's
a subset of bijections in $S^S$. We then follow that this subset of bijections
is a group (permutations), and hence we conclude that for each set there's a
group, and thus a collection of groups is a proper class, as desired.

We can also skip all this stuff, and state singleton of any set $S$ is
a trivial group under projection. Trivial group (if I'm not mistaken) is a
group that's got only an identity in it.

Permutation Group practically states that a set of bijections over a
set constitutes a group under composition. There are no further
restrictions, which is pretty neat.

Matrix group leads us to an interesting idea, which is also bleeding
into dihedral groups: as long as a subset of the set of bijections is
closed under composition, includes inverses, and includes an
indentity, given subset is a group (associativity is a property of
composition). Can we lose some restrictions though?

Suppose that there's a set of bijections over a set and it is closed
under composition. If a set is finite, then we can follow that the set of pairs of
bijections is larger than the set of bijections, hence there's got to be a pair
$$S \circ C = S$$
or something like that, which would imply that $C$ is an identity.
If the set is infinite, then this proof will not do.
Finality of the set does not imply the existence of inverses: $\set{C, e}$ will be
closed under composition, will have an identity, and will not have inverses.

Given a set of nonzero naturals $N$, for each $n \in N$ we can have $S_n: \omega \to \omega$
$$S_n(x) = x + n$$
For each $n, m \in N$ we've got that
$$(S_n \circ S_m)(x) = S_n(x + m) = x + m + n = S_{m + n}(x)$$
which gives us a set of bijections, that is closed under a
composition, does not include an identity ($0 \notin N$ by our
restriction), and no element has an inverse.

Thus we conclude that if a set of bijections is finite and is a singleton, then it's a group.
If it's not a singleton and is finite, then it's a monoid (i.e. associativity and identity).
If it's not finite, then it's just a semigroup (i.e. associativity exclusively).

We can't have a group of functions, where the set is not a singleton and
domain and range of the function are distinct, since two functions are supposed
to compose. We can't have non-bijections be present, since we've got to have
both left-hand side and right-hand side inverses of any given element.

\subsection{}

\textit{Let $G$ be a finite cyclic group of order $n$, and let $g$ be
a agenerator of $G$.  Prove that $g^k$ is a generator of $G$ if and
only uf $gcd(k, n) = 1$.}

Suppose that $g^k$ is a generator of $G$. If $k = 1$,
then we follow that $gcd(k, n) = gcd(1, n) = 1$, thus
assume that $k \neq 1$. 

We follow that there's $m \in \omega$ such that $m \neq 0$
and
$$(g^k)^m = g^{km} = g$$
thus
$$g^{km - 1}  = e$$
and hence we follow that $km - 1$ is multiple of order of $G$.  Thus
there exits $j \in Z$ such that $km - 1 = jn$.  Thus $km - jn = 1$,
which implies that $gcd(k, n) = 1$, as desired.

Every implication in the forward direction is pretty much a
biconditional (not exactly though, we need to add some quantifiers to
the mix), so it works in the reverse direction as well.

Also, we haven't gone into a proof that $gcd(m, n) = l$ if and only if
$l$ is the lowest positive number such that there exist $i, j \in Z$
such that
$$mi + jn = l$$
but we've essentially proven the result for $gcd(m, n) = 1$ in 1.35

\subsection{}

\textit{Skip}

\subsection{}

\textit{Prove that the Dihedral group $D_n$, as described in Example
  2.22, has exactly 2n elements}

We can follow that there are $n$ vertices, and each of them gotta go
to one of the other $n$ spaces. There are $n$ ways to put the first vertex
into any of those places. Whenever we put the first vertex $n$ into some place,
there are 2 places where the second vertiex can go, so now we're down to $2n$
possible positions. Whenever we put the second vertex down, positions of the
rest are determined, and hence we can follow that there are total of $2n$
possible positions, as desired.

\subsection{}

\textit{(a) Let $Q^*$ be the set of non-zero rational numbers, with
  the group law being multiplication. Prove that $Q^*$ is a group.}

We follow that $1 * q = q$, and hence we've got an identity.  If $q
\in Q^* \ra q \neq 0 \land q \in Q$, then we follow that $1/q$ exists,
and thus there's $q\inv$ such that $q q\inv = 1$. Multiplication's
associativity does not require a proof (or can be provided, but in
this course it is assumed), and thus we conclude that $Q^*$ is a group
(moreover, an abelian group), as desired.

\textit{the rest is skipped}

\subsection{}

\textit{(b) Let $p$ be a prime number. Prove that the set of non-zero
  elements of $Z/pZ$ is a group using multiplication as the group law}

We fillow that for each $i \in Z/pZ$, $i * 1 = i$, and thus $i$ is our
identity. Associativity of multiplication under $Z$ implies our associativity.

From group theory we know that there's $x$ such that $ax \equiv 1 (mod
m)$ if and only if $gcd(a, m) = 1$.
We then follow that for any given $i \in Z/pZ$ we've got that
$gcd(i, p) = 1$ since $p$ is prime, and thus there's $a \in Z/pZ$
such that $a * i = 1$. Thus we've got the inverse, and $Z/pZ$ is a
group under multiplication, as desired.

We can probably also prove the other direction if we want. If $p$ is
not $1$ and is not a prime, then there's a divisor $q$ of $p$ in
$Z/pZ$, which implies that only multiple of $q$ is its multiple, or
$0$, which implies that it's got no inverse and hence $Z/pZ$ is not a
group, as desired. If $p$ is not a prime, however, then we've got that
$Z/pZ$ is a trivial group. We can probably reword this thing so it
becomes a theorem, but I'm too lazy to do this.

\textit{(c) Heck, I've proven this already and I didnt even read this section}

\textit{(d) blah blah blah}

Got pretty much the same proof as part (a). The only thing of note is
that we've got new notation:
$$(Z/pZ)^*$$
which is a group under multiplication of numbers relatively prime to $p$.

\subsection{}

\textit{Let $C$ be the set of complex numbers, that is, the (... a
crude definition of complex numbers)}

\textit{(a) We make $C$ into ...}

It's a group under addition, yes. Proof is trivial

\textit{(b) $C^* = C \setminus \set{0}$ is a group under multiplication.}

Yes it is. Multiplicative inverse is kinda whacky in complex numbers, let me find it:

$$\frac{1}{a + bi} = \frac{(a - bi)}{(a + bi)(a - bi)} =
\frac{(a - bi)}{a^2 - b^2i^2} = \frac{(a - bi)}{a^2 - b^2 * (-1)}  =
\frac{(a - bi)}{a^2 + b^2}  = \frac{a}{a^2 + b^2}  - \frac{b}{a^2 + b^2}i$$
which I would never ever find on my own, if I wouldn't start a complex
analysis course at some time in the past (probably should look into 
it not, this thing looks pretty fun)

\subsection{}

\textit{(a) Let
$$GL_2(R) = \set{
  \begin{pmatrix}
    a & b \\
    c & d \\
  \end{pmatrix}: a, b, c, d \in R, ad - bc \neq 0}
  $$
  be the indicated set of 2-by-2 matrices, with composition law being matrix multiplication.
  Prove that $GL_2(R)$ is a group.
}

It's a set of bijections, that is closed under composition (see linear
algebra course), each element is invertible by the restriction, and
has identity, thus it's a group.  We can also scale this thing to
complex numbers, and we can also increase dimensions of the underlying
sets and substitute restriction from the determinant-based to just
being invertible (which are equivalent).

\textit{(b)Let
$$SL_2(R) = \set{
  \begin{pmatrix}
    a & b \\
    c & d \\
  \end{pmatrix}: a, b, c, d \in R, ad - bc = 1}
  $$
  be the indicated set of 2-by-2 matrices, with composition law being matrix multiplication.
  Prove that $SL_2(R)$ is a group.
}

Same logic as before. We can scale with determinants here as well.

This group does not include all the isometries though, which is kinda
interesting.  Some isometries (e.g. $1, -1$ on diagonal) will not be
included here. TODO: research this thing a bit more

\textit{(c) ... }

Pretty much is solved by my discussion in previous points. A thing to remeber though:

\textbf{General} linear group is a set of invertible linear functions.

\textbf{Special} linear group is a set of matrices with 1 for the
determinant.

\subsection{}

\textit{Let $GL_2(R)$ be the general linear group. Prove or disprove
  that each of the following subsets of $GL_2(R)$ is a group. In the
  case of non-groups, indicate which of the group conditions fail.
}

\textit{(a)
  $$\set{
  \begin{pmatrix}
    a & b \\
    c & d \\
  \end{pmatrix} \in GL_2(R): a, b, c, d \in R, ad - bc = 2}
  $$
}
Not a group, identity is not in there, and composition is not closed ($|A B | = |A| |B| = 2 * 2 = 4$)

\textit{(b)
  $$\set{
  \begin{pmatrix}
    a & b \\
    c & d \\
  \end{pmatrix} \in GL_2(R): a, b, c, d \in R, ad - bc \in \set{1, -1}}
  $$
}

Is a group, pretty sure that those describe the isometries. 

\textit{(c)
  $$\set{
  \begin{pmatrix}
    a & b \\
    c & d \\
  \end{pmatrix} \in GL_2(R): a, b, c, d \in R, c = 0}
  $$
}

Is a set of upper-triangulars, which are closed under composition, and contain the identity.
Inverses of upper-triangular are also upper-triangular, thus we've got a group.

\textit{(d)
  $$\set{
  \begin{pmatrix}
    a & b \\
    c & d \\
  \end{pmatrix} \in GL_2(R): a, b, c, d \in R, d = 0}
  $$
}

Does not contain the identity, and inverses arent present

\textit{(e)
  $$\set{
  \begin{pmatrix}
    a & b \\
    c & d \\
  \end{pmatrix} \in GL_2(R): a, b, c, d \in R, a = d = 1, c = 0}
  $$
}

Pretty sure that this once is a group as well.

\subsection{}

\textit{Let $Q = \set{\pm 1, \pm i, \pm j, \pm k}$ be the group of quaternions.
  We clamed that the group law for $Q$ is determined by
  $$i^2 = j^k = k^2 = i j k = -1$$
  Use there formulas to prove following formulas, which completely determine the group
  opetrations on $Q$:
}

We firstly follow that $(-1)\inv = -1$. Thus
$$i * i = -1$$
$$(i * i)\inv = (-1)\inv$$
$$i\inv * i\inv = -1$$
$$i\inv = -1 * i $$
$$i\inv = -i$$
where the last part comes from the note that $-1$ commutes with eveything.
We can follow that $k\inv = -k$ and $j\inv = -j$ by the same logic.

We follow now that
$$i * j = i * j * 1 = i * j * (k * -k) = (i * j * k) * -k = -1 * -k = k$$
$$j * k = 1 * j * k = (-i * i) * j * k = -i * -1 = i$$
$$k * i =  1 * k * i = -j * j * k * i = -j * (j * k) * i = -j * i * i = -j * -1 = j$$
we then follow that
$$-k = k\inv = (i * j) \inv = -j * -i = j * i$$
and so on for all the inverses.

\subsection{}

Skip

\subsection{}

\textit{Practically continue with the functions in the notes. Part (a)
is handled}

\textit{(b) If $X$ is a finite set with $n$ elements, Prove that
  $\mathcal{E}_X$ is a finite monoid and compute how many elements it has.}

We follow that the number of functions is $n^n$, the rest is handled in the notes

\textit{(c) If $|X| \geq 3$, prove that $\mathcal{E}_X$ is not commutative, i.e.
  show that there are elements $\phi, \psi \in \mathcal{E}_X$ satisfying
  $\phi \circ \psi \neq \psi \circ \phi$.}

We can map a portion of this set to ${1, 2, 3}$, and then get functions
$$\phi(x) =
\begin{cases}
  1 \to 2 \\
  2 \to 2 \\
  3 \to 2 \\ 
\end{cases}
$$
$$\psi(x) =
\begin{cases}
  1 \to 3 \\
  2 \to 3 \\
  3 \to 3 \\ 
\end{cases}
$$
which will suffice.

\section{Group Homomorphism}

\subsection{}

\textit{Recall that two groups $G_1$, $G_2$ are said to be isomorphic
  if there is a bijective homomorphism
  $$\phi: G_1 \to G_2$$
  The fact that $\phi$ is bijective means that the inverse map $\phi \inv : G_2 \to G_1$ exists.
  Prove that $\phi_1$ is a homomorphism from $G_2$ to $G_1$. 
}

We essentially need to prove that
$$\phi\inv(h_1 * h_2) = \phi \inv (h_1) * \phi \inv (h_2)$$
for all $h_1, h_2 \in G_2$.

$\phi$ is a bijection, and thus for $h_1, h_2 \in G_2$ there are $g_1, g_2 \in G_1$ such that
$\phi(g_1) = h_1, \phi(g_2) = h_2$. By the fact that $\phi$ is a bijection we also follow that
$g_1 = \phi\inv(h_1), g_2 = \phi\inv(h_2)$. Thus we've got that 
$$\phi\inv(h_1 * h_2) = \phi\inv(\phi(g_1) * \phi(g_2)) =
\phi\inv(\phi(g_1 * g_2)) = g_1 * g_2  = \phi\inv(h_1) * \phi\inv(h_2) $$
as desired

\subsection{}

\textit{Let $G$ be a group, and consider the functiion
  $$\phi: G \to G, \phi(g) = g\inv$$
}

\textit{(a) Prove that $\phi(\phi(g)) = g$ for all $g \in G$}

$$\phi(\phi(g)) = \phi(g\inv) = (g\inv)\inv = g$$

\textit{(b) Prove that $\phi$ is a bijection.}

We follow that for each $g \in G$ there's $g\inv \in G$, thus
$\phi(g\inv) = g$, which implies that the range of $\phi$ is $G$.

We follow that if $g_1 \neg g_2$, then $g_1\inv \neq g_2\inv$, and thus
$\phi$ is injective.

\textit{(c) Prove that $\phi$ is a group homomorphism if and ony if
  $G$ is an abelian group.}

\textbf{Forward direction:}
We know that $\phi$ is bijective, and thus for all $h_1, h_2 \in G$
there are $g_1, g_2 \in G$ such that $\phi(g_1) = h_1, \phi(g_2) =
h_2$. If $G$ is a group homomorphism, then we follow that
$$h_1 * h_2 = \phi(g_1) \phi(g_2) = \phi(g_1 g_2) = (g_1 g_2)\inv = g_2\inv g_1 \inv = h_2 * h_1$$
thus the group is abelian.

\textbf{Reverse direction:}

Let us keep assumptions about our variables. If $G$ is abelian, then
we follow that
$$h_2 * h_1 = g_2\inv g_1 \inv  =  (g_1 g_2)\inv  = \phi(g_1 g_2)$$
$$h_1 * h_2 = g_1\inv g_2 \inv  =  \phi(g_1) * \phi(g_2)$$
Since $G$ is abelian we follow that $h_1 h_2 = h_2 h_1$, and thus
we can equate given equalities to get the desired result.

\subsection{}

Skip

\subsection{}

\textit{Let $G_1$ and $G_2$ be groups, and suppose that $\phi: G_1 \to
  G_2$ is an isomorphism}

\textit{(a) Prove that if $G_1$ is finite, then $G_2$ is also finite, and
  that they satify $|G_1| =_c |G_2|$}

Isomorphism is a bijection, thus sets have equal cardinality.

\textit{(b) Suppose that $G_1$ is abelian. Prove that $G_2$ is
abelian}

Let $g_1, g_2 \in G_2$. Since $\phi$ is a bijection we follow that there are
$h_1, h_2 \in G_1$ such that $\phi(h_1) = g_1$ and $\phi(h_2) = g_2$.
We follow that
$$g_1 g_2 = \phi(h_1) \phi(h_2) = \phi(h_1 h_2) = \phi(h_2 h_1) = \phi(h_2) \phi(h_1) = g_2 g_1$$
as desired.

\textit{The solution for the rest of this exercise practically copies
  the structure of part (a) }

\subsection{}

\textit{In this exericse, $C_n$ is a cyclic group of order $n$, $D_n$
  is the $n$h dihedral group, and $S_n$ is the $n$th symmetric group.}

\textit{(a) Prove that $C_2$ and $S_2$ are isomorphic.}

We follow that both $C_2$ and $S_2$ have $2$ and $2! = 2$ elements
respectively. Non-identity elements can interact with identity in a
predictible way, or with themselves, where they've got an order of
$2$. This pretty much concludes the proof.

\textit{(b) Prove that $D_3$ is isomorphic to $S_3$.}

We know that $D_3$ is a subgroup (although we haven't defined at this
point what a subgroup is, it doesn't take a genius to guess) of $S_3$,
with the same cardinality.

\textit{(c) Let $m \geq 3$. Prove that for every $n$, the groups $C_m$ and $S_n$ are not
  isomorphic.}

If $m \neq n!$, then we follow that groups have different cardinalities, thus
there's no bijection between the two, and thus no isomorphisms.

If $m = n!$, then we follow that since $m \geq 3$ that $n \geq 3$ as
well.  We follow that there are elements of $S_n$ that map $(1, 2, 3)$
to $(1, 3, 2)$ and $(2, 1, 3)$ and the rest to identity. Composition
of those elements is not commutative ($(2, 3, 1)$ and $(3, 1, 2)$ in
forward and reverse respectively), and thus $S_n$ is not abelian.
$C_n$ on the other hand is commutative, and thus we follow
that they aren't isomorhic. This also proves that $S_n$ is non-abelian
for any $n \geq 3$.

\textit{(d) Prove that for every $n \geq 4$ the groups $D_n$ and $S_n$
  aren't isomorphic}

They dont have the same cardinalities.

\textit{(e) More generally, let $m \geq 4$, and let $n \geq 4$. Prove
  that the groups $D_m$ and $S_n$ aren't isomorphic.}

We once again look at the case when $2m = n!$. We probably want to
find some irregularities with orders of the elements here.  We follow
that if $m$ is odd, then there is only 1 element of order $2$ ($f$).
If $m$ is even, then we've got 3 such elements ($f, r^{m/2}, fr^{m/2}$).
For $S_n$ for $n \geq 4$ we follow that there are at least 6 of them
(bijections that swap elements, there are $C(4, 2) = 6$ of them at least).

\textit{(f) The cyclic group $C_8$, the dihedral group $D_4$, and the quaternion group
  $Q$ are non-abelian groups of order $8$. Prove that they ren't isomorphic.}

First of all, $C_8$ is abelian (as is any cyclic group for that
matter). If we ignore that, then we can follow that there is one
element of order 4 ($2$) in $C_8$, and there are 6 ($\pm i, \pm j, \pm
k$) in $Q$. There are 3 elements of order $2$ in $D_4$, but only one
in $C_8$. $Q$ has 1 element of order $2$.

\subsection{}

Skip

\section{Subgroups, Cosets, and Lagrange's Theorem}

\subsection*{Notes}

Important definition from the exercises:
\textit{Let $G$ be a group, and let $S \subseteq G$ be a subset of
$G$. The subgroup of $G$ generated by $S$, which we denote by
$\eangle{S}$, is the intersection of all of the subgroups of $G$ that
contains $S$; i.e.
  $$\eangle{S} = \bigcap_{S \subseteq H \subseteq G: H\text{ is a subgroup of }G} H$$}
Exercise proves that $\eangle{S}$ is essentially a subgroup, where
every element is a multiple of some combination (i.e. ordered list in
non-abelian and multiset in abelian) of elements of $S$ or their
inverses, and also the smallest subgroup, that contains the entirety
of $S$.

Another important definition is the definition of centralizer.  The
centralizer of the subset ($Z_G(S)$) is the most general of them, they
are provided below

We also follow that if $H_1, H_2$ are subgroups of $G$, then $H_1 \cap
H_2$ is a subgroup of $G$, which is sort of trivial to prove. Importatnt
to note that union of two subgroups is not necessarily a group.

\subsection{}

\textit{Let $G$ be a group and let $g \in G$ be an element of order
  $n$, and let $k \geq 1$.}

\textit{(a) Prove that $g^k$ has order $n / gcd(n, k)$.}

We follow that $g$ is a generator for a cyclic group, in which
multiples of $g^k$ comprises a subgroup. Since $g$ has order $n$ we
follow that $\eangle{g}$ is a cyclic group of order $n$.

We know that $gcd(n, k)$ is the lowest number that is a positive sum
of multiples of $n$ and $k$ respectively. Thus there are integers
$a, v$ such that
$$an + vk = gcd(n, k)$$
We then follow that
$$(g^n)^a * (g^k)^v = g^{gcd(n, k)}$$
order of $g$ is $n$ and thus
$$(g^n)^a * (g^k)^v = (e)^a * (g^k)^v = e * (g^k)^v = (g^k)^v = g^{gcd(n, k)}$$
We thus follow that multiples of $g^{gcd(n, k)}$ are in cyclic
subgroup of $g^k$. There are exacltly $n / gcd(n, k)$ such elements.

Suppose that $v$ is not a multiple of $gcd(n, k)$ and suppose that
there's $j \in Z_+$ such that $(g^k)^j = g^{k * j} = g^v$. Since $v$ is
not a multiple of $gcd(n, k)$, it is not a multiple of $k$. Thus we
follow that $k * j > n$. We thus conclude that there's a maximal $h \in Z_+$
such that
$$kj - hn = v$$
$$kj + (-h)n = v$$
thus we conclude that $v$ is a multiple of $gcd(n, k)$, which is a
contradiction.

\textit{(b) Use (a) to give a quick proof of Exercise 2.10, which says
  that $G = \eangle{g}$ is a cyclic group of order $n$, then $g^k$
  generates $G$ if and only if $gcd(n, k) = 1$ }

We follow that if $g^k$ generates $G$, then $|\eangle{g^k}| = n / gcd(n,
k) = n$. Thus $gcd(n, k) = 1$. Reverse case is pretty much the same.

\subsection{}

\textit{Let $G$ be a cyclic group of order $n$, and let $d \geq 1$ be
  an integer.}

\textit{(a) Prove that every subgroup of $G$ is cyclic.}

Since $G$ is cyclic we follow that there's $g \in G$ that generates
$G$.

Suppose that $H$ is a subgroup of $G$. If $g \in H$, then $G = H$, and
thus we're done. Thus suppose that $g \notin H$. We then follow that
there is a lowest possible $i \in Z_+$ such that $g^i \in H$. We
follow that every multiple of $i$ is in $H$. Suppose that $k \in Z_+$
is not a multiple of $i$ and such that $g^k \in H$. We follow that
$gcd(k, i) < i$ since $k$ is not a multiple of $i$. Thus we follow
that there are $a, v \in Z$ such that
$$ak + vi = gcd(k, i)$$
and thus
$$g^{ak} + g^{vi} = g^{gcd(k, i)}$$
thus $g^{gcd(k, i)} \in H$, which implies that $i$ is not the lowest element
of $Z_+$ such that $g^i \in H$, which is a contradiction. Thus we
conclude that multiples of $i$ are the only elements of $H$, which implies that
$H$ is cyclic, as desired.

Important note: this argument also works for groups $G$, and so we can follow that
any subgroup of any cyclic group is cyclic.

\textit{(b) If $d$ divides $n$, prove that $G$ has a unique subgroup of order $d$.}

Let $H_1$ and $H_2$ be two subgroups of order $d$. We follow that for
both of there are least elements $i_1, i_2$ of $Z_+$ such that
$g^{i_1} \in H_1$ $g^{i_2} \in H_2$. Suppose that $i_1 \neq i_2$ and
assume that $i_1 < i_2$.  Previous arguments in (a) imply that $g^{i_1}$ and
$g^{i_2}$ generate $H_1$ and $H_2$. We know also that $H_1$ is comrised entirely of
powers of $g^{i_1}$. If $gcd(i_1, d) \neq i$, we follow that $gcd(i_1, d) < i$,
and thus there are $a, v \in Z$ such that
$$(g^{i_1})^a * (g^{d})^v = g^{gcd(i_1, d)}$$
$$(g^{i_1})^a * (e)^v = g^{gcd(i_1, d)}$$
$$(g^{i_1})^a  = g^{gcd(i_1, d)}$$
We thus follow that $g^{gcd(i_1, d)} \in H_1$, and thus $i_1$ is not
the lowest positive integer such that $g^{i_1} \in H$, which is a
contradiction. We thus conclude that $gcd(i_1, d) = i_1$, and thus
$i_1 | d$. This implies that $|H_1| = |\eangle{g^{i_1}}| = d / i_1$.
Thus if $i_1 \neq i_2$ we follow that $|H_1| \neq |H_2|$, which is a
contradiction. Thus we conclude that $i_1 = i_2$, and thus $H_1 = H_2$,
as desired.

\textit{(c) If $d$ does not divide $n$, prove that $G$ does not have a subgroup of order $d$.}

We follow that number of cosets of $G$ of order $d$ is a natural number, and thus
Lagrange's Theorem implies that order of any subgroup of $G$ divides $n$, as desired.

\subsection{}

\textit{Let $G$ be a group, and let $H \subseteq G$. Prove that $H$ is a subgroup
  if and only if it has the following properties:}

\textit{(1) $H \neq \emptyset$}

\textit{(2) For every $h_1, h_2 \in H$, $h_1 * h_2 \inv \in H$}

Forward direction is covered by axioms.

Suppose that $H \neq \emptyset$ and for every $h_1, h_2 \in H$, $h_1 *
h_2 \inv \in H$. We follow that there's $h \in H$, and thus
$h * h\inv = e \in H$. Thus we follow the identity axiom.

Suppose that $k \in H$. We follow that $k \in H$ and $e \in H$, thus
$e * k\inv = k\inv \in H$. Thus we follow the inverse axiom.

Let $h_1, h_2 \in H$. We follow that $h_2 \inv \in H$, and thus
$h_1 * h_2 \inv \inv = h_1 * h_2 \in H$, thus giving us closure,
which completes the definition of the subgroup, as desired.

\subsection{}

\textit{Let $G$ be a group, and let $H \subseteq G$ be a nonempty subset of $G$. Prove that
  $H$ is a subgroup if and only if it has the following property:}

\textit{For every $a \in H$, we have $H = \set{ah: h \in H}$}

If $H$ is a subgroup, then we follow that $a \in H$ implies that
$a\inv \in H$, and thus for every $h \in H$ we follow that $a \inv h
\in H$, and thus $h = e * h = a a\inv h \in \set{ah: h \in H}$, thus
$H \subseteq \set{ah: h \in H}$. Reverse subset argument is trivial, and thus
those two sets are equal, as desired.

qSuppose that for every $a \in H$ we have $H = \set{ah: h \in H}$.
Let $b \in H$ be an arbitrary element. We follow that $a b \in
\set{ah: h \in H} = H$.  We thus follow that all $H = \set{ab * h: h
\in H}$. Thus there's an element $k$ of $h$ such that $abk = a$, which
implies that $bk = e$, and thus $k = b\inv$. Thus we conclude that
$b\inv = k \in H$. Since $b$ is arbitrary, we conclude that $H$
contains inverses of its elements. Since $b\inv \in H$, we follow that
$$H = \set{a h: h \in H}$$
and thus $a b\inv \in H$, which implies that $H$ is a subgroup, as desired.

\subsection{}

\textit{This exercise generalizes the notion of the cyclic subgroup
  generated by an element of a group as described in Example 2.37.
  Let $G$ be a group, and let $S \subseteq G$ be a subset of $G$. The
  subgroup of $G$ generated by $S$, which we denote by $\eangle{S}$,
  is the intersection of all of the subgroups of $G$ that contains $S$;
  i.e.
  $$\eangle{S} = \bigcap_{S \subseteq H \subseteq G: H\text{ is a subgroup of }G} H$$
}

We also note that the set of subgroups of $G$ is a subset of $\pow(G)$,
and thus let us define
$$K = \set{H \in \pow(G): H\text{ is a subgroup of }G \text{ and }S \subseteq H \subseteq G}$$
which is a set and our original definition can be rewritten as
$$\eangle{S} = \bigcap_{H \in K}{H}$$

\textit{(a) Prove that $S$ is not an empty set.}

We follow that $S$ is a subset of $G$, $G \subseteq G$, and $G$ is a
subgroup of $G$, and thus
$$g \in K$$
Thus $K$ is nonempty, which is also important, since there are no
intersections of empty sets.  We follow that every subgroup of $G$
contains $e$, and thus $e \in \eangle{S}$, as desired.

\textit{(b) Prove that $\eangle{S}$ is a subgroup of $G$}

Let $a, b \in \eangle{S}$. We follow that every element of $K$ contains $a, b$, and
since every element of $K$ is a subgroup that $b\inv$ is also in every element of $K$.
Thus $a b\inv$ is also in every element of $K$, and thus $a b\inv \in S$, which implies that
$S$ is a subgroup, as desired.

\textit{(c) Suppose that $L$ is a subgroup of $G$ and that $S \subseteq L$. Prove that
  $\eangle{S} \subseteq L$. (The whole text of this part of the
  exercise is one big typo)
}

We follow that $L \in K$, and thus $\eangle{S} = \bigcap_{H \in K}{H}
\subseteq L$ by definition.

\textit{(d) Let $T$ ve the set of inverses of the elements in $S$; i.e.
  $$T = \set{g\inv: g \in S}$$
  Prove that $\eangle{S}$ is equal to the following set of products
  $$\eangle{S} = \set{e * g_1 g_2 ... g_n: g_{1..n} \in S \cup T}$$
  (I've deviated a tad bit from the definition in the book since we gotta
  take care of the $S = \emptyset$ case))
}

We firstly follow that $S \subseteq \eangle{S}$, and since
$\eangle{S}$ is a subgroup of $G$ we conclude that $T \subseteq
\eangle{S}$ by definition of $T$. We thus follow that for any
$$q \in \set{e * g_1 g_2 ... g_n: g_{1..n} \in S \cup T}$$
$q \in \eangle{S}$ by closure of composition in the subgroup, thus
we've got the $\supseteq$ case.

We now follow that $rhs$ is nonempty by definition (in case when $S =
\emptyset$ we follow that $e \in rhs$). If $a, b \in rhs$, then
there are finite lists $l_1, l_2$ over elements of $S \cap T$ such that
$$\prod{l_1} = a; \prod{l_2} = b$$
we thus follow that $b\inv$ is also in $rhs$ since inverses of $S \cup
T$ is closed under inverses by definition, and thus we've gotta invert
elements of $l_2$ and order them in reverse to get $b\inv$. Thus we conclude
that $a b\inv \in rhs$ since $a b\inv$ will be just a longer list of
elements of $S \cup T$, and thus we follow that $rhs$ is a subgroup of $G$ that
contains $S$. This implies that $\eangle{S} \subseteq rhs$ by definition, which
in tandem with previous paragraph gives us the desired equality.

\subsection{}

\textit{Let $G$ be a finite group, let $m \geq 1$ be an integer, and let
  $$G[m] = \set{g \in G: g^m = e}$$
  be the set of elements of $G$ whose order divides $m$. }

\textit{(a) If $G$ is an abelian group, prove that $G[m]$ is a subgroup of $G$.}

We follow that $e^m = e$, thus $e \in G[m]$, and thus $G[m] \neq
\emptyset$.

We follow that if $b \in G$, then $b^m = e$, thus $(b\inv)^m = e$, and
thus $b\inv \in G[m]$.  Let $a, b \in G[m]$. We follow that $(a b\inv
)^m = a^m b^{-m} = e * e = e$ where we can use the first equality due
to the fact that $G$ is abelian, which implies that $ab\inv \in G[m]$,
which implies that $G[m]$ is a subgroup, as desired.

\textit{(b) Give an example of non-abelian group and $m \geq 1$ such
  that $G[m]$ is not a subgroup of $G$.}

We are aware of several non-abelian groups: permutation group, special
linear group, and dihedral group.

Ler us look at $D_3$ and get all orders of that group:
$$|e| = 1$$
$$|r| = 3$$
$$|r^2| = 3$$
$$|s| = 2$$
$$|sr| = 2$$
$$|sr^2| = 2$$
We thus follow that $D_3[2] = \set{e, s, sr, sr^2}$. We then follow
that $|s sr| = |r| = 3$, and thus we conclude that this set is not
closed under composition, and thus it is not a group, as desired.

\subsection{}

\textit{Skip}

\subsection{}

\textit{Let $G$ be a group, let $A, B \subseteq G$ be subgroups of
  $G$, and let $\phi$ be the map
  $$\phi: A \times B \to G, \phi(a, b) = ab$$}

\textit{(a) Prove that $A \cap B = \set{e}$ if and only if the map $\phi$ is an injection}

Suppose that $A \cap B = \set{e}$. Let $a_1, a_2 \in A$, $b_1, b_2 \in
B$, and suppose that
$$\phi(a_1, b_1) = \phi(a_2, b_2)$$
we follow that
$$a_1 b_1 = a_2 b_2$$
$$a_1  = a_2 b_2 b_1\inv$$
$$a_2 \inv a_1  = b_2 b_1\inv$$
We follow that $a_2 \inv a_1 \in A$ and $b_2 b_1 \inv \in B$, and
since they are equal we conclude that
$$a_2 \inv a_1 = b_2 b_1\inv  = e$$
which implies that $a_1 = a_2$ and $b_1 = b_2$, which in turn implies
that $\phi$ is an injection, as desired.

Now suppose that $\phi$ is an injection. Supposde that there's $q \in
A \cap B$ such that $q \neq e$. Intersection of any two subgroups is a
subgroup, and thus $q\inv \in A \cap B$. And since $q \neq e$ we
follow that $q\inv \neq e$. Thus we follow that $\phi(e, e) = e * e =
e = q q\inv = phi(q, q\inv)$ while $\eangle{e, e} \neq \eangle{q,
q\inv}$, which implies that $\phi$ is not an injection, which is a
contradiction.

\textit{(b) We can turn $A \times B$ into a group by using the group
  operation
  $$\eangle{a_1, b_1} * \eangle{a_2, b_2}  = \eangle{a_1* a_2, b_1 * b_2}$$
  Prove that $\phi$ is a homeomorphism of groups if and only if element of $A$
  commutes with every element of $B$.}

Suppose that $\phi$ is a homeomorphism. This implies that
$$\phi(\eangle{a_1, b_1} * \eangle{a_2, b_2}) =
\phi(\eangle{a_1, b_1}) * \phi(\eangle{a_2, b_2}) =
a_1 b_1 a_2 b_2 $$
it also implies that 
$$\phi(\eangle{a_1, b_1} * \eangle{a_2, b_2}) =
\phi(\eangle{a_1, b_1} * \eangle{a_2, b_2}) =
\phi(\eangle{a_1 * a_2 , b_1 * b_2}) =
a_1 a_2 b_1 b_2 $$
thus
$$a_1 a_2 b_1 b_2 = a_1 b_1 a_2 b_2$$
for all $a_1 a_2 \in A, b_1, b_2 \in B$. By setting $a_1 = b_2 = e$ we
get the desired implication.

By not setting $a_1 = b_1 = e$ and running the argument for the
forward implication (with possible reorder of equalities) in reverse
we can also get the reverse implication, which gives us the desired
result.

\subsection{}

\textit{Let $g$ be a finite group, and let $A, B \subseteq G$ be
subgroup of $G$, and supposet that $gcd(|A|, |B|) = 1$. Prove that $A
\cap B = e$}

$A \cap B$ is a subgroup of both $A$ and $B$, and thus its order gotta
divide orders of both $A$ and $B$. Since $gcd(|A|, |B|) = 1$ we
conclude that order of $A \cap B$ is $1$, which gives us the desired
result.

\subsection{}

\textit{Let $G$ be a group. The center of $G$, denoted by $Z(G)$, is
the seet of elements of $G$ that commute with every other element;
i.e.
$$Z(G) = \set{g \in G: (\forall h \in G)(gh = hg) }$$}

\textit{(a) Prove thtat $Z(G)$ is a subgroup of $G$.}

We follow that for all $h \in H$ we've got that $eh = h = he$, thus
$e \in Z(G)$, and hence $Z(G) \neq \emptyset$

Let $a, b \in Z(G)$. Let $h \in G$ be arbitrary. We follow that
$$bh = hb$$
$$(bh)b\inv = h b b\inv$$
$$(bh)b\inv = h$$
$$b\inv b h b\inv = b\inv h$$
$$h b\inv = b\inv h$$
we thus follow that since $h, b$ are arbitrary that $b \in Z(G)$
implies that $b\inv \in Z(G)$.

With the same assumptions we then follow that
$$(a b\inv) h = a (b\inv h) = a h b\inv = (a h) b\inv = h (a b\inv)$$
thus $a b\inv \in Z(G)$, and thus we conclude that $a, b \in Z(G)$
implies that $ab\inv \in Z(G)$, which with nonemptyness of $Z(G)$
gives us the desired result.

\textit{(b) When does $Z(G)$ equal $G$?}

The obvious case of equality comes when $G$ is abelian. Assume that
$G$ is non-abelian. This implies that there are $a, b \in G$ such that
$ab \neq b a $. We thus conclude that neither $a$ nor $b$ are in
$Z(G)$, and thus $Z(G) \neq G$, which implies that $Z(G) = G$ if and
only if $G$ is abelian.

\textit{(c) Compute the center of the symmetric group $S_n$.}

We follow that for $n \in \set{1, 2}$, $S_n$ is abelian, and thus
$Z(S_2) = S_2$
Let us enumerate elements of $S_3$:
$$S_3 = \set{(1, 2, 3), (1, 3, 2), (2, 1, 3),
  (2, 3, 1), (3, 1, 2), (3, 2, 1) } $$
we follow that $Z(S_3)$ includes at least $e$. Something tells me that
$Z(S_3)$ consists exclusively of $e$. Let us particularly look at the
"shift" operator $(2, 3, 1)$ (the one that maps one element to the
next with mapping the last element to the first). We follow that any
given permutation group contains one. For $n \geq 3$ we can also find
an element, that is distinct from the shift element, that swaps two
elements in place (we're gonna take $(2, 1, 3)$ in this case). We
follow that
$$(2, 3, 1) \circ (2, 1, 3) = (1, 3, 2)$$
$$(2, 1, 3) \circ (2, 3, 1) = (3, 2, 1)$$
thus we follow that shift operator is not in the $Z(S_3)$.

Generally let $a \in \set{1, ..., n}$, let $q \in S_n$ be an element
such that it maps $a$ to itself, and maps $(a + 1) \mod n$ not to
itself, and let $s \in S_n$ be the shift operator. We follow that $(q
\circ s) (a) = q(s(a)) = q(a + 1)$ and $s(q(a)) = s(a) = a + 1$. By
definition $q(a + 1) \neq a + 1$, and thus we conclude that neither
shift operator, nor the element that has partial identity are in
$Z(S_n)$.

With a given example we can generalize even more to create a proof
for our conjecture:

Let $n \geq 3$ and let $q \in S_n$ be such that $q \neq e$. Since $q
\neq e$ we follow that there's an element $a \in \set{1, ..., n}$ such
that $q(a) \neq a$. We then follow that there exists an element $s \in
S_n$ such that $s(a) = a$, and $s(q(a)) \neq q(a)$ (we can do this
because there are at least 3 elements in $\set{1, ..., n}$: we take
the $q(a)$ and some element $b \in \set{1, ..., n} \setminus
\set{q(a), a}$, then map $a$ to itself, and swap $q(a)$ and $b$; the
fact that there's an element that is distinct from both $q(a)$ and $a$
is the restriction that fails for $S_2$). We follow that
$$(s \circ q) (a) = s(q(a)) \neq q(a) = q(s(a)) = (q \circ s) (a)$$
and thus $s \circ q \neq q \circ s$.  Therefore we conclude that $q
\notin Z(S_n)$. Since the only restriction on $q$ is that $q \in S_n
\setminus \set{e}$ we follow that $Z(S_n) = \set{e}$ for all $n \geq
3$, as desired.

\textit{(d) Compute the center of the dihedral group $D_n$.}

We follow that
$$s * r \neq r * s = s * r^{n - 1}$$
thus neither $r$ nor $s$ are in $Z(D_n)$ for $n \geq 3$. We follow
that
$$sr^j * r = sr^{j + 1}$$
$$rsr^j = s r\inv r^j = sr^{j - 1}$$
thus for $n \geq 3$ we follow that $sr^{j + 1} \neq sr^{j - 1}$, and
thus for all $j \in \set{1, ..., n - 1}$ we follow that $sr^j \notin
Z(D_n)$.
We also follow that
$$r^j * s = sr^{n - j}$$
$$s * r^{j} = sr^{j}$$
thus if $j \neq n/2$ we follow that $r^j \notin Z(D_n)$. Thus we
follow that if $n$ is odd that $Z(D_n) = \set{e}$. We follow that
$$r^{n/2} * s r^k = sr^{n - n/2 + k} = sr^{n/2 + k}$$
$$s r^k * r^{n/2} = sr^{n/2 + k}$$
for all $k \in \set{0, ..., n}$. Case with $r^k$ for arbitrary $k$ is
trivial, and thus we conclude that if $n$ is even that $Z(D_n) =
\set{e, r^{n/2}}$, which is kinda surprising.

\textit{(e) Compute the center of the quaternion group $Q$.}

We follow that none of the $i, j, k$ are commuting with each other (by
definition), nor do their inverses, and thus none of them are in
$Z(Q)$. $\pm 1$ on the other hand commute with everything (once again,
by definition), and thus $Z(Q) = \set{\pm 1}$.

\subsection{}

\textit{Let $G$ be a group, and let $g \in G$. The centralizer of $g$,
  denoted $Z_G(g)$ is the set of elements of $G$ that commute with $g$,
  i.e.
  $$Z_G(g) = \set{g' \in G: gg' = g'g}$$
}

\textit{(a) Prove that $Z_G(g)$ is a subgroup of $G$.}

The proof is pretty much the same as with centralizer of the entire
group. We can also note that the definition implies that $Z(G)$ is a
subgroup of $Z_G(g)$ for any given $g \in G$. Also important to note
that $g \in Z_G(g)$.

\textit{(b) Compue the centralizer $Z_G(g)$ for the following groups:}

\textit{(i) $G = D_4$ and $g = r$}

$4$ is even, and thus $\set{e, r^2}$ are in $Z_G(g)$. Any rotation is
also there.  $sr \neq sr^3$, and thus $s$ is not included there. $sr^j
r = sr^j \neq sr{j - 1} = r sr^j$, which implies that none of the
$sr^j$ are in there

\textit{(ii) $G = D_4$ and $g = sr^j$}

$r$ and $r^3$ aren't in there. If $j = 2$, then $s$ is also in there,
otherwise it isn't.  If $j = 2$, then $sr sr^2 = r^{3}$ and $sr^2 sr =
r$, thus $sr$ is not in there.  $sr^3$ is the inverse of $sr$, and
thus it's not in there. The rest can be handled pretty easily.

\textit{(iii) $G = GL_2(R)$ and $g = \
  \begin{pmatrix}
    a & 0 \\
    0 & d
  \end{pmatrix}
  $
}

We follow that this matrix is diagnogal, and thus it commutes with other
diagonal matrices. If matrix is not diagonal then
$$
  \begin{pmatrix}
    a & 0 \\
    0 & d
  \end{pmatrix}
  \begin{pmatrix}
    q & w \\
    e & r
  \end{pmatrix} =
  \begin{pmatrix}
    qa & wa \\
    ed & rd
  \end{pmatrix}
  $$
  $$
  \begin{pmatrix}
    q & w \\
    e & r
  \end{pmatrix}
  \begin{pmatrix}
    a & 0 \\
    0 & d
  \end{pmatrix} =
  \begin{pmatrix}
    qa & wd \\
    ea & rd
  \end{pmatrix}
  $$
  thus we need $wa = wd$ and $ea = ed$. Therefore if $a \neq d$ we
need $w = e = 0$ and this would include only include other diagonal
matrices, that we've already included. If $a = d$, then any matrix
will do.

\textit{(c) Prove that $Z_G(g) = G$ if and only if $g \in Z(G)$.}

IF $g \in Z(G)$, then by definition every element of $G$ commutes
with $G$, and thus $Z_G(g) = G$.

Assume that $Z_G(g) = G$. We follow that every element of $G$ commutes
with $g$, and thus $g \in Z(G)$.

\textit{(d) More generally, if $S \subseteq G$ is any subset of $G$,
then the centralizer of $S$ is the set
  $$Z_G(S) = \set{g \in G: (\forall s \in S)(sg = gs)}$$
  Prove that$ Z_G(S)$ is a subgroup of $G$ }

Proof is borderline identical to the case of centralizer of
group/element.

\subsection{}

\textit{This exercise explains when two elements of $G$ determine the
  same coset of $H$. Let $G$ be a group, let $H$ be a subgroup of $G$,
  and let $g_1, g_2 \in G$. Prove that the following three statements
  are equivalent:}

\textit{(1) $g_1 H = g_2 H$. }

\textit{(2) There are $h \in H$ such that $g_1 = g_2 h$. }

\textit{(3) $g_2\inv g_1 \in H$. }

Assume that $g_1 H = g_2 H$. We follow that $e \in H$ and thus
$g_1 e \in g_2 H$. Thus there's $h \in H$ such that $g_1 e = g_1 = g_2 h$
by definition. Thus
$$g_1 = g_2 h$$
$$g_2 \inv g_1 = h \in H$$
We can run this in reverse to establish equivalence of (2) and (3).
We follow that since $g_1 = g_2 h$ that $g_1 H \cap g_2 H \neq \emptyset$,
and thus they are equal.

\subsection{}

\textit{Let $G$ be a finite group whose only subgroups are $\set{e}$
and $G$.  Prove that either $G = \set{e}$ or else that $G$ is a cyclic
group whose order is prime.}

Set with $G = \set{e}$ is trivial, thus assume that $G \neq \set{e}$.
Let $q \in G \neq e$. We follow that since $G$ is finite that $q$ has
an order, and thus its powers constitute a subgroup
$\eangle{q}$. Since order of $G$ is a prime we follow that order of
$\eangle{q}$ divides the order of $G$ and thus order of  $\eangle{q}$ is
either $1$ or order of $G$. Since $q \neq e$ we conclude the latter,
and thus $\eangle{q} = G$, which implies that $G$ is cyclical, as desired.

\subsection{}

\textit{Let $G$ be a group, and let $K \subseteq H \subseteq G$ be
subgroups. We may thus view $K$ as a subgroup of $G$ or as a subgroup
of $H$. We also recall that the index of a subgroup is its numbder of
distinct cosets. }

\textit{(a) If $G$ is finite, prove the Index Multiplication Rule:
  $$(G:K) = (G:H) (H:K)$$
}

We follow that
$$|H| = |K| (H:K)$$
$$|G| = |H| (G:H)$$
and
$$|G| = |K| (G:K)$$
therefore
$$|G| / |K| =  (G:K)$$
thus
$$|G| = |K| (H:K) (G:H)$$
$$|G| / |K| =  (H:K) (G:H)$$
$$|G| / |K| =  (G:H) (H:K) $$
$$(G:K) =  (G:H) (H:K) $$
as desired.

\textit{(b) Prove that the Index Multiplication Rule is true even if
  $G, H, K$ are allowed to be infinite groups, provided that we assume
  that $(G:K)$ is finite}

Let $Q'$ be the set of distinct cosets of $H$ in $G$.  Let $W'$ be the
set of distinct cosets of $K$ in $H$.  For each $q' \in Q'$ there's an
element $q$ of $G$ such that $qH = q'$. By AoC we can define subset
$Q$ of $G$ that correspond to $Q'$ For each $w' \in W'$ there's an
element $w$ of $H$ such that $wK = w'$. By AoC we can define subset
$W$ of $H$ that correspond to $W'$

Let $j$ be a coset of $K$ in $G$. We follow that there's $g \in G$
such that $gK = j$. We follow that $gH$ is a coset of $H$, and thus
there's a unique $q \in Q$ such that $qH = gH$. We follow then that
$g\inv q \in H$ (see a couple of exercises above), and thus $(q\inv g)
K$ is a coset of $K$ in $H$. Thus there's a unique $w \in W$ such that
$$w K = (q\inv g ) K$$
thus
$$q w K = gK$$
therefore for each coset of $K$ in $G$ there's a pair $\eangle{q, w}
\in Q \times W$. By uniqueness of $q$ and $w$ in $Q$ and $W$,
and the fact that $j$ is an arbitrary coset we
conclude that there's an injection from a set
of cosets of $K$ in $G$ to $Q \times W$.

Let $\eangle{q_1, w_1}, \eangle{q_2, w_2} \in Q \times W$ are such that
$\eangle{q_1, w_1} \neq \eangle{q_2, w_2}$. We follow that
if $q_1 \neq q_2$, and thus $q_1 H \neq q_2 H$ by definition.
$w_1 w_2 \in H$, and thus $w_1 w_2 H = H$. Since $K \subseteq H$ we follow that
$$q_1 w_1 K \subseteq q_1 H$$
$$q_2 w_2 K \subseteq q_2 H$$
and thus
$$q_1 w_1 K \neq q_2 w_2 K$$
If $q_1 = q_2$ we follow that $w_1 \neq w_2$. This implies that
$$w_1 K \neq w_2 K$$
and thus
$$q_1 w_1 K \neq q_1 w_2 K$$
and since $q_1 = q_2$ we follow that
$$q_1 w_1 K \neq q_2 w_2 K$$
This implies that if $\eangle{q_1, w_1}, \eangle{q_2, w_2} \in Q \times W$ then
$q_1 w_1 K \neq q_2 w_2 K$, and thus we conclude that there's an injection from
$Q \times W$ to a set of cosets of $K$. Shroder-Bernstein now implies that
the set of cosets of $K$ in $G$ is equal to $|Q \times W|$. If the latter is finite,
then we follow that $(G:K) = |Q| * |W|$. Definition of $Q$ and $W$ respectively
imply that $|Q| = (G:H)$ and $|W| = (H:K)$, which implies that
$$(G:K) = (G:H) (H:K)$$
as desired.

\subsection{}

\textit{Let $p$ be a prime and let $a \in Z$ be an inteeger satisfying
  $p \not | a$. Use Exercise 2.13(b) and Lagrange's theorem to prove}

\textit{\textbf{Fermat's Little Theorem}: $a^{p - 1} \equiv 1 (\mod p)$}

We follow that if $a \not | p$, then $a \not \equiv 0 \mod p$. We thus follow that
there's $r \in Z_+$ such that $r < p$ and there's an integer $k$ such that
$a = kp + r$. This implies that
$$a^{n} = (kp + r)^n = \sum_{i \in \set{0, ...,  n}}{bc(n, i)(kp)^{i} r^{n - i}}$$
by the binomial theorem ($bc$ is the binomial coefficient). We then
follow that the only element of the sum that does not have $p$ as its
multiple is the one where $i = 0$. Thus we follow that
$$a^{n} \equiv  \sum_{i \in \set{0, ...,  n}}{bc(n, i)(kp)^{i} r^{n - i}} \equiv r^n \mod p$$
thus we follow that the original theorem is equivalent to the
statement that for all $r \in Z_+$ such that $r < p$ we've got that
$$r^{p - 1} \equiv 1 \mod p$$

We know that $1$ is the identity in the multiplicative group
$(Z/pZ)^*$ We also follow that order of $(Z/pZ)^*$ is $p - 1$. We thus
follow that any cyclic subgroup of $(Z/pZ)^*$ has order that divides
$p - 1$. Therefore we follow that for each provided $r$ we've got that
order of $r$ is some divisor $n$ of $p - 1$, and thus
$$r^{p - 1} = r^{n * k} = e^k = e = 1$$
which gives us the desired result.

\section{Products of groups}

\subsection{}

\textit{Let $G_1, G_2, ..., G_n$ be groups. Generalize definition 2.54
  by explaining why
  $$G_1 \times ... \times G_n$$
  with component-wise operation is a group.}

We can follow this case by induction. Interesting thing is to ask that
if $K$ is an indexed set of groups, then is $\prod{K}$ a group?

We can follow that the product of identities is an identity. Inverses
are also handled.  Component-wise association gives us also
association for the product group. $\prod{K}$ is abelian if and only
if all the constructing groups are abelian.

\subsection{}

\textit{Let $G$ be a group, let $A$ and $B$ be subgroups of $G$ and
  consider the map
  $$\phi: A \times B \to G, \phi(a, b) = ab$$}

\textit{(a) If $G$ is an abelian group, prove that $\phi$ is a
  homomorphism.}

Follows from the exercise that we've already finished (there we've handled
the specific case that $B \subseteq Z_G(A)$)

\textit{(b) If $G$ is an abelian group, prove that
  $$ker(\phi) = \set{\eangle{c, c\inv}: c \in A \cap B}$$}

We can follow that application of $\phi$ to the element of $rhs$ will give us
identity by definition, and thus we've got $\supseteq$.

If $\phi(i, j) = e$, then by definition $i j = e$, and thus $i = j\inv$. Thus
we've got the $\subseteq$.

\textit{(c) Suppose that there are elements $a \in A$, $b \in B$ such
thaht $ab \neq ba$.  Prove that $\phi$ is not a homeomorphism}

Exercise that handled this premise before concludes with
biconditional, which gives us this implication

\textit{the rest of the exercises are skipped}

\chapter{Rings - Part 1}

\section{Abstract Rings and Ring Homomorphisms}

\subsection*{Notes}

Distributive law can be generalized through application of
associativity, which will give us
$$a * (b + c + d) = ab + ac + ad$$
and the same for the other direction. We can also by induction
do the same with a general sum.
$$b \sum{a_i} = \sum{b a_i}$$

The idea of ring homomorphism can be summed up to preserving
functions upon application of the function, and mapping identities
to identities.

\subsection{}

\textit{Let $R$ be a ring, and let $a, b \in R$.}

\textit{(a) Prove that
  $$(-a) * (-b) = a * b$$}

$$(-a) * (-b) = (0 + (- a)) * (-b) = (a + (-a) + (-a)) * (-b) = a * (-b) + (-a) * (-b) + (-a) * (-b)$$
(we've added 0 to the $-a$ and then distibuted over it) thus
$$(-a) * (-b) = a * (-b) + (-a) * (-b) + (-a) * (-b)$$
and cancelation law implies that
$$0 = a * (-b) + (-a) * (-b)$$
this in turn implies that
$$(- (a * (-b))) =  (-a) * (-b)$$
we can do a similar thing for $(-b)$, which will result in
$$(- ((-a) * b)) =  (-a) * (-b)$$
Given that everything has additive inverse, we can follow that
$$a + b = (-((-a) * b)) = (- (a * (-b)))$$
which will imply that
$$(-a) * (-b) = (- (a * (-b))) = (- ((-(-a)) * (-b))) = (- ((-(-a)) * (-b))) = (- (- (-(-a)) * b)) = $$
$$ =  (- (- (a * b))) = a * b$$
as desired.

\textit{(b) Our proof of Proposition 3.2(a) used the multiplicative
  identity element $1_R$ of $R$. Find a proof that works even if $R$ does not
  have a multiplicative identity}

We follow that for all $b \in R$
$$ab = a * (b + 0) = ab + a0$$
thus by cancelation law of the group we follow that
$$-(ab) + ab   = -(ab) + ab   + a0$$
$$0 = 0 + a0$$
$$0 = a0$$
as desired.

\subsection{}

\textit{Let $R$ be a commutative ring}

\textit{(a) Suppose that the map
  $$f: R \to R, f(a) = a^2$$
  is a ring homomorphism. Prove that $1_R + 1_R = 0_R$.}

By definition of ring homeomorphism we follow that
$$1_R = f(1_R)$$
and thus we follow that
$$1_R + 1_R = f(1_R) + f(1_R) = f(1_R + 1_R) = (1_R + 1_R)^2 = (1_R + 1_R) (1_R + 1_R) = $$
$$ =  1_R(1_R + 1_R) + 1_R(1_R + 1_R) = (1_R + 1_R) + (1_R + 1_R)$$
thus
$$(1_R + 1_R) = (1_R + 1_R) + (1_R + 1_R)$$
$+$ is a group, thus we're justified to use cancelation to get that
$$1_R + 1_R = 0$$
as desired


\textit{(b) Conversely, if $2 = 0$ in the ring $R$, prove that
  $f(a)^2$ is a homomorphism from $R$ to $R$}

We follow that
$$0^2 = 0$$
$$1^2 = 1$$
so identities are there.

One thing to point out is that $R$ is commutative, which is used in
deriviations here
$$f(a + b) = (a + b)^2 = (a + b)(a + b) = a(a + b) + b(a + b) = $$
$$ = a^2 + ab + ab + b^2 = f(a) + ab + ab + f(b) = f(a) + f(b) + ab(1 + 1) = $$
$$ = f(a) + f(b) +  * 0 = f(a) + f(b)$$
thus we've got the addition
$$f(a * b) = (a * b)^2 = (a * b) * (a * b) = a * b * a * b = a^2 * b^2 = f(a) * f(b)$$
and multiplication, that now satisfies the definition of the
homomorphism.

\textit{(c) Suppose that the map
  $$f: R \to R, f(a) = a^3$$
  is a ring homomorphism. Prove that $6 = 0$ in the ring $R$}

$$ f(1 + 1) = (1 + 1)^3 = (1 + 1) * (1 + 1) * (1 + 1) = 8$$
$$ f(1 + 1) = f(1) + f(1) = 1 + 1 = 2$$
thus
$$8 = 2$$
$$6 = 0$$
as desired. In general we follow that if $f(a) = a^n$ is a
homomorphism, then $2^n - 2 = 0$

\section{Interesting Examples of Rings}

\subsection*{Notes}

When checking for subrings, we follow that associativity of both operations
and their distributivity gets grandfathered by the subset, and thus
the only things that we need to check for are the identities and closures
of operations 

When we're using generalized rings with polynomials, it is possible
that different lists of coefficients might give the same functions.
This is easily observed, when we remember that there exist finite
rings, and thus the space of functions from this ring to itself is
finite, but the list of possible coefficients is infinite.

When we're talking about polynomial equality, we mean that they've got
the same list of coefficients. This essentially means that a
polynomial is practically a finite list of elements of the ring (and
thus the set of all polynomials is a set of all finite lists of
coefficients), and not the function.

It's also important to remember, that coefficients of polynomials are
the elements of the underlying ring, but powers are natural
numbers by definition.

We can also theoretically have polynomials over polynomials, and so on,
and so forth.

In the book, the ring of quaternions is defined over coefficients in
$R$, but the subrings of $R$ can be also used to define appropriate
subrings.

Matrix ring is an important example of the fact that two non-zero
elements when multiplied can produce a zero.

Although we haven't observed any product rings (i.e. rings that are
made up by sticking two rings into cartesian product), it's pretty
obvious that we can define them and everything will line up

We can rephrase the condition of uniqueness of homomorphism from $ZZ$:
if $f$ and $g$ are homomorphims from $Z$ to some ring $R$, then $f =
g$.


\subsection{}

\textit{Let $m \geq 1$ be an integer, and define a map
  $$\phi: Z \to Z/ZmZ, \phi(a) = a \mod m$$
  Prove that $\phi$ is a ring homomorphism.}

We firstly follow that $\phi(0) = 0, \phi(1) = 1$ by definition, and
thus identities are mapped properly. Fact that
$$a \equiv c \mod m, b \equiv d \mod m \ra a + b \equiv c + d \mod m, a * b \equiv c * d \mod m$$
implies that all the function properties are satisfied, and thus $\phi$ is a
homeomorphism, as desired.

\subsection{}

\textit{(a) Let $\alpha = 7$ and $\beta = 11$ be elements of the ring
  $Z/17Z$. Compute $\alpha + \beta$ and $\alpha * \beta$.}

We follow that
$$7 + 11 = 18 \equiv 1 \mod 17$$
$$7 * 11 = 77 \equiv 9 \mod 17$$

\textit{(b) Let $\alpha = 2 + 4x$ and $\beta = 1 + 4x + 3x^2$ be
  elements of the ring $(Z/7Z)[x]$. Compute $\alpha + \beta$ and $\alpha *
  \beta$.}

$$(2 + 4x) + (1 + 4x + 3x^2) = 3 + x + 3x^2$$
$$(2 + 4x) * (1 + 4x + 3x^2) = (2 + 1x + 6x^2) + (4x + 2x^2 + 5x^3) = 2 + 5x + x^2 + 5x^3$$

\textit{(c) Let $\alpha = 3 + 2i$ and $\beta = 2 - 3i$ be
  elements of the ring $Z[i]$. Compute $\alpha + \beta$ and $\alpha *
  \beta$.}

$$3 + 2i + 2 - 3i = 5 - i$$
$$(3 + 2i) * (2 - 3i) = 6 - 9i + 4i + 6 = 12 - 5i$$

\textit{(d) Let $\alpha = 3 + 2x - x^2$ and $\beta = 2 - 3x + x^2$ be
  elements of the ring $Z[x]$. Compute $\alpha + \beta$ and $\alpha *
  \beta$.}

$$(3 + 2x - x^2) + (2 - 3x + x^2) = 5 - x$$
$$(3 + 2x - x^2) * (2 - 3x + x^2) =
(6 - 9x + 3x^2) + (4x - 6x^2 + 2x^3) + (-2x^2 + 3x^3 - x^4) = $$
$$= 6 - 15x - 5x^2 + 5x^3 - x^4$$

\textit{(e) Let $R = Z[i]$ and let $\alpha = (1 + i) + (2 - i)x - x^2$
  and $\beta = (2 + i) + (1 + 3i) x$ be elements of the ring
  $Z[x]$. Compute $\alpha + \beta$ and $\alpha * \beta$.}

$$(1 + i) + (2 - i)x - x^2 + (2 + i) + (1 + 3i) x = (2 + 2i) + (3 + 2i)x - x^2$$
$$((1 + i) + (2 - i)x - x^2) * ((2 + i) + (1 + 3i) x) = $$
$$ = 
(2 + 2i + i - 1) + (4 - 2i + 2i + 1)x + (-2 - i)x^2 +
(1 + i + 3i - 3)x  + (2 - i + 6i + 3)x^2 + (-1 - 3i)x^3 = 
$$
$$ =  (1 + 3i) + 5x + (-2 - i)x^2 + (-2 + 4i)x  + (5 + 5i)x^2 + (-1 - 3i)x^3 = $$
$$ =  (1 + 3i) + (3 + 4i)x + (3 + 4i)x^2  + (-1 - 3i)x^3$$

\textit{(f) Let $\alpha = 1 + 2i - j + k$ and $\beta = 2 - i + 3j - k$ be
  elements of the ring $H$. Compute $\alpha + \beta$ and $\alpha *
  \beta$.}

$$1 + 2i - j + k +  2 - i + 3j - k = 3 + i + 2j$$

$$(1 + 2i - j + k) *  (2 - i + 3j - k) = $$
$$ = (2 - i + 3j - k) + 2i(2 - i + 3j - k) - j(2 - i + 3j - k) + k(2 - i + 3j - k) = $$
$$ = (2 - i + 3j - k) + (4i + 2 + 6k + 2j) - (2j + k - 3 - i) + (2k - j - 3i + 1) = $$
$$ = (2 - i + 3j - k) + (2 + 4i + 2j +  6k) + (3 + i - 2j - k) + (1 - 3i - j + 2k) = $$
$$ = 8 + i + 2j + 6k $$

\subsection{}

\textit{For any integer $D$ that is not the square root of an integer,
we can form a ring
  $$Z[\sqrt{D}] = \set{a + b\sqrt{D}: a, b \in Z}$$
  If $D > 0$, then $Z[\sqrt{D}]$ is a subring of $R$, while if $D < 0$,
  then in any case it is a subring of $C$.}

\textit{(a) Let $\alpha = 2 + 3 \sqrt{5}$ and let $\beta = 1 - 2\sqrt{5}$. Compute
  the quntities
  $$a + b, a * b, a^2$$}

We follow that
$$2 + 3 \sqrt{5} + 1 - 2 \sqrt{5} = 3 + \sqrt{5}$$
$$(2 + 3 \sqrt{5}) * (1 - 2 \sqrt{5}) = 2 - 4\sqrt{5} + 3 \sqrt{5} - 30 = -28 - \sqrt{5}$$
$$(2 + 3 \sqrt{5})^2 = 4 + 12 \sqrt{5} + 45 = 49 + 12 \sqrt{5}$$

\textit{(b) Prove that the map
  $$\phi: Z[\sqrt{D}] \to Z[\sqrt{D}], \phi(a + b\sqrt{D}) = a - b \sqrt{D}$$
  is a ring homomorphism. }

We follow that identities are golden.
$$\phi(a + b\sqrt{D} + c + d\sqrt{D}) =  \phi((a + c) + (b + d)\sqrt{D}) = $$
$$ = 
(a + c) - (b + d)\sqrt{D} = a - b\sqrt{D} + c - d\sqrt{D} = \phi(a + b\sqrt{D}) + \phi(c + d\sqrt{D})$$

$$\phi((a + b\sqrt{D}) * (c + d\sqrt{D})) = \phi(ac + ad\sqrt{D} + bc\sqrt{D} + bdD) =
ac + bdD - ad\sqrt{D} - bc\sqrt{D}$$
$$\phi(a + b\sqrt{D}) * \phi(c + d\sqrt{D}) = (a - b\sqrt{D}) * (c - d \sqrt{D}) = $$
$$ ac  + bdD  - ad\sqrt{D} - bc\sqrt{D}$$
thus everything checks out, as desired.

\textit{(c) Prove that for all $\alpha \in Z[\sqrt{D}]$
  $$\alpha *  \overline{\alpha} \in Z$$}
$$\alpha *  \overline{\alpha} = (a + b\sqrt{D}) * (a - b\sqrt{D}) = $$
$$ = a^2 - ab\sqrt{D} + ab\sqrt{D} + b^2D = a^2 + b^2D$$
given that $a, b, D$ are integers, we follow the desired conclusion.

\subsection{}

\textit{Let $\rho$ be the complex number $\rho = \frac{-1 + i\sqrt{3}}{2} \in C$, and let
  $$Z[\rho] = \set{a + b\rho: a, b \in Z}$$}

\textit{(a) Prove tht $\rho^3 = 1$. Thus $\rho$ is a cube root of unity.}

$$\left(\frac{-1 + i\sqrt{3}}{2}\right)^3 =
\left(\frac{\sqrt{3}}{2} i - \frac{1}{2}\right)^3 =
\left(\frac{\sqrt{3}}{2} i \right)^3 -
3 \left(\frac{\sqrt{3}}{2} i \right)^2 \left(\frac{1}{2} \right) +
3 \left(\frac{\sqrt{3}}{2} i \right) \left(\frac{1}{2}\right)^2 -
\left(\frac{1}{2}\right)^3 = $$
$$ = - \frac{3 \sqrt{3}}{8} i  + \frac{9}{8} + \frac{3 \sqrt{3} i}{8}i  - \frac{1}{8} =
1 + 0 = 1$$
as desired.

\textit{(b) Prove that $\rho^2 + \rho + 1 = 0$}

$$\rho^3 - 1 = (\rho - 1)(\rho^2 + \rho + 1)$$
we follow that lhs is not zero, and thus lhs is zero, as desired.

\textit{(c) Prove that the polynomial $X^3 - 1$ factors as
  $$X^3 - 1 = (X - 1)(X - \rho)(X - \rho^2)$$}


We follow that
$$(X - \rho) (X - \rho^2) = X^2 - \rho^2 X - \rho X + 1 = X^2 - \rho^2 X - \rho X - X + X + 1 =$$
$$ = X^2 - X(\rho^2 X+ \rho - 1) + X + 1 = = X^2 - 0X + X + 1 = X^2 + X + 1$$
thus
$$(X - 1) (X - \rho) (X - \rho^2)  = (X - 1) (X^2 + X + 1)$$
and
$$X^3 - 1 = (X - 1)(X^2 + X + 1)$$
as desired.

\textit{(d) Prove that $Z[\rho] $ is a subring of $C$.}

By setting $a, b$ appropriately we get that identites are present. Additive group is
closed by general properties of $C$.

$$(a + b \rho) * (c + d \rho) = ac + ad \rho + bc \rho + bd \rho^2 =
ac + ad \rho + bc \rho + bd \rho^2 + bd \rho + bd - bd\rho - bd = $$
$$ = ac + ad \rho + bc \rho + bd (\rho^2 + \rho + 1) - bd\rho - bd =
ac + ad \rho + bc \rho + bd 0 - bd\rho - bd =$$
$$ = (ac - bd) + \rho(ad + bc - bd)$$
thus this thing is also closed under multiplication, which makes $Z[\rho]$ a
subring, as desired.

\subsection{}

\textit{This one is reasoned in the notes, and is a bunch of arithmetic}

\subsection{}

\textit{Let $R$ be commutative ring, let $c \in R$, and let $E_c: R[x]
  \to R$ be evaluation map $E_c(f) = f(c)$.}

\textit{(a) Prove that $E_c$ is a ring homomorphism.}

We follow that there's a polynomial $x - c$ and $x - c + 1$ that maps
identities.
We follow that
$$E_c(f + g) = (f + g)(c) = f(c) + g(c) = E_c(f) + E_c(g)$$
$$E_c(f * g) = (f * g)(c) = f(c) * g(c) = E_c(f) * E_c(g)$$
by algebraic properties of functions.

\textit{(b) Prove that $E_c(f) = 0$ if ans only if there's a
  polynomial $g(x) \in R[x]$ satisfying $f(x) = (x - c)g(x)$}

$$f(x) = f(x) - 0 = f(x) - f(c) = \sum{a_i (x^j - c^j)}$$
We know that
$$x^n - c^n = (x - c)\sum_{i = 1}^n{x^{n - i} c^{i - 1}}$$
Which gives us the desired result (needed to look this one up).
Reverse is trivial

\subsection{}

\textit{Let $R$ be a non-commutative ring, with elements $\alpha,
  \beta \in R$ such that $\alpha \neq \beta$. We can still turn $R[x]$
  into a ring by always writing polynomials in the form
  $$p(x) = \sum{a_i x^i}$$
  and using the usual rule that $x$ commutes with all of the elements
  of $R$.  We can also define an evaluation map $E_\alpha: R[x] \to R$
  by setting
  $$E\alpha(\sum{a_i x^i}) = \sum{a_i \alpha^i}$$
  Prove that if $R$ is non-commutative, then $E_\alpha$ is not a
  homomorphism.}

We follow that
$$E\alpha(0) = 0$$
$$E\alpha(1) = 1$$
thus we've got identities, and so there's nothing there.

We know that both $\alpha$ and $\beta$ are polynomials, but
$$E_\alpha(x * \beta) = E_\alpha(\beta x) = \beta \alpha$$
$$E_\alpha(x) * E_\alpha(\beta) = \alpha * \beta$$
which implies that $E_\alpha(x) * E_\alpha(\beta) \neq E_\alpha(x *
\beta)$, which in turn implies that $E_\alpha$ is not a homomorphism,
as desired.

\subsection{}

\textit{Skip}

\subsection{}

\textit{For any commutative ring $R$, let
  $$M_2(R) = \set{
    \begin{pmatrix}
      a & d \\ c & d
    \end{pmatrix}: a, b, c, d \in R
  }$$
  be the set of 2-by-2 matrices with entries in $R$. Define addition
  and multiplication in an obvious way }

\textit{(a) Prove that $M_2(R)$ is a ring.}

Additive identity is the matrix with zeros in it. Multiplicative
identity is the one with multiplicative identites on the diagonal, and
the rest being zeroes.

Properties that this thing is an abelian group
under addition are following from the fact that $R$ is an abelian
group.

Closure under multiplication is derived from the definition of
multiplication.  Detiviations of associativity and distributivity take
a lok of paper, but are pretty uneventful.

\textit{(b) Prove that $M_2(R)$ is non-commutative}

This is where our assumtions that $0 \neq 1$ are kicking in: if
a trivial ring is a singleton, then this thing is commutative.
But we don't assume that, and thus
$$
\begin{pmatrix}
  0 & 1 \\
  1 & 0
\end{pmatrix}
\begin{pmatrix}
  1 & 1 \\
  1 & 0
\end{pmatrix}
=
\begin{pmatrix}
  1 & 0 \\
  1 & 1
\end{pmatrix}
$$
$$
\begin{pmatrix}
  1 & 1 \\
  1 & 0
\end{pmatrix}
\begin{pmatrix}
  0 & 1 \\
  1 & 0
\end{pmatrix}
=
\begin{pmatrix}
  1 & 1 \\
  0 & 1
\end{pmatrix}
$$
which given that $1 \neq 0$ implies the desired result.

\textit{(c) Find non-zero elements $A, B \in M_2(R)$ such that $AB = 0$}
$$
\begin{pmatrix}
  1 & 0 \\
  0 & 0
\end{pmatrix}
\begin{pmatrix}
  0 & 0 \\
  0 & 1
\end{pmatrix} = 0
$$
as desired.

\textit{(d) Let $A \in M_2(R)$. Prove that there's $B \in M_2(R)$
satisfying $AB = I$ if and only if $ad - bc$ has a multiplicative
inverse in $R$.}

Let's start with the forward direction, and assume the appropriate
things. We're gonna implicitly derive the determinant here. We follow that
$$
\begin{pmatrix}
  a & b \\
  c & d
\end{pmatrix}
\begin{pmatrix}
  p & q \\
  r & s
\end{pmatrix} =
\begin{pmatrix}
  pa + rb & qa + sb \\
  pc + rd & qc + sd  
\end{pmatrix} = I
$$
thus we follow that
$$
\begin{cases}
  pa + rb = qc + sd = 1 \\
  qa + sb = pc + rd = 0 \\
\end{cases}
$$
thus

$$
\begin{cases}
  (pa + rb) * (qc + sd) = 1  \\
  (qa + sb) * (pc + rd) = 0
\end{cases}
$$
$$
\begin{cases}
  paqc + pasd + rbqc + rbsd = 1 \\
  qapc + qard + sbpc + sbrd = 0
\end{cases}
$$
$$
\begin{cases}
  paqc + pasd + rbqc + rbsd = 1 \\
  paqc + qard + sbpc + rbsd = 0
\end{cases}
$$
thus
$$ paqc + pasd + rbqc + rbsd -   (paqc + qard + sbpc + rbsd) = 1 $$
$$ pasd + rbqc - (qard + sbpc) = 1 $$
$$ psad + rqbc - rqad - psbc = 1 $$
$$ ps(ad - bc) + rq(bc - ad) = 1 $$
$$ ps(ad - bc) - rq(ad - bc) = 1 $$
$$ (ps - rq) (ad - bc) = 1 $$
which implies that there's $ps - rq \in R$ that is a multiplicative
inverse of $(ad - bc)$, as desired.

Assume that $ad - bc$ has a multiplicative inverse. The following is
just a simple formula for computing inverse of the matrix in $R^2$. We
follow that
$$
\begin{pmatrix}
  a & b \\
  c & d
\end{pmatrix}
\begin{pmatrix}
  d & -b \\
  -c & a
\end{pmatrix} =
\begin{pmatrix}
  da - cb & -ba + ba \\
  -dc + cd & -bc + ad 
\end{pmatrix} =
\begin{pmatrix}
  ad - bc & 0 \\
  0 & ad - bc 
\end{pmatrix}
$$
We then follow that if we would've changed every entry $e$ in the
right matrix for $e / (ad - bc)$, then we would've gotten identity as
the result. Thus we conclude that there's a matrix $B$ such that
$AB = I$, as desired

\textit{(e) Let $A, B \in M_2(R)$. Prove that $AB = I$ if and onyl if $BA = I$}

Given previous point and lots of notation, we can conclude the desired result.

\textit{(f) More generally, prove all of the above for $M_n(R)$ for $n \geq 2$.}

I'm not gonna prove this thing rigorously, but I'll outline a proof
because of the fact that most of it is tedious and not particualarly
interesting.

When it comes to a ring, we get grop properties for free, closure of
multiplication is somewhat trivial. Districutivity can be handled piecewise:
for $(A + B) * C$ let $q_{i, j}$ be some entry in the resulting matrix 
$$q_{i, j} = \sum_{k = 1}^{n}{(a_{i, k} + b_{i, k}) c_{k, j} } =
\sum{c_{k, j}a_{i, k}} + \sum{c_{k, j}b_{i, k}}$$
and through lots of notation we can follow the rest pretty easily,
thus making $M_n(R)$ a ring.

By putting our original matrix for $M_2(R)$ into the upper left corner
of the matrix we get non-commutativity.

Proof that $A A^{adj} = (\det A) I$ is heavily dependent on what we
choose to be the definition of the determinant. If we start with the
Laplace's formula as the definition of the determinant, we can follow
that matrices with identical rows/columns are zero, and thus
everything that is not on diagonal is zero. We would then follow that
diagonals are determinants by definition, and thus we get desired result.

$A A^{adj} = (\det A) I$ implies that if $(\det A)$ has a
multiplicative inverse in $R$, then we would get that $(\det
A)A^{adj}$ is an inverse of $A$ (we would need to weave through or
define scalar multiplication for matrices), which gives the desired
result. Proof that invertible $A$ commutes with $A\inv$ can be shown
piecewise.

One important point to note is that here we don't get any benefits for
going through determinant to define such things. We would get
practically the same results, if we would've gone through the usual
definitions of linear transformations and whatnot.

\subsection{}

\textit{Let $R_1$ and $R_2$ be commutative rings, and let $\phi: R_1
  \to R_2$ be homomorphism. We define a map $\Phi: R_1[x] \to R_2[x]$ by
  $$\Phi(\sum{a_i x^i}) = \sum {\phi(a_i) x^i}$$
}

\textit{(a) Prove that $\Phi$ is a ring homomorphism.}

We get identities by the fact that $\phi$ is a homomorphism. We follow that

$$\Phi(\sum{a_i x^i}) + \Phi(\sum{b_i x^i}) = \sum{\phi(a_i) x^i} + \sum{\phi(b_i) x^i} = $$
$$ =  \sum{\phi(a_i) x^i + \phi(b_i) x^i}= \sum{(\phi(a_i) + \phi(b_i)) x^i} = $$
$$ =  \sum{(\phi(a_i + b_i)) x^i}= \sum{(\phi(a_i + b_i) x^i}= \Phi(\sum{(a_i + b_i) x^i}) =
\Phi(\sum{a_i x^i} + \sum{b_i x^i} )$$

$$ \Phi(\sum{a_i x^i}) + \Phi(\sum{b_i x^i}) =
(\sum{\phi(a_i) x^i}) * ( \sum{\phi(b_i) x^i}) = \sum{\sum{\phi(a_i) \phi(b_j) x^{i + j}}} = $$
$$ = \sum{\sum{\phi(a_i b_j) x^{i + j}}} = \Phi(\sum{\sum{a_i b_j x^{i + j}}}) =
\Phi((\sum{a_i x^i}) * (\sum{b_i x^i}))$$
as desired.

\textit{(b) If $\phi$ is a ring isomorphism, prove that $\Phi$ is a
ring isomorphism}

$R_1[x]$ is a polynomial, which by definition is the list. There's a
general theorem (if not proven explicitly, then derived pretty
easily), that piecewise application of a bijection on elements of a
given list gives us a bijection. This implies that if $\phi$ is a
bijection, then $\Phi$ is a bijection, which given homomorphism
implies isomorphism, as desired.

\subsection{}

\textit{Lots of notation, but the essense boils down to the fact that
we can define $R[x. y]$ - a set of polynomials with two variables. All
the things that concern evaluation and whatnot are essentially the
same in reagards to this version. We can also define such a thing for
a polynomial in an arbitrary number of variables.}

\subsection{}

\textit{Let $R[x, y]$ be the ring of polynomials in two variables with
coefficients in R, as described in Exercise 3.13. In this exercise we
will look at polynomials that don’t change if we swap x and y. For
example, the polynomials
$$x + y, xy, x^2 + y^2$$
are invariant under $x \lra y$ swap. We observe that our third example
can be expresset using the first two examples,
$$x^2 + y^2 = (x + y)^2 - 2xy$$
In other words, if we let $g_2(u, v) = u^2 - 2v$, then $x^2 + y^2 = g_2(x + y, xy)$}

\textit{(a) Do the same for $x^3 + y^3$ and $x^4 + y^4$}

We follow that generally speaking there's a binomial expansion
$$(x + y)^n = \sum_{i = 0}^n {bc(i, n)x^iy^{n - i}}$$
from which we follow that
$$(x + y)^n = \sum_{i = 1}^{n - 1} {bc(i, n)x^iy^{n - i}} + x^n + y^n$$
which implies that
$$ x^n + y^n = (x + y)^n -  \sum_{i = 1}^{n - 1} {bc(i, n)x^iy^{n - i}} =
(x + y)^n -  xy \sum_{i = 1}^{n - 1} {bc(i, n)x^{i - 1}y^{n - i - 1}} = $$
$$ = (x + y)^n -  xy \sum_{i = 0}^{n - 2} {bc(i, n)x^{i}y^{n - i}} $$

Let us look at the problem at hand:
$$(x + y)^3 = x^3 + 3x^2y + 3xy^2 + y^3 = x^3 + y^3 + 3xy(x + y)$$

We follow that for $n = 3$ we've got that
$$\sum_{i = 1}^{n - 1}{bc(n, i) x^i} = 3x^2y + 3xy^2 = 3xy(x + y)$$
thus $\sum_{i = 1}^{n - 1}{bc(n, i) x^i} = g(x + y, xy)$ for $n = 3$.
Assume that for all $j$ such that $j < n$ we've got that $\sum_{i =
1}^{n - 1}{bc(n, i) x^i} = g(x + y, xy)$
we then follow that for arbitrary $n$
$$ \sum_{i = 1}^{n - 1} {bc(i, n)x^iy^{n - i}} =
xy \sum_{i = 1}^{n - 1} {bc(i, n)x^{i - 1}y^{n - i - 1}} = $$
$$ = xy \sum_{i = 0}^{n - 2} {bc(i, n)x^{i}y^{n - i}}  = xy * g(x + y, xy)$$
thus we follow that we've got the inductive step, and thus for all $n
\geq 3$ we've got the IH. This implies that for all $n \geq 3$
$$x^n + y^n = (x + y)^n -  \sum_{i = 1}^{n - 1} {bc(i, n)x^iy^{n - i}} =
(x + y)^n - g(x + y, xy) = g'(x + y, xy)$$
as desired.

\textit{(b) Handled in the previous point.}

\textit{(c) Even more generally, suppose that $f(x, y) \in R[x, y]$ is any polynomial
  with the symmetry property
  $$f(x, y) = f(y, x)$$
  Prove that there's a polynomial $g$ such that
  $$f(x, y) = g(x + y, xy)$$}

If $f(x, y) = f(y, x)$, then we follow that for every element of the
sum $a_i x^i y^j$ there's an element $a_i x^j y^i$. This implies that
$f(x, y)$ can be partitioned into pairs $a_i x^i y^j + a_i x^j
y^i$. This implies that for each such pair we can take
$$a_i x^i y^j + a_i x^j y^i = a x^{j - i}y^{j - i}(x^i + y^i)$$
latter part ($x^i + y^i$) of this product is a polynomial of form $g(x
+ y, xy)$ by the previous point, which in turn implies that the whole
polynomial is a sum of pollynomials in form $g(x + y, xy)$, which is
itself a polynomial in form $g(x + y, xy)$, which implies the desired
conclusion.

\subsection{}

\textit{I've practically done this one before in the real analysis
  course. There might be some deviations, but I'm pretty sure that they
  are minor.  If it isn't the case, then skip.}

\subsection{}

\textit{For a quaternion $\alpha = a + bi + cj + dk \in H$, let
  $\overline \alpha = a - bi - cj - dk$. }

\textit{(a) Prove that $\alpha \overline \alpha \in \real$.}

$$(a + bi + cj + dk) * (a - bi - cj - dk) = $$
$$ = 
a^2 - abi - acj - adk + abi + b^2 - bicj - bidk
+ acj - cjbi + c^2 - cjdk + adk - dkbi - dkcj + d^2 = $$
$$ = 
a^2 - abi - acj - adk + abi + b^2 - bck + bdj
+ acj + bck + c^2 - cdi + adk - bdj + cdi + d^2 = $$
$$ =  a^2 + b^2  + c^2 + d^2 \in R$$
as desired.

\textit{(b) Prove that $\alpha \overline \alpha = 0$ if and only if $\alpha = 0$}

Follows from the fact that $a^2 + b^2 + c^2 + d^2 = 0 \lra a = b = c =
d = 0$ for any elements of $\real$

\textit{(c) Suppose that $\alpha, \beta \in H$ and that $\alpha \beta = 0$. Prove that either
  $\alpha = 0$ or $\beta = 0$.}

Assume that $\alpha \neq 0$. We follow that there's $q \in \real$ such that
$\alpha \overline \alpha = q = a^2 + b^2 + c^2 + d^2$. We then follow that 
$$\overline{\alpha} * \frac{1}{q}$$
is a multiplicative inverses of $\alpha$. This in turn implies that
$$\alpha\inv \alpha \beta = \alpha \inv 0$$
$$1 \beta =  0$$
$$\beta =  0$$

Same idea applies to $\beta$, but from the different side, which implies the desired result.

\textit{(d) Let $\alpha, \beta \in H$. Prove that
  $$\overline{\alpha + \beta} = \overline \alpha + \overline \beta$$
  and
  $$\overline{\alpha * \beta} = \overline \alpha * \overline \beta, $$
}

$$\overline \alpha + \overline \beta = (a - bi - cj - dk) + (a' - b'i - c'j - d'k) = $$
$$= (a + a') - (b + b')i - (c + c')j - (d + d')k =  \overline{\alpha + \beta}$$
by distributivity and whatnot. Multiplicity is proven in the same way,
but it's more tedious.

\textit{(e) Let $\alpha \neq 0$. Prove that there's $\beta \in H$ such
  that $\alpha \beta = \beta \alpha = 1$}

handled in point (c)

\subsection{}

\textit{let $R$ be a possibly non-commutative ring. The
\textbf{center} of $R$ consists of the elements of $R$ that commute
with every other element of $R$.
  $$R^{center} = \set{a \in R: \alpha \beta = \beta \alpha \text{ for every } \beta \in R}$$
}

\textit{(a) Prove that $R^{center}$ is a commutative subring of $R$}

Proof goes along the samy way as the center of group.

\textit{(b) What is the center of the ring of quaternions $H$?}

We follow that every real entry is commutative.

$$(a + bi + cj + dk) * k = ak + bik + cjk + dk^2 = ak - bj + ci - d$$
$$k * (a + bi + cj + dk) = ak + bki + ckj + dk^2 = ak + bj - ci - d$$
this implies that $(a + bi + cj + dk)$ is not commutative with $k$ if
$b = 0$ and $c = 0$. Multiplying with $j$ will give the same result,
but for $c$ and $k$, which in sum implies that only reals are in the
center.

\textit{Skipped the rest}

\section{Some Important Special Types of Rings}

\subsection*{Notes}

Important note: definition of the field requires a commutative
ring. Thus we can follow that field is a couple of abelian groups,
that are stich together, have distributivity, and thus one of them is
not defined over anothers identity.

For cancelation property we require we require biconditional for some
reason, but $b = c$ or $a = 0$ implies that $ab = ac$ for any given
ring, and thus I'm assunming that there's a typo there (in the text
of the theorem we even see the word "implication")

\subsection{}

\textit{Let $m$ be a positive integet.}

\textit{(a) Prove that $Z/mZ$ is an integral domain if and ony if $m$
  is a prime}

Suppose that $Z/mZ$ is an ingeral domain. Assume that it's not a
prime.  if $m = 1$, then this thing is trivial, and thus assume that
$m > 1$.  This implies that there's a divisor $n$ of $m$ that is less
than $m$.  We follow that $m / n$ is a nonzero element of $Z/mZ$, and
thus
$$n * (m / n) = m = 0$$
which implies that $Z/mZ$ is not an integral domain, which is a
contradiction.

Suppose that $m$ is a prime. We follow that $(Z/mZ)^*$ is a group
under multiplication, and thus product of two nonzeroes is closed
there, and thus is nonzero. This implies that $Z/mZ$ is an integral
domain, as desired.

\textit{(b) Prove that $Z/mZ$ is a field if and ony if $m$
  is a prime}

If $Z/mZ$ is a field, then it's an integral domain, and thus $m$ is a prime.
Implications from the fact that $(Z/mZ)^*$ is a group give us the fact that
all non-zeroes got inverses, and thus $Z/mZ$ is a field, as desired.

\subsection{}

\textit{Let $R$ be a field. Prove that $R$ is an integral domain.}

Suppose that $R$ is a field, but not integral domain. The fact that
it's not an integral domain implies that there are nonzero
$\alpha, beta \in R$ such that 
$$\alpha \beta = 0$$
the fact that $\alpha$ is a nonzero in a field implies that there exists $\alpha \inv$.
This implies that
$$\alpha \inv \alpha \beta = \alpha \inv 0 $$
$$ \beta =  0 $$
which is a contradiction.

\subsection{}

\textit{Prove that each of the following rings is not a field}

\textit{(a) $Z[i]$}

We follow that $2 * (a + bi) = 2a + 2bi$, where the latter part is not identity.

\textit{(b) $\real[x]$}

We follow that $x * 1 = x$ and $x * \sum{a_i x^i} = \sum{a_i x^{i +
    1}}$, which implies the desired.

\textit{(c) $H$}

This one took me by surprise a bit, but it's important to note that
$H$ \textbf{has inverses}, but since it's not mutliplicative, we
follow that it's not a field. Same is true for the next point, but it
is also not an intefral domain.

\subsection{}

\textit{Let $R$ be a commutative ring. Prove that $R$ is an integral
  domain if and only of it's got cancellation property.}

Let $R$ be an integral domain.  Let $a, b, c \in R$ be such that $ab =
ac$ and $a \neq 0$.
$$ab = ac \lra ab - ac = 0 \lra a(b - c) = 0$$
Given that $a \neq 0$ we follow that there are no nonzero elements $q$
of $R$ such that $aq = 0$, and thus we conclude that
$$b - c = 0$$
which implies the desired result. Given that every implication here is
biconditional, we get the reverse case for free.

\subsection{}

\textit{Let $R$ be a finite integral domain. Prove that $R$ is a
  field.}

Let $R$ be a finite integral domain, and let $a \in R \neq 0$. We
follow that if $ab = ac$, then $b = c$, which implies that
multiplication by $a$ is an injection. Given that $R$ is finite and
multiplication by $a$ is a function with domain equal to its codomain,
we follow that multipication by $a$ is a bijection, and thus there's
$b \in R$ such that $a * b = 1$, which implies that $R$ is a field, as
desired.

\subsection{}

\textit{This one is a lot of notation, but not a lot of substance}

\subsection{}

\textit{Let $R$ be a commutative ring. The degree of a non-zero
polynomial $f(x) \in R[x]$ is the highest power of $x$ that appears in
$f(x)$.}

\textit{(a) Prove that for all $f(x), g(x) \in R[x]$ we have
  $$deg(f(x) + g(x)) \leq \max \set{deg(f(x)), deg(g(x))}$$
}

We know that sum of polynomials is the piecewise addition of the
coefficients, that correspond to each degree, thus implying that
the desired result

\textit{(b) If $deg(f) \neq deg(g)$, prove that the previous equation
  is an equality}

Once again, follows directly from the definition

\textit{(c) Suppose that $R$ is an integral domain. Prove that
  for all non-zero $f(x), g(x) \in R[x]$ we have
  $$deg(f(x) g(x)) = deg(f(x)) + deg(g(x)))$$}

We follow that the largest powers of $f(x)$ and $g(x)$ are
in the form $a_i x^i$ and $b_j x^j$, and thus the largest power of
the product is
$$a_i x^i b_j x^j = (a_i b_j)x^{i + j}$$
the fact that $R$ is an integral domain implies that $a_i b_j$ is
not zero, thus implying the desired result.

\textit{(d) Let $R = Z/6Z$. Find polynomials $f(x), g(x) \in R[x]$
  with $deg(f) = deg(g) = 1$ such that previous equality does not hold.}

We follow that
$$3x * 2x = 6x = 0x = 0$$
as desired.

\subsection{}

\textit{Let $R$ be a commutative ring.}

\textit{(a) Prove that there's exactly one integral domain $R$ such
that the map
$$f: R \to R, f(a) = a^6$$
is a ring homomorphism.}

Firstly, we want to state what exactly what "exactly one integral
domain" mean. We want to follow that firstly there exists an integral
domain such that all the rules apply, and that if there's another ring
that that satisfies those conditions, then it's isomorphic to the
ring, that we've made firstly.

We know that if there's a ring homomorphism from $f(a) = a^n$, then
$2^n - 2 = 0$ (from the second exercise in this chapter). In this case
we've got $62 = 0$. It is important to note that this statement is an
implication, and not a biconditional.

We know that prime decomposition of $62$ is
$$62 = 31 * 2$$

With $Z/2Z$ we follow that $f$ is an identity ($f(0) = 0^6 = 0, f(1) =
1^6 = 1$), and this thing is a ring homomorphism. We also follow that
$Z/2Z$ is a field, and thus an integral domain. Thus we follow that there's a
ring such that $f$ is a homomorphism (and isomorphism also).

With $Z/3Z$ we follow that $f(0) = 0, f(1) = 1, f(2) = 1$, which
implies that
$$f(2) + f(1) = 1$$
$$f(2 + 1) = f(3) = 0$$
which means that $Z/3Z$ does not satisfy the conditions.

For multiplication we follow that
$$f(a * b) = (ab)^6 = a^6 b^6 = f(a) * f(b)$$
which is true for all commutative rings. Identities are also pretty
trivial.

Addition is a bit more interesting:
$$f(a + b) = (a + b)^6 = \sum{bc(n, i)a^n b^{n - i}}$$
$$f(a) + f(b) = a^6 + b^6$$
thus we need
$$\sum_{i = 1}^{n - 1}{bc(n, i)a^n b^{n - i}} = 0$$
list of binomial coefficients in this case is
$$6, 15, 20, 15, 6$$

$$(a + b)^6 = a^6 + 6a^5b + 15a^4b^2 + 20a^3b^3 + 15a^2b^4 + 6ab^5 + b^6$$

We can also follow that
$$\sum_{i = 1}^{n - 1}{bc(n, i)a^n b^{n - i}} =
ab \sum_{i = 1}^{n - 1}{bc(n, i)a^{n - 1} b^{n - i - 2}}$$
We'll then follow that both $a, b$ are nonzero, then $ab$ is also
nonzero, and thus
$$\sum_{i = 1}^{n - 1}{bc(n, i)a^{n - 1} b^{n - i - 2}}$$
must be zero. We then follow that none of the products of powers of
$a$ and $b$ are zeroes. We then can also follow that

We follow that
$$f(1 + 1) = 2^6 = f(1) + f(1) = 2$$
thus
$$2^6 = 2$$
we then follow by induction that
$$a^6 = a$$
thus by cancellation property $a^5 = 1$ or $a = 0$.

Suppose that there's a ring, that's got distinct elements $0, 1,
k$. We follow that
$$f(k + 1) = (k + 1)^6$$
$$f(k) + f(1) = f(k) + 1 = k^6 + 1$$


Let's start over:

Let $0, 1, k$ be all distinct items of $R$. We follow that
$$f(k) + f(1) = k^6 + 1$$

$$f(k + 1) + f(k - 1) + f(1 - k) =
f(k) + f(1) + f(k) + f(-1) + f(k) + f(-1) = $$
$$ = k^6 + 1 + k^6 + 1 + k^6 + 1 = 3k^6 + 3$$

$$f(k + 1 + k - 1 + 1 - k) = f(k + 1) = k^6 + 1$$
thus we follow that
$$3k^6 + 3 = k^6 + 1$$
$$3(k^6 + 1) = k^6 + 1$$
thus
$$3 = 1$$


\subsection{}

\textit{Let $R$ be a ring. We define three properties that ann element
  $a \in R$ may posess.}

\textit{$a$ is nilpotent if $a^n = 0$ for some $n \geq 1$.}

\textit{$a$ is unipotent if $a - 1$ is nilpotent}

\textit{$a$ is idempotent if $a^2 = a$.}

\textit{(a) If $R$ is an integral domain, describe all of the
  nilpotent, unipotent, and idempotent elements of $R$.}

We follow that if $a^n = 0$, then $a = 0$ by cancellation property,
which implies that there's one nilpotent element.

If $a - 1$ is nilpotent, then $(a - 1)^n = 0$, thus $(a - 1) = 0$, and therefore $a = 1$.

If $a$ is idempotent, then $a^2 = a$, thus $a = 0$ or $a = 1$ by
cancelation property.

\textit{(b) Let $p \in Z$ be a prime and let $k \geq 1$. Describe all of the
  nilpotent elements in $Z/p^kZ$}

\textit{Skipped the rest of the exercises for later.}

\section{Unit groups and Product rings}

\subsection{}

\textit{(a) Compute the unit group $Z^*$}

We follow that anything times zero is zero, so it's out of the
question for all the subsequent things.

We know that $-1, 1$ have themselves as inverses, and any other number
fails to have one.

\textit{(b) Compute the unit group $Q^*$}

All of the $Q^*$ is the unit group.

\textit{(c) Compute the unit group $Z[i]^*$}

We follow that $-1, 1$ are in the unit group. $i^4 = 1$, and thus $i,
-i$ are also in the group. Thus we follow that for the more general
case
$$a + bi$$
that if one of the numbers $a, b$ are equal to $1$, while the other is
zero, we've got it in the unit group. If $a = b = 1$, then we're out
of luck though.

We know that
$$1 / (a + bi) = \frac{a}{a^2 + b^2}  - \frac{b}{a^2 + b^2}i$$
and thus in order to $a + bi$ to be in $Z[i]$ we need both
$\frac{a}{a^2 + b^2}$ and $\frac{b}{a^2 + b^2}$ to be integers.
We follow that
$$|a| \leq a^2 \leq a^2 + b^2$$
and for $a \notin \set{-1, 0, 1}$ we follow that
$$|a| < a^2 \leq a^2 + b^2$$
which implies that there for the case $a \notin \set{-1, 0, 1}$ there are
no inverses for $a + bi$. We've got the same conclusions for $b$,
and thus we conclude that $\set{1, -1, i, -i}$ is the unit group
(which seems to be pretty straightforward).

\textit{(d) Consider the ring $Z[\sqrt{2}]$. Prove that $1 + \sqrt{2}
  \in Z[\sqrt{2}]^*$. Prove that the powers of $1 + \sqrt{2}$ are all
  different, and use that fact to deduce that $Z[\sqrt{2}]$ has
  infinitely many elements.}

We follow that
$$\frac{1}{1 + \sqrt{2}} =
\frac{(1 - \sqrt{2})}{(1 - \sqrt{2})(1 + \sqrt{2})} =
\frac{(1 - \sqrt{2})}{1 - 2} = -(1 - \sqrt{2}) = -1 + \sqrt{2}$$ by
pretty much same deriviation we can follow that $(1 + \sqrt{2})^n$
(and by extension $(1 - \sqrt{2})^n$) are in $Z[\sqrt{2}]^*$.
We follow that power function is strictly increasing, and thus
$(1 + \sqrt{2})^n$ are all different for all $n \in \omega$, and
thus $Z[\sqrt{2}]^*$ has infinite amount of elements.

\textit{(e) Prove that $\real[x]^* = \real^*$}

We follow that by multiplying any nonzero polynomial in $\real$ by
other polynomial nonzero we get only polynomials of equal or greater
degree, and thus follow that the only one order of polynomial that
has an option of having multiplicative inverses is the polynomial of
degree $1$.

\textit{(f) Prove that $1 + 2x$ is a unit in the ring $(Z/4Z)[x]$ }

$$1 + 2 * 0 = 1$$
$$1 + 2 * 1 = 3$$
$$1 + 2 * 2 = 1$$
$$1 + 2 * 3 = 3$$

we then follow that in order to be constant, elements that result in
$3$ should be, when multiplied, be equal to $1$. We follow that
$3 * 3 = 1$ in this case, and thus we follow that
$$(1 + 2x)^2 = 1$$
which implies that $1 + 2x$ has itself for an inverse, as desired

\textit{Maybe I'll tackle the last one later}

\subsection{}

\textit{(a) Let $R$ be commutative ring, and suppose that its unit
  group $R^*$ is finite with $|R^*| = n$. Prove that every element of $a
  \in R^*$ satisfies
  $$a^n = 1$$
}

We follow that since $R^*$ is finite that $a$ has an order $k$. This
order $k$ divides order of the group $n$, and thus
$$a^n = a^{k * j} = 1^j = 1$$
as desired.

\textit{(b) - already taken care of in previous exercises}

\subsection{}

\textit{(a) Compute the unit group $(Z/pZ)^*$ for each of the primes
  $7, 11, 13$. Which ones are cyclic? }

Since they are all primes, we follow that groups are exactly $(Z/pZ)
\setminus \set{0}$.

In $(Z/7Z)^*$ we follow that order of $3$ is $6$. In $(Z/11Z)^*$ we
follow that order of $2$ is $10$. For $13$ we've got that $2$ has
order $12$.

\textit{(b) Compute the unit group $(Z/mZ)^*$ for each of the
  composite numbers $m = 8, 9, 15$. Which ones are cyclic?}

We follow that
$$(Z/8Z)^* = \set{1, 3, 5, 7}$$
we then follow that $3^2 = 5^2 = 7^2 = 1$, and thus none of them are
cyclic.

We follow that
$$(Z/9Z)^* = \set{1, 2, 4, 5, 7, 8}$$
Order of $2$ is $6$, which means that this thing is cyclc.


We follow that
$$(Z/15Z)^* = \set{1, 2, 4, 7, 8, 11, 13, 14}$$
Orders of those elements get up to $4$, so this group is not cyclic.

\subsection{}

\textit{Let $R_1, ..., R_n$ be rings.}

\textit{(a) Prove that the product $R_1 \times ... \times R_n$ is a ring.}

Follows directly from piecewise application of standart rules of the
respective rings.

\textit{(b) Yeah, projection is a homomorphism}

\textit{Skip the rest}

\subsection{}

\textit{Prove that the product ring $Z/2Z \times Z/3Z$ is isomorphic
to the ring $Z/6Z$ by writing down an explicit isomorphism $Z/6Z \to
Z/2Z \times Z/3Z$}

Let $f$ be a presumed isomorphism between the rings. 
We follow that
$$0 \to (0, 0)$$
$$1 \to (1, 1)$$
by properties of homomorphism.  The fact that $Z/6Z$ is cyclic and
generated by $1$ gives us an idea on how to proceed:
$$2 \to (0, 2)$$
$$3 \to (1, 0)$$
$$4 \to (0, 1)$$
$$5 \to (1, 2)$$

From this we can follow that the first index of value of our function
in $Z/2Z \times Z/3Z$ is $0$ on even numbers and $1$ on odd, which
kinda gives us an idea that it might have to do something with $\%$
operation. Same goes for the latter index, which gives us an explicit
definition.
$$f(x) = (x \% 2, x \% 3)$$
Explicit enumeration here gives us the proof that the function is a
bijection. Properties of homomorphism can be followed by either
explicit enumeration, of perhaps operations are transferred by $\%$
function.

\subsection{}

\textit{Let $R$ be a non-commutative ring. The group os units of $R$ is
  $$R^* = \set{a \in R: \exists b, c \in R: ab = ca = 1}$$}

\textit{(a) If $ab = ca = 1$, prove that $b = c$.}

We follow that
$$b = e b = c a b  = c e = c$$

\textit{(b) Prove that $R^*$ is a group, where we use multiplication
  for the group law.}

Identity is an inverse for intself, and hence is prersent in the
thing.  Associativity is taken care of my the properties of the ring,
and thus we need only to handle the closure on inverses.  Let $a \in
R^*$. We follow that there are $b, c \in R$ such that
$$ab = ca = 1$$
Previous point implies that $b = c$, and thus
$$ab = ba = 1$$
and thus we conclude that $b$ is the inverse of $a$. By the same point
we follow that $b \in R^*$, and thus we conclude that $a \in R^* \ra
a\inv \in R^*$, as desired.

\subsection{}

\textit{Skip}

\subsection{}

\textit{Let $R$ be a ring. $e$ is idempotent if $e^2 = e$.}

\textit{Errata shows that there's an error in this exercise, either
  $R$ should be commutative, or $e$ should be in the center.}

\textit{(a) Let $e \in R$ be idempotent. Prove that $1 - e$ is idempotent and
  that the product of $e$ and $1 - e$ is $0$.}

We follow that
$$(1 - e)^2 = (1 - e)(1 - e) = 1 * 1 - 1 * e - e * 1 + e * e =
1 - e - e  + e = 1 - e$$
thus $1 - e$ is idempotent by definition.

We follow that
$$(1 - e) e = e - e * e = e - e = 0$$
once again by definitions.

\textit{(b) Let $e \in R$ be idempotent with $e \neq 0$. Prove that we
can turn $eR = \set{ea: a \in R}$ into a ring by treating $e$ as the
multiplicative identity.}

We follow that $0 \in eR$, $a \in R \ra -a \in R$, and thus $ea \in eR
\ra -ea \in eR$, thus we've got inverses, and associativity and
commutativity is handled by properties of $R$, thus making $R$ into an
abelian group under $+$.

Let $q \in eR$. We follow that there's $a \in R$ such that $q = ea$,
and thus $eq = eea = ea = q$, thus making $e$ into an identity over
$R$. Since $1 \in R$, we follow also that $e \in eR$. Associativity
and distibutivity over $eR$ is inherited by $R$, and thus we conclude that
$eR$ is indeed a ring, as desired.

\textit{(c) Let $e \in R$ be idempotent with $e \neq 0, e \neq
1$. Prove that the map
  $$R \to eR \times (1 - e)R$$
  $$a \to (ea, (1 - e)a)$$
  is a ring isomorphism.}

Under assumtion that $R$ is commutative we follow that
$$ e (a + b) = e a + eb $$
$$ e (a * b) = e * e * a * b = e * a * e * b $$
for any idempotent $e$, which implies that given function is a product
of homomorphisms, which means that the whole thing is a homomorhism.

Let $a, b \in R$ be arbitrary.  If $f(a) = f(b)$ we follow that $ea =
eb$ and $(1 - e)a = (1 - e)b$.  Since $ea = eb$ we follow that
$$(1 - e)a = (1 - e)b$$
$$b - ea = a - eb$$
$$b - ea = a - ea$$
$$b = a$$
which gives us an implication $f(a) = f(b) \ra a = b$, which implies
that $f$ is injective.

Let $q \in eR \times (1 - e)R$. We follow that there are $a, b \in R$
such that
$$q = (ea, (1 - e)b)$$
We follow that
$$f(ea) = (eea, (1 - e)ea) = (ea, 0a) = (ea, 0)$$
$$f((1 - e)b) = (e(1 - e)b, (1 - e)(1 - e)b) = (0 b, b) = (0, b)$$
and thus
$$f(ea + (1 - e)b) = f(ea) + f((1 - e)b)
= (ea, 0) + (0, (1 - e)b) = (ea, (1 - e)b) = q$$
by homomorphisms and whathot, which implies that for arbitrary $q$
there's an elements of $R$ that is mapped into $q$, which means that
$f$ is surjective, which gives us isomorphism, as desired.

\textit{(d) Let $R_1, R_2$ be rings. Find idempotents $e_1, e_2 \in
R_1 \times R_2$ that also satisfy $e_1 e_2 = 0$. Prove that if $\alpha
\in R_1 \times R_2$, then there are unique elements $a_1 \in R_1$ and
$a_2 \in R_2$ such that $\alpha = a_1 e_1 + a_2 e_2$.}

Something tells me that we can use $\eangle{1, 0}$ and $\eangle{0, 1}$
as our $e_1$ and $e_2$ respectively. We then follow that if
$\alpha \in R_1 \times R_2$, then it's in the form $\eangle{a,b}$,
which in turn implies that we can take
$$a * e_1, + b * e_2$$
to be our value. The fact that this factorization is unique is kinda trivial


\subsection{}

\textit{We use a symbol $L$ to define a ring
  $$R = \set{a + bL: a, b \in Z}$$
  where definition is defined in an obvious way and multiplication is defined
  by using the rules
  $$L^2 = 1, La = aL$$
}

\textit{For the rest of this thing we're assuming that $L \notin Z$, since
otherwise some things don't hold}

\textit{(a) Fint a formula for the product $(a + bL) * (c + dL)$.}

We follow that
$$(a + bL) * (c + dL) = ac + adL + bLc + bLdL = ac + bd + (ad + bc)L$$ 

\textit{(b) Prove that the ring $R$ has zero divisors}

If $L = 1$, then it does not. But if we add an assumption that $L \notin Z$,
then
$$(1 - L) (1 + L) = 1 + L - L - L^2 = 1 - 1 =  0$$

\textit{(c) Prove that the ring $R$ is not isomorphic to the ring $Z \times Z$}

Assume that there's an isomorphism $f$. Identites are mapped pretty
easily,
$$0 \to (0, 0)$$
$$1 \to (1, 1)$$
but we then need to map $L$ somewhere. We follow that there is $(a,
b)$ such that $f(L) = (a, b)$. Since $L$ is not in the $Z$, we
conclude that $a \neq b$. We then follow that $L * L = 1$, and thus
$$a * a = 1$$
$$b * b = 1$$
and since there are no multiplicative inverses for any $a, b \in Z$
apart from $1$ and $-1$ we follow that $a, b \in \set{1, -1}$.  This
in turn implies that either $f(L) = (-1, 1)$ or $f(L) = (1, -1)$.
Assume the latter. We then follow that
$$f(1 + L) = (2, 0)$$
$$f(1 - L) = (0, 2)$$

Now let's try to square some numbers:
$$f((1 + L)(1 - L)) = f(1 - L + L - L^2) = f(0) = 0$$
$$f((1 + L)(1 - L)) = (2, 0) * (0, 2) = 0$$
which is unhelpful.

$$f((1 - L)(1 - L)) = f(1 - L - L + L^2) = f(2 - 2L) = (0, 4)$$
$$f((1 - L)(1 - L)) = (0, 2) * (0, 2) = (0, 4)$$
which is also unhelpful.

I'm pretty sure that we somehow need to prove that the $R$ has
multiplicative inverses for some elements, but $Z \times Z$ does not.
We can follow that
$$(2 - L) * (1 + L) = 2 + 2L - L - 1 = 1 + L$$
while
$$((2, 2) - (1, -1)) * ((1, 1) + (1, -1)) = (1, 3) * (2, 0) = (2, 0)$$
which might be wrong again.

Let us try to prove the thing then, and maybe we'll have some problems
then. All the identities are handled, additivity is trivial, and for
multiplication we've got
$$f((a + bL) * (c + dL)) = f((ac + bd) + (ad + bc)L)
= (ac + bd + ad + bc, ac + bd - ad - bc)$$
$$f(a + bL) * f(c + dL) = (a + b, a - b) * (c + d, c - d)
= (ac + ad + bc + bd, ac - ad - bc + bd)$$
so we're kinda shafted in this regard. Maybe the $f$ is not a
bijection then. It's ceirtanly a function from $R$ to $Z \times Z$,
and is defined on all $R$. If there are $a, b \in R$ such that $a \neq
b$ and $f(a) = f(b)$, then we conclude that $f(a - b) = f(a) - f(b) =
0$, and hence there's a nonzero element $q \in R$ such that it maps to
$0$. Construction of $f$ though prohibits such a thing, and thus we
must conclude that it's an injection.

Okay, now I've got it. Let $a + bL \in R$. We follow that
$$f(a + bL) = (a + b, a - b)$$
we then conclude that the difference between those two
$$a + b - a + b = 2b$$
is always even. Thus we conclude that there's no way for example to
map the thing to $(1, 0)$, since the difference between indices is odd,
and thus we conclude that $f$ is not surjective, and thus we conclude that
$R$ and $Z \times Z$ are not isomorphic. 

\textit{(d) However, prove that the ring
  $$S = \set{a + bL: a, b \in Q}$$
  is isomorphic to the ring $Q \times Q$.
}

This case grants us on the other hand the ability to get surjectivity,
that we can prove through a lot of notation, but it's easy to see why
it is the case from the last paragraph of the previous point.

\subsection{}

\textit{Let $R_1, R_2, ...$ be an infinite list of
rings. blah-blah-blah, we can define rings from infinite list of rings
by the same reason as for the multiple rings (i.e. piecewise
application). Part (b) comes fromm the fact that the ring that is
there's an inection from $2^\omega$ to $Z/2Z^\omega$}

\textit{Skip the rest}

\section{Ideals and Quotient Rings}

\subsection*{Notes}

Firstly it's important to note that ideals are mostly not subrings due
to the fact that the ones that aren't equal to the whole space $R$
don't have multiplicative identitites in them. They are subgroups
under addition however, which is evident from their definition.

Given that ideals are subgroups under the addition, we can conclude
from our endeavours in group theory that the set of distinct cosets
of an ideal constitutes a partition of a ring $R$. A partition of a
ring $R$ with respect to an ideal $I$ is denoted by $R/I$. Apart from
the initial ideal $I$ we can follow that no other element of $R/I$ is
an ideal due to the fact that any given ideal $I$ has got to have $0$
in it.

Suppose that we've got an $R/I$. Let $q \in R/I$ and $b \in q$.  Since
$q \in R/I$ we follow that by definition there is an element $j \in R$
such that
$$q = j + I$$
(where sum of an element of $R$ and a subset of $R$ is defined in an
obvious way). $b \in q$, and thus by defnition of addition of a set to
an element we conclude that there's an element $k \in I$
such that
$$b = j + k$$
and thus
$$j = b - k$$
and therefore if $w \in q$ is an arbitrary element, then there's $e
\in I$ such that
$$w = j + e$$ 
thus
$$w = b - k + e$$
since $k, e \in I$ we follow that $-k \in I$, thus $-k + e \in I$, and
therefore $w \in b + I$. Thus $q \subseteq b + I$. The case with
$\supseteq$ is kinda similar, which gives us an identity $q = b + I$,
which together with the fact that $b$ and $q$ are arbitrary element
gives us a biconditional
$$q \in R/I \land b \in q \lra q = b + I$$
thus any given element of $R/I$ can be represented as a sum of any of
its elements and $I$.

Formulas that were given in the book don't give out too much
rigor, and thus let us clarify the whole shebang:

In the book we're given formulas
$$(a + I) + (b + I) = (a + b) + I$$
$$(a + I) * (b + I) = (a * b) + I$$
and without any context this thing does not make much sense. We know
that for any given coset $q$ there's an element $a \in R$ such that $q
= a + I$ (which we're gonna call an \textit{additive representation}
of a coset). Additive representations are only representations, and
from our previous point in the notes we can follow that any given
coset can have several additive representations, and a single representation
refers only to one coset. Thus this representation can be formulated
as "one-to-many" from cosets to representations.

With our notion of representations we can now make sense of the
formulas. Given two representations we follow that there exists a
single representation, that conforms to the provided formulas. Given
that there is a single coset that is represented by a representation,
we conclude that formula essentially is a binary function from
representations to cosets. In the book (and further in the exercises)
we are proving the fact that those formulas are "well defined",
i.e. that given two cosets, there exists only one coset, to which
their respective representations are being sent, and thus that we can
define a proper function from pairs of cosets to cosets, that conforms
to the formula.

Although with proper maintenance we can define sums and procuts
in a way, that they were defined in the book, I wonder if we could
define the same thing in terms of ranges. 
Given $a, b \in R/I$, we can define sum and
product functions $+: R/I \times R/I \to \pow(R)$
$*: R/I \times R/I \to \pow(R)$ by
$$a + b = \set{q + w \in R: q \in a \land w \in b}$$
$$a * b = \set{q * w \in R: q \in a \land w \in b}$$
where it is essentially a range of underlying $+$ and $*$ respectively
over a set of pairs between elements of $a$ and $b$. Problem is that
those functions are having a codomain of $\pow(R)$, but we want to
prove specifically that we can restrict codomains of those functions
to $R/I$. Material in the book and the exercises certainly implies
that ranges of functions over underlying sets are subsets of
a single coset (i.e. given $a, b \in R/I$ there's $c \in R/I$
such that $a + b \subseteq c$ and the same goes for $*$).

UPDATE: after doing some exercises, I became convinced that this thing
wont work after all. Defining product as a sum of products (as with
ideals) might solve the problem, but I wont investigate this thing
further. 

In order to try to prove that our definitions are indeed equivalent to
the ones in the book, we might want to prove some other things about
cosets. Notably we need to show that if $b' \in R/I$, $w \in I$, and
$b \in b'$, then $b + w \in b'$. We can do that by a straightforward
application of the definition of coset.

Let $a', b', w \in R/I$ be such that $a' + b' \subseteq w$. Let $q \in w$.
We follow that since $q \in w$ that $w = q + I$. Let $a \in a'$, $b \in b'$.
We follow that $a + b \in w$, and thus $a + b \in q + I$. Therefore
there is an element $e \in I$ such that $a + b = q + e$. Therefore
$a + b - e = q$. Since $e \in I$ we follow that $-e \in I$, thus
$b - e \in b'$, and therefore we conclude that for $q \in w$ there
is a pair of element $\eangle{a, b - e} \in a' \times b'$ such that
$q = a + b - e$, and thus $w \subseteq a' + b'$, which implies that
$a' + b' = w$.


With our definitions we can try to produce formulas, that were
given in the book:
Let $a', b' \in R/I$.  Let $a \in a', b \in b'$. We follow that
$$a' = a + I$$
and
$$b' = b + I$$
as discussed previously. Now let us prove the identity in the book. Let
$$q \in (a + b) + I$$
We follow that it is the caseif and only if there's $w \in I$ such that
$$q = (a + b) + w \lra q = (a + 0) + (b + w)$$
which implies that
$$q \in (a + I) + (b + I)$$

Now assume that $q \in (a + I) + (b + I)$. We follow that there are
$w, e \in I$ such that
$$q = a + w + b + e$$
thus
$$q = (a + b) + (w + e)$$
given  that $w, e \in I$ we follow that $w + e \in I$ and thus
$$q \in (a + b) + I$$
thus giving us an identitty
$$(a + I) + (b + I) = (a + b) + I$$
which gives us the formula from the book.

Let $a, b \in R$. We follow that
$$q \in (a + I)  * (b + I)$$
implies that there are $c, d \in I$ such that
$$q = (a + c) * (b + d)$$
$$q = ab + cb + ad + cd$$
we follow that $cb, ad, cd \in I$ by definition of ideal, and thus
$$cb + ad + cd \in I$$
therefore
$$q \in ab + I$$
which gives us a statement
$$(a + I)  * (b + I) \subseteq a * b + I$$


\subsection{}

\textit{Let $R$ be a commutative ring.}

\textit{(a) Let $c \in R$. Prove that
  $$\set{cR: r \in R}$$
  is an ideal of $R$.
}

Let's name the unnamed set $Q$ and assume that $q, w \in Q$. We follow that
there are $a, b \in R$ such that $q = ca$ and $w = cb$. We then follow that
$$q + w = ca + cb = c(a + b) \in Q$$
where we derive the last relation by definition of $Q$.

If $r \in R$, then we conclude that
$$rq = rca = cra \in Q$$
where deriviations come from the fact that $R$ is commutative. Hence we conclude that
$Q$ is an ideal by definition. 

\textit{(b) ... straightforward applications of definitions and
commutative/distributive properties of $R$ will give us this
conclusion}

\subsection{}

\textit{Let $R$ be a commutative ring. Prove that $R$ is a field if
and only if its only ideals are the zero ideal and the entire ring
$R$.}

Assume that $R$ is a field and let $c \in R$ be nonzero. We follow that
there's $c\inv \in R$, and thus $c\inv R$ is a subset of $R$ by closure. Multiplying
$c$ by this subset we get that $R \subseteq cR$, and thus $c \neq 0 \ra cR = R$.
If $c = 0$ we can trivially conclude that $cR = \set{0}$.

Now assume that $R$ is a commutative ring whose only ideals are
$\set{0}$ and $R$. Let $q \in R$ be nonzero. We follow that $qR = R$,
and thus there's $w \in R$ such that $qw = 1$. This implies that $w =
q\inv$, and thus we conclude that the fact that $q$ is arbitrary
together with this implication constitutes a definition of a field,
as desired.

\subsection{}

\textit{Let $R$ be a commutative ring, and let $I$ be an ideal of $R$. The radical of $I$
  is defined to be the set
  $$Rad(I) = \set{a \in R: (\exists n \in \omega)(n \geq 1 \land a^n \in I)}$$
  Prove thaht $Rad(I)$ is an ideal of $R$.
}

Let $a \in Rad(I)$ and $b \in R$. We follow that there's $n \in \omega$ such that $a^n \in I$
by definition of $Rad(I)$. We also follow that since $b \in R$ that $b^n \in R$ and thus
we follow that $(ab)^n = a^n b^n$ is a product of an element of $I$ and an element of $R$, and
thus is in $I$ by the fact that $I$ is an ideal.


Let $a \in Rad(I)$. We follow that there's a $n \in \omega$ such that
$a^n \in I$. We then follow that for all $m \geq n$ that
$a^m = a^n a^{n - i}$ and thus $a^m$ is a multiple of an element of $I$ ($a^n$)
and an element of $R$, and thus it is itself an element of $I$.

Now assume that $a, b \in Rad(I)$ and $m, n \in \omega$ are such that
$a^n, b^m \in I$. We follow that
$$(a + b)^k = \sum_{i \in \set{0..k}}{bc(i, k)a^i b^{k - i}}$$
for any $k \in \omega$. Thus if $k \geq 2\max(m, n)$, then for any
element of the sum
$$bc(i, k)a^i b^{k - i}$$
either $i$ or $k - i$ is greater than $\max(m, n)$, and thus $a^i$ or
$b^{k - i}$ is in $I$, thus the whole product is in $I$, and thus then
the sum is a sum of elements of $I$, thus making the whole sum an
element of $I$. Thus we follow that there's an element $n$ of $\omega$
such that $(a + b)^n \in I$, and therefore we conclude that $a + b \in
Rad(I)$.

Those two properties constitute a definition of the ideal, as desired.

\subsection{}

\textit{The goal of this exercise is to prove that every iideal in $Z$
is principal ideal.}

\textit{(a) Let $I$ be a non-zero ideal in $Z$. Prove that $I$
contains a positive integer.}

If $I$ is a non-zero ideal, then it's got $a \in I$ such that $a \neq
0$. If $a > 0$, then we're done, and if not, then we follow that $-1 *
a > 0 \in I$.

\textit{(b) Let $I$ be a non-zero ideal in $Z$. Let $c$ be the
smallest positive integer in $I$. Prove that every element of $I$ is a
multiple of $c$.}

We follow that all the multiples of $c$ are in $I$. If there's some
element $q \in I$ that is not a multiple of $c$, then we follow that
$gcd(q, c) < c$, and thus there are integers $a, v \in Z$ such that
$aq + vc = gcd(q, c)$. Since $q, c \in I$ we follow that $gcd(q, c)
\in I$, and thus we conclude that there's a positive element of $I$
that is less than $c$, which is a contradictions

\textit{(c) Prove that every ideal in $Z$ is principal.}

Directly follows from the previous point.

\subsection{}

\textit{Prove the remaining parts of Proposition 3.32. Let $R$ be a
commutative ring, and let $I$ be an ideal of $R$.}

\textit{(a) Let $a + I$ and $a' + I$ be two cosets. Prove that $a + I
= a' + I$ if and only of $a - a' \in I$.}

Suppose that $a + I = a' + I$. We follow that there are elements $i,
i' \in I$ such that $a + i = a' + i'$. Then we follow that $a - a' =
i' - i$. Since $i', i \in I$ we follow that $-i \in I$ and $i' - i \in
I$, and thsus $a - a' \in I$.

If $a - a' \in I$, then we follow that there's and element $i \in I$
such that $a - a' = i$, thus $a = a' + i$. We then follow that for
every element in $q \in a + I$ there's an element $i' \in I$ such that
$q = a + i'$, and thus $q = a' + i + i'$. By the additive closure of
$I$ we follow that $i + i' \in I$, and thus we conclude that $q \in a'
+ I$, which gives us the $\subseteq$ relation between those
sets. Reverse direction is kinda similar, and thus we conclude the
desired reverse implication.

\textit{(b) Prove that addition of cosets is well-defined.}

Was taken care of in the notes.

\textit{(c) Prove that addition and multiplication of cosets turns
  $R/I$ into a commutative ring.}

Definition of the sum and the axioms of underlying ring imply all the
necessary conditions in a straightforward fashion.

\subsection{}

\textit{Let $R$ be a commutative ring.}

\textit{(a) Let $I$ be an ideal of $R$. Prove that the map $f$
  $$R \to R/I, a \to a + I$$
  sending an element to its coset is a surjective ring homomorphism whose kernel is $I$.
  This is proposition 3.34(a)}

$R/I$ is a partition of $R$, and thus if $q \in R/I$, then there's $w
\in q \subseteq R$, thus $f(w) = q$, which implies that $f$ is
surjective. Identities are mapped properly by definition, and the fact
that the thing is a homomorphism is handled by definitions of $R/I$. If $q \in I$,
then $f(q) = q + I = 0 + I$, and otherwise it isn't, which implies that the whole
$f(q) = 0$ if and only if $q \in I$, thus making $I$ into the kernel of $f$.

\textit{(b) Let $I$ and $J$ be ideals of $R$. Prove that the map $g$
  $$R \to R/I \times R/J, a \to (a + I, a + J)$$
is homomorphism. What is its kernel? Give an example where it is
surjective and an example for which it is not surjective.}

Although this thing looks unconventional, it is nothing but a
concatenation of homomorphisms and thus a homomorphism itself. Kernel
of this thing is the intersection of underlying rings ($I$ and $J$
respectively in this case).

We can set $I = 2Z$, $J = 3Z$, and thus if $(a, b) \in Z/2Z \times Z/3Z$,
then we follow that we have set $b + 3Z$ has both even and odd elements,
and thus we're able to produce proper element depending on $a$, thus making the
map surjective. We can also set $I = J = 2Z$, from which we can't produce element
$(1, 0)$, which gives us a non-surjective map.

\subsection{}

\textit{Let $I$ be the principal ideal of $\real[x]$ generated by
  polynomial $x^2 + 1$. Prove that the map
  $$\phi: \real[x]/I \to C, \phi(f(x) + I) = f(i)$$
  is a well-defined isomorphism, where $i = \sqrt{-1}$, as usual.}

If $q$ is an element in $R/I$, then we follow that for all $f \in q$ we get
$$q = f + I$$
Thus if for any $q \in R/I$ we follow that for all $f_1, f_2 \in q$
$$f_1(i) = f_2(i)$$
then can conclude that any element of $q$ maps $i$ to the same value,
and thus there's a function from $R/I$ to $C$ that maps the things
exacly in a way that the exercise describes. Now let us try to prove
that and let $q \in R/I$ and $f_1, f_2 \in q$. We follow that there's
$w \in C$ such that
$$f_1(i) = w$$
Since $f_1, f_2 \in q$ we follow that $f_1 - f_2 \in I$, and by
definition of $I$ we follow that there's $c \in \real[x]$ such that
$$f_1 - f_2 = c(x^2 + 1)$$
therefore
$$f_2 = f_1 - c(x^2 + 1)$$
and thus
$$f_2(i) = f_1(i) - c(x^2 + 1) = w - c(-1 + 1) = w - c0 = w$$
which all implies the fact that the whole shebang is "well-defined".

Now we need to prove that this thing is an isomorphism. Identities are
easy, functions are as well, and hence this thing is a
homomorphism. Bijectivity is kinda interesting though.

If $r \in I$, then $r(i) = c(-1 + 1) = 0$. From our previous exercises
know that for any polynomials in commutative rings
$$r(k) = 0 \iff r(x) = (x - k)f(x)$$
for some $f \in R$, and thus $r(i) \neq 0$ implies that there's no
polynomial $g \in R$ such that $r = (x - i) g$. In polynomials in real
coefficients it's also not too complicated to prove that
$\overline{r(x)} = r(\overline{i})$, which means that $-i$ is also not
a root for this polynomial, which gives us the fact that $r \neq (x -
i) (x + i) g = (x^2 + 1)$.  This implies that $r \notin I$, and thus
kernel of $\phi$ is $I$. Theorem in the book proves then that $\phi$
is injective.

If $k \in C$, then we follow that there are $a, b \in R$ such that $k
= a + bi$, and thus
$$f(x) = a + bx$$
is an element of $R$ such that $f(i) = k$, which implies that $\phi$
is surjective, which gives us the fact that $\phi$ is an isomorphism,
as desired.

\subsection{}

\textit{Let $R$ be a commutative ring and let $I$ be an ideal of
  $R$. For $a \in R$, we will denote the coset $a + I$ by
  $\overline{a}$, i.e.
  $$R \to R/I: a \to \overline{a}$$
  is the reduction modulo $I$ homomorphism. We define a map of
  polynnomial rings
  $$\phi: R[x] \to (R/I)[x], \phi(\sum {a_i x^i}) = \sum {\overline{a_i} x^i}$$
  by reducing the coefficieents modulo $I$. Prove that $\phi$ is a ring
  homomorphism.}

We can easily follow that identities are mapped as expected. The rest is basically taken care of
by the definitions and previous proves:
$$\phi(a + b) = \phi(\sum {a_i x^i} + \sum {b_i x^i}) = \phi(\sum {(a_i + b_i) x^i}) =
\sum {\overline{(a_i + b_i)} x^i}$$
$$\phi(a)  + \phi(b) = \sum {\overline{a_i} x^i} + \sum {\overline{b_i} x^i}) =
\sum {\overline{(a_i + b_i)} x^i}$$


$$\phi(a * b) = \phi(\sum {a_i x^i} * \sum {b_i x^i}) = \phi(\sum{\sum {a_i b_j x^{i + j}}}) =
\sum{\sum {\overline{a_i b_j} x^{i + j}}}$$
$$\phi(a) * \phi(b) = \sum {\overline{a_i} x^i} * \sum {\overline{b_i} x^i})
= \sum{\sum {\overline{a_i b_j} x^{i + j}}}$$
as desired.

\subsection{}

\textit{Let $R$ be a commutative ring and let $I$ and $J$ be ideals of $R$}

\textit{(a) Prove that the intersection $I \cap J$ is an ideal of $R$.}

We follow that if $w, e \in I \cap J$, then $w + e \in I$ by axioms of
$I$ and $w + e \in J$ by axioms of $J$, and thus $w + e \in I \cap
J$. If $q \in R$ is arbitrary, then $wq \in I$ by axioms of $I$ and
$wq \in J$ by axioms of $J$, which implies that $wq \in I \cap
J$. This constitutes a definition of an ideal, as desired.

\textit{(b) Prove that the ideal sum (that is defined in an obvious
way) is an ideal of $R$ }

If $q, w \in I + J$, then there are $a, b \in I$ and $c, d \in J$ such
that $q = a + c, w = b + d$, and we can follow that then
$$q + w = a + b + c + d = (a + c) + (b + d) \in I + J$$
If $e \in R$ is arbitrary, then
$$e + q = e + a + b = (e + a) + b \in I + J$$
which implies that $I + J$ is indeed an ideal, as desired.

\textit{(c) The ideal product of two ideals is defined to be
  $$IJ = \set{\sum{a_i b_i}: n \geq 1 \land a_n \in J \land b_n \in J}$$
  Prove that IJ is an ideal of $R$}

Pretty much the same idea as before: sums do not even need to be rearranged,
and
$$c\sum{a_i b_i} = \sum{c a_i b_i} = \sum{(c a_i) b_i}$$
which by power of the fact that $I$ is an ideal gives us the desired result.

\textit{(d) One might ask why the product $IJ$ of ideals isn't simply
defined as the set of products. The answer is that the set of product
need not be an ideal. Here is an example. Let $R = Z[x]$, and let $I$
and $J$ be the ideals
$$I = 2Z[x] + xZ[x], J = 3Z[x] + xZ[x]$$
Prove  that the set of products is not an ideal.}

We follow that
$$a_1 = x, a_2 = -2 \in I$$
$$b_1 = x, b_2 = 3 \in I$$
We then follow that $a_1 b_2 = 3x$, $a_2 b_1 = -2x$ are in the set of
products. We then follow that $3x - 2x = x$ is a sum of elements of
the set of products. Let $i \in I$ and $j \in J$ be arbitrary.
We follow that if $i * j = x$, then degrees of $i$ and $j$ got
to be $1$ and $0$ in order for the product to be of order $1$.
Since $1$ is a multiple of neither $2$ or $3$ we can follow that
there are no polynomials, which implies that the set is not closed
under addition, and thus is not an ideal.

\textit{(e) On the other hand, prove in general that if either $I$ or
  $J$ is a principal ideal, then the set of products is an ideal.}

Assume that $I$ is a principal ideal. We follow that if $a', c' \in I$
and $b, d \in J$ are elements, then their pairwise products are in the
set of procuts, and thus for arbitrary $e \in R$ we've got that
$$c a'b = (ea') * b$$
which satisfies the multiplication part of the definition of an ideal.
Since $I$ is a principal ideal, we follow that there's $q \in R$ that
is a defining element of $I$, and thus there are $a, c \in R$ such
that $a' = qa, c' = qc$. Thus
$$a'b + c'd = qab + qcd = q(ab + cd)$$
$b, d \in J$, and thus $ab \in J$ and $cd \in J$. Together with the
fact that $q \in I$ we follow that the abovementioned element is
indeed an element of the set of products, which gives us the closure
of the set under addition. ] Together with the previous part it
constitutes a definition of an ideal, as desired.

\subsection{}

\textit{Let $R$ be a ring, let $I$ be an ideal of $R$, and for any other ideal $J$ of $R$,
  let $\overline{J}$ be the following subset of the quotient ring $R/I$:
  $$\overline{J} = \set{a + I: a \in J}$$}

\textit{(a) Prove that $\overline{J}$ is an ideal of $R/I$.}

If $q', w' \in \overline{J}$, we follow that
$$q' + w' = q + I + w + I = (q + w) + I$$
$q + w \in J$ by the powers of $J$, and thus the whole thing is
$\overline{J}$. By similar logic product of any element of $R/I$ and
$\overline{J}$ is in $\overline{J}$, which concludes the proof.

\textit{(b) Let $\overline{K}$ be an ideal of $R/I$. Prove that the
  set
  $$Q = \bigcup_{a + I \in \overline{K}}{(a + I)}$$
  is an ideal of $R$ that contains $I$.}

By definition of $\overline{K}$ and  underlying $K$ we follow that
$0 \in K$, and thus $0 + I = I \in \overline{K}$, which implies that
$I \subseteq Q$. If $a, b \in Q$, then there are appropriate $a', b' \in K$,
and thus
$$ca = c (a + k) = ca + ck$$
$ck \in K$, and thus the abovementioned thing is an element of $ca + K$.
$$a + b = a_1 + k_1 + b_1 + k_2 = (a_1 + b_1) + (k_1 + k_2)$$
and thus blah-blah-blah, it is all covered.

\textit{(c) Conclude that there is  a bijective map
  $$ \set{\text{ideals of }R\text{ that contain } I} \to \set{\text{ideals of }R/I}, J \to J/I$$}

By point (a) for every ideal of $R$ (whether or not it contains $I$)
there's an appropriate ideal of $R/I$. If $R_1 \neq R_2$ are ideals that contain $I$,
then we follow that there is $r_1 \in R_1 \setminus I$ such that $r_1 \notin R_2$
(if not, then swap'em around and assume so). Thus $r_1 + I$ is in the $\overline{R_1}$
but not in $\overline{R_2}$ by their respective definitions. Thus the whole thing
is injective. If $Q$ is an ideal of $R/I$, then by point $(b)$ we've got an ideal of
$R$ that contains $I$. If $I \subseteq J$ and $J$ is an ideal, then we
can follow that point (b) is essentially an inverse function of
construction of $\overline{J}$, which produced the desired result.


\subsection{}

\textit{Prove that the set of nilpotent elements is an ideal of $R$.}

It's the radical of ideal $\set{0}$.

\section{Prime Ideals and Maximal Ideals}

\subsection*{Notes}

There's gotta be an easier way to do this, but we gonna try to prove
that for every non-invertible element of an $R$ there's a maximum
ideal, that contains it. Let us assume that $q \in R$ is such that
$q \notin R^*$. We follow that all of the $\pow(R)$ is a set, and thus
there's a set of all ideals of $R$. Therefore there's a set of all
ideals of $R$ that contain $q$, and if we remove $R$ from this set,
then we get a set of all ideals of $R$, that are not $R$, but contain
$q$ (this set is not empty due to restriction on $q$; this can be
derived, but it's not that complicated, key is to assume that since
$q \notin R^*$ that there's no nonzero power of $q$ that is equal to
$1$ ). We then follow that this thing is a poset under $\subseteq$,
and let $C$ be a chain in it. We then conclude that
$$1 \in \cup C \ra \exists Q' \in C: 1 \in Q' \ra Q' = R$$
which implies that $1 \notin \cup C$. Now if $r, w \in \cup C$,
then we follow that $r \in Q_1, w \in Q_2$, and by the
fact that $C$ is a chain we follow that $r \in Q_2$ or $w \in Q_1$,
which in turn implies that $r + w \in Q_1$ or $r + w \in Q_2$,
and thus $r + w \in \cup C$. By similar process we conclude that for
arbitrary $r \in R$ and $c \in \cup C$ we've got that $rc \in \cup C$,
which implies that $\cup C$ is an ideal that contains $q$, and thus
$\cup C \in C$.

By Zorn's lemma then we follow that there's a maximal element $M$ in
the set of ideals of $R$ that aren't $R$ and contain $q$ under
$\subseteq$. Let now $r \in R \setminus M$. We follow that if $J$ is
an ideal such that $M \subseteq J$ and $r \in J$, then by definition
of $M$ and $r$, $J$ is not an ideal of $R$ that isn't $R$, and thus $J
= R$, which implies that $M$ is a maximal ideal . Thus we conclude the
following:

\textbf{Let $R$ be a commutative ring. If $q \notin R^*$, then there's
a maximal ideal $M$ that contains $q$.}


\subsection{}

\textit{Let $I$ be the following subset of the ring $Z[x]$ of
  polynomials having integer coefficient.
  $$I = \set{2a(x) + x b(x) : a(x), b(x) \in Z[x]}$$ 
}

\textit{(a) Prove that $I$ is an ideal of $Z[x]$.}

We follow that both $a(x), b(x)$ are ideals, both $2 a(x)$ and $x
b(x)$ are principal ideals, and thus $I$ is a set of sums of two ideals,
and thus itself  an ideal (as proven in the exercises).

\textit{(b) Prove that $I \neq Z[x]$}

$1$ is not present in $I$

\textit{(c) Prove that $I$ is not a principal ideal}

Assume that $q(x)$ is an element, that generates $I$. We follow that it
generates both $2$ and $x$, for which the only common multiple is $1$,
which implies that $q = 1$, which implies that $I = Z[x]$, which is
not true.

\textit{(d) Prove that $I$ is a maximal ideal of $Z[x]$}

Let $J$ be an ideal of $Z[x]$ that properly contains $I$.  Assume that
$q \notin I$. We follow that since $q \notin I$ that it cannot be a
multiple of $x$, and thus it has to contain an element $b$ of sum of
degree $0$. Since $q \notin I$ we follow that this element is not even
(and thus odd). Thus we then follow that $q - b$ is a multiple of $x$
that does not have an element of the sum of degree $0$, and thus is a
multiple of $x$. Therefore we can state that $q - b \in I$, which
implies that $q - (q - b) = b \in J$. Since $b \in J$ is an odd
number, we follow that $b - 2n \in J$ for all $n \in Z$, which implies
thaat $1 \in J$, which in turn implies that $J = Z[x]$, as desired.

\subsection{}

\textit{(a) Let $m \neq 0$ be an integer. Prove that the ideal $mZ$ is
a maximal ideal if and only if $|m|$ is a prime number}

Assume that $mZ$ is a maximal ideal. If $m$ is not a prime, then there are
$a \in Z \setminus \set{0, 1}, b \geq 2$ such that $m = ab$. By the restriction on
$b$ we follow that $a, b$ aren't multiples of $m$, and thus $mZ$ is not a prime ideal.
Thus contrapositive of this statement implies that $m$ is a prime, as desired.

Assume now that $m$ is a prime and let $b \in Z \setminus mZ$. We
follow that $gcd(m, b) = 1$, thus there are integers $a, v$ such that
$am + bv = 1$, which implies that if $J$ is an ideal that contains both $mZ$ and $b$,
then $J = Z$, which implies that $mZ$ is a maximal ideal, as desired.

\textit{(b) Let $F$ be a field, and let $a, b \in F$ with $a \neq
  0$. Prove that the principal ideal $(ax + b)F[x]$ is a maximal ideal
  of the polynomial ring $F[x]$.}

Since $a \in F$ and $a \neq 0$ we follow that $a \inv \in F$. Thus by
pretty much the same logic as in previous exercise (part c) we
conclude the same result.

\textit{(c) Skip}

\subsection{}

\textit{Let $R$ be a ring, let $M \subseteq R$ be a maximal ideal, and
let $R^*$ be its group of units. Prove that the following is
equivalent}

\textit{(a) $M$ is the only maximal ideal of $R$}

\textit{(b) $R = M \cup R^*$.}

\textit{A ring having a unique maximal ideal is called a local ring.}

What is interesting is that if $r \in R^*$, then there is $r\inv \in
R^*$, and thus if $r \in M$ then $r r\inv = 1 \in M$ by properties of
an ideal, and thus $M = R$, which is not true. This implies that $M
\cap R^* = \emptyset$ for any maximal ideal $M$. Thus the whole thing
is a partition.

Assume (a) and contradict (b). Let $q \in R \setminus (M \cup
R^*)$. Theorem in the notes implies that there's a maximal ideal,
that contains $q$, and thus definition of $q$ implies that $M$ is
not the only maximal ideal, which is a contradiction (holy heck,
that was an overkill)

Let us assume (b). Prerequisites of the exercise imply that $M$ has
got to be a maximal ideal, so we are only required to derive the
uniqueness of it. Assume that $M'$ is a maximal ideal of $R$,
that is different from $M$. We follow that $M'$ cannot
be a subset of $M$, since $M'$ is a maximal ideal, and thus
there's $q \in M' \setminus M$. (b) now implies that $q \in R^*$,
and thus there's $q\inv \in R$, which means that $q q\inv = 1 \in M'$,
which implies that $M' = R$, which is false, and thus we conclude the
desired result, which was pretty easy.

\textit{The rest of the exercises are skipped for now}

\chapter{Vector Spaces - Part 1}

\textit{Practically all the material here was covered in my course on
  linear algebra }

\chapter{Fields - Part 1} 

\section{Introduction to Fields}

\section{Abstract Fields and Homomorphisms}

\subsection{}

\textit{Let $F$ be a field and let $f(x) \in F[x]$ be a non-zero polynomial.}

\textit{(a) Suppose that $\alpha \in F$ is a root of $f(x)$ Prove that
there is a polynomial $g(x) \in F[x]$ such that $f(x) = (x - \alpha)
g(x)$.}

This one is a repeat of an exercise two chapters before (3.8(b))

\textit{(b) More generally, suppose that $\alpha_1, ..., \alpha_n \in
F$ are distinct roots of $f(x)$. Prove that there is a polynomial
$g(x)$ such that
  $$f(x) = (x - \alpha_1) (x - \alpha_2) ... (x - \alpha_n) g(x)$$
}

We're gonna use the induction on this one. Part (a) will suffice as a
base case. Let $f(x) = (x - \alpha_1) g(x)$ and $\alpha_2 \neq
\alpha_1$ is a root of $f(x)$. We follow then that
$$(\alpha_2 - \alpha_1) g(\alpha_2) = 0$$
Since $F$ is a field, it's an integral domain, and thus it's got
cancellation property.  $(\alpha_2 - \alpha_1) \neq 0$ by our
assumption, and thus we follow that $g(\alpha) = 0$.  Part(a) implies
all the things that it needs to imply, which gives us the desired
result.

\textit{(c) Use (b) to deduce the following important result:}

\textit{\textbf{Theorem} Let $F$ be a field, and let $f(x) \in F[x]$
be a nonzero polynomial.  Then $f(x)$ has at most $deg(f)$ distinct
roots in $F$.}

If there are more roots, than the degree, then part (b) implies that
it's got such a decomposition, which gives it greater degree than it's
got, which gives us a contradiction.

\subsection{}

Done this one.

\section{Interesting Examples of Fields}

\subsection{}

\textit{Prove that each of the following subsets of $R$ is a field}

\textit{(a) $Q(\sqrt{3})$}

The same idea as with $\sqrt{2}$ works here as well: inverses are
found in a same way as inverses of complex numbers, closure is kinda
trivial, and the rest is inherited from $R$. 

\textit{(b) $Q(\sqrt{4})$}

Since $\sqrt{4} = 2$, it's just $Q$

\textit{(c) Skip}

\subsection{}

\textit{Prove that each of the following rings is not a field:}

\textit{(a) $Z[i]$}

We can follow that there's no inverse of $2$ (or any other non-1
element), simpliest proof of that comes from the fact that by
multiplying any given element by that thing gives us a tuple,
that constists of multiples of that number, and $1$ is not a multiple
of any non-1 number

\textit{(b) $Z[\sqrt{2}]$}

Same idea as in (a)

\textit{(c) $Z/p^2Z$, where $p$ is a prime}

We can follow that $p \neq 0$, but $p^2 = 0$, which implies that
the whole thing is not an integral domain, and thus not a field.

\textit{(d) $Z/mnZ$, where $m, n \geq 2$}

Same idea as in (c)

\subsection{}

\textit{The fields $Q(i)$ and $Q(\sqrt{2})$ are both contained in the field $C$.}

\textit{(a) Prove that $Q(i)$ is not contained in $Q(\sqrt{2})$}

$i$ is not present in $Q(\sqrt{2}$

\textit{(b) Prove that $Q(\sqrt{2})$ is not contained in $Q(i)$}

$\sqrt{2}$ is not present in $Q(i)$.

\subsection{}

\textit{(a) Prove that
  $$\sqrt{6} \notin \set{a + b\sqrt{2} + c\sqrt{3}: a, b, c \in Q}$$
  and conclude that this set of real numbers is not a ring }

Suppose that there are $a, b, c \in Q$ such that
$$\sqrt{6} = a + b\sqrt{2} + c\sqrt{3}$$
we follow then that
$$\sqrt{6} - a = b\sqrt{2} + c\sqrt{3}$$
$$(\sqrt{6} - a)^2 = (b\sqrt{2} + c\sqrt{3})^2$$
$$6 - 2a\sqrt{6} + a^2 = 2b^2 + 2bc\sqrt{6} + 3c^2$$
$$6  + a^2 - 3c^2 - 2b^2 = 2bc\sqrt{6} + 2a\sqrt{6} $$
$$6  + a^2 - 3c^2 - 2b^2 = \sqrt{6}(2bc + 2a) $$
lhs here is rational, and thus if $(2bc + 2a) \neq 0$ we've got a contradiction. Thus
we conclude that $2bc + 2a = 0 \ra bc + a = 0 \ra a = -bc$.

$$\sqrt{6} - c\sqrt{3}= a + b\sqrt{2} $$
$$6 + c^23 - 6c \sqrt{2}= a^2 + 2b^2 + 2ab\sqrt{2}$$
$$6 + c^23 - a^2 - 2b^2 =  2ab\sqrt{2} +  6c \sqrt{2}$$
$$6 + c^23 - a^2 - 2b^2 =  \sqrt{2}(2ab + 6c)$$
by the same logic $2ab + 6c = 0$

$$\sqrt{6} - b\sqrt{2}= a + c\sqrt{3} $$
$$6 + 2b^2 - 4b\sqrt{3}= a^2 + 3c^2 + ac\sqrt{3}$$
$$6 + 2b^2 - (a^2 + 3c^2) = ac\sqrt{3} + 4b\sqrt{3}$$
$$6 + 2b^2 - (a^2 + 3c^2) = \sqrt{3}(ac + 4b)$$
by the same logic $ac + 4b = 0$
thus

$$
\begin{cases}
  a = -bc \\
  2ab + 6c = 0 \\
  ac + 4b = 0
\end{cases}
$$
$$
\begin{cases}
  2(-bc)b + 6c = 0 \\
  (-bc)c + 4b = 0
\end{cases}
$$
$$
\begin{cases}
  -2b^2c + 6c = 0 \\
  -bc^2 + 4b = 0
\end{cases}
$$
$$
\begin{cases}
  c(-b^2 + 3) = 0 \\
  b(-c^2 + 4) = 0
\end{cases}
$$
the fact that $b \in Q$ implies that $b^2 \neq 3$, and thus $c =
0$. Therefore $-c^2 + 4 \neq 0$, which implies that $b = 0$, and thus
$a = 0$. Therefore we conclude that $a = b = c = 0$, which implies
$$a + b\sqrt{2} + c\sqrt{3} = 0$$
which is a contradiction (I might've messed up some of the
calculations here, but it has got to get the same result w/ the right
ones). More general case of this thing is somewhat whacky, maybe I'll try
to get a hold of it sometime in the future.

\textit{(b) Prove that the set
  $$Q(\sqrt{2}, \sqrt{3}) = \set{a + b\sqrt{2} + c\sqrt{3} + d\sqrt{6}, a, b, c, d \in Q}$$
  is a subring of $R$.
}

It's closed under all the respective properties and it has zeroes.

\textit{(c) Prove that $Q(\sqrt{2}, \sqrt{3})$ is a subfield of $R$}

Assume that
$$a + b\sqrt{2} + c\sqrt{3} + d\sqrt{6} \in Q(\sqrt{2}, \sqrt{3})$$
we want to find $a' + b'\sqrt{2} + c'\sqrt{3} + d'\sqrt{6} \in Q(\sqrt{2}, \sqrt{3})$ such that
$$ (a + b\sqrt{2} + c\sqrt{3} + d\sqrt{6}) * (a' + b'\sqrt{2} + c'\sqrt{3} + d'\sqrt{6}) = 1$$
moreover, we can relax this thing, to get any given number
$$ (a + b\sqrt{2} + c\sqrt{3} + d\sqrt{6}) * (a' + b'\sqrt{2} + c'\sqrt{3} + d'\sqrt{6}) = j$$
so that $j \in Q$ and $j \neq 0$
we follow that

$$(a + b\sqrt{2} + c\sqrt{3} + d\sqrt{6}) * (a' + b'\sqrt{2} + c'\sqrt{3} + d'\sqrt{6}) = $$

$$ (a a' + a b'\sqrt{2} + a c'\sqrt{3} + a d'\sqrt{6}) + $$
$$ (b \sqrt{2} a' + 2 b b' + b \sqrt{2} c'\sqrt{3} + b \sqrt{2} d'\sqrt{6}) + $$
$$ ( c\sqrt{3} a' + c\sqrt{3} b'\sqrt{2}  + 3 c c' + c\sqrt{3} d'\sqrt{6}) + $$
$$ ( d\sqrt{6} a' + d\sqrt{6} b'\sqrt{2}  + d\sqrt{6} c'\sqrt{3}  + 6 d d') = $$

$$ (a a' +     a b'\sqrt{2} +              a c'\sqrt{3} +              a d'\sqrt{6}) + $$
$$ (2 b b' +   a' b \sqrt{2} +             b  c'\sqrt{6} +             3 b  \sqrt{3} d') + $$
$$ (3 c c' +   c\sqrt{3} a' +              cb'\sqrt{6} +               3 c \sqrt{2} d') + $$
$$ (6 d d' +   d\sqrt{6} a' +              d2\sqrt{3} b' +             3 d\sqrt{2} c') = $$

$$ (a a' +     a b'\sqrt{2} +              a c'\sqrt{3} +              a d'\sqrt{6}  ) + $$
$$ (2 b b' +   a' b \sqrt{2} +             3 b d' \sqrt{3}  +           b c'\sqrt{6}  ) + $$
$$ (3 c c' +   3 c d' \sqrt{2} +           c a' \sqrt{3} +              cb'\sqrt{6}   ) + $$
$$ (6 d d' +   3 c'd \sqrt{2} +           2 db'\sqrt{3}  +            da'\sqrt{6} ) $$

thus given (presumed) linear independence of roots of integers under $Q$, we follow that we need
to find such $a', b', c', d' \in Q$ so that
$$
\begin{cases}
  ab' + a'b + 3cd' + 3c'd = 0 \\
  ac' + 3bd' + ca' + 2db' = 0  \\
  ad' + bc' + cb' + da' = 0  \\  
\end{cases}
$$
now let us move some variables around (the ones without the dash are
our given constants in this case)

$$
\begin{cases}
  b a' + a b' + 3d c' + 3c d' = 0 \\
  da' + cb' + bc' + ad' = 0 \\
  -ca' + 2db' + ac' + 3bd' = 0 \\
\end{cases}
$$

and probably this one leads to the satisfying answer TODO

\subsection{}

\textit{Let $F$ be a finite field. Prove that
  $$\prod_{\alpha \in F^*}{\alpha} = -1$$
}

Given $\alpha \in F^*$ we can follow that there is $\alpha\inv \in
F^*$ as well, but they do not need to be distinct (i.e. there's a case
when $\alpha = \alpha\inv$). Thus we can partition the whole set into
subsets, that contain one or two elements, where each subset that
contains two elements is a set of an element and its inverse.
Product of those sets is then $1$, and thus we follow that
the entire product is the product of all elements, that are equal
to its inverses.

For our not-so general case of $F_{13}$ we have a case that
the whole product can be partitioned into
$$\set{2, 7}, \set{3, 9}, \set{4, 10}, \set{5, 8},
\set{6, 11}, \set{12}$$

(needed a hint on this one)
Suppose that there is an element $q \in F$ such thaht $q^2 = 1$ and
such that $q$ is distinct from $1$ and $-1$. We follow that
$$(q - 1) (q + 1) = q^2 - 1 = 1 - 1 = 0$$
thus
$$(q - 1)( q + 1) = 0$$
this implies that $q - 1 = 0$ or $q + 1 = 0$, which implies that $q =
\pm 1$, which is a contradiction. Thus we conclude several things,
chief among them is the fact that any element that is not equal
to $\pm 1$ is distinct from its inverse. Thus we conclude by previous
point (1st paragraph) the desired result.

\textit{The rest of this section is TODO}

\section{Subfield and Extension Fields}

\subsection{}

\textit{(a) Let $K/F$ be an extension of fields. Prove that
  $$[K:F] = 1 \lra K = F$$
}

Since $K/F$ is an extension, by definition we have that $F \subseteq
K$.

Suppose that $[K:F] = 1$. This means that $dim_F(K) = 1$, and
therefore $\set{1}$ spans $K$. Thus if $k \in K$, then there is $f \in
F$ such that $f * 1 = k$, and thus $f = k$. Therefore we have that $k
\in F$, and thus $K \subseteq F$, which implies the right
side. Reverse is trivial.

\textit{(b) Let $L/F$ be a finite extension of fields, and suppose
that $[L:F]$ is a prime. Suppose further that $K$ is a fiels such that
$F \subseteq K \subseteq L$. Prove that either $K = F$ or $K = L$.}

We follow that $[L:K]$ is finite (if it's infinite, then $[L:F]$
is infinite, which is not the case), and thus
$$[L:F] = [L:K] * [K:F]$$
the fact that $[L:F]$ is prime implies that $[L:K]$ or $[K:F]$
is equal to one, which by previous point implies the desired result

\subsection{}

\textit{Let $K/F$ be a finite extension of fields. Prove that there is a
  finite set of elements $\alpha_1, ... \alpha_n \in K$ such that
  $$K = F(\alpha_1, ..., \alpha_n)$$
}

The fact that $K/F$ is finite extension by definition implies that
there is $n \in \omega$ such that
$$dim_F(K) = n$$
thus there is a basis (don't know whether or not we need AoC in this
case) $A$ of elements of $K$. Linear combination under $F$ of elements
of $A$ will span $K$ by the fact that it is the basis, which
gives us the desired conclusion.

\subsection{}

\textit{let $L/K/F$ be extensions of fields.}

\textit{(a) Assume that $L/K$ and $K/F$ are finite extensions. During
the proof of Theorem 5.18 we defined a set
  $$C = \set{a_i b_j: 1 \leq i \leq m, 1 \leq j \leq n}$$
  claimed that $C$ is a basis for $L$ as an $F$-fector space, and
  proved that $C$ is linearly independent. Prove that $C$ is a spanning
  set}

Let $l \in L$. We follow that there are $c_1, ..., c_n \in K$ such that
$$l = \sum{c_i b_i}$$
each $c_i$ is an element of $K$ and thus there are $q_1, ..., q_m \in F$
such that
$$c_i = \sum{q_{i, j} a_j}$$
and thus
$$l = \sum{\sum{q_{i, j} a_j} b_i}$$
by unfolding this sum we get the desired result.

\textit{(b) Prove that $|C| = mn$, i.e. prove that the $mn$ products
  of $a_i b_j$ are distinct.}

Assume that there are $a_i b_j$ and $a_q b_w$ such that
$a_i \neq a_q$ and $b_j \neq b_w$, but
$$a_i b_j = a_q b_w$$
One thing that we can follow is that $a_i, a_q, b_j, b_w \neq 0$ since
they are part of a basis for $L$ and $K$ respectively.
We then follow that
$$a_i a_q \inv b_j = b_w$$
which implies that $b_j$ is a nonzero multiple of $b_w$ in $L/K$,
which implies that the list is then not linearly independent, thus
making $B$ not a basis, which gives us a contradiction

\textit{(c) Prove that $L/F$ is an infinite extension if and only
  if at least one of $L/K$ or $K/F$ is
  an infinite extension}

Part (a) of this theorem is fully cooked at this point, which gives us
an easy forward implication. Thus assume that one of the $L/K$ or $K/F$
is infinite. We can then see that the proof for finite case does not
depend on finality of $A$ nor $B$, and thus we conclude that $C$
in that case is indeed a basis for $L/F$. We then see that
$$[L:F] = \dim_F(L) = |C|$$
which is still the case for infinite sets, which gives us the desired
conclusion

\textit{The rest of this section was previously attempted in the notes,
not foint to repeat it here}

\section{Polynomial Rings}

\subsection*{Notes}

Let us reiterate a couple of things here: given a ring $R$ there is a
set of finite lists of elements of $R$, whose first element is not
zero. We call those lists polynomials, and denote the set of
them by $R[x]$. Degree of a polynomial is the length of a given list
minus 1 for nonzero polynomials, and $-\infty$ for the zero
polynomial. Thus $\deg$ is a function
$$\deg: R[x] \to \omega \cup \set{-\infty}$$
for any given $R$. Purpose of including $-\infty$ to the mix is to
ensure that we've got an identity
$$\deg(f_1 f_2) = \deg(f_1) + \deg(f_2)$$
where we define addntion between a random element of omega (or
$-\infty$) with $-\infty$ in an obvious manner.

\subsection{}

\textit{Repeat of a previous exercise}

\subsection{}

\textit{Let $F$ be a field, let $f(x), g(x) \in F[x]$ be polynomials
with $g(x) \neq 0$, and suppose that there are polynomials $q_1(x),
q_2(x), r_1(x), r_2(x) \in F[x]$ satisfying
$$f(x) = g(x) q_1(x) + r_1(x)\text{ with } \deg(r_1) < \deg(g)$$
$$f(x) = g(x) q_2(x) + r_2(x)\text{ with } \deg(r_2) < \deg(g)$$
prove that $q_1(x) = q_2(x)$ and $r_1(x) = r_2(x)$. }

If $\deg(f) < \deg(g)$, then we follow that if $q_1 \neq q_2$, then one of
$q_1, q_2$ aren't zero. Assume that $q_1 \neq 0$, and then
$$\deg(g(x)q_1(x)) = \deg(g(x)) + \deg(q_1(x))$$
since $\deg(r(x)) < \deg(g(x))$, we follow that
$$\deg(g(x)q_1(x) + r(x)) = \deg(g(x)) + \deg(q_1(x)) \geq \deg(g(x)) > \deg(f(x))$$
and thus
$$g(x)q_1(x) + r(x) \neq f(x)$$
which gives us a contradiction.

Since
$$f(x) = g(x) q_1(x) + r_1(x)$$
$$f(x) = g(x) q_2(x) + r_2(x)$$
we follow that
$$g(x) q_1(x) + r_1(x) - (g(x) q_2(x) + r_2(x)) = f(x) - f(x) = 0$$
$$g(x) q_1(x) + r_1(x) - (g(x) q_2(x) + r_2(x))  = 0$$
$$g(x) (q_1(x) - q_2(x)) + (r_1(x) -  r_2(x))  = 0$$
$$g(x) (q_1(x) - q_2(x)) = r_2(x) - r_1(x)$$
thus
$$\deg(g(x) (q_1(x) - q_2(x))) = \deg(r_2(x) - r_1(x))$$
If $q_1 \neq q_2$, then we follow that  $q_1 - q_2 \neq 0$, and thus
$\deg(g(x) (q_1(x) - q_2(x))) \geq \deg(g(x))$. Degree of rhs here is
less than $\deg(g(x))$, which gives us a contradiction. Thus we
conclude that $q_1 = q_2$, which then implies that
$$r_2(x) - r_1(x) = g(x) * 0 = 0$$
which implies the desired result.

\section{Building Extension Fields}

\subsection*{Notes}

One thing that we can say is that if a polynomial has roots,
then an exercise from chapter 3 implies that it is reductible.
Equivalently irreductible polynomials dont have roots.
Reverse is not necessarily true, we can have rootless reductible
polynomials.

We also follow that if $f \in F[x]$ is of degree 1, then
we can follow that $f(x) = a_1 x + a_0$
and thus
$$a_1 x + a_0 = 0$$
$$a_1 x = - a_0$$
$$ x = - a_0 / a_1$$
and thus $f$ has roots. 

\subsection{}

\textit{Let $F$ be a field}

\textit{(a) Prove that every polynomial of degree $1$ in $F[x]$ is irreductible}

Let $g \in F[x]$ be a polynomial of degree 1. We follow that if there
are $h, k \in F[x]$ such that
$$g = h * k$$
then
$$\deg(g) = \deg(h * k) = \deg(h) + \deg(k)$$
thus
$$\deg(h) + \deg(k) = 1$$
Thus we conclude that $\deg(h) = 0$ or $\deg(k) = 0$, which implies
the desired result.

\textit{(b) Let $f \in F[x]$ be a polynomial of degree 2. Prove that
  $f(x)$ is irreductible if and only if it has no roots in $F$.}

Irreductibility implies lack of roots in general, as indicated in the
notes. Now assume that $f$ has no roots in $F$. We follow that if $f$
is reductible, then there are nonconstant $g, h \in F[x]$
such that
$$f(x) = g(x) * h(x)$$
we then follow that
$$\deg(f(x)) = \deg(g(x) * h(x)) = \deg(g(x)) + \deg(h(x)) = 2$$
Since $g, h$ are nonconstant, we follow that $\deg(g(x)) = \deg(h(x))
= 1$, which implies that they have roots, and thus $f$ has roots as
well, which is a contradiction.

\textit{(c) Let $f \in F[x]$ be a polynomial of degree 3. Prove that
  $f(x)$ is irreductible if and only if it has no roots in $F$.}

Same kind of idea as in point (b), but we follow that either $g$
or $h$ has roots.

\textit{(d) Let $f(x) = x^4 + 2$. Prove that $f(x)$ is irreductible in
  $Q[x]$.}

We can follow that if $f$ has roots, then
$$x^4 + 2 = 0$$
$$x^4 = -2$$
which is a contradiction, and thus $f$ has no roots. If $f$ is
reductible, then there are $g, h \in Q[x]$
such that
$$f(x) = g(x) h(x)$$
this implies that either $g(x)$ or $h(x)$ has degree $1$ or $2$ (other needs to
have degreee 3 or 2).

If $g(x)$ has degree 1, then it's got roots, and thus $f$ has roots,
which is not the case.

Suppose that there is $g(x)$ such that there is $h(x)$
so that
$$x^4 + 2 = g(x) h(x)$$
we follow that both $g$ and $h$ have degree 2. Assume that $g, h$ are
both monic polynomials. Then we can give indicate indices of $g$ as $a$ and
indices of $h$ as $b$, which gives us
$$(x^2 + a_1 x + a_2) * (x^2 + b_1 x + b_2) = x^4 + 2$$
$$x^4 + b_1 x^3 + b_2 x^2 + a_1 x^3 + a_1 b_1 x^2 + a_1 b_2 x + a_2 x^2 + a_2 b_1 x + a_2 b_2 = x^4 + 2$$
$$x^4 + (b_1 + a_1)x^3 + (b_2 + a_1 b_1 + a_2)x^2 + (a_1 b_2 + a_2 b_1) x + a_2 b_2 = x^4 + 2$$
$$(b_1 + a_1)x^3 + (b_2 + a_1 b_1 + a_2)x^2 + (a_1 b_2 + a_2 b_1) x + a_2 b_2 = 2$$
therefore
$$
\begin{cases}
  b_1 + a_1 = 0 \\
  b_2 + a_1b_1 + a_2 = 0 \\
  a_1 b_2 + a_2 b_1 = 0 \\
  a_2 b_2 = 2
\end{cases}
$$
$$
\begin{cases}
  a_1 = -b_1 \\
  b_2 -  b_1^2 + b_2\inv 2 = 0 \\
  -b_1 b_2 +  b_2\inv 2 b_1 = 0 \\
  a_2 = b_2\inv 2
\end{cases}
$$
$$
\begin{cases}
  a_1 = -b_1 \\
  b_2 -  b_1^2 + b_2\inv 2 = 0 \\
  b_1 (- b_2 +  b_2\inv 2) = 0 \\
  a_2 = b_2\inv 2
\end{cases}
$$
Let us concentrate on the third equation, which by the fact that $Q$ is a field,
and thus an integral domain, gives us two cases: $b_1 = 0$ or $-b_2 +
b_2\inv 2 = 0$. If we assume the former, then second equation becomes
$$b_2 + b_2 \inv 2 = 0$$
$$b_2^2 + 2 = 0$$
and no such elements of $Q$ exist. Thus we conclude the latter case, which gives us that
$$- b_2  + b\inv 2 = 0$$
$$- b_2^2  + 2 = 0$$
$$b_2^2  = 2$$
and once again, no such element of $Q$ exists, which implies that
there are no monic polynomials of degree $2$, that multiply to $f$. Now if
$g(x)$ and $h(x)$ are arbitrary polynomials of degree 2, that multiply to $f$,
then we conclude that coefficients for $x^2$ in both $g$ and $h$ has got to
multiply to $1$. Thus we conclude that multiplying both polynomials by
inverses of their coefficients for $x^2$ gives us a pair of monic
polynomials, which gives us a hasty contradiction, and thus implies the desired
result.

One last thing to point out: if we generalize the case, and include 


\textit{(e) Let $f = x^4 + 4$. Prove tht $f$ is reductible in $Q[x]$,
  despite the fact that it has no roots in $Q$.}

$$x^4 + 4 = (x^2 + 2x + 2) * (x^2 - 2x + 2)$$

\subsection{}

\textit{Let $F$ be a field, and suppose that the polynomial $X^2 + X
  + 1$ is irreductible in $F[X]$. Let
  $$K = F[X] / (X^2 + X + 1)F[X]$$
  be the quotient ring.}

\textit{(a) Find a polynomial $p(X) \in F[x]$ of degree at most 1 such that
  $$\overline{p(X)} = \overline{(X + 3)} * \overline{(2X + 1)}$$
}

We follow that
$$(X + 3) * (2X + 1) = 2X^2 + 7X + 3$$
we then want to get an element of $(X^2 + X + 1)F[X]$,
whose sum with given element is of power of 1 or less. This is easy,
we can get
$$2(X^2 + X + 1) = 2X^2 + 2X + 21$$
and then substract one from another to get
$$2X^2 + 7X + 3 - 2X^2 - 2X - 2 = 5X + 1$$

\textit{(b) Find a polynomial $q(X) \in F[X]$ of degree at most 1 satisfying
  $$\overline{q(X)} * \overline{X + 1} = \overline{1}$$
}

Exercise says that we can get an element of degree 1 in here, so let's
try that
$$x * (x + 1) = x^2  + x$$
initial guess wont do, so let us try dividing the initial polynomial
with the remainter
$$(x^2 + x + 1) = x(x + 1) + 1$$
thus we might get a guess of plugging $-x$, which gives us
$$(-x) * (x + 1) = -x^2 - x \equiv 1$$
as desired. Here we might try to develop a proper way of finding
multiplicative inverses, but I'm too eager to mmove forward to
actually do that


\textit{(c) Find a polynomial $r \in F[x]$ of degree at most 1 satisfying
  $$\overline{r}^2 = \overline{- 3}$$
}

We can follow that
$$\overline{- 3} = \overline{x^2 + x + 1 - 3} =
\overline{x^2 + x - 2} = \overline{(x - 1) * (x + 2)}$$
we then try to plug in some coefficients into $x^2 + x + 1$
so that its sum with $-3$ is a square. We follow that
$4$ is such a number:
$$4x^2 + 4x + 1 = (2x + 1)^2$$
which in turn implies that
$$\overline{(2x + 1)}^2 = \overline{4x^2 + 4x + 1} = \overline{-3}$$
as desired.

\subsection{}

\textit{Let
  $$f(x) = \sum_{0 \leq i \leq d}{c_i x^i} \in Z[x]$$
  be a polynomial of degree $d \geq 1$}

\textit{(a) Prove that if $a/b \in Q$ is a root of $f(x)$, written in lowert terms,
  then $a | c_0$ and $b | c_d$}

We follow that
$$\sum_{0 \leq i \leq d}{c_i x^i}  = c_0 + \sum_{1 \leq i \leq d}{c_i x^i} $$
and thus
$$f(a/b) = c_0 + \sum_{1 \leq i \leq d}{c_i (a/b)^i} =
c_0 + \sum_{1 \leq i \leq d}{c_i a^i b^{-i}} =
c_0 + a \sum_{1 \leq i \leq d}{c_i a^{i - 1} b^{-i}}$$
since $a/b$ is a root of $f(x)$ we follow that
$$f(a/b) = c_0 + a \sum_{1 \leq i \leq d}{c_i a^{i - 1} b^{-i}}$$
$$0 = c_0 + a \sum_{1 \leq i \leq d}{c_i a^{i - 1} b^{-i}}$$
$$-c_0 = a \sum_{1 \leq i \leq d}{c_i a^{i - 1} b^{-i}}$$
$$-c_0 b^d = a b^d \sum_{1 \leq i \leq d}{c_i a^{i - 1} b^{-i}}$$
$$-c_0 b^d = a \sum_{1 \leq i \leq d}{c_i a^{i - 1} b^{d - i}}$$
by our multiplication by $b^d$ we got rid of elements of $Q$, and now
this whole shebang is in $Z$.  We then follow that since $a/b$ are in
their lowest terms that they dont have any common primes as multiples,
which implies that
$$a | c_0$$
as desired.

We can then do similar trich with $c_d$ to get 
$$0 = f(a / b) = \sum_{0 \leq i \leq d}{c_i a^i b^{-i}} =
c_d b^{-d} +  \sum_{0 \leq i \leq d}{c_i a^i b^{-i}}$$

$$0 = c_d a^d b^{-d} +  \sum_{0 \leq i \leq d - 1}{c_i a^i b^{-i}}$$
$$0 = c_da^d  +  b^d \sum_{0 \leq i \leq d - 1}{c_i a^i b^{-i}}$$
$$-c_da^d =  b^d \sum_{0 \leq i \leq d - 1}{c_i a^i b^{-i}}$$
$$-c_da^d =  b \sum_{0 \leq i \leq d}{c_i a^i b^{d - 1 - i}}$$
which also implies the desired result.

\textit{(b) Use (a) to find the small finite set of integers $A$ and $B$ so that
  all of the rational roots of the polynomial
  $$g(x) = 45x^4 + 62x^3 + 56x^2 + 11x - 6$$
  are in the set
  $$\set{\pm \frac a b : a \in A, b \in B}$$}

Part (a) here gives us the idea that we can get a total set of
divisors of the first and the last coefficients of any given
polynomials, then get a set of fractions out of those two sets, and
this set in the end will contain all of the rational roots of the
given polynomial. I've written the desired program (polyq.py in
progs/, bruteforced a couple of things), and the result is pretty fun

\subsection{}

\textit{Let $F$ be a field, and let $f(x, y) \in F[x, y]$ be a
  polynomial in two variables. For example, you could take $f$ to be
  irreductible}

Definitions of irreducibility (and so on) are transferring nicely
to two variables. On top of that, polynomials of one variable can be
viewed as a polynomial in two variables, which allows us to conclude
that for example $(x + 1)$ is an irreductible polynomial of two
variables.

Definitions of degrees is kinda non-applicable, but the fact that we
include only non-constant polynomials are applicable in this case

\textit{(a) Suppose that there are values $a, b \in F$ such that $f(a,
  b) = 0$. Let $I$ be the ideal
  $$I = \set{(x - a) g(x, y) + (y - b) h(x, y): g, h \in F[x, y]}$$
  Prove that
  $$f(x, y)F[x, y] \subset I \subset R$$
  (proper subsets) and conclude that the principal ideal $f(x, y)F[x,
  y]$ generated by $f(x, y)$ is not maximal }


We can follow that if $I$ contains $f$, then it contains $f(x, y)F[x,
y]$ properly by the fact that $I$ is an ideal (proof of the fact that
$I$ is an ideal is kind of trivial).

We can follow that since $f$ is a polynomial implies that there is a
maximal power of either $x$ or $y$ that we're going to denote by $d$,
and thus we can set
$$f(x, y) = \sum_{0 \leq i, j \leq d}{c_{i, j} x^i y^j} $$
from the fact that
$$x^n - c^n = (x - c) \sum_{0 \leq i \leq n - 1}{x^i c^{n - 1 - i}}$$

$$f(x,y) = f(x, y) - 0 = f(x, y) - f(a, b) = $$
$$ \sum_{0 \leq i, j \leq d}{c_{i, j} x^i y^j}  - \sum_{0 \leq i, j \leq d}{c_{i, j} a^i b^j} =
\sum_{0 \leq i, j \leq d}{c_{i, j} (x^i y^j - a^i b^j)}= $$
TODO

\textit{the rest of this section is marked as TODO}

\section{Finite Fields}

\subsection{}

\textit{Let $F$ be a finite field with $q$ elements}

\textit{(a) Prove taht every non-zero element of $F$ is a root of the
polynomial $x^{q - 1} - 1$}

We can follow that $F^*$ is an abelian group under muultiplication,
and since it is finite, we follow that for $a \in F: a \neq 0$
its order $m$ divides the order of $F^*$, which is precicely $q - 1$.
We then follow that
$$a^{q - 1} = (a^m){(q - 1) / m} = 1^{(q - 1) / m} = 1$$
which implies the desired result.

\textit{(b) Prove that every element of $F$ is a root
  of the polynomial $x^q - x$}

We follow that
$$x^q - x = x(x^{q - 1} - 1)$$
and thus its zeroes are all nonzero elements, and zero
(i.e. the whole field)

\textit{(c) Prove the formula $$\prod_{a \in F}{(x - a)} = x^q - x$$ }

Since every $a \in F$ is a root of $x^q  - x$ we follow that
it can be deconstructed into lhs (exercise 5.1(b) directly implies it)


\subsection{}

\textit{Let $m$ and $n$ be positive integers}

\textit{(a) Prove that $x^m - 1$ divides $x^n - 1$ in $F[x]$ if and
  only if $m$ divides $n$ in $Z$.}

Suppose that $m$ divides $n$. Definition states that there exists
$k \in N_+$ such that $n = mk$. Thus
$$x^n - 1 = x^{mk} - 1$$
and thus
$$(x^m - 1) * \sum_{0 \leq i \leq k - 1}{x^{im}} =
\sum_{0 \leq i \leq k - 1}{x^{im} * (x^m - 1)} =
\sum_{0 \leq i \leq k - 1}{x^{(i + 1)m} - x^{im}} =
$$
$$ =
\sum_{0 \leq i \leq k - 1}{x^{(i + 1)m}} - \sum_{0 \leq i \leq k - 1}{x^{im}} =
\sum_{1 \leq i \leq k}{x^{im}} - \sum_{1 \leq i \leq k - 1}{(x^{im})} - 1 = $$
$$ =
x^{km} + \sum_{1 \leq i \leq k - 1}{x^{im}} - \sum_{1 \leq i \leq k - 1}{(x^{im})} - 1 =
x^{km} - 1
$$
as desired (deriviation of the desired polynomial was done by
observing polynomial division for lower powers, i.e. $x^6 - 1$ and
$x^2 - 1$, and then generalizing the hypothesis).

Now assume that $x^m - 1$ divides $x^n - 1$. Thus there is a polynomial $h(x)$
such that
$$(x^m - 1) * h(x) = x^n - 1$$
We follow that there is $g(x)$ such that
$$(x^m - 1) * (x^{n - m} + g(x)) = x^n - 1$$
$$(x^m - 1) * (x^{n - m} + g(x)) = x^n - 1$$
$$x^n + x^mg(x) - x^{n - m} - g(x)  = x^n - 1$$
$$x^mg(x) - x^{n - m} - g(x)  =  - 1$$
$$(x^m - 1)g(x) = x^{n - m}  - 1$$
and this is the moment where induction should kick somewhere here.

Our hypothesis is that if $x^m - 1$ divides $x^n - 1$, then for all $k
\in N_+$ such that $mk \leq n$ we have that $x^m - 1$ divides $x^{n -
mk} - 1$. Base case is provided, and the inductive step is not much
different from the base case, which gives us the desired conclusion. We
then can state that if $m$ does not divide $m$, then there is $k \in
N_+$ and $0 < r < m$ such that
$$n = km + r$$
and thus
$$x^{n} - 1 = x^{km + r} - 1$$
our little lemma gives us that since $x^m - 1$ divides $x^n - 1$
that $x^m - 1$ divides $x^{r} - 1$, which is impossible by the
degrees rule, which gives us the desired contradiction.

\textit{(b) Let $a \geq 2$ be an integer. Prove that $a^m$ divides
  $a^n - 1$ in $Z$ if and only if $m$ divides $n$ in $Z$.}

Proof is similar to the proof of previous point, where we swap
appropriate things.

\textit{The rest of this section is marked as TODO}

\chapter{Groups - Part  2}

\section{Normal Subgroups and Quotient Groups}

\subsection{}

\textit{Let $\psi: G \to G'$ be a homomorphism of groups.}

\textit{(a) Prove that the image of $\psi$ is a subgroup of $G'$}

We can follow that if $h', g' \in \psi[G]$, then there are by
definition $h, g \in G$ such that
$$\psi(h) = h', \psi(g) = g'$$
thus
$$\psi(h\inv g) = \psi(h\inv) \psi(g) = \psi(h)\inv \psi(g) = h'\inv g$$
as desired.

\textit{(b) Suppose that $G$ is a finite group. Prove that
  $$|G| = |\ker(\phi)| * |\phi[G]|$$}

Let $a * \ker(\phi)$ be a left coset of $G$. We follow that if $q \in
a * \ker(\phi)$, then there is $b \in \ker(\phi)$ such that
$$q = a * b$$
and thus
$$\phi(q) = \phi(a * b) = \phi(a) * \phi(b) = \phi(a) * 1 = \phi(a)$$
Thus we can follow that different element of the same cosets are sent
to the same element of $G'$ , which implies that there is a bijection
between a set of cosets, and elements in the range of $\phi[G]$.
Thus there is by definition the same number of elements of $\phi[G]$
and cosets of $\ker(\phi)$. Then Lagrange's theorem implies the desired
conclusion

\subsection{}

\textit{Let $G$ be a group, and let $H \subseteq G$ be a subgroup.}

\textit{(a) Suppose that $g\inv H g \subseteq H$ for all $g \in G$.
  Prove that $H$ is a normal subgroup of $G$.}

Fix $g \in G$ and let us assume that
$$g\inv H g \subseteq H$$
we can also follow that by our assumptions we've got that
$$g H g\inv \subseteq H$$

Now let $h \in H$. We want to prove that there is $h' \in H$
such that
$$h = g\inv h' g$$
we follow that
$$h' = g h g\inv$$
thus $h' \in g H g\inv$, and since $g H g\inv \subseteq H$ we follow that
$h' \in H$, as desired.

\textit{(b) Let $g \in G$. Prove that $g\inv H g$ is a subgroup of $G$}

Since $e \in H$, we follow that
$$e = g\inv g = g\inv e g \in h\inv H G$$
We can also follow taht if $q, w \in g\inv H g$, then
there are $h, h' \in H$ such that
$$q = g\inv h g, w = g\inv h' g$$
then
$$q\inv w = g\inv h g g\inv h' g = g\inv h h' g$$
and since $h h' \in H$ we conclude that $q\inv w \in g\inv H g$, thus making
$g\inv H g$ a subgroup

\textit{(c) Let $g \in G$. Prove that the following map is a group
  isomorphism:
  $$H \to g\inv H g, h \to g\inv h g$$
}

Multiplication by an element is injective, and definition of $g\inv H
g$ implies that the map is surjective, thus making the map bijective.

We can also see that
$$\psi(a b) = g\inv a b g = g \inv a e b g = g \inv a g g\inv  b g = \psi(a) \psi(b)$$
as desired

\textit{(d) If $H$ is finite. prove that every conjugate of $H$ has
  the same number of elements as $H$}

Bijection in the previous point implies that

\subsection{}

\textit{Skip}

\subsection{}

\textit{Prove that every subgroup of the quaternion group is normal}

Quaternion group has $8$ elements, which implies that its only
subgroups are of order $1, 2, 4, 8$. We then follow that $1$ and $8$
are trivial and the entire group respectively, the only group of order
$2$ is $\set{\pm 1}$, and thus the only order that is left is $4$.  We
then follow that the only groups of order $4$ are in the form
$$\set{\pm 1, \pm x}$$
where $x \in \set{i, j, k}$. After this we can proceed with a simple
enumeration, that we're gonna skip, which implies the desired result

\subsection{}

\textit{Let $G$ be a group and let $H \subseteq G$ be a subgroup of
  $G$ of index 2.  Prove that $H$ is a normal subgroup of $G$.}

Suppose that there is an element $g \in G$ such that there exists $h
\in H$ so that $g\inv h g \notin H$. We then follow that if
$g\inv h \in H$, then $g\inv \in H$, and thus $g \in H$
and $g\inv h g \in H$, which is a contradiction, and thus
$g, g\inv \notin H$. This implies that $g\inv H, g H \neq H$,
and thus we follow that $g\inv h g \in gH$. Thus
there is $h' \in H$ such that
$$g\inv h g = g\inv h'$$
$$h g = h'$$
$$g = h\inv h'$$
which implies that $g \in H$, which is a contradiction. Thus
we conclude that there is no element $g \in G$ such that
there is $h \in H$ so that $g\inv h g \notin H$. This implies that
for all $g \in G$ and all $h \in H$ we have that 
$$g\inv h g \in H$$
which implies that
$$g\inv H g \subseteq H$$
which implies the desired result. (did this one een without a hit,
what a rush)

\subsection{}

\textit{Let $G$ be a group, let $H \subseteq G$ and $K \subseteq G$
  be subgroups, and assume that $K$ is a normal subgroup of $G$}

\textit{(a) Prove that $HK = \set{hk: h \in H, k \in k}$ is a subgroup of $G$}

We can follow that $e \in H$ and $e \in K$, and thus $e = e * e \in HK$.
Now let $q, w \in HK$. We folllow that they can be deconstructed to
$q = hk, w = h' k'$. We then conclude that 
$$q w\inv  = hk (h' k')\inv = h k k' \inv h'\inv = h e k k'\inv h'\inv  =
h h'\inv h' k k'\inv h'\inv = h h'\inv (h' k k'\inv h'\inv)$$
We then can see that $h h'\inv \in H$ and by the fact that $K$ is normal
we follow that $(h' k k'\inv h'\inv) \in K$, which implies that the whole thing
is in $HK$, as desired.

\textit{(b) Prove that $H \cap K$ is a normal subgroup of $H$ and that $K$
  is a normal subgroup of $HK$}

We can follow that $H \cap K$ is a subgroup of $H$ by properties
of subgroups.
Let $h \in H$ and $q \in H \cap K$ be arbitrary. We follow that
$h\inv q h \in H$ by the fact that it's a product of elements of $H$,
and $h\inv q h \in K$ by the fact that $K$ is normal subgroup of $G$.
Thus $h\inv q h \in H \cap K$, which means that $H \cap K$ is a
normal subgroup of $H$, as desired

$K$ is a normal subgroup of $G$, and thus it's a normal subgroup of any
subgroup that contains it by a pretty straightforward application
of definition.

\textit{(c) Prove that $HK/K$ is isomorphic to $H/(H \cap K)$}

We can follow that $f_1: H \to HK$ defined by identity is a homomorphism,
and since $K$ is a normal subgroup of $HK$ we can follow there is
a homomorphism $f_2: HK \to HK/K$. Thus there is a homomorphism
$f_3: H \to HK/K$. We follow that if
$q \in H \cap K$, then $q \in K$, and thus
$$f_3(q) = qK = K = f_3(0)$$
If $w \in H \setminus K$,then
$$f_3(w) = wK \neq K = f_3(0)$$
which implies that $H \cap K$ is a kernel of $f_3$. Thus we can define
$f_4: H/(H \cap K) \to HK / K$, which will be an injective homomorphism.
We can also follow that if $C \in HK/K$, then there is $q \in HK$
where $q = hk$ such that
$$C = qK = hkK = hK = f_4(h(H \cap K))$$
which implies that $f_4$ is surjejctive, and thus it is bijective,
which implies the desired result.

\textit{(d) Skip}

\subsection{}

\textit{Let $G$ be a group, let $K \subseteq H \subseteq G$ be
  subgroups, and assume that $K$ is normal.}

\textit{(a) Prove that $H/K$ is naturally a subgroup of $G/K$}

We can follow that if $C \in H/K$, then there is $h \in H$ such that
$C = hK$, and since $H \subseteq G$ we follow that $C \in G/K$, which
implies that $f: H/K \to G/K$ defined as an identity with the expanded
codomain is an injective homomorphism

\textit{(b) Conversely, prove that every subgroup of $G/K$ looks like
  $H/K$ for some subgroup $H$ satisfying $K \subseteq H \subseteq G$}

Let $Q \subseteq G/K$ be a subgroup. We follow that each $q \in Q$ is
of the form $q = gK$ for some $g \in G$.  We then can unionize $Q$ in
order to get
$$H = \bigcup Q$$
we then follow that if $w, r \in H$, then there are cosets $wK, rK \in
Q$, and since $Q$ is a group there is $w\inv in K$.  Thus
$$w\inv K r K = w\inv r K \in Q$$
this implies that $w\inv r K \subseteq H$, and thus
$$w\inv r e = w\inv r \in H$$
which implies that $H$ is a subgroup of $G$. We also follow that since
$Q$ is a group that $eK = K \in Q$, which implies that $K \subseteq
H$. Then we follow that if $q \in Q$, then there for all $t \in q$ we
have that $q = tK$. Thus $t \in H$ and therefore $tK \in H/K$, which
implies that $Q \subseteq H/K$. Similar argument shows the inverse
inclusion, which implies the desired result.

\textit{(c) Prove tha $H$ is a normal subgroup of $G$ if and only if
  $H/K$ is a normal subgroup of $G/K$}

Assume that $H$ is normal, and let $gK \in G/K, hK \in H/K$. We follow that
$$g\inv K h K g K = (g\inv h g) K$$
since $h$ is normal we follow that $(g\inv h g) \in H$ and this implies that
$H/K$ is normal.

Assume that $H/K$ is normal. We follow that if $g \in G, h \in H$, then
$$g\inv h g \in g\inv h g H = g\inv H h H g H = g\inv H g H = H$$
which implies that $H$ is normal

\textit{(d) If $H$ is a normal subgroup of $G$, prove that
  $$\frac{G/K}{H/K} \cong G/H$$}

There's a surjective homomorpshism $f_1: G \to G/H$ with kernel $H$, and
since $K \subseteq H$ we follow that there is a surjective
homomorphism $f_2: G/K \to G/H$.

Let $q \in G/K$.  If $q \in H/K$, then $q = hK$ for some $h \in H$ and thus
$$f_2(q) = f_2(hK) = f_1(h) = e$$
if $q \notin H/K$, then there is $g \in G \setminus H$ such that $q = gK$. Thus
$$f_2(q) = f_2(gK) = f_1(g) \neq 0$$
which implies that $H/K$ is the kernel of $f_2$. This implies that
there is an isomorphism $f_3: \frac{G/K}{H/K} \to G/H$, which implies
the desired result.

\subsection{}

\textit{Skip}

\section{Group Acting on Sets}

\subsection*{Notes}

Initial definition of the group action is a tad bit lacking, by action
we denote a function $\cdot: G \times A \to A$ with all the axioms
attached

Then, I want to follow some things. If $x, y \in A$ and $g \in G$,
then we follow that
$$gx = gy$$
implies that
$$g\inv (g x) = g\inv (gy)$$
$$(g\inv g) x = (g\inv g)y$$
$$e x = e y$$
$$x = y$$
which implies that with fixed $g \in G$, our initial map is injective.

If $x \in A$ is fixed, then if $g_1, g_2 \in G$ we have
$$g_1 x = g_2 x$$
$$g_1 \inv g_1 x = g_1 \inv g_2 x$$
$$e x = (g_1 \inv g_2) x$$
and we might not have cancelation properties here, so we're assuming
that injectivity is not implied here.

Orbit is the image of an element of $X$, and the stabilizer is
the values of $G$ that leave particular $x \in X$ unchanged. 
Thus orbit is a subset of $X$ and stabilizer is a subset of $G$.
Both of them are indexed by a particular $x \in X$.

Now I want to re-word the proof of 6.19(b):

With fixed $x \in X$ our initial map becomes a function of $G$,
whose range is the orbit of $X$. A union of the set of all orbits
constitutes $X$ (we have that $e * x = x$, which implies this).
And on top of that, if there are two elements $x, y$ such that
$$Gx \cap Gy \neq \emptyset$$
then there are $g_1, g_2 \in G$ such that
$$g_1 x = g_2 y$$
$$ x = g_1 \inv g_2 y$$
$$ x = (g_1 \inv g_2) y$$
which means that $x \in Gy$ and $y \in Gx$.

If $q \in Gx$, then there is $g' \in G$ such that
$q = gx \lra x = g\inv q$, and then we follow that if $r \in Gq$, then
$$(\exists g \in G) (r = gq) \ra (\exists g, g' \in G) (r = gg'x) \ra r \in Gx$$
thus $Gq \subseteq Gx$. By the similar logic if $r \in Gx$ then
$$(\exists g \in G) (r = gx) \ra (\exists g, g' \in G) (r = gg'q) \ra r \in Gq$$
which implies that $Gq = Gx$.

Combining our discovery into the previous point, we get that
$$Gy = Gx$$
which means that orbits of elements of $X$ constitute partition of $X$
This gives us a thing, that is similar to cosets: they can be indexed
by any on their elements. Moreover, if we have $x \in X$, then
$ex = x  \in Gx$, which implies that $x \in Gy \iff Gy = Gx$

Let us also re-word the 6.19(c): \textit{If $G$ is a group that acts
on $X$, then for every $x \in X$ there is a bijection
$$\alpha: G/G_x \to Gx$$
such that
$$(\forall C \in G/G_x): (\forall g \in C)(\alpha(C) = gx)$$}

\textit{Proof: } If $C \in G/G_x$ and $g, g' \in C$, then by definition of
the coset for all $q \in C$ we have $C = qG_x$. Thus
$C = gG_x = g'G_x$. This implies that there is $h \in G_x$
such that
$$g = g' h$$
and therefore
$$gx = g' h x = g' (h x) =  = g'x$$
where $hx = x$ is derived from the definition of $G_x$.
This implies that
$$(\forall C \in G/G_x)(\forall g \in C)(\exists !  y \in X)(gx = y)$$
which means that there is a function $\beta: G/G_x \to X$ that satisfies
$$(\forall C \in G/G_x)( \forall g \in C)(\beta(C) = gx)$$
Furthermore, since for every $C \in G/G_x$ we have that $\beta(C) = gx$
for some $g \in G$ we can follow that $\beta[G/G_x] \subseteq Gx$,
and thus we can restrict its codomain to $\beta: G/G_x \to Gx$.

We then follow that if $q \in G_x$, then $qx = x$, and thus
$$\beta(G_x) = x$$
Moreover, if $g \in G$, then $g \in gG_x$, and thus by restriction on $\beta$
we have that
$$\beta(gG_x) = gx$$
thus if $y \in Gx$, then there is $g \in G$ such that $y = gx$ by definition,
and thus
$$\beta(gG_x) = gx = y$$
which implies that $\beta$ is surjective.

Let  $C_1, C_2 \in G/G_x$ and assume that $\beta(C_1) = \beta(C_2)$.
We follow then that for all $g \in C_1$ and $g' \in C_2$ we have
$$gx = g'x$$
by definition of $\beta$. This gives us
$$x = g\inv g'x \ra g\inv g' \in G_x$$
Thus there is $r \in G_x$ such that $g\inv g' = r$, and therefore
$g' = gr$. Therefore we conclude that
$$C_1 = g'G_x = gr G_x = g G_x = C_2$$
which implies that $\beta$ is injective, and thus we have the desired
bijection.

\subsection{}

\textit{Let $G = C_n$ be a cyclic group of order $n$, and let $X =
\set{x_1, ..., x_n}$ be a set containing $n$ elements. For each part,
find an action of $G$ on $X$ that there is an element $x \in X$ having
the indicated property}

Firstly, for convenience's sake we note that every cyclic group of
order $n$ is isomorphic to $Z/nZ$, and thus we're gonna take
$G = Z/nZ$. We're also gonna take $X = Z/nZ$ for some easy notation.

\textit{(a) The stabilizer is $G_x = \set{e}$}

We can set $\times$ to be normal $+$, which gives this thing trivial.

We can also see that $|G| / |G_x| = |Gx|$, which knocks out point (d)

\textit{(b) The stabilizer is $G_x = G$}

We can set
$$j \times x_i = x_i$$
same idea applies to point (c)

\subsection{}

\textit{Let $G$ be a group, let $X$ be a set, and let $S_X$ be the
  symmetry group of $X$. Let
  $$\alpha: G \to S_X$$
  be a function, and for $g \in G, x \in X$, let $g * x = \alpha
  (g)(x)$. Prove that this defined a group action if and only if $\alpha$
  is a group homomorphism.}

Suppose that this thing defines a group action. This means that
we can define a function $\times: G \times X \to X$ by
$$g \times x = \times(g, x) = \alpha(g)(x)$$
with properties that
$$e \times x = \times (e, x) = x$$
and
$$(g * g')  \times x = \times(g * g', x) = \alpha (g * g')(x)
= \times(g, \times (g', x)) = \alpha(g)(\alpha(g')(x))$$

Thus for all $g, g' \in G, x \in X$
$$ \alpha(g  g')(x) = (g g') \times x =
g (g' \times x) = \alpha(g)( \alpha(g')(x))$$
which gives us
$$\alpha(g * g') = \alpha(g) \cdot \alpha(g')$$
which proves that this thing is a homomorpism.

Now if $\alpha$ is a homomorphism, then
$$\alpha(e) = Id$$
and thus
$$\alpha(e)(x) = x$$
thus satisfying the first condition. We also have the same
deriviation, but in reverse for the rest of conditions.

\subsection{}

\textit{(a) Prove that $G$ acts transitively in $X$ if and only there
  is at least one $x \in X$ such that $Gx = X$.}

Forward implication is handled by definition.

Suppose that there exists $x \in X$ such that $Gx = X$. This means
that for all $y \in X$ there is $g \in G$ such that $gx = y \lra x = g\inv y$.
Thus we follow that if $j \in X$, then there is $g' \in G$ such that
$g'x = j$, and therefore $g' g\inv y = j$, thus implying that $j \in
Gy$, thus implying that $X \subseteq Gy$, which gives us the desired result.

\textit{(b) Prove that $G$ acts transitively on $X$ if and only if for
  every pair of elements $x, y \in X$ there exists a group element $g \in G$
  such that $gx = y$.}

If $G$ acts transitively, then if $x, y \in X$ we follow that $y \in X = Gx$,
and thus there is $g \in G$ such that $y = gx$.

We then follow that if for all $x, y \in X$ there is $g \in G$ such that
$gx = y$ then by fixing $x$ and shoving $y$ into the thing we get a
surjective function from $Gx$ to $G$, which implies the desired result.

\textit{(c) If $G$ acts transitively in $X$, prove that $|X|$ divides
  $|G|$.}

Probably we're assuming here that everything is finite.

If $G$ acts transitively, then $X = Gx$ for all $x \in X$, and thus
$$|X| = |Gx| = |G| / |G_x|$$
which implies the desired result.

\subsection{}

\textit{Let $G$ be a group that acts on a set $X$. We say that the
  action is doubly transitive if it has the following property: }

\textit{For all $x_1, x_2, y_1, y_2 \in X$ with $x_1 \neq x_2$ and
  $y_1 \neq y_2$ there exists an element $g \in G$ of the group
  satisfying $g x_1 = y_1$, $g x_2 = y_2$. }


\textit{(a) Let $Z$ be the following set of ordered pairs:
  $$Z = \set{(z_1, z_2) \in X \times X: z_1 \neq z_2}$$
  Let $G$ act on $Z$ by the rule
  $$g(z_1, z_2) = (gz_1, gz_2)$$
  Prove that the action of $G$ on $X$ is doubly transitive if and only
  if the action of $G$ on $Z$ is transitive.
}

Suppose that the action on $X$ is doubly transitive. Fix $q = (q_1,
q_2) \in Z$. Since $q \in Z$ we follow that $q_1 \neq q_2$.  Let $z
\in Z$. We follow that $z = (z_1, z_2)$ for some $z_1, z_2 \in X$ such
that $z_1 \neq z_2$. We then follow that for $q_1, q_2 \in X$ there is
$g \in G$ such that $gq_1 = z_1$ and $gq_2 = z_2$.  Thus
$g(q_1, q_2) = (z_1, z_2)$ and thus $z \in Gq$. Since $z$ is arbitrary
we conclude that $Z \subseteq Gq$, which implies the desired result.
Running about the same result but in reverse gives us the reverse
implication.

\textit{(b) skip}

\section{The Orbit-Stabilizer Counting Theorem}

\subsection{}

\textit{Let $G$ be a group, and let $X$ be a set on which $G$ acts.}

\textit{(a) Suppose that $|G| = 15$ and $|X| = 7$. Prove that there is
  some element of $X$ that is fixed by every element of $G$.}

OSCT states that
$$|X| = \sum_{i = 1}^{k}{\frac{|G|}{|G_{x_i}|}}$$
elements in the sum in rhs are nonzero divisors of $15$ (by the fact
that $|G| = 15$), of which there are $1, 3, 5$. The only way that we
can have them sum up to $|X| = 7$ is
$$3 + 3 + 1$$
or
$$5 + 1 + 1$$
which implies that there's gotta be an element $x_i$ such that
$|G_{x_i}| = |G|$, which implies that there is $x_i$ such that
$G_{x_i} = G$, which implies the desired result.

\textit{(b) What goes wrong with your proof in (a) if $|G| = 15$ and
  $|X| = 6$ or $|X| = 8$?}

The fact that then we can sum up the divisors into this number:
$$6 = 3 + 3$$
$$8 = 5 + 3$$

\subsection{}

\textit{Skip}

\subsection{}

\textit{Let $G$ act on itself by conjugation as described in 6.25.
  Let $x \in G$. The conjugacy class of $x$ is the orbit of $X$ for this
  action.}

\textit{(a) Prove that $G$ is the disjoint union of its conjugacy classes.}

We can follow that for every $g \in G$ we have $g = e g e$, thus
implying that every element of $G$ is in some conjugacy class, which
implies the union part.

We then follow that if $C$ is a conjugacy class, and  $q \in C$,
then there is $g, g' \in C$ such that $g$ generates this class,
and $q = g' g g'\inv$. Thus $g = g'\inv q g'$, and if $w \in C$,
then there is $r \in G$ such that
$$w = r g r' = r g'\inv q g' r\inv = r g'\inv q (r g'inv)\inv $$
which implies that $w$ is in conjugacy class of $q$, and thus $C$
is a subset of the conjugacy class of $q$. Reverse inclusion is similar
to this one, which implies that
$$C_1 \cap C_2 \neq \emptyset \ra C_1 = C_2$$
which implies the desired result.

\textit{(b) Prove that $G$ is abelian if and only if each conjugacy class
  of $G$ contains a single element.}

If $G$ is abelian, then $g q g\inv = g g\inv q = q$ for all $g \in G$,
which implies that conjugacy classes are singletons.

If $G$'s conjugacy classes are singletons, then for every pair $g, q \in G$
we have
$$g q g\inv = q$$
(the only element in conjugacy class is $q$ since $e \in G$).
This implies that
$$gq = qg$$
which implies the desired result.

\textit{Skip the rest}

\subsection{}

\textit{Let $p$ be a prime. We proved in 6.26 that a group with $p^2$
  elements must be abelian. Let $G$ be a group with $p^3$ elements}

\textit{(a) Mimic the proof of corollary 6.26 to try to prove that $G$ is
  abelian. Where does the proof do wrong?}

We firstly can show that $p$ divides $|Z(G)|$, since nothing else in
that part of the proof is dependent on the square. We then follow that
$$G/Z$$
does not have to be cyclic, which diverts from our initial proof.


\textit{(b) Give two examples of non-abelian groups with $2^3$ elements}

$Q$ and $D_4$.

\textit{(c) What sort of information about $G$ can you deduce from the
  proof in (a) that failed?}

That $G/Z(G)$ is an abelian group (since its order is $p^2$)

\subsection{}

\textit{Let $G$ be a group, and let $a, b \in G$. We say that $a$
commutes with $b$ if $ab = ba$. Is "commutes with" an equivalence
relation}

By direct implication of valious definitions, "commutes with"
is reflexiva and symmetric, but I think that transitiveness
is not exaclty covered here. For example $a$ commutes with $e$
and $e$ commutes with $b$ does not imply that $a$ commutes with
$b$ (e.g. $i * 1$, $1 * k$, $ik \neq ki$ in $Q$)

\subsection{}

\textit{Let $G$ be a group.}

\textit{(a) Prove that the center $Z(G)$ of $G$ is a normal subgroup of $G$.}

We follwo that $e$ commutees with evverything, and thus $e \in Z(G)$.
Let $g, h \in Z(G)$, then let $q$ be arbitrary. We follow that
$$q g\inv = (g\inv g )q g\inv = g\inv (g q) g\inv = g\inv qg  g\inv = g\inv q$$
which gives us that $g\inv in Z(G)$. We also have that
$$ghq = gqh = qgh$$
thus implkying that $Z(G)$ is a subgroup, as desired.

Now assume that $g \in G, z \in Z(G)$. We follow that
$$g\inv z g = g\inv g z = z \in Z(G)$$
which implies that $Z(G) \subseteq g\inv Z(G) g$, thus making $Z(G)$
normal, as desired

\textit{(b) Let $H \subseteq G$ be a subgroup of $G$. Prove that
  the centralizer $Z_G(H)$ is a subgroup of $G$. }

We follow that $e \in Z_G(H)$ since $eHe = H$. If $g \in Z_G(H)$, then
for all $q \in H$
$$q g\inv = (g\inv g )q g\inv = g\inv (g q) g\inv = g\inv qg  g\inv = g\inv q$$
and if $g, g' \in Z_G(H)$ then
$$gg'h = ghg' = hgg'$$
thus implying the desired result

\textit{(c) Let $H \subseteq G$ be a subgroup of $G$.  Prove the the
  normalizer $N_G(H)$ is a subgroup of $G$.}

We once again follow that $e \in N_G(H)$. If $g, g' \in N_G(H)$ then
$$g H g\inv = g g\inv H g g\inv = H$$
thus $g\inv \in N_G(H)$ and
$$g g' H (g g') \inv = g g' H g' g = g H g = H$$
as desired.

\subsection{}

\textit{Skip}

\section{Sylow's Theorem}

\subsection{}

\textit{Fix a prime $p$ and an integer $n \geq 1$. For $m \geq 1$, we
  consider the product
  $$A_m = \prod_{i = 0}^{p^n - 1}{p^nm - i}$$
  Suppose that we factor $A_m$ as $A_m = p^kB_m$ with $p \not | B_m$
}

\textit{(a) Assume that $p \not | m$. Find a simple closed formula for $k$ that depends
  only on $p$ and $n$.}

Let us look at $p^nm - i$. Let us factor $p$ out of $i$
and get $i = p^js$, where $p$ does not divide $s$. We follow then that
$$p^nm - i = p^j(p^{n - j}m - s)$$
For every $i \in \omega$ such that $i \neq 0$ there is $j$ such that
$i = p^js$ and $p \not | s$,
and thus we conclude that there is a function $f: \omega_+ \to \omega$
such that $f(i) = j$. We then follow that if $a, b \in \omega_+$, then
$$f(a * b) = f(a) + f(b)$$
(as can be proven quite trivially) and we also note that $k = f(A_m)$
in our conditions. Thus
$$A_m = \prod_{i = 0}^{p^n - 1}{p^nm - i}$$
$$k = f(A_m) = f(\prod_{i = 0}^{p^n - 1}{p^nm - i}) = \sum_{i = 0}^{p^n - 1}{f(p^nm - i)}$$
(none of the elements in the product are zero, which gives us this
result) we then can follow that if $p | a$, and $b \neq 0$ then
$$f(a + b) = \min(f(a), f(b))$$
and in our case where $0 \leq i \leq p^n - 1$ we have that
$$f(p^nm - i) = f(i)$$
for $i \neq 0$ and
$$f(p^nm - 0) = n$$

If $S$ is a finite sequence in $\omega$, then we can follow an
interesting thing.  Namely that if $J$ is a finite sequence of
elements of $\omega$, then we can define $Q_i: J \to \omega$ as
$$Q_i(S) = \text{the number of the elements of $S$ that are greater or equal to }i$$
or more formally
$$Q_i(S) = |\set{j \in S[\omega]: j \geq i}|$$
and then
$$\sum{S} = \sum_{i = 1}^{\omega}{Q_i(S)}$$
and also if $j$ is the highest element of $S$, then
$$\sum{S} = \sum_{i = 1}^{j}{Q_i(S)}$$
thus we have that
$$k = \sum_{i = 0}^{p^n - 1}{f(p^nm - i)} = f(p^nm) + \sum_{i = 1}^{p^n - 1}{f(p^nm - i)} =$$
$$ = f(p^nm) + \sum_{i = 1}^{\omega}{Q_i(f[\set{p^nm - j: 1 \leq j \leq p^n - 1}])}$$
we then follow that since $j \neq 0$ and $j \leq p^n - 1 < p^nm$ we have
$$k = f(p^nm) + \sum_{i = 1}^{\omega}{Q_i(f[\set{j: 1 \leq j \leq p^n - 1}])}$$
then we follow that the maximum possible value of $ f[\set{j: 1 \leq j \leq p^n - 1}]$
is $n - 1$, which gives us
$$k = f(p^nm) + \sum_{i = 1}^{n - 1}{Q_i(f[\set{j: 1 \leq j \leq p^n - 1}])}$$
and then we can also follow that for $i \in N_+$
$$Q_i(f[\set{j: 1 \leq j \leq p^n - 1}])$$
is the number of divisors of $p^i$ in the set $1 \leq j \leq p^n - 1$.
We then can follow that for $j \in N_+$ we have $p^{n - j} - 1$ (from
$p^{i}$ to $p^n - p^i = p^{n - i}p^i - p^i = p^i(p^{n - i} - 1)$)
multiples of $p^i$. Thus
$$k = f(p^nm) + \sum_{i = 1}^{n - 1}{Q_i(f[\set{j: 1 \leq j \leq p^n - 1}])} = $$
$$ = 
f(p^nm) + \sum_{i = 1}^{n - 1}{(p^{n - i} - 1)} = f(p^nm) - n + 1 + \sum_{i = 1}^{n - 1}{(p^{n - i})}$$
then we can refactor the sum to get
$$\sum_{i = 1}^{n - 1}{(p^{n - i})} = p^{n - 1 - 1} + ... + p^1 = \sum_{i = 1}^{n - 2}{p^i} =
\sum_{i = 0}^{n - 2}{p^i} - 1 = \frac{p^{n} - 1}{p - 1} - 1$$
and thus we have
$$k = f(p^nm) - n + 1 + \frac{p^{n} - 1}{p - 1} - 1
= f(p^nm) - n  + \frac{p^{n} - 1}{p - 1}$$
Since $p$ does not divide $m$ we conclude that $f(p^nm)  = n$ and thus
$$k = \frac{p^{n} - 1}{p - 1}$$
as desired

\textit{(b) Use (a) to give a quick proof of Lemma 6.30}

Proof is essentialy the same, but we note that the maximal power of
$p$ in both numerator and denominator are the same, and thus the whole
shebang is not divisible by $p$

\textit{(c) What happens if we allow $p$ to divide $m$?}

We get that if $j$ is the highest power of $p$ that divides $m$ then
the whole product's highest power of $p$ is also $j$ (as seen in the
proof of (a)).

\subsection{}

\textit{Let $p$ be a prime, and let $G$ be a group of order $p^n$. Prove
  that for evert $0 \leq r \leq n$ there is a subgroup $H$ of order $p^r$.}

Sylow's theorem implies the existence for $p^n$ (i.e. the whole group), and
we can state also that $\set{e}$ is also a subgroup of order $p^0$.

Assume that $|G| \geq 2$, and let $q \in G$ be non-identitiy. We
follow that its order $j$ divides $|G|$, and thus since it's a
non-identity, its order is $p^k$ for some $1 \leq k \leq n$. Then we
follow that order of $j^k$ is $p$, which implies that there is a
subgroup of $G$, whose order is $p$.

Assume that for some $n \in \omega$ we have the desired property.
Let $G$ be an arbitrary group of order $p^{n + 1}$. We follow that
its center is non-trivial, and then follow that its center has
a subgroup $H$ of order $p$. Since $H$ is a subgroup of the center,
we can quickly follow that it's normal, and thus $G/H$ constitues
a group. Order of $G/H$ is going to be $p^n$ by Lagrange's theorem. We
then follow that by our assumption, there are subgroups of $G/H$
of order $p^0, p^1, ..., p^n$. Let us label them by $C_0, C_1, ...,
C_n$. Exercise 6.7(b) (one, that is also solved in this work),
implies that for each $C_j$ there is a subgroup $Q_j \subseteq G$
such that $C_j = Q_j / H$. In the proof of that exercise we
essentially state that the union of all the elements of $C_j$ is
itself a subgroup of $G$, and now we're adding that by the fact that
all the elements of $C_j$ are disjoint (by properties of cosets) we
conclude that $|Q_j| = |C_j| * |H| = p^j * p = p^{j + 1}$. Thus we
conclude that for all $0 \leq j \leq n$ there are subgroups
of $G$ of order $p^{j + 1}$. In combination with the fact that
the trivial subgroup $\set{e}$ is a subgroup of $G$, we
conclude that $G$ has subgroups of $p^j$ for all $0 \leq j \leq n + 1$.
By the fact that $G$ is arbitrary, we conclude that it is true for
all groups of order $p^{n + 1}$.

If we now take our previous proof, and combine it with one of the
deriviation at the start, we get a proof by induction of the fact all
groups $G$ of order $p^n$ with $n \in \omega$ have the desired
property. We also note, that we did not restrict $p$, which also
implies that the result is true for any prime $p$.

\subsection{}

\textit{This exercise asks you to give two different proofs of the following
  stronger version of the first part of Sylow's Theorem}

\textit{\textbf{Theorem: } Let $G$ be a finite group. let $p$ be a
  prime, and suppose that $|G|$ is divisible by $p^r$ for $r \in
  \omega$.  There is a subgroup of $G$ of order $p^r$.}

\textit{(a) Give a proof that directly mimics the proof in the book.}

Let $n$ be the highest power of $p$ that divides $|G|$. We follow then
that
$$|G| = p^nm$$
where $p \not | m$, and thus for any given $1 \leq j \leq n - 1$ we have
$$|G| = p^{n - j}p^j{m}$$
then we state that there are $C(p^{n - j}p^j{m}, p^j)$ subsets of
$|G|$ that are of size $p^j$, and thus
$$C(p^{n - j}p^jm, p^j) = \sum_{i = 1}^{r}{\frac{p^{n - j}p^jm}{|G_{A_r}|}}$$
we then follow that by first exercise in this section, the maximal power
of $p$ in $C(p^{n - j}p^jm, p^j)$ is $p^j$, and thus there is $B$ such that
$p \not | B$ and 
$$C(p^{n - j}p^jm, p^j) = p^j B = \sum_{i = 1}^{r}{\frac{p^{n - j}p^jm}{|G_{A_r}|}}$$
thus
$$p^j B = \sum_{i = 1}^{r}{\frac{p^{n - j}p^jm}{|G_{A_r}|}}$$
$$p^j B_m = p^j \sum_{i = 1}^{r}{\frac{p^{n - j}m}{|G_{A_r}|}}$$
$$B_m = \sum_{i = 1}^{r}{\frac{p^{n - j}m}{|G_{A_r}|}} = $$
which implies that there is $r$ such that $|G_{A_r}| | p^{n - j}$.
After this point, the proof is practically the same as well, thus giving
us implication that for all $1 \leq j \leq n - 1$ we have a subgroup of $G$
of order $p^{n - j}$, which is equivalent to the desired conclusion.

\textit{(b) Combine the version of Sylow's Theorem that we did prove
  with the previous exercise}.

Let $G$ be a finite group. We follow that if $n$ is the highest order
of a prime $p$ such that $p^n | |G|$, then there is a subgroup of $G$
of order $p^n$. Previous exercise implies that the resulting subgroup
has subgroups of order $p^0, p^1, ..., p^n$ of its own, and since
subgroup $J$ of a subgroup $K$ of a group $L$ is a subgroup of $L$ we
get the desired result.

\subsection{}

\textit{In example 6.36 we showed that there are exactly two groups of
  order $10$. Do a similar calculation to find all groups of order $15$}

15 is divided by $5$ and $3$, and thus we follow that there are $H_5$
and $H_3$. We then once again follow that the only $k$ such that
$k \mid |G|$ and $k \equiv 1 (\mod 5)$ is $1$, thus implying that
$H_5$ is the only subgroup of order $5$ and thus normal

Then let us uncombine those groups to get
$$H_3 = \set{e, a, a^2}, H_5 = \set{e, b, b^2, b^3, b^4}$$
we then note that since $gcd(3, 5)$ that $H_3 \cap H_5 = \set{e}$,
and thus $a, a^2 \notin H_5$. This in turn gives us that
$$eH_5 = H_5, aH_5, a^2H_5$$
are cosets of $H_5$ that decompose the group, and thus any given
$g \in G$ is of the form $g = a^ib^j$ for some $i, j \in \omega$.

We then note that $a\inv ba \in H_5$ and thus there is $0 \leq j \leq 4$
such that
$$a b a\inv = b^j$$
which is equivalent to
$$b = a\inv b^j a$$
thus we can follow that
$$b = a\inv b^j a = (a\inv b a)^j = (a\inv (a\inv b^j a) a)^j = $$
$$ = (a\inv (a\inv b a)^j a)^j = (a\inv (a\inv (a\inv b^j a) a)^j a)^j = $$
$$ = (a\inv (a\inv b^j a) a)^{j^2} = (a\inv b^j a)^{j^2} = b^{j^3}$$
which is practically the same deriviation as in the book, but with an
extra appplication of substitution of $b$ for $a\inv b^j a$.
Thus we have that
$$b = b^{j^3} \lra e = b^{j^3 - 1}$$
thus we have that
$$j^3 \equiv 1 (\mod 5)$$
we then follow that the only such $j$ is $j = 1$, which implies that
$G$ is abelian, and then by the same deriviation we conclude that $G$
is cyclic group of order 15. 

\subsection{}

\textit{Let $G$ be a finite group, and let $H_1, H_2$ be normal subgroups
  having the property that $gcd(|H_1|, |H_2|) = 1$. Prove that elements
  of $H_1$ and $H_2$ commute.}

Let us look at
$$aba\inv b\inv = (a b a\inv) b\inv $$
since $H_2$ is normal we follow that $aba\inv \in H_2$ and thus
$$aba\inv b\inv \in H_2 b\inv$$
by the fact that $b\inv \in H_2$ we conclude thta
$$H_2 b\inv = H_2$$
and thus
$$aba\inv b\inv \in H_2$$
by similar logic we have
$$aba\inv b\inv = a(ba\inv b) \in a H_1 = H_1$$
and thus
$$aba\inv b\inv \in H_1 \cap H_2$$

If $H_1 \cap H_2$ is non-trivial, then we follow that there is $h \in
H_1 \cap H_2$ such that $h \neq e$. Thus the order of $h$ is some
$j > 1$ and therefore $h$ generates a cyclic subgroup of
order $j$. Therefore we follow that $j$ divides both $|H_1|$
and $|H_2|$ by Lagrange's theorem, which implies that
$gcd(|H_1|, |H_2|) > 1$, which is a contradiction. Thus we conclude that
$H_1 \cap H_2$ is trivial, and therefore
$$aba\inv b\inv = e \ra ab = ba$$
which is the desired result.

\subsection{}

\textit{Let $G$ be a finite group of order $|G| = pq$ where $p, q$ are
  primes satisfying $p > q$. Assume further that $p \not \equiv 1 (\mod
  q)$. }

\textit{(a) Prove that $G$ is an abelian group. }

We follow that there are $H_p, H_q$ - respective $p$-Sylow and
$q$-Sylow subgroups of $G$ of order $p$ and $q$ respectively.  Their
order implies that $H_p$ and $H_q$ are abelian. Example 6.37 implies
that $H_p$ is normal (while we know nothing of the $H_q$).  We then
follow that our assumption that $p \not \equiv 1 (\mod q)$ gives us an
ability to use example 6.37 to prove by same idea mutatis mutandis
that $H_q$ is also normal. Thus previous exrcise implies that elements
of $H_p$ and $H_q$ commute.

We then follow that every element $g \in G$ can be expressed as
$$g = a b$$
for some $a \in H_p$ and $b \in H_q$ (by the same deriviation as in
example 6.36 or two exercises ago). Thus let $g_1, g_2 \in G$. We
follow that there are $a, a' \in H_p$ and $b, b' \in H_q$ such that
$$g_1 = a b, g_2 = a' b'$$
and thus
$$g_1 g_2 = ab a' b' = a (b a') b' = a a' b b'
= (a a') (b b') = a' (a b') b = a' b' a b = g_2 g_1$$ which implies
the desired result.

\textit{(b) Prove that $G$ is a cyclic group}

We can use the same deriviation as in example 6.36

\subsection{}

\textit{Let $G$ be a group. An isomorphism from $G$ to itself is
  called an automorphism of $G$. The set of automorphisms is denoted
  $$Aut(G) = \set{\text{ group isomorphisms } G \to G}$$
  We define a composition law on $Aut(G)$ by composition of functions
}

\textit{(a) Prove that this composition law makes $Aut(G)$ into a
group}

We can follow that the general identity is the identity for the group.
If $f \in Aut(G)$ is an element, then $f\inv$ is a function by the
fact that everything in $Aut(G)$ is a bijection. Inverse of an
isomorphism is an isomorphism itself, and thus we have inrverses in
the $Aug(G)$.  Association is taken care of by the general association
of the composition of functions, which gives us the desired result

\textit{(b) Let $a \in G$. Define a map $\phi_a: G \to G$ by the
  formula
  $$\phi_a(g) = a g a\inv$$
  Prove that $\phi_a \in Aut(G)$ and that the map
  $$q: G \to Aut(G), q(a) = \phi_a$$
  is a group homomorphism.
}

We follow that $\phi_a$ is a composition of products of elements of a
group, and thus is injective. If $h \in G$, then
$$\phi_a(a\inv h a) = h$$
thus making $h$ surjective, which implies that $\phi_a \in Aut(G)$.
Then we follow that
$$q(ab) = \phi_{ab}$$
$$q(a)q(b) = \phi_a\phi_b$$
now let $g \in G$. We follow that
$$\phi_{ab}(g) = ab g(ab)\inv = abgb\inv a$$
$$\phi_{a}\phi_{b}(g) = \phi_a(bgb\inv) = a b g b\inv a\inv$$
which implies that $q$ is indeed a homomorphism, as desired.

\textit{(c) Prove that the kernel of homomorphism 6.23 is the center
  $Z(G)$ of $G$.}

Let $c \in Z(G)$. We follow that for $g \in G$
$$\phi_c(g) = c g c\inv = c c\inv g = eg = g$$
and if $v \notin Z(G)$, then if $q(v) = id$ then for all $g \in G$
$$\phi_v(g) = g$$
thus
$$v g v\inv = g$$
$$v g = gv$$
which implies that $v \in Z(G)$, which is a contradiction. Thus we
have the desired result.

\textit{(d) Elements of $Aut(G)$ that are equal to $\phi_a$ for some
  $a \in G$ are called inner automorphisms, and all other elements of
  $Aut(G)$ are called outer automorphisms. Prove that $G$ is abelian if
  and only if its only inner automorphism is the identity map}

Inner automorphisms are essentially images of $q$.  If $G$ is abelian,
then $Z(G) = G$, and thus $q(g) = id$ as per the previous exercise. If
the range of $q$ is $\set{id}$, then for all $v ,g \in G$
$$\phi_v(g) = g$$
$$v g v\inv = g$$
$$v g  = gv$$
which implies the desired result.

\textit{(e) More generally, if $H$ is a normal subgroup of $G$, prove that
  there is a well-defined group homomorphism
  $$w: G \to Aut(H), q(a)(h) = a h a\inv $$
  and that the kernel of this homomorphism is the centralizer of $H$ in $G$
}

We can follow that if $f$ is an inner automorphism of $G$, then for $h
\in H$ we have
$$f(h) = a h a\inv$$
and more generally
$$f[H] = a H a\inv = H$$
thus we can make $f$ into an automorphism on $H$ by restricting its
domain. We call the function that does such a thing $j$.  Thus by
composing homomorphism from (b) with $j$ we get the desired function.
We then follow that if $k \in Z_G(H)$, then
$$w(k)(h) = k h k\inv = k k\inv h = h$$
for all $h \in H$, thus implying that $w(k) = id$. Now assume that
$k\notin Z_G(H)$ and we have that $w(k) = id$. Then for all $h \in H$
$$h = w(k)(h) = k h k\inv  \ra kh = hk \ra k \in Z_G(H)$$
which is a desired contradiction.

\subsection{}

\textit{Let $C_n$ be a cyclic subgroup of order $n$, and let
  $Aut(C_n)$ be the automorphism group of $C_n$; Prove that
  $Aut(C_n)$ is isomorphism to $(Z/nZ)^*$.}

We notice that $C_n$ by definition is generated by some element
$g \in C_n$. That is,
$$C_n = \set{g^i: 0 \leq i < n}$$
Let $q \in [0, n) \cap \omega$ be such that it's relatively prime to
$n$.  We follow that $g^q$ also generates $C_n$, as was proven
somewhere in the book. Thus we can follow that there is a function $h:
C_n \to C_n$
$$h(g^j) = (g^q)^j$$
that is bijective. We then can trivially follow that this thing is a
homomorphism, and thus isomorphism, and thus an element of $Aut(C_n)$.

We then follow that if $k: C_n \to C_n$ is an isomorphism, then there is
$0 < j < n$ such that
$$k(g) = g^j$$
which gives us the fact that for every element of $Aut(C_n)$ there is
a unique $0 < j < n$ and vice versa. We then notice that $j$ can only
be a number that is relatively prime to $n$, which in combination with
the previous paragraph gives us the idea that $\phi: (Z/nZ)^* \to Aut(C_n)$
defined by
$$phi(j)(g) = g^j$$
is the desired isomorphism


\subsection{}

\textit{Let $G$ be a group of order $|G| = 75$.}

\textit{(a) Prove that $G$ has a subgroup $H$ with all three of the
  folllowing properties: $|H| = 25$, $H$ is a normal subgroup of $G$,
  $H$ is abelian}

Let $H$ be a 5-Sylow subgroup of $G$. We follow that its order is
25. Since its order is a square of a prime, it's abelan. Numbers that
divide 75 are
$$1, 3, 5, 15, 25, 75$$
none of which give 1 mod 5, except for $1$, which implies that this is
the only 5-Sylow subgroup, which implies that the thing is normal.

\textit{(b) Suppose that the subgroup $H$ in (a) is cyclic of order
  $25$.  Prove that $G$ is abelian.}

We follow that there is also a 3-Sylow subgroup. Since its order is
$3$ we follow that it's cyclic. By the example's deriviation we have 
$$aba\inv = b^j$$
$$b = a\inv b^j a$$
$$b = b^{j^3}$$
$$j^3 \equiv 1 \mod 25$$
the only such $j$ is $1$, which implies that 3-Sylow and $H$ commute.
Let $a$ now generate $H$ and $b$ generate the 3-Sylow subgroup. We
follow that $ab$'s order is finite and let this order be $j$. We
follow that
$$e = (ab)^j \ra e = a^j b^j \ra a^{-j} = b^j \in H \cap H_3
= \set{e} \ra$$
$$\ra a^k = b^k = e \ra 3 | k \land 25 | k \ra k = 75$$
which is pretty much the same deriviation as in one of the examples in
the chapter

\section{Double Cosets and Sylow's Theorem}

\subsection{}

\textit{Let $G$ be a finite group of order $|G| = pm$, where $p$ is
  prime and $p \not | m$. Assume further that $m$ has the following
  property: The only divisor $d$ of $m$, that satisfies $d \equiv m
  (\mod p)$ is $d = m$.}

\textit{(a) Prove the $p$-Sylow subgroup of $G$ is a normal subgroup
  of $G$.}

Let us name the $p$-Sylow sugroup of $G$ as $H$.

We firstly follow that if $p$-Sylow subgroup is normal, then its
conjugates are equal to the group, thus making this group the one
and only $p$-Sylow subgroup of $G$. Same goes in the other
direction, thus $p$-Sylow is normal if and only if $k = 1$ (variables
definition here is the same as in the text in the book).

Now assume that $H$ is not normal. We follow that $k > 1$. We also
have that $k | pm$. If $p | k$, then $k \equiv 0 \mod p$, which is a
contradiction, which implies that $p \not | k$. Since $p \not | m$
we follow then that
$$k | m$$
which implies that there is $n \in N_+$ such that $m = kn$. Moreover
we have that $n < m$ since $k > 1$. Now we can follow that
$$k \equiv 1 (\mod p)$$
$$nk \equiv n (\mod p)$$
$$m \equiv n (\mod p)$$
$$n \equiv m (\mod p)$$
which directly contradicts the premise (this proof is somewhat
awkward, and can be easily converted to the direct one).



\textit{Skipping the rest of those exercises, maybe will come back for
  them later}

\chapter{Rings - Part 2}

\section{Irreducible Elements and Unique Factorization Domains}

\subsection{}

\textit{Let $R$ be an integral domain, and let $a, b \in R$. Prove
  that the following are equivalent:}

\textit{(a) $a | b$ and $b | a$ }

\textit{(b) There is a unit $u \in R^*$ satisfying $a = bu$. }

\textit{(c) There is an equality of principal ideals $aR = bR$.}

There's probably a cyclical imlication here, so assume that $a | b$ and $b | a$.
Definition implies that $a, b \neq 0$ and also that there are $k_1,
k_1 \in R$ such that $a = bk_1$ and $b = a k_2$. We follow then that
$$a = bk_1 = a k_2 k_1$$
thus
$$a * 1 = a k_2 k_1$$
which by the fact that $a \neq 0$ and $R$ being integral domain implies that
$$k_2 k_1 = 1$$
thus implying that $k_1, k_2 \in R^*$, which implies (b) from deriviation that
$a = bk_1$.

Now if $a = bu$ for some $u \in R^*$, then we follow that
$$h \in aR \ra (\exists r \in R)( h = ar) \ra
(\exists r \in R)( h = bur) \ra  (\exists ur \in R)( h = b(ur)) \ra h \in bR$$
thus implying that $aR \subseteq bR$. Reverse inclusion is kinda similar here,
which implies (c)

If $aR = bR$, then $a \in bR$, which implies that there is $r \in R$ such
that $a = br$, which implies that $b | a$. Similar implication holds
for the other condition.

\subsection{}

\textit{Let $R$ be a ring, and let $a, b \in R$. }

\textit{(a) Suppose that $b = au$ for some unit $u \in R^*$. Prove
that the principal ideals $aR$ and $bR$ are equal.}

In the previous exercise we had the assumption that $R$ is an integral
domain, which we don't have here. However out proof holds the same
since the deriviation from (b) to (c) does not involve any properties
of an integral domain

\section{Eucledian Domains and Principal Ideal Domains}

\subsection{}

\textit{Let $R$ be a PID, let $a, b \in R$ be elements that are not
  both $0$, and let $d \in R$ be an element that generates the
  (principal) ideal generated by $a$ and $b$:
  $$dR = aR + bR$$}

\textit{(a) Prove that $d$ divides both $a$ and $b$}

We can follow that since $dR = aR + bR$ we have that $0 \in bR$ and
thus $aR = aR + 0 \subseteq dR$. Therefore we have that $a \in dR$ and
thus there is $r \in R$ such that $a = dr$, which implies that $d |
a$, as desired. Similar thing holds for $b$.


\textit{(b) Let $c \in R$ be a non-zero element that divides both $a$
  and $b$. Prove that $c$ divides $d$.}

Since $c \vert a$ and $c \vert b$ we follow that there are $n, k \in R$
such that $nc = a$ and $kc = b$. Since $dR = aR + bR$ and $d \in dR$
we follow that there are $r, r' \in R$ such that
$$d = ar + br' = ncr + kcr' = c(nr + kr')$$
which implies the desired result

\subsection{}

\textit{Let $R$ be a Eucledian domain with size function $\sigma$ and
  let $a \in R$ be a nonzero element of $R$. Prove that
  $$a \in R^* \iff \sigma(a) = \sigma(1)$$
}

$a \in R^*$ implies that for all $b \in R \setminus \set{0}$ we have
$$\sigma(ba) = \sigma(b)$$
thus since $1 \neq 0$ we have
$$\sigma(a) = \sigma(1 * a) = \sigma(1)$$
which gives us the forward implication.

Since $\sigma(1) = \sigma(a)  = \sigma(1 * a)$
we follow the desired result by 7.15

\subsection{}

\textit{Let $R$ be an integral domain, and let
  $$\sigma: R \setminus \set{0} \to \omega$$
  be a function that has property (1) in the definition of a Eucledian
  domain. Define a new function
  $$\tau: R \setminus \set{0} \to \omega,
  \tau(r) = \min \set{\sigma(rc): c \in R \setminus \set{0}}$$
  Prove that $\tau$ has both properties (1) and (2)
}

Suppose that $b, c \in R$ are arbitrary nonzero elements.  We follow
that there are $a, a' \in R \setminus \set{0}$ such that $\tau(b) =
\sigma(ab)$, $\tau(cb) = \sigma(acb)$. Since $R$ is an integral domain
and $a, c$ are nonzero we follow that $ac \neq 0$. By definition of
$\tau$ we follow that for all $q \in R \setminus \set{0}$ we have that
$\tau(b) \leq \sigma(qb)$, and thus $\tau(b) \leq \sigma((ac)b) =
\tau(cb)$ which implies the part (2) of the definition.


Assume that $b, r \in R \setminus 0$ are such that $\sigma(b) >
\sigma(r)$. We firstly can follow that $\sigma(r) \geq \tau(r)$
by definition of $\tau$, and $\tau(b) = \sigma(cb) \geq \sigma(b)$ by
previous point, which implies that
$$\tau(b) \geq \sigma(b) > \sigma(r) \geq \tau(r)$$
which gives us point (1) and the desired result.

\subsection{}

\textit{this one was taken care of in the linear algebra course}

\subsection{}

\textit{Explain why proof of the fact that $Z[i]$ is a Eucledian domain
  won't work on $Z[\sqrt{3}]$ }

\textit{here i've copied the exercise wrongly}

We firstly follow that $Z[\sqrt{3}]$ is a subring of $R$, and thus
it's got to be an integral domain. We can also state that
there is a bijection between $Z[\sqrt{3}]$ and $Z^2$ which
can also give us a plane, just as in $Z[i]$ case. We can also define
$\sigma: Z[\sqrt{3}] \to R$ by setting
$$\sigma(a + b \sqrt{3}) = a^2 + b^2$$
we can follow that $\sigma(\zeta) = 0 \lra \zeta = 0$, but can we do the
second part? We can see that
$$\sigma(\sqrt{3} * \sqrt{3}) = \sigma(3) = 3^2 + 0^2 = 9$$
$$\sigma(\sqrt{3}) * \sigma(\sqrt{3}) = 1 * 1 = 1 \neq 9$$
which gives us problems.

\subsection{}

\textit{For each of the following rings $R$ and elements $\alpha$,
  determine whether $\alpha$ is irreductible in $R$. Justify
  your answer by either factoring $\alpha$ or proving that it is
  irreductible}

\textit{(a) $R = Z[i]$ and $\alpha = 2 + 3i$}

We can follow that we have a function $\sigma(x + yi) = x^2 + y^2$ is
having a property $\sigma(ab) = \sigma(a)\sigma(b)$. We then follow that
$$\sigma(\alpha) = \sigma(2 + 3i) = 4 + 9 = 15 = 3 * 5$$
thus we can try to make up elements with sizes $3$ and $5$. The problem is
that $3$ is not a sum of squares, and thus we can follow that there is
no element $a$ of $Z[i]$ such that $\sigma(a) = 3$, which implies that the
element is irreductible.

\textit{(b) $R = Z[i]$ and $\alpha = 4 + 3i$}

We follow that
$$\sigma(\alpha) = \sigma(4 + 3i) = 16 + 9 = 25 = 5^2$$
we then can heck around with $Z[i]$ to get that
$$(2 - i) * (1 + 2i) = 2 + 4i - i + 2 = 4 + 3i$$
thus giving us the desired result.

\textit{(c) $R = F_2[x]$ and $\alpha = x^5 + x + 1$}

Here we firstly note that the size function is $\deg$, which has
property $\deg(a * b) = \deg(a) + \deg(b)$, thus stating that our
resulting elements are of order $1, 4$ or $2, 3$. We also note that
both elements should have a non-zero constant value in order
to result in a non-zero constant value in the result. There
are only a finite amount of polynomials of both degrees, and thus
we can try to go through all of them
$$(x + 1) (x^4 + 1) = x^5 + x^4 + x + 1$$
$$(x + 1) (x^4 + x + 1) = ... $$
we then note an interesting thing: polynomial multiplication in $F_2$
looks kinda similar to multiplication in a base 2. We follow that in
base 2
$$\alpha = 2^5 + 2 + 1 = 35 = 5 * 7$$
and $5 = x^2 + 1$, $7 = x^2 + x + 1$. We then follow that
$$(x^2 + 1) * (x^2 + x + 1) =
x^4 + x^3 + x^2 + x^2 + x + 1 = x^4 + x^3 + x + 1$$ thus concluding
that we were wrong on this one (but it's still pretty interesting how
exactly this thing translates to binary). Thus let us go back to
just calculating the things
$$(x + 1) (x^4 + 1) = x^5 + x^4 + x + 1$$
$$(x + 1) (x^4 + x + 1) = x^5 + x^2 + x + x^4 + x + 1 = x^5 + x^4 + x^2 + x + 1$$
$$(x + 1) (x^4 + x^2 + x + 1) = x^5 + x^3 + x^2 + x + x^4 + x^2 + x + 1 = x^5 + x^4 + x^3 + 1$$
$$(x + 1) (x^4 + x^3 + x^2 + x + 1) =
x^5 + x^4 + x^3 + x^2 + x + x^4 + x^3 + x^2 + x + 1 = x^5 + 1$$
after doing this thing for a while, I thought that maybe a better idea
is to think about theproblem a bit. If we're multiplying $(x + 1)$ by any
polynomial with 4th power, then the fourth power will be in the resulting
polynomial, which does not give us the desired result. Thus we can
skip this thing to get to the powers $3, 2$:
$$(x^2 + 1) (x^3 + 1) = x^5 + x^2 + x^3 + 1$$
$$(x^2 + 1) (x^3 + x + 1) = x^5 + x^3 + x^2 + x^3 + x + 1 = x^5 + x^2 + x + 1$$
$$(x^2 + 1) (x^3 + x^2 + 1) = x^5 + x^4 + x^2 + x^3 + x^2 + 1 = x^5 + x^4 + x^3 + 1$$
$$(x^2 + 1) (x^3 + x^2 + x + 1) = x^5 + x^4 + x^3 + x^2 + x^3 + x^2 + x + 1 =
x^5 + x^4 + x + 1$$
$$(x^2 + x + 1) (x^3 + 1) = x^5 + x^4 + x^3 + x^2 + x + 1$$
$$(x^2 + x + 1) (x^3 + x + 1) = x^5 + x^3 + x^2 + x^4 + x^2 + x + x^3 + x + 1 =
x^5 + x^4 + 1$$
$$(x^2 + x + 1) (x^3 + x^2 + 1) = x^5 + x^4 + x^2 + x^4 + x^3 + x + x^3 + x^2 + 1
x^5 + x + 1$$
which gives us $\alpha$, and thus the desired result.

\textit{(d) $R = F_2[x]$ and $\alpha = x^5 + x^2 + 1$}
we've got only one polynomial to check, thus let us do this:
$$(x^2 + x + 1) (x^3 + x^2 + x + 1) = x^5 + x^4 + x^3 + x^2 + x^4 + x^3 + x^2 + x + x^3 + x^2 + x + 1
x^5 + x^3 + x^2 + 1$$
thus we've gone through all the applicable polynomials, which implies that
there $\alpha$ is irreductible. 

\subsection{}

\textit{This one mirrors the proof of 3.51(c) since we haven't used
  degrees in that proof}

\section{Factorization in PID}

\subsection{}

\textit{Let $R$ be a principal ideal domain, and let $a, b c \in R$. Prove that
  if $a$ is irreductible, does not divide $b$, and $a | c$, $b | c$, then $ab | c$.}

We can follow that there are nonzero $n ,k \in R$ such that
$$an = c, bk = c$$
thus
$$an = bk \ra a | bk$$
since $a$ does not divice $b$ we follow that $a | k$. Thus there is $j \in R$
such that $aj = k$, and therefore
$$bk = b aj = c$$
which implies that $ba = ab$ divides $c$, as desired.

\subsection{}

\textit{Let $R$ be a PID. Prove the 7.19 without the assumption that
  $R$ is a Eucledian domain by assuming that $a \in R$ is non-zero
  non-unit, not a product of irreductibles, and following the following
  steps.
}

\textit{(a) Show that you can write $a = a_1 b_1$ as a product of
  non-units such that $b_1$ is not a product of irreductibles and then
  that you can write $a = a_1 a_2 b_2$ as a product of non-units such
  that $b_2$ is not a product of irreductibles etc. In this way, for
  every $n$ you get a product of non-units $a = a_1 a_2 ... a_n b_n$
  such that $b_n$ is not a product of irreductibles. 
}

Since $a$ is not a non-unit, and not irreductible, we follow taht
there are $a_1, b_1$ such that
$$a = a_1 b_1$$
We follow that if both of $a_1, b_1$ are irreductibles, then $a$ is a
product of irreductibles, which is a contradiction. Therefore we
conclude that at least one of $a_1, b_1$ is not irreductible. We then
follow that at least one of $a_1, b_1$ also has to be a product of
non-irreductibles, otherwise we have that $a$ is a product of
irreductibles. By relabeling if necessary and proceeding with induction
we get the desired result.

\textit{(b) Prove that the ideals $I_n = b_nR$ satisfy $I_1 \subseteq
  I_2 \subseteq I_3 ... $}

We can follow that by construction of $b_n$ we have that $b_{n + 1} | b_n$,
and thus $b_{n + 1}R \subseteq  b_n$, which implies this result.

\textit{(c) Let $J = I_1 \cup I_2 ...$ be the union of all of the
  $I_n$. Prove that $J$ is an ideal of $R$. Then since we have assumed that
  $R$ is a PID, the ideal $J$ is principal, so we can write $J = cR$
  for some $c \in R$.}

We can follow that any given $a_1, a_2 \in J$ are both contained in
some $I_n$, and thus their sum is also in $I_n$. We can also
follow that $a_1 c \in I_n$ by the fact that $I_n$ is an ideal,
thus making $J$ an ideal.

\textit{(d) Since $J$ is the union of the $I_n$, there is an index $k$
  so that $c \in I_k$. Prove that $I_k = I_{k + 1} = I_{k + 2} = ...$}

We know that $cR = J$, $cR \subseteq I_k$ and $I_k \subseteq J$,
which implies that $I_k = J$. By similar reasoning we have the rest.

\textit{(e) Use $I_k = I_{k + 1}$ to deduce that $a_{k + 1}b{k + 1}R =
  b_{k + 1}R$, and from this, deduce that $a_{k + 1} \in R^*$.}

We can follow that $I_{k + 1} = b_{k + 1}R$ by definition, and
$b_k = a_{k + 1}b_{k + 1}$ by its definition, thus implying that
$$b_k R = a_{k + 1}b_{k + 1}R =  I_k = I_{k + 1} = b_{k + 1}R$$
thus we have that there is $c \in R$ such that
$$a_{k + 1}c b_{k + 1} = b_{k + 1}$$
we then follow that $a_{k + 1}c = 1$ by cancellation property, that
implies in turn that $a_{k + 1}\in R^*$.

\subsection{}

\textit{Let $R = Z[\sqrt{-3}]$ be the ring that we studies in Example 7.21.}

\textit{(a) Prove that $\pm 1$ are the only units in $R$.}

$\pm 1$ are obviously units here, as can be proven trivially.

If we take a number $q$ that is in $Z$ and not equal to $\pm 1$, then
we follow that if we multipy it by another member of $Z$, then we're
not getting $1$ (ddue to basic properties of $Z$). If we multiply them
by any $a + b\sqrt{-3}$ with non-zero $b$, then we're getting non-zero
$b$ in the result as well, thus proving that no member of $Z$ is a
unit.

If $q \notin Z$, then we follow by previous point that its supposed
multiplicative inverse cannot be in $Z$ by previous point. If $q =
b\sqrt{-3}$, then we follow that by multiplicating this thing by some
$b'\sqrt{-3}$ will give us $-3bb'$, which cannot be equal to $1$ with
$b, b' \in Z$. Multiplication by $a' + b'\sqrt{3}$ with nonzero $a'$
will give us nonzero $\sqrt{-3}$ in the result, so it won't do. Thus
the only other case happens if $q = a + b\sqrt{-3}$ with $a, b \neq
0$. Multiplicating this $q$ by any $a' + b'\sqrt{-3}$ with zero $a$ or
$b$ will give us previous point, thus we need to assume that $a', b'
\neq 0$. We follow that
$$(a + b\sqrt{-3})(a' + b'\sqrt{-3}) = 1$$
we then follow that
$$(a - b\sqrt{-3})(a' - b'\sqrt{-3}) = 1$$
and mmultyplying them we get
$$(a^2 + 3b^2)(a'^2 + 3b'^2) = 1$$
we then follow that $(a^2 + 3b^2)$ for nonzero $a, b$ is greater or
equal to $4$, thus implying that there is no way to pick such numbers,
thus impying the desired result


\textit{(b) Prove that $1 + \sqrt{-3}$ and $1 - \sqrt{-3}$ are
  irreductible elements of $R$.}

Suppose that there are non-units $q, w \in R$ such that
$$qw = 1 + \sqrt{-3}$$
we follow that
$$(a + b\sqrt{-3})(a' + b'\sqrt{-3}) = 1 + \sqrt{-3}$$
and therefore
$$(a - b\sqrt{-3})(a' - b'\sqrt{-3}) = 1 - \sqrt{-3}$$
and thus
$$(a^2 + 3b^2)(a'^2 + 3b'^2) = 1^2 + 3 = 4$$
$$(a^2 + 3b^2)(a'^2 + 3b'^2) = 4$$
and then we're going with pretty much the same deriviation as in the
example.

\textit{(c) Prove that $2$ is ireeductible in $R$}

Skip

\textit{(d) Use previous point to dedufe that $R$ is not a UFD}

Directly follows from definitions and trivial deriviations

\section{The Chinese Remainder Theorem}

\subsection{}

\textit{Solve the following simultanenous congruences:}

\textit{(a) $x \equiv 3(mod 7)$ abd $x \equiv 5 (\mod 11)$}

$38 + 77n $ for $n \in Z$.

\textit{(b) $x \equiv 31(mod 117)$ abd $x \equiv 41 (\mod 119)$}

$13726 + 117 * 119 * n $ for $n \in Z$.

\textit{Next one is 423, and the rest is skipped}

\subsection{}

\textit{Let $R$ be a commutative ring}

\textit{(a) Suppose that $a, b \in R$ have the property that $aR + bR
= R$. Prove that for all $m, n \geq 1$ we have
  $$a^mR + b^nR = R$$
}

We follow that there are $v, w \in R$ such that
$$a^2v + baw = a$$
and thus we have that
$$1 = av + bw = (a^2v + baw)v + bw =
a^2v^2 + baww + bw = a^2v^2 + b(aw^2 + w)$$
and therefore we follow that $a^2R + bR = R$. By induction we can
prove that $a^mR + bR = R$, which would imply the desired result 

\textit{Next one is also proven by induction}

\subsection{}

\textit{Let $R$ be a PID}

\textit{(a) Let $a, b \in R$. Prove that
  $$aR + bR = R \lra aR \cap bR = abR$$
}

Assume the former. We follow that if $x \in abR$ then there is $q \in
R$ such that $x = abq$, and thus $x \in aR, x \in bR$, which implies
that
$$abR \subseteq aR \cap bR$$
If $x \in aR \cap bR$, then we follow that there are $u, w \in R$ such that
$au = bw = x$. Since $aR + bR = R$ we follow also that there are $q, e \in R$
such that
$$aq + be = 1$$
and thus
$$u = u * 1 = u(aq + be) = auq + bue = bwq + bue = b(wq + ue)$$
and thus we have that
$$x = au = a b(wq + ue) \in abR$$
which gives us the forward implication.

Since $R$ is a PID we follow that there is $d \in R$ such that
$$aR + bR = dR$$
thus there are $u, w \in R$ such that
$$au + bw = d$$
We then follow that by one of the previous exercises
$d | a$ and $d | b$. Therefore there are $n, j \in R$
such that
$$d n = a$$
$$d k = b$$
and thus we follow that
$$dnk = ak = bn \in aR \cap bR = abR$$
therefore we have that there is $q \in R$ such that
$$dnk = abq$$
and thus
$$dnk = dndkq$$
since $R$ is a PID we follow that it's an integral domain, and thus
we can cancel out the terms to obtain
$$dnk = dndkq$$
$$1 * (dnk) = (dnk) * dq$$
$$1 = dq$$
which implies that $d \in R^*$, and thus $dR = R$, which gives us the desired result.

\textit{(b) Let $c_1, ..., c_n \in R$, and suppose that $c_iR + c_jR =
R$ for all $i \neq j$. Prove that
  $$c_1 R \cap c_2 R ... \cap c_nR = c_1 c_2 ... c_n R$$
}

As before, we can follow that $c_1 c_2 ... c_n R \subseteq c_1 R \cap
c_2 R ... \cap c_nR$ by a straightforward application of the
definition.

Now the proof of 7.25 (particularly the IH part) gives us that
$$c_1 c_2 ... c_n R + c_{n + 1} R = R$$
namely throught the deriviation that
$$c_1 R + c_3 R = R$$
$$c_2 R + c_3 R = R$$
$$c_1 u_1 + c_3 w_1 = 1$$
$$c_2 u_2 + c_3 w_2 = 1$$
$$(c_1 u_1 + c_3 w_1) (c_2 u_2 + c_3 w_2) = 1$$
$$c_1 c_2 u_1 u_2 + c_3 c_1 u_1 w_2 + c_3 ... = c_1 c_2 u_1 u_2 + c_3 (...) = 1$$
thus we fllow that
$$c_1 R \cap c_2 R \cap c_3 R =
(c_1 R \cap c_2 R) \cap c_3 R = c_1 c_2 R \cap c_3 R = c_1 c_2 c_3R$$
which by induction givesus the desired result.

\textit{(c) Prove that (a) may be false if $R$ is not a PID by giving
a counterexample for the ring $F[x, y]$ where $F$ is a field.  }

Let us look at $xF[x, y]$ and $yF[x, y]$. We follow that if $q \in
xF[x, y] \cap yF[x, y]$, then every sum element of $q$ is a multiple
of $x$ and $y$, and thus $xy$, thus implying that $q \in xyR$, and the
reverse inclusion is also trivial. However, we have that
$$1 \notin xF[x, y] + yF[x, y]$$
which implies the desired result.

\subsection{}

\textit{Prove the generalized version of the CRT}

We're going to try to replicate the proof of the version 2
here, but with additional points. Thus we firstly need to
prove the version for 2 ideals.

Let us denote $I = I_1 \cap I_2$.  We follow that since $I_1 + I_2 =
R$, that there are $i \in I_1, j \in I_2$ such that
$$i + j = 1$$
there is an obvious map
$$\psi: R \to R/I_1 \times R/I_2, \psi(r) = (r \mod I_1, r \mod I_2)$$
we then follow that if $x \in I = I_1 \cap I_2$, then $\psi(x) = (x
\mod I_1, x \mod I_2) = (0, 0)$, thus implying that $I \subseteq
\ker(\psi)$.
(by $\mod$ here we mean standart addition of the cosets, the same as
in the chapter).

Now assume that $r \in \ker(\psi)$. We follow that $r + I_1 = I_1$ and
$r + I_2 = I_2$, thus we have that $r \in I_1$ and $r \in I_2$,
and thus $r \in I$. Therefore we conclude that $I \subseteq \ker(\psi)$,
and thus $\ker(\psi) = I$.

Thus we have an injective homomorphism
$$\phi: R/I \to R/I_1 \times R/I_2$$
we've aasumed in the beginning that $i + j = 1$, and thus we have that
$i = 1 - j$, and $j = 1 - i$. We follow that
$$\phi(1 - j) = \phi(i) = (i + I_1, 1 - j + I_2) = (0, 1)$$
and similar for the other case, which implies that $\psi$ is
surjective, as desired.

Now for the inductive step. We're going to state it only for the
case of 3, which can be generalized further. We follow that there are
$i_1 \in I_1, i_2 \in I_2, i_{3, 1}, i_{3, b2} \in I_3$ such that
$$i_1 + i_{3, 1} = 1$$
$$i_2 + i_{3, 2} = 1$$
thus giving us that
$$(i_1 + i_{3, 1})(i_2 + i_{3, 2}) = 1$$
$$i_1 i_2 + i_1 i_{3, 2} + i_{3, 1} i_2 + i_{3, 1} i_{3, 2} = 1$$
we follow that
$$i_1 i_{3, 2} + i_{3, 1} i_2 + i_{3, 1} i_{3, 2} \in I_3$$
and
$$i_1 i_2 \in I_1 \cap I_2$$
which implies that there are $q \in I_1 \cap I_2, r \in I_3$ such that 
$$q + r = 1$$
thus givin us the fact that
$$I_1 \cap I_2 + I_3 = R$$
this is the point where our previous proof will kick in, and by the same
deriviation as in the original proof, we get the desired result.

\textit{Skipping the rest of the exercises}

\section{Field of Fractions}

\subsection{}

\textit{Prove the following results, which are needed to complete the
  proof of Theorem 7.30}

\textit{(a) Prove that the relation $(a, b) \equiv (a', b')$ defined
  by $ab' = a'b$ is an equivalence relation.}

Firstly it's obvious that
$$ab = ab$$
thus giving us the reflexivily. Symmetry is also kinda trivial.  If
$$(a_1, b_1) \equiv (a_2, b_2), (a_2, b_2) \equiv (a_3, b_3)$$
then we follow that
$$a_1 b_2 = a_2 b_1, a_2 b_3 = a_3 b_2$$
we thus can multiply both equations to get 
$$a_1 b_2 a_2 b_3 = a_2 b_1 a_3 b_2$$
since $R$ is an indegral domain, we follow that we can cancel and thus
$$a_1 a_2 b_3 = a_2 b_1 a_3$$
$$a_1  b_3 =  b_1 a_3$$
thus implying that
$$(a_1, b_1) \equiv (a_3, b_3)$$
which is the desired conclusion.

\textit{(b) If $(a_1', b_1') \equiv (a_1, b_1)$ and $(a_2', b_2')
  \equiv (a_2, b_2)$, prove that their multiplication is well-defined
}

We can follow that we need to prove that
$$(a_1' a_2', b_1' b_2') \equiv (a_1 a_2, b_1 b_2)$$
by definition of the multiplication. Thus we need to prove that
$$a_1' a_2'b_1 b_2 = a_1 a_2 b_1' b_2'$$
by definition of $\equiv$. By our assumptions we have that
$$a_1' b_1 = a_1 b_1'$$
$$a_2' b_2 = a_2 b_2'$$

Thus we have
$$a_1' a_2'b_1 b_2 = a_1 a_2 b_1' b_2'$$
$$a_1' b_1 a_2' b_2  = a_1 b_1' a_2 b_2'$$
$$(a_1' b_1) (a_2' b_2)  = a_1 b_1' a_2 b_2'$$
$$(a_1 b_1') (a_2 b_2')  = a_1 b_1' a_2 b_2'$$
$$a_1 b_1' a_2 b_2'  = a_1 b_1' a_2 b_2'$$
(which you can chain, or use the same thing as in the proof for the
addition to get a solid proof).

\textit{(c) Prove that the map $\lambda$ defined by 7.23 is a field
  homomorphism from $F$ to $K$.}

We can follow that
$$\lambda(1, 1) = \phi(1) * \phi(1)\inv = \phi(1) * \phi(1) = 1 * 1 = 1$$
where we follow that $\phi(1) = 1$ by the fact that it's a
homomorphism (with 1's in their respective rings/fields).

We also have
$$\lambda((a, b) + (c, d)) = \lambda((ad + bc, bd)) = $$
$$ = \phi(ad + bc) * \phi(bd)\inv$$
$$\lambda(a, b) + \lambda(c, d) = \phi(a) * \phi(b)\inv + \phi(c) * \phi(d)\inv = $$
$$= \phi(a) * \phi(b)\inv * 1 + \phi(c) * \phi(d)\inv * 1 =
\phi(a) * \phi(b)\inv * 1 + \phi(c) * \phi(d)\inv * 1 = $$
$$ =
\phi(a) * \phi(b)\inv * \phi(d) * \phi(d)\inv +
\phi(c) * \phi(d)\inv * \phi(b) * \phi(b)\inv = $$
$$ =
\phi(ad) * \phi(b)\inv * \phi(d)\inv +
\phi(cb) * \phi(d)\inv  * \phi(b)\inv = $$
$$=  (\phi(ad) + \phi(cb)) * \phi(b)\inv * \phi(d)\inv =
(\phi(ad + cb)) * \phi(bd)\inv $$
which states that addition is fine, and we also have that
$$\lambda((a, b) * (c, d)) = \lambda((ac, bd))= \phi(ac) * \phi(bd)\inv$$
$$\lambda((a, b)) * \lambda((c, d)) =
\phi(a) * \phi(b)\inv * \phi(c) * \phi(d)\inv = \phi(ac) * \phi(bd)\inv$$
which provess that multiplication is also fine, thus proving that
$\lambda$ is indeed a homomorphism, as desired

\subsection{}

\textit{Let $F$ be a field. An informal definition of the projective
  line is that it is the union
  $$P^1(F) = F \cup \set{\infty}$$
  of the field $F$ and one extra point "at infinity". Then for $\alpha
  \in F$, we define an evaluation map
  $$E_\alpha: F(X) \to P^1(F)$$
  as follows. For rational functions $\phi \in F(X)$, we write
  $\phi(X) = f(X)/g(X)$ as a ration of polynomials having no common
  factors in $F[X]$, and then we set
  $$E_\alpha(\phi) =
  \begin{cases}
    f(\alpha)/g(\alpha) \text{ if } g(\alpha) \neq 0 \\
    \infty \text{ if } g(\alpha) = 0 \\
  \end{cases}
  $$
}

\textit{(a) Prove that $E_\alpha(\phi)$ is well-defined; i.e. that is
  does not depend on the choice of $f$ and $g$.}

Let $\phi \in F(X)$. We know that $F(X)$ is a field of fractions, and
thus assume that $(f_1, g_1), (f_2, g_2)$ are both representations of
$\phi$. We follow that by definition
$$f_1 g_2 = f_2 g_1$$
We know also that $F[X]$ is PID, and thus a UFD. Therefore we follow
that we can deconstruct $f_1, g_1$ into irreductible polynomials, and
then remove common factors, (or the factors that differ by multiplication
by a unit). Thus we can get $f, g \in F[X]$ with such property.
Since we've removed common factors, we follow that there is a
product of those factors $q$, and thus we have that
$$f_1 = fq, g_1 = gq$$
and thus we have that
$$f_1 gq = fq g_1$$
and since $F[X]$ is an integral domain we follow that
$$f_1 g = f g_1$$
thus concluding that
$$(f_1, g_1) \sim (f, g)$$
thansitivity of $\sim$ implies that
$$(f_2, g_2) \sim (f, g)$$

Now we've proven that two representations of a rational "function" can
be mapped into one single pair of polynomials, that have no common
factors, but we don't have that outputting pair is unique. And it does
not have to be, since we can differ for example factors of polynomials
in $R$ by a unit, and get different pair. Thus we still need to prove
that if $(f, g)$ and $(f', g')$ are both polynomials, that have
no common factors that are equivalent to each other, then their
quotient at $\alpha$ is the same.

Since $(f, g) \sim (f', g')$ we have that
$$fg' = f'g$$
and therefore we have
$$(fg')(\alpha) = (f'g)(\alpha)$$
or in other words
$$E_\alpha(f g') = E_\alpha(f' g)$$
evaluation of $F[X]$ is a ring homomorphism, thus giving us that
$$E_\alpha(f g') = E_\alpha(f) E_\alpha(g')  = E_\alpha(f g') = E_\alpha(f') E_\alpha(g)$$
or in other words
$$f(\alpha) g'(\alpha) = f'(\alpha) g(\alpha)$$
we then can follow that if $g'(\alpha), g(\alpha) \neq 0$, then
$$f(\alpha) g(\alpha)\inv = f'(\alpha) g'(\alpha) \inv$$
which gets us the desired result.
If $g(\alpha) = 0$ and $g'(\alpha) \neq 0$, then we follow that
there is $q \in F[X]$ such that
$$g = q(x - \alpha)$$
by exercise 5.1. Since $g'(\alpha) \neq 0$ we follow that $(x - \alpha)$ is
not a factor of $g'$. But we have that
$$fg'(\alpha) = f'g(\alpha) = 0$$
$$f(\alpha) g'(\alpha) = 0$$
$$f(\alpha) = 0$$
thus implying that $f$ is also factored by $(x - \alpha)$, thus implying that
$f$ and $g$ have common factors, which is false.
And in conclusion we have that if $g(\alpha) = g'(\alpha) = 0$, then
$E_\alpha(\phi) = \infty$, which implies the desired result.

\textit{Skip this one}

\subsection{}

\textit{This exercise describes a generalization of the construction
  of the field of fractions. Let $R$ be a ring. A subset $S \subseteq R$ is
  said to be multiplicatively closed if it has the following properties:
  $$1 \in S \text{ and } 0 \notin S$$
  and
  $$\text{If } a, b \in S \text{ then } ab \in S$$
  We define an equivalence relation $\sim_S$ on the set of paints $(a,
  b) \in R \times S$ by
  $$(a, b) \sim_S (a', b') \iff cab' = ca'b \text{ for some } c \in S$$
  and we write $R_S$ for the set of equivalence classes. We define
  addition and multiplication on $R_S$ exactly as we did for the field
  of fractions.
}

Little note: for this chapter we assume that rings are all associative,
and also we are not assuming that $R_S$ is a afield, only indicating that
it is a ring.

\textit{(a) Prove that $\sim_S$ is an equivalence relation}

We follow that for $(a, b), (a', b'), (a'', b'') \in R \times S$ we have
$$ab = ab \ra (a, b) \sim_S (a, b)$$
thus giving us reflexivity. We also follow that if
$$(a, b) \sim_S (a', b')$$
then there is $c \in S$ such that 
$$cab' = ca'b \ra (a', b') \sim_S (a, b)$$
thus giving us symmetry. We also have that if
$$(a, b) \sim_S (a', b') \land (a', b') \sim_S (a'', b'') $$
then there are $c, c' \in S$ such that
$$cab' = ca'b \land c'a'b'' = c'a''b'$$
and therefore we can just as before multiply both sides to get
$$cab'c'a'b'' = ca'bc'a''b' $$
$$cb'c'a'(ab'') = ca'c'b'(a''b) $$
setting $q = cb'c'a'$ we get
$$q (ab'') = q(a''b)$$
thus proving that
$$(a, b) \sim_S (a'', b'')$$
thus giving us transitivity and the desired result.

\textit{(b) Prove that addition and multiplication on $R_S$ are
  well-defined.}

Let $(a_1, b_1), (a_1', b_1'), (a_2, b_2), (a_2', b_2') \in R \times
S$ be such that
$$(a_1, b_1) \sim_S (a_1', b_1'), (a_2, b_2) \sim_S (a_2', b_2')$$
We follow that
$$(a_1, b_1) + (a_2, b_2) = (a_1 b_2 + a_2 b_1, b_1 b_2)$$
$$(a_1', b_1') + (a_2', b_2') = (a_1' b_2' + a_2' b_1', b_1' b_2')$$
thus we need to show that there is $c \in S$ such that 
$$c (a_1 b_2 + a_2 b_1) (b_1' b_2') = c (a_1' b_2' + a_2' b_1') (b_1 b_2)$$
we then pull the same trick as in the book
$$0 = 0$$
$$(c_1 a_1b_1' - c a_1' b_1 ) = (c_2 a_2' b_2 - c' a_2  b_2')$$
$$c_1 (a_1b_1' - a_1' b_1 ) = c_2 (a_2' b_2 -  a_2  b_2')$$
we then multiply lhs by $c_2$ and rhs by $c_1$ (which we can do since both are still zero)
$$c_1 c_2 (a_1b_1' - a_1' b_1 ) = c_1 c_2 (a_2' b_2 -  a_2  b_2')$$
we then set $c = c_1 c_2$ and get that 
$$c b_2 b_2' (a_1b_1' - a_1' b_1 ) = c b_1 b_1'(a_2' b_2 - a_2  b_2')$$
$$c a_1 b_2 b_1' b_2' - c a_1' b_2' b_1 b_2 = c a_2' b_1' b_1 b_2 - c a_2 b_1 b_1' b_2'$$
$$c a_1 b_2 b_1' b_2' + c a_2 b_1 b_1' b_2' = c a_1' b_2' b_1 b_2 + c a_2' b_1' b_1 b_2$$
$$(c a_1 b_2 + c a_2 b_1) (b_1' b_2') = (c a_1' b_2' + c a_2' b_1') (b_1 b_2)$$
$$c (a_1 b_2 + a_2 b_1) (b_1' b_2') = c (a_1' b_2' + a_2' b_1') (b_1 b_2)$$
which is precisely the desired result.

We could've also simplified this proof by firstly stating that if
$$(a_1, b_1) \sim_S (a_1', b_1'), (a_2, b_2) \sim_S (a_2', b_2')$$
then there are $c_1, c_2 \in S$ such that
$$c_1 a_1 b_1' = c_1 a_1' b_1, c_2 a_2 b_2' = c_2 a_2' b_2$$
and therefore we follow that
$$c_1c_2  a_1 b_1' = c_1c_2 a_1' b_1, c_1 c_2 a_2 b_2' = c_1 c_2 a_2' b_2$$
and thus there is a single $c = c_1 c_2 \in S$ (product is in $S$
since it's multiplicatively closed) such that
$$c a_1 b_1' = ca_1' b_1, c a_2 b_2' = c a_2' b_2$$

For multiplication we gotta prove that
$$(a_1, b_1) * (a_2, b_2) = (a_1 a_2, b_1 b_2)$$
$$(a_1', b_1') * (a_2', b_2') = (a_1' a_2', b_1' b_2')$$
By our assumptions and previous paragraph we have that there is $c \in
S$ such that
$$ca_1' b_1 = ca_1 b_1'$$
$$ca_2' b_2 = ca_2 b_2'$$
which then workd out in the same manner, as our proof for intergal
domains

\textit{(c) Prove that $R_S$ with the indicated addition and
multiplication is a ring. The ring $R_S$ is called the localization of
$R$ in $S$.}

We firstly follow that the identitiies proof is the same as in the
usual field of fractions, same goes for distributivity. For identity
we need to prove that $(0, 1) \not \sim_S (1, 1)$, thus firstly assume
that $(0, 1) \sim_S (1, 1)$.  Then we have that there is $c \in S$
such that
$$0 * 1 * c = 1 * 1 * c \ra 0 = 1$$
which is not true by the assumption that $R$ is a ring.

The fact that
$R_S$ under plus is commutative follows directly from the definition,
we follow that for $(a, b) \in R_S $ we have that
$$(a, b) + (-a, b) = (ab - ba, bb) = (0, ab) = (0, 1)$$
thus proving inverses for the addition. We also follow that
$$((a_1, b_1) + (a_2, b_2)) + (a_3, b_3) =
(a_1 b_2 + a_2 b_1, b_1 b_2) + (a_3, b_3) =
(a_1 b_2 b_3 + a_2 b_1 b_3 + a_3 b_1 b_2, b_1 b_2 b_3)$$
$$(a_1, b_1) + ((a_2, b_2) + (a_3, b_3)) =
(a_1, b_1) + (a_2 b_3 + a_3 b_2, b_2 b_3) = 
(a_1 b_2 b_3 + a_2 b_3 b_1 + a_3 b_2 b_1, b_1 b_2 b_3)$$
thus giving us the associativity and proving that $R$ is
indeed an abelian group under $+$.

The only thing that is left for the ring properties is associativity,
which is also trivial:
$$((a_1, b_1) * (a_2, b_2)) * (a_3, b_3) = (a_1 a_2 a_3, b_1 b_2 b_3) =
(a_1, b_1) * ((a_2, b_2) * (a_3, b_3))$$

One note: we see that in all of the given proofs we have
denominators being products of elements of $S$, which by assumption
of $S$ give us the fact that they aren't 0. 

Here we are also note why this thing is a ring, and not a field:
$R_S$ is a cartesian of $R$ and $S$, which implies that if
$q \in R \setminus S$ and $q \neq 0$, then $(q, 1) \neq 0$ for example
may not have an inverse in $R_S$ since $(1, q) \notin R_S$.  If $R
\setminus S = \set{0}$ however, then
$$(a, b) * (b, a) = (ab, ab)$$
but here we can have a problem that there might be the case that $ab =
0$ since $R$ is not an integral domain, which would imply that $(ab,
ab) \notin R_S$.

\textit{(d) Suppose that $R$ is an integral domain and that $S$ is
  multiplicatively closed. Prove that $R_S$ is a subring of the field
  of fractions of $R$.}

If $R$ is an integral domain, then we follow that there is indeed a
field of fractions. $R_S$ in this case by the way it's defined is
indeed a subset of a field of fractions, with the same addition and
multiplication functions, and as proven in the previous point it is a
ring, thus making it a subring of the field of fractions.

\textit{(e) Let $I$ be an ideal of $R$ and let
  $$S = R \setminus I$$
  Prove that $S$ is multiplicatively closed if and only if $I$
  is a prime ideal.}

Here we're gonna assume that $I$ is given as an ideal, and then
multiplicative closuse of $S$ is followed directly, as indicated in the
definition in the book.

\textit{(f) Let $b \in R$ be an element, and let
  $$\eangle{b} = \set{1, b^1, b^2, ...}$$
  be the set of powers of $b$. Prove that $\eangle{b}$ is multiplicatively
  closed if and only if $b$ is not nilpotent.}

Suppose that $\eangle{b}$ is multiplicatively closed. We follow that
there are no $q, w \in \eangle{b}$ such that $qw = 0$ since $qw \in
\eangle{b}$ and $0 \notin \eangle{b}$. Thus we conclude that there
are no $m, n \in Z$ such thaht $b^m b^n = 0$, and thus $b$
is not nilpotent, as desired.

If $b$ is not nilpotent, then we follow that there is no $m \in Z$
such that $b^m = 0$. We follow that $1 \in \eangle{b}$,
$0 \notin \eangle{b}$, and if $b^m, b^n \in \eangle{b}$ for some
$m, n \in Z$ then $b^mb^n \in \eangle{b}$, as desired.

\textit{Skipping the rest due to triviality}

\chapter{Fields - Part 2}

\section{Algebraic Numbers and Transcendental Numbers}

\subsection{}

\textit{Let $F$ be a field, let $f(x) \in F[x]$ be a non-zero
  polynomial, and let $\alpha$ be a root of $f(x)$ in some extension
  field of $F$. Prove that
  $$[F(\alpha):F] = \deg(f) \ra f(x) \text{ is irreductible in } F[x]$$
}

By definition we have that
$$[F(\alpha):F] = \dim_F(F(\alpha))$$
and thus
$$\dim_F(F(\alpha)) = \deg(f) = \dim_F F[x]/f(x)F[x]$$
By the same reasoning as in proof of 8.6(b) we have that there is a
well-defined surjective ring homomorphism
$$F[x]/f(x)F[x] \to F(a)$$
which is also a linear map between two $F$-spaces.
Since this map is a linear map of spaces with the same dimension,
we follow that surjectivity here implies bijectivity (see LADR for
clarifications), thus making this map an isomorphism, which implies that
$F[x]/f(x)F[x]$ is a field, for which by 7.16 we follow that $f(x)$
is irreductible, as desired.

\subsection{}

\textit{Prove that each of the following numbers is algebraic over $Q$
  by finding a polynomial in $Q[x]$ for which it is a root:}

\textit{(a) $\sqrt[n]{c}$ where $c \in Q$ and $n \geq 1$.}

We can say that
$$f(x): x^n - c$$
is such a polynomial.

\textit{(b) $\sqrt{2} + \sqrt{3}$}

We can follow that
$$(\sqrt{2} + \sqrt{3})^2 = 2\sqrt{6} + 5$$
thus
$$((\sqrt{2} + \sqrt{3})^2 - 5) / 2 = \sqrt{6}$$
and therefore
$$(((\sqrt{2} + \sqrt{3})^2 - 5) / 2)^2 = 6$$
$$(((\sqrt{2} + \sqrt{3})^2 - 5) / 2)^2 - 6 = 0$$
then we sub $\sqrt{2} + \sqrt{3}$ for $x$ and get
$$((x^2 - 5) / 2)^2 - 6 = 0.25 (x^2 - 5)^2 - 6 = 0.25 (x^4 - 10x^2 + 25)  - 6 =
\frac{1}{4}x^4 - \frac{10}{4}x^2 + \frac{1}{4}$$
which is the desired polynomial.

\textit{(c) $\sqrt[3]{2} + \sqrt{3}$.}

We can follow that
$$(\sqrt[3]{2} + \sqrt{3})^0 = 1$$
$$(\sqrt[3]{2} + \sqrt{3})^1 = \sqrt[3]{2} + \sqrt{3}$$
$$(\sqrt[3]{2} + \sqrt{3})^2 = (\sqrt[3]{2})^2 + 2\sqrt[3]{2}\sqrt{3} + 3$$
$$(\sqrt[3]{2} + \sqrt{3})^3 = 2 + 3 \sqrt[3]{2}^2 \sqrt{3} + 9 \sqrt[3]{2}  + 3\sqrt{3}$$
and so on. In general we can later create a matrix with coefficients
in
$$1, \sqrt[3]{2}, (\sqrt[3]{2})^2, \sqrt{3}, \sqrt{3}\sqrt[3]{2}, \sqrt{3}(\sqrt[3]{2})^2$$
from powers to coefficients, to get a surjective matrix with nonzero
kernel, which would in turn give us the desired polynomial (we need to go for at least
6th power here, and that's too tedious).

\subsection{}

\textit{This exercise sketches a proof of the following resut, which says that if a number is the root
  of a polynomial in $Q[x]$, then it cannot be too closely approximated by rational numbers.}

\textit{Let $f(x) \in Q[x]$ be a polynomial of degree $d \geq
1$. There is a positive constant $C_f > 0$ such that if $\alpha \in C
\setminus Q$ is a non-rational root if $f(x)$, then
$$|p/q - \alpha| \geq C_f/q^d \text{ for all }p/q \in Q$$}

Skipping the rest due to the fact that this exercise is wrong, and
fixing it is a mess

\section{Polynomial Roots and Multiplicative Subgroups}

\subsection{}

\textit{Suppose that $R$ is a commutative ring that is not an integral
  domain. Prove that threre is a polynomial $f(x) \in R[x]$ that has more distinct
  roots in $R$ than its degree.}

Let $a, b \in R$ be nonzero and such that $ab = 0$. We can follow that
$$(x - a)(x - b) = x^2 - bx - ax + ab = x^2 + (-a - b)x + ab = x^2 + (-a - b)x$$
We then follow that roots for this polynomial are $a, b, 0$, which
gives us the desired result.

\subsection{}

\textit{Let $R$ be a ring, let $a_1, ..., a_n \in R$, and define a set of polynomials
  $$K_a = \set{f(x) \in R[x]: f(a_1) = ... = f(a_n) = 0}$$}

\textit{(a) Prove that $K_a$ is an ideal of $R[x]$.}

Let $f, g \in K_a$. We follow that
$$(f + g)(a_j) = f(j) + g(j) = 0 + 0 = 0$$
and if $q \in R$ then
$$(fq)(a_j) = f(j) q(j) = 0 * q(j) = 0$$
thus proving to us that $K_a$ is indeed an ideal, as desired.

\textit{(b) Suppose that $R$ is an integral domain, and that $a_1,
..., a_n$ are distinct. Prove that $K_a$ is the principal ideal of
$R[x]$ generated by the polynomial $(x - a_1) ... (x - a_n)$.}

Firstly we follow that
$$(x - a_1) ... (x - a_n) \in K_a$$
which is pretty evident.
Then we follow that if $f \in K_a$ then it can be destructed as
$$(x - a_1) ... (x - a_n) g(x)$$
by theorem 8.8(b), which implies the desired result.

\subsection{}

\textit{Let $H$ be the ring of quaternions.}

\textit{(a) It was noted that the quadratic polynomial $x^2 + 1$ has
  more than two roots in $H$. Explain which step of the proof Theorem
  8.8 goes wrong for $H$.}

Theorem 8.8 is not applicable to $H$ due to the fact that $H$ is a
non-commutative ring. For the case of (a) we can follow that
$$x^k - a^k = (x - a) \sum_{i = 0}^{k - 1}{x^{k - i - 1} a^i }$$
is wrong:
$$i^2 - j^2 = (-1) - (-1) = 0$$
$$(i - j)(i + j) = i^2 + ij - ji + j^2 = (-1) + k + k + (-1) = k - 2$$

\textit{(b) Prove that $x^2 + 1$ has infinitely many roots in $H$.}

We can follow that for some $h \in H$ we have that
$$h = q + xi + yj + zk$$
for some $q, x, y, z \in R$, and thus we have that 
$$h^2 = (q + xi + yj + zk)^2 = (q + xi + yj + zk)(q + xi + yj + zk) = $$

$$= (q^2 + qxi+ qyj + qzk) + (qxi - x^2 + xyij + xzik) + $$
$$+ (qyj + xyji - y^2 + yzjk) + (qzk + zxki + zykj - z^2) = $$

$$= (q^2 + qxi+ qyj + qzk) + (qxi - x^2 + xyk - xzj) + $$
$$+ (qyj - xyk - y^2 + yzi) + (qzk + zxj - zyi - z^2) = $$

$$= (q^2 - x^2 - y^2 - z^2) + i(qx + qx + yz - zy) +
j(qy - xz + qy + zx) + k(qz + xy - xy + qz) = $$
$$= (q^2 - x^2 - y^2 - z^2) + (2qx) i + (2qy) j + (2qz) k$$

We then follow that if we set $q = 0$, then $(2qx) i = (2qy) j = (2qz) k = 0$.
Thus we only need to find some $x, y, z \in R$ so that
$$- x^2 - y^2 - z^2 = -1$$
in order to find appropriate root. For example we can set $x = 0$ and get
$$- y^2 - z^2 = -1$$
$$- y^2  = -1 + z^2$$
$$y^2  = 1 - z^2$$
$$y  = \sqrt{1 - z^2}$$
we then follow that for any $z \in [-1, 1]$ there  is an appropriate $y$, which
gives us that
$$(\sqrt{1 - z^2} j + zk)^2 = -1$$
which gives us the desired result.

\textit{(c) Describe all the roots of $x^2 + 1$ in $H$}

This can be easily derived from our previous formula for the square.

\textit{(d) Skip}

\subsection{}

\textit{Let $F$ be a finite field of order $q$, and assume that $q$ is
  odd.}

\textit{(a) Let $a, b \in F^*$. If $a^2 = b^2$, prove that either $a =
  b$ or $a = -b$.}

Here we only use the assumption that $F$ is an integral domain. Namely we have that
$$a^2 = b^2 \lra a^2 - b^2 = 0 \ra (a - b) (a + b) = 0 \ra a = b \lor a = -b$$
as desired.

\textit{(b) Show by the way of example that (a) is not true for the
  rings $Z/8Z$ and $Z/15Z$}

We follow that
$$4^2 = 16 \equiv 0 = 0^2$$
and
$$11^2 = 121 \equiv 1 \equiv 196 = 14^2$$
as desired.

\textit{(c) Let
  $$R = \set{a^2: a \in F^*}, N = F^* \setminus R$$
  be, respectively, the set of squares and non-squares in $F^*$. Prove
  that $R$ and $N$ each contain exactly $(q - 1) / 2$ elements.
}

Since $F^*$ is a finite subset of itself, and $F$ is a finite field,
we follow that $F^*$ is a cyclic group under multiplication. This
gives us the fact that there is $w$ such that for each $a \in F^*$
there is \ $n \in N$ such that
$$a = w^n$$
thus we then follow that either $n$ is odd or even, which in
conjuction with the fact that $|F^*| = q - 1$ gives us the desired
result.

\textit{(d) Let $f(x)$ be the polynomial
  $$f(x) = x^{(q - 1)/2} - 1$$
  Prove that $R$ is exactly the set of roots of $f(x)$ in $F$.}

We follow that if $a^2 \in R$ then
$$(a^2)^{(q - 1)/2} = a^{q - 1}$$
since $a \in F^*$ we follow that its order $j$ (which is finite since
$F^*$ is finite) divideds the order of $F^*$, and thus
$$a^{q - 1} = a^{jk} = e^k = 1$$
which gives us that all elements of $R$ are roots of the given
polynomial. Since
$$|R| = (q - 1)  / 2$$
we follow that $f$ is fully destructed by elements of $R$, which
implies that $R$ is a comlete set of roots of $f$, as desired.

\textit{(e) Let $c \in F^*$. Prove that
$$c^{(q - 1)/2)} \equiv
  \begin{cases}
    c \in R \ra 1 \\
    c \in N \ra -1
  \end{cases}
$$}

Presuming that there is a typo and the author meant $R$ instead of $Q$.

We follow that if $x \in F^*$ then
$$x^{q - 1} = 1$$
we also follow that
$$(x^{(q - 1)/2} - 1)(x^{(q - 1)/2} + 1) = x^{q - 1} - 1$$
The fact that $(x^{(q - 1)/2} - 1)$ destructs by $R$ is the previous
point, which gives us the desired resut.

\end{document}
