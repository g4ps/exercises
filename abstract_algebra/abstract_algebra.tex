\documentclass[11pt,oneside,titlepage]{book}
\title{My abstract algebra exercises}
\usepackage{amsmath, amssymb}
\usepackage{geometry}
\usepackage{hyperref}
\author{Evgeny Markin}
\date{2023}

\DeclareMathOperator \map {\mathcal {L}}
\DeclareMathOperator \pow {\mathcal {P}}
\DeclareMathOperator \ns {null}
\DeclareMathOperator \range {range}
\DeclareMathOperator \inv {^{-1}}
\DeclareMathOperator \Span {span}
\DeclareMathOperator \imp {\Rightarrow}
\DeclareMathOperator \lra {\Leftrightarrow}
\DeclareMathOperator \eqv {\Leftrightarrow}
\DeclareMathOperator \la {\Leftarrow}
\DeclareMathOperator \ra {\Rightarrow}
\DeclareMathOperator \true {true}
\DeclareMathOperator \false {false}
\DeclareMathOperator \dom {dom}
\DeclareMathOperator \ran {ran}
\newcommand{\eangle}[1]{\langle #1 \rangle}
\newcommand{\set}[1]{\{ #1 \}}


\begin{document}
\maketitle
\tableofcontents

\part{Preliminaries}

\chapter{Relations and Functions}

\chapter{The Integers and Modular Arithmetic}

\part{Groups}

\chapter{Introduction to Groups}

\section{An Important Example}

\subsection{}

\textit{In $S_4$, let $\sigma = \
  \begin{pmatrix}
    1 & 2 & 3 & 4 \\
    3 & 1 & 4 & 2
  \end{pmatrix}
  $,
  and $\tau =
  \begin{pmatrix}
    1 & 2 & 3 & 4 \\
    3 & 4 & 1 & 2
  \end{pmatrix}
  $. Calculate $\sigma \tau$, $\tau \sigma$ and $\sigma \inv$.
}

$$\sigma \tau =
\begin{pmatrix}
  1 & 2 & 3 & 4\\
  4 & 2 & 3 & 1 
\end{pmatrix}
$$
$$ \tau \sigma =
\begin{pmatrix}
  1 & 2 & 3 & 4\\
  1 & 3 & 2 & 4 
\end{pmatrix}
$$
$$\sigma \inv =
\begin{pmatrix}
  1 & 2 & 3 & 4 \\
  2 & 4 & 1 & 3
\end{pmatrix}
$$

\subsection{}

\textit{In $S_5$, let $\sigma =
  \begin{pmatrix}
    1 & 2 & 3 & 4 & 5 \\
    5 & 3 & 2 & 1 & 4
  \end{pmatrix}
  $
  and
  $\tau =
  \begin{pmatrix}
    1 & 2 & 3 & 4 & 5 \\
    2 & 4 & 1 & 3 & 5
  \end{pmatrix}
  $
  calculate $\sigma \tau \sigma$, $\sigma \sigma \tau$, $\sigma \inv$.
}

$$\sigma \tau \sigma =
\begin{pmatrix}
  1 & 2 & 3 & 4 & 5 \\
  4 & 5 & 1 & 3 & 2
\end{pmatrix}
$$
$$\sigma \sigma \tau =
\begin{pmatrix}
  1 & 2 & 3 & 4 & 5 \\
  2 & 5 & 4 & 3 & 1 
\end{pmatrix}
$$
$$\sigma \inv =
\begin{pmatrix}
  1 & 2 & 3 & 4 & 5 \\
  4 & 3 & 2 & 5 & 1
\end{pmatrix}
$$

\subsection{}

\textit{How mamy permutations are there in $S_n$? In $S_5$, how many permutations $\alpha$
  satisfy $\alpha(2) = 2$?}

We can follow that there are $n!$ permutations total, and if we've got a restriction
$\alpha(2) = 2$, then we've got $(n - 1)!$ permutation. For the case $S_5$ it means
that there are $4! = 24$ such permutations.

\subsection{}

\textit{Let $H$ be the set of all permutations $\alpha \in S_5$ satisfying $\alpha(2) = 2$.
  Which of the properties of closure, associativity, identit, inverses does $H$ enjoy
  under composition?}

All of them

\subsection{}

\textit{Consider the set of all functions from $6$ to $6$. Which of the ...}

Everything other then inverse

\subsection{}

\textit{Let $G$ be the set of all ...}

All of them

\section{Groups}

\subsection{}

\textit{Give group tables for following additive grops: $Z_3$, $Z_3 \times Z_2$}

$$
\begin{tabular}{c | c c c}
   &  0 & 1 & 2 \\
  \hline
  0 & 0 & 1 & 2 \\
  1 & 1 & 2 & 0 \\
  2 & 2 & 0 & 1 \\
  \hline
\end{tabular}
$$

last one is ommited

\subsection{}

\textit{Give group tables for the following groups: $U(12), S_3$}

We follow that $U(12) = \set{1, 5, 7, 11}$. THus
$$
\begin{tabular}{c | c c c c}
   &  1 & 5 & 7 & 11 \\
  \hline
  1 & 1 & 5 & 7 & 11 \\
  5 & 5 & 1 & 11 & 7 \\
  7 & 7 & 11 & 1 & 5 \\
  11 & 11 & 7 & 5 & 1 \\
  \hline
\end{tabular}
$$

One of the programs in progs folder produces desired table for $S_3$ (and can
produce one for any $S_n$ for that matter).

\subsection{}

\textit{Show that $G \times H$ is abelian iff $G$ and $H$ are both abelian}

Was proven in dummit and foote, check 1.1.29

\textit{Rest of the exercises in this section were either already proven in D\&F, are trivial, or
could be solved at a later time if I encounter some gaps in the theory.}

\section{}
\section{}
\section{}

\section{Cyclic Groups}

\subsection{}

\textit{Let $G = \eangle{a}$ be a cyclic group of order 12. List every subgroup of $G$.
List every group of $Z_{12}$}

12's divisors are $\set{1, 2, 3, 4, 6, 12}$, therefore subgroups of $G$ are
$\eangle{a^i}$ for $i \in \set{0, 1, 2, 3, 4, 6}$

Since $Z_{12}$ is cyclic, we follow that  $\eangle{[0, 1, 2, 3, 4, 6]}$
are the subgroups of $Z_{12}$.

\subsection{}

\textit{Let $G = \eangle{a}$ be a cyclic group of order 120. List all of the groups of order 120.
  List all of the elements of order 12 in $G$. }

Divisors of $120$ are $\set{1, 2, 3, 4, 5, 6, 8, 10, 12, 24, 60, 120}$, thus
we can state that subgroups of a cyclic group are $a$ to powers of those numbers

According to the theorems, there should be $\phi(12) = 4$ elements of order 12.
All of them lie in a subgroup $\eangle{a^{120 / 12}} = \eangle{a^{10}}$
and are in form $(a^{10})^k$ where $k \in {1, 5, 7, 11}$. 

\textit{How many element of order 12 are there in a cyclic group of order 1200?}

Also 4.

\subsection{}

\textit{Let $p$ be a prime  and $n$ a positive integer. Show that $\phi(p^n) = p^n - p^{n - 1}$}

If $j \in Z_+$ is such that $j = pi$ for some $i \in Z_+$, then we follow that
$(p^n, j) = p$, therefore they are not relatively prime. Suppose that $(p^n, j) = 1$ for some
$j \in Z_+$. Let $S$ be a multiset of prime divisors of $p^nN$ and $T$ be a multiset
of divisors of $j$. Then we follow that $S \cap T = \emptyset$, since otherwise we
would've had that $j$ is a multiple of $p$, which is not relatively prime to $p^n$.
Thus we follow that the set of not relatively prime numbers to  $p^n$ is equal to
the set of multiples of $p$.

We can follow that there are pricicely $p^{n - 1}$ of multiples of $p$ that are
less or equal to $p^n$ (don't think that we need to prove that),
therefore the total amount of numbers that are less or equal to
$p^n$, which are relatively prime to $p^n$ is $p^n - p^{n - 1}$, as desired.

\subsection{}

\textit{Find all positive integers $n$ such that $|U(n)| = 24$.}

We can follow that $\phi(n)$ is an function that tends to infinity
(i.e. for every $n \in Z_+$ there exists $j \in Z_+$ such that
$m > n$ implies that $\phi(m) > j$) since $\phi(n)$ is larger than
the number of prime numbers that is in the set $Z_+ \cap [1, n)$.
Therefore we conclude that there is an upper bound for a number of
numbers $n$ such that $\phi(n) = 24$.

Brute-force shows that those numbers are
$$35, 39, 45, 52, 56, 70, 72, 78, 84, 90$$

Can't come up with a better answer than that, but I'm sure that it's there.

\subsection{}

\textit{Let $G$ be a nonabelian group. If $H$ and $K$ are cyclic subgroups of $G$,
  does it follow that $H \cap K$ is also a cyclyc subgroup? Prove that it does,
  or provide a counterexample.}

We follow that every subgroup has an identity in it, thus $e \in H \cap K$.
Suppose that $j \in H \cap K$. We follow that $j \in H \land j \in K$. Since $H$
and $K$ are both subgroups, we follow that $j \inv \in H \land j\inv \in K$. Thus
$j \inv \in H \cap K$. Therefore $H \cap K$ is closed under inverses.
We can follow also by the same logic that $j, l \in H \cap K$ implies that
$jl \in H \cap K$. Therefore we can conclude that $H \cap K$ is a subgroup.

We can follow that if $H \cap K = \set{e}$, then it's cyclic. We can follow that
$H \cap K$ can be not only a trivial subgroup by setting $H = K$. Suppose that
$H \cap K \neq \set{e}$. By the fact that both $H$ and $K$ are cyclic we follow that
$H \cap K = \set{a^i: i \in \text{ some subset of } Z_+}$.
Since $H \cap K \neq \set{e}$, we follow that there exists an element $a \in G$ and
two sets $H', K' \in \pow(Z_+)$ such that $H = \set{a^i: i \in H'}$ and
$K = \set{a^i: i \in K'}$. Since both $H$ and $K$ are cyclic we follow that
both $H'$ and $K'$ are the sets of multiples of some number. Thus
$H' \cap K'$ is a set of multiples of some number as well (proof ommited). Thus we
follow that $H \cap K = \set{a^i: i \in H' \cap K'}$ is a cyclic group as well.

\subsection{}

\textit{Let $G = \eangle{a}$ be an infinite cyclic. If $m$ and $n$ are positive integers,
find a generator for $\eangle{a^m} \cap \eangle{a^n}$.}

If $m = n$, then $\eangle{a^m} \cap \eangle{a^n} = \eangle{a^m} = \eangle{a^n}$.
Suppose that $m \neq n$. 


\end{document}

%%% Local Variables:
%%% mode: latex
%%% TeX-master: t
%%% End:
