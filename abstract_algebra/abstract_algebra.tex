\documentclass[11pt,oneside,titlepage]{book}
\title{My abstract algebra exercises}
\usepackage{amsmath, amssymb}
\usepackage{geometry}
\usepackage{hyperref}
\author{Evgeny Markin}
\date{2023}

\DeclareMathOperator \map {\mathcal {L}}
\DeclareMathOperator \pow {\mathcal {P}}
\DeclareMathOperator \ns {null}
\DeclareMathOperator \range {range}
\DeclareMathOperator \inv {^{-1}}
\DeclareMathOperator \Span {span}
\DeclareMathOperator \imp {\Rightarrow}
\DeclareMathOperator \lra {\Leftrightarrow}
\DeclareMathOperator \eqv {\Leftrightarrow}
\DeclareMathOperator \la {\Leftarrow}
\DeclareMathOperator \ra {\Rightarrow}
\DeclareMathOperator \true {true}
\DeclareMathOperator \false {false}
\DeclareMathOperator \dom {dom}
\DeclareMathOperator \ran {ran}
\newcommand{\eangle}[1]{\langle #1 \rangle}


\begin{document}
\maketitle
\tableofcontents

\chapter*{Prelinimaries}

\section{Basics}

\subsection{}

\textit{Determine which of the following elements of $A$ lie in $B$}

$M$ is defined to be
$$
\begin{pmatrix}
  1 & 1\\
  0 & 1
\end{pmatrix}
$$

and
$$B = \{x \in A : MX = XM\}$$

thus all of the following are in $B$.
$$
\begin{pmatrix}
  1 & 1\\
  0 & 1
\end{pmatrix}
$$
$$
\begin{pmatrix}
  0 & 0\\
  0 & 0
\end{pmatrix}
$$
$$
\begin{pmatrix}
  0 & 0\\
  0 & 0
\end{pmatrix}
$$
$$
\begin{pmatrix}
  1 & 0\\
  0 & 1
\end{pmatrix}
$$
$$
\begin{pmatrix}
  1 & 0\\
  0 & 1
\end{pmatrix}
$$

\subsection{}

\textit{Prove that $P, Q \in B \ra P + Q \in B$}

Suppose that $P, Q \in B$. Then we follow that
$$(P + Q)M = PM + QM = QM + PM = (Q + P)M$$
where we've used distributive and commutativity under addition for matrices

\subsection{}

\textit{Prove that $P, Q \in B \ra PQ \in B$}

Suppose that $P, Q \in B$. Thus we follow that $PM = MP$ and $QM = MQ$. Thus
$$(PQ)M = PQM = P(QM) = P(MQ) = PMQ = (PM)Q = (MP)Q = M(PQ)$$
as desired.

\subsection{}

\textit{Find conditions on $p, q, r, s$, which determine precisely when
$$
\begin{pmatrix}
  p & q \\
  r & s
\end{pmatrix} \in B
$$ }

$$
\begin{pmatrix}
  p & q \\
  r & s
\end{pmatrix}
\begin{pmatrix}
  1 & 1 \\
  0 & 1
\end{pmatrix} =
\begin{pmatrix}
  p & p + q \\
  r & r + s
\end{pmatrix}
$$
$$
\begin{pmatrix}
  1 & 1 \\
  0 & 1
\end{pmatrix}
\begin{pmatrix}
  p & q \\
  r & s
\end{pmatrix} = 
\begin{pmatrix}
  p + r & q + s\\
  r & s
\end{pmatrix}
$$
thus we follow that we need to have
$$
\begin{pmatrix}
  p + r & q + s\\
  r & s
\end{pmatrix} =
\begin{pmatrix}
  p & p + q \\
  r & r + s
\end{pmatrix}
$$
thus we follow that the matrix is in $B$ if and only if $r = 0$ and $p = s$. (ocave seems
to support this point).

\subsection{}

\textit{Determine whether the following functions $f$ are well-defined: }

\textit{(a)}
$$f: Q \to Z: f(a/b) = a$$
If we assume that $a/b$ is in form, where $b > 0$ and $a/b$ in their lower terms, then the
function is well-defined. Otherwise, we've got that
$$2/4 = 1/2$$
but
$$f(2/4) = 2 \neq 1 = f(1/2)$$

\textit{(b)}
$$f: Q \to Q: f(a/b) = a^2/b^2$$
is indeed well-defined, since for every $a \in Q$ there is only one square.

\subsection{}
\textit{Determine whether the function $f: R^+ \to Z$ defined by mapping a real number $r$
  to the first digit to the right of the decimal point in a decimal expansion of $r$ is
  well-defined.}

This is a somewhat trick question, since we've got that
$$1 = 0.99999999...$$
which in this case gives us that $f$ is not well-defined.

\subsection{}

\textit{Let $f: A \to B$ be a surjective map of sets. Prove that the relation
  $$a \sim b \lra f(a) = f(b)$$
  is an equivalence relation whose equivalence classes are the fibers of $f$.}

$$f(a) = f(a) \ra a \sim a$$
$$(f(a) = f(b) \land f(b) = f(c) \ra f(a) = f(c)) \ra (a \sim b \land b \sim c \ra a \sim c )$$
$$a \sim b \ra f(a) = f(b) \ra f(b) = f(a) \ra b \sim a$$
which gives us reflexive, transitive and symmetric properties, thus $\sim$ is an
equivalence relation.

We follow that if $x \in B$ and $a, b \in f\inv(\{x\})$, then $a \sim b$ by definition.
Suppose that $a \sim b$. Then we follow that $f(a) = f(b)$, therefore
$a \in f\inv(\{f(a)\}) \land b \in f\inv(\{f(a)\})$. Thus we follow that
if $a \sim b$, then they are fibers for the same value.
Thus we follow that $a \sim b$ if and only if $(\exists x \in B)( a, b \in f\inv(\{x\})$.
Thus we follow that fibers of $f$ are indeed the equivalence classes for $\sim$.

\section{Properties of the Integers}

\subsection{}

\textit{Find GCD and LCM for following numbers and find integers $x$ and $y$ such that
  $ax + by = gcd(a, b)$}

\begin{verbatim}
gcd:   1; lcm:        260, 2 * 20 + -3 * 13 = 1
gcd:   3; lcm:       8556, 27 * 69 + -5 * 372 = 3
gcd:  11; lcm:      19800, 8 * 792 + -23 * 275 = 11
gcd:   3; lcm:   21540381, -126 * 11391 + 253 * 5673 = 3
gcd:   1; lcm:    2759487, -105 * 1761 + 118 * 1567 = 1
gcd: 691; lcm:   44693880, -17 * 507885 + 142 * 60808 = 691
\end{verbatim}

\subsection{}

\textit{Prove that if the integer $k$ divides the integers $a$ and $b$, then $k$ divides
  $as + bt$ for every pair of integers $s$ and $t$}

We follow that because $k$ divides both $a$ and $b$ it also divides $(a, b)$.
Since $(a, b)$ divides both $a$ and $b$ we follow that there exist $q, w \in Z$ such that
$a = q(a, b), b = w(a, b)$.
Thus
$$as + bt = q(a, b) + w(a, b) = (q + w) (a, b)$$
thus we follow that $(a, b)$ divides $as + bt$. Since $|$ is transitive, we follow that
$k | (a, b)$ and $(a, b)| as + bt$ implies that $k | as + bt$, as desired.

(We could've actually skip this part, don't know why I've used it)

\subsection{}

\textit{Let $a, b, N$ be fixed integers with $a, b \neq 0$ and let $d = (a, b)$. Suppose that
  $x_0, y_0 \in Z$ are such that $ax_0 + by_0 = N$. Prove that
  $$a(x_0 + \frac{b}{d}t) + b(y_0 - \frac{a}{d}t) = N$$
}

$$a(x_0 + \frac{b}{d}t) + b(y_0 - \frac{a}{d}t) =
a x_0 + a \frac{b}{d}t + by_0 - b\frac{a}{d}t =
a x_0 + by_0 + t( \frac{ab}{d} - \frac{ab}{d}) =$$
$$ = 
a x_0 + by_0 + t( 0) =
N + 0 = N
$$

\subsection{}

\textit{Determine the value $\phi(n)$ for each integer $n \leq 30$ where $\phi$
  denotes the Euler $\phi$-function}

\begin{verbatim}
phi(1) = 1
phi(2) = 1
phi(3) = 2
phi(4) = 2
phi(5) = 4
phi(6) = 2
phi(7) = 6
phi(8) = 4
phi(9) = 6
phi(10) = 4
phi(11) = 10
phi(12) = 4
phi(13) = 12
phi(14) = 6
phi(15) = 8
phi(16) = 8
phi(17) = 16
phi(18) = 6
phi(19) = 18
phi(20) = 8
phi(21) = 12
phi(22) = 10
phi(23) = 22
phi(24) = 8
phi(25) = 20
phi(26) = 12
phi(27) = 18
phi(28) = 12
phi(29) = 28
phi(30) = 8
\end{verbatim}

\subsection{}

\textit{Prove the WOP of $Z$ by induction and prove the minimal element is and prove
  the minimal element is unique.}

GOTO set theory book

\subsection{}

\textit{If $f$ is a prime prove that there do noe exist nonzero integers $a$ and $b$
  such that $a^2 = pb^2$}

We follow that $a$ and $b$ can be represented as multiples of primes. Therefore
the powers of primes, that represent $a^2$ and $b^2$ are even.
Since the power of $p$ in $pb^2$ is not even, we follow that such numebers do not exist,
as desired

\subsection{}

\textit{Let $p$ be a prime, $n \in Z^+$. Find a formula for the largest power of $p$
  which divides $n!$}

We follow that every $p$'th number is a multiple of $p$. Thus the amount of
of multiples of $p$ in the list $1, 2, ..., n$ is $\lfloor n / p \rfloor$.
To those we need to add the number of multiples of $p^2$, of which there will be
$\lfloor n / p^2 \rfloor$, and thus we follow that the number of multiples of $p$ in $n$ is
$$\sum_{i = 1}^n{\lfloor n / p^i\rfloor}$$
Since for every prime  number we've got that $p^n > n$, we can follow that this formula will do.

\subsection{}

\textit{Write a computer program to determine ....}

Way ahead of you, check congr.py in progs.

% \subsection{}

% \textit{Prove for any $N \in Z^+$ there exists only finitely many integers $n$ such that
%   $\phi(n) = N$.}




\section{$Z/n Z$: The Integers Modulo $n$}

\subsection{}
\textit{Write down explicitly all the elements in the residue classes $Z/18Z$.}

$$\overline{1}, \overline{2}, ... , \overline{17}$$

\subsection{}

\textit{Prove that the distinct equivalence classes in $Z/nZ$ are precisely
  $\overline{0}, ..., \overline{n - 1}$.}

Suppose that $q \in N$. We follow that $q = an + r$, where $0 \leq r < n$, thus we follow that
$q \in \overline{r}$. Therefore every integer is in one of those sets. Since $r$ is
unique, we follow that $q$ is only in one of those sets.

\subsection{}

\textit{Prove that of $a = a_n 10^n + a_{n - 1}10 ^{n - 1} + .... + a_1 10 + a_0$ is any positive
  integer then $a \equiv \sum{a_n} \mod 9$.}

We follow that $10 \equiv 1 \mod 9$, and therefore $10^n \equiv 1 \mod 9$ for any $n \in Z$.
Thus we can follow that
$$10 a_n \equiv a_n \mod 9$$
and in general
$$10^n a_n \equiv a_n \mod 9$$
therefore
$$\overline{a_n 10^n} = \overline{a_n}$$
and since
$$\sum {\overline{a_n}} = \overline{\sum a_n}$$
we follow the desired result.

\subsection{}

\textit{Compute the remainder when $37^{100}$ is divided by $29$}

We follow that 
$$37^{100} \equiv 8^{100} \mod {29}$$
thus
$$8^ 1 \equiv 8 \mod{29}$$
$$8^ 2 \equiv 6 \mod{29}$$
$$8^ 4 \equiv 36 \equiv 7 \mod{29}$$
$$8^{8} \equiv 49 \equiv 20 \mod{29}$$
$$8^{10} \equiv 120 \equiv 4 \mod{29}$$
$$8^{20} \equiv 16 \mod{29}$$
$$8^{40} \equiv 256 \equiv 24 \mod{29}$$
$$8^{50} \equiv 96 \equiv 9 \mod{29}$$
$$8^{100} \equiv 81 \equiv 23 \mod{29}$$
thus we follow that $37^{100}$ divided by 29 gives us the answer 23.

\subsection{}

$$9^{1500} = ...01$$

\subsection{}

\textit{Prove that the squares of the elements in $Z/4Z$ are jsut $0$ and $1$}

We follow that
$$0^2 = 0$$
$$1^2 = 1$$
$$2^2 = 4 \equiv 0 \mod 4 $$
$$3^2 = 9 \equiv 1\mod 4 $$
so yeah

\subsection{}

\textit{Prove for any integers $a$ and $b$ that $a^2 = b^2$ never leaves a remainder of $3$
  when divided by 4}

From previous exercise we follow that
$$a^2 \equiv [0, 1] \mod 4$$
$$b^2 \equiv [0, 1] \mod 4$$
thus
$$a^2 + b^2  \equiv [0, 1, 2] \mod 4$$

\subsection{}

\textit{Prove that the equation $a^2 + b^2 = 3c^2$ has no nonzero integer solutions}

We follow from previous exercise that $a^2 + b^2 \equiv [0, 1, 2] \mod 4$, and
$c^2 \equiv [0, 1] \mod 4$, therefore $3c^2 \equiv [0, 3] \mod 4$. Thus we follow that
the only possible case is when $a^2 + b^2 \equiv 3c^2 \equiv 0 \mod 4$. Thus we follow
all of the $a^2$, $b^2$ and $c^4$ have the factor of $4$. Thus there exist
$a_0, b_0, c_0$ such that $a^2 = 4^n a_0^2$, $b^2 = 4^n b_0^2$ $c^2 = 4^n c_0^@$
and $a_0^2, b_0^2, c_0^2$ are not divisible by $4$ (otherwise we get a contradiction).
Thus we follow that
$$a_0^2 +  b_0^2 = 3c_0^2$$
all of which are not divisible by 4, which gets us a contradiction, as desired.

\subsection{}

\textit{Prove that the square of any odd integer always leaves a remainder of $1$ when
  divided by $8$}

We follow that remainders  of squares of congruent classes of $8$ are
$$0 1 4 1 0 1 4 1$$
thus we follow the desired conclusion.

% \subsection{}

% \textit{Prove that the number of elements of $(Z/nZ)^\times$ is $\phi(n)$ where $\phi(n)$
%   denotes the Euler function.}


\chapter{Introduction to Groups}

\section{Basic Axioms and Examples}

\textit{Let $G$ be a group}

\subsection*{1.1.1}

\textit{Determine which of the follow ing binary operations are associative}

\textit{(a) }
$$Z$$
$$a \star b = a - b$$
$$(a - b) - c = a - b - c = a - (b + c)$$
therefore it is not associative

\textit{(b) }
$$R$$ 
$$a \star b = a + b + ab$$
$$(a \star b ) \star c = (a + b + ab) \star c = (a + b + ab) + (c) + (ca + cb + abc)$$
$$a \star (b \star c) = a \star (b + c + bc) = (a) + (b + c + bc) + (ab + ac + abc)$$
$$(a + b + ab) + (c) + (ca + cb + abc) - ( (a) + (b + c + bc) + (ab + ac + abc)) = $$
$$ = 0$$
therefore it is associative.

\textit{(c) }
$$Q$$ 
$$a \star b = \frac{a + b}{5}$$
$$(a \star b ) \star c = \frac{a + b}{5} \star c = \frac{\frac{a + b}{5} + c}{5} =
\frac{a + b + 5c}{25}$$
$$a \star (b \star c) = a \star \frac{b + c}{5} = \frac{a + \frac{b + c}{5}}{5} =
\frac{5a + b + c}{25}$$
therefore it is not associative.

\textit{(d) }
$$Z \times Z$$ 
$$(a, b) \star (c, d) = (ad + bc, bd)$$
$$((a, b) \star (c, d)) \star (e, f) = (ad + bc, bd) \star (e, f) =
((ad + bc) f + bde, bdf) = (adf + bcf + bde, bdf) $$
$$(a, b) \star ((c, d) \star (e, f) = (a, b) \star (cf + de, df) =
(adf + b(cf + de), bdf) = (adf + bcf + bde, bdf)$$
therefore it is associative.

\textit{(e) }
$$Q \setminus \{0\}$$ 
$$a \star b = \frac{a}{b}$$
$$a \star (b \star c) = a \star \frac b c = \frac{a}{\frac{b}{c}} = \frac{ac}{b}$$
$$(a \star b) \star c = \frac{a}{b} \star c = \frac{\frac{a}{b}}{c} = \frac{ac}{b}$$
therefore it is associative.

\subsection*{1.1.2}

\textit{Decide which of the binary operations in the preceeding exercise are commutative}

b and c

\subsection*{1.1.3}

\textit{Prove that addition of residue clases in $Z/nZ$ is associative}

$$\overline{a} + (\overline{b} + \overline{c}) = \overline a + \overline{b + c} =
\overline{a + b + c} = \overline{a + b} + \overline{c} = (\overline{a} + \overline{b} ) +
\overline{c}$$

\subsection*{1.1.4}

\textit{Prove that multiplication of residue clases in $Z/nZ$ is associative}

analogous to previous

\subsection*{1.1.5}

\textit{Prove that for all $n > 1$ that $Z/nZ$ is not a group under multiplication}

We follow that there is no invevrse of $\overline{0}$, $Z/nZ$ is not a group

\subsection*{1.1.6}

\textit{Determine which of the following sets are groups under addition}

Sums are associative for every following group, therefore I'll skip discussion about them.

\textit{(a) Rationals, whose denominators are odd}

Is a group, denominator of the sum is a divisor
of product of two odd numbers, and therefore it is odd itself, and inverses of given
elements have same denominators as the elements themselves.

\textit{(b) Rationals. whose denominators are even}

$$1/2 + 1/2 = 1/1$$
therefore it is not closed under addition, and therefore it is not a group.

\textit{(c) The set of rational numbers of absolute value $< 1$}

$$0.7 + 0.7 = 1.4$$
therefore it is not closed under addition, and therefore it is not a group.

\textit{(d) the set of rationals of absolute value $ \geq 1$, together with $0$}

$$-1.5 + 1 = 0.5$$
therefore it is not closed under addition, and therefore it is not a group.

\textit{(e) the set of rationals, with denominators equal to $1$ or $2$ (or $3$)}

Is a group under addition.

\subsection*{1.1.7}

\textit{Let $G = \{x \in R: 0 \leq x < 1\}$ and for $x, y \in G$ let $x \star y$
  be the fractional part of $x + y$. Prove that $\star$ is a well-defined binary
  operation on $G$ and that $G$ is an abelian group under $\star$.}

Since $\lfloor \cdot \rfloor$ is a well-defined function, we follow that
there exists unique $\lfloor x + y \rfloor$ for given $x, y \in R$,
and therefore $x + y - \lfloor x + y \rfloor$ is a well-defined on $R \times R \to R$.

Let $x, y \in G$. Then we follow that $0 \leq x + y < 2$, therefore
we've got two cases: if $x + y < 1$ or $1 \leq x + y  \leq 2$. In former
case we follow that $\lfloor x + y \rfloor = 0$, therefore
$$0 \leq x \star y = x + y < 1$$. In the latter case we've got that
$$\lfloor x + y \rfloor = 1$$, therefore $x \star y = x + y - 1$ and thus
$0 \leq x \star y = x + y - 1 < 1$. 
Therefore $x, y \in G \ra x \star y \in G$, therefore we can state
that $\star: G \times G \to G$ is a well-defined function on $G$.

Thus
$$x \star (y \star z) = x \star (y + z - \lfloor y + z \rfloor )
= x + y + z - \lfloor y + z \rfloor - \lfloor x + y + z - \lfloor y + z \rfloor  \rfloor $$
$$(x \star y) \star z = (x  + y - \lfloor x + y \rfloor) \star z =
x + y + z - \lfloor x + y \rfloor - \lfloor x + y + z - \lfloor x + y \rfloor \rfloor $$

We follow that for every $n \in Z$ we've got that $\lfloor n \rfloor = n$. It is also
pretty straightforward (although I can't seem to produce a concrete proof for now)
to check that $\lfloor x + n \rfloor = \lfloor x \rfloor + n$
for every $n \in Z$. Since $\lfloor x \rfloor \in Z$ for every $x \in R$ we follow that
$$(x \star y) \star z = $$
$$ = x + y + z - \lfloor y + z \rfloor - \lfloor x + y + z - \lfloor y + z \rfloor  \rfloor  =
x + y + z - \lfloor y + z \rfloor - \lfloor x + y + z   \rfloor + \lfloor y + z \rfloor =
$$
$$ =  x + y + z - \lfloor x + y + z   \rfloor = 
x + y + z - \lfloor x + y \rfloor - \lfloor x + y + z \rfloor + \lfloor x + y \rfloor = $$
$$ =  x + y + z - \lfloor y + z \rfloor - \lfloor x + y + z - \lfloor y + z \rfloor  \rfloor =
x \star (y \star z)$$
therefore $\star$ is an associative function.

By definition, $0 \in G$, therefore
$$0 \star x = 0 + x + \lfloor 0 + x \rfloor = x + \lfloor x \rfloor = x = x \star 0$$
thus we've got the indentity in $G$.

For $0$ we've got that it is an inverse of itself. 
For every $x \in G \setminus \{0\}$  we can follow that $1 - x \in G$ therefore
$1 - x \star x = x + 1 - x + \lfloor 1 \rfloor = 0$
therefore for every $x \in G$ we've got the identity.

Thus we can follow that $\eangle{G, \star}$ is indeed a group.

We also follow that $x \star y = x + y - \lfloor x + y \rfloor
= y + x - \lfloor y + x \rfloor  = y \star x$
therefore given group is also abelian, as desired.

\subsection*{1.1.9}

\textit{Let $G = \{a + b \sqrt{2} \in R: a, b \in Q\}$.}

\textit{Prove that $G$ is a group under addition.}

Let $x, y, z \in G$. Thus 
$$x + y = a_x + b_x \sqrt{2}  + a_y + b_y\sqrt{2}  = (a_x + a_y) + \sqrt{2}(b_x + b_y)$$
therefore $G$ is closed under addition, thus we follow that $+: G \times G \to G$.
Sums are associative in general, therefore gonna skip that.
$0$ is the usual identity, which can be represented as $0  + 0 \sqrt{2}$, thus $0 \in G$.
For $x \in G$ we can define $x \inv = -a_x - b_x \sqrt{2}$, which is also in $G$. Thus
we follow that $\eangle{G, +}$ is indeed a group, as desired.

\subsection*{1.1.11}

\textit{Find the orders of each element of the additive group $Z/12Z$.}

we've got that
\begin{verbatim}
ord (0) = 1
ord (1) = 12
ord (2) = 6
ord (3) = 4
ord (4) = 3
ord (5) = 12
ord (6) = 2
ord (7) = 12
ord (8) = 3
ord (9) = 4
ord (10) = 6
ord (11) = 12
\end{verbatim}

\subsection*{1.1.13}

\textit{Find the orders of the following elements of the additive group $Z/36Z$:
  ...}

we've got that 
\begin{verbatim}
ord (0) = 1
ord (1) = 36
ord (2) = 18
ord (3) = 12
ord (4) = 9
ord (5) = 36
ord (6) = 6
ord (7) = 36
ord (8) = 9
ord (9) = 4
ord (10) = 18
ord (11) = 36
ord (12) = 3
ord (13) = 36
ord (14) = 18
ord (15) = 12
ord (16) = 9
ord (17) = 36
ord (18) = 2
ord (26) = 18
ord (35) = 36
\end{verbatim}

And since $\overline{-1} = \overline{35}$ and so on, we've got the desired result.

\subsection*{1.1.15}

\textit{Prove that $(a_1 a_2 ... a_n)\inv = a_n \inv a_{n - 1}\inv ... a_1 \inv$}

We already know that $(a_1 a_2)\inv = a_2 \inv a_1 \inv$. Suppose that chain of length $n - 1$
has this property. Then we follow that
$$ (a_1 ... a_{n - 1} a_n )\inv = ((a_1 ... a_{n - 1} )(a_n) )\inv =
a_n \inv (a_1 ... a_{n - 1})\inv = a_n\inv a_{n - 1} \inv .... a_1 \inv$$
thus we follow that if $k$ has this property, then $k + 1$ has this property, thus we follow
that this property holds for all $n \geq 2$, as desired.

\subsection*{1.1.17}

\textit{Let $x$ be an element of $G$.  Prove that if $|x| = n$, for some positive integer $n$,
  then $x \inv = x^{n - 1}$.}

We follow that $|x| = n$ means that $x^n = e$, where $e$ denoted identity. Thus
$$e = x^n$$
$$e x \inv  = x^n x \inv$$
$$x \inv  = x^{n - 1} (x x \inv)$$
$$x \inv  = x^{n - 1} (e)$$
$$x \inv  = x^{n - 1}$$
as desired.

\subsection*{1.1.18}

\textit{Let $x, y \in G$. Prove that $xy = yx$ iff $y \inv xy  = x$ iff $x \inv y \inv xy = 1$}

$$xy = yx \lra y \inv x y = y \inv yx \lra y \inv x y = e x \lra y \inv x y = x \lra
x \inv y \inv x y = x \inv x \lra x \inv y \inv x y = e$$
where $e = 1$, and we've got the reverse implication in $\lra$ by cancelation laws.

\subsection*{1.1.21}

\textit{Let $G$ be a finite group and let $x$ be an element of $G$ of order $n$. Prove that
  if $n$ is odd, then $x = (x^2)^k$.}

We follow that $n = 2k - 1$ for some $k \in Z$, thus
$$e = x^n$$
$$e = x^{2k - 1}$$
$$ex = x^{2k - 1}x$$
$$x = x^{2k}$$
$$x = (x^{2})^k $$
as desired.

\subsection*{1.1.23}

\textit{Suppose $x \in G$ and $|x| = n < \infty$. If $n = st$ for some positive integers $s, t$,
  prove that $|x^s| = t$}

We follow that $n$ is the lowest integer such that
$$x^n = e$$
thus
$$x^{st} = e$$
$$(x^{s})^t = e$$
thus $t$ is the smallest integer such that $x^s = e$, therefore $|x^s| = t$, as desired.

\subsection*{1.1.25}

\textit{Prove that if $x^2 = 1$ for all $x \in G$, then $G$ is abelian.}

Suppose that $x, y \in G$. We follow that
$$x^2 = e$$
$$x\inv x^2 = x\inv$$
$$x = x\inv$$
by the same logic
$$y = y\inv$$
and since $xy \in G$ we've got that 
$$xy = (xy)\inv$$
thus
$$xy = (xy)\inv = y\inv x\inv = yx$$
as desired.

\subsection*{1.1.27}

\textit{Prove that if $x$ is an element of the group $G$ then $\{x^n: n \in Z\}$ is
  a subgroup of $G$.}

Let us denote this set by $H$ and let $y, z, w \in H$.
Then we follow that there exist $i, j, k \in Z$ such that
$$y = x^i$$
$$z = x^j$$
$$w = x^k$$
thus
$$yz = x^i x^j = x^{i + j}$$
thus we follow that $H$ is closed under $\star$. We also have that
$$y(wz) = x^{i + j + k} = (yw)z$$
Since $x^0 = 1$, we follow that $1 \in H$. Also, if $x^n \in H$, then $x^{-n} \in H$, and
since $x^n x^{-n} = x^0 = 1$ we follow that every element has an inverse. THus we
conclude that $H$ is indeed a group.

\subsection*{1.1.29}

\textit{Prove that $A \times B$ is an abelian group iff both $A$ and $B$ are
  abelian groups.}

Let $x, y \in A \times B$. Then we follow that
$$xy = yx \lra (a_x, b_x) (a_y, b_y ) = (a_y, b_y) (a_x, b_x ) \lra $$
$$ \lra (a_x a_y, b_x b_y) = (a_y a_x, b_y b_x) \lra a_x a_y = a_y a_x \land b_x b_y = b_y b_x$$
as desired.

\subsection*{1.1.33}

\textit{Let $x$ be an element of finite order in $G$.}

\textit{(a) Prove that if $n$ is odd then $x^i \neq x^{-i}$ for all $i = 1, 2, ..., n - 1$.}

We follow firstly that
$$x \neq x^2 \neq x^3 ... \neq x^{n - 1}$$
becase if we have that $x^i = x^j$ for $1 \leq i < j \leq n - 1$, then
$$x^i = x^j$$
$$x^{-i} x^i  = x^{j- i}$$
$$1  = x^{j- i}$$
which contradicts that $n$ is the order of $x$.

Thus we've got that 
$$x^i \neq x^{-i}$$
is equivalent to 
$$x^{2i} \neq 1$$
If $2i < n$, then we folow that this is given by the fact that order of $x$ is $n$. If $2i \geq n$,
then we follow that $2i < 2n$, therefore $2i - n < n$, and thus 
$$x^n x^{2i - n} \neq 1$$
$$x^{2i - n} \neq 1$$
which is given to us by the fact that $n$ is the order of $x$.



\section{Dihedral Groups}

We firstly state that
$$D_{2n} = \eangle{r, s| r^n = s^2 = 1, rs = sr\inv}$$

\subsection*{1.2.1}

\textit{Compute the order of each of the elements in the following groups:}

In general, we're going to have that
$$|1| = 1$$
$$|r| = n$$
$$|r^j| = lcm(j, n) / j$$
$$|s| = |sr^j| = 2$$

\textit{(a) $D_6$}

$$|1| = 1$$
$$|r| = 3$$
$$(r^2)^2 = r^4 = r r^3 = r$$
$$(r^2)^3 = r^6 = (r^3)^2 = 1^2 = 1$$
$$|r^2| = 3$$
$$|s| = 2$$
$$(sr)^2 = srsr = ssr\inv r = 1$$
$$|sr| = 2$$
$$(sr^2)^2 = sr^2sr^2 = srrsrr = srsr\inv rr = ssr\inv r\inv r r = 1$$
$$|sr^2| = 2$$

\textit{(b) $D_8$}

$$|1| = 1$$
$$|r| = 4$$
$$(r^2)^2 = r^4 = 1$$
$$|r^2| = 2$$
$$(r^3)^2 = r^6 = r^2$$
$$(r^3)^3 = r^9 = r$$
$$(r^3)^4 = r^{12} = 1$$ 
$$|r^3| = 4$$
$$|s| = 2$$
$$|sr| = 2$$
$$srrsrr = srs r\inv rr = ssr\inv r\inv rr = 1$$
$$|sr^2| = 2$$
$$srrrsrrr = srr sr\inv rrr = sr s r\inv r\inv rrr = ss r^{-3}r^3 = 2$$
$$|sr^3| = 2$$

\textit{Not gonna repeat for $D_{10}$, ourlined general case in the beggining of the exercise}

\subsection*{1.2.3}

\textit{Use the generators and relations above to show that every element of $D_{2n}$,
  which is not a power of $r$ has order $2$. Deduce that $D_{2n}$ is generated by the
  two elements $s$ and $sr$, both of which have order $2$.}

We follow that
$$rs = s r\inv$$
and since $(r\inv)\inv = r$, we follow that
$$r\inv s = sr$$
Now suppose that $j \in N$ such that $sr^j = r^{-j}s$. Then we follow that
$$sr^{j + 1} = sr^{j} r = r^{-j} s r = r^{-j} r \inv s = r^{-(j + 1)}s$$
thus we conclude that for every $n \in N$ we've got that $sr^j = r^{-j}s$. Therefore
we can follow that
$$(sr^j)^2 = sr^j s r^j = s r^j r{-j} s = ss = 1$$
therefore $|sr^j| = 2$ for every $j \in Z$.

Suppose that $x \in D_{2n}$. Then we follow that
$x = s^j r^i$
therefore $x = s^{j - i}(sr)^{i}$, as desired.

\subsection*{1.2.5}

\textit{IF $n$ is odd and $n \geq 3$,show that the identity is the only element off $D_{2n}$
  which commutes with all elements of $D_{2n}$.}

Suppose that $x, y \in D_{2_n}$. Then we follow that if $x$ is not the identity, then
$x = r^j$ or $x = sr^j$. Then we follow that if $x = sr^j$, then
$$xr = sr^{j + 1}$$
and
$$rx = r s r^j = s r^{j - 1}$$
thus we follow that $x$ does not commute with $r^i$.

If we let $x = r^j$ then we follow that
$$sx = sr^j$$
and
$$xs = r^j s = s r^{-j} = s r^{n - j}$$
We follow that if $n$ is odd, then there does not exist $j$ such that $j = n - j$, therefore
we follow that $x$ does not commute with $s$.

Thus we can conclude that the only element that commutes with all elements in $D_{2n}$ is
the identity, as desired.

\subsection*{1.2.7}

\textit{Show that $(a, b| a^2 = b^2 = (ab)^n = 1)$ gives a presentation for $D_{2_n}$ }

Let $a = s$ and $b = sr$. Then we follow that 
$$a^2 = s^2 = 1$$
from which we follow that
$$s = s\inv$$
$$(ab)^n = s^n s^n r^n = (s^2)^n r^n = 1 r^n = r^n = 1$$
$$b^2 = 1 \lra (sr)^2 = 1 \lra sr = (sr)\inv \lra sr = r\inv s\inv = sr = r\inv s$$
thus we conclude that given representation is equivalent to our original representation.

\subsection*{1.2.9}

\textit{Let $G$ be the group of rigid motions in $R^3$ of tetrahedron. Show that
  $|G| = 12$.}

We follow that we can send every vertex into another 3 vertices, which gives us 4
motions (don't forget the identity). Then we've got 3 other vertices, with which
we can label the second vertex, thus giving us 24 total cases (in general we've
got that $|G|$ is equal to number of vertices of the solid multiplied by the number of
neighbors of any given vertex.)

\subsection*{1.2.11, 1.2.13}

Same logic as in 1.2.9

\subsection*{1.2.15}

\textit{Find a set of generators and relations for $Z/nZ$.}

$1$ is the obvious canditate, since $j = \sum 1$ for every $j \in Z/nZ$.
We can state that $1^{n + 1} = 1$, which will give us a relation. Not sure how to
show that this is the extensive list of relations, but here's mine.

\subsection*{1.2.17}

\textit{Let $X_{2n}$ be a group whose presentation is displayed in (1.2)}

$$X_{2n} = \eangle{x, y | x^n = y^2  = 1, xy = yx^2}$$

\textit{(a) Show that if $n = 3k$, then $X_{2n}$ has order $6$, and it has the same generators
  and relations as $D_6$ when $x$ is replaces by $r$ and $y$ by $s$.}

We follow that $1, x, y \in X_{2n}$.

If $z \neq  \in X_{2n}$, then it is represented by $z = x^j y^i$ or $z = y^i x^j$. In former
case we can use the relation $y^2 = 1$ to follow that
$$x^j y^i = x^j y$$
from which by induction we can follow that
$$x^j y = y x^{2j}$$
In latter case we follow that $z = yx^j$ or $z = x^j$ by the relation $y^2 = 1$.
In any case we follow that
$$z \in X_{2n} \ra (\exists j \in Z)(z = y x^j \lor z = x^j)$$

Now it would be nice to get that $x^3 = 1$.
From the identity in the chapter we follow that
$$x = x^4$$
and therefore $x^3 = 1$, which is neat.

Since $n = 3k$, we follow that
$$x^n = x^{3k} = (x^3)^k = 1$$
thus we follow that $x^n = 1$ does not restrict our set in any way.

Therefore we follow that all the elements of $X_{2n}$ are $1, x, x^2, y, yx, yx^2$, therefore
it has order $6$, as desired.


Now suppose that we let $x = r$ and $y = s$. Then we follow that we've got relations
$$x^3 = 1$$
$$s^2 = 1$$
$$rs = sr^2 \lra rsr = sr^3 \lra rsr = s \lra sr = r \inv s$$
which gives us the desired correspondense

\textit{(b) Show that if $(3, n) = 1$, then $x$ satisfies additional relation $x = 1$}

We follow that if $(3, n) = 1$, then $3a + qn = 1$, and thus $qn = 1 - 3a$. Thus
$$1 = x^n = x^{qn} = x^n = x^{3a - 1}= (x^3)^a x$$
and since $x^3 = 1$ we follow that
$$1 = x^{3a} x = 1x = x$$
from which we follow that the only elements of $X_{2n}$ are $1, y$. Thus $|X_{2n}| = 2$, as
desired.

\section{Symmetric Groups}

\subsection*{1.3.1}

\textit{Let $\sigma$ and $\tau$ be the given permutations. Find the cycle decomposition
  of their compositions}

$$\sigma = (1, 3, 5)(2, 4)$$
$$\tau = (1, 5)(2, 3)$$
$$\tau \sigma = (1, 2, 4, 3)$$
$$\sigma \tau = (1)(2, 5, 3, 4)$$
$$\tau^2 = (1)(2)(3)(4)(5)$$
$$\tau^2 \sigma = (1, 3, 5)(2, 4)$$

\subsection*{1.3.3}

\textit{For each of the permutations whose cycle decompositions were computed in the preceeding
  (I've got only one) exercises compute its order}

We follow that the general case is the $lcm(l_1, l_2, ...)$, where $l_j$ is the length of
$j$'th cycle


\subsection*{1.3.5}

\textit{Find the order of $(1, 12, 8, 10, 4)(2, 13)(5, 11, 7)(6, 9)$.}

It's 12.

\subsection*{1.3.7}

\textit{Write out the cycle decomposition of each element of order $2$ in $S_4$.}

Skip

\subsection*{1.3.9}

\textit{(a) Let $\sigma$ be the 12-cycle $(1, 2, 3, ..., 12)$. For which positive integers $i$
  is $\sigma^i$ also a 12-cycle}

Ones that have $(i, 12) = 1$

Rest is similar.

\subsection*{1.3.11}

\textit{Let $\sigma$ be the $m$-cycle $(1, 2, ..., m)$. Show taht $\sigma^i$ is also a ...}

Trivial

\subsection*{1.3.13}

\textit{Show that an element has order $2$ in $S_n$ iff its cycle decomposition is a
  product of commuting 2-cycles.}

We follow that $|S_n| = lcm(l_1, ...., l_n)$, thus if we omit 1-cycles, then
$S_n$ has indeed order $2$ iff it's a product of commuting 2-cycles.

\subsection*{1.3.15}

\textit{Prove that the  .... }

Trivial

\subsection*{1.3.17}

\textit{Show that if $n \geq 4$, then the number of permutations in $S_n$ which are the
  product of two disjoint 2-cycles is $n(n - 1)(n - 2)(n - 3)/8$}

Let $C$ denote binomial coefficient. 
We follow that there are $C(n, 2)$ ways to choose the elements in the first cycle,
and $C(n - 2, 2)$ for the second. Thus there are
$$C(n, 2)C(n - 2, 2) = \frac{n(n - 1)}{2} \frac{(n - 2)(n - 3)}{2} =
\frac{n(n - 1)(n - 2)(n - 3)}{4}$$
total elements, if we care about order. Since we don't care about the order of the product,
we follow that we can divide this number by $2! = 2$ to get the number of unordered
products of disjoint cycles, which will be
$$\frac{n(n - 1)(n - 2)(n - 3)}{8}$$
as desired.

\subsection*{1.3.19}

\textit{Find all numbers $n$ such that $S_7$ contains an element of order $n$}

We follow that if element is of order $n$, then $lcm(l_1, l_2, ..., l_n) = n$.
Since $lcm(n, 1, 1, ...) = n$, we follow that all the numbers 1 through $n$ are there.
We can also brute-forse this thing and get that $10, 12$ are also present. $9$ is out,
and so is 8. And that's about it.

\section{Matrix Groups}

\subsection*{1.4.1}

\textit{Prove that $|GL_2(F_2)| = 6$}

We follow that there are only $2$ elements in $F$, therefore there are only 16 matrices
in genenral.

We follow that matrices
$$
\begin{pmatrix}
  1 & 0 \\
  1 & 0
\end{pmatrix}
$$
its four rotation,
$$
\begin{pmatrix}
  1 & 0 \\
  0 & 0
\end{pmatrix}
$$
its four rotation and
$$
\begin{pmatrix}
  0 & 0 \\
  0 & 0
\end{pmatrix}
$$
$$
\begin{pmatrix}
  1 & 1 \\
  1 & 1
\end{pmatrix}
$$
are all non-invertible. Every other one is invertible, so we follow that $|GL_2(F_2)| = 6$,
as desired.

\subsection*{1.4.3}

\textit{Show that $GL_2(F_2)$ is non-abelian}
$$
\begin{pmatrix}
  0 & 1 \\
  1 & 0
\end{pmatrix}
\begin{pmatrix}
  1 & 1 \\
  1 & 0
\end{pmatrix} =
\begin{pmatrix}
  1 & 0 \\
  1 & 1
\end{pmatrix}
$$
$$
\begin{pmatrix}
  1 & 1 \\
  1 & 0
\end{pmatrix}
\begin{pmatrix}
  0 & 1 \\
  1 & 0
\end{pmatrix} = 
\begin{pmatrix}
  1 & 1 \\
  0 & 1
\end{pmatrix}
$$
thus we follow that elements in this group do not commute.

\subsection{1.4.5}

\textit{Show that $GL_n(F)$ is a finite group if and only if $F$ has a finite number of
  elements}

We can follow that of $F$ is finite, then there are only $|F|^{n^2}$ matrices in total,
and invertible matrices are a subset of this set, thus we follow that $GL_n(F)$ is a
finite set.

Conversely, suppose that $GL_n(F)$ is a finite group and $F$ is infinite. Then we follow that
for every $x \in F$ we've got $xI \in GL_n(F)$, thus we follow that finite set has
an infiinte subset, which is a contradiction.

\subsection*{1.4.7}

\textit{Let $p$ be a prime. Prove that the order of $GL_2(F_p)$ is $p^4 - p^3 - p^2 + p$}
There are a total of $p^{4}$ distinct matrices. We follow that there
are $p^2$ distinct rows, one of which is zero. If the row is not zero, then it has $p$
distinct scalar multiples. For zero there is $p^2$ rows, such that one of them is
the scalar multiple of the other. Thus we follow that there are
$$(p^2 - 1)p + p^2 = p^3 - p + p^2$$
non-invertible matrices. From this we follow that the total number of invertible matrices
is
$$p^4 - p^3 - p^2 + p$$
as desired.


\subsection*{1.4.9}

\textit{Prove that the binary operation of matrix multiplication of $2 \times 2$ matrices
  with real number entries in associative.}

Follows from definition of matrix multiplication, true in general for $n \times n$
matrices.

\section{Quaternion Group}

\subsection*{1.5.1}

\textit{Compute the order of each of the elements in $Q_8$}

$$|1| = 1$$
$$|-1| = 2$$
$$|i| = 4$$
$$(-i)^2 = (kj)^2 = kjkj = kik = jk = i$$
$$(-i)^3 = (kj)^3 = ikj = (-j)j = 1$$
$$|-i| = 3$$
$$|j| = 4$$
$$(-j)^2 = ikik = ijk = kk = -1$$
$$(-j)^3 = -1(-j) = j$$
$$(-j)^4 = j(-j) = 1$$
$$|-j| = 4$$
$$|k| = 4$$
$$(-k)^2 = jiji = jki = i^2 = -1$$
$$|-k| = 4$$

\section{Homomorphisms and Isomorphisms}

\subsection*{1.6.1}

\textit{Let $\phi: G \to H$ be a homomorphism}

\textit{(a) Prove that $\phi(x^n) = \phi(x)^n$}

We follow that $\phi(x^2) = \phi(x)^2$ by definition.

Suppose that $\phi(x^n) = \phi(x)^n$. Then we follow that
$$\phi(x^{n + 1}) = \phi(x^{n} x) = \phi(x^{n}) \phi(x) =
\phi(x)^n \phi(x) = \phi(x)^{n + 1}
$$
thus we follow the desired conclusion from induction.

\textit{(b) Do part (a) byt with $n = 1$ and conclude that the same is true for $n \in Z$}

We follow that $$\phi(1x) = \phi(1) \phi(x)$$
and $$\phi(x1) = \phi(x) \phi(1)$$
thus we can conclude that $\phi$ maps identity to the identity. Therefore we follow that
$$\phi(x^0) = \phi(1) = 1 = \phi(x)^0$$

$$\phi(x\inv) = 1 \phi(x\inv) = \phi(x)\inv \phi(x) \phi(x\inv) = \phi(x)\inv \phi(x x \inv) =
\phi(x)\inv \phi(1) = \phi(x)\inv$$
thus we follow that if $\phi(x\inv) = \phi(x)\inv$. Suppose now that
$$\phi(x^{-j}) = \phi(x)^{-j}$$
then we follow that
$$\phi(x^{-(j + 1)}) = \phi(x^{-j} x\inv) = \phi(x^{-j}) \phi(x\inv) = \phi(x)^{-j} \phi(x)\inv =
\phi(x)^{-(j + 1)}$$

Thus we've got the desired conclusion.

\subsection*{1.6.3}

\textit{If $\phi: G \to H$ is an isomorphism, prove that $G$ is abelian iff $H$ is abelian.
  If $\phi: G \to H$ is a homomorphism, what additional conditions on $\phi$ are sufficient to
  ensure that if $G$ is abelean, then so $H$?}

Suppose that $G$ is abelian. Now let $x, y \in H$. Because $\phi$ is a bijection,
we follow that it is surjective. Thus we follow that there exist $x', y' \in G$
such that $\phi(x') = x \land \phi(y') = y$. Thus we follow that
$$xy = \phi(x') \phi(y') = \phi(x'y') = \phi(y' x') = \phi(y') \phi(x') = yx$$
thus we follow that $H$ is abelian.

If a funtion is a bijection, then the  inverse of this function is also surjective. Thus
we've got converse case from the forward implication.

If $\phi$ is a homomorphism, then we follow that if $\phi$ is surejctive and
$G$ is abelian, then $H$ is abelian as well.

\subsection*{1.6.5}

\textit{Prove that the additive groups $R$ and $Q$ are not isomorphic.}

Since there are no bijections from $R$ to $Q$, we follow that no such functions exist (for
the proof GOTO either first chapter of real analysis book, or just google it).

\subsection*{1.6.7}

\textit{Prove that $D_8$ and $Q_8$ are not isomorphic.}

$Q_8$ has an element of order $3$, but $D_8$ does not.

\subsection*{1.6.9}

\textit{Prove that $D_{24}$ and $S_4$ are not isomorphic.}

We've got that $r^{11} \in D_{24}$ has order $132$, and the maximum order of an element in $S_4$
is below 16 (not gonna compute the exact thing, for more info GOTO 1.3.19)

\subsection*{1.6.11}

\textit{Let $A, B$ be groups. Prove that $A \times B \cong B \times A$.}

Define $\phi((a, b)) = (b, a)$. Then we follow that $\phi$ is a bijection, and
the fact that is a homomorphism is easily followed from definitions.

\subsection*{1.6.13}

\textit{Let $G$ and $H$ be groups and let $\phi: G \to H$ be a homomorphism. Prove that
  the image of $\phi, \phi(G)$ is a subgroup of $H$. Prove that if $\phi$ is injective,
  then $G \cong \phi(G)$.}

We follow that if $1 \in \phi(G)$, as proven earlier. Associativity comes naturally and
inverses are handled in exercise 1.6.1. Thus we follow that $\phi(G)$ is indeded a subgroup.

If $\phi$ is injective, then we follow that $\phi: G \to \phi(G)$ is a bijection,
thus we've got isomorphism, as desired.

\subsection*{1.6.15}

\textit{Define a map $\pi: R^2 \to R$ by $\pi((x, y)) = x$. Prove that $\pi$ is a homomorphism
  and find the kernel of $\pi$.}

Suppose that $x = (a, b), y = (c, d) \in R$. Then we follow that
$$\phi(xy) = \phi(ac, bd) = ac = \phi((a, b)) \phi((b, d))$$
therefore $\phi$ is a homomorphism.

We follow that $(0, y)\in R^2$ is a kernel of $\pi$.

\subsection*{1.6.17}

\textit{Let $G$ be any group. Prove that the map from $G$ to itself, defined by $g \to g \inv$
  is a homomorphism iff $G$ is abelian.}

Suppose that $g \to g \inv$ is a homomorphism and let $x, y \in G$. Then we follow that
$$xy = ((xy)\inv) \inv = \phi(xy) \inv = (\phi(x) \phi(y))\inv = (x\inv y\inv )\inv = yx$$
thus we follow that $G$ is abelian.

Suppose that $G$ is abelian. Then we follow that 
$$\phi(xy) = \phi(yx) = (yx)\inv = x\inv y\inv = \phi(x) \phi(y)$$
thus we follow that $\phi$ is a homomorphism.

\subsection*{1.6.19}

\textit{Let $G = \{z \in C: (\exists n \in Z^+)(z^n = 1)\}$. Prove that for any
  fixed integer $k > 1$ the map from $G$ to itself defined by $z \to z^k$ is
  surjective homomorphism but not an isomorphism.}

Suppose that $x, y \in G$. Then we follow that
$$\phi(xy) = (xy)^k = x^ky^k = \phi(x) \phi(k)$$
which proves that $\phi$ is a homomorphism.

Suppose that $x \in G$. Then we follow that $x = \in C$ and there exists $n \in Z^+$ such that
$x^n = 1$. Thus we follow that $x \neq 0$ and we're going to have some $x^{-k + 1}$ such that
$\phi(x^{-k + 1}) = x^1 = x$.
We can also follow that $(x^{-k + 1})^n = x^n = 1$, thus $x^{-k + 1} \in G$. Thus we follow that
$x \in \phi(G)$, and therefore $\phi$ is surjective.

We follow that for every number $x$ in $C$ there exist $k$ elements of $C$ such that $$x^k = 1$$.
Since $1 \in G$, and $k > 1$ we follow that there exist $x_1 \neq x_2 \in C$ such that 
$$\phi(x_1) = 1 = \phi(x_2)$$
thus we follow that $\phi$ is not injective, and thus it is not an isomorphism.

\subsection*{1.6.21}

\textit{Prove that for each fixed nonzero $k \in Q$ the map from $Q$ to itself defined by
  $q \to kq$ is an automorphism of $Q$.}

Assuming that by $kq$ we mean multiplication and letting $\phi(q) = kq$, then we follow that
$$\phi(x + y) = k(x + y) = kx + ky = \phi(x) + \phi(y)$$
thus it is an isomorphism.

We follow that if $x \neq y$, then
$$\phi(x) - \phi(y) = k(x - y) \neq 0$$
thus we follow that $x \neq y$ implies that $\phi$ is injective.

For $x \in Q$ we follow that $\phi(k\inv x) = k k\inv x = x$, thus $\phi$ is
also surjective. Thus $\phi$ is an automorphism.

\subsection*{1.6.23}

\textit{let $G$ be a finite group which posseses an automorphism $\sigma$.
  such that $\sigma(g) = g$ iff $g = 1$. If $\sigma^2$ is the identity map, prove that
  $G$ is abelian.}

Suppose that $\sigma^2$ is the identity. We need to show that for all $x, y \in G$
we've got that
$$xy = yx$$

Suppose that $x \in G$. We follow that if $x \neq 1$, then
$$\sigma(x) \neq x$$
but
$$\sigma(x^2) = \sigma(x)^2 = x^2$$
thus we follow that for every $x \in G$, $x^2 = 1$. Thus we follow that $x\inv = x$ for
every $x \in G$. Therefore we follow that for every $x, y \in G$ we've got that
$xy \in G$ as well, thus
$$xy = (xy)\inv = y\inv x\inv = yx$$
thus the group is abelian, as desired.

\subsection*{1.6.25}

Followes some exercise in my linear algebra book

\section{Group Actions}

\subsection*{1.7.1}

\textit{Let $F$ be a field. Show that the multiplicative group of nonzero elements of $F$
  (denoted by $F^\times$)
  acts on the set $F$ by $g \cdot a = ga$, where $g \in F^\times$, $a \in F$ and $ga$ is
  the usual product in $F$ of the two field elemets.}

We follow that for all $g_1, g_2, g, a \in F^\times$
$$g_1 \cdot (g_2 \cdot a) = (g_1 \cdot g_2) \cdot a$$
by associative property of multiplication in the field.
We also follow that
$$1 \in F$$
by the fact that multiplicative identity is in $F^\times$, and
$$1 \cdot a = a$$
by the fact that $1$ is the multiplicative identity.

\subsection*{1.7.3}

\textit{Shwo that the additive group $R$ acts on the $x, y$ plane $R \times R$ by
  $$r \cdot (x, y) = (x + ry, y)$$
}

We follow that for $r_1, r_2 \in R$ we've got 
$$(r_1 + r_2) \cdot (x, y) = (x + r_1y + r_2 y, y)$$
$$r_1 \cdot (r_2 \cdot (x, y) = r_1 \cdot (x + r_2 y, y) = (x + r_1 y + r_2 y, y)$$
thus we follow the first part of the definition of group action.


We follow that in this case the group identity in $R$ is 0, therefore for all
$(x, y) \in R \cdot R$ is 
$$0\cdot (x, y) = (x + 0y, y) = (x, y)$$

Thus given function satisfies all the properties of group action, and therefore
it is a group action itself, as desired.

\subsection*{1.7.5}

\textit{Prove that the kernel of an action of the group $G$ on the set $A$ is the same
  as the kernel of corresponding permutation representation $G \to S_A$.}

Skip, cannot figure out wording. Kernel of the action is a subset of $G \times A$ and
kernel of permutation representation is a subset of $G$, which means that their
sole common subset is $\emptyset$. Kernel of permutation representation may be non-empty,
thus we follow that the conclusion of the exercise is false.

\subsection*{1.7.7}

\textit{Prove that in Example $2$ in this section the addition is faithful.}

Let $x, y \in F^\times$ be such that $x \neq y$. Let $\phi_1, \phi_2$ be permutations, produced
by
$$\phi_1(a) = x \cdot a$$
$$\phi_2(a) = y \cdot a$$

Let $v \in V$ be defined by
$$v = (1, 1, ...)$$
We follow that
$$\phi_1(v) = (x, x, ...)$$
$$\phi_2(v) = (y, y, ...)$$
and since $x \neq y$, we follow that $\phi_1(v) \neq \phi_2(v)$. Thus we follow that
$x \neq y$ implies that $\phi_x \neq \phi_y$, as desired.

\subsection*{1.7.8}

\textit{Let $A$ be a nonempty set and let $k$ be a positive integer with $k \leq |A|$. The
  symmetric group $S_A$ acts of the set $B$ consisting of all subsets of $A$ of
  cardinality $k$ by $\sigma \cdot \{a_1, ..., a_k\} =   \{\sigma(a_1), ..., \sigma(a_k)\}$}.

Don't have a clue on why do we need a finite cardinality here, but here we go.

Presumably we define $\cdot: S_A \times B \to B$ to be
$$\cdot(\eangle{\sigma, b}) = \sigma[b]$$
where $[]$ is defined to be image of $B$.
Since every $\sigma \in S_A$ is a bijection, we follow that for every $b \in B$
there quite literally exists a bijection between $b$ and $\sigma[b]$. Thus we follow that
$$(\forall \sigma \in S_A, b \in B)(|\sigma \cdot b| = |b| = k)$$


\textit{(a) Prove that this is a group action.}

Suppose that $\sigma_1, \sigma_2 \in S_A$ and  $b \in B$ are arbitrary. We follow that
$$\sigma_1 \cdot (\sigma_2 \cdot b) = \sigma_1 \cdot \sigma[b] = \sigma_1 [\sigma_2[b]] =
(\sigma_1 \cdot \sigma_2)[b]$$
where the last equation quite easily comes from definition of range.

We also follow that if $\sigma$ is an identity, then
$$\sigma[b] = b$$
by the definition (or minor deriviation and application of definition of range) of identity.

Thus we follow that $\cdot$ is indeed a group action.

\textit{Describe explicitly how the elements $(1, 2)$ and $(1, 2, 3)$ act on the six 2-element
  subsets of $\{1, 2, 3, 4\}$}

We follow that
$$B = \{\{1, 2\}, \{2, 3\}, \{3, 4\}, \{1, 3\}, \{1, 4\}, \{2, 4\}\}$$
where we're gonna refer to each element of $B$ by $b_j$, where $j$ is the order of
the element in the above-mentioned representation. Then we follow that
$$(1, 2) \cdot b_1 = \{1, 2\}$$
$$(1, 2) \cdot b_2 = \{1, 3\}$$
$$(1, 2) \cdot b_3 = \{3, 4\}$$
$$(1, 2) \cdot b_4 = \{2, 3\}$$
$$(1, 2) \cdot b_5 = \{2, 4\}$$
$$(1, 2) \cdot b_6 = \{1, 4\}$$
$$(1, 2, 3) \cdot b_1 = \{2, 3\}$$
$$(1, 2, 3) \cdot b_2 = \{1, 3\}$$
$$(1, 2, 3) \cdot b_3 = \{1, 4\}$$
$$(1, 2, 3) \cdot b_4 = \{1, 2\}$$
$$(1, 2, 3) \cdot b_5 = \{2, 4\}$$
$$(1, 2, 3) \cdot b_6 = \{3, 4\}$$

\subsection{1.7.11}

\textit{Write out the cycle decompositionof the eight permutations in $S_4$, corresponding
  to the elements of $D_8$ given by the action of $D_8$ on the vertices of a square.}

$$()$$
$$(1, 2, 3, 4)$$
$$(1, 3)(2, 4)$$
$$(1, 4, 3, 2)$$
$$(2, 4)$$
$$(1, 2)(3, 4)$$
$$(1, 3)$$
$$(1, 4)(2, 3)$$

\subsection{1.7.13}

\textit{Find the kernel of the left regular action}

We follow that the kernel of left regular action is the set
$$A: \{\eangle{x, y} \in G \times G: x = y\inv\}$$

\subsection*{1.7.15}

\textit{Let $G$ be any group and let $A = G$. Show that the maps defined by
  $g \cdot a = a g \inv$ for all $g, a \in G$ do satisfy the axioms of group action
  of $G$ on itself}

Suppose that $x, y, z \in G$. Then we follow that
$$x \cdot (y \cdot z) = x \cdot (z y \inv) = z y\inv x \inv$$
$$(x y) \cdot z = z (x y) \inv = z y \inv x \inv$$

And we follow that
$$1 \cdot z = z  1 \inv = z$$


\subsection*{1.7.17}

\textit{Let $G$ be a group and let $G$ act on itself by left conjugation (i.e.
  $$cdot: G \times G \to G$$
  $$g \cdot x = g x g \inv$$
  )
}

For $g_1, g_2, a \in G$
$$g_1 \cdot (g_2 \cdot a) = g_1 g_2 a g_2 \inv g_1 \inv$$
$$(g_1g_2) \cdot a = g_1 g_2 a (g_1 g_2)\inv = g_1 g_2 a g_2 \inv g_1 \inv $$
as desired.

The fact that if we fix $g$ and make a function
$$\sigma_g: G \to G$$
$$\sigma_g(a) = g \cdot a$$
is an automorphism comes from the fact that permutation representation produces permutations
(i.e. bijections from $G$ to $G$), which is precisely the definition of automorrphism.

\subsection{1.7.18}

\textit{Let $H$ be a group acting on a set $A$. Prove that the relation $\equiv$ on $A$
  defined by
  $$a \equiv b \iff a = hb$$
  is an equivalence relation. }

We follow that
$$a = 1a \ra a \equiv a$$
$$a \equiv b \land b \equiv c \ra a = hb \land b = h'c \ra a = hh' c \ra a \equiv c$$
$$a \equiv b \ra a = hb \ra b = h\inv a \ra b \equiv a$$

\chapter{Subgroups}

\section{Definition and Examples}

Let $G$ be a group

\subsection{2.1.1}

\textit{Prove that the specified subset is a subgroup of the given group}

Let us denote the desctibed group by $H$

\textit{(a) the set of complex number of the form $a + ai$, $a \in R$ (addition)}

$0 = 0 + 0i \in H$, therefore $H \neq \emptyset$.
Suppose that $x, y \in H$ $x = a + ai$, $y = b + bi$,
then we follow that $y\inv = -b -bi \in H$ and
$$xy\inv = a + ai - b - bi = (a - b) + (a - b)i \in H$$
therefore we follow that $H$ is a subgroup.

\textit{(b) The set of complex numbers of absolute value 1, i.e. the unit circlel in
  the complex plane (under multiplication)}

We follow that $1 \in H$, therefore $H \neq \emptyset$. Suppose that $x, y \in H$.
Then we follow that $|y\inv| = 1$, therefore $y\inv \in H$.

$$|xy\inv | = |1| \in H$$
(concrete proof follows from porerties of complex numbers in linear algebra course)
therefore $H$ is a subgroup.

\textit{(c) For fixed $n \in Z^+$ the set of rational number whose denominators divide $n$
  (under addition)}

We follow that $1/n \in H$. Suppose that $y \in H$. We follow that $y = m/k$, and thus
$y\inv = -m/k$, thus $y\inv \in H$. Suppose that $x \in H$. We follow that $x = j/l$,
and
$$xy\inv = s/ lcm(l, k)$$
where $s$ is some number and denominator also is a multiple of $n$, thus $H$ is a subgroup.

\textit{(d) rationals whose denominators are relatively prime to $n$ (addition)}

If $p$ is prime, then $1/p \in H$. For $y = m/k \in H$, we follow that $y \inv = -m/k \in H$,
and for some $x = k/l \in H$ we follwo that
$$xy\inv = s/lcm(l, k) \in H$$
Thus $H$ is a subgroup

\textit{(e) The set of nonzero real number whose square is a
  rational number (under multiplication)}

Here we've got $G = H$.


\subsection*{2.1.2}

\textit{Prove that subset is not a subgroup}

\textit{(a) the set of $2$-cycles in $S_n$ for $n \geq 3$.}

Suppose that $s$ is defined by  $(1, 2)$. We follow that $s s \inv$ 
$(1, 2) (2, 1) = (1)(2)$
which is not a 2-cycle.

\textit{(b) The set of reflections in $D_{2n}$ for $n \geq 3$.}

We follow that $1 \notin H$.

\textit{(c) For $n$ composite integer $> 1$ and $G$ is a group containing element of order $n$,
  the set $\{x \in G: |x| = n\} \cup \{1\}$.}

We follow that if $x \in H$, then $x^{n} \notin H$

\textit{(d) odd integers and 0}

$$3 + 3 = 6$$

\textit{(e) reals whose square is a rational number (under addition)}

$$(\sqrt{2} + \sqrt{3})^2 = 5 + 2\sqrt{6}$$

\subsection*{2.1.4}

\textit{Give an explicit example of a group $G$ and an infinite subset $H$ of $G$ is closed
  under the group operation but is not a subgroup of $G$.}

$Z^+$ closed under addition, but does not have inverses.

\subsection*{2.1.5}

\textit{Prove that $G$ cannot have a subgroup $H$ with $|H| = n - 1$, where $n = |G| > 2$}

Suppose that $x$ is the removed element. Then we follow that it cannot be identity by definition
of the group. If $x$ is not an identity, then we follow that $xy $ must be removes as well,
thus we follow that $|H|$ cannot be $n - 1$.

\subsection*{2.1.6}

\textit{Let $G$ be an abelian group. Prove that $H = \{g \in G: |g| < \infty\}$ is a subgroup
  of $G$ (called the torsion subgroup of $G$). Give an explicit example where
  this set is not a subgroup when $G$ is non-abelian.}

We follow that $1 \in H$. Suppose that $y \in H$. We follow that $|y| < \infty$. Thus
$$y^n = 1$$ for some $n \in Z^+$. Thus $$y^{-n} = 1$$, therefore $|-y| < \infty$ as well.
Suppose that $x \in H$. We follow that there exists $j \in Z^+$ such that
$$x^j = 1$$
thus 
$$(xy{\inv})^{lcm(j, n)}) = 1$$
where we can use this fact only because $G$ is abelian, and therefore we can use the fact that
$$(xy)^n = x^n y^n$$

We can have an operator
$$
\begin{pmatrix}
  1 & 2 \\
  0 & 4
\end{pmatrix}
$$
for which the inverse is
$$
\begin{pmatrix}
  1 & -0.5 \\
  0 & 0.25
\end{pmatrix}
$$
and whose square root is
$$
\begin{pmatrix}
  1 & 1/3 \\
  0 & 2
\end{pmatrix}
$$
where we have that inverse times square root has an infinite order.

\subsection*{2.1.7}

\textit{Fix some $n \in Z$ with $n > 1$. Find the torsion subgroup of $Z \times (Z/nZ)$. Show
  taht the set of elements of infinite order together with the identity is not a subgroup
  of this direct product.}

We follow that the only element of $Z$ that has a finite order is $0$ and
$Z/nZ$ can have $\phi(n)$ (not sure about this, but at least finite ) number of elements
that have finite order. Thus we follow that the cartesian product of those two sets
are the torsion subgroup of $Z \times (Z/nZ)$.

Identity in this case is $\eangle{0, 0}$, and we follow that
$$\eangle{1, 0} \eangle{-1, 1} = \eangle{0, 1}$$

\subsection*{2.1.8}

\textit{Let $H$ and $K$ be subgroups of $G$. Prove that $H \cup K$ is a subgroup if and
  only if either $H \subseteq K$ or $K \subseteq H$.}

Suppose that $H \cup K$ is a subgroup of $G$. Let $x \in K \setminus H$ and
$y \in H \setminus K$. If $x \inv \in H$, then we follow that $(x\inv)\inv \notin H$, therefore
$H$ is not closed under inverses. Thus we follow that $x\inv \in K \setminus H$. Same
applies to $y$, thus $x, x\inv \in K \setminus H$ and $y, y\inv \in H \setminus K$.

Now suppose that $xy \in K$. We follow that since $x\inv \in K$, then $x\inv xy \in K$, therefore
$y \in K$, which is a contradiction. Simular case holds for $xy \in H$. Thus we follow that
our assumption that $H \setminus K$ and $K \setminus H$ are both nonempty is false. Thus we
follow that $H \subseteq K$ or $K \subseteq H$, as desired.

Conversely, if $H \subseteq K$ or $K \subseteq H$, then
we follow that $H \cup K = H$ or  $H \cup K = K$, thus it's a subgroup.

\subsection*{2.1.9}

\textit{Let $G = GL_n(F)$ where $G$ is any field. Define
  $$SL_n(F) = \{A \in GL_n(F): det(A) = 1\}$$
  Prove that $SL_n(F) \leq GL_n(F)$.}

We follow that $I \in SL_n(F)$, thus it is nonempty. Suppose that
$x, y \in SL_n(F)$. Then we follow that $x, y$ are both invertible (followed by
nonzero determinant), and also $y\inv \in SL_n(F)$ (because $\det(A\inv) = \det(A)\inv$).
From this we follow that
$$\det(x y \inv ) = 1$$
since $\det (A) \det (B) = \det(AB)$.
Thus $SL_n(F) \leq GL_n(F)$, as desired.

\subsection*{2.1.10}

Simular case was handled in linear algebra book, gonna skip this one.

\subsection*{2.1.11}

Trivial; skip

\subsection*{2.1.12}

\textit{Let $A$ be an abelian group and fix some $n \in Z$. Prove that the following sets are
  subgroups of $A$:}

\textit{(a) $\{a^n: a \in A\}$}

repeat of 1.1.27

\textit{(b) $$H = \{a \in A: a^n = 1\}$$}

We follow that $1 \in H$, therefore $H \neq \emptyset$. Suppose that $x, y \in H$. Then we
follow that $y^n = 1$. Therefore $y^{-n} = 1$, therefore $y\inv \in H$.
Thus we follow that
$$(xy\inv)^n = x^n (y\inv)^n = 11 = 1$$


\subsection*{2.1.13}

\textit{Let $H$ be a subgroup of the additive group of rational numbers with the property
  that $1/x \in H$ for every nonzero $x \in H$. Prove that $H = 0$ or $H = Q$.}

We follow that $H = 0$ and $H = Q$ are cases, where everything holds. Now suppose that
there exist two nonzero elements $x, y$ of $Q$ such that $x \in H$ and $y' \notin H$.
Then we follow that if $y' < 0$, then $y'\inv > 0$ and $y'\inv \notin H \iff y \notin H$,
thus we can let $y = \max\{y, y'\}$. (Not gonna remember about it afterwards, this was
done so that we've got $x, y > 0$ for simplicity's sake).

Now we get that $x > 0$ and $y > 0$, $x \in H$, $y \notin H$. Since $x, y$ are
positive rationals we follow that $x = m/n, y = r/q$ for some $m, n, r, q \in Z^+$.

Since $x \in H$, we follow that every integer multiple of $x$ is in $H$. Thus we follow that
$$q(m/n) = (qm)/n \in H$$
we can also state by definition of $H$ that
$$n/(qm) \in H$$
therefore every integer multiple of it is also in $H$. Thus
$$jn/lm \in H$$
for every $j, l \in Z^+$. Thus we follow that
$$(j/l)(n/m)  \in H$$
and thus if we set $j/l =  (m/n) / (r/q)$
then we can follow that
$$r/q \in H$$
therefore $y\in H$, which is a contradiction. Thus we follow that the only possible cases for
$H$ are $0$ and $Q$, as desired.

\subsection*{2.1.14}

\textit{Show that $H = \{x \in D_{2n}: x^2 = 1\}$ is not a subgroup of $D_{2n}$ (for $n \geq 3$).}

We follow that $1, s  \in H$. If $n$ is even, then we follow that $r^{n/2},  sr^{n/2} \in H$.
We also follow that $sr^j \in H$ for every $j \in Z^+$ since
$$(sr^j)^2 =  1$$
For which we follow that $s sr^j = r^j \in H$, which is a contradiction.

\subsection*{2.1.15}

\textit{Let $H_1 \leq H_2 .... $ be an ascending chain of subgroups of $G$. Prove that
  $$\bigcup_{i = 1}^{\infty} H_i$$
  is a subgroup of $G$.}

Let us denote
$$S = \bigcup_{i = 1}^{\infty} H_i$$

Since $1 \in H_1$ and $H_1 \subseteq S$, we follow that $1 \in S$, therefore $S$ is non-empty.

Suppose that $x, y \in S$. Then we follow that $(\exists j, k \in Z^+)(y \in H_j \land y \in H_k)$.
From this we follow by group property of $H_k$ that
$$(\exists j, k \in Z^+)(y \in H_j \land y\inv \in H_k)$$
Let $l = \max\{j, k\}$. Then we follow by definition of $S$ that $H_j \subseteq H_l$
and $H_k \subseteq H_l$. Thus
$$(\exists l \in Z^+)(y \in H_l \land y\inv \in H_l)$$
from which by group property of $H_l$ we follow that
$$(\exists l \in Z^+)(xy\inv \in H_l)$$
thus by definition of $S$ we follow that
$$xy\inv \in S$$
Thus we conclude that $x, y \in S \ra xy\inv \in S$. Therefore $S$ satisfies all the
properties of a subgroup of $G$. Thus $S$ is a subgroup of $G$, as desired.

\subsection*{2.1.16}


\textit{Let $n \in Z^+$ and let $F$ be a field. Prove that the set
  $$\{a_{i, j}) \in GL_n(F): (\forall i > j )(a_{ij} = 0)\}$$
  os a subgroup of $GL_n(F)$}

Identity is in there and we follow closure under multiplication and inverces by properties
in linear algebra book. Same goes for the next exercise as well.

\section{Centralizers and Normalizers, Stabilizers and Kernels}



\end{document}
%%% Local Variables:
%%% mode: latex
%%% TeX-master: t
%%% End:

