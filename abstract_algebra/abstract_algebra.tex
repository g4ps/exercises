\documentclass[11pt,oneside,titlepage]{book}
\title{My abstract algebra exercises}
\usepackage{amsmath, amssymb}
\usepackage{geometry}
\usepackage{hyperref}
\author{Evgeny Markin}
\date{2023}

\DeclareMathOperator \map {\mathcal {L}}
\DeclareMathOperator \pow {\mathcal {P}}
\DeclareMathOperator \ns {null}
\DeclareMathOperator \range {range}
\DeclareMathOperator \inv {^{-1}}
\DeclareMathOperator \Span {span}
\DeclareMathOperator \imp {\Rightarrow}
\DeclareMathOperator \lra {\Leftrightarrow}
\DeclareMathOperator \eqv {\Leftrightarrow}
\DeclareMathOperator \la {\Leftarrow}
\DeclareMathOperator \ra {\Rightarrow}
\DeclareMathOperator \true {true}
\DeclareMathOperator \false {false}
\DeclareMathOperator \dom {dom}
\DeclareMathOperator \ran {ran}
\newcommand{\eangle}[1]{\langle #1 \rangle}
\newcommand{\set}[1]{\{ #1 \}}


\begin{document}
\maketitle
\tableofcontents

\part{Preliminaries}

\chapter{Relations and Functions}

\chapter{The Integers and Modular Arithmetic}

\part{Groups}

\chapter{Introduction to Groups}

\section{An Important Example}

\subsection{}

\textit{In $S_4$, let $\sigma = \
  \begin{pmatrix}
    1 & 2 & 3 & 4 \\
    3 & 1 & 4 & 2
  \end{pmatrix}
  $,
  and $\tau =
  \begin{pmatrix}
    1 & 2 & 3 & 4 \\
    3 & 4 & 1 & 2
  \end{pmatrix}
  $. Calculate $\sigma \tau$, $\tau \sigma$ and $\sigma \inv$.
}

$$\sigma \tau =
\begin{pmatrix}
  1 & 2 & 3 & 4\\
  4 & 2 & 3 & 1 
\end{pmatrix}
$$
$$ \tau \sigma =
\begin{pmatrix}
  1 & 2 & 3 & 4\\
  1 & 3 & 2 & 4 
\end{pmatrix}
$$
$$\sigma \inv =
\begin{pmatrix}
  1 & 2 & 3 & 4 \\
  2 & 4 & 1 & 3
\end{pmatrix}
$$

\subsection{}

\textit{In $S_5$, let $\sigma =
  \begin{pmatrix}
    1 & 2 & 3 & 4 & 5 \\
    5 & 3 & 2 & 1 & 4
  \end{pmatrix}
  $
  and
  $\tau =
  \begin{pmatrix}
    1 & 2 & 3 & 4 & 5 \\
    2 & 4 & 1 & 3 & 5
  \end{pmatrix}
  $
  calculate $\sigma \tau \sigma$, $\sigma \sigma \tau$, $\sigma \inv$.
}

$$\sigma \tau \sigma =
\begin{pmatrix}
  1 & 2 & 3 & 4 & 5 \\
  4 & 5 & 1 & 3 & 2
\end{pmatrix}
$$
$$\sigma \sigma \tau =
\begin{pmatrix}
  1 & 2 & 3 & 4 & 5 \\
  2 & 5 & 4 & 3 & 1 
\end{pmatrix}
$$
$$\sigma \inv =
\begin{pmatrix}
  1 & 2 & 3 & 4 & 5 \\
  4 & 3 & 2 & 5 & 1
\end{pmatrix}
$$

\subsection{}

\textit{How mamy permutations are there in $S_n$? In $S_5$, how many permutations $\alpha$
  satisfy $\alpha(2) = 2$?}

We can follow that there are $n!$ permutations total, and if we've got a restriction
$\alpha(2) = 2$, then we've got $(n - 1)!$ permutation. For the case $S_5$ it means
that there are $4! = 24$ such permutations.

\subsection{}

\textit{Let $H$ be the set of all permutations $\alpha \in S_5$ satisfying $\alpha(2) = 2$.
  Which of the properties of closure, associativity, identit, inverses does $H$ enjoy
  under composition?}

All of them

\subsection{}

\textit{Consider the set of all functions from $6$ to $6$. Which of the ...}

Everything other then inverse

\subsection{}

\textit{Let $G$ be the set of all ...}

All of them

\section{Groups}

\subsection{}

\textit{Give group tables for following additive grops: $Z_3$, $Z_3 \times Z_2$}

$$
\begin{tabular}{c | c c c}
   &  0 & 1 & 2 \\
  \hline
  0 & 0 & 1 & 2 \\
  1 & 1 & 2 & 0 \\
  2 & 2 & 0 & 1 \\
  \hline
\end{tabular}
$$

last one is ommited

\subsection{}

\textit{Give group tables for the following groups: $U(12), S_3$}

We follow that $U(12) = \set{1, 5, 7, 11}$. THus
$$
\begin{tabular}{c | c c c c}
   &  1 & 5 & 7 & 11 \\
  \hline
  1 & 1 & 5 & 7 & 11 \\
  5 & 5 & 1 & 11 & 7 \\
  7 & 7 & 11 & 1 & 5 \\
  11 & 11 & 7 & 5 & 1 \\
  \hline
\end{tabular}
$$

One of the programs in progs folder produces desired table for $S_3$ (and can
produce one for any $S_n$ for that matter).

\subsection{}

\textit{Show that $G \times H$ is abelian iff $G$ and $H$ are both abelian}

Was proven in dummit and foote, check 1.1.29

\textit{Rest of the exercises in this section were either already proven in D\&F, are trivial, or
could be solved at a later time if I encounter some gaps in the theory.}

\section{}
\section{}
\section{}

\section{Cyclic Groups}

\subsection{}

\textit{Let $G = \eangle{a}$ be a cyclic group of order 12. List every subgroup of $G$.
List every group of $Z_{12}$}

12's divisors are $\set{1, 2, 3, 4, 6}$, therefore subgroups of $G$ are
$\eangle{a^i}$ for $i \in \set{1, 2, 3, 4, 6}$.

Since $Z_{12}$ is cyclic, we follow that  $\eangle{[1, 2, 3, 4, 6]}$
are the subgroups of $Z_{12}$.

\subsection{}

\textit{Let $G = \eangle{a}$ be a cyclic group of order 120. List all of the groups of order 120.
List all of the elements of order 12 in $G$. }


\end{document}

%%% Local Variables:
%%% mode: latex
%%% TeX-master: t
%%% End:
