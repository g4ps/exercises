\documentclass[11pt,oneside,titlepage]{book}
\title{My abstract algebra exercises}
\usepackage{amsmath, amssymb}
\usepackage{geometry}
\usepackage{hyperref}
\author{Evgeny Markin}
\date{2023}

\DeclareMathOperator \map {\mathcal {L}}
\DeclareMathOperator \ns {null}
\DeclareMathOperator \range {range}
\DeclareMathOperator \inv {^{-1}}
\DeclareMathOperator \Span {span}
\DeclareMathOperator \imp {\Rightarrow}
\DeclareMathOperator \lra {\Leftrightarrow}
\DeclareMathOperator \eqv {\Leftrightarrow}
\DeclareMathOperator \la {\Leftarrow}
\DeclareMathOperator \ra {\Rightarrow}
\DeclareMathOperator \true {true}
\DeclareMathOperator \false {false}
\DeclareMathOperator \dom {dom}
\DeclareMathOperator \ran {ran}
\newcommand{\eangle}[1]{\langle #1 \rangle}


\begin{document}
\maketitle
\tableofcontents

\chapter*{Prelinimaries}

\section{Basics}

\subsection{}

\textit{Determine which of the following elements of $A$ lie in $B$}

$M$ is defined to be
$$
\begin{pmatrix}
  1 & 1\\
  0 & 1
\end{pmatrix}
$$

and
$$B = \{x \in A : MX = XM\}$$

thus all of the following are in $B$.
$$
\begin{pmatrix}
  1 & 1\\
  0 & 1
\end{pmatrix}
$$
$$
\begin{pmatrix}
  0 & 0\\
  0 & 0
\end{pmatrix}
$$
$$
\begin{pmatrix}
  0 & 0\\
  0 & 0
\end{pmatrix}
$$
$$
\begin{pmatrix}
  1 & 0\\
  0 & 1
\end{pmatrix}
$$
$$
\begin{pmatrix}
  1 & 0\\
  0 & 1
\end{pmatrix}
$$

\subsection{}

\textit{Prove that $P, Q \in B \ra P + Q \in B$}

Suppose that $P, Q \in B$. Then we follow that
$$(P + Q)M = PM + QM = QM + PM = (Q + P)M$$
where we've used distributive and commutativity under addition for matrices

\subsection{}

\textit{Prove that $P, Q \in B \ra PQ \in B$}

Suppose that $P, Q \in B$. Thus we follow that $PM = MP$ and $QM = MQ$. Thus
$$(PQ)M = PQM = P(QM) = P(MQ) = PMQ = (PM)Q = (MP)Q = M(PQ)$$
as desired.

\subsection{}

\textit{Find conditions on $p, q, r, s$, which determine precisely when
$$
\begin{pmatrix}
  p & q \\
  r & s
\end{pmatrix} \in B
$$ }

$$
\begin{pmatrix}
  p & q \\
  r & s
\end{pmatrix}
\begin{pmatrix}
  1 & 1 \\
  0 & 1
\end{pmatrix} =
\begin{pmatrix}
  p & p + q \\
  r & r + s
\end{pmatrix}
$$
$$
\begin{pmatrix}
  1 & 1 \\
  0 & 1
\end{pmatrix}
\begin{pmatrix}
  p & q \\
  r & s
\end{pmatrix} = 
\begin{pmatrix}
  p + r & q + s\\
  r & s
\end{pmatrix}
$$
thus we follow that we need to have
$$
\begin{pmatrix}
  p + r & q + s\\
  r & s
\end{pmatrix} =
\begin{pmatrix}
  p & p + q \\
  r & r + s
\end{pmatrix}
$$
thus we follow that the matrix is in $B$ if and only if $r = 0$ and $p = s$. (ocave seems
to support this point).

\subsection{}

\textit{Determine whether the following functions $f$ are well-defined: }

\textit{(a)}
$$f: Q \to Z: f(a/b) = a$$
If we assume that $a/b$ is in form, where $b > 0$ and $a/b$ in their lower terms, then the
function is well-defined. Otherwise, we've got that
$$2/4 = 1/2$$
but
$$f(2/4) = 2 \neq 1 = f(1/2)$$

\textit{(b)}
$$f: Q \to Q: f(a/b) = a^2/b^2$$
is indeed well-defined, since for every $a \in Q$ there is only one square.

\subsection{}
\textit{Determine whether the function $f: R^+ \to Z$ defined by mapping a real number $r$
  to the first digit to the right of the decimal point in a decimal expansion of $r$ is
  well-defined.}

This is a somewhat trick question, since we've got that
$$1 = 0.99999999...$$
which in this case gives us that $f$ is not well-defined.

\subsection{}

\textit{Let $f: A \to B$ be a surjective map of sets. Prove that the relation
  $$a \sim b \lra f(a) = f(b)$$
  is an equivalence relation whose equivalence classes are the fibers of $f$.}

$$f(a) = f(a) \ra a \sim a$$
$$(f(a) = f(b) \land f(b) = f(c) \ra f(a) = f(c)) \ra (a \sim b \land b \sim c \ra a \sim c )$$
$$a \sim b \ra f(a) = f(b) \ra f(b) = f(a) \ra b \sim a$$
which gives us reflexive, transitive and symmetric properties, thus $\sim$ is an
equivalence relation.

We follow that if $x \in B$ and $a, b \in f\inv(\{x\})$, then $a \sim b$ by definition.
Suppose that $a \sim b$. Then we follow that $f(a) = f(b)$, therefore
$a \in f\inv(\{f(a)\}) \land b \in f\inv(\{f(a)\})$. Thus we follow that
if $a \sim b$, then they are fibers for the same value.
Thus we follow that $a \sim b$ if and only if $(\exists x \in B)( a, b \in f\inv(\{x\})$.
Thus we follow that fibers of $f$ are indeed the equivalence classes for $\sim$.

\section{Properties of the Integers}

\subsection{}

\textit{Find GCD and LCM for following numbers and find integers $x$ and $y$ such that
  $ax + by = gcd(a, b)$}

\begin{verbatim}
gcd:   1; lcm:        260, 2 * 20 + -3 * 13 = 1
gcd:   3; lcm:       8556, 27 * 69 + -5 * 372 = 3
gcd:  11; lcm:      19800, 8 * 792 + -23 * 275 = 11
gcd:   3; lcm:   21540381, -126 * 11391 + 253 * 5673 = 3
gcd:   1; lcm:    2759487, -105 * 1761 + 118 * 1567 = 1
gcd: 691; lcm:   44693880, -17 * 507885 + 142 * 60808 = 691
\end{verbatim}

\subsection{}

\textit{Prove that if the integer $k$ divides the integers $a$ and $b$, then $k$ divides
  $as + bt$ for every pair of integers $s$ and $t$}

We follow that because $k$ divides both $a$ and $b$ it also divides $(a, b)$.
Since $(a, b)$ divides both $a$ and $b$ we follow that there exist $q, w \in Z$ such that
$a = q(a, b), b = w(a, b)$.
Thus
$$as + bt = q(a, b) + w(a, b) = (q + w) (a, b)$$
thus we follow that $(a, b)$ divides $as + bt$. Since $|$ is transitive, we follow that
$k | (a, b)$ and $(a, b)| as + bt$ implies that $k | as + bt$, as desired.

(We could've actually skip this part, don't know why I've used it)

\subsection{}

\textit{Let $a, b, N$ be fixed integers with $a, b \neq 0$ and let $d = (a, b)$. Suppose that
  $x_0, y_0 \in Z$ are such that $ax_0 + by_0 = N$. Prove that
  $$a(x_0 + \frac{b}{d}t) + b(y_0 - \frac{a}{d}t) = N$$
}

$$a(x_0 + \frac{b}{d}t) + b(y_0 - \frac{a}{d}t) =
a x_0 + a \frac{b}{d}t + by_0 - b\frac{a}{d}t =
a x_0 + by_0 + t( \frac{ab}{d} - \frac{ab}{d}) =$$
$$ = 
a x_0 + by_0 + t( 0) =
N + 0 = N
$$

\subsection{}

\textit{Determine the value $\phi(n)$ for each integer $n \leq 30$ where $\phi$
  denotes the Euler $\phi$-function}

\begin{verbatim}
phi(1) = 1
phi(2) = 1
phi(3) = 2
phi(4) = 2
phi(5) = 4
phi(6) = 2
phi(7) = 6
phi(8) = 4
phi(9) = 6
phi(10) = 4
phi(11) = 10
phi(12) = 4
phi(13) = 12
phi(14) = 6
phi(15) = 8
phi(16) = 8
phi(17) = 16
phi(18) = 6
phi(19) = 18
phi(20) = 8
phi(21) = 12
phi(22) = 10
phi(23) = 22
phi(24) = 8
phi(25) = 20
phi(26) = 12
phi(27) = 18
phi(28) = 12
phi(29) = 28
phi(30) = 8
\end{verbatim}

\subsection{}

\textit{Prove the WOP of $Z$ by induction and prove the minimal element is and prove
  the minimal element is unique.}

GOTO set theory book

\subsection{}

\textit{If $f$ is a prime prove that there do noe exist nonzero integers $a$ and $b$
  such that $a^2 = pb^2$}

We follow that $a$ and $b$ can be represented as multiples of primes. Therefore
the powers of primes, that represent $a^2$ and $b^2$ are even.
Since the power of $p$ in $pb^2$ is not even, we follow that such numebers do not exist,
as desired

\subsection{}

\textit{Let $p$ be a prime, $n \in Z^+$. Find a formula for the largest power of $p$
  which divides $n!$}

We follow that every $p$'th number is a multiple of $p$. Thus the amount of
of multiples of $p$ in the list $1, 2, ..., n$ is $\lfloor n / p \rfloor$.
To those we need to add the number of multiples of $p^2$, of which there will be
$\lfloor n / p^2 \rfloor$, and thus we follow that the number of multiples of $p$ in $n$ is
$$\sum_{i = 1}^n{\lfloor n / p^i\rfloor}$$
Since for every prime  number we've got that $p^n > n$, we can follow that this formula will do.

\subsection{}

\textit{Write a computer program to determine ....}

Way ahead of you, check congr.py in progs.

\textit{Rest is left for later}

\section{$Z/n Z$: The Integers Modulo $n$}

\subsection{}
\textit{Write down explicitly all the elements in the residue classes $Z/18Z$.}

$$\overline{1}, \overline{2}, ... , \overline{17}$$

\subsection{}

\textit{Prove that the distinct equivalence classes in $Z/nZ$ are precisely
  $\overline{0}, ..., \overline{n - 1}$.}

Suppose that $q \in N$. We follow that $q = an + r$, where $0 \leq r < n$, thus we follow that
$q \in \overline{r}$. Therefore every integer is in one of those sets. Since $r$ is
unique, we follow that $q$ is only in one of those sets.

\subsection{}

\textit{Prove that of $a = a_n 10^n + a_{n - 1}10 ^{n - 1} + .... + a_1 10 + a_0$ is any positive
  integer then $a \equiv \sum{a_n} \mod 9$.}

We follow that $10 \equiv 1 \mod 9$, and therefore $10^n \equiv 1 \mod 9$ for any $n \in Z$.
Thus we can follow that
$$10 a_n \equiv a_n \mod 9$$
and in general
$$10^n a_n \equiv a_n \mod 9$$
therefore
$$\overline{a_n 10^n} = \overline{a_n}$$
and since
$$\sum {\overline{a_n}} = \overline{\sum a_n}$$
we follow the desired result.

\subsection{}

\textit{Compute the remainder when $37^{100}$ is divided by $29$}

We follow that 
$$37^{100} \equiv 8^{100} \mod {29}$$
thus
$$8^ 1 \equiv 8 \mod{29}$$
$$8^ 2 \equiv 6 \mod{29}$$
$$8^ 4 \equiv 36 \equiv 7 \mod{29}$$
$$8^{8} \equiv 49 \equiv 20 \mod{29}$$
$$8^{10} \equiv 120 \equiv 4 \mod{29}$$
$$8^{20} \equiv 16 \mod{29}$$
$$8^{40} \equiv 256 \equiv 24 \mod{29}$$
$$8^{50} \equiv 96 \equiv 9 \mod{29}$$
$$8^{100} \equiv 81 \equiv 23 \mod{29}$$
thus we follow that $37^{100}$ divided by 29 gives us the answer 23.

\subsection{}

$$9^{1500} = ...01$$

\subsection{}

\textit{Prove that the squares of the elements in $Z/4Z$ are jsut $0$ and $1$}

We follow that
$$0^2 = 0$$
$$1^2 = 1$$
$$2^2 = 4 \equiv 0 \mod 4 $$
$$3^2 = 9 \equiv 1\mod 4 $$
so yeah

\subsection{}

\textit{Prove for any integers $a$ and $b$ that $a^2 = b^2$ never leaves a remainder of $3$
  when divided by 4}

From previous exercise we follow that
$$a^2 \equiv [0, 1] \mod 4$$
$$b^2 \equiv [0, 1] \mod 4$$
thus
$$a^2 + b^2  \equiv [0, 1, 2] \mod 4$$

\subsection{}

\textit{Prove that the equation $a^2 + b^2 = 3c^2$ has no nonzero integer solutions}

We follow from previous exercise that $a^2 + b^2 \equiv [0, 1, 2] \mod 4$, and
$c^2 \equiv [0, 1] \mod 4$, therefore $3c^2 \equiv [0, 3] \mod 4$. Thus we follow that
the only possible case is when $a^2 + b^2 \equiv 3c^2 \equiv 0 \mod 4$. Thus we follow
all of the $a^2$, $b^2$ and $c^4$ have the factor of $4$. Thus there exist
$a_0, b_0, c_0$ such that $a^2 = 4^n a_0^2$, $b^2 = 4^n b_0^2$ $c^2 = 4^n c_0^@$
and $a_0^2, b_0^2, c_0^2$ are not divisible by $4$ (otherwise we get a contradiction).
Thus we follow that
$$a_0^2 +  b_0^2 = 3c_0^2$$
all of which are not divisible by 4, which gets us a contradiction, as desired.

\subsection{}

\textit{Prove that the square of any odd integer always leaves a remainder of $1$ when
  divided by $8$}

We follow that remainders  of squares of congruent classes of $8$ are
$$0 1 4 1 0 1 4 1$$
thus we follow the desired conclusion.

\textit{Rest of the exercises is are left for the better times}

\end{document}
%%% Local Variables:
%%% mode: latex
%%% TeX-master: t
%%% End:
