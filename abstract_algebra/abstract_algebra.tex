\documentclass[11pt,oneside,titlepage]{book}
\title{My abstract algebra exercises}
\usepackage{amsmath, amssymb}
\usepackage{geometry}
\usepackage{hyperref}
\author{Evgeny (Gene) Markin}
\date{2024}

\DeclareMathOperator \map {\mathcal {L}}
\DeclareMathOperator \pow {\mathcal {P}}
\DeclareMathOperator \topol {\mathcal {T}}
\DeclareMathOperator \basis {\mathcal {B}}
\DeclareMathOperator \ns {null}
\DeclareMathOperator \range {range}
\DeclareMathOperator \fld {fld}
\DeclareMathOperator \inv {^{-1}}
\DeclareMathOperator \Span {span}
\DeclareMathOperator \lra {\Leftrightarrow}
\DeclareMathOperator \eqv {\Leftrightarrow}
\DeclareMathOperator \la {\Leftarrow}
\DeclareMathOperator \ra {\Rightarrow}
\DeclareMathOperator \imp {\Rightarrow}
\DeclareMathOperator \true {true}
\DeclareMathOperator \false {false}
\DeclareMathOperator \dom {dom}
\DeclareMathOperator \ran {ran}
\newcommand{\eangle}[1]{\langle #1 \rangle}
\newcommand{\set}[1]{\{ #1 \}}
\newcommand{\qed}{\hfill $\blacksquare$}



\begin{document}
\maketitle
\tableofcontents

\chapter*{Preface}

This is another yet another attempt at making any progress with
abstract algebra, this time with 'Abstract Algebra: An Integrated
Approach' by Joseph H. Silverman. I really hope that it works out this
time.

So far, it's been most pleasurable journey. This book embodies all the
things, that I like in the mathematics books: lots of rigor and a bit of
lightheartedness, that really lights it up.

Some of the notation was migrated from my previous endeavours in the maths.


\chapter{A Potpourri of Preliminary Topics}

\textit{All of the topics, discussed in this chapter I know already; skip}

\chapter{Groups - Part 1}

\section*{Notes}

I'm gonna use the symbol $*$ as the generic group function and $e$ as
an identity untill stated otherwise since it's most convinient to me.
Sometimes I'll omit $*$ whenever it's clear what's going on. Also,
sometimes I omit parenthesis, but given that we've got associativity,
we can omit them without problems. For rigorousness' sake, I'll
define it to mean left-associative (i.e. $a * b * c = (a * b) * c$).

\section{Introduction to Groups}

\subsection{}

By substituting shapes for numbers we get a trivial exercise


\subsection{}

\textit{Let $n$ be a positive integer, and let $S_n$ be the group of
  permutations of the set $\set{1, 2, ... , n}$ as described in Example
  2.19. Prove that $S_n$ is a finite group, and give a formula for the
  order of $S_n$.}

From the combinatorics we know that the number of permutations is
exacly the factorial of the cardinality of the underlying set.

\subsection{}

\textit{(a) Let $S$ be a finite set, and let $\phi: S \to S$ be a function. Prove that the
  injectivity, surjectivity, and bijectivity of this function are equivatent}

I'm pretty sure that we've proven that rigorously in the set theory
course. If not, then the proof comes from contradiction and theh
cardinality of the codomain of, which proves that injectivity and
surjectivity are equivalent, and bijectivity comes from definition.

\textit{(b) Give an example of an infinite set $S$ and a function
  $\phi: S \to S$ such that $\phi$ is injective but not surjective}

We can let $S = \omega$, $\phi(x) = 2x$, which gives us range of even numbers.

\textit{(c) Give an example of an infinite set $S$ and a function
  $\phi: S \to S$ such that $\phi$ is surjjective but nnot bijective}

We can set $S = \omega$ and
$$\phi(x) =
\begin{cases}
  x = 0 \to 1 \\
  x - 1 \text{ otherwise}
\end{cases}
$$

\subsection{}

\textit{This one involves drawing and is pretty trivial; skip}

\section{Abstract Groups}

\textit{Let $G$ be a group. In this exercise you will prove the remaining parts of
  Proposition 2.9. Be sure to justify each step using the group axioms or by reference to a
  previously proven fact}

\textit{(a) $G$ has exactly one identity element.}

Suppose that $e_1$ and $e_2$ both satisfy identity axiom. We follow
that both of them are in $G$ and thus
$$e_1 = e_1 e_2 = e_2 e_1 = e_2$$
which comes directly from the Identity Axiom.

\textit{(b) $g, h \in G \ra (g * h)\inv = h\inv g\inv$}

We follow that
$$(g * h)\inv  (g * h) = e$$
by definition of identity. Thus
$$(g * h)\inv  (g * h) * h\inv  = e h\inv$$
$$(g * h)\inv  (g * h) * h\inv  g\inv  = e h\inv g\inv$$
by the fact that $*$ is a binary function. Thus
$$(g * h)\inv  g * (h * h\inv) * g\inv  = e h\inv g\inv$$
$$(g * h)\inv  g * g\inv  = e h\inv g\inv$$
$$(g * h)\inv  (g * g\inv)  = e h\inv g\inv$$
$$(g * h)\inv = e h\inv g\inv$$
by associative laws, and then we've got that
$$(g * h)\inv = h\inv g\inv$$
by the identity axiom, as desired

\textit{(c) $g \in G \ra (g\inv) \inv = g$}

We follow that
$$(g\inv) \inv * (g\inv) = e$$
by the inverse axiom. Thus
$$(g\inv)\inv * (g\inv) * g  = e * g$$
$$(g\inv)\inv * (g\inv) * g  = g$$
by identity and properties of functions. Thus
$$(g\inv)\inv * ((g\inv) * g)  = g$$
$$(g\inv)\inv * e  = g$$
$$(g\inv)\inv  = g$$
by associativity and so on, as desired.

\subsection{}

\textit{Let $G$ be a group, let $g, h \in G$, and suppose that $g$ has
  order $n$ and that $h$ has order $m$.}

\textit{(a) If $G$ is an abelian group and if $gcd(m, n) = 1$, prove that
  the order of $gh$ is $mn$.}

Firstly, I want to follow a couple of things:

If order of $g$ is $m$, and order of $g\inv$ is not $n < m$, then
$$e = e * e = g^m (g\inv)^n = g^{m - n} = e$$
which is a contradiction. Similar case holds for $n > m$.  Thus if
order of $g$ is $n$, then order of $g\inv$ is also $n$.

We also follow that if $g$ has finite order $n$, then $(g^k)^n =
(g^n)^k = e$, and thus $g^k$ has finite order for any $k \in Z$.
Moreover, since $(g^k)^n = e$ we follow that $g^k$'s order divides $n$.

We also follow that if $g$ has finite order $n$, then
$$g * g^{n - 1} = e$$
$$g^{n - 1} = g\inv$$


Now back to our exercise. We follow that
$$(gh)^{mn} = g^{mn} h^{mn} = (g^{n})^m (h^m)^n = e^m e^n = e$$
where we can split it this way since $G$ is abelian. Thus we follow
that order of $gh$ is finite and divides $mn$.

Suppose that the order of $gh$ is $k$. We follow that $k \leq mn$
since it divides $mn$.
$$(gh)^k = e$$
thus
$$(gh)^k = g^k h^k = e$$
if $g^k \neq e$, then we follow that $h^k = (g^k) \inv \neq e$, thus
$h^k = (g\inv)^k = (g^{n - 1})^k$. Thus we follow that $h^k$ is a
multiple of $g$, and thus its order divides order of $g$ $m$. Since
$h^k$ is both multiple of $g$ and $h$, we follow that its order
divides both $m$ and $n$, and since $gcd(m, n) = 1$, we follow that
its order is 1. Thus $h^k = e$, which is a contradiction.

Thus we conclude that $g^k = e$. For $h^k$ we've got a similar
case. Thus we follow that $k$ divides both $m$ and $n$, and thus it's
either $1$ or $mn$. If $k = 1$, then we follow that $g = h\inv$, and
thus $gcd(m, n) = 1$ implies that the order of both $g$ and $h$ cannot
be anything other than $1$, and thus $k = mn = 1$.
If $k \neq 1$, then we follow that $k = mn$, as desired.

\textit{(b) Give an example showing that (a) need not be true of we
allow $gcd(m, n) > 1$}

We can have some group where order of $g$ is $n > 1$ (for example $g =
1$ in $G = Z/5$) and set $h = g\inv$, for which we'll have that order
of $gh$ is 1.

\textit{(c) Give an example of a nonabelian group showing that (a) need not
  be true even if we retain the requirement that $gcd(m, n) = 1$.}

A dihedral group of a triangle with $g = r$ and $h = f$ will do. We'll have that
order of $rf$ is 2:
$$(rf)^2 = (rfrf) = f\inv r r f = e$$
with order of $g$ being $2$ and order of $h$ being $3$

\textit{(d) Again assume that $G$ is an abelian group, and let $l = mn
  / gcd(m, n)$ (i.e.  $l = lcm(m, n)$).  Prove that $G$ has an element
  of order $l$.}

We follow that $n$ divides order of $g^m$. We also follow that $m$
divides order of $h^n$.  Since $l = lcm(m, n)$ is divided by both $m$
and $n$ we follow that $g^m h^n$'s order divides $l$.

If there's $k \leq lcm(m, n)$ such that
$$(g^m h^n)^k = e$$
then we follow that
$$g^{mk} h^{nk} = e$$

Here we're gonna employ a more generalized version of an argument in part (a):
If $g^{mk} \neq e$, then $k < lcm(m, n)$, thus  $h^{n}$'s order is a
multiple of both $m$ and $n$. $h^{nk}$ cannot be $1$, and its order
must be then $lcm(m, n)$ or $1$, and it can be neither, thus we've got
a contradiction. Thus we conclude that $g^{mk} = h^{nk} = 1$, which implies that $k$
divides both $m$ and $n$, which implies that $k = lcm(m, n)$, as desired.

\subsection{}

\textit{Definition 2.6 says that a group is a set $G$ with a
composition law satisfying three axioms.  In particular, it says that
there's an identity element $e \in G$ that works on both sides and
that every element $g \in G$ has an inverse that works on both
sides. Suppose that we weaken the requirements to specify that the
identity and inverse work only on one side. In other words, we suppose
that $G$ is a set with a composition law satisfying the following
weaker axioms: }

\textit{(a) (Right-Identity Axiom) There is an element $e \in G$ so
  that $g * e = g$ for all $g \in G$}

\textit{(b) (Right-Inverse Axiom) For all $g \in G$ there is an
  element $h \in G$ so that $g * h = e$}

\textit{(c) (Associative Law) $(g_1 * g_2) * g_3 = g_1 * (g_2 * g_3)$
for all $g_1, g_2, g_3 \in G$.}

\textit{Prove that $G$ is a group.}

We're gonna start with establishing the inverse axiom for $G$,
as suffested in the hint to the exercise.
Suppose that for $g \in G$ there's $h \in G$ such that 
$$g * h = e$$
we also follow that for $h$ there's an element $k$ such that
$h * k = e$ by the same axiom. We thus follow that
$$ h * g = h * g * e = h * g * h * k = h * (g * h) * k = h * e * k = (h * e) * k = h * k = e $$
where we use justification of
$$a \to (\text{h * k = e}) \to c \to b \to c \to a \to (\text{h * k = e})$$
in our equalities (given axioms are presented by letters).
Thus we've got that $h * g = e = g * h$ (i.e. normal inverse axiom)

Now suppose that $g \in G$. We follow that
$$e * g = g * g\inv * g = g * (g\inv * g) =  g * e = g$$
which practially establishes the Identity axiom. The associative law
is unchanged from the stadart definition of the group, and thus we're
following that $G$ is indeed a group, as desired.

\subsection{}

\textit{There are other sorts of algebraic structures that are similar to groups in that
  they are sets $S$ that have a composition law}
$$S \times S \to S, (s_1, s_2) \to s_1 * s_2$$
\textit{but they have fewer or different asiomx than a group. In this exercise we explore two
  of these structures.}

\textit{The set $S$ with its composition law is a monoid if it has an
identity element $e \in S$ and satisfies the associative law, but
elements are not required to have inverses.}

\textit{The set $S$ with its composition law is a semigroup if its
composition law is associative, but it need not have an identity
element or inverses.}

i.e. monoid satisfies associative and identity, semigroup satisfies associative,
and group satisfies all of them. Hence we've got nested classes of structures:

semigroup $\subseteq$ monoid $\subseteq$ group

\textit{For each of the following sets $S$ and composition laws $*$
determine if $(S, *)$ is a group, a monoid, or a semigroup.}

\textit{(a) The set of natural numbers $N = \set{1, 2, 3,
    ...}$ with the composition law being addition.}

I usually do include $0$ into the set of naturals, but in this particular case
it seems that we don't do it.

In this particular case we follow that $N$ has associativity with $+$,
rigorous proof of which comes from the construction of the naturals.
In the course of the set theory I've gone through that proof, but
I'm pretty sure that it's not required here.

If we don't include $0$ into $N$, then we don't have an identity in
$N$, which can be proven by defining order $>$ on $N$ and proceeding
from there (which was also handled in the course on set theory).
Hence it is only a semigroup. If we include it, however, then we get an
identity, and thus this set becomes a monoid.

We can also state that it's not got identities.

\textit{(b) The set of extended  natural numbers $N_0 = \set{0, 1, 2, 3,
    ...}$ with the composition law being addition.}

handled in part (a)

\textit{(c) $(Z, +)$}

We can follow that it's a superset of $N$, which gets its inverses ($a\inv = -a$)
and hence becomes a full-blown group.

\textit{(d) $(N, *)$}

We follow that it's associative, has the identity $1$, and hence is at
least a monoid. It's not got multiplicative inverses (because that
would be $Q_+$), and hence is not a group.

\textit{(e) $(N_0, *)$}

Same as previous, $0$ does not change associativity and identities. Is
not a group for the same reason as the previous case.

\textit{(f) $(Z, *)$}

Same as previous for the same reason.

\textit{(g) The set of integets $Z$ with the composition law $m * n = max\set{m, n}$}

If there's $m \in Z$, then there's $n_0, n_1 \in Z$ such that
$$n_0 < m < n_1$$
and hence
$$n_0 * m = n_0, n_1 * m = m$$
and hence we follow that there's no identity.

We follow that
$$(n * m) * k = max\set{max\set{n, m}, k} = max\set{n, m, k} = max\set{n, max\set{m, k}} = n * (m * k)$$
hence we've got associaativity. Thus the given structure is a semigroup.

\textit{(h) The set of naturals $N$ with the composition law $m * n = max\set{m, n}$}

It's got associativity and hence this thing is a semigroup. We can follow that
$1 \leq m$ for all $m \in N$, and thus $max\set{1, m} = m$. Thus we follow that we also
have a monoid. Inverses are not present, and thus it's not a group.

\textit{(i) The set of naturals $N$ with the composition law $m * n = min\set{m, n}$}

We've got associativity. If $m \in N$, then there's $n \in N$ such that $n > m$, and thus
$min\set{m, n} = m$, which means that there's no identity, and hence this thing is only a semigroup.

\textit{(j) The set of naturals $N$ with the composition law $m * n = mn^2$}

We follow that
$$(m * n) * k = (mn^2) * k = mn^2k^2$$
$$m * (n * k) = m * nk^2 = mn^2k^4$$
setting m, n, k to primes we can follow that $(m * n) * k \neq m * (n * k)$, which implies that
this thing is neither a group, a monoid, nor a semigroup.

\end{document}
