\documentclass[11pt,oneside,titlepage]{book}
\title{My abstract algebra exercises}
\usepackage{amsmath, amssymb}
\usepackage{geometry}
\usepackage{hyperref}
\author{Evgeny (Gene) Markin}
\date{2024}

\DeclareMathOperator \map {\mathcal {L}}
\DeclareMathOperator \pow {\mathcal {P}}
\DeclareMathOperator \topol {\mathcal {T}}
\DeclareMathOperator \basis {\mathcal {B}}
\DeclareMathOperator \ns {null}
\DeclareMathOperator \range {range}
\DeclareMathOperator \fld {fld}
\DeclareMathOperator \inv {^{-1}}
\DeclareMathOperator \Span {span}
\DeclareMathOperator \lra {\Leftrightarrow}
\DeclareMathOperator \eqv {\Leftrightarrow}
\DeclareMathOperator \la {\Leftarrow}
\DeclareMathOperator \ra {\Rightarrow}
\DeclareMathOperator \imp {\Rightarrow}
\DeclareMathOperator \true {true}
\DeclareMathOperator \false {false}
\DeclareMathOperator \dom {dom}
\DeclareMathOperator \ran {ran}
\newcommand{\eangle}[1]{\langle #1 \rangle}
\newcommand{\set}[1]{\{ #1 \}}
\newcommand{\qed}{\hfill $\blacksquare$}



\begin{document}
\maketitle
\tableofcontents

\chapter*{Preface}

This is yet another attempt at making any progress with abstract
algebra, this time with 'Fundamentals of Abstract Algebra' by Mark
J. DeBonis. Hopefully this time I wont encounter any deal-breakers

\chapter{Background Material}

\section{Equivalence Relations}

\subsection{}

\textit{For the examples in Example 1.1, list three elements in each relation}

$$ a \equiv b; b \equiv b; c \equiv d$$
for the first one,
$$ 1 < 2 < 3 < 4$$
for the second one, and
$$\emptyset \subseteq \set{0} \subseteq \set{0, 1} \subseteq \set{0, 1, 2}$$
for the third one. The rest is trivial as well.

\textit{The majority of the exercises are pretty trivial or were dealt
with before in my previous endeavours in various parts of
math. Instead of repeating, I'm just gonna complete the ones, that I
like the most}

\subsection*{1.1.13}

\textit{Suppose a relation $\equiv$ on a set $A$ has the following two
properties:}

\textit{(a) For all $a \in A, a \equiv a$.}

\textit{(b) For all $a, b, c \in A, a \equiv b \land b \equiv c \ra c
\equiv a$.}

\textit{Prove that $\equiv$ is an equivalence relation on $A$}

Part (a) gives us reflexivity. If $a \equiv b$, then we follow that $b
\equiv b$ by part (a), and thus by part (b) we've got $b \equiv a$,
which is symmetry. Now we follow that $c \equiv a$ implies that $a
\equiv c$, which when appended to the end of part (b) gives us
trasitivity, which concludes the proof.

\section{Functions}

\subsection*{Notes}

There's no reason why this chapter is following the previous chapter,
they should be swapped in place. Also, definition of 'well-defined
map' is nonsensical

All the material, that is presented in this section mirrors the
content of the similar section in set theory course, so exercises
are skipped. This also applies to all the material in this chapter

\end{document}