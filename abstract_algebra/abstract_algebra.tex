\documentclass[11pt,oneside,titlepage]{book}
\title{My abstract algebra exercises}
\usepackage{amsmath, amssymb}
\usepackage{geometry}
\usepackage{hyperref}
\author{Evgeny Markin}
\date{2023}

\DeclareMathOperator \map {\mathcal {L}}
\DeclareMathOperator \pow {\mathcal {P}}
\DeclareMathOperator \topol {\mathcal {T}}
\DeclareMathOperator \basis {\mathcal {B}}
\DeclareMathOperator \ns {null}
\DeclareMathOperator \range {range}
\DeclareMathOperator \fld {fld}
\DeclareMathOperator \inv {^{-1}}
\DeclareMathOperator \Span {span}
\DeclareMathOperator \lra {\Leftrightarrow}
\DeclareMathOperator \eqv {\Leftrightarrow}
\DeclareMathOperator \la {\Leftarrow}
\DeclareMathOperator \ra {\Rightarrow}
\DeclareMathOperator \imp {\Rightarrow}
\DeclareMathOperator \true {true}
\DeclareMathOperator \false {false}
\DeclareMathOperator \dom {dom}
\DeclareMathOperator \ran {ran}
\newcommand{\eangle}[1]{\langle #1 \rangle}
\newcommand{\set}[1]{\{ #1 \}}

\begin{document}
\maketitle
\tableofcontents

\chapter{Groups}

\section{Symmetries of  a Regular Polygon}

\textit{Content of this section was pretty much taken care of in a previous try at an
  abstract algebra coutse}

\section{Introduction to Groups}

\textit{For the next 14 exercises decide whether or not hte given pair forms a group.}

\subsection{}

\textit{The pair $(N, +)$}

No, since there are no inverses for nonzero elements

\subsection{}

\textit{The pair $(Q \setminus \set{-1}, \star)$, where $a \star b = a + b + ab$}

$$a \star (b \star c) = a \star (b + c + bc) = a + (b + c + bc) + ab + ac + abc$$
so associativity checks out.

We can follow that $0$ is an identity, since
$$a \star 0 = a + 0 + a0 = a$$

Suppose that $a \in Q \setminus \set{-1}$. We follow that
$$a + b + ab = 0$$
$$b = -a(1 + b)$$
$$b/(1 + b) = -a$$
$$- b/(1 + b) = a$$
since $b \in Q \setminus \set{-1}$, we follow that $b = m/n$, and thus
$$- \frac{m/n}{1 + m/n} = a$$
$$- \frac{m/n}{(n + m)/n} = a$$
$$- \frac{m}{n + m} = a$$
since $a \in Q \setminus \set{-1}$ we follow that $a = k/l$, and thus
$$- \frac{m}{n + m} = k/l$$
$$\frac{- m}{n + m} = \frac{k}{l}$$
$$
\begin{cases}
  m = -k \\
  n = l + k
\end{cases}
$$
thus we follow that as long as $n \neq 0$, $a$ will have an inverse. $n = 0 \iff l = -k \iff a = -1$,
and since $a \neq -1$, we conclude that any given element in the given set is an inverse, and thus
the given set satisfies all the axioms of a group.

\subsection{}

\textit{The pair $\eangle{Q \setminus \set{0}, /}$}

We follow that if $a \in lhs$, then $a = m/n$, and thus $n/m$ is the inverse, thus every element
got an inverse ($a \neq 0$, thus $m \neq 0$).

$$a / (b / c) = a / \frac{b}{c} = a \frac{c}{b} = \frac{ac}{b}$$
$$(a / b) / c = \frac{a}{b} / c = \frac{a}{b} \frac{1}{c} = \frac{a}{bc}$$
nonzero $a, b, c$ ($\eangle{1, 2, 3}$ should do the trick) will give us a concrete
proof that $/$ is not associative, which means that there's no group

\subsection{}

\textit{The pair $\eangle{A, +}$ where $A = \set{x \in Q: |x| < 1}$}

Assuming that $|\star|$ means absolute value, we follow that $+$ won't be a binary operation on $A$.

\textit{The rest of the exercises are left for better times}

\section{Properties of Group Elements}

\subsection{}

\textit{Find the orders of $\overline{5}$ and $\overline{6}$ in $(Z/21Z, +)$}

We follow that order of $\overline{5}$ is $21$ and $7$ for $\overline{6}$.

\subsection{}

\textit{Find the orders of $\overline{21}$ in $Z/52$}

It's' 13

\subsection{}

\textit{Calculate the order of $\overline{285}$ in the group $Z/360Z$}

$$(285 * 24) / 360 = 19$$
thus the order is 19


\textit{The rest of the exercises (or exercises similar to those given in a book) were
  taken care of previously in previous books}

\subsection*{1.3.18}

\textit{Prove that $(Q, +)$ is not a cyclic group.}

We can follow that $q \in Q$ is either positive, negative or zero. Thus $q^n$ is either positive,
negative or zero respectively for all $n \in \omega$, thus proving that no element of $Q$ can be
a generator, which means that $Q$ has no generator.



\section{Concept of a Classification Theorem}

\end{document}

%%% Local Variables:
%%% mode: latex
%%% TeX-master: t
%%% End:
