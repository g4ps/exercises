\documentclass[11pt,oneside,titlepage]{book}
\title{My abstract algebra exercises}
\usepackage{amsmath, amssymb}
\usepackage{geometry}
\usepackage{hyperref}
\author{Evgeny Markin}
\date{2023}

\DeclareMathOperator \map {\mathcal {L}}
\DeclareMathOperator \pow {\mathcal {P}}
\DeclareMathOperator \topol {\mathcal {T}}
\DeclareMathOperator \basis {\mathcal {B}}
\DeclareMathOperator \ns {null}
\DeclareMathOperator \range {range}
\DeclareMathOperator \fld {fld}
\DeclareMathOperator \inv {^{-1}}
\DeclareMathOperator \Span {span}
\DeclareMathOperator \lra {\Leftrightarrow}
\DeclareMathOperator \eqv {\Leftrightarrow}
\DeclareMathOperator \la {\Leftarrow}
\DeclareMathOperator \ra {\Rightarrow}
\DeclareMathOperator \imp {\Rightarrow}
\DeclareMathOperator \true {true}
\DeclareMathOperator \false {false}
\DeclareMathOperator \dom {dom}
\DeclareMathOperator \ran {ran}
\newcommand{\eangle}[1]{\langle #1 \rangle}
\newcommand{\set}[1]{\{ #1 \}}

\begin{document}
\maketitle
\tableofcontents

\chapter{Groups}

\section{Symmetries of  a Regular Polygon}

\textit{Content of this section was pretty much taken care of in a previous try at an
  abstract algebra coutse}

\section{Introduction to Groups}

\textit{For the next 14 exercises decide whether or not hte given pair forms a group.}

\subsection{}

\textit{The pair $(N, +)$}

No, since there are no inverses for nonzero elements

\subsection{}

\textit{The pair $(Q \setminus \set{-1}, \star)$, where $a \star b = a + b + ab$}

$$a \star (b \star c) = a \star (b + c + bc) = a + (b + c + bc) + ab + ac + abc$$
so associativity checks out.

We can follow that $0$ is an identity, since
$$a \star 0 = a + 0 + a0 = a$$

Suppose that $a \in Q \setminus \set{-1}$. We follow that
$$a + b + ab = 0$$
$$b = -a(1 + b)$$
$$b/(1 + b) = -a$$
$$- b/(1 + b) = a$$
since $b \in Q \setminus \set{-1}$, we follow that $b = m/n$, and thus
$$- \frac{m/n}{1 + m/n} = a$$
$$- \frac{m/n}{(n + m)/n} = a$$
$$- \frac{m}{n + m} = a$$
since $a \in Q \setminus \set{-1}$ we follow that $a = k/l$, and thus
$$- \frac{m}{n + m} = k/l$$
$$\frac{- m}{n + m} = \frac{k}{l}$$
$$
\begin{cases}
  m = -k \\
  n = l + k
\end{cases}
$$
thus we follow that as long as $n \neq 0$, $a$ will have an inverse. $n = 0 \iff l = -k \iff a = -1$,
and since $a \neq -1$, we conclude that any given element in the given set is an inverse, and thus
the given set satisfies all the axioms of a group.

\subsection{}

\textit{The pair $\eangle{Q \setminus \set{0}, /}$}

We follow that if $a \in lhs$, then $a = m/n$, and thus $n/m$ is the inverse, thus every element
got an inverse ($a \neq 0$, thus $m \neq 0$).

$$a / (b / c) = a / \frac{b}{c} = a \frac{c}{b} = \frac{ac}{b}$$
$$(a / b) / c = \frac{a}{b} / c = \frac{a}{b} \frac{1}{c} = \frac{a}{bc}$$
nonzero $a, b, c$ ($\eangle{1, 2, 3}$ should do the trick) will give us a concrete
proof that $/$ is not associative, which means that there's no group

\subsection{}

\textit{The pair $\eangle{A, +}$ where $A = \set{x \in Q: |x| < 1}$}

Assuming that $|\star|$ means absolute value, we follow that $+$ won't be a binary operation on $A$.

\textit{The rest of the exercises are left for better times}

\section{Properties of Group Elements}

\subsection*{Notes}

Order of a group is defined as cardinality of $G$, which is a functional and not a function. This
is not that big of a deal, all things considered. Order of an element is a separate entity
altogether, that is defined as a function from a set $G$, to an extended natural line
with excluded $0$ (i.e. $\omega \setminus 0 \cup \set{\infty}$), where we define order
in the latter by obvious means.

\subsection{}

\textit{Find the orders of $\overline{5}$ and $\overline{6}$ in $(Z/21Z, +)$}

We follow that order of $\overline{5}$ is $21$ and $7$ for $\overline{6}$.

\subsection{}

\textit{Find the orders of $\overline{21}$ in $Z/52$}

It's' 13

\subsection{}

\textit{Calculate the order of $\overline{285}$ in the group $Z/360Z$}

$$(285 * 24) / 360 = 19$$
thus the order is 19

\subsection{}

\textit{Calculate the order of $r^{16}$ in $D_{24}$}

We follow that $|r| = 24$, and thus
$$|r^{16}| = \frac{24}{gcd(16, 24)} = \frac{24}{gcd(16, 24)} = 3$$
$$(r^{16})^3 = r^{48} = (r^{24})^2 = e^2 = e$$

\subsection*{1.3.11}

\textit{Prove 1.2.12}

The definition of powers in the book as not as rigorous, as one might want. We can rigorously
a function $f_x: \omega \to G$ for an arbitrary group $G$ and arbitrary $x \in G$
by setting
$$f_x(0) = e$$
and
$$f_x(n^+) = x f(n)$$
which will give us a proper function by recursive definition. Thus we can create
a function from $G$ to a set of functions, defined this way, and then can expand the domains to $Z$
of resulting function by setting
$$f_x(-n) = f_{x\inv}(n)$$
to then get a function $\pow: G \times Z \to G$, which we're gonna call the power function.
That way we don't have to prove that the power function is indeed a function and all that nonsense.

Now we can follow that
$$\pow(x, 0) = e$$
$$\pow(x, n + 1) = \pow(x, n + 1) = \pow(x, n) n = \pow(x, n - 1) n n =
n \pow(x, n - 1) n = n n \pow(x, n - 1) = \pow(x, n + 1)$$
and the same thing for negative numbers, which by induction will give us that
$$\pow(x, n) x = x \pow(x, n)$$
for arbitrary $x \in G$ and $n \in Z$.

Now we want to prove that
$$x^m x^n = x^{m + n}$$
with a functional notation, we want to prove that
$$\pow(x, m) \pow(x, n) = \pow(x, m + n)$$
We firstly can follow that
$$\pow(x, m) \pow(x, 0) = \pow(x, m) e = \pow(x, m) = \pow(x, m + 0)$$
then we follow that
$$\pow(x, m) \pow(x, n^+) = \pow(x, m) x \pow(x, n) = \pow(x, m) \pow(x, n) x = \pow(x, m + n) x =
\pow(x, m + n^+)$$
and this will give us an inductive proof that $x^m x^n = x^{m + n}$ for arbitrary $m \in Z$ and
$n \in \omega$. Some burocracy with regards to domains, maybe a trivial proof of the
fact that $\pow(x, m) x\inv = \pow(x, m - 1)$  and whatnot will give us inductive proof for
arbirtrary pairs of $m, n \in Z$. Same kind of reasoning (i.e. setting arbitrary $m$ and
then do the inductive proof over $n$) can be applied to the latter part of the theorem,
which is gonna be as boring as this one.



\subsection*{1.3.18}

\textit{Prove that $(Q, +)$ is not a cyclic group.}

We can follow that $q \in Q$ is either positive, negative or zero. Thus $q^n$ is either positive,
negative or zero respectively for all $n \in \omega$, thus proving that no element of $Q$ can be
a generator, which means that $Q$ has no generator.

\subsection*{1.3.19}

\textit{Prove 1.3.5}

1.3.5 states that $|x\inv| = |x|$. Let $n = |x|$. Assume that $|x| \in \omega$.
If $|x\inv| = m \neq n$,
then we follow that if  $m < n$ then
$$x^n (x\inv)^m = x^{n - m}$$
which gives us that either $|x| \neq n$ or that our properties of powers don't work,
both of which are contraciction. Same logic (with some obvious handling of a case
when $|x\inv| = \infty$) can be applied for
$m > n$, thus giving us the desired conclusion for $|x| \in \omega$. If $|x| = \infty$
and $|x\inv| = n$ for $n \in \omega$ we follow practically the same thing: 
$x^n (x\inv)^n$ is either not equal to $e$, or equal to it, both of which aren't good for
not having contradictions.

\subsection*{1.3.23}

\textit{Let $x \in G$ be an element of finite order $n$. Prove that $e, x, x^2, ..., x^n - 1$
  are all distict. Deduce that $|x| \leq |G|$}

\textbf{The premise of the given exercise should be given as a proposition in the book. Don't
put the theorems in exercises, it doesn't help anyone}

If $0 < i < j < n$ are such that $x^i = x^j$, then $n - i \neq n - j$ but
$$e = x^n = x^{n - i} x^i$$
$$e = x^n = x^{n - j} x^j$$
and thus
$$e = x^{n - j} x^j = x^{n - j} x^i = x^{n - j + i}$$
since $i < j $ we follow that $-j + i < 0$ thus $n - j + i < n$ and therefore $n$ is not
an order of $|x|$, as desired.


\subsection*{1.3.29}

\textit{Using a CAS find all the orders of all the elements in $GL_2(F_3)$}

We can use
\begin{verbatim}
for i in GL(2, GF(3)):
    print(i.order())
\end{verbatim}
in SAGE to ge the desired result

\textit{The rest of the exercises (or exercises similar to those given in a book) were
  taken care of previously in previous books}

\section{Concept of a Classification Theorem}

\subsection*{Notes}

An obvious remark: if $G$ and $H$ are finite, then $|G \oplus H| = |G \times H| = |G||H|$.

\subsection{}

\textit{Find all orders of all elements in $Z_4 \oplus Z_2$}

We can follow that
$$|\eangle{0, 0}| = 1$$
$$|\eangle{1, 0}| = 4$$
$$|\eangle{2, 0}| = 2$$
$$|\eangle{3, 0}| = 4$$
$$|\eangle{0, 1}| = 2$$
$$|\eangle{1, 1}| = 4$$
$$|\eangle{2, 1}| = 2$$
$$|\eangle{3, 1}| = 4$$

\subsection{}

\textit{What is the largest order of an element in $Z_{75} \oplus Z_{100}$? Illustrate with a
  specific element}

We follow that for $\eangle{x, y} \in Z_{75} \oplus Z_{100}$ we've got that 
$$|\eangle{x, y}| = lcm(|x|, |y|)$$
we thus want to maximize the desired value of $lcm$. Both $Z_{75}$ and $Z_{100}$ are
cyclic, and thus
$$|n| = \frac{75}{gcd(n, 75)}$$
for $n \in Z_{75}$ and it's simular for a $Z_{100}$. We thus want to maximize the function
$$lcm(75/gcd(n, 75), 100/gcd(m, 100))$$
fundamental theorem of arithmetics essentially states that eny given positive number greater than
2 can be destructed to a multiset of primes, whose product is gonna be that number. $lcm$ in
that matter presents some sort of a uniom of multisets, that are connected to a given number,
and thus we can practically follow that we want $n$ and $m$ such that
$$n * m = lcm(75, 100)$$
since
$$75 = 3 * 5^2$$
and
$$100 = 2^2 * 5^2$$
let's take $n = 5^2 = 25$ so that $|n| = 3$ and let us take $m = 1$ so that $|m| = 2^2 * 5^2$.
this way we'll have that
$$lcm(n, m) = 3 * 2^2 * 5^2 = 300$$
Since we were'nt required to present a proper proof that a given number is an absolute maximum,
I'm gonna leave this exercisee at that.

\subsection{}

\textit{Show that $Z_5 \oplus Z_2$ is cyclic}

We follow that $|Z_5 \oplus Z_2| = 5 * 2 = 10$ and that
$$|\eangle{1, 1}| = 10$$

\subsection{}

\textit{Show that $Z_4 \oplus Z_2$ is not cyclic}

We've seen the orders of elements of those groups previously, and none of them are $8$.

\subsection{}

\textit{Skip}

\subsection{}

\textit{Let $A$ and $B$ be groups. Prove that the direct sum $A \oplus B$ is abelian of and
  only if $A$ and $B$ are both abelian}

Let's start with reverse implication: if $A$ and $B$ are abelian, then
$$\eangle{a, b} \eangle{c, d} = \eangle{ac, bd} = \eangle{ca, db} = \eangle{c, d} \eangle{a, b}$$
for arbitatry blah-blah-blah and thus as desired.

If $A \oplus B$ is abelian, then assume that $e$ is an identity for $B$ and $a, b \in A$ are
such that $ab \neq ba$. We follow then that $\eangle{ab, e} \neq \eangle{ba, e}$ but we've got that
$$\eangle{a, e} \eangle{b, e} = eangle{ab, e} = \eangle{b, e} \eangle{a, e}$$
which contradicts. Thus we conclude that $A$ is abelian, and the same can be followed by the
same thread of logic for $B$ and in general for arbitrary (but finite) direct sum of groups.

\subsection{}

\textit{Let $G$ and $H$ be two finite groups. Prove that $G \oplus H$ is cyclic if and only
  if $G$ and $H$ are both cyclic with $gcd(|G|, |H|) = 1$ }

if $G, H$ are cyclic and $gcd(|G|, |H|) = 1$, then we can take generators $a, b$ of both groups
to get
$$|\eangle{a, b}| = lcm(|a|, |b|) = lcm(|G|, |H|) = |G||H|$$
thus making the direct sum cycic, as desired.

$G \oplus H$ is cyclic if and only if there's an element $\eangle{a, b} \in G \oplus H$ such that
$$|\eangle{a, b}| = |G \oplus H|$$
i.e.
$$|\eangle{a, b}| = |G| |H|$$
we know that $|\eangle{a, b}| = lcm(|a|, |b|)$ and therefore $|\eangle{a, b}| = |G| |H|$ iff
$$lcm(|a|, |b|) = |G||H|$$
for all elements $k$ of an arbitrary finite group $K$ we've got that $|k| \leq |K|$ and thus 
if $|G|$ is not cycic, then $|a| < |G|$, and thus this equality won't hold. Same goes for $|H|$,
thus we follow that both $G, H$ are cyclic. We also follow that the equality won't hold if
$gcd(|G|, |H|) \neq 1$, which gives the desired conclusion.

\subsection{}

This one is trivial, skip.

\subsection{}

\textit{Find all groups of order 5}

Cyclic group is one of those.

If $|x| = 4$ then $e, x, x^2, x^3$ are all distinct. We follow that $|x^2| = 4/2 = 2$ and
$|x^3| = 4$. We then follow that $x\inv = x^3$ and $x^2$ is an inverse of itself. Thus we follow
that the last element $k$ is an inverse of itself, and thus has order of $2$. We then follow that
if $xk = k$, then $x = e$, which is not the case. Thus $xk = x^n$, which means that $k = x^{n - 1}$,
which is also not the case, thus giving us a contradiction.

If $|x| = 3$ and the group is not cyclic, then $\eangle{e, x, x^2}$ are all distinct. Let's name
the other elements as $a, b$ and thus we'll have a group $\set{e, x, x^2, a, b}$.
We follow that $ax \neq x^2$ since that would imply that $a = x$. We also follow that
$ax = a \ra x = e$, $ax = x \ra a = e$ and $ax = e \ra x\inv = a \ra a = x^2$, all of which
are contradictions. Thus we conclude that $ax = b$. Same reasoning leads us to a conclusion
that $bx = a$. Thus $bx^2 = ax = b$, and thus $bx^2 = b$, which implies that $x^2 = e$, which is
a contradiction. Thus we conclude that there's no element of order $3$.

If $|x| = 2$ and the group is not cyclic, then $e, x$ are distinct. This means that
we've got a group $\set{e, x, a, b, c}$. We follow from previous paragraph that there are
no elements of order $3$ or $4$, which implies that $|x| = |a| = |b| = |c| = 2$. We now
can follow that since all of the elements are equal to their inverses
$$ab = (ab)\inv = b\inv a\inv = ba$$
thus making the group abelian. We can also follow without loss of generality that
$ab = e \ra a = b\inv \ra a = b$, which gives us a contradiction, thus proving that
$ab \notin set{a, b}$. If $x = ab$, then $xc = abc$, therefore
$x \neq abc$, and thus $abc \in \set{a, b, c}$. If $abc = a$, then $bc = e$ and therefore $b = c$,
which is a contradiction. In general we follow that $abc \notin \set{a, b, c}$, and thus
$abc = e$. This implies that $xc = e$, which is a contradiction. Thus we conclude that
$xc$ is cannot be equal to non of the elements, which implies that there's no element,
whose order is equal to $2$ and the group is not cyclic, as desired.


\subsection{}

\textit{We consider groups of order 6. We know that $Z_6$ is a group of order $6$. We now
  look for all the others. Let $G$ be any group of order $6$ that is not cyclic.}

\textit{(a) Show that $G$ cannot have an element of order $7$ or higher}

Order of an element of a group is less than the order of the group, in which it is located. There's
an exercise that proves it.

\textit{(b) Show that $G$ cannot have an element of order $5$}

If $|x| = 5$ then $G = \set{e, x, x^2, x^3, x^4, a}$, therefore $ax = x^n$, which gives us
a countradiction.

\textit{(c) Show that $G$ cannot have an element of order $4$.}

Let $G = \set{e, x, x^2, x^3, a, b}$. We follow that
$$xa = e \ra a = x^3$$
$$xa = x \ra a = e$$
$$xa = x^2 \ra a = x$$
$$xa = x^3 \ra a = x^2$$
$$xa = a \ra x = e$$
thus $xa = b$. We then follow for the same reason that $xb \notin \set{e, x, x^2, x^3, b}$,
thus $xb = a$. Therefore $xa = xxb = x^2b = b$, thus $x^2 = e$, which gives us a contradiction.

\textit{(d) Show that the nonidentity elements of $G$ have order $2$ or $3$}

We follow that it's got to be either 2, 3, or 6. 6 is not an option since $G$ is not cyclic.

\textit{(e) Conclude that there exist only two subgroups of order $6$. In particular,
  there exists one abelian group of order $6$ (cyclic) and one nonabelian group of order $6$
  ($D_3$ is such a group)}

We follow that
$$|0| = 1, |1| = 6, |2| = 3, |3| = 2, |4| = 3, |5| = 6$$
for the cyclic group and
$$|e| = 1, |r| = 3, |r^2| = 3, |s| = 2, |sr| = 2, |sr^2| = 2$$
for the dihedral group.

We follow that order of all nonidentity elements cannot be equal to $3$, since there are
5 of those and none of them are equal to their inverses. Thus there's got to be an element of
order $2$.

If all the elements are of order $2$, then we follow that the group is abelian.
Let us denote first nonidentity element by $a$ and the second one by $b$. We follow that
$ab \notin \eangle{e, a, b}$, thus let $c = ab$. We now follow that $abc = c^2 = e$. We then
follow that there are also another two elements $d, f$. If $df = a$, then we follow that
$f = ad$ and $d = fa$ and thus we've got that
$$db = e \ra d = b$$
$$db = a \ra d = ab \ra d = c$$
$$db = b \ra d = e$$
$$db = c \ra db = ab \ra d = a$$
$$db = d \ra b = e$$
$$db = f \ra fdb = e \ra ab = e \ra c = e $$
thus we've got ourselves a much desired contradiction. Thus we can conclude that there's an
element of $G$ that is not of order $2$, which means that at least one of the elements of $G$
has order $3$. We then can follow that there are at least two of those, since it's got to
have an inverse.

Now suppose that there are 4 elements of order 3. We name'em by $a, a\inv, b, b\inv$. We follow
that $|a^2| = 3$ and $a^2 \notin \set{e, a\inv, b}$, thus $a^2 = b\inv$. For the same
reason $b^2 = a\inv$. Thus $ab = (b\inv a\inv)\inv = (a^2 b^2)\inv = b^{-2} a^{-2}$. Thus
$$ab \notin \set{e, a, b}$$
$$ab = b\inv \ra ab = a^2 \ra b = a$$
$$ab = a\inv \ra ab = b^2 \ra a = b$$
thus we conclude that $ab = c$, which is our only element of order $2$. Thus
$$ab = (ab)\inv$$.
Therefore $ab = (ab)\inv = b\inv a\inv = a^2 b^2$. THus
$$ab = a^2 b^2 \ra b = ab^2 \ra ab = e$$
which is a contradiction. Thus we conclude that there couldn't be 4 elements of order 3 and thus
we conclude that there are only 2 of them.

Now let $G$ be a group of order $6$, that is not cyclic. We follow that it's got $e$, an
element $s$ of order $2$ and an element $r$ of order $3$. We then follow that
$\set{e, r, r^2}$ are all distinct and that $s \neq r^2$ since $|s| = 2$ and
$|r^2| = 3/gcd(3, 2) = 3$. Since $r$ and $r^2$ are the only elements of order $3$, we conclude
that $r\inv = r^2$. We now follow that $sr \notin \set{e, s, r, r^2}$ and thus it's its own element.
$$sr^2 \notin \set{e, s, r, r^2}$$, thus it's its own element as well. We then can create
a Caley table fot all those elements to prove that they don't have no contradiction, thus concluding
that this is the only possible non-cyclic group of order $6$.

The cyclic group is unique, since all 6 powers of a generator have got to be unique. Thus we
conclude that the given group (i.e. $D_3$) and cyclic groups are the only ones that have
order $6$.

\textit{The rest is skipped}

\end{document}

