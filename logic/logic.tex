\documentclass[11pt,oneside,titlepage]{book}
\title{My topology exercises}
\usepackage{amsmath, amssymb}
\usepackage{geometry}
\usepackage{hyperref}
\author{Evgeny Markin}
\date{2023}

\DeclareMathOperator \map {\mathcal {L}}
\DeclareMathOperator \pow {\mathcal {P}}
\DeclareMathOperator \topol {\mathcal {T}}
\DeclareMathOperator \basis {\mathcal {B}}
\DeclareMathOperator \ns {null}
\DeclareMathOperator \range {range}
\DeclareMathOperator \fld {fld}
\DeclareMathOperator \inv {^{-1}}
\DeclareMathOperator \Span {span}
\DeclareMathOperator \lra {\Leftrightarrow}
\DeclareMathOperator \eqv {\Leftrightarrow}
\DeclareMathOperator \la {\Leftarrow}
\DeclareMathOperator \ra {\Rightarrow}
\DeclareMathOperator \imp {\Rightarrow}
\DeclareMathOperator \true {true}
\DeclareMathOperator \false {false}
\DeclareMathOperator \dom {dom}
\DeclareMathOperator \ran {ran}
\newcommand{\eangle}[1]{\langle #1 \rangle}
\newcommand{\set}[1]{\{ #1 \}}
\newcommand{\qed}{\hfill $\blacksquare$}

\begin{document}
\maketitle
\tableofcontents

\chapter*{Preface}

Those are my solutions and notes for "A Concise Introduction to Mathematical Logic" (3rd edition)
by Wolfgang Rautenberg

\chapter{Propositional Logic}

\section{Boolean Functions and Formulas}

\subsection{}

\textit{$f \in B_n$ is called linear if $f(x_1, ..., x_n) = a_0 + a_1 x_1 + ... + a_n x_n$ for
  suitable coefficients $a_0, ..., a_n \in \set{0, 1}$}

We firstly going to assume that $+$ is associative and commutative.

\textit{(a) Show that the above representation of a linear function $f$ is unique}

By constructing an appropriate table we can prove that
$$a_0 + a_1 x_1 = b_0 + b_1 x_1 \iff a_0 = b_0 \land a_1 = b_1$$

Assume that
$$\sum_{i < n}{a_i x_i} = \sum_{i < n}{b_i x_i} \iff \set{a_n} = \set{b_n}$$
Now assume that
$$\sum_{i < n}{a_i x_i} + a_n x_n = \sum_{i < n}{b_i x_i} + b_n x_n$$
we follow that if $a_n \neq b_n$, then without loss of generality we can assume that
$a_n = 0$ and $b_n = 1$. Thus
$$\sum_{i < n}{a_i x_i} + x_n = \sum_{i < n}{b_i x_i}$$
Let $\set{q_n}$ be a vector of boolean variables. Substituting all the $x$'s in
$\sum_{i < n}{a_i x_i}$ for $q$'s we're going to get result $m$. If $m = 0$, then we can
set $x_n$ to $1$ to follow that
$$\sum_{i < n}{a_i q_i} + q_n = 1 \neq  \sum_{i < n}{b_i x_i}$$
and if $m = 1$, then we can set $q_n = 1$ to also get
$$\sum_{i < n}{a_i q_i} + q_n = 0 \neq  \sum_{i < n}{b_i x_i}$$
thus concluding that (attention to $\leq$)
$$\sum_{i \leq n}{a_i x_i} + a_n x_n = \sum_{i \leq n}{b_i x_i} + b_n x_n \lra \set{a_n} = \set{b_n}$$
now we can use the induction to conclude the desired result.

\textit{(b) Determine the number of $n$-ary Boolean functions}

Since linear functions are determined uniquely by their coefficients, it's easy to say
that there are $2^{n + 1}$ $n$-ary linear Boolean functions

\textit{(c) Prove that each formula $\alpha$ in $\neg, +$ (i.e. $\alpha$ is a formula of the
  logical signature $\set{\neg, +}$) represents a linear Boolean functions.}

We follow that
$$\neg(x + y) = 0 + x + y$$
Given associativity and commutativity of the $+$ we follow the desired result

\textit{The rest of the exercises are pretty trivial, so I'm gonna leave them alone}

\subsection{Semantic Equivalence and Normal Forms}

\end{document}