\documentclass[11pt,oneside,titlepage]{book}
\title{My probability and statistics exercises}
\usepackage{amsmath, amssymb}
\usepackage{geometry}
\usepackage{hyperref}
\author{Evgeny Markin}
\date{2023}

\DeclareMathOperator \map {\mathcal {L}}
\DeclareMathOperator \ns {null}
\DeclareMathOperator \range {range}
\DeclareMathOperator \inv {^{-1}}
\DeclareMathOperator \Span {span}
\DeclareMathOperator \imp {\Rightarrow}
\DeclareMathOperator \lra {\Leftrightarrow}
\newcommand{\eangle}[1]{\langle #1 \rangle}


\begin{document}
\maketitle
\tableofcontents

\chapter{Introduction to Probability}

\section{The History of Probability}

\section{Interpretations of Probability}

\section{Experiments and Events}

\section{Set Theory }

\textit{Exercises in this section (or exercises similar to them) are handled in the set theory
  course}

\section{The Definition of Probability}

\begin{tabular}[center]{||c | c|| }
  \hline
  1 & 2/5 \\
  2 & 0.7 \\
  3a & 1/2 \\
  3b & 1/6 \\
  3c & 3/8 \\
  4 & 0.6 \\
  5 & 0.4 \\
  6 & - \\
  \hline 
\end{tabular}


\end{document}
%%% Local Variables:
%%% mode: latex
%%% TeX-master: t
%%% End:
