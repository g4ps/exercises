\documentclass[11pt,oneside,titlepage]{book}
\title{My probability and statistics exercises}
\usepackage{amsmath, amssymb}
\usepackage{geometry}
\usepackage{hyperref}
\author{Evgeny Markin}
\date{2023}

\DeclareMathOperator \map {\mathcal {L}}
\DeclareMathOperator \pow {\mathcal {P}}
\DeclareMathOperator \topol {\mathcal {T}}
\DeclareMathOperator \basis {\mathcal {B}}
\DeclareMathOperator \ns {null}
\DeclareMathOperator \range {range}
\DeclareMathOperator \fld {fld}
\DeclareMathOperator \inv {^{-1}}
\DeclareMathOperator \Span {span}
\DeclareMathOperator \lra {\Leftrightarrow}
\DeclareMathOperator \eqv {\Leftrightarrow}
\DeclareMathOperator \la {\Leftarrow}
\DeclareMathOperator \ra {\Rightarrow}
\DeclareMathOperator \imp {\Rightarrow}
\DeclareMathOperator \true {true}
\DeclareMathOperator \false {false}
\DeclareMathOperator \dom {dom}
\DeclareMathOperator \ran {ran}
\newcommand{\eangle}[1]{\langle #1 \rangle}
\newcommand{\set}[1]{\{ #1 \}}
\newcommand{\qed}{\hfill $\blacksquare$}



\begin{document}
\maketitle
\tableofcontents

\chapter*{Preface}

\section{Notation}

Some notable deviations from book's notation are presented here
\begin{enumerate}
\item Sometimes instead of p.d.f. and the likes of it, we write PDF
\item Countable set is defined as a set that has an injection into naturals (i.e. countably infinite
  and finite)
\end{enumerate}

\chapter{Introduction to Probability}

\section{The History of Probability}

\section{Interpretations of Probability}

\section{Experiments and Events}

\section{Set Theory}

Although the section is not pretty complex, some definitions can use a tad bit of rigor.

Sample space is a set. An element of this set is called an outcome.

Event is a subspace of a sample space. Sometimes elements of sample space (i.e. outcomes)
are called elementary events in some litrature. It might also be the case that elementary
outcomes denote singletons (i.e. sets with one element) that contain an element of a sample space
(i.e. an outcome). Same idea can be applies to the term "outcome" (i.e. sometimes
singletons are reffered to as an outcome)

Although this books says that any given subset of a sample space is an event, Cormen's
book notes that there are some restrictions when it comes to whacky subsets of uncountably infinite
sets. I guess that eventually MIRA will place everything in order.

Event that is equal to the sample space is called a certain event. Empty event is
called a null set. Disjoint events are called mutually exclusive. It doesn't take
a genius to figure out that elementary events are mutually exclusive.

If $S$ is a sample space, and $f: \pow(S) \to R$ is a function such that
\begin{enumerate}
\item $\range(f) \subseteq [0, \infty)$
\item $f(S) = 1$
\item if $A$ is a countable set of events (i.e. $K \subseteq \pow(S)$ and $|K| \leq_c \omega$),
  and it is indexed by $\omega$  (i.e. $A = \set{A_1, A_2, ...}$) then
  $f(\bigcup_{i \in \omega}{A_i}) = \sum{f(A_i)}$
\end{enumerate}
then $f$ is called a probability measure, or just probability. Canonical representation of
probability measure is $Pr$ and above mentioned items are usually refered to as
axioms of probability.

Definitions for this note were compiled from the book itself, book on measure theory (MIRA by Axler)
and general notes from the internet. Some definitions and terms were also borrowed from
the Cormen's (et. al) book on algorithms (Appendix C). As of the time of writing this, the
course on Cormen's book is currently ongoing within the same project, and MIRA is in the backlog.

\textit{Exercises in this section (or exercises similar to them) are handled in the set theory
  course}

\section{The Definition of Probability}

\begin{tabular}[center]{||c | c|| }
  \hline
  1 & 2/5 \\
  2 & 0.7 \\
  3a & 1/2 \\
  3b & 1/6 \\
  3c & 3/8 \\
  4 & 0.6 \\
  5 & 0.4 \\
  6 & 0.5 \\
  8 & 30 \\
  11a & 1 - $\pi/4$ \\
  11b & 0.75 \\
  11c & 2/3\\
  11d & 0 \\
  14a & 0.38, 0.16 \\
  14b & 0.04 \\
  \hline 
\end{tabular}


A little notation, related to 6: 
$$Pr(A) = 0.5$$
$$Pr(B) = 0.2$$
$$Pr(A \cap B) = 0.1$$
$$Pr(A \cup B) = 0.6$$
$$Pr((A \cup B) \cap (A \cap B)^c) = P(A \cup B) - P((A \cup B) \cap (A \cap B)) =
P(A \cup B) - P(A \cap B) = 0.5$$

\subsection*{1.5.7}

If $Pr(A) = 0.4$ and $Pr(B) = 0.7$, then we follow that the maxium $Pr(A \cap B)$ is attained
if $A \subset B$, in which case $Pr(A \cap B) = Pr(A) = 0.4$. The minimum is obtained
if $A \cup B = S$, in which case $Pr(A \cap B) = 0.1$

\subsection*{1.5.9}

The event that exaclty one of the events occurs can be expressed as
$$(A \cap B^c) \cup (A^c \cap B)$$
which comes from either the definition of xor, common sense or something else, depending on
your preferences. Thus we follow that
$$Pr((A \cap B^c) \cup (A^c \cap B)) = Pr(A \cap B^c) + Pr(A^c \cap B) -
Pr((A \cap B^c) \cap (A^c \cap B)) = $$
$$ = 
Pr(A \cap B^c) + Pr(A^c \cap B) - Pr((A \cap A^c) \cap (B^c \cap B)) = $$
$$ =   Pr(A \cap B^c) + Pr(A^c \cap B) = Pr(A) - Pr(A \cap B) + Pr(B) - Pr(B \cap A) = $$
$$= Pr(A) - Pr(A \cap B) + Pr(B) - Pr(A \cap B) =  Pr(A) + Pr(B) - 2Pr(A \cap B)$$
as desired (rules used in this derivitation: association of unions, $A \cap A^c = \emptyset$
and other trivial stuff)

\subsection*{1.5.10}

$$Pr(A \cap B^c) = Pr(A) - Pr(A \cap B)$$
$$Pr(A \cap B^c) + Pr(A \cap B) = Pr(A) $$
as desired.

\subsection*{1.5.12}

Suppose that $n > m \in N$. Then we follow that by definition
$$B_m \subseteq A_m$$
and
$$B_n \subseteq A_m^c$$
thus we follow that
$$B_m \cap B_n \subseteq A_m \cap A_m^c = \emptyset$$
thus
$$B_m \cap B_n = \emptyset$$
therefore we conclude that $B_1, B_2 ...$ are disjoint sets. Thus we follow that
$$Pr(\bigcup_{i = 1}^n B_i) = \sum_{i = 1}^n {Pr(B_i)}$$
For $n = 2$ we've got that
$$B_1 \cup B_2 = A_1 \cup (A_1^c \cap A_2) = (A_1 \cup A_1^c) \cap (A_1 \cup A_2) = A_1 \cup A_2$$
and by induction we can follow that
$$\bigcup_{i = 1}^n {B_i} = \bigcup_{i = 1}^n {A_i}$$
thus
$$Pr(\bigcup_{i = 1}^n B_i) = \sum_{i = 1}^n {Pr(B_i)}$$
implies that
$$Pr(\bigcup_{i = 1}^n A_i) = \sum_{i = 1}^n {Pr(B_i)}$$
for $n \in N$. Given that $n$ is arbirtary, we can follow that
$$Pr(\bigcup_{i = 1}^\infty A_i) = \sum_{i = 1}^\infty {Pr(B_i)}$$
as desired.

\subsection*{1.5.13}

First equation follow from induction on the result that
$$Pr(A \cup B) \leq Pr(A) + Pr(B)$$
the second equation follows from the first equation, DeMorgan laws and induction on the
form 
$$Pr(A \cap B) = Pr((A^c \cup B^c)^c) = 1 - Pr(A^c \cup B^c) \geq 1 - (Pr(A^c) + Pr(B^c))$$

\subsection*{1.5.14}

$$Pr(A) = 0.34$$
$$Pr(B) = 0.12$$
$$Pr(O) = 0.5$$
$$Pr(AB) = 1 - 0.34 - 0.12 - 0.5 = 0.04$$
$$Pr(a-A) = 0.34 + 0.04 = 0.38$$
$$Pr(a-B) = 0.12 + 0.04 = 0.16$$

\section{Finite Sample Spaces}

\begin{tabular}[center]{||c | c|| }
  \hline
  1 & 1/2 \\
  2 & 1/2 \\
  3 & 2/3 \\
  4 & 1/7 \\
  5 & 4/7 \\
  6 & 1/4 \\
  8b & 1/4 \\
  \hline 
\end{tabular}

\subsection*{1.6.7}

The possinble genotypes are $Aa$ and $aa$ with probabilities $1/2$ and $1/2$ respectively

\subsection*{1.6.8a}

The sample space of the experiment is $\{heads, tails\} \times \{1, 2, 3, 4, 5, 6\}$,

\section{Counting Methods}

\begin{tabular}[center]{||c | c|| }
  \hline
  1 & 14 \\
  2 & 9000 \\
  3 & 120 \\
  4 & 24 \\
  5 & 5/18 \\
  6 & 5/324 \\
  7 & ~0.014731 \\
  8 & 360 / 2401 \\
  9 & 1 / 20\\
  10a & r/100 \\
  10b & r/100 \\
  10c & r/100 \\
  \hline 
\end{tabular}

\subsection*{1.7.11}

$$s(n) = \frac{1}{2} \log(2 \pi) + (n + \frac{1}{2})\log{n} - n \approx \log{n!}$$
$$\log{n!} - \log{(n - m)!} = \log{\frac{n!}{(n - m)!}}$$
$$s(n) - s(n - m) = \frac{1}{2} \log(2 \pi) + (n + \frac{1}{2})\log{n} - n
- (\frac{1}{2} \log(2 \pi) + ((n - m) + \frac{1}{2})\log{n - m} - (n - m)) = $$
$$ = (n + \frac{1}{2})\log{n} - n
-   ((n - m) + \frac{1}{2})\log{(n - m)} + (n - m) = $$
$$ = (n + \frac{1}{2})\log{n}
-   ((n - m) + \frac{1}{2})\log{(n - m)}  - m \approx \log{\frac{n!}{(n - m)!}}$$
$P(n, m) = \frac{n!}{(n - m)!} = \exp(s(n) - s(n - m))$


\section{Combinatorial Methods}

\begin{tabular}[center]{||c | c|| }
  \hline
  1 & 184756 \\
  2 & latter \\
  3 & equal \\
  4 & 1 / 10626 \\
  5 & - \\
  6 & 2/n \\
  7 & (n - k - 1)/C(n, k) \\
  8 & (n - k)/C(n, k) \\
  9 & (n + 1)/C(2n, n)\\
  10 & $15/92 \approx 0.16304$ \\
  11 & $1 / 75 \approx 0.01333$ \\
  12 & $69 / 119 \approx 0.57983$ \\
  13 & $173 / 1518 \approx 0.114$ \\
  14 & - \\
  15 &  - \\
  16a & $48 / 175 \approx 0.27429$ \\
  16b & $2^{50}/C(100, 50) \approx 0$ \\
  17 & $4 C(13, 4)/ C(52, 4) = 44/ 4165 \approx 0.0105$\\
  18 & $C(20, 2)^5 / C(100, 10) \approx 0.0143$ \\
  19 & - \\
  20 & - \\
  21 & C(365 + 7 - 1, 7) \\
  22 & - \\
  \hline 
\end{tabular}


\subsection*{1.8.5}

\textit{Prove that
  $$ \frac{\prod_{4155 \leq i \leq 4251}{i}}{\prod_{2 \leq i \leq 97}{i}}$$
  is an integer}

$$\frac{\prod_{4155 \leq i \leq 4251}{i}}{\prod_{2 \leq i \leq 97}{i}} =
\frac{\prod_{4155 \leq i \leq 4251}{i}}{\prod_{1 \leq i \leq 97}{i}} = $$
$$ =  \frac{\prod_{4155 \leq i \leq 4251}{i}}{97!} =
\frac{4251!}{4154!97!} = \frac{4251!}{4154!(4251 - 4174)!} = C(4251, 4154)$$
and binomial coefficients are integers (pretty sure that we can follow that by induction
in some more advanced course).

\subsection*{1.8.10}

There are total of $C(24, 10)$ possible subsets of length 10 in the space of 24.
We follow that there are $C(22, 8)$ ways to pick 8 normal bulbs, which is what required to pick
2 defective bulbs. Therefore the probability is 
$$\frac{C(22, 8)}{C(24, 10)} = 15/92 \approx 0.16304...$$

\subsection*{1.8.12}

Using the same logic as in 1.8.10, there is a possibility $\frac{C(33, 8)}{C(35, 10)}$
that same two guys will be in the first team, and probability of  $\frac{C(33, 23)}{C(35, 10)}$
that they'll be in the other team. Thus the total probability is the sum of two.



\subsection*{1.8.14}

\textit{Prove that for all positive integers $n, k$ such that $n \geq k$
  $$C(n, k) + C(n, k - 1) = C(n + 1, k)$$
}

$$ C(n, k) + C(n, k - 1) = \frac{n!}{(n - k)!k!} + \frac{n!}{(n - k + 1)!(k - 1)!} =$$
$$ =
\frac{n!}{k (n - k)!(k - 1)!} + \frac{n!}{(n - k + 1)(n - k)!(k - 1)!} =$$
$$ =
\frac{(n - k + 1)n!}{k(n - k + 1) (n - k)!(k - 1)!} + \frac{kn!}{k(n - k + 1)(n - k)!(k - 1)!} =
$$
$$ =
\frac{(n - k + 1)n! + kn!}{k(n - k + 1) (n - k)!(k - 1)!} =
\frac{n!((n - k + 1) + k)}{k(n - k + 1) (n - k)!(k - 1)!} = 
$$
$$ =
\frac{n!(n + 1)}{k(n - k + 1) (n - k)!(k - 1)!} =
\frac{(n + 1)!}{((n + 1) - k)!k!} = C(n + 1, k)
$$
as desired.

\subsection*{1.8.15}

\textit{(a) Prove that 
  $$\sum_{i = 0}^n{C(n, i)} = 2^n$$
}

We can follow that from the fact that there are $2^n$ subsets of any given finite set,
which means that the number of subsets of different lengths sums up to $2^n$.

Another way to do this is to use binomial theorem:

$$(x + y)^n = \sum_{i = 0}^n{C(n, i) x^k y^{n - k}}$$
thus if we subisitute $x$ and $y$ for $1$, we get
$$(1 + 1)^n = \sum_{i = 0}^n{C(n, i) 1^k 1^{n - k}}$$
$$2^n = \sum_{i = 0}^n{C(n, i)}$$

\textit{(b) Prove that
  $$\sum_{i = 0}^n{(-1)^iC(n, i)} = 0$$
}

I'm sure that there is a neat explanation for this one as well, but using the binomial
theorem once again, but now substituting $1$ for x and $-1$ for $y$ we get
$$(1 - 1)^n = \sum_{i = 0}^n{C(n, i) 1^i (-1)^{n - i}}$$
$$\sum_{i = 0}^n{C(n, i) 1^i (-1)^{n - i}} = 0$$
we can follow through the even-odd argument that $1^i (-1)^{n - i} = (-1)^i$, but I'll
skip it.

\subsection*{1.8.19}

\textit{(rewording) Prove the formula for unordered sampling with replacement.}

This thing is ought to be covered rigorously in a course for discrete maths, combinatorics or
something of sorts.  Currentry there is a better
proof at Belcastro's "Discrete mathematics with ducks".

\subsection*{1.8.20}

\textit{Prove the binomial theorem 1.8.2}

1.8.2 states that
$$(x + y)^n = \sum_{i = 0}^{n}{C(n, i) x^i y^{n - i}}$$

Let
$$I = \set{n \in \omega: (x + y)^n = \sum_{i = 0}^{n}{C(n, i) x^i y^{n - i}}}$$
We follow that
$$(x + y)^0 = C(0, 0)x^0 y^0 = 1$$
Thus $0 \in I$. (we can start with a base case of $1$ as well for a more clear example,
but I like this one more, and it suffices as well).

Now suppose that $n \in I$. We follow that
$$(x + y)^n = \sum_{i = 0}^{n}{C(n, i) x^i y^{n - i}}$$
thus we follow that
$$(x + y)(x + y)^n = (x + y)\left[\sum_{i = 0}^{n}{C(n, i) x^i y^{n - i}}\right]$$

Left-hand side is reduced to
$$(x + y)(x + y)^n = (x + y)^{n + 1}$$
Right-hand side is obviously a bit trickier, but we can follow
$$
(x + y) \sum_{i = 0}^{n}{C(n, i)x^i y^{n - i}} =
$$
$$ =
x \sum_{i = 0}^{n}{C(n, i)x^{i} y^{n - i}} + y \sum_{i = 0}^{n}{C(n, i)x^i y^{n - i}} =
$$
$$ =
\sum_{i = 0}^{n}{C(n, i)x^{i + 1} y^{n - i}} + \sum_{i = 0}^{n}{C(n, i)x^i y^{n + 1 - i}} =
$$
$$ =
\sum_{i = 0}^{n}{C(n, i)x^i y^{n + 1 - i}} +  \sum_{i = 0}^{n}{C(n, i)x^{i + 1} y^{n - i}} =
$$
$$ =
C(n, n)x^{n + 1}y^0 + \sum_{i = 0}^{n}{C(n, i)x^i y^{n + 1 - i}}
+  \sum_{i = 0}^{n - 1}{C(n, i)x^{i + 1} y^{n - i}} =
$$
$$ =
x^{n + 1} + \sum_{i = 0}^{n}{C(n, i)x^i y^{n + 1 - i}}
+  \sum_{i = 0}^{n - 1}{C(n, i)x^{i + 1} y^{n - i}} =
$$
$$ =
x^{n + 1} + \sum_{i = 0}^{n}{C(n, i)x^i y^{n + 1 - i}}
+ x \sum_{i = 0}^{n - 1}{C(n, i)x^{i} y^{n - i}} =
$$
$$ =
x^{n + 1} + \sum_{i = 0}^{n}{C(n, i)x^i y^{n + 1 - i}}
+ x \sum_{i = 1}^{n}{C(n, i - 1)x^{i - 1} y^{n - (i - 1)}} =
$$
$$ =
x^{n + 1} + C(n, 0)x^0 y^{n + 1}  + \sum_{i = 1}^{n}{C(n, i)x^i y^{n + 1 - i}}
+ \sum_{i = 1}^{n}{C(n, i - 1)x^i y^{n + 1 - i}} =
$$
$$ =
x^{n + 1} + y^{n + 1} + \sum_{i = 1}^{n}{C(n, i)x^i y^{n + 1 - i}}
+ \sum_{i = 1}^{n}{C(n, i - 1)x^i y^{n + 1 - i}} =
$$
$$ = 
 x^{n + 1} + y^{n + 1} + \sum_{i = 1}^{n}{(C(n, i) + C(n, i - 1))x^i y^{n + 1 - i}}  = 
$$
$$
= x^{n + 1} + y^{n + 1} + \sum_{i = 1}^{n}{C(n + 1, i)x^i y^{n + 1 - i}}  
= x^{n + 1} + C(n + 1, 0)x^0 y^{n + 1 - 0} + \sum_{i = 1}^{n}{C(n + 1, i)x^i y^{n + 1 - i}}  =
$$
$$
= x^{n + 1} + \sum_{i = 0}^{n}{C(n + 1, i)x^i y^{n + 1 - i}} 
= x^{n + 1}y^{0} + \sum_{i = 0}^{n}{C(n + 1, i)x^i y^{n + 1 - i}} =  $$
$$ = C(n + 1, n + 1)x^{n + 1}y^{n + 1 - (n + 1)} + \sum_{i = 0}^{n}{C(n + 1, i)x^i y^{n + 1 - i}} 
= \sum_{i = 0}^{n + 1}{C(n + 1, i)x^i y^{n + 1 - i}}$$
Thus we follow
$$(x + y)^{n + 1} = \sum_{i = 0}^{n + 1}{C(n + 1, i)x^i y^{n + 1 - i}}$$
or
$$(x + y)^{n^+} = \sum_{i = 0}^{n^+}{C(n^+, i)x^i y^{n^+ - i}}$$
which means that $n \in I \Rightarrow n^+ \in I$, from which we conclude that $I = \omega$, and
thus 
$$(x + y)^n = \sum_{i = 0}^{n}{C(n, i) x^i y^{n - i}}$$
for all $n \in \omega$, as desired.

\subsection*{1.8.22}

Skip

\section{Multinomial Coefficients}

\begin{tabular}[center]{||c | c|| }
  \hline
  1 & $(21!)/(7! * 7! * 7!)$\\
  2 & $50!/(18! * 12! * 12! * 8!)$ \\
  3 & $300!/(5! * 8 ! * 287!)$ \\
  4 & $(3! 3! 2!)/10! = 1/50400$\\
  5 & $M(n, (n_1,..., n_6)) / 6^n$ \\
  6 & $(7!) / (2 * 6^7)$ \\  
  7 & $M(12, (6, 2, 4)) * M(13, (4, 6, 3)) / M(25, (10, 8, 7))$ \\
  8 & $M(12, (3, 3, 3, 3) * M(40, (10, 10, 10, 10))/ M(52, (13, 13, 13, 13)$\\
  9 & $4! / M(52, (13, 13, 13, 13)$ \\
  10 & $(2! * 3! * 4!) / 9!$ \\
  \hline 
\end{tabular}

\section{The Probability of a Union of Events}

\begin{tabular}[center]{||c | c|| }
  \hline
  1 & $\approx 0.11913$ \\
  2 & 85 \\
  3 & 45 \\
  \hline 
\end{tabular}

\subsection*{1.10.1}

$$Pr(A_1) = Pr(A_2) = Pr(A_3) = C(4, 2) * C(48, 3) /C(52, 5)$$
$$Pr(A_1 \cup A_2) = Pr(A_1 \cup A_3) = Pr(A_2 \cup A_3) =
C(4, 2) * C(48, 3) * C(45, 3) / C(52, 5)^2$$
$$Pr(A_1 \cup A_2 \cup A_3) = 0$$

$$Pr(A_1 \cup A_2 \cup A_3) = 3 * C(4, 2) * C(49, 3) /C(52, 5) -
3 C(4, 2) * C(49, 3) * C(46, 3) / C(52, 5)^2$$


TODO later (probably never).

\chapter{Conditional Probability}

\section{Definition of Conditional Probability}

\begin{tabular}[center]{||c | c|| }
  \hline
  1 & $Pr(A) / Pr(B)$ \\
  2 & 0 \\
  3 & $Pr(A)$ \\
  4 & $1/27 \approx 0.037037$ \\
  5 & - \\
  6 & $2/3$ \\
  7 & $1/3$ \\
  8 & $0.6 / 0.85 \approx 0.706$ \\
  9a & $3/4$ \\
  9b & $3/5$ \\
  10 & $0.4485884485884486$ \\
  11 & - \\
  12 & - \\
  13 & $4/9$ \\
  14 & $0.056$ \\
  15 & $0.47$ \\
  16 & $5/12$ \\
  17 & - \\
  \hline 
\end{tabular}

$$Pr(A|B) = \frac{Pr(A \cap B)}{Pr(B)}$$

\subsection*{2.1.5}

$$\frac{r}{r + b} *  \frac{(r + k)}{(r + k) + b} * \frac{(r + 2k)}{(r + 2k) + b} *
\frac{b}{(r + 3k) + b} $$

\subsection*{2.1.6}

Let $A$ be an event, that we've picked up a card, looked at its side and that the side is green.
We can follow that
$$Pr(A) = 1/2$$
Let $B$ be an event that we've picked up a card, and it's green on both sides. We follow that
$$Pr(B) = 1/3$$
Probability that both $A$ and $B$ happened are $1/3$. Thus we follow that 
$$Pr(B | A) = \frac{Pr(A \cap B)}{Pr(A)} = \frac{1/3}{1/2} = 2/3$$
This makes me think about Monty Hall problem, as those two are (probably) closely related.

\subsection*{2.1.11}

We want to prove that
$$Pr(A^c|B) = 1 - Pr(A|B)$$
we follow that by
$$Pr(A^c|B) = \frac{Pr(A^c \cap B)}{Pr(B)} = \frac{Pr(B) - Pr(A \cap B)}{Pr(B)} =
1 - \frac{Pr(A \cap B)}{Pr(B)} = 1 - Pr(A|B)$$
where
$$Pr(A^c \cap B) = Pr(B) - Pr(A \cap B)$$
is proven in Theorem 1.5.6.
as desired.

\subsection*{2.1.12}

$$Pr(A \cup B | D) = \frac{Pr((A \cup B) \cap D)}{Pr(D)} =
\frac{Pr((A \cap D) \cup (B \cap D))}{Pr(D)} = $$
$$ =
\frac{Pr(A \cap D) + Pr(B \cap D) - Pr(A \cap D \cap B \cap D)}{Pr(D)} = $$
$$ =  \frac{Pr(A \cap D) + Pr(B \cap D) - Pr(A \cap B \cap D)}{Pr(D)} =  $$
$$ =  \frac{Pr(A \cap D)}{Pr(D)} + \frac{Pr(B \cap D)}{Pr(D)}
- \frac{Pr(A \cap B \cap D)}{Pr(D)} =  Pr(A|D) + Pr(B|D) - Pr(A \cap B|D)$$
every deriviation that was done here was either justified by a theorem in section 1.5 or
is a property of set operations.

\subsection*{2.1.17}

We can't have
$$Pr((A|C)|B)$$
on the account that $A|C$ is not an event, but just a funky notation introduced with the
probability function. What this notation gives is just a syntactic sugar.

$$Pr(A|C) = \frac{Pr(A \cap C)}{Pr(C)} = \frac{1}{Pr(C)} Pr(A \cap C) =
\frac{1}{Pr(C)} \sum_{j = 1}^n{Pr(B_j)Pr(A \cap C | B_j)} = 
$$
$$ =
\frac{1}{Pr(C)} \sum_{j = 1}^n{Pr(B_j)\frac{Pr(A \cap C \cap B_j)}{Pr(B_j)}} =
\sum_{j = 1}^n{Pr(B_j)\frac{Pr(A \cap C \cap B_j)}{Pr(B_j)Pr(C)}} = 
$$
$$ =
\sum_{j = 1}^n{\frac{Pr(A \cap C \cap B_j)}{Pr(C)}} =
\sum_{j = 1}^n{\frac{Pr(B_j \cap C) Pr(A \cap C \cap B_j)}{Pr(B_j \cap C)Pr(C)}} = 
$$
$$ =
\sum_{j = 1}^n{\frac{Pr(B_j \cap C) Pr(A  \cap B_j \cap C)}{Pr(C) Pr(B_j \cap C)}} = 
$$
$$ = \sum_{j = 1}^n{\frac{Pr(B_j \cap C)}{Pr(C)} *
  \frac{Pr(A \cap B_j \cap C)}{Pr(B_j \cap C)}} = \sum_{j = 1}^n{Pr(B_j|C)Pr(A|B_j \cap C)}$$
assuming that $Pr(B_j \cap C), Pr(C) \neq 0$ for all $1 \leq j \leq n$.

\section{Independent Events}

\begin{tabular}[center]{||c | c|| }
  \hline
  1 & $Pr(A^c)$ \\
  2 & - \\
  3 & - \\
  4 & $1/216$ \\
  5 & $1 - 10^{-6}$\\
  6 & $149/5000 = 0.0298$\\
  7a & $23/25 = 0.92$ \\
  7b & $20/23 \approx 0.869565$ \\
  8 & $1/36 \approx 0.0277778$ \\
  9 & $1/7 \approx 0.142857$ \\
  10 & $\frac{106}{781} \approx 0.1357234314980794$ \\
  11 & $67/256 = 0.26171875$ \\
  12a & $3/4 = 0.75$ \\
  12b & $11/24 \approx 0.4583333333$ \\
  13 & $0.09135172474836409$ \\
  14 & $0.09561792499119552$ \\
  15 & $161$ \\ 
  \hline 
\end{tabular}

\subsection*{2.2.1}

Suppose that $A$ and $B$ are independent events. Thus
$$P(A|B) = P(A)$$
and
$$P(B|A) = P(B)$$

thus 
$$Pr(A^c|B^c) = \frac{Pr(A^c \cap B^c)}{Pr(B^c)} = \frac{Pr((A \cup B)^c)}{Pr(B^c)} =
\frac{1 - Pr(A \cup B)}{Pr(B^c)} =  $$
$$ =
\frac{1 - (Pr(A) + Pr(B) - Pr(A) Pr(B))}{Pr(B^c)} =
\frac{1 - Pr(A) - Pr(B) + Pr(A) Pr(B))}{Pr(B^c)} =  
$$
$$ =
\frac{1  - Pr(B) - Pr(A) + Pr(A) Pr(B))}{Pr(B^c)} =
\frac{1  - Pr(B)}{Pr(B^c)} +  \frac{ - Pr(A) + Pr(A) Pr(B))}{Pr(B^c)} =  
$$
$$
=
1 +  \frac{ Pr(A)( -1 + Pr(B))}{Pr(B^c)} =
1 -  \frac{ Pr(A)(1 - Pr(B))}{Pr(B^c)} =
1 -  Pr(A) \frac{ 1 - Pr(B)}{Pr(B^c)} =  
$$
$$
= 1 - Pr(A) = Pr(A^c)
$$

Same goes for $Pr(B^c|A^c)$

\subsection*{2.2.2}

2.2.1 implies that
$$Pr(A^c) = Pr(A^c|B^c)$$
and
$$Pr(B^c) = Pr(B^c|A^c)$$
for the nonzero cases, and if $Pr(A) = 0$ or $Pr(B) = 0$, then the cases are trivial. 

\subsection*{2.2.3}

Suppose that $A$ is an event and $Pr(A) = 0$ and $B$ is another event.
We follow that
$$Pr(A \cap B) \leq Pr(A)$$
and thus
$$Pr(A \cap B) = 0$$
as desired.

\subsection*{2.2.7b}

$$Pr(A|A \cup B) = \frac{Pr(A \cap (A \cup B))}{Pr(A \cup B)} =
\frac{Pr(A)}{Pr(A \cup B)} $$

\subsection*{2.2.9}

Assuming $1 \leq n \leq \infty$
$$\sum{(p_n)^3} = \sum{(2^{-n})^3} = \sum{2^{-3n}} = \sum{(1/8)^{n}} = \frac{1/8}{1 - 1/8} = 1/7$$

\subsection*{2.2.10}

Let $A$ be an event that at least 1 child in the family has blue eyes and
let $B$ be an event that at least 3 children have blue eyes. We follow that
$$Pr(B|A) = \frac{Pr(A \cap B)}{Pr(A)}$$
given that $B \subseteq A$, we follow that
$$Pr(B|A) = \frac{Pr(B)}{Pr(A)}$$
We follow that
$$Pr(A) = 1 - (1 - 1/4)^5 = 781/1024$$
and
$$Pr(B) = \sum_{i \in \set{3, 4, 5}}{C(n, i)1/4 * C(n, n - i)(1 - 1/4)} =
\sum_{i \in \set{3, 4, 5}}{C(n, i)(1/4)^i(3/4)^{5 - i}} = 53/512
$$
thus
$$Pr(B|A) = \frac{Pr(B)}{Pr(A)} = \frac{106}{781} \approx 0.1357234314980794$$

\subsection*{2.2.11}

If the youngest child in the family has the blue eyes, then we can't say that $B \subseteq A$.
Given that the probabilitiees of children having different colored eyes are independent,
we follow that we can rewrite this problem as "what's the probability of that the
remaining 4 children have at least 2 blue-eyed children among them". This happens to be equal to
$$\sum_{i \in \set{2, 3, 4}}{C(4, i)(1/4)^i(3/4)^{4 - i}} = 67/256 = 0.26171875$$

\textit{Done with this section; moving on}

\section{Bayes' Theorem}

\begin{tabular}[center]{||c | c|| }
  \hline
  1 & - \\
  2 & 3 \\
  3 & 0.3 \\
  4 & 0.0001899658061548921 \\
  5 & 0.30508474576271183 \\
  6a & 0.9896907216494846 \\
  6b & 0.9846153846153847 \\
  7a & 0, 1/10, 1/5, 3/10, 2/5 \\
  8 & skip \\
  16 & - \\
  \hline 
\end{tabular}

\subsection*{2.3.1}

Suppose that $S$ can be partitioned into $B_1, ..., B_k$. Suppose also  that $A$
is an event such that $Pr(A) > 0$ and 
$$Pr(B_1|A) < Pr(B_1)$$
and
$$Pr(B_i|A) \leq Pr(B_i)$$
for all $1 < i \leq k$.
Thus we follow that
$$\sum{Pr(B_i|A)} < \sum{Pr(B_i)} = 1$$
thus
$$\sum{Pr(B_i|A)} <  1$$
$$\sum{\frac{Pr(B_i \cap A)}{Pr(A)}} <  1$$
$$\sum{Pr(B_i \cap A)} <  Pr(A)$$
Given that $B_i$ is a partition of $S$, we follow that $B_i$'s are disjoint (BTW if several sets
are all pairwise disjoint, then all of them are disjoint), therefore we follow that
$B_j \cap A$ is disjoint from $B_l \cap A$ for all $1 \leq j, l \leq k$. Thus
$$\sum{Pr(B_i \cap A)} = Pr(\bigcup{[B_i \cap A]}) = Pr(\bigcup{[B_i]} \cap A) = Pr(S \cap A)
= Pr(A) <  Pr(A)$$
which is a contradiction.

\subsection*{2.3.16}

\textit{(a)}

Suppose that $D_1$ is independent of $B$. That is,
$$Pr(D_1) = Pr(D_1 | B) = 0.01$$

Assume that for some $n$ we've got that
$$Pr(D_n) = 0.01$$
We follow that
$$Pr(D_{n + 1}|B) = 0.01$$ 
If $B^c$ is true and we know that $n$'th item is normal, then we can follow that 
$$Pr(D_{n + 1}| D_n^c \cap B^c ) = 1/165$$
If $n$'th item is defective, then
$$Pr(D_{n + 1}| D_n \cap B^c ) = 2/5$$
therefore, because $D$ and $D^c$ are partitioning space, we follow that
$$Pr(D_{n + 1}|B^c) = Pr(D_n^c) * 1/165 + Pr(D_n) * 2/5 = 0.01$$
thus we now can follow that
$$Pr(D_{n + 1}) = 0.1 * 0.7 + 0.01 * 0.3 = 0.1$$
therefore by induction we can conclude that $Pr(D_n) = 0.01$ for all $n \in N$

\textit{(b)}

Let us assume that we've got a typo in the text, and we actually need to compute $Pr(B|E)$.
From our initial assumptions we follow that
$$Pr(E|B) = 0.99^4 * 0.01^2 = 9.65 * 10^{-5}$$
thus we need to compute
$$Pr(B|E) = \frac{Pr(E|B) * Pr(B)}{Pr(E|B) * Pr(B) + Pr(E|B^c) * Pr(B^c)}$$
thus the only thing that we need to compute is $Pr(E|B^c)$.
We follow that
$$Pr(E | B^c) = $$
$$ = Pr(D_1^c \cap D_2^c \cap D_3 \cap D_4 \cap D_5^c \cap D_6^c | B^c) =
Pr(D_1^c|B^c) Pr(D_2^c |D_1^c \cap B) Pr(D_3| D_2^c \cap B) ...  = $$
$$ = 0.99 * 164/165 * 1/165 * 2/5 * 3/5 * 164/165 = 0.99 * (164/165)^2 * 1/165 * 2/5 * 3/5  = $$
$$ =  0.001422598347107438$$
thus we can now compute the rest and state that
$$Pr(B|E) = 0.11898006688921978 \approx 12\%$$

\section{The Gambler's Ruin Promlem}

\begin{tabular}[center]{||c | c|| }
  \hline
  1 & - \\
  2 & all the same \\
  3 & a \\
  4 & c \\
  5 & 198 \\
  6 & 7 \\
  7 & - \\
  \hline 
\end{tabular}

\subsection*{2.4.1}

Suppose that we've got conditions from Example 2.4.2. Let $i$ be a natural number such that
$i \leq 98$. Probability that gambler $A$'s gonna win $i$ dollars before losing $100 - i$ is
$$a_i = \frac{(3/2)^i - 1}{(3/2)^{100} - 1}$$
we follow that $a_i$ is an increasing function and thus we can conclude that in order to
get the desired conclusion, we need to calculate the case $i = 98$. We follow that
$$a_{98} = \frac{(3/2)^{98} - 1}{(3/2)^{100} - 1} \approx 0.444444$$
BTW, it's not a pretty rational number.

\subsection*{2.4.7}

we follow that
$$f_i = \frac{(1/3)^i - 1}{(1/3)^{i + 2} - 1}$$
is the desired function. We want to show that the function is decreasing and $a_1 < 1/4$.
Simple calculation show that $a_1 \approx 0.14285714285714282$. We also follow that
$$f_n - f_{n + 1} = \frac{(1/3)^n - 1}{(1/3)^{n + 2} - 1} -
\frac{(1/3)^{n + 1} - 1}{(1/3)^{n + 3} - 1}$$
Maxima shows that this thing is equal to
$$- \frac{16 * 3^{n + 2}}{something . positive}$$
which is good enough for me to prove that this thing is always below $1/4$, as desired.

\textit{Done with this section}

\chapter{Random Variables and Distributions}

\section{Random Variables and Discrete Distributions}

\subsection*{Notes}

Let $S$ be a sample space

A random variable is a function $f: S \to R$ (which is confusing). Canonical representations
of random variables are capital English letters (i.e. $X, Y, etc.$)

If $X$ is a random variable, then $R$ can be thought of as a sample space, and then
we can define $g: \pow(R) \to R$ such that
$$g(C) = Pr(\set{s \in S: X(s) \in C})$$
Function $g$ is then called distribution of $X$, and it is a probability measure on $S$
(in the book, definition of probability is somewhat un-intuitive, but the not just below the
definition produces given definition)

Random variable $X$ has a discrete distribution (we also say that $X$ is a discrete variable)
if $\range(X)$ is countable. Distribution of the variable does not show up in the definition, which
is not helpful.

In general, we abuse notation in this book a lot. It's not hard to decode what $Pr(X = c)$ is
supposed to mean.

If $X$ is discrete and $g$ is a distribution of $X$,
then a probability function $f: R \to R$ is a function such that
$f(x) = g(\set{x})$
Closure of $\set{x \in R: f(x) > 0}$ is called a support of $X$. 
Although the term "closure" was not defined anywhere in this book, from the internet it seems that
it is the topological notion of closure. It is nice that in the course of that book we've defined
union, but we haven't mentioned closure at all.

If random variable $X$ has range $\set{0, 1}$ with $Pr(X = 1) = p$ for some $p \in R$, then we say
that $X$ has the Bernoulli distribution with parameter $p$. Although it might seem that
any given random variable with range of size of 2 has Bernoulli distribution, I'm
pretty sure that we won't encounter such variables in the wild, but if for some reason
I will ever have to use such a function, I'll name its distribution proto-Bernoulli.

Random variable $X$ has uniform distribution of the integers $a, ..., b$
(i.e. $Z \cap [a, b]$ for some $a, B \in Z$ such that $a \leq b$) if $f(m) = f(n)$ for
all $m, n \in Z \cap [a, b]$. 

\begin{tabular}[center]{||c | c|| }
  \hline
  1 & 6/11 \\
  2 & 1/15 \\
  3 & no \\
  4 & binomial with 10 and 1/2 \\
  5 & skip \\
  6 & 0.15087890625\\
  7 & 0.80589565 \\
  8 & 0.13295332343433508 \\
  9 & 1/2 \\
  10a & 1/120 (x + 1)(8 - x) \\
  10b & 1/3 \\
  11 & harmonics\\
  \hline 
\end{tabular}

\section{Continous Distributions}

\begin{tabular}[center]{||c | c|| }
  \hline
  1 & 4/9 \\
  2 & 31/48, 9/16, 136/243 \\
  3 & 1/2, 13/27, 2/27 \\  
  \hline 
\end{tabular}

\textit{The rest of that damned section is just exercises in trivial calculus. Skipping all
  this stuff.}

\section{The Cumulative Distribution Function}

\subsection*{Notes}

c.d.f. is a really nice way to describe distribution of a given random variable. Firstly,
we don't care whether or not the variable is discrete, continous, or whatever, it's got to
have a c.d.f. Secondly, it's an increasing bounded function from reals to reals, which implies
that any discontinuity in a given function is a jump discontinuity (see section 4.6. in real
analysis course) and the set of its discontinuities is countable. Also, given distribution
of a random variable, the c.d.f. is unique.

If we get into deeper parts of the book, we can conclude that we can ditch somewhat
non-rigorous notions of p.d.f and p.f. and concentrate exclusively on c.d.f. of a given
random variable for all the theoretical parts.



\section{Bivariate Distributions}

\subsection*{Notes}

It's kinda hard to describe the terminology here on the account of the
fact that it doesn't make any sense.

We've already defined distibution as the probability on the set of
reals. By that logic, we can follow that bivariate distribution is
some whacky kind of a distribution, that we want to describe further,
right?  Well, no. Bivariate distribution is a probability, that is
defined on the $R^2$. Hense it's a completely different beast altogether. 

Suppose that we've got a couple of random variables (which are the
functions, so in turn we gotta have a couple of sample spaces, two
probabilities for those sample spaces, etc.).  We follow that we can
in that case see $R^2$ as the
sample space, that we've concocted out of all of those things. In that
case we say that the probability on the $R^2$ is
a joint (or a bivariate) distribution. Going a bit further I can also
foreshadow that multivariate distribution is also not a distribution,
but a probability, that is defined on $R^n$ for arbitrary $n \in
\omega$ that we get out of several different random variables (and all
of the things that go with a random variable).

It's also important to note that if we take two discrete variables and
make a joint distribution out of them, we get a discrete joint
distribution.

In a fashion similar to a standart distribution we can define discrete
and continous joint distributions. Discrete joint distribution is a
joint distribution that is nonzero on a countable subset of $R^2$ and
continous joint distribution is the one whose probability can be
described by an integral. 

By taking half-opened (where we include the higher bound of the
interval, i.e. the oppositve of the $R_l$) intervals and making a
cartesian product of them we can also define a joint cumulative
distribution (or joint distribution) function.
There are similar things goint on with joint distributions and
joint cumulative distribution functions as with the standart ones:
given a joint distribution we can always have its joint c.d.f.

\begin{tabular}[center]{||c | c|| }
  \hline
  1 & 1/2, 1/4 \\
  2 & 0.27, ... \\
  3 & 1/40, 1/20, 0.175(7/40), 7/10 \\
  4 & 3/2, 3/8, 1/8, 1/2, 0 \\
  5 & 5/4, 49/256, 13/16, 0\\
  \hline 
\end{tabular}

Exercises are a bunch of integrals/sums and other borderline trivial stuff. This is practically
an exercise in Maxima. Skip

\section{Marginal Distributions}

\subsection*{Notes}

Given a joint distribution (or probably multivariable too) we can
get a joint c.d.f. Given a joint c.d.f, we can use it on subsets, that
consist of cartesian product of a half-closed interval, that are bounded by some $k \in R$
and the whole $R$. Out of those things we can make a real functions:
$F_x: R \to R$ and $F_y: R \to R$ such that
$$F_x(x) = F(x, \infty)$$
$$F_y(y) = F(\infty, y)$$
those things are called marginal c.d.f.'s, and out of them we can create either probability
density functions, probability functions and all sorts of other stuff.

Given a joint probability function (i.e. the one for the discrete
joint distribution) we can sum it over a variable to get a marginal
p.f. of a specific variable. By using integrals over the whole real
line instead of sums we get p.d.f's out of joint p.d.f.'s.

Remember independent events? Me neither. To refresh the memory: if $A$
and $B$ are events (i.e. subsets of the sample space), then they are
called independent (given a probability, of course) if
$$Pr(A \cap B) = Pr(A) Pr(B)$$
At this point the book goes completely off the rails and starts
abusing notation like there's no tomorrow. Instead of doing that, I'm
just gonna note that joint c.d.f. is called independent in its variables if and only if 
$$F(x, y) = F_x(x) F_y(y)$$
where in the rhs we've got marginal c.d.f.'s. We then follow that if
you think about it a little bit, then you can see how the original
notion of independence is connected to the presented one. For instance,
given a probability $Pr$ over $R^2$ and two subsets $A, B$ of $R$ we've got that 
$$Pr(A \times B) = Pr(A \times R \cap R \times B)$$
and hence
$$Pr(A \times B) = Pr(A \times R) Pr(R \times B)$$
if and only if $A \times R$ and $R \times B$ are independent.

Out of definition of independence in variables of c.d.f. we can define
various independence of p.f's, p.d.f's and so on and so
forth. Thankfully, it's usually clear from context what exactly we're
trying to do in each particular case. The only thing of note that is
that our original idea in the definition of independence of a joint c.d.f
translates neatly into various p.d.f's and p.f.'s and whatnot:
given independent joint thing, we can state that 
$$f(x, y) = f_x(x) f_y(y)$$
where rhs are marginal things for $x, y \in R$.

Sometimes, when we look at some formula for some joint p.f/p.d.f. or
whatever, we see how the formula might break down into product of two
formulas in one variable. Although it might look like the initial
joint thing is independent, it's not necessarily true. Theorem 3.5.6
states that it's true for continous joint p.d.f.'s if and only if the
support (i.e. region in which the function is positive) for the
initial p.d.f. is a cartesian of two intervals. It's probably also
somehow true for p.f.'s and whatnot, but let's not digress.


\begin{tabular}[center]{||c | c|| }
  \hline
  1 & f(y) = (d - c)k, f(y) = (b - a) k \\
  2 & $f(y) = \frac{y + 1}{30}, f(x) = \frac{2x + 3}{15}$, dependent \\
  3 & $f(y) = 3 y^2, f(x) = 1/2$, yes, yes \\
  \hline 
\end{tabular}

this section is also an exercise in calculus.

\section{Conditional Distributions}

\subsection*{Notes}

In this book we concern ourselves with conditional probabilities
exclusively with joint p.f.'s, p.d.f's, or joint p.f./p.d.f's.

From the standpoint of sets and whatnot we've got that
$$Pr(A | B) = \frac{Pr(A \cap B)}{Pr(B)}$$
where the lhs is just a syntactic sugar for the rhs.

In the book, if we've got joint p.f (or p.d.f, or p.f./p.d.f), then we
define conditional probability as the family of functions
$g(.|y): R \to R$ as
$$g(x|y) = \frac{f(x, y)}{f_y(y)}$$
for $y \in R$ such that $f_y(y) > 0$ and 
where the funky notation in the lhs is nothing, but a funky notation,
and $f_y$ in the rhs is the marginal p.f./p.d.f. Quick note: it works for
both $x$'s, and $y$'s, (i.e. first and second positions in the vector).

In the book we also define p.d.f's for cases $y \in R$ such that
$f_y(y) = 0$ as $g(x|y)$ being some sort of a p.d.f. (no matter which
one).  The only implication that we get out (as far as I undesrtand
it) of it is that $g(x|y)$ is a p.d.f. for all $y \in R$. I don't like
that definition, and hence want to define it to be $g(x|y) = 0$ for
all $y \in R$ such that $f_y(y) = 0$. We lose that implication, yes,
but we can reformat it a bit and state that $g(x|y)$ is a p.d.f. given
that $f_y(y) > 0$. If we define the function this way, then out of
somewhat sane p.d.f we don't get a whacky family of functions, but a
pretty well-defined function. We can also do practically the same thing
for a p.f., and out of all of that we can define $g$ to be $R^2 \to R$. 

If we've got a conditional function $g(.|y)$ and $f_y$, then we
can reconstitute original joint thing by using some algebra on the original
definition:
$$f(x, y) = g(x|y) f_y(y)$$
We also include here marginal cases with $f_y(y) = 0$, and both the book's
definition and my definition work out pretty much fine.

If the joint thing is independent in its variables, then we follow that
$$g(x|y) = f_x(x)$$
which we follow from the corresponding definitions and whatnot.

As the consequence of all things said, we can follow that
$$f(x, y) = g(x|y) f_y(y)$$
and
$$f(x, y) = g(y|x) f_x(x)$$
and hence
$$g(x|y) f_y(y) = g(y|x) f_x(x)$$
which gives us the Bayes rule or something like that. Although
everything here desperatly lacks rigor, if you think about it a little
bit, it's not hard to figure out all the missing pieces.

\subsection{}

$$f(x|y) = \frac{3y^2}{2{x^2 - 1}^{3/2}}$$
for appropriate values of $x$ and $y$.

The rest is calculus and counting

\section{Multivariate Distributions}

\subsection*{Notes}

Multivariate distributions are originally defined from making a c.d.f
from the random functions (and spaces, and probabilities, etc.) in an
obvious way:
$$F(x_1, ..., x_n) = Pr((-\infty, x_1] \times ... \times (-\infty, x_n))$$
We also start denoting this thing as joint as well, but we say that
it's joint in $n$ variables, which does not help. Important to note
that joint might denote either bivariate or multivariate (bivariate by
that logic is a part of multivariate). If we want clarity, we usually
use terms "univariate" for 1 variable, "bivariate" for 2 and
"multivariate" for 2 or more. Shoulda seen it coming and use the
term "bivariate" exclusively way earlier.


If every given initial distribution has a p.d.f, then in an obvious
way we define multivariate p.d.f, and so on. Same thing applies to
discrete distributions. Thus taking a whaky sum or a whacky integral
out of whose p.f./p.d.f/whatnots gives us a probability of an event.
Just as with the joint p.f/p.d.f we can take sums and integrals
together whenever it's appropriate.

If we've got p.f.'s and p.d.f's mixed together into one happy function,
then we call it that distribution mixed.

You don't have to be a genious in order to get a marginal c.d.f from
the original c.d.f: just restrict one of the variables, set the rest
to infinity, and you're golden. If you've got a p.d.f or some other
such thing, integrate/sum over the rest of the variables over $R$.
The only thing of note is that we can fix several variables and have a
whacky marginal c.d.f over several variables. 

Fun starts whenever we get to independence. Firstly, if
$$F(x_1, ..., x_n) = F_1(x_1)...F_n(x_n)$$
then we've got independence in all of the variables (as expected).
Same thing applies to p.f.'s and p.d.f's.
$$f(x_1, ..., x_n) = f_1(x_1)...f_n(x_n)$$
if some of the  $f_j$'s are equal, then
(according to the book) the random variables, that correspond to those
things are called independent and identically distributed (or i.i.d.).
I don't like this definition on the account of the fact that it breaks
the original definition of the random variable, and I don't see the
reason not to call the marginal c.d.f.'s with the same name.
For obvious reasons, the multivariate distribution that we get out of
initial $F$ cannot be mixed. Length of the equal marginal things in
that case is called sample size. According to the book, "random
sample" is the collection of those random variables, whatever that
means.

For conditional stuff we once again constraint ourselves with p.f's,
p.d.f's or something mixed (no funny stuff). We also start using the vector
notation extensively here: we denote
$$\bold x  := x_1, ..., x_n$$
there's actually no rigorous definitions for this stuff, we just kinda
start abusing notation a great bit.
There are subvectors and whatnot, for example we can split
$$\bold x  := x_1, ..., x_n$$
into
$$\bold z  := x_1, ..., x_j$$
$$\bold y  := x_{j + 1}, ..., x_n$$
we can also mix'em together a bit, and have
$$\bold q := x_1, x_3, x_j, x_n$$
and so forth. When we deconstruct and reconstitute vector back
together, we tend to use commans (e.g. $\bold x = \bold y, \bold
z$). It's usually clear from the context what exactly what we mean.

Conditional stuff is defined in an intuitive way: given that $\bold x$
destructs into two subvectors $\bold y$ and $\bold z$ we have
$$g_1(\bold y | \bold z) = \frac{f(\bold y, \bold z)}{f_2(\bold z)}$$
where in the rhs $f_2$ denotes the marginal p.f/p.d.f over variables
of $\bold z$ and provided that $f_2(\bold z)$ is positive. We can use
the same thing as in the bivariate distribution and in both p.f. and
p.d.f. cases define $g_1(. | \bold z)$ to be zero in zero cases for
$f_2(\bold z)$. We also got
$$f(y, z) = g_1(y| z) f_2(z)$$
just as with the bivariate stuff, and we also get the Bayes theorem
out of this
$$g_2(z| y) f_1(y)= g_1(y| z) f_2(z)$$

We also can define something interesting: we say that variables are independent
given some other variables if
$$g(\bold x | \bold z) = \prod_{x \in \bold x}{g_x(x|\bold z)}$$
definition in the book kinda implies that everything is consecutive in
there, which it might not be.

Then the book goes on a tanget about histograms for some reason.



\begin{tabular}[center]{||c | c|| }
  \hline
  1 & $1/3$,  $x_3 + 1/3 x_1 + 1/3$, $5/13 \approx 0.38461538461538464$\\
  2 & $6$, $f(0, 0) = f(1, 1) = 3/10; f(0, 1) = f(1, 1) = 1/5$, $f(x_1) = 20 x_1^3 (1 - x_1)$\\
  \hline 
\end{tabular}


\section{Functions of a Random Variable}

\subsection*{Notes}

If we've got a random variable (and everything that usually
accompanies a random variable), we can create another random variable
through a composition with another real function.

Generally speaking, whenever we have a fandom variable $X$, there's a
c.d.f $F$ of that variable. $F$ is defined as
$$F(y) = Pr(X \leq y)$$
this can be restated as
$$F(y) = Pr(x \in (-\infty, y])$$
if $g: R \to R$ is some function of a random variable, then its
c.d.f $G$ is
$$G(y) = Pr(g(x) \in (-\infty, y])$$
since the $Pr$ is the same. We then can use the identity
$$g(x) \in A \iff x \in g\inv[A]$$
to conclude that
$$G(y) = Pr(x \in g\inv[(-\infty, y]])$$
By computing the $g\inv[(-\infty, y]]$ and corresponding probability,
we can compute the proper c.d.f. of a function of a random variable,
given the variable and the corresponding function.

For example, if $g(x) = x^2$, then
$$G(y) = Pr(g(x) \in (-\infty, y])$$
$$G(y) = Pr(x \in [-\sqrt{y}, \sqrt{y}])$$

If $h$ is a linear function (i.e. $h(x) = ax + b$), then
$$G(y) = Pr(ax + b \leq y)$$
$$G(y) = Pr(ax + b \in (-\infty, y])$$
$$G(y) = Pr(ax \in (-\infty, y - b])$$
if $a > 0$, then 
$$G(y) = Pr(x \in (-\infty, \frac{y - b}{a}])$$
$$G(y) = Pr(x \leq \frac{y - b}{a}) = F(\frac{y - b}{a})$$
and if $a < 0$, then
$$G(y) = Pr(x \in [\frac{y - b}{a}, \infty)) =
1 - Pr(x \in (\infty, \frac{y - b}{a})) = 1 - F(\frac{y - b}{a})$$
and hence if the initial function $f$ is continous with c.d.f $F$,
then p.d.f. $g$ of $G$ is
$$g(y) = \frac{d}{dy}G(y)$$
if $a > 0$, then
$$g(y)= \frac{d}{dy}F(\frac{y - b}{a}) = f(\frac{y - b}{a}) \frac{1}{a}$$
where the last part was derived from the chain rule of the differentiation. 
If $a < 0$, then
$$g(y)= \frac{d}{dy}(1 - F(\frac{y - b}{a})) = -f(\frac{y - b}{a}) \frac{1}{a}$$
by using absolute value of $a$ we can combine previous 2 cases into
one case for $a \neq 0$
$$g(y) = f(\frac{y - b}{a}) \frac{1}{|a|}$$

Probability Integral Theorem states that if the function of the random
variable is the c.d.f. of that function, then the resulting function
has unifom distribution on $[0, 1]$. Its main use is the implication
that if we've got a random variable $X$ with the uniform distribution
over $[0, 1]$, and a random variable $Y$, whose quantile function is
$G\inv$, then  $G\inv(X)$ is the function with the same distribution as
$Y$. This gives us an idea on how to implement a halfway decent random
variable, when we're given a random variable with uniform distribution
over $[0, 1]$ in a given programming environment.

\subsection{}

\textit{Suppose that a p.d.f. of a random variable $X$ is as follows}
$$
  f(x) = 
  \begin{cases}
    3x^2 \text{ for } 0 < x < 1 \\
    0 \text{ otherwise}
\end{cases}
$$
\textit{Also, suppose that $Y = 1 - X^2$. Determine the p.d.f of $Y$.}

We follow that
$$G(y) = Pr(g(x) \leq  y) = Pr(1 - x^2 \leq y) = Pr(x^2 \geq 1 - y) = Pr(x \geq \sqrt{1 - y})$$
where the latest implication does not include the $Pr(x \leq -\sqrt{1
- y})$ on the account of the fact that it's zero due to the
restrictions on the original function.
We then follow that
$$G(y) = Pr(x \geq \sqrt{1 - y}) = \int_{\sqrt{1 - y}}^{1}{3x^2} = x^3]^{1}_{\sqrt{1 - y}} = 1 - (1 - y)^{3/2}$$
and hence
$$g(y) = d/dx 1 - (1 - y)^{3/2} = -\sqrt{1 - y}3/2$$
as desired

\subsection{}

\textit{Suppose that a random variable $X$ can have each of the seven
  values $\set{-3, ..., 3}$ with equal probability. Determine the p.f.
  of $Y = X^2 - X$.}

We've got
$$f(x) =
\begin{cases}
  x = 12 \to 1/7 \\
  x = 6 \to 2/7 \\
  x = 2 \to 2/7 \\
  x = 0 \to 2/7 \\
\end{cases}
$$

\subsection{}

\textit{Suppose that the p.d.f. of a random variable $X$ is as follows:}
$$f(x) =
\begin{cases}
  \frac{1}{2} x \text{ for } 0 < x < 2 \\
  0 \text{ otherwise}
\end{cases}
$$
\textit{Also suppose thta $Y = X(2 - X)$. Determine the c.d.f and the p.d.f. of
  $Y$}

We've got that (i've skipped some calculus and algebra in the deriviation,
but the idea is pretty straightforward):
$$G(y) = Pr(Y \leq y) = Pr(X \leq 1 - \sqrt{1 - y}) + Pr(X \geq 1 + \sqrt{1 - y}) =
1 - \sqrt{1 - y}$$
differentiating over the thing we get
$$g(y) = \frac{1}{2\sqrt{1 - y}}$$
for $y \in [0, 1]$

\subsection{}

\textit{Suppose that the p.d.f. of $X$ is as given in Exercise 3. Determine the p.d.f.
  of $Y = 4 - X^3$}

We follow that
$$G(y) = Pr(Y \leq y) = Pr(4 - X^3 \leq y) = Pr(X \geq (4 - y)^{1/3}) = 1 - Pr(X < (4 - y)^{1/3}) $$
thus for $y \in (-4, 4)$
$$G(y) = 1 - \int_{-\infty}^{(4 - y)^{1/3}}{f(x) dx} =
1 - \int_{0}^{(4 - y)^{1/3}}{\frac{1}{2} x dx} =
1 - [x^2/4]_{0}^{(4 - y)^{1/3}} = 1 - \frac{(4 - y)^{2/3}}{4}
$$
and appropriate values otherwise. For p.d.f. we've got
$$g(y) = \frac{1}{6(4 - y)^{1/3}}$$
for $y \in (-4, 4)$ and $0$ otherwise

\subsection{}

\textit{Taken care of in the notes to the section}

\textit{The rest of the section is left for the better days}

\section{Functions  of Two or More Random Variables}

\subsection*{Notes}

There aren't any fundamentally new ideas in this chapter: all we do is
combinating ideas of functions of random variables and joint
distributions. All that we do is pretty much trying to find the
inverses of the required functions (sometimes in a some whacky way if
we talk about specific functions), and carrrying out the obvious
steps afterwards.

Whenever we talk about functions of several variables, we usually
differentiate between scalar functions (i.e. $R^n \to R$) and vector
functions (i.e. $R^n \to R^m$). This chapter only discusses scalar
functions. Not only that, here we start to discuss exclusively continous
and discrete functions. Nothing mixed and/or arbitrary is discussed at all.

One thing of note is that it appears that there's a lot more depth to
some of the concepts, that were defined in the chapter. It seems that
convolution, for example, is a somewhat notable concept in functional
analysis, so the planned MIRA jorney will hopefully shed light on some
of the concepts, that I'm currently not fully grasping in regards to
this chapter

\subsection{}

\textit{Suppose that $X_1$ and $X_2$ are i.i.d. and each of them has
  the uniform distribution on the interval $[0, 1]$. Fint the p.d.f. of $Y = X_1 + X_2$}

We follow that
$$f(x) =
\begin{cases}
  x \in [0, 1] \to 1 \\
  \text{0 otherwise}
  
\end{cases}
$$
is a p.d.f for $X_1$ and $X_2$. Although there's a convolution theorem, it's
somewhat easier to wrap your head around an idea of brute-forcing this thing.

We can dance for a long time around the more general concepts of supports
and their respective intersections with the desired sets, but it'll
be more straightforward to see the desired function as the area of intersection
of the triangle with a square at the origin. After a brief interaction
with a graph calculator one might obtain a function
$$g(y) =
\begin{cases}
  y \in [0, 1] \to y \\
  y \in (1, 0] \to 2 - y \\
\end{cases}
$$

\subsection{}

\textit{Fot the conditions of Exercise 1, find the p.d.f. of the average $(X_1 + X_2) / 2$}

Although there's a arguable reason to use the linear case, we can also
see, that the desired function is nothing more than a function of the
function, that we've already derived.

By using methods from the previous chapter, we can follow that c.d.f. $H$
of the newfound function is 
$$H(y) = Pr(Y/2 < y) = Pr(Y < 2y) = \int_0^{2y}{g(x)} =
\begin{cases}
  2y \in [0, 1] \to 2y^2 \\
  2y \in [1, 2] \to 1/2 +  4y - 2y^2
\end{cases} = 
$$
$$
= \begin{cases}
  y \in [0, 1/2] \to 2y^2 \\
  y \in [1/2, 1] \to 1/2 + 4y - 2y^2
\end{cases} = 
$$
and the appropriate p.d.f. is
$$h(y) =
\begin{cases}
  y \in [0, 1/2] \to 4y \\
  y \in [1/2, 1] \to 4 - 4y \\
\end{cases}
$$
and the appropriate values otherwise.

\subsection{}

\textit{}

\section{Markov Chains}

\subsection{}

\textit{Consider Markov chain in Example 3.10.2 with initial
probability vector $v = (1/2, 1/2)$}

\textit{a) Find the probability vector specifying the probabilities of
  the states at the time $n = 2$}

It doesnt seem to change and will be $(1/2, 1/2)$

\textit{(b) Find the two-step transition matrix}

$$
\begin{pmatrix}
  5/9 & 4/9 \\
  4/9 & 5/9
\end{pmatrix}
$$

\end{document}