\documentclass[11pt,oneside,titlepage]{book}
\title{My probability and statistics exercises}
\usepackage{amsmath, amssymb}
\usepackage{geometry}
\usepackage{hyperref}
\author{Evgeny Markin}
\date{2023}

\DeclareMathOperator \map {\mathcal {L}}
\DeclareMathOperator \ns {null}
\DeclareMathOperator \range {range}
\DeclareMathOperator \inv {^{-1}}
\DeclareMathOperator \Span {span}
\DeclareMathOperator \imp {\Rightarrow}
\DeclareMathOperator \lra {\Leftrightarrow}
\newcommand{\eangle}[1]{\langle #1 \rangle}


\begin{document}
\maketitle
\tableofcontents

\chapter{Introduction to Probability}

\section{The History of Probability}

\section{Interpretations of Probability}

\section{Experiments and Events}

\section{Set Theory }

\textit{Exercises in this section (or exercises similar to them) are handled in the set theory
  course}

\section{The Definition of Probability}

\begin{tabular}[center]{||c | c|| }
  \hline
  1 & 2/5 \\
  2 & 0.7 \\
  3a & 1/2 \\
  3b & 1/6 \\
  3c & 3/8 \\
  4 & 0.6 \\
  5 & 0.4 \\
  6 & 0.5 \\
  8 & 30 \\
  11a & 1 - $\pi/4$ \\
  11b & 0.75 \\
  11c & 2/3\\
  11d & 0 \\
  14a & 0.38, 0.16 \\
  14b & 0.04 \\
  \hline 
\end{tabular}


A little notation, related to 6: 
$$Pr(A) = 0.5$$
$$Pr(B) = 0.2$$
$$Pr(A \cap B) = 0.1$$
$$Pr(A \cup B) = 0.6$$
$$Pr((A \cup B) \cap (A \cap B)^c) = P(A \cup B) - P((A \cup B) \cap (A \cap B)) =
P(A \cup B) - P(A \cap B) = 0.5$$

\subsection*{1.5.7}

If $Pr(A) = 0.4$ and $Pr(B) = 0.7$, then we follow that the maxium $Pr(A \cap B)$ is attained
if $A \subset B$, in which case $Pr(A \cap B) = Pr(A) = 0.4$. The minimum is obtained
if $A \cup B = S$, in which case $Pr(A \cap B) = 0.1$

\subsection*{1.5.9}

The event that exaclty one of the events occurs can be expressed as
$$(A \cap B^c) \cup (A^c \cap B)$$
which comes from either the definition of xor, common sense or something else, depending on
your preferences. Thus we follow that
$$Pr((A \cap B^c) \cup (A^c \cap B)) = Pr(A \cap B^c) + Pr(A^c \cap B) -
Pr((A \cap B^c) \cap (A^c \cap B)) = $$
$$ = 
Pr(A \cap B^c) + Pr(A^c \cap B) - Pr((A \cap A^c) \cap (B^c \cap B)) = $$
$$ =   Pr(A \cap B^c) + Pr(A^c \cap B) = Pr(A) - Pr(A \cap B) + Pr(B) - Pr(B \cap A) = $$
$$= Pr(A) - Pr(A \cap B) + Pr(B) - Pr(A \cap B) =  Pr(A) + Pr(B) - 2Pr(A \cap B)$$
as desired (rules used in this derivitation: association of unions, $A \cap A^c = \emptyset$
and other trivial stuff)

\subsection*{1.5.10}

$$Pr(A \cap B^c) = Pr(A) - Pr(A \cap B)$$
$$Pr(A \cap B^c) + Pr(A \cap B) = Pr(A) $$
as desired.

\subsection*{1.5.12}

Suppose that $n > m \in N$. Then we follow that by definition
$$B_m \subseteq A_m$$
and
$$B_n \subseteq A_m^c$$
thus we follow that
$$B_m \cap B_n \subseteq A_m \cap A_m^c = \emptyset$$
thus
$$B_m \cap B_n = \emptyset$$
therefore we conclude that $B_1, B_2 ...$ are disjoint sets. Thus we follow that
$$Pr(\bigcup_{i = 1}^n B_i) = \sum_{i = 1}^n {Pr(B_i)}$$
For $n = 2$ we've got that
$$B_1 \cup B_2 = A_1 \cup (A_1^c \cap A_2) = (A_1 \cup A_1^c) \cap (A_1 \cup A_2) = A_1 \cup A_2$$
and by induction we can follow that
$$\bigcup_{i = 1}^n {B_i} = \bigcup_{i = 1}^n {A_i}$$
thus
$$Pr(\bigcup_{i = 1}^n B_i) = \sum_{i = 1}^n {Pr(B_i)}$$
implies that
$$Pr(\bigcup_{i = 1}^n A_i) = \sum_{i = 1}^n {Pr(B_i)}$$
for $n \in N$. Given that $n$ is arbirtary, we can follow that
$$Pr(\bigcup_{i = 1}^\infty A_i) = \sum_{i = 1}^\infty {Pr(B_i)}$$
as desired.

\subsection*{1.5.13}

First equation follow from induction on the result that
$$Pr(A \cup B) \leq Pr(A) + Pr(B)$$
the second equation follows from the first equation, DeMorgan laws and induction on the
form 
$$Pr(A \cap B) = Pr((A^c \cup B^c)^c) = 1 - Pr(A^c \cup B^c) \geq 1 - (Pr(A^c) + Pr(B^c))$$

\subsection*{1.5.14}

$$Pr(A) = 0.34$$
$$Pr(B) = 0.12$$
$$Pr(O) = 0.5$$
$$Pr(AB) = 1 - 0.34 - 0.12 - 0.5 = 0.04$$
$$Pr(a-A) = 0.34 + 0.04 = 0.38$$
$$Pr(a-B) = 0.12 + 0.04 = 0.16$$

\section{Finite Sample Spaces}

\begin{tabular}[center]{||c | c|| }
  \hline
  1 & 1/2 \\
  2 & 1/2 \\
  3 & 2/3 \\
  4 & 1/7 \\
  5 & 4/7 \\
  6 & 1/4 \\
  8b & 1/4 \\
  \hline 
\end{tabular}

\subsection*{1.6.7}

The possinble genotypes are $Aa$ and $aa$ with probabilities $1/2$ and $1/2$ respectively

\subsection*{1.6.8a}

The sample space of the experiment is $\{heads, tails\} \times \{1, 2, 3, 4, 5, 6\}$,

\section{Counting Methods}

\begin{tabular}[center]{||c | c|| }
  \hline
  1 & 14 \\
  2 & 9000 \\
  3 & 120 \\
  4 & 24 \\
  5 & 5/18 \\
  6 & 5/324 \\
  7 & ~0.014731 \\
  8 & 360 / 2401 \\
  9 & 1 / 20\\
  10a & r/100 \\
  10b & r/100 \\
  10c & r/100 \\
  \hline 
\end{tabular}

\subsection*{1.7.11}

$$s(n) = \frac{1}{2} \log(2 \pi) + (n + \frac{1}{2})\log{n} - n \approx \log{n!}$$
$$\log{n!} - \log{(n - m)!} = \log{\frac{n!}{(n - m)!}}$$
$$s(n) - s(n - m) = \frac{1}{2} \log(2 \pi) + (n + \frac{1}{2})\log{n} - n
- (\frac{1}{2} \log(2 \pi) + ((n - m) + \frac{1}{2})\log{n - m} - (n - m)) = $$
$$ = (n + \frac{1}{2})\log{n} - n
-   ((n - m) + \frac{1}{2})\log{(n - m)} + (n - m) = $$
$$ = (n + \frac{1}{2})\log{n}
-   ((n - m) + \frac{1}{2})\log{(n - m)}  - m \approx \log{\frac{n!}{(n - m)!}}$$
$P(n, m) = \frac{n!}{(n - m)!} = \exp(s(n) - s(n - m))$


\section{Combinatorial Methods}

\begin{tabular}[center]{||c | c|| }
  \hline
  1 & 184756 \\
  2 & latter \\
  3 & equal \\
  4 & 1 / 10626 \\
  5 & - \\
  6 & 2/n \\
  7 & (n - k - 1)/C(n, k) \\
  8 & (n - k)/C(n, k) \\
  9 & (n + 1)/C(2n, n)\\
  \hline 
\end{tabular}

\subsection*{1.8.5}

\textit{Prove that
  $$ \frac{\prod_{4155 \leq i \leq 4251}{i}}{\prod_{2 \leq i \leq 97}{i}}$$
  is an integer}

$$\frac{\prod_{4155 \leq i \leq 4251}{i}}{\prod_{2 \leq i \leq 97}{i}} =
\frac{\prod_{4155 \leq i \leq 4251}{i}}{\prod_{1 \leq i \leq 97}{i}} = $$
$$ =  \frac{\prod_{4155 \leq i \leq 4251}{i}}{97!} =
\frac{4251!}{4154!97!} = \frac{4251!}{4154!(4251 - 4174)!} = C(4251, 4154)$$
and binomial coefficients are integers (pretty sure that we can follow that by induction
in some more advanced course).

\subsection*{1.8.14}

\textit{Prove that for all positive integers $n, k$ such that $n \geq k$
  $$C(n, k) + C(n, k - 1) = C(n + 1, k)$$
}

$$ C(n, k) + C(n, k - 1) = \frac{n!}{(n - k)!k!} + \frac{n!}{(n - k + 1)!(k - 1)!} =$$
$$ =
\frac{n!}{k (n - k)!(k - 1)!} + \frac{n!}{(n - k + 1)(n - k)!(k - 1)!} =$$
$$ =
\frac{(n - k + 1)n!}{k(n - k + 1) (n - k)!(k - 1)!} + \frac{kn!}{k(n - k + 1)(n - k)!(k - 1)!} =
$$
$$ =
\frac{(n - k + 1)n! + kn!}{k(n - k + 1) (n - k)!(k - 1)!} =
\frac{n!((n - k + 1) + k)}{k(n - k + 1) (n - k)!(k - 1)!} = 
$$
$$ =
\frac{n!(n + 1)}{k(n - k + 1) (n - k)!(k - 1)!} =
\frac{(n + 1)!}{((n + 1) - k)!k!} = C(n + 1, k)
$$
as desired.





\end{document}
%%% Local Variables:
%%% mode: latex
%%% TeX-master: t
%%% End:
