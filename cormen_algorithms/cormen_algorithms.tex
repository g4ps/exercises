\documentclass[11pt,oneside,titlepage]{book}
\title{My algorithms exercises}
\usepackage{amsmath,amssymb}
\usepackage{geometry}
\usepackage[linesnumbered, ruled]{algorithm2e}
\usepackage{hyperref}
\author{Evgeny Markin}
\date{2023}

\begin{document}
\maketitle
\tableofcontents

\chapter*{Preface}
Exercises for Introduction to Algorithms by Cormen et al., 4th ed. It has
exercises, that should be written down, moslty in math and whatnot.

Some of the exercises require that you code something (sometimes it's not
explicitely required, but that would be nice to code it anyways), and this
code is presented in the progs folder. Everything is written in C, because I'm most
familiar with it. 

Pseudocode is written by using package \textbf{algorithm2e}, which does not really correspond
to the one, that is used in the book, but it still does the job.

Same rules as usual apply, if you want to use this book for any reason -- go right ahead, it's
free, just be aware that it is full of mistakes.

\part{Foundations}

\chapter{The Role of Algorithms in Computing}

\section{Algorithms}

\subsection{}

\textit{Describe your own real-world example that required sorting. Describe one
  that required finding the shortest distance between two points.}

I've needed both when I was creating 8-puzzle program

\subsection{}

\textit{Other than speed, what other measures of efficiency migh you need to consideer
  in a real-world setting?}

Memory and parallellability.

\subsection{}

\textit{Select a data structure that you've seen, and discuss its strengths and limitations.}

Linked lists. They are perfect in everything, apart from sorting; but even then you can define
any data structure through linked lists, which makes them just perfect (especially
omnidirectional ones).

\subsection{}

\textit{Suggest a real-world problem in which only the best solution will do. Then come up
  with one in which "approximately" the best solution is good enough.}

Sorting has to be perfect, otherwise it's borderline useless. Estimated time to
complete the task can tolerate imperfections.

\subsection{}

\textit{Describe a real-world problem in which sometimes the entire input is available
  before you need to solve the problem, but other times the input is not
  entirely available in advance and arrives over time}

Traffic on maps does this thing. Sometimes you have all the input, sometimes it changes.


\section{Algorithms as a technology}

\subsection{}

\textit{Give an example of an application that requires algorithmic content at the
  application level, and discuss the function of the algorithms involved.}

Path finding on maps will do. It requires to triverse graphs and whatnot.

\subsection{}

\textit{Suppose that for inputs of size $n$ on a particular computer, insertion sort
  runs in $8n^2$ steps and merge sort runs in $64n \lg n$ steps. For which values of $n$
  does insertion sort beat merge sort?}

For
$$8 n^2 < 64 n \lg n$$
$$n <8 \lg n$$
$$\frac{n}{lg n} < 8$$
$$n \approx 44$$
cases.


\subsection{}

\textit{What is the smallest value of $n$ such that an algorithm whose running time is $100n^2$
  runs faster than an algorithm whose running time is $2^n$ on the same machine?}

Calculator says 15

\chapter{Getting Started}


\section{Inserition sort}

\subsection{}

\textit{Using Figure 2.2 as a model, illustrate the operation of Insertion-Sort on an
  array initially containing the sequence $[31, 41, 59, 26, 41, 58]$}
$$[31, 41, 59, 26, 41, 58]$$
$$[26, 31, 41, 59, 41, 58]$$
$$[26, 31, 41, 41, 59, 58]$$
$$[26, 31, 41, 41, 58, 59]$$

\subsection{}

\textit{State loop invariant for the Sum-Array procedure.}

\textbf{Initialization: }

Firstly, we've got 0 as the sum. Given that we've summed 0 elements so far, we can conclude
that this is indeed a correct value to set it.

\textbf{Maintenance: }

For each iteration of $i$ we've got that we add a i'th element from the array to our sum and
incrementing $i$. Thus before iterating through $i$ we had a sum of all of the elements
before $i$, and after iterating through it we create a sum of elements before $i$ and
the $i$'th element as well. Thus the sum after iterating through $i$ is correct.

\textbf{Termination: }

Given that the array is finite, we follow that because we are incrementing $i$ at each
iteration the algoritm will terminate. Because we increment through elements, we follow that
we've added every element of the array to the sum at the point of termination.

\subsection{}

\textit{Rewrite the Insertion-Sort procedure to sort into monotonically decreasing instead
  of monotonically increasing order.}

Done it in the progs section; long story short: reverse the ordering function in the inner loop,
replace  $A[j] > key$ with $A[j] < key$.

\subsection{}

\textit{Consider the seqrching problem}

\textit{Input: A sequence of $n$ numbers $[a_1, ... a_n]$ stored in array $A[1:n]$ and a
  value x.}

\textit{Output: An index $i$ such that $x$ equals $A[i]$ or the special value $NIL$ if $x$
  does not appear in $A$.}

\textit{Write pseudocode for linear search, which scans through the array from beginning to
  end, looking for $x$. Using a loop invariant, prove that your algorythm is correct.
  Make sure that your loop invariant fufills the three necessary properties.}

\begin{function}
  \caption{Linear-search (A, x)}
  \For{($i = 1 \to n$)}{
    \If{$A[i] = x$} {
      \Return {i} \;
    }
  }
  \Return{NIL}\;

\end{function}

\textbf{Initialization: }
Null case is when we haven't gone through any of the elements yet;
because we haven't searched anything, we
can follow that we haven't found anything, thus the base case is correct

\textbf{Maintenance: }
Suppose that $i = j + 1$. Then we follow that $A[1:j]$ does not contain our element by our
induction hypothesis (not sure that we can use this kind of language here, but it's
my book and I can do whatever I want). Then $i$'th element is cheched for the necessary
equality and returned in case of the equality; otherwise we increment $i$ and make it so
$A[1: j + 1]$ is the array of processed values. Thus before and after the loop iteration
we have a correct solution.

\textbf{Termination: }
We terminate either after going through every element, or some time before it.

Example of this thing in C is presented in progs directory


\subsection{}

\textit{Consider the problem of adding two $n$-bit binary integers $a$ and $b$, stored in
  two $n$-element arrays $A[0: n - 1]$ and $B[0: n - 1]$, where each element is either 0 or 1,
  $a = \sum_{n = 0}^{n - 1}A[i] * 2^i$ and $b = \sum_{n = 0}^{n - 1}B[i] * 2^i$. The
  sum $c = a + b$ of the two integers should be stored in binary form in an $(n + 1)$-element array
  $C[0:n]$, where   $c = \sum_{n = 0}^{n - 1}C[i] * 2^i$. Write a procedure Add-Binary-Integers
  that takes an input arrays $A$ and $B$, along with the length $n$, and returns array $C$ holding
  the sum.}


\begin{function}
  \caption{Add-Binary-Integers (A, B, n)}
  define C[0: n + 1] and fill it with zeroes\;
  \tcc{carry stores overflow from the previous iteration} 
  $carry \leftarrow 0$\;
  \For{($i = 0 \to n + 1$)}{
    \tcc{Initializing temporary variable with carry bit; we need to sum carry, A[i], and B[i],
    so we can just initialize temp with carry}
  $r \leftarrow carry$ \;
  \tcc{if it is not the last bit, where A nor B are not defined; We can just define it to be zero
  with the same result}
    \If{$i \neq n + 1$} {
      $r \leftarrow A[i] + B[i] + r$ \;
    }
    \tcc{If we've got an overflow as the result, set carry bit and result of summation
      appropriately  }
    \eIf{$r > 1$}{
      $carry \leftarrow 1$ \;
      $r \leftarrow r \% 2$\;
    }{
      \tcc{otherwise zero the carry bit}
      $carry \leftarrow 0$\;
    }
    \tcc{lastly, put the result of the partial summation into the resulting array}
    $C[i] \leftarrow r$\;
  }
  \Return{C}\;

\end{function}


\part{Appendix: Mathematical Background}
\chapter{Summations}

\section*{1-1}
\textit{Prove that $\sum_{k  = 1}^n{O(f_k(i))} = O(\sum_{k  = 1}^n{f_k(i)})$}

Short answer:
$$\sum cg(x) = c\sum{g(x)}$$

Long answer:

Suppose that $g \in O(f_k(i))$. It follows that there exists $n_i$ and $c_i$
such that $0 \leq g(n) \leq cf_i(n)$. Thus we can pick
$n = \max\{n_0, n_1, ...\}$ and $c = \max\{c_0, c_1, ... \}$. We know that
both $n$ and $c$ will work all of functions $f_k$. Therefore by
linearity of summations
$$\sum_{k  = 1}^n{O(f_k(i))}
= \sum_{k  = 1}^n{cf_k(i)} =
= c\sum_{k  = 1}^n{f_k(i)} = 
= O(\sum_{k  = 1}^n{f_k(i)})$$
(notation is a little abused and there is nothing is rigorously
proven, but it'll do).

\section*{1-2}
\textit{Find a simple formula for $\sum_{k = 1}^n{(2k - 1)}$.}

$$\sum_{k = 1}^n{(2k - 1)} =
\sum_{k = 1}^n{(2k)} - \sum_{k = 1}^n{(1)} =
2\sum_{k = 1}^n{(k)} - n =
2\frac{n(n + 1)}{2} - n =
n(n + 1) - n =
n^2
$$


\section*{1-3}
\textit{Interpret the decimal number $111,111,111$ in light of equation A.6}

$$111,111,111 = \sum_{k = 0}^{9}{10^k} = \frac{10^{10} - 1}{10 - 1}$$

\section*{1-4}
\textit{Evaluate the infinite series $1 - \frac 1 2 + \frac 1 4 - \frac 1 8
  + \frac{1}{16} - ...$}

The series converges absolutely to 2, so we are free to do anything with it.

$$1 - \frac 1 2 + \frac 1 4 - \frac 1 8
+ \frac{1}{16} - ... = 
\sum_{k = 0}^{\infty}{\frac{1}{2}^{2k}}
- \sum_{k = 0}^{\infty}{\frac{1}{2}^{1 + 2k}} =
\sum_{k = 0}^{\infty}{\frac{1}{2}^{2k}}
- \frac{1}{2}\sum_{k = 0}^{\infty}{\frac{1}{2}^{2k}} =
\left(1 - \frac{1}{2}\right)\sum_{k = 0}^{\infty}{\frac{1}{2}^{2k}} = 
$$
$$
= \left(1 - \frac{1}{2}\right)\sum_{k = 0}^{\infty}{\frac{1}{4}^{k}}
= \left(1 - \frac{1}{2}\right)\frac{1}{1 - \frac{1}{4}}
= \frac 1 2 *  \frac 4 3 = \frac 2 3
$$

\section*{1-5}
\textit{Let $c \geq 0$ be a constant. Show that
  $\sum_{k = 1}^{n}{k^c} = \Theta(n^{c + 1})$}

$$
\sum_{k = 1}^{n}{k^c} = \sum_{k = 1}^{n - 1}{k^c} + n^c =
n^c\sum_{k = 1}^{n}{\frac{k^c}{n^c}} =
$$

Let $f(n) = n^c$.
It can be seen from the graph that
$$\int_{0}^{n}{f(x)dx} \leq \sum_{k = 1}^{n}{k^c} \leq
\int_{0}^{n}{f(x + 1)dx}$$

Thus 
$$\int_0^n{f(x)dx} = \int_0^n{x^c} = \frac{n^{c + 1}}{c + 1} \in$$
$$\int_0^n{f(x + 1)dx} = \int_0^n{(x + 1)^c} = \frac{(n + 1)^{c + 1}}{c + 1}$$

Thus we can state that $\sum_{k = 1}^{n}{k^c} = \Theta(n^{c + 1})$
(I'm not good enough yet to show that $\frac{(n + 1)^{c + 1}}{c + 1} \in
\Theta(n^{c + 1})$, but I'm pretty sure that it's true TODO).



\section*{1-6}
\textit{Show that $\sum_{k=0}^{\infty}{k^2 x^k} = x(1 + x)/(1 - x)^3$ for
  $|x| < 1$}

We know that for $|x| < 1$
$$\sum_{k = 0}^{\infty}{kx^k} = \frac{x}{(1 - x)^2}$$
thus if we differentiate both sides we get
$$\sum_{k = 0}^{\infty}{k^2x^{k - 1}} = \frac{2x}{(1 - x)^3}
+ \frac{1}{(1 - x)^2} $$
and then if we multiply all of it by $x$ we'll get
$$\sum_{k = 0}^{\infty}{k^2x^k} = \frac{2x^2}{(1 - x)^3}
+ \frac{x}{(1 - x)^2} $$
thus if we factor all of this jazz we'll get
$$\sum_{k = 0}^{\infty}{k^2x^k} = - \frac{x(x + 1)}{(x - 1)^3}$$
and if we tuck this minus into denominator we'll get (which we can do because
the power is odd)
$$\sum_{k = 0}^{\infty}{k^2x^k} = \frac{x(x + 1)}{(1 - x)^3}$$
as desired.

\section*{1-7}
\textit{Prove that $\sum_{k = 1}^n{\sqrt{k \lg k}} = \Theta(n^{3/2}\lg^{1/2}n)$}

$$\int \sqrt{k \lg k} = $$

\section*{1-9}
\textit{Show that}
$$\sum_{k = 0}^{\infty}{(k - 1)/2^k} = 0$$

$$\sum_{k = 0}^{\infty}{(k - 1)/2^k} = \sum_{k = 0}^{\infty}{k/2^k} - \sum_{k = 0}^{\infty}{1/2^k} =
\sum_{k = 0}^{\infty}{k/2^k} - 2 = 0$$

$$\sum_{k = 0}^{\infty}{k/2^k} - 2 = 0$$
$$\sum_{k = 0}^{\infty}{k/2^k} = 2$$
$$\sum_{k = 0}^{\infty}{k/2^k - 2} = 0$$
$$\sum_{k = 0}^{\infty}{\frac{k - 2^{k + 1}}{2^k}} = 0$$


\chapter{Sets, Etc.}

\section*{1-1}
\textit{Draw Venn diagrams that illustrate the first of the distributive laws
(B.1)}

TODO, add picture here

\section*{1-2}
\textit{Prove the generalization of DeMorgan's laws to any finite collection
  of sets}

\textit{Copy from real analysis exercises}

Suppose that $x \in \left(\cup_{\lambda \in \Lambda} E_\lambda \right)^c$. It
follows, that $x$ is not in the union of given sets. Therefore there is no
set $E_n$ such that $x \in E_n$ (because if there would be such a set, then $x$
wouldn't be in $\left(\cup_{\lambda \in \Lambda} E_\lambda \right)^c$).
Therefore $x \in \cap_{\lambda \in \Lambda} E_\lambda^c$. Therefore 
$$\left(\cup_{\lambda \in \Lambda} E_\lambda \right)^c \subseteq
\cap_{\lambda \in \Lambda} E_\lambda^c$$

The proof of reverse inclusion is the same as with the forward, but in reverse
order.

$x \in \left(\cap_{\lambda \in \Lambda} E_\lambda \right)^c$ implies that
$x$ is not in every $E_n$. Therefore there exists $x \in E_n^c$ for some $E_n$.
therefore it is in $\cup_{\lambda \in \Lambda} E_\lambda^c$. The proof of
reverse inclusion uses the same argument, but in other direction.

\section*{1-3}
TODO

\section*{1-4}
\textit{Show that the set of odd natural numbers is countable.}

Let us set a function $f: A \to N$, where $A$ denotes the set of
odd natural numbers
$$f(n) = (n + 1) / 2$$
for this function  we've got
$$f^{-1}(n) = 2n - 1$$

Both funcitons are injective and therefore $f$ is bijective. Therefore
we've got a bijective function betweeen $A$ and $N$, therefore
$A \sim N$, therefore it is conuntable, as desired.

\section*{1-5}
\textit{Show that for any finite set $S$, the power set $2^S$ has
  $2^{|S|}$ elements (that is, there are $2^{|S|}$ distinct subsets of $S$).}

\textit{Another copy from real analysis}

This proof is dumb, but intuitive:

Every subset is corresponding to a number in binary system: 0 for excluded,
1 for included. Therefore there exist $2^n$ possible combinations.

For a more concrete proof let's resort to induction.

Base case(s): subsets of $\emptyset$ are $\emptyset$ itseft
($2^0 = 1$ in total). Subsets of
set with one element are $\emptyset$ and set itself ($2^1 = 1$ in total).

Proposition is that set with n elements has $2^n$ subsets.

Inductive step is that for set with $n + 1$ elements can either have or hot
have the $n + 1$'th element. Therefore there exist $2^n + 2^n = 2 * 2^n =
2^{n + 1}$ subsets, as desired.

\section*{1-6}
\textit{Give an inductive definition for an $n$-tuple by extending the
  set-theoretic definition for an ordered pair.}

The tuple is actually just a re-writing of particular set
$$(a_1, a_2, ..., a_n) = \{\{a_1\}, \{a_1, a_2\}, \{a_1, a_2, a_3\} ...
\{a_1, a_2, a_3, ..., a_n\}\}$$

\end{document}

%%% Local Variables:
%%% mode: latex
%%% TeX-master: t
%%% End:
