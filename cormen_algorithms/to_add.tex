\documentclass[11pt,oneside,titlepage]{book}
\title{Additions to the algorithm text}
\usepackage{amsmath,amssymb}
\usepackage{geometry}
\usepackage[linesnumbered, ruled]{algorithm2e}
\usepackage{hyperref}
\author{Evgeny Markin}
\date{2023}

\begin{document}

\part{Appendix: Mathematical Background}
\chapter{Summations}

\section*{A.1 Summation formulas and properties}

\subsection*{A.1-1}
\textit{Prove that $\sum_{k  = 1}^n{O(f_k(i))} = O(\sum_{k  = 1}^n{f_k(i)})$}

Short answer:
$$\sum cg(x) = c\sum{g(x)}$$

Long answer:

Suppose that $g \in O(f_k(i))$. It follows that there exists $n_i$ and $c_i$
such that $0 \leq g(n) \leq cf_i(n)$. Thus we can pick
$n = \max\{n_0, n_1, ...\}$ and $c = \max\{c_0, c_1, ... \}$. We know that
both $n$ and $c$ will work all of functions $f_k$. Therefore by
linearity of summations
$$\sum_{k  = 1}^n{O(f_k(i))}
= \sum_{k  = 1}^n{cf_k(i)} =
= c\sum_{k  = 1}^n{f_k(i)} = 
= O(\sum_{k  = 1}^n{f_k(i)})$$
(notation is a little abused and there is nothing is rigorously
proven, but it'll do).

\subsection*{A.1-2}
\textit{Find a simple formula for $\sum_{k = 1}^n{(2k - 1)}$.}

$$\sum_{k = 1}^n{(2k - 1)} =
\sum_{k = 1}^n{(2k)} - \sum_{k = 1}^n{(1)} =
2\sum_{k = 1}^n{(k)} - n =
2\frac{n(n + 1)}{2} - n =
n(n + 1) - n =
n^2
$$


\subsection*{A.1-3}
\textit{Interpret the decimal number $111,111,111$ in light of equation A.6}

$$111,111,111 = \sum_{k = 0}^{9}{10^k} = \frac{10^{10} - 1}{10 - 1}$$

\subsection*{A.1-4}
\textit{Evaluate the infinite series $1 - \frac 1 2 + \frac 1 4 - \frac 1 8
  + \frac{1}{16} - ...$}

The series converges absolutely to 2, so we are free to do anything with it.

$$1 - \frac 1 2 + \frac 1 4 - \frac 1 8
+ \frac{1}{16} - ... = 
\sum_{k = 0}^{\infty}{\frac{1}{2}^{2k}}
- \sum_{k = 0}^{\infty}{\frac{1}{2}^{1 + 2k}} =
\sum_{k = 0}^{\infty}{\frac{1}{2}^{2k}}
- \frac{1}{2}\sum_{k = 0}^{\infty}{\frac{1}{2}^{2k}} =
\left(1 - \frac{1}{2}\right)\sum_{k = 0}^{\infty}{\frac{1}{2}^{2k}} = 
$$
$$
= \left(1 - \frac{1}{2}\right)\sum_{k = 0}^{\infty}{\frac{1}{4}^{k}}
= \left(1 - \frac{1}{2}\right)\frac{1}{1 - \frac{1}{4}}
= \frac 1 2 *  \frac 4 3 = \frac 2 3
$$

\subsection*{A.1-5}
\textit{Let $c \geq 0$ be a constant. Show that
  $\sum_{k = 1}^{n}{k^c} = \Theta(n^{c + 1})$}

$$
\sum_{k = 1}^{n}{k^c} = \sum_{k = 1}^{n - 1}{k^c} + n^c =
n^c\sum_{k = 1}^{n}{\frac{k^c}{n^c}} =
$$

Let $f(n) = n^c$.
It can be seen from the graph that
$$\int_{0}^{n}{f(x)dx} \leq \sum_{k = 1}^{n}{k^c} \leq
\int_{0}^{n}{f(x + 1)dx}$$

Thus 
$$\int_0^n{f(x)dx} = \int_0^n{x^c} = \frac{n^{c + 1}}{c + 1} \in$$
$$\int_0^n{f(x + 1)dx} = \int_0^n{(x + 1)^c} = \frac{(n + 1)^{c + 1}}{c + 1}$$

Thus we can state that $\sum_{k = 1}^{n}{k^c} = \Theta(n^{c + 1})$
(I'm not good enough yet to show that $\frac{(n + 1)^{c + 1}}{c + 1} \in
\Theta(n^{c + 1})$, but I'm pretty sure that it's true TODO).



\subsection*{A.1-6}
\textit{Show that $\sum_{k=0}^{\infty}{k^2 x^k} = x(1 + x)/(1 - x)^3$ for
  $|x| < 1$}

We know that for $|x| < 1$
$$\sum_{k = 0}^{\infty}{kx^k} = \frac{x}{(1 - x)^2}$$
thus if we differentiate both sides we get
$$\sum_{k = 0}^{\infty}{k^2x^{k - 1}} = \frac{2x}{(1 - x)^3}
+ \frac{1}{(1 - x)^2} $$
and then if we multiply all of it by $x$ we'll get
$$\sum_{k = 0}^{\infty}{k^2x^k} = \frac{2x^2}{(1 - x)^3}
+ \frac{x}{(1 - x)^2} $$
thus if we factor all of this jazz we'll get
$$\sum_{k = 0}^{\infty}{k^2x^k} = - \frac{x(x + 1)}{(x - 1)^3}$$
and if we tuck this minus into denominator we'll get (which we can do because
the power is odd)
$$\sum_{k = 0}^{\infty}{k^2x^k} = \frac{x(x + 1)}{(1 - x)^3}$$
as desired.

\subsection*{A.1-7}
\textit{Prove that $\sum_{k = 1}^n{\sqrt{k \lg k}} = \Theta(n^{3/2}\lg^{1/2}n)$}

$$\int \sqrt{k \lg k} = $$
TODO

\subsection*{A.1-8}
\textit{Show that
  $$\sum_{k = 1}^{n}{1/(2k - 1)} = \ln(\sqrt{n}) + O(1)$$
  by manipulating the harmonic series
}

In the book we're reassured that
$$\sum_{k = 1}^n{\frac{1}{k}} = \ln(n) + O(1)$$

We want to find the sum of reciprocals of odd numebers. Since $n \in Z_+$ is
either odd or even, but not both, we follow that
$$\sum_{k = 1}^{n}{1/(2k - 1)} = \sum_{k = 1}^n{\frac{1}{k}} -
\sum_{k = 1}^{n}{\frac{1}{2k}} =
\sum_{k = 1}^n{\frac{1}{k}} - \sum_{k = 1}^{n}{\frac{1}{2} \frac{1}{k}} =
\sum_{k = 1}^n{\frac{1}{k}} - \frac{1}{2}\sum_{k = 1}^{n}{\frac{1}{k}} =
\frac{1}{2}\sum_{k = 1}^{n}{\frac{1}{k}} $$
and since
$$\sum_{k = 1}^n{\frac{1}{k}} = \ln(n) + O(1)$$
we follow that
$$\sum_{k = 1}^{n}{1/(2k - 1)} = \frac{1}{2}\sum_{k = 1}^{n}{\frac{1}{k}}  =
\frac{1}{2}(\ln(n) + O(1)) = \ln(n^{1/2}) + 1/2 O(1) =
\ln(\sqrt{n}) + O(1) $$
as desired (justification that $1/2 O(1) = O(1)$ follows directly from the
definition of $O$).

\subsection*{A.1-9}
\textit{Show that}
$$\sum_{k = 0}^{\infty}{(k - 1)/2^k} = 0$$

We can use standard series manipulations to get
$$\sum_{k = 0}^\infty{(k - 1) / 2^k} = -1 + \sum_{k = 1}^\infty{(k - 1) / 2^k} =
-1 + \sum_{k = 2}^\infty{(k - 1) / 2^k} = -1 + \sum_{k = 1}^\infty{k / 2^{k + 1}} = $$
$$ = -1 + \frac{1}{2} \sum_{k = 1}^\infty{k / 2^{k}}$$

We can also manipulate it differently to get
$$\sum_{k = 0}^\infty{(k - 1) / 2^k} = \sum_{k = 0}^\infty{k/ 2^k - 1/2^k} =
\sum_{k = 0}^\infty{k/ 2^k} -  \sum_{k = 0}^\infty{1/2^k} =
\sum_{k = 0}^\infty{k/ 2^k} - 2 =  \sum_{k = 1}^\infty{k/ 2^k} - 2$$

Now assuming that the original sum converges we get an equation
$$\sum_{k = 1}^\infty{k/ 2^k} - 2 = -1 + \frac{1}{2} \sum_{k = 1}^\infty{k / 2^{k}}$$
$$\sum_{k = 1}^\infty{k/ 2^k} - \frac{1}{2} \sum_{k = 1}^\infty{k / 2^{k}}  = 1 $$
$$\frac{1}{2} \sum_{k = 1}^\infty{k/ 2^k}  = 1 $$
$$\sum_{k = 1}^\infty{k/ 2^k}  = 2$$
and by substituting the result into any of the previous results (I'll take the first) we get that
$$\sum_{k = 0}^\infty{(k - 1) / 2^k} = -1 + \frac{1}{2} \sum_{k = 1}^\infty{k / 2^{k}} =
-1 + 1 = 0$$
as desired.

\subsection*{A.1-11}

\textit{Evaluate the product
  $$\prod_{k = 2}^{n}{1 - \frac{1}{k^2}}$$
}

$$\prod_{k = 2}^{n}{1 - \frac{1}{k^2}} = \prod_{k = 2}^{n}{\frac{k^2 - 1}{k^2}} =
\prod_{k = 2}^{n}{\frac{(k + 1)(k - 1)}{k^2}} =
\frac{\prod_{k = 2}^{n}{(k + 1)}\prod_{k = 2}^{n}{(k - 1)}}{\prod_{k = 2}^{n}{(k^2)}} = $$
$$ =
\frac{\prod_{k = 3}^{n + 1}{k}\prod_{k = 1}^{n - 1}{k}}{(\prod_{k = 2}^{n}{k})^2} =
\frac{\frac{1}{2} * 1 * 2 * \prod_{k = 3}^{n}{k}   * (n + 1) * \frac{1}{n} * n *
  \prod_{k = 1}^{n - 1}{k}}
{(1 * \prod_{k = 2}^{n}{k})^2} =
\frac{\frac{1}{2} * \prod_{k = 1}^{n}{k}   * (n + 1) * \frac{1}{n} * 
  \prod_{k = 1}^{n}{k}}
{(1 * \prod_{k = 2}^{n}{k})^2} =$$
$$ = 
\frac{\frac{1}{2} * (n + 1) * \frac{1}{n} * (\prod_{k = 1}^{n}{k})^2}
{(\prod_{k = 1}^{n}{k})^2} = \frac{1}{2} * (n + 1) * \frac{1}{n} = \frac{1}{2n} + \frac{1}{2}$$
as desired.

\section*{A.2 Bounding summations}



\chapter{Sets, Etc.}

\section*{1-1}
\textit{Draw Venn diagrams that illustrate the first of the distributive laws
(B.1)}

TODO, add picture here

\section*{1-2}
\textit{Prove the generalization of DeMorgan's laws to any finite collection
  of sets}

\textit{Copy from real analysis exercises}

Suppose that $x \in \left(\cup_{\lambda \in \Lambda} E_\lambda \right)^c$. It
follows, that $x$ is not in the union of given sets. Therefore there is no
set $E_n$ such that $x \in E_n$ (because if there would be such a set, then $x$
wouldn't be in $\left(\cup_{\lambda \in \Lambda} E_\lambda \right)^c$).
Therefore $x \in \cap_{\lambda \in \Lambda} E_\lambda^c$. Therefore 
$$\left(\cup_{\lambda \in \Lambda} E_\lambda \right)^c \subseteq
\cap_{\lambda \in \Lambda} E_\lambda^c$$

The proof of reverse inclusion is the same as with the forward, but in reverse
order.

$x \in \left(\cap_{\lambda \in \Lambda} E_\lambda \right)^c$ implies that
$x$ is not in every $E_n$. Therefore there exists $x \in E_n^c$ for some $E_n$.
therefore it is in $\cup_{\lambda \in \Lambda} E_\lambda^c$. The proof of
reverse inclusion uses the same argument, but in other direction.

\section*{1-3}
TODO

\section*{1-4}
\textit{Show that the set of odd natural numbers is countable.}

Let us set a function $f: A \to N$, where $A$ denotes the set of
odd natural numbers
$$f(n) = (n + 1) / 2$$
for this function  we've got
$$f^{-1}(n) = 2n - 1$$

Both funcitons are injective and therefore $f$ is bijective. Therefore
we've got a bijective function betweeen $A$ and $N$, therefore
$A \sim N$, therefore it is conuntable, as desired.

\section*{1-5}
\textit{Show that for any finite set $S$, the power set $2^S$ has
  $2^{|S|}$ elements (that is, there are $2^{|S|}$ distinct subsets of $S$).}

\textit{Another copy from real analysis}

This proof is dumb, but intuitive:

Every subset is corresponding to a number in binary system: 0 for excluded,
1 for included. Therefore there exist $2^n$ possible combinations.

For a more concrete proof let's resort to induction.

Base case(s): subsets of $\emptyset$ are $\emptyset$ itseft
($2^0 = 1$ in total). Subsets of
set with one element are $\emptyset$ and set itself ($2^1 = 1$ in total).

Proposition is that set with n elements has $2^n$ subsets.

Inductive step is that for set with $n + 1$ elements can either have or hot
have the $n + 1$'th element. Therefore there exist $2^n + 2^n = 2 * 2^n =
2^{n + 1}$ subsets, as desired.

\section*{1-6}
\textit{Give an inductive definition for an $n$-tuple by extending the
  set-theoretic definition for an ordered pair.}

The tuple is actually just a re-writing of particular set
$$(a_1, a_2, ..., a_n) = \{\{a_1\}, \{a_1, a_2\}, \{a_1, a_2, a_3\} ...
\{a_1, a_2, a_3, ..., a_n\}\}$$


\end{document}
%%% Local Variables:
%%% mode: latex
%%% TeX-master: t
%%% End:
