\documentclass[11pt,oneside,titlepage]{book}
\title{My advanced calculus exercises}
\usepackage{amsmath, amssymb}
\usepackage{geometry}
\usepackage{hyperref}
\author{Evgeny Markin}
\date{2023}

\DeclareMathOperator \inv {^{-1}}

\begin{document}
\maketitle
\tableofcontents


\chapter{Starting points}

\section{}

\textit{Evaluate }
$$\int_0^\infty{\frac{dx}{1 + x^2}}$$
\textit{and}
$$\int_{-\infty}^1{\frac{dx}{1 + x^2}}$$

$$\int_0^\infty{\frac{dx}{1 + x^2}} = [\tan \inv]_0^\infty = \pi/2 - 0 = \pi/2 $$

$$\int_{-\infty}^1{\frac{dx}{1 + x^2}} = [\tan \inv]_{-\infty}^1 = [\pi/4 + \pi/2] = \frac{3}{4} \pi$$


\section{}

\textit{Determine}
$$\int{\frac {x dx}{1 + x^2}}$$
\textit{Which type of substitution did you use?}

$$\int{\frac {x dx}{1 + x^2}} = \frac 1 2 \int{ \frac {2 x  dx}{1 + x^2}}$$
let $u(x) = 1 + x^2$. Then $u'(x) = 2x$ and therefore
$$\frac 1 2 \int{ \frac {2 x  dx}{1 + x^2}} =
\frac 1 2 \int{ \frac {du}{u}} = \frac{1}{2} \ln(u) = \frac{1}{2} \ln(1 + x^2)$$

I've used push-forward substitution here.

\section{}

\textit{Carry out a change of variables to evaluate the integral and determite the type
of substitution used.}
$$\int_{-R}^{R}{\sqrt{R^2 - x^2}dx}$$



Let's try $x = R \sin(s)$ (the idea is to use the identity $\sin^2(x) + \cos^2(x) = 1$ here
somewhere). It follows that $s = \arcsin(\frac{x}{R})$



$$\int_{-R}^{R}{\sqrt{R^2 - x^2}dx} =
\int_{-R}^{R}{\sqrt{R^2 - R^2 \sin^2(s)} R\cos(x) ds} =
\int_{-R}^{R}{R^2 \sqrt{1 -  \sin^2(s)} \cos(s) ds} =$$
$$ = 
R^2  \int_{-R}^{R}{\cos^2(s) ds} =
\frac{R^2}{2}  \int_{-R}^{R}{1 + \cos(2s)  ds} =
\frac{R^2}{2}\left[s + \frac{1}{2} \sin(2s)\right]_{-R}^R
=
$$

$$
= \frac{R^2}{2}\left[\arcsin(\frac{x}{R}) + \frac{1}{2} \sin(2\arcsin(\frac{x}{R}))\right]_{-R}^R =
$$
$$
=
\frac{R^2}{2} \left[\arcsin(1) + \frac{1}{2} \sin (2 \arcsin(1)) - \arcsin(-1) -
  \frac{1}{2} \sin (2 \arcsin(-1))\right]= 
$$
$$
=
\frac{R^2}{2} \left[\pi/2 + \frac{1}{2} \sin (\pi) + \pi/2 -
  \frac{1}{2} \sin (-\pi)\right]=
\frac{R^2}{2} \pi = \frac{\pi R^2}{2}
$$
Which is goot enough for me. We used the pullback approach here

\section{}

\textit{Determine }
$$\int {\frac{\arctan(x)}{1 + x^2} dx}$$
\textit{and show }
$$\int_0^\infty {\frac{\arctan(x)}{1 + x^2} dx} = \pi^2/8$$

$$\int {\frac{\arctan(x)}{1 + x^2} dx} $$
let $u = \arctan(x)$. Then
$$\int {\frac{\arctan(x)}{1 + x^2} dx} = \int u = u^2/2 = \arctan(x)^2/2$$
thus
$$\int_0^\infty {\frac{\arctan(x)}{1 + x^2} dx} = \pi^2/8$$
as desired.

\section{}

\textit{(a) Detrmine }
$$\int{\frac{1}{w (\ln(w))^p} dw}$$.
\textit{Which type of substitution did you use?}

Let $u = \ln(w)$. It follows that $du = \frac{1}{w} dw$. Thus
$$\int{\frac{1}{w (\ln(w))^p} dw} =
\int{\frac{1}{w } (\ln(w))^{-p} dw} =
\int{(u)^{-p} du} =  \frac{u^{-p - 1}}{-p - 1} = \frac{(\ln(w))^{-p - 1}}{-p - 1}
$$.
Given that $p \neq -1$.

Othersise it is $\ln(\ln(w))$.

\textit{(b) Evaluate }
$$I = \int_2^\infty{\frac{1}{w (\ln(w))^p} dw}$$.
\textit{for which values of $p$ is $I$ finite?}

For $p = 1$ we've got that
$$\lim{\ln(\ln(w))} = \infty$$
thus it diverges

For $p \neq -1$ we've got
$$I = \int_2^\infty{\frac{1}{w (\ln(w))^p} dw} =
\lim_{w \to \infty}{\left[\frac{(\ln(w))^{-p - 1}}{-p - 1}\right]} - \frac{(\ln(2))^{-p - 1}}{-p - 1}
$$
The only thing that bothers us is that whether
$$(\ln(w))^{-p - 1} = (\frac{1}{(\ln(w))})^{p - 1}$$
converges. This happens whenever $p > 1$ (in that case we got that $\ln(w) \to \infty$).
Otherwise it diverges.

\section{}

\textit{State a condition that guarantees a function $x = \phi(s)$ has an inverse. Then use
  your condition to decide whether each of the following functions is invertible. When possible,
  find a formula for the inverse of each function that is invertible.}

Such a condition is bijectivity on a given domain and codomain.

\textit{(a)}
$$x = 1/s$$

Is bijective on a domain $R \setminus \{0\}$. Inverse is $s = 1/x$.

\textit{(b)}
$$x = s + s^3$$

This one is bijective (because it is strictly increasing and unbounded below and above).
$$x = s + s^3$$
$$s^3 + s - x = 0$$
Maxima gives some god-awful result for an inverse function, but it can be obtained by solving the
cubic polinomial.

\textit{(c)}
$$x = \frac{s}{1 + s^2}$$

It is not surjective on $R$, and in addition, it is not injective. Thus it cannot be used
without some heavy restrictions on the domain and codomain.

\textit{(d)}
$$x = \sinh s = \frac{e^s - e^{-s}}{2}$$

Looks solid to me.

$$s = \text{asinh}(s)$$
is a desired inverse function;

\textit{(e)}
$$x = \frac{s}{\sqrt{1 - s^2}}$$

Is bijective on restricted domain. I'm not sure if we can have an analytical inverse of this
function.

\textit{(f)}
$$x = ms + b$$

Is a standart linear function. If $m \neq 0$, then we've got inverse on whole $R$.

\textit{(g)}
$$x = cosh(s)$$

Is bijective on restricted domain. Reverse is $s = \text{acosh}(x)$.

\textit{(h)}
$$x = s - s^3$$

Is bijective on restricted domain. Inverse is terrible.


\textit{(i)}
$$x = tanh(s)$$

Also bijective on restricted domain. Inverse is
$$s = \text{atanh}(x)$$

\textit{(j)}
$$x = \frac{1 - s}{1 + s}$$

BIjective on restricted domain.

\section{}

\textit{(a) Obtain formulas for $f(s) = \cos(\arcsin(s))$ and $g(s) = \tan(\arcsin(s))$
  directly as funtions of $s$ that involve neither trigonometric nor inverse trigonometric
  functions. Your answer will involve the square root function and polynomical expressions
  in $s$.}

$$f(s) = \cos(\arcsin(s)) $$
$$\cos(\arcsin(s)) = \sin(\pi/2 - \arcsin(s)) =
\sin(\pi/2)\cos(\arcsin(s)) - \sin(\arcsin(s)) \cos(\pi/2) = $$
$$ = \sin(\pi/2)\cos(\arcsin(s)) = \cos(\arcsin(s)) $$

$$\cos(\arcsin(s)) = \sqrt{1 - \sin^2(\arcsin(s))} = \sqrt{1 - s^2}$$

$$g(s) = \tan(\arcsin(s))$$
$$\tan(\arcsin(s)) = \frac{\sin(\arcsin(s))}{\cos(\arcsin(s))} = \frac{s}{\sqrt{1 - s^2}}$$


\textit{(b) Compute teh derivative of $\cos(\arcsin(s))$ using the chain rule and the
  derivatives of $\cos u$ and $\arcsin s$. Then compute the derivative of $f(s)$ using
  your expressions in part (a). Compare the two derivatives. Do the same for $\tan(\arcsin(s))$
  and $g(s)$}

$$f'(s) = -\sin(\arcsin(s)) \frac{1}{\sqrt{1 - s^2}} = - \frac{s}{\sqrt{1 - s^2}}$$
$$f'(s) = -2s \frac{1}{2\sqrt{1 - s^2}} = - \frac{s}{\sqrt{1 - s^2}}$$
They are the same.

$$g'(s) = \sec^2(\arcsin(s)) \frac{1}{\sqrt{1 - s^2}} =
\frac{1}{\cos^2(\arcsin(s))} \frac{1}{\sqrt{1 - s^2}} =
\frac{1}{(1 - s^2)\sqrt{1 - s^2}} 
$$

$$g'(s) = \frac{\sqrt{1 - s^2} - s(- \frac{s}{\sqrt{1 - s^2}})}{1 - s^2} =
\frac{\sqrt{1 - s^2} + \frac{s^2}{\sqrt{1 - s^2}}}{1 - s^2} =
\frac{\frac{1 - s^2 + s^2}{\sqrt{1 - s^2}}}{1 - s^2} =
\frac{\frac{1}{\sqrt{1 - s^2}}}{1 - s^2} =
\frac{1}{(1 - s^2) \sqrt{1 - s^2}} 
$$

They are also the same.

\section{}

\textit{Use $x = \arcsin s$ to show $\int {\cos^3 x dx} = \sin(x) - \frac{\sin^3 x}{3}$.}

$$s = \sin(x)$$

$$dx = \frac{1}{\sqrt{1 - x^2}}ds$$

$$\int{\cos^3 (x) dx} = \int{\cos^3 (\arcsin(s)) \frac{1}{\sqrt{1 - s^2}} ds} =
\int{(\sqrt{1 - s^2})^3 \frac{1}{\sqrt{1 - s^2}} ds} =
$$
$$
\int{1 - s^2 ds} = s - \frac{s^3}{3} = \sin(x) - \frac{\sin^3(x)}{3}
$$

as desired.

\section{}

\textit{(a) Write the microscope equation (i.e. the linear approximation) for
  $\phi(s) = \sqrt{s}$ at $s = 100$.}

$$\phi'(s) = \frac{1}{2 \sqrt{s}}$$
$$\phi'(100) = \frac{1}{2 \sqrt{100}} = 1/20 = 0.05$$
Thus
$$\Delta \phi = 0.05 \Delta s$$

\textit{(b) Use the microscope equation from part (a) to estimate $\sqrt{102}$ and $\sqrt{99.4}$}

For the first one we've got that
$$\Delta s = |102 - 100| = 2$$
thus
$$\Delta \phi = 0.05 * 2 = 0.1$$
Therefore
$$\phi(102) \approx \phi(100) + 0.1 = 10 + 0.1 = 10.1$$

For the second one we've got
$$\Delta s = |100 - 99.4| = 0.06$$
thus
$$\Delta \phi = 0.06 * 0.05 = 0.03$$
and since $99.4 < 100$ we follow that
$$\phi(99.4) \approx \phi(100) - \Delta \phi = 10 - 0.03 = 9.97$$

\textit{(c) How far are your estimate from those given by a calculator?}

$$\sqrt{102} \approx 10.0995049$$
$$\sqrt{99.4} \approx 9.969955$$
thus we've got  error of approximately $10^{-3}$ in the first case and $10^{-4}$ in the second.

\textit{(d) Your estimates should be greater than the calculator values; use the graph of
  $x = \phi(s)$ to explain why this is so.}

This is because derivative of this function is a  decreasing function around 100.

\section{}

\textit{(a) Write a microscope equation for $\phi(s) = 1/s$ as $s = 2$ and use it to estimate
  $1/2.03$ and $1/98$.}

$$\phi'(s) = -\frac{1}{s^2}$$
thus
$$\phi'(2) = -\frac{1}{4} = -0.25$$

And
$$\Delta \phi = -0.25 \Delta s$$

We can get
$$\Delta s = |2 - 2.03| = 0.03$$
therefore

$$\phi(2.03) \approx 1/2 - 0.25 * 0.03 = 0.4925$$

For the second one we've got
$$\Delta s = |2 - 1.98| = 0.02$$
$$\phi(1.98) \approx 1/2 + 0.25 * 0.02 = 0.505$$

\textit{(b) How far are your estimates from the values given by a calculator?}

$$\phi(2.03) = 0.49261083743842...$$
thus our error is
$$0.00011083743842371652$$

and for the latter we've got
$$\phi(1.98) = 0.5050505050505051...$$
and our error is
$$\approx -5 * 10^{-5}$$

\textit{(c) Your estimates should be lower than the calculator values; use the graph of
  $x = \phi(s)$ to explain why this is so.}

This is because the derivative is increasing, and thus our estimates will be lower.


\section{}

\textit{Show that $\sqrt{1 + 2h} \approx 1 + h$ when $h \approx 0$}

Let
$$f(h) = \sqrt{1 + 2h}$$
$$g(h) = 1 + h$$
then
$$f'(x) = \frac{1}{\sqrt{1 + 2h}}$$
and thus
$$f(0) = 1$$
$$f(0) = 1$$
$$f'(0) = 1$$
$$g'(0) = 1$$
Given that the function is contionus and differentiable around zero, we can conclude by the
same reasoning as in our  "microscope equation" section, that functions are approximately
equal in this range.

In case if we've got some shenannigans happening right after the 0, I've looked at the graphs
of those functions side by side in the range $[-0.1, 0.1]$ and concluded that they look pretty
simular at this range.

\section{}

\textit{(a) Determine the microscope equation for $x = \tan(s)$ at $s = \pi/4$}

$$x' = \sec^2(s)$$
$$x'(0) = 1$$
thus
$$\Delta x \approx \Delta s$$

\textit{(b) Show that $\tan(h + \pi/4) \approx 1 + 2h$ when $h \approx 0$. Is this estimate larger
  or smaller than the true value? Explain why.}

Their derivatives are equal at this point, and they are equal at zero too. Thus we can conclude
that they are simular around zero.

Also, I don't know why it didn't cross my mind earlier, but it also happens
because $1 + 2h$ is a linear approximation of this function at this point.

This estimate is lower then the true value, because the derivative of the function is increasing
around zero.

\section{}

\textit{Determine the local lentgth multiplier for $x = \sin(s)$ at each of the
  points $\{0, \pi/4, \pi/2, 2\pi/3, \pi\}$}

$$x' = \cos(x)$$
$$x'(0) = 1$$
$$x'(\pi/4) = 1/\sqrt{2}$$
$$x'(\pi/2) = 0$$
$$x'(2 \pi/3) = -1/2$$
$$x'(\pi) = -1$$

\section{}

\textit{What is true about the map $\phi: s \to x$ at a point $s_0$ where the local length
  multiplier is negative?}

It means that the funciton is decreasing

\section{}

\textit{Consider the hyperbolic sine and hyperbolic cosine functions, $\sinh s$ and
  $\cosh s$. Show each is the derivative of the other, and show }
$$\cosh^2 s - \sinh^2 s = 1$$
\textit{for all $s$}

$$f(x) = \cosh(x) = \frac{e^x + e^{-x}}{2}$$
by algebraic properties of the derivative and the chain rule we've got that
$$f'(x) = \frac{1}{2}(e^x - e^{-x}) = \sinh(x)$$

$$g(x) = \sinh(s) = \frac{e^x - e^{-x}}{2}$$
same applies to this one
$$g'(x) = \frac{1}{2}(e^x +  e^{-x}) = \cosh(x)$$

$$\cosh^2(x) - \sinh^2(x) = \left(\frac{e^x + e^{-x}}{2} \right)^2 -
\left(\frac{e^x - e^{-x}}{2} \right)^2 = 
\frac{(e^x + e^{-x})^2 - (e^x - e^{-x})^2}{4} = 
$$
$$ =
\frac{e^{2x} + 2e^xe^{-x} + e^{-2x} - (e^{2x} - 2e^xe^{-x} + e^{-2x})}{4} =
$$
$$ =
\frac{4e^xe^{-x}}{4} = \frac{4e^0}{4} = \frac{4}{4} = 1
$$
as desired.

\section{}

\textit{Use the substitution $x = \sinh s$ to determine }
$$\int{\frac{dx}{\sqrt{1 + x^2}}}$$

Let $x = \sinh s$. It follows that
$$\frac{dx}{ds} = \cosh(s)$$
$$dx = \cosh(s) ds$$
(just a reminder: this is not a  rigorous equation, just a useful mneumonic)

Thus we follow that
$$\int{\frac{dx}{\sqrt{1 + x^2}}}
= \int{\frac{\cosh(s) ds}{\sqrt{1 + (\sinh( s))^2}}}
$$
we know that 
$$\cosh^2 (x) - \sinh^2(x) = 1 \to \cosh^2 (x) = 1 + \sinh^2(x)$$
thus
$$\int{\frac{\cosh(s) ds}{\sqrt{1 + (\sinh( s))^2}}} =
\int{\frac{\cosh(s) ds}{\cosh(s)}} = \int{ds} = s = \text{asinh}(x)$$

\section{}

\textit{Determine the work done by the constant force $F = (2, -3)$ in displacing an
  object along}

\textit{(a) $\Delta x = (1, 2)$}

$$\Delta x \cdot F = 2 - 6 = -4$$

\textit{(b) $\Delta x = (1, -2)$}

$$\Delta x \cdot F = 2 + 6 = 8$$

\textit{(c) $\Delta x = (-1, 0)$}

$$\Delta x \cdot F = 2 $$

\section{}

\textit{Determine the work done by the constant force $F = (7, -1, 2)$ in displacing an
  object along}

\textit{(a) $\Delta x = (0, 1, 1)$}

$$\Delta x \cdot F = 0 - 1 + 2 = 1 $$

\textit{(b) $\Delta x = (1, -2, 0)$}

$$\Delta x \cdot F = 7 + 2 + 0 = 9$$

\textit{(c) $\Delta x = (0, 0, 1)$}

$$\Delta x \cdot F = 0 + 0 + 2  = 2$$

\section{}

\textit{Suppsoe a constant force $F$ in the place does $7$ units of work in displacing an
  object along $\Delta x = (2, -1)$ and $-3$ units of work along $\Delta x = (4, 1)$. How
  much work does $F$ do in displacing an object along $\Delta x = (1, 0)$? Along
  $\Delta x = (0, 1)$? Find a nonzero displacement $\Delta x$ along which $F$ does no work.}

We can follow that $F = (v_1, v_2)$ for which it will be true that
$$
\begin{cases}
  4 v_1 + v_2 = 7 \\
  2 v_1 - v_2 = -3
\end{cases}
$$

$$
\begin{cases}
  3v_2  = 13 \\
  2 v_1 - v_2 = -3
\end{cases}
$$

$$
\begin{cases}
  v_2  = 4 + 1/3 \\
  2 v_1 - 4 - 1/3 = -3
\end{cases}
$$

$$
\begin{cases}
  v_1 = 2/3 \\ 
  v_2  = 13/3 \\
\end{cases}
$$

$$F = (2/3, 13/3)$$

Thus for $\Delta x = (1, 0)$ we've got
$$W  = 2/3$$
and for $\Delta x = (0, 1)$ we've got
$$W  = 13/3$$

In order to have a desired vector we've gotta have $v = (x_1, x_2)$ such that $F \cdot v = 0$.
Applying some formulas we get 
$$2/3 x_1 + 13/3 x_2 = 0$$
$$2 x_1 + 13 x_2 = 0$$
$$x_1 = -6.5 x_2$$
Thus we can get $v = (1, -6.5)$ with the desired properties.

\section{}

\textit{Let $W(F, \Delta x)$ be the work done by the constant force $F$ along the linear
  displacement $\Delta x$. Show that $W$ is a linear function of the vectors $F$ and
  $\Delta x $.}

Given that $W = F \cdot \Delta x$ and we are working with the real numbers here only,
we can follow, that by properties of the inner product (specifically linearity in the first slot
and conjugate symmetry) we've got the desired linearity in both slots, as desired.

\section{}

\textit{Suppose $F = (P, Q)$. Determine the unit displacement $\Delta u$
  that yield the maximum and minimum values of $W$.}

$$u = (P, Q) \frac{1}{\sqrt{P^2 + Q^2}}$$
gives the maximum. $-u$ will give us minimum.

\section{}

\textit{Suppose the constant force $F = (P, Q)$ does the work $A$ along the displacement
  $(a, c)$ and the work $B$ along the displacement $(b, d)$. Determine $P$ and $Q$. What
  condition (on $a, b, c$ and $d$) must be satisfied for $P$ and $Q$ to be found?}

$$W_1 = (P, Q) \cdot (a, c) = Pa + Qc$$
$$W_2 = (P, Q) \cdot (b, d) = Pc + Qd$$

$$P = \frac{W_1 - Qa}{c}$$
$$P = \frac{W_2 - Qb}{d}$$

$$\frac{W_1 - Qa}{c} = \frac{W_2 - Qb}{d}$$
$$\frac{W_1}{c} - Q \frac{a}{ c} = \frac{W_2}{d} - Q\frac{b}{d}$$
$$\frac{W_1}{c} - \frac{W_2}{d}  =  - Q\frac{b}{d} + Q \frac{a}{ c}$$
$$\frac{W_1}{c} - \frac{W_2}{d}  =   Q \frac{a}{ c} - Q\frac{b}{d} $$
$$\frac{W_1}{c} - \frac{W_2}{d}  =   Q \left( \frac{a}{ c} - \frac{b}{d} \right) $$
$$\frac{W_1 d - W_2 c}{cd}  =   Q  \frac{ad - bc }{ cd} $$
$$\frac{W_1 d - W_2 c cd}{cd (ad - bc)}  =   Q   $$
$$
\begin{cases}
  Q = \frac{W_1 d - W_2 c }{ad - bc} \\
  P = \frac{W_1 - \frac{W_1 d - W_2 c }{ad - bc} a}{c}
\end{cases}
$$

We need $(a, c)$ and $(b, d)$ to be linearly independent. In this particular case we need
the $(a, c) \neq \lambda (b, d)$ for some $\lambda \neq 0 \in R$.


\end{document}

%%% Local Variables:
%%% mode: latex
%%% TeX-master: t
%%% End:
