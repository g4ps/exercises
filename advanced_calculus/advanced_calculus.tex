\documentclass[11pt,oneside,titlepage]{book}
\title{My advanced calculus exercises}
\usepackage{amsmath, amssymb}
\usepackage{geometry}
\usepackage{hyperref}
\author{Evgeny Markin}
\date{2023}

\DeclareMathOperator \inv {^{-1}}

\begin{document}
\maketitle
\tableofcontents


\chapter{Starting points}

\section{}

\textit{Evaluate }
$$\int_0^\infty{\frac{dx}{1 + x^2}}$$
\textit{and}
$$\int_{-\infty}^1{\frac{dx}{1 + x^2}}$$

$$\int_0^\infty{\frac{dx}{1 + x^2}} = [\tan \inv]_0^\infty = \pi/2 - 0 = \pi/2 $$

$$\int_{-\infty}^1{\frac{dx}{1 + x^2}} = [\tan \inv]_{-\infty}^1 = [\pi/4 + \pi/2] = \frac{3}{4} \pi$$


\section{}

\textit{Determine}
$$\int{\frac {x dx}{1 + x^2}}$$
\textit{Which type of substitution did you use?}

$$\int{\frac {x dx}{1 + x^2}} = \frac 1 2 \int{ \frac {2 x  dx}{1 + x^2}}$$
let $u(x) = 1 + x^2$. Then $u'(x) = 2x$ and therefore
$$\frac 1 2 \int{ \frac {2 x  dx}{1 + x^2}} =
\frac 1 2 \int{ \frac {du}{u}} = \frac{1}{2} \ln(u) = \frac{1}{2} \ln(1 + x^2)$$

I've used push-forward substitution here.

\section{}

\textit{Carry out a change of variables to evaluate the integral and determite the type
of substitution used.}
$$\int_{-R}^{R}{\sqrt{R^2 - x^2}dx}$$



Let's try $x = R \sin(s)$ (the idea is to use the identity $\sin^2(x) + \cos^2(x) = 1$ here
somewhere). It follows that $s = \arcsin(\frac{x}{R})$



$$\int_{-R}^{R}{\sqrt{R^2 - x^2}dx} =
\int_{-R}^{R}{\sqrt{R^2 - R^2 \sin^2(s)} R\cos(x) ds} =
\int_{-R}^{R}{R^2 \sqrt{1 -  \sin^2(s)} \cos(s) ds} =$$
$$ = 
R^2  \int_{-R}^{R}{\cos^2(s) ds} =
\frac{R^2}{2}  \int_{-R}^{R}{1 + \cos(2s)  ds} =
\frac{R^2}{2}\left[s + \frac{1}{2} \sin(2s)\right]_{-R}^R
=
$$

$$
= \frac{R^2}{2}\left[\arcsin(\frac{x}{R}) + \frac{1}{2} \sin(2\arcsin(\frac{x}{R}))\right]_{-R}^R =
$$
$$
=
\frac{R^2}{2} \left[\arcsin(1) + \frac{1}{2} \sin (2 \arcsin(1)) - \arcsin(-1) -
  \frac{1}{2} \sin (2 \arcsin(-1))\right]= 
$$
$$
=
\frac{R^2}{2} \left[\pi/2 + \frac{1}{2} \sin (\pi) + \pi/2 -
  \frac{1}{2} \sin (-\pi)\right]=
\frac{R^2}{2} \pi = \frac{\pi R^2}{2}
$$
Which is goot enough for me. We used the pullback approach here

\section{}

\textit{Determine }
$$\int {\frac{\arctan(x)}{1 + x^2} dx}$$
\textit{and show }
$$\int_0^\infty {\frac{\arctan(x)}{1 + x^2} dx} = \pi^2/8$$

$$\int {\frac{\arctan(x)}{1 + x^2} dx} $$
let $u = \arctan(x)$. Then
$$\int {\frac{\arctan(x)}{1 + x^2} dx} = \int u = u^2/2 = \arctan(x)^2/2$$
thus
$$\int_0^\infty {\frac{\arctan(x)}{1 + x^2} dx} = \pi^2/8$$
as desired.

\section{}

\textit{(a) Detrmine }
$$\int{\frac{1}{w (\ln(w))^p} dw}$$.
\textit{Which type of substitution did you use?}

Let $u = \ln(w)$. It follows that $du = \frac{1}{w} dw$. Thus
$$\int{\frac{1}{w (\ln(w))^p} dw} =
\int{\frac{1}{w } (\ln(w))^{-p} dw} =
\int{(u)^{-p} du} =  \frac{u^{-p - 1}}{-p - 1} = \frac{(\ln(w))^{-p - 1}}{-p - 1}
$$.
Given that $p \neq -1$.

Othersise it is $\ln(\ln(w))$.

\textit{(b) Evaluate }
$$I = \int_2^\infty{\frac{1}{w (\ln(w))^p} dw}$$.
\textit{for which values of $p$ is $I$ finite?}

For $p = 1$ we've got that
$$\lim{\ln(\ln(w))} = \infty$$
thus it diverges

For $p \neq -1$ we've got
$$I = \int_2^\infty{\frac{1}{w (\ln(w))^p} dw} =
\lim_{w \to \infty}{\left[\frac{(\ln(w))^{-p - 1}}{-p - 1}\right]} - \frac{(\ln(2))^{-p - 1}}{-p - 1}
$$
The only thing that bothers us is that whether
$$(\ln(w))^{-p - 1} = (\frac{1}{(\ln(w))})^{p - 1}$$
converges. This happens whenever $p > 1$ (in that case we got that $\ln(w) \to \infty$).
Otherwise it diverges.

\section{}

\textit{State a condition that guarantees a function $x = \phi(s)$ has an inverse. Then use
  your condition to decide whether each of the following functions is invertible. When possible,
  find a formula for the inverse of each function that is invertible.}

Such a condition is bijectivity on a given domain and codomain.

\textit{(a)}
$$x = 1/s$$

Is bijective on a domain $R \setminus \{0\}$. Inverse is $s = 1/x$.

\textit{(b)}
$$x = s + s^3$$

This one is bijective (because it is strictly increasing and unbounded below and above).
$$x = s + s^3$$
$$s^3 + s - x = 0$$
Maxima gives some god-awful result for an inverse function, but it can be obtained by solving the
cubic polinomial.

\textit{(c)}
$$x = \frac{s}{1 + s^2}$$

It is not surjective on $R$, and in addition, it is not injective. Thus it cannot be used
without some heavy restrictions on the domain and codomain.

\textit{(d)}
$$x = \sinh s = \frac{e^s - e^{-s}}{2}$$

Looks solid to me.

$$s = \text{asinh}(s)$$
is a desired inverse function;

\textit{(e)}
$$x = \frac{s}{\sqrt{1 - s^2}}$$

Is bijective on restricted domain. I'm not sure if we can have an analytical inverse of this
function.

\textit{(f)}
$$x = ms + b$$

Is a standart linear function. If $m \neq 0$, then we've got inverse on whole $R$.

\textit{(g)}
$$x = cosh(s)$$

Is bijective on restricted domain. Reverse is $s = \text{acosh}(x)$.

\textit{(h)}
$$x = s - s^3$$

Is bijective on restricted domain. Inverse is terrible.


\textit{(i)}
$$x = tanh(s)$$

Also bijective on restricted domain. Inverse is
$$s = \text{atanh}(x)$$

\textit{(j)}
$$x = \frac{1 - s}{1 + s}$$

BIjective on restricted domain.

\section{}

\textit{(a) Obtain formulas for $f(s) = \cos(\arcsin(s))$ and $g(s) = \tan(\arcsin(s))$
  directly as funtions of $s$ that involve neither trigonometric nor inverse trigonometric
  functions. Your answer will involve the square root function and polynomical expressions
  in $s$.}

$$f(s) = \cos(\arcsin(s)) $$
$$\cos(\arcsin(s)) = \sin(\pi/2 - \arcsin(s)) =
\sin(\pi/2)\cos(\arcsin(s)) - \sin(\arcsin(s)) \cos(\pi/2) = $$
$$ = \sin(\pi/2)\cos(\arcsin(s)) = \cos(\arcsin(s)) $$

$$\cos(\arcsin(s)) = \sqrt{1 - \sin^2(\arcsin(s))} = \sqrt{1 - s^2}$$

$$g(s) = \tan(\arcsin(s))$$
$$\tan(\arcsin(s)) = \frac{\sin(\arcsin(s))}{\cos(\arcsin(s))} = \frac{s}{\sqrt{1 - s^2}}$$


\textit{(b) Compute teh derivative of $\cos(\arcsin(s))$ using the chain rule and the
  derivatives of $\cos u$ and $\arcsin s$. Then compute the derivative of $f(s)$ using
  your expressions in part (a). Compare the two derivatives. Do the same for $\tan(\arcsin(s))$
  and $g(s)$}

$$f'(s) = -\sin(\arcsin(s)) \frac{1}{\sqrt{1 - s^2}} = - \frac{s}{\sqrt{1 - s^2}}$$
$$f'(s) = -2s \frac{1}{2\sqrt{1 - s^2}} = - \frac{s}{\sqrt{1 - s^2}}$$
They are the same.

$$g'(s) = \sec^2(\arcsin(s)) \frac{1}{\sqrt{1 - s^2}} =
\frac{1}{\cos^2(\arcsin(s))} \frac{1}{\sqrt{1 - s^2}} =
\frac{1}{(1 - s^2)\sqrt{1 - s^2}} 
$$

$$g'(s) = \frac{\sqrt{1 - s^2} - s(- \frac{s}{\sqrt{1 - s^2}})}{1 - s^2} =
\frac{\sqrt{1 - s^2} + \frac{s^2}{\sqrt{1 - s^2}}}{1 - s^2} =
\frac{\frac{1 - s^2 + s^2}{\sqrt{1 - s^2}}}{1 - s^2} =
\frac{\frac{1}{\sqrt{1 - s^2}}}{1 - s^2} =
\frac{1}{(1 - s^2) \sqrt{1 - s^2}} 
$$

They are also the same.

\section{}

\textit{Use $x = \arcsin s$ to show $\int {\cos^3 x dx} = \sin(x) - \frac{\sin^3 x}{3}$.}

$$s = \sin(x)$$

$$dx = \frac{1}{\sqrt{1 - x^2}}ds$$

$$\int{\cos^3 (x) dx} = \int{\cos^3 (\arcsin(s)) \frac{1}{\sqrt{1 - s^2}} ds} =
\int{(\sqrt{1 - s^2})^3 \frac{1}{\sqrt{1 - s^2}} ds} =
$$
$$
\int{1 - s^2 ds} = s - \frac{s^3}{3} = \sin(x) - \frac{\sin^3(x)}{3}
$$

as desired.

\section{}

\textit{(a) Write the microscope equation (i.e. the linear approximation) for
  $\phi(s) = \sqrt{s}$ at $s = 100$.}

$$\phi'(s) = \frac{1}{2 \sqrt{s}}$$
$$\phi'(100) = \frac{1}{2 \sqrt{100}} = 1/20 = 0.05$$
Thus
$$\Delta \phi = 0.05 \Delta s$$

\textit{(b) Use the microscope equation from part (a) to estimate $\sqrt{102}$ and $\sqrt{99.4}$}

For the first one we've got that
$$\Delta s = |102 - 100| = 2$$
thus
$$\Delta \phi = 0.05 * 2 = 0.1$$
Therefore
$$\phi(102) \approx \phi(100) + 0.1 = 10 + 0.1 = 10.1$$

For the second one we've got
$$\Delta s = |100 - 99.4| = 0.06$$
thus
$$\Delta \phi = 0.06 * 0.05 = 0.03$$
and since $99.4 < 100$ we follow that
$$\phi(99.4) \approx \phi(100) - \Delta \phi = 10 - 0.03 = 9.97$$

\textit{(c) How far are your estimate from those given by a calculator?}

$$\sqrt{102} \approx 10.0995049$$
$$\sqrt{99.4} \approx 9.969955$$
thus we've got  error of approximately $10^{-3}$ in the first case and $10^{-4}$ in the second.

\textit{(d) Your estimates should be greater than the calculator values; use the graph of
  $x = \phi(s)$ to explain why this is so.}

This is because derivative of this function is a  decreasing function around 100.


\end{document}

%%% Local Variables:
%%% mode: latex
%%% TeX-master: t
%%% End:
