\documentclass[11pt,oneside,titlepage]{book}
\title{My Measure, Integration, \& Real Analysis exercises}
\usepackage{amsmath, amssymb}
\usepackage{geometry}
\usepackage{hyperref}
\author{Evgeny (Gene) Markin}
\date{2025}

\DeclareMathOperator \map {\mathcal {L}}
\DeclareMathOperator \pow {\mathcal {P}}
\DeclareMathOperator \real {\mathbb {R}}
\DeclareMathOperator \topol {\mathcal {T}}
\DeclareMathOperator \basis {\mathcal {B}}
\DeclareMathOperator \ns {null}
\DeclareMathOperator \range {range}
\DeclareMathOperator \fld {fld}
\DeclareMathOperator \inv {^{-1}}
\DeclareMathOperator \Span {span}
\DeclareMathOperator \lra {\Leftrightarrow}
\DeclareMathOperator \eqv {\Leftrightarrow}
\DeclareMathOperator \la {\Leftarrow}
\DeclareMathOperator \ra {\Rightarrow}
\DeclareMathOperator \imp {\Rightarrow}
\DeclareMathOperator \true {true}
\DeclareMathOperator \false {false}
\DeclareMathOperator \dom {dom}
\DeclareMathOperator \ran {ran}
\newcommand{\eangle}[1]{\langle #1 \rangle}
\newcommand{\set}[1]{\{ #1 \}}
\newcommand{\qed}{\hfill $\blacksquare$}



\begin{document}
\maketitle
\tableofcontents


\chapter{Riemann Integration}

\section{Riemann Integral}

\subsection{}

\textit{Suppose $f: [a, b] \to R$ is a bounded function such that
  $$L(f, P, [a, b]) = U(f, P, [a, b])$$
  for some partition $P$ of $[a, b]$. Prove that $f$ is a constant
  function on $[a, b]$.}

Suppose that $f$ is not contant. We want to follow that there is a
section of the partition $P$ whose elements are not all equal.

Assume that all images of elements of any subinterval of a partition
are equal. This means that for a given $x_j, x_{j + 1}$ we have that
if $x_1, x_2 \in [x_j, x_{j + 1}$, then $f(x_1) = f(x_2)$. We then
follow that $f(x_1) = f(x_2)$, $f(x_2) = f(x_3)$, and so on, which
implies that images of all elemets of the partition are equal. By our
assumption we have that all the elements in between elements of the
partition are also equal, which implies that $f$ is a constant
function.

Thus if $f$ is not a constant function, then there is a
subinterval of the partition $[x_j, x_{j + 1}]$ such that there are
$q_1, q_2 \in [x_j, x_{j + 1}]$ for which $f(q_1) \neq f(q_2)$.
Now relabel $q_1, q_2$ so that $f(q_1) > f(q_2)$. We follow that
$$\sup_{[x_j, x_{j + 1}]} f \geq f(q_1) > f(q_2) \geq \inf_{[x_j, x_{j + 1}]} f $$
and thus
$$\sup_{[x_j, x_{j + 1}]} f >  \inf_{[x_j, x_{j + 1}]} f $$
$$(x_{j + 1} - x_j) \sup_{[x_j, x_{j + 1}]} f >  (x_{j + 1} - x_j)  \inf_{[x_j, x_{j + 1}]} f $$

We then can state that for any $x_k, x_{k +1}$ we have that
$$(x_{k + 1} - x_k) \inf_{[x_k, x_{k +1}]} f
\leq (x_{k + 1} - x_k) \sup_{[x_k, x_{k +1}]} f$$
which further implies that
$$\sum_{k \in P \setminus \set{j}}(x_{k + 1} - x_k) \inf_{[x_k, x_{k +1}]} f
\leq \sum_{k \in P \setminus \set{j}} (x_{k + 1} - x_k) \sup_{[x_k, x_{k +1}]} f$$
by then adding the partition in question to both sides we have that
$$(x_{j + 1} - x_j)  \inf_{[x_j, x_{j + 1}]} f  +
\sum_{k \in P \setminus \set{j}}(x_{k + 1} - x_k) \inf_{[x_k, x_{k +1}]} f
<$$
$$ < (x_{j + 1} - x_j)  \sup_{[x_j, x_{j + 1}]} f  +
\sum_{k \in P \setminus \set{j}} (x_{k + 1} - x_k) \sup_{[x_k, x_{k +1}]} f$$
thus
$$
\sum_{k \in P}(x_{k + 1} - x_k) \inf_{[x_k, x_{k +1}]} f
< \sum_{k \in P} (x_{k + 1} - x_k) \sup_{[x_k, x_{k +1}]} f$$
and when we apply definitions we get that
$$L(f, P, [a, b]) < U(f, P, [a, b])$$
as desired.

\subsection{}

\textit{Suppose $a \leq s < t \leq b$. Define $f: [a, b] \to R$ by
$$f(x) =
\begin{cases}
  s < x < t \to 1 \\
  0 \text{ otherwise }
\end{cases}
$$
Prove that $f$ is Riemann integrable on $[a, b]$ and that $\int_a^b{f} = t - s$}

We can follow that every part of the sum in the definition
$$L(f, P, [a, b]) = \sum_{j = 1}^n{(x_j - x_{j - 1})\inf_{[x_{j - 1}, x_j]}{f}}$$
is nonnegative since $f(x) \geq 0$ for all $x \in [a, b]$. Same goes
for $U(f, P, [a, b])$. We then follow that there is $n \in N$ such that
$$s < s + 1/n < t - 1/n < t$$
and for all $m > n$ we have 
$$s < s + 1/m < t - 1/m < t < b$$
We then can create a set of partitions
$$P_m = a, s, s + 1/m, t - 1/m, t, b$$
for which we have
$$L(f, P_m, [a, b]) =$$
$$ = (s - a) * 0 + (s + 1/m - s) * 0 + (t - 1/m - s - 1/m) * 1 + (t - t + 1/m) * 0 + (b - t) * 0 = $$
$$ = t - 1/m - s - 1/m = t - s - 2/m$$
and
$$U(f, P_m, [a, b]) =$$
$$ = (s - a) * 0 + (s + 1/m - s) * 1 + (t - 1/m - s - 1/m) * 1 + (t - t + 1/m) * 1 + (b - t) * 0 = $$
$$ = s + 1/m - s + t - 1/m - s - 1/m + t - t + 1/m = $$
$$ = t - s = $$
thus we have
$$U(f, P_m, [a, b]) - L(f, P_m, [a, b]) = 2/m$$
and since for all $\epsilon > 0$ we have $m \in N$ such that $n \geq m
\ra 2/m < \epsilon$ we follow that for all $\epsilon > 0$ there is $m
\in N$ such that
$$U(f, [a, b]) - L(f, [a, b]) \leq U(f, P_m, [a, b]) - L(f, P_m, [a, b]) < \epsilon$$
thus proving that $f$ is indeed Riemann integrable on $[a, b]$. We
then follow that
$$L(f, P_m, [a, b]) \leq \int_a^b{f} \leq U(f, P_m, [a, b])$$
for all $m$, and thus
$$t - s - 2/m \leq \int_a^b{f} \leq t - s$$
for all $m$. This in turn implies that
$$\int_a^b{f} = t - s$$
as desired.

\subsection{}

\textit{Suppose $f: [a, b] \to R$ is a bounded function. Prove that $f$ is Riemann integrable
  if and only if for each $\epsilon > 0$ there exists a partition $P$ of $[a, b]$ such that
  $$U(f, P, [a, b]) - L(f, P, [a, b]) < \epsilon$$
}

We can follow that
$$U(f, P, [a, b]) \geq U(f, [a, b]) $$
$$L(f, [a, b]) \geq L(f, [a, b]) $$
by defnition. Thus we have that
$$U(f, P, [a, b]) - L(f, P, [a, b]) \geq U(f, [a, b]) - L(f, P, [a, b])
\geq U(f, [a, b]) - L(f, [a, b]) $$
and thus for every $\epsilon > 0$ we have that
$$U(f, [a, b]) - L(f, [a, b]) < \epsilon$$
Since $U(f, [a, b]) \geq L(f, [a, b])$ we follow then that $U(f, [a,
b]) - L(f, [a, b]) \geq 0$ in general, and in particular we've got
that
$$U(f, [a, b]) - L(f, [a, b]) = 0 \lra U(f, [a, b]) = L(f, [a, b]) $$
thus implying the desired result by definition.

Assume now that $f$ is Riemann integrable. Let $\epsilon > 0$. We
follow that there are partions $P_1, P_2$ such that
$$U(f, P_1, [a, b]) - U(f, [a, b]) < \epsilon/2$$
$$L(f, [a, b]) - L(f, P_2, [a, b])  < \epsilon/2$$
we then can sum up the previous inequalities to get
$$U(f, P_1, [a, b]) - U(f, [a, b]) + L(f, [a, b]) - L(f, P_2, [a, b]) < \epsilon$$
Since $U(f, [a, b]) = L(f, [a, b])$ by the fact that $f$ is Riemann
integrable on $[a, b]$, which implies that
$$U(f, P_1, [a, b]) - L(f, P_2, [a, b]) < \epsilon$$
We then follow that there is $P_3 = P_1 \cup P_2$ for which we have
$$U(f, P_3, [a, b]) - L(f, P_3, [a, b]) < \epsilon$$
by 1.5, as desired.

\subsection{}

\textit{Suppose $f, g: [a, b] \to R$ are Riemann integrable. Prove that $f + g$ is
  Riemann integrable on $[a, b]$ and
  $$\int_a^b{f + g} = \int_a^b{f} + \int_a^b{g} $$
}

We can follow that
$$L(f + g, P, [a, b]) =
\sum_{j = 1}^n{(x_j - x_{j - 1}) \inf_{[x_{j - 1}, x_j]}{f + g}}$$
$$\inf_{[x_{j - 1}, x_j]}{f + g} =
\inf_{[x_{j - 1}, x_j]}{f} + \inf_{[x_{j - 1}, x_j]}{g}$$
and similar for $\sup$ and $U(f, P, [a, b])$. Don't know whether or
not we've got the latter equality proven, but it's trivial to prove if
not. Similar case holds for $U$.

We then follow that for a given
$\epsilon/2 > 0$ we have $P_1, P_2$ such that
$$U(f, P_1, [a, b]) - L(f, P_1, [a, b]) < \epsilon/2$$
$$U(g, P_2, [a, b]) - L(g, P_2, [a, b]) < \epsilon/2$$
we thus have that by $P_3 = P_1 \cup P_2$ we have
$$U(f, P_3, [a, b]) - L(f, P_3, [a, b]) < \epsilon/2$$
$$U(g, P_3, [a, b]) - L(g, P_3, [a, b]) < \epsilon/2$$
and thus we can add those up to get 
$$U(f, P_3, [a, b]) - L(f, P_3, [a, b]) + U(g, P_3, [a, b]) - L(g, P_3, [a, b]) < \epsilon$$
thus by the first paragraph we've got
$$U(f + g, P_3, [a, b]) - L(f + g, P_3, [a, b]) < \epsilon$$
which by previous exercise proves that $f + g$ is Riemann integrable, as desired.

By the first paragraph as well we can follow that
$$\int_a^b{f + g} = L(f + g, [a, b]) = \sup_P{L(f + g, P, [a, b])} =
\sup_P{L(f, P, [a, b]) + L(g, P, [a, b])} =
\sup_P{L(f, P, [a, b])} + \sup_P{L(g, P, [a, b])} = \int_a^b{f} + \int_a^b{g}$$
as desired.

\textit{The rest was handled in my real analysis course}

\section{Riemann Integral Is Not Good Enough}

\subsection{}

\textit{Define $f: [0, 1] \to R$ as follows:
$$ f(a) = 
\begin{cases}
  0 \text{ if } a \text{ is irrational} \\
  1/n \text{ if } a \text{ is rational and } n \text{
  is the smallest positive integer such that } a = m/n \\
\end{cases}
$$}

Show that $f$ is Riemann integrable and comute $\int_0^1{f}$. We
firstly follow that for any given $0 \leq x_1 < x_2 \leq 1$ we have that
there is $x_1 < x_3 < x_2$ such that
$$f(x_3) = 0$$
and thus for any partition $P$ we've got that
$$L(f, P, [a, b]) = 0$$
thus
$$L(f, [a, b]) = 0$$
We then follow that for any given $\epsilon > 0$ there is $n \in N$
such that $1/n < \epsilon$, and thus we can create a set
$$R = \set{i/j: j \leq n, 0 \leq i \leq n}$$
that is finite. Thus we can follow that
$$\sup_{[0, 1] \setminus R}{f} = 1/(n + 1) < \epsilon$$
and thus we have that for every $\epsilon > 0$ there is $P$ such that
$$U(f, P, [a, b]) < \epsilon$$
thus implying that
$$U(f, [a, b]) = 0$$
therefore we conclude that
$$\int_0^1{f} = 0$$

\chapter{Measures}

\section{Outer Measure on $R$}

\subsection{}

\textit{Prove that if $A$ and $B$ are subsets of $R$ and $|B| = 0$,
  then $|A \cup B| = |A|$}

2.8 implies that
$$|A \cup B| \leq |A| + |B| = |A|$$
and since $A \subseteq A \cup B$ we follow by 2.5 that
$$|A| \leq |A \cup B|$$
which gives us the desired result

\subsection{}

\textit{Suppose $A \subseteq R$ and $t \in R$. Let $tA = \set{ta: a
    \in A}$. Prove that $|tA| = |t||A|$.}

If $t = 0$, then we follow that $tA = \set{0}$, and thus $|tA| = 0 = 0
* |A|$. Thus suppose taht $t \neq 0$.

Let $I_k$ be a family of intervals, whose union covers $A$. We follow
then that if $q \in tA$, then there is $a \in A$ such that $q =
ta$. Thus there is $n \in N$ such that $a \in I_n$. Since $I_n$ is an
open interval, by definition there are $b, c$ such that $I_n = (b,
c)$ (where both $b$ and $c$ can be infinities), and thus $a \in (b, c)
\lra b < a < c$. Therefore we have that if $t > 0$ that
$$tb < ta < tc$$
and if $t < 0$ then
$$tb > ta > tc$$
which implies that $ta \in t(b, c)$ ($(b, c) \subseteq R$, thus same
definition as in the text of the exercise applies). Thus we have that
$$q = ta \in tI_n$$
thus we have that if $I_n$ is a family of intervals that covers $A$, then
$tT_n$ is a family of intervals, that cover $tA$.

Now suppose that $I_n$ is a family of intervals that covers $tA$. Setting
$t' = t\inv$ by previous point we get that $t'I_n$ covers $A$.

Then simple application of definition gives us that
$$l(t(b, c)) = |t| l((b - c))$$
which in turn implies that
$$\sum{l(t I_k)} = |t| \sum{l(I_k)}$$
which implies that if $J \subseteq R$ is such that
$$J = \set{\sum{I_k}: I_1, ... \text{ are intervals whose union covers } A}$$
then
$$|t| J = \set{\sum{I_k}: I_1, ... \text{ are intervals whose union covers } tA}$$
and since for nonzero $t$ we have that
$$\inf {|t|J} = |t| \inf {J}$$
we follow that
$$|tA| = \inf {|t|J} = |t| \inf {J} = |t||A|$$
as desired


\end{document}